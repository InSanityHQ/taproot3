% Created 2021-09-11 Sat 09:35
% Intended LaTeX compiler: xelatex
\documentclass[letterpaper]{article}
\usepackage{graphicx}
\usepackage{grffile}
\usepackage{longtable}
\usepackage{wrapfig}
\usepackage{rotating}
\usepackage[normalem]{ulem}
\usepackage{amsmath}
\usepackage{textcomp}
\usepackage{amssymb}
\usepackage{capt-of}
\usepackage{hyperref}
\usepackage[margin=1in]{geometry}
\usepackage{fontspec}
\usepackage{indentfirst}
\setmainfont[ItalicFont = LiberationSans-Italic, BoldFont = LiberationSans-Bold, BoldItalicFont = LiberationSans-BoldItalic]{LiberationSans}
\newfontfamily\NHLight[ItalicFont = LiberationSansNarrow-Italic, BoldFont       = LiberationSansNarrow-Bold, BoldItalicFont = LiberationSansNarrow-BoldItalic]{LiberationSansNarrow}
\newcommand\textrmlf[1]{{\NHLight#1}}
\newcommand\textitlf[1]{{\NHLight\itshape#1}}
\let\textbflf\textrm
\newcommand\textulf[1]{{\NHLight\bfseries#1}}
\newcommand\textuitlf[1]{{\NHLight\bfseries\itshape#1}}
\usepackage{fancyhdr}
\pagestyle{fancy}
\usepackage{titlesec}
\usepackage{titling}
\makeatletter
\lhead{\textbf{\@title}}
\makeatother
\rhead{\textrmlf{Compiled} \today}
\lfoot{\theauthor\ \textbullet \ \textbf{2021-2022}}
\cfoot{}
\rfoot{\textrmlf{Page} \thepage}
\titleformat{\section} {\Large} {\textrmlf{\thesection} {|}} {0.3em} {\textbf}
\titleformat{\subsection} {\large} {\textrmlf{\thesubsection} {|}} {0.2em} {\textbf}
\titleformat{\subsubsection} {\large} {\textrmlf{\thesubsubsection} {|}} {0.1em} {\textbf}
\setlength{\parskip}{0.45em}
\renewcommand\maketitle{}
\date{\today}
\title{Dogma HW Notes}
\hypersetup{
 pdfauthor={},
 pdftitle={Dogma HW Notes},
 pdfkeywords={},
 pdfsubject={},
 pdfcreator={Emacs 27.2 (Org mode 9.4.4)}, 
 pdflang={English}}
\begin{document}

\maketitle


\subsection{\#flo \#ret}
\label{sec:org2d6d656}
\section{Notes}
\label{sec:org9bdb35e}
\subsubsection{DNA/RNA}
\label{sec:org5685e00}
\begin{itemize}
\item Nucleotides make up DNA and RNA

\begin{itemize}
\item DNA and RNA have different levels of stability

\begin{itemize}
\item This is all because of the hydroxyl group vs just the hydrogen
\end{itemize}

\item If the second carbon does not have an oxygen then it is DNA

\begin{itemize}
\item You would write the letter d
\end{itemize}

\item If there is a T then put a T
\item Count the number of phosphates and use the code

\begin{itemize}
\item 1 is Mono - M
\item 2 is Di - D
\item 3 Tri - T
\end{itemize}

\item Phosphate groups determine the amount of energy in the molecule
\end{itemize}
\end{itemize}

\subsubsection{Central Dogma of Biology}
\label{sec:org943da7b}
\begin{itemize}
\item From the DNA an RNA is created
\item From the RNA a protein is generated
\item The central dogma says that protein instructions are found in DNA
\item RNA carries these instructions somewhere which manufactures proteins
\item Parts of DNA are transcribed into RNA
\item In short DNA -- > RNA -- > Protein

\begin{itemize}
\item This is done because moving DNA doesn't make sense as it's the
master copy
\item RNA is moved to the place that the protein can be made from the
instructions
\end{itemize}

\item This is done through transcription and translation
\end{itemize}
\end{document}
