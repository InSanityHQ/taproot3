% Created 2021-09-11 Sat 08:17
% Intended LaTeX compiler: xelatex
\documentclass[letterpaper]{article}
\usepackage{graphicx}
\usepackage{grffile}
\usepackage{longtable}
\usepackage{wrapfig}
\usepackage{rotating}
\usepackage[normalem]{ulem}
\usepackage{amsmath}
\usepackage{textcomp}
\usepackage{amssymb}
\usepackage{capt-of}
\usepackage{hyperref}
\usepackage[margin=1in]{geometry}
\usepackage{fontspec}
\usepackage{indentfirst}
\setmainfont[ItalicFont = LiberationSans-Italic, BoldFont = LiberationSans-Bold, BoldItalicFont = LiberationSans-BoldItalic]{LiberationSans}
\newfontfamily\NHLight[ItalicFont = LiberationSansNarrow-Italic, BoldFont       = LiberationSansNarrow-Bold, BoldItalicFont = LiberationSansNarrow-BoldItalic]{LiberationSansNarrow}
\newcommand\textrmlf[1]{{\NHLight#1}}
\newcommand\textitlf[1]{{\NHLight\itshape#1}}
\let\textbflf\textrm
\newcommand\textulf[1]{{\NHLight\bfseries#1}}
\newcommand\textuitlf[1]{{\NHLight\bfseries\itshape#1}}
\usepackage{fancyhdr}
\pagestyle{fancy}
\usepackage{titlesec}
\usepackage{titling}
\makeatletter
\lhead{\textbf{\@title}}
\makeatother
\rhead{\textrmlf{Compiled} \today}
\lfoot{\theauthor\ \textbullet \ \textbf{2021-2022}}
\cfoot{}
\rfoot{\textrmlf{Page} \thepage}
\titleformat{\section} {\Large} {\textrmlf{\thesection} {|}} {0.3em} {\textbf}
\titleformat{\subsection} {\large} {\textrmlf{\thesubsection} {|}} {0.2em} {\textbf}
\titleformat{\subsubsection} {\large} {\textrmlf{\thesubsubsection} {|}} {0.1em} {\textbf}
\setlength{\parskip}{0.45em}
\renewcommand\maketitle{}
\author{Zachary Sayyah}
\date{\today}
\title{Eukaryotic Gene Expression}
\hypersetup{
 pdfauthor={Zachary Sayyah},
 pdftitle={Eukaryotic Gene Expression},
 pdfkeywords={},
 pdfsubject={},
 pdfcreator={Emacs 27.2 (Org mode 9.4.4)}, 
 pdflang={English}}
\begin{document}

\maketitle


\subsection{\#flo \#ret}
\label{sec:org355709d}
\section{Notes}
\label{sec:org6bbeb16}
\subsubsection{Overview}
\label{sec:orge305bee}
\begin{itemize}
\item Organisms turn genes on and off which is called Gene Expression

\begin{itemize}
\item This can be done in response to external and internal signals

\begin{itemize}
\item These signals are based off of environmental factors
\end{itemize}

\item This is also be done in order to specialize cells

\begin{itemize}
\item Certain cells need certain genes to preform their specific role
\end{itemize}
\end{itemize}
\end{itemize}

\#\#\# Differential Gene Expression

\begin{itemize}
\item Human Cells can express about 20\% of it's protein coded genes at any
given time
\item Most cells contain the same genome

\begin{itemize}
\item Each cell type must use specific parts of this genome

\begin{itemize}
\item This is called Differential gene expression
\end{itemize}

\item Exception would be cells of the immune system
\end{itemize}

\item Due to the importance of gene expression when it has issues it can
affect the organism significantly
\item Process of Gene expression in a Eukaryotic cell

\begin{itemize}
\item Chromatin (DNA unpacking) -->
\item RNA processing -->
\item Transport to cytoplasm -->
\item Translation -->
\item Protein processing -->
\item Transport to cellular destination-->
\end{itemize}

\item This process can often be equated to transcription for Prokaryote
cells
\end{itemize}

\subsubsection{Regulation of Chromatin Structure}
\label{sec:org075fdeb}
\begin{itemize}
\item The chromatin structure itself allows for the regulation of gene
expression

\begin{itemize}
\item This is partially due to the location of the promoter
\end{itemize}

\item Chemical modifications to the histone proteins can affect the
structure

\begin{itemize}
\item This in turn can affect gene expression
\item Histone proteins are the proteins in which the DNA is wrapped
\item There are many types of modifications that can take place

\begin{itemize}
\item Histone acetylation can tend to promote transcriptions by opening
up the chromatin
\item Additional methyl groups tend to close up the chromatin and
decrease transcription
\end{itemize}

\item DNA methylation occurs in most plants and animals as well as fungi
\item Methylated DNA will stay methalated through cell divisions

\begin{itemize}
\item This accounts for genomic imprinting
\item These epigenetic markers can be inherited

\begin{itemize}
\item There is continually more evidence for the importance of
epigentics in gene expression \#\#\# Regulation of Transcription
\end{itemize}
\end{itemize}
\end{itemize}

\item Chromatin changes are not permanent and can be reversed
\item The next step of gene expression regulation is in the transcription
factors

\begin{itemize}
\item These either allow for or inhibit transcription
\end{itemize}

\item These factors usually bind to proteins, but some of them bind to DNA
\item High levels of transcription factors created for specific genes are
associated with another protein thought creatively of as specific
transcription factors
\item Gene expression is dramatically increased or decreased by the binding
of specific transcription factors

\begin{itemize}
\item These are either activators or repressors
\end{itemize}

\item There are many transcription factors

\begin{itemize}
\item Repressors act in many different ways, but some bind directly to
control element DNA blocking activator binding
\item Others interfere with the activator itself
\end{itemize}

\item Coordinated control of genes can need to happen when multiple genes
need to be expressed at the same time for something to function

\begin{itemize}
\item These can often be signaled from the outside with something like a
hormone
\end{itemize}

\item The activation of receptors on the surface of the cell can release
specific repressors and activators
\end{itemize}

\#\#\# Mechanisms of Post-Transcriptional Regulation - Transcription is not
the only thing that regulates gene expression - How much of the protein
is created once the RNA is received is also a factor - RNA can be
interpreted in different ways with different things being introns and
others being exons - This allows for the creation of multiple proteins
from the same strand of RNA - RNA splicing is critical since it allows a
lot of information to be fit on a single strand of RNA - Around 75-100\%
of human genes with multiple exons undergo RNA splicing allowing for our
genome to describe a lot of complexity without needing as many genes -
Translation is another stage at which gene expression occurs - Some
regulatory proteins can bock translation of an mRNA by preventing
attachment to a ribisome - Length by which an mRNA is around is also
crucial - This can vary greatly depending on the cell - Cells can mark
proteins for destruction using something called ubiquitin
\end{document}
