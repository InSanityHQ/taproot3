% Created 2021-09-11 Sat 08:17
% Intended LaTeX compiler: xelatex
\documentclass[letterpaper]{article}
\usepackage{graphicx}
\usepackage{grffile}
\usepackage{longtable}
\usepackage{wrapfig}
\usepackage{rotating}
\usepackage[normalem]{ulem}
\usepackage{amsmath}
\usepackage{textcomp}
\usepackage{amssymb}
\usepackage{capt-of}
\usepackage{hyperref}
\usepackage[margin=1in]{geometry}
\usepackage{fontspec}
\usepackage{indentfirst}
\setmainfont[ItalicFont = LiberationSans-Italic, BoldFont = LiberationSans-Bold, BoldItalicFont = LiberationSans-BoldItalic]{LiberationSans}
\newfontfamily\NHLight[ItalicFont = LiberationSansNarrow-Italic, BoldFont       = LiberationSansNarrow-Bold, BoldItalicFont = LiberationSansNarrow-BoldItalic]{LiberationSansNarrow}
\newcommand\textrmlf[1]{{\NHLight#1}}
\newcommand\textitlf[1]{{\NHLight\itshape#1}}
\let\textbflf\textrm
\newcommand\textulf[1]{{\NHLight\bfseries#1}}
\newcommand\textuitlf[1]{{\NHLight\bfseries\itshape#1}}
\usepackage{fancyhdr}
\pagestyle{fancy}
\usepackage{titlesec}
\usepackage{titling}
\makeatletter
\lhead{\textbf{\@title}}
\makeatother
\rhead{\textrmlf{Compiled} \today}
\lfoot{\theauthor\ \textbullet \ \textbf{2021-2022}}
\cfoot{}
\rfoot{\textrmlf{Page} \thepage}
\titleformat{\section} {\Large} {\textrmlf{\thesection} {|}} {0.3em} {\textbf}
\titleformat{\subsection} {\large} {\textrmlf{\thesubsection} {|}} {0.2em} {\textbf}
\titleformat{\subsubsection} {\large} {\textrmlf{\thesubsubsection} {|}} {0.1em} {\textbf}
\setlength{\parskip}{0.45em}
\renewcommand\maketitle{}
\author{Houjun Liu}
\date{\today}
\title{Cell Reproduction}
\hypersetup{
 pdfauthor={Houjun Liu},
 pdftitle={Cell Reproduction},
 pdfkeywords={},
 pdfsubject={},
 pdfcreator={Emacs 27.2 (Org mode 9.4.4)}, 
 pdflang={English}}
\begin{document}

\maketitle


\section{Cell Reproduction}
\label{sec:org1cec798}
\textbf{How Cells Make New Cells}

=> Cell division is used as a process for asexual production, and growth
\& development + tissue renewal.

At some point, stuff ends up in different positions and that causes the
specialization of cells. (Outside cells become skin cells, etc.)

\textbf{Most cell division results in genetically identical daughter cell.}
Importantly, \emph{most}: meaning that there is a critical difference between
\href{KBhBIO101Meiosis.org}{KBhBIO101Meiosis} and
\href{KBhBIO101Mitosis.org}{KBhBIO101Mitosis}.

\subsection{Mitosis}
\label{sec:orga002e3a}
What you think of as "cell division". It takes somatic cells ---
"normal" cells (not sperm/egg cells) --- and makes two somatic cells
with the same DNA; essentially cloning the somatic cell. Basically, all
cells divide this way \emph{except} for reproductive cells. For more, see
\href{KBhBIO101Mitosis.org}{KBhBIO101Mitosis}.

\subsection{Meiosis}
\label{sec:orgcf1f767}
Meiosis is the process by which gametes ("sperm and egg cells") becomes
\emph{created}. NOTE! Not the word "divided", because gametes comes from the
division of germline cells. Unlike mitosis, the 23 \textbf{pairs} of chromosome
in the germline cells gets randomly mixed (see link for details) and
result in two somatic cells with 23 chromosomes each. Why? When the
sperm and egg connect together, they pair up and result back into 23
chromosome pairs for the child's gamete cells. See
\href{KBhBIO101Meiosis.org}{KBhBIO101Meiosis}.

\begin{html}
<!--
Also, know what\ldots{}

!\href{Pasted image 20210405130601.png.org}{Pasted image 20210405130601.png}

is doing.

\textbf{*}

To form a pair of sister chromatids: genetic material is duplicated as part of \href{KBhBIO101DNAReplication.org}{KBhBIO101DNAReplication}.

!\href{Pasted image 20210405132325.png.org}{Pasted image 20210405132325.png}

You get a set of chromatids for each chromosome. If the chromosome gets over-duplicated, you don't have a lot of fun: 18 repeat => down's syndrome.

\textbf{*}

-->
\end{html}
\end{document}
