% Created 2021-09-11 Sat 08:17
% Intended LaTeX compiler: xelatex
\documentclass[letterpaper]{article}
\usepackage{graphicx}
\usepackage{grffile}
\usepackage{longtable}
\usepackage{wrapfig}
\usepackage{rotating}
\usepackage[normalem]{ulem}
\usepackage{amsmath}
\usepackage{textcomp}
\usepackage{amssymb}
\usepackage{capt-of}
\usepackage{hyperref}
\usepackage[margin=1in]{geometry}
\usepackage{fontspec}
\usepackage{indentfirst}
\setmainfont[ItalicFont = LiberationSans-Italic, BoldFont = LiberationSans-Bold, BoldItalicFont = LiberationSans-BoldItalic]{LiberationSans}
\newfontfamily\NHLight[ItalicFont = LiberationSansNarrow-Italic, BoldFont       = LiberationSansNarrow-Bold, BoldItalicFont = LiberationSansNarrow-BoldItalic]{LiberationSansNarrow}
\newcommand\textrmlf[1]{{\NHLight#1}}
\newcommand\textitlf[1]{{\NHLight\itshape#1}}
\let\textbflf\textrm
\newcommand\textulf[1]{{\NHLight\bfseries#1}}
\newcommand\textuitlf[1]{{\NHLight\bfseries\itshape#1}}
\usepackage{fancyhdr}
\pagestyle{fancy}
\usepackage{titlesec}
\usepackage{titling}
\makeatletter
\lhead{\textbf{\@title}}
\makeatother
\rhead{\textrmlf{Compiled} \today}
\lfoot{\theauthor\ \textbullet \ \textbf{2021-2022}}
\cfoot{}
\rfoot{\textrmlf{Page} \thepage}
\titleformat{\section} {\Large} {\textrmlf{\thesection} {|}} {0.3em} {\textbf}
\titleformat{\subsection} {\large} {\textrmlf{\thesubsection} {|}} {0.2em} {\textbf}
\titleformat{\subsubsection} {\large} {\textrmlf{\thesubsubsection} {|}} {0.1em} {\textbf}
\setlength{\parskip}{0.45em}
\renewcommand\maketitle{}
\author{Houjun Liu}
\date{\today}
\title{Types of Genetic Inheritance}
\hypersetup{
 pdfauthor={Houjun Liu},
 pdftitle={Types of Genetic Inheritance},
 pdfkeywords={},
 pdfsubject={},
 pdfcreator={Emacs 27.2 (Org mode 9.4.4)}, 
 pdflang={English}}
\begin{document}

\maketitle


\section{Genetic Inheritance}
\label{sec:orgfd63505}
How to deal with \textbf{Heterozygus} (two different alleals of one gene) genes

\begin{itemize}
\item Mendelian: dominant vs recessive versions of genes (Mendel's pea
plants)
\item Incomplete dominance (snap dragons)
\item Codominance (human blood types)
\item Polygenic inheritance (human height \& skin color)
\item Epistasis (dog coat color)
\item Sex-linked inheritance (color-blindness)
\end{itemize}

\subsection{Mendelian Inheritance}
\label{sec:orge63d6aa}
\textbf{If two alleal for a gene differ, one could dominate the phenotype.}

\begin{itemize}
\item In order to see the recessive gene, a plant needs two copies of their
traits.
\item In order to see the dominant gene, the plant only need one copy of the
trait
\end{itemize}

But\ldots{}.. What's actualyl happening?

\begin{center}
\includegraphics[width=.9\linewidth]{Pasted image 20210426132223.png}
\end{center}

The "recessive" gene usually is a gene that does not code for the
functional enzyme. Hence, if you have one alleal with the functional
DNA, even if the other alleal is broken, a functional enzyme is created
and hence the individual will "express" this trait. It is \emph{only} with
both copies being broken that the enzyme that create that trait will not
exist and hence can't function.

\subsection{Incomplete Dominance}
\label{sec:org5f271d6}
Both alleals ale visible in the phenotype, and so neither is dominant
really. Think about the genetic explanation of inherintance above. In
the case of "incomplete dominance", not enough enzymes is created to
fully express a trait (like "red pigment") such that the resulting
organism will have an "incompletely" dominant trait.

\begin{center}
\includegraphics[width=.9\linewidth]{Pasted image 20210426132605.png}
\end{center}

\subsection{Codominance}
\label{sec:orga1f2050}
Both alleals are \emph{fully present}. For instance, in blood types, the AB
alleals will result in their codominance to created AB blood. This is
different from incomplete dominance in that that is simply a half-mix.

\subsection{Polygenic Dominance}
\label{sec:org83f321f}
Where a trait exists on the gradient of the combination multiple genes
than results in a phenotype.

\subsection{Epistasis}
\label{sec:orgb231bbf}
Alleals that could only be expressed if another alleal is already
expressed. For instance, the \texttt{Ee} gene in labrador retrivers control
whether a pigment could be deposited. So, if a dog has \texttt{ee} gene, it
will have golden coat whether or not the black-ness \texttt{Bb} gene is
expressed b/c the lack of colour expression.

\begin{center}
\includegraphics[width=.9\linewidth]{Pasted image 20210426133425.png}
\end{center}

\subsection{Sex-Linked Inheritance}
\label{sec:org072e011}
\begin{itemize}
\item Two X chromasomes: most women
\item XY chromasomes: most man
\end{itemize}

Because men usually only have one X chromasome, even if a sex-linked
mutation carries recessively, they do not have a chance of being
dominated. Examples of these include red-green colour blindness.
\end{document}
