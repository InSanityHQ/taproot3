% Created 2021-09-11 Sat 08:17
% Intended LaTeX compiler: xelatex
\documentclass[letterpaper]{article}
\usepackage{graphicx}
\usepackage{grffile}
\usepackage{longtable}
\usepackage{wrapfig}
\usepackage{rotating}
\usepackage[normalem]{ulem}
\usepackage{amsmath}
\usepackage{textcomp}
\usepackage{amssymb}
\usepackage{capt-of}
\usepackage{hyperref}
\usepackage[margin=1in]{geometry}
\usepackage{fontspec}
\usepackage{indentfirst}
\setmainfont[ItalicFont = LiberationSans-Italic, BoldFont = LiberationSans-Bold, BoldItalicFont = LiberationSans-BoldItalic]{LiberationSans}
\newfontfamily\NHLight[ItalicFont = LiberationSansNarrow-Italic, BoldFont       = LiberationSansNarrow-Bold, BoldItalicFont = LiberationSansNarrow-BoldItalic]{LiberationSansNarrow}
\newcommand\textrmlf[1]{{\NHLight#1}}
\newcommand\textitlf[1]{{\NHLight\itshape#1}}
\let\textbflf\textrm
\newcommand\textulf[1]{{\NHLight\bfseries#1}}
\newcommand\textuitlf[1]{{\NHLight\bfseries\itshape#1}}
\usepackage{fancyhdr}
\pagestyle{fancy}
\usepackage{titlesec}
\usepackage{titling}
\makeatletter
\lhead{\textbf{\@title}}
\makeatother
\rhead{\textrmlf{Compiled} \today}
\lfoot{\theauthor\ \textbullet \ \textbf{2021-2022}}
\cfoot{}
\rfoot{\textrmlf{Page} \thepage}
\titleformat{\section} {\Large} {\textrmlf{\thesection} {|}} {0.3em} {\textbf}
\titleformat{\subsection} {\large} {\textrmlf{\thesubsection} {|}} {0.2em} {\textbf}
\titleformat{\subsubsection} {\large} {\textrmlf{\thesubsubsection} {|}} {0.1em} {\textbf}
\setlength{\parskip}{0.45em}
\renewcommand\maketitle{}
\author{Houjun Liu}
\date{\today}
\title{Meiosis}
\hypersetup{
 pdfauthor={Houjun Liu},
 pdftitle={Meiosis},
 pdfkeywords={},
 pdfsubject={},
 pdfcreator={Emacs 27.2 (Org mode 9.4.4)}, 
 pdflang={English}}
\begin{document}

\maketitle


\section{Meiosis}
\label{sec:org1bf7251}
Meiosis is the process by which sex cells (gametes cells) are produced.
These cells have only 23 chromasomes (compared to somatic cell's 23
\emph{pairs}), and they contain a variety of mechanisms for genetic
variation.

Meiosis happens in two phases, which happens each in 4 phases:

\subsection{Meiosis 1}
\label{sec:org611601b}
The purpose of meiosis 1 is to take the 23 \emph{pairs} of 2-chromatid
chomasomes in germline cells (2n diploid, contains two sets of
homologous chromosomes) and mix them to separate into two cells
containing 23 singular 2-chromatid chromasomes (1n haploid, contains
only one set of genes).

\begin{itemize}
\item \textbf{(P)rophase 1}: the starting cell, a diploid, dissolves its nucleaus
and genetic information flows out. Also,
\href{KBhBIO101GeneticVariation.org}{KBhBIO101GeneticVariation} by
crossing over and independent assortment happens.
\item \textbf{(M)etaphase 1}: homogous PAIRS of chromosomes (\textbf{note!} pairs!!! not
the chromasomes) line up along the metaphase plate, forming a
double-filed lines
\item \textbf{(A)naphase 1}: seperate the homologous pairs to the opposite ends of
the cell
\item \textbf{(T)elophase 1}: the two new half-cells proceed to seperate further,
creating new nuclear envelopes enveloping the 23-unpaired sister
chromatids
\end{itemize}

\begin{figure}[htbp]
\centering
\includegraphics[width=.9\linewidth]{./Pasted image 20210426220455.png}
\caption{Pasted image 20210426220455.png}
\end{figure}

\subsection{Meiosis 2}
\label{sec:org242a360}
The 23 2-chromatid Chromasomes becomes seperated into two more cells
each with 23 1-chromatids. This is more similar to a good-ol
\href{KBhBIO101Mitosis.org}{KBhBIO101Mitosis}.

\begin{itemize}
\item \textbf{(P)rophase 2}: new spindles form, again! and the new haploids'
nuclear envelope will start dissoving
\item \textbf{(M)etaphase 2}: the sister chromatids (chromasomes) align themselves
along the metaphase plate, attaching themselves to the spindles
\item \textbf{(A)naphase 2}: spindles pull the sister chromatids away from each
other
\item \textbf{(T)elophase 2}: new nuclear envelope forms and the chromasomes
dissolves
\end{itemize}
\end{document}
