% Created 2021-09-11 Sat 09:35
% Intended LaTeX compiler: xelatex
\documentclass[letterpaper]{article}
\usepackage{graphicx}
\usepackage{grffile}
\usepackage{longtable}
\usepackage{wrapfig}
\usepackage{rotating}
\usepackage[normalem]{ulem}
\usepackage{amsmath}
\usepackage{textcomp}
\usepackage{amssymb}
\usepackage{capt-of}
\usepackage{hyperref}
\usepackage[margin=1in]{geometry}
\usepackage{fontspec}
\usepackage{indentfirst}
\setmainfont[ItalicFont = LiberationSans-Italic, BoldFont = LiberationSans-Bold, BoldItalicFont = LiberationSans-BoldItalic]{LiberationSans}
\newfontfamily\NHLight[ItalicFont = LiberationSansNarrow-Italic, BoldFont       = LiberationSansNarrow-Bold, BoldItalicFont = LiberationSansNarrow-BoldItalic]{LiberationSansNarrow}
\newcommand\textrmlf[1]{{\NHLight#1}}
\newcommand\textitlf[1]{{\NHLight\itshape#1}}
\let\textbflf\textrm
\newcommand\textulf[1]{{\NHLight\bfseries#1}}
\newcommand\textuitlf[1]{{\NHLight\bfseries\itshape#1}}
\usepackage{fancyhdr}
\pagestyle{fancy}
\usepackage{titlesec}
\usepackage{titling}
\makeatletter
\lhead{\textbf{\@title}}
\makeatother
\rhead{\textrmlf{Compiled} \today}
\lfoot{\theauthor\ \textbullet \ \textbf{2021-2022}}
\cfoot{}
\rfoot{\textrmlf{Page} \thepage}
\titleformat{\section} {\Large} {\textrmlf{\thesection} {|}} {0.3em} {\textbf}
\titleformat{\subsection} {\large} {\textrmlf{\thesubsection} {|}} {0.2em} {\textbf}
\titleformat{\subsubsection} {\large} {\textrmlf{\thesubsubsection} {|}} {0.1em} {\textbf}
\setlength{\parskip}{0.45em}
\renewcommand\maketitle{}
\author{Houjun Liu}
\date{\today}
\title{More PCR Stuff}
\hypersetup{
 pdfauthor={Houjun Liu},
 pdftitle={More PCR Stuff},
 pdfkeywords={},
 pdfsubject={},
 pdfcreator={Emacs 27.2 (Org mode 9.4.4)}, 
 pdflang={English}}
\begin{document}

\maketitle


\section{DNA Extraction}
\label{sec:orgbe5efc3}
\#flo

\begin{itemize}
\item DNA Extraction
\item Two main steps

\begin{itemize}
\item Breaking the cells open chemically, by breaking the membrane of the
cell. Some substance/method also has to break open the nucleus after
the previous one is done. => (Of course, you could also
acoustically, by ultrasonic waves to break, or mechanically, by like
breaking breaking it)

\begin{itemize}
\item The chemical usually used to do this is a \textbf{detergent}, like soap
\item Lipid bi-layer's hydrophobic tails becomes attracted to the
detergent, and it gets ripped apart
\item Or, changing the pH of the solution may also be one method by
which this could take place
\end{itemize}

\item Then, DNA is separated from other proteins \& debris => This is
usually done with either enzymes or using heat
\end{itemize}

\item How the QuickExtract possibly work

\begin{itemize}
\item A \textbf{detergent} breaks the outside lipid layer and the nucleus
\item Then, a \textbf{protease} chews up the enzyme and other random proteins
\item Finally, the whole thing's heated to make sure that the only thing
left is the DNA
\end{itemize}
\end{itemize}
\end{document}
