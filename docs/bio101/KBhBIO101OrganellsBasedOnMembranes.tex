% Created 2021-09-11 Sat 08:17
% Intended LaTeX compiler: xelatex
\documentclass[letterpaper]{article}
\usepackage{graphicx}
\usepackage{grffile}
\usepackage{longtable}
\usepackage{wrapfig}
\usepackage{rotating}
\usepackage[normalem]{ulem}
\usepackage{amsmath}
\usepackage{textcomp}
\usepackage{amssymb}
\usepackage{capt-of}
\usepackage{hyperref}
\usepackage[margin=1in]{geometry}
\usepackage{fontspec}
\usepackage{indentfirst}
\setmainfont[ItalicFont = LiberationSans-Italic, BoldFont = LiberationSans-Bold, BoldItalicFont = LiberationSans-BoldItalic]{LiberationSans}
\newfontfamily\NHLight[ItalicFont = LiberationSansNarrow-Italic, BoldFont       = LiberationSansNarrow-Bold, BoldItalicFont = LiberationSansNarrow-BoldItalic]{LiberationSansNarrow}
\newcommand\textrmlf[1]{{\NHLight#1}}
\newcommand\textitlf[1]{{\NHLight\itshape#1}}
\let\textbflf\textrm
\newcommand\textulf[1]{{\NHLight\bfseries#1}}
\newcommand\textuitlf[1]{{\NHLight\bfseries\itshape#1}}
\usepackage{fancyhdr}
\pagestyle{fancy}
\usepackage{titlesec}
\usepackage{titling}
\makeatletter
\lhead{\textbf{\@title}}
\makeatother
\rhead{\textrmlf{Compiled} \today}
\lfoot{\theauthor\ \textbullet \ \textbf{2021-2022}}
\cfoot{}
\rfoot{\textrmlf{Page} \thepage}
\titleformat{\section} {\Large} {\textrmlf{\thesection} {|}} {0.3em} {\textbf}
\titleformat{\subsection} {\large} {\textrmlf{\thesubsection} {|}} {0.2em} {\textbf}
\titleformat{\subsubsection} {\large} {\textrmlf{\thesubsubsection} {|}} {0.1em} {\textbf}
\setlength{\parskip}{0.45em}
\renewcommand\maketitle{}
\author{Houjun Liu}
\date{\today}
\title{Organizing Organelle Based on Membranes}
\hypersetup{
 pdfauthor={Houjun Liu},
 pdftitle={Organizing Organelle Based on Membranes},
 pdfkeywords={},
 pdfsubject={},
 pdfcreator={Emacs 27.2 (Org mode 9.4.4)}, 
 pdflang={English}}
\begin{document}

\maketitle


\section{Organizing Organelles Based on Membranes}
\label{sec:org23df127}
By organizing
\href{KBhBIO101EukaryoticOrganells.org}{KBhBIO101EukaryoticOrganells}
based on whether or not they have membranes, it helps us gauge the
evolutionary history of cells.

\subsection{Membranous Organelles}
\label{sec:org34b7cc8}
These have membranes! They have their own plasma, regulates their own
macromolecule consumption, hormones, etc. Based on
\href{KBhBIO101Endosymbiotic.org}{KBhBIO101Endosymbiotic}
endosymbiotic theory, double-memranous organelles may perhaps be the
organelles that were originally independent prokaryotic cells that
evolve to coexist with Eukarotes; by the same token, single-membraneous
organelles may be fragements of prokaryotic cells.

\subsubsection{Double Membranes}
\label{sec:org9f3c0bb}
\begin{itemize}
\item Mitochrondria => store ATP and extract energy from ATP
\item Chloroplasts => Does photosynthesis
\end{itemize}

\subsubsection{Double Membranes, Evolved Later}
\label{sec:orga9f8be9}
\begin{itemize}
\item Endoplasmic reticulum => forms the network of transferring proteins
and other elements
\item Golgi body/Gioli apparatus => packs, sorts, and modifies proteins and
other elements throughout the cell
\end{itemize}

\subsubsection{Single Membranes}
\label{sec:orgc1b41d7}
\begin{itemize}
\item Vesticles
\item Lysomoes => breaking stuff down and garbage dumps
\item Vacuoles => storing water, nutrients, waste
\end{itemize}

\subsection{Non-Membranous Organells}
\label{sec:orgadc6ca3}
These organelles does not process their own plasma, and they are mostly
part of the cytoskeleton of a cell.

\begin{itemize}
\item Ribosomes => protein synthesizer in the cell
\item Centrosome => forms flangella, cilla, and handles cells divisions
\item Plastids => creates colours displayed in the chromoplasts
\end{itemize}
\end{document}
