% Created 2021-09-11 Sat 09:35
% Intended LaTeX compiler: xelatex
\documentclass[letterpaper]{article}
\usepackage{graphicx}
\usepackage{grffile}
\usepackage{longtable}
\usepackage{wrapfig}
\usepackage{rotating}
\usepackage[normalem]{ulem}
\usepackage{amsmath}
\usepackage{textcomp}
\usepackage{amssymb}
\usepackage{capt-of}
\usepackage{hyperref}
\usepackage[margin=1in]{geometry}
\usepackage{fontspec}
\usepackage{indentfirst}
\setmainfont[ItalicFont = LiberationSans-Italic, BoldFont = LiberationSans-Bold, BoldItalicFont = LiberationSans-BoldItalic]{LiberationSans}
\newfontfamily\NHLight[ItalicFont = LiberationSansNarrow-Italic, BoldFont       = LiberationSansNarrow-Bold, BoldItalicFont = LiberationSansNarrow-BoldItalic]{LiberationSansNarrow}
\newcommand\textrmlf[1]{{\NHLight#1}}
\newcommand\textitlf[1]{{\NHLight\itshape#1}}
\let\textbflf\textrm
\newcommand\textulf[1]{{\NHLight\bfseries#1}}
\newcommand\textuitlf[1]{{\NHLight\bfseries\itshape#1}}
\usepackage{fancyhdr}
\pagestyle{fancy}
\usepackage{titlesec}
\usepackage{titling}
\makeatletter
\lhead{\textbf{\@title}}
\makeatother
\rhead{\textrmlf{Compiled} \today}
\lfoot{\theauthor\ \textbullet \ \textbf{2021-2022}}
\cfoot{}
\rfoot{\textrmlf{Page} \thepage}
\titleformat{\section} {\Large} {\textrmlf{\thesection} {|}} {0.3em} {\textbf}
\titleformat{\subsection} {\large} {\textrmlf{\thesubsection} {|}} {0.2em} {\textbf}
\titleformat{\subsubsection} {\large} {\textrmlf{\thesubsubsection} {|}} {0.1em} {\textbf}
\setlength{\parskip}{0.45em}
\renewcommand\maketitle{}
\author{Houjun Liu}
\date{\today}
\title{Sense and Antisense DNA/RNA}
\hypersetup{
 pdfauthor={Houjun Liu},
 pdftitle={Sense and Antisense DNA/RNA},
 pdfkeywords={},
 pdfsubject={},
 pdfcreator={Emacs 27.2 (Org mode 9.4.4)}, 
 pdflang={English}}
\begin{document}

\maketitle


\section{Sense and Antisense DNA/RNA}
\label{sec:org8b1543a}
Although DNA is usually complementary and RNA usually single-stranded,
they both have named complementary pairs that perform different
function. Here's a table to help decode the vocab and find the purpose
of each thing:

\begin{center}
\begin{tabular}{lll}
Type & Identifier & Purpose\\
\hline
DNA 3'\ldots{}5' & DNA Antisense/Noncoding/Template Strand & Used as a template for transcription\\
DNA 5'\ldots{}3' & DNA Sense/Coding/Nontemplate Strand & The complement to the template strand + what is being used (bar urisil) in RNA form to perform protein synthesis\\
RNA 3'\ldots{}5' & mRNA Antisense/-ss Strand & Pretty useless (unless you are trying to avoid detection) RNA strand that serve only as the template for the RNA that codes for a protein\\
RNA 5'\ldots{}3' & mRNA Sense/+ss Strand & RNA transcribed from the template antisense strand that codes for a protein\\
\end{tabular}
\end{center}
\end{document}
