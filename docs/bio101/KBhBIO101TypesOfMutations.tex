% Created 2021-09-11 Sat 09:35
% Intended LaTeX compiler: xelatex
\documentclass[letterpaper]{article}
\usepackage{graphicx}
\usepackage{grffile}
\usepackage{longtable}
\usepackage{wrapfig}
\usepackage{rotating}
\usepackage[normalem]{ulem}
\usepackage{amsmath}
\usepackage{textcomp}
\usepackage{amssymb}
\usepackage{capt-of}
\usepackage{hyperref}
\usepackage[margin=1in]{geometry}
\usepackage{fontspec}
\usepackage{indentfirst}
\setmainfont[ItalicFont = LiberationSans-Italic, BoldFont = LiberationSans-Bold, BoldItalicFont = LiberationSans-BoldItalic]{LiberationSans}
\newfontfamily\NHLight[ItalicFont = LiberationSansNarrow-Italic, BoldFont       = LiberationSansNarrow-Bold, BoldItalicFont = LiberationSansNarrow-BoldItalic]{LiberationSansNarrow}
\newcommand\textrmlf[1]{{\NHLight#1}}
\newcommand\textitlf[1]{{\NHLight\itshape#1}}
\let\textbflf\textrm
\newcommand\textulf[1]{{\NHLight\bfseries#1}}
\newcommand\textuitlf[1]{{\NHLight\bfseries\itshape#1}}
\usepackage{fancyhdr}
\pagestyle{fancy}
\usepackage{titlesec}
\usepackage{titling}
\makeatletter
\lhead{\textbf{\@title}}
\makeatother
\rhead{\textrmlf{Compiled} \today}
\lfoot{\theauthor\ \textbullet \ \textbf{2021-2022}}
\cfoot{}
\rfoot{\textrmlf{Page} \thepage}
\titleformat{\section} {\Large} {\textrmlf{\thesection} {|}} {0.3em} {\textbf}
\titleformat{\subsection} {\large} {\textrmlf{\thesubsection} {|}} {0.2em} {\textbf}
\titleformat{\subsubsection} {\large} {\textrmlf{\thesubsubsection} {|}} {0.1em} {\textbf}
\setlength{\parskip}{0.45em}
\renewcommand\maketitle{}
\author{Houjun Liu}
\date{\today}
\title{Types of Mutations}
\hypersetup{
 pdfauthor={Houjun Liu},
 pdftitle={Types of Mutations},
 pdfkeywords={},
 pdfsubject={},
 pdfcreator={Emacs 27.2 (Org mode 9.4.4)}, 
 pdflang={English}}
\begin{document}

\maketitle


\section{Types of Mutations}
\label{sec:org527adeb}
\subsection{By Place}
\label{sec:orgda2fbc4}
\textbf{Germline mutations} mutate the egg/cell's causes no/local problems but
pass the mutated gene down to the children fully through gametes.

\textbf{Somatic mutations} mutated somatic cell causes local mutations that
does not influence much (cancer, but like shhh).

\subsection{By Method}
\label{sec:org1711bb4}
\subsubsection{Point mutations}
\label{sec:org8a20541}
Change one codon on the gene and potentially cause something.

\begin{itemize}
\item Slient mutation: has no effect on protein
\item Missense: result in amino acid substitution
\item Nonsense: substitutes a stop codon for an amino acid
\end{itemize}

\subsubsection{Indel/Frameshift mutation}
\label{sec:orged1b9ac}
Shift by adding/substracting codons and shift the gene. Everything
downstream to the point of mutation will be completely incorrect.

\subsection{Mutations in other places}
\label{sec:orgf1b07d5}
\textbf{Promoter/Enhancer mutation}: control the level of expression for genes,
which could relate to cancer (over-activation) or a protein deficiency
(lack of activation)

\textbf{Splice donor and acceptor site mutation}: including extra intron or
cutting out required exon

\textbf{Ribosome binding sites}: prevents the ribosome from binding

\subsection{Large scale DNA changes}
\label{sec:org7110fd4}
Taking whole chunks of DNA or swapping them; usually caused by your DNA
wholly breaking (Radioactivity? Incorrectly functioning enzymes?) and
then your repair machinary stitching it up wrongly.

\begin{center}
\includegraphics[width=.9\linewidth]{Pasted image 20210423135639.png}
\end{center}
\end{document}
