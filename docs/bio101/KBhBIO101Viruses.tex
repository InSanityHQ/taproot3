% Created 2021-09-11 Sat 08:17
% Intended LaTeX compiler: xelatex
\documentclass[letterpaper]{article}
\usepackage{graphicx}
\usepackage{grffile}
\usepackage{longtable}
\usepackage{wrapfig}
\usepackage{rotating}
\usepackage[normalem]{ulem}
\usepackage{amsmath}
\usepackage{textcomp}
\usepackage{amssymb}
\usepackage{capt-of}
\usepackage{hyperref}
\usepackage[margin=1in]{geometry}
\usepackage{fontspec}
\usepackage{indentfirst}
\setmainfont[ItalicFont = LiberationSans-Italic, BoldFont = LiberationSans-Bold, BoldItalicFont = LiberationSans-BoldItalic]{LiberationSans}
\newfontfamily\NHLight[ItalicFont = LiberationSansNarrow-Italic, BoldFont       = LiberationSansNarrow-Bold, BoldItalicFont = LiberationSansNarrow-BoldItalic]{LiberationSansNarrow}
\newcommand\textrmlf[1]{{\NHLight#1}}
\newcommand\textitlf[1]{{\NHLight\itshape#1}}
\let\textbflf\textrm
\newcommand\textulf[1]{{\NHLight\bfseries#1}}
\newcommand\textuitlf[1]{{\NHLight\bfseries\itshape#1}}
\usepackage{fancyhdr}
\pagestyle{fancy}
\usepackage{titlesec}
\usepackage{titling}
\makeatletter
\lhead{\textbf{\@title}}
\makeatother
\rhead{\textrmlf{Compiled} \today}
\lfoot{\theauthor\ \textbullet \ \textbf{2021-2022}}
\cfoot{}
\rfoot{\textrmlf{Page} \thepage}
\titleformat{\section} {\Large} {\textrmlf{\thesection} {|}} {0.3em} {\textbf}
\titleformat{\subsection} {\large} {\textrmlf{\thesubsection} {|}} {0.2em} {\textbf}
\titleformat{\subsubsection} {\large} {\textrmlf{\thesubsubsection} {|}} {0.1em} {\textbf}
\setlength{\parskip}{0.45em}
\renewcommand\maketitle{}
\author{Houjun Liu}
\date{\today}
\title{Viruses}
\hypersetup{
 pdfauthor={Houjun Liu},
 pdftitle={Viruses},
 pdfkeywords={},
 pdfsubject={},
 pdfcreator={Emacs 27.2 (Org mode 9.4.4)}, 
 pdflang={English}}
\begin{document}

\maketitle


\section{Viruses}
\label{sec:org1bd1b43}
\definition{Viruses}{Acellular Macromolecular Assemblies}
Viruses\ldots{}

\begin{itemize}
\item \ldots{}contain protein coat called \textbf{capsid}
\item \ldots{}use DNA or RNA, but not both
\item \ldots{}are obligate parasites that could only replicate within host
\end{itemize}

Assembled and mature viral particles => \textbf{virions}. They usually have
three different parts

\begin{enumerate}
\item Capsid --- the protein coat
\item Genetic material --- what they are transmitting/replicating
\item Occationally outside lipid layer
\end{enumerate}

=> Viruses exist on the nanometre scale, but they are difference in
share and size

\subsection{Structure of Viruses}
\label{sec:org98fa7c0}
See
\href{KBhBIO101StructureOfViruses.org}{KBhBIO101StructureOfViruses}

\subsection{Types of Viruses}
\label{sec:org5a940bc}
Two types of viruses: the prokaryote-frequenting \textbf{DNA viruses} which
replicates themselves using DNA and the eukaryote-frequenting \textbf{RNA
viruses} which replicates themselves using RNA.

See \href{KBhBIO101TypesOfViruses.org}{KBhBIO101TypesOfViruses}

\subsection{Virus Lifecycle + Infectivity}
\label{sec:org855c870}
How do viruses infect people? Basically, they come into your body,
hijack the \href{KBhBIO101CentralDogma.org}{KBhBIO101CentralDogma}
system of your body, and leverage it to create more copies of itself.

To see more about this, head on over to
\href{KBhBIO101ViralInfection.org}{KBhBIO101ViralInfection}. This is
important and cool.

\subsection{Viral Genetic Shift + Viral Genetic Drift}
\label{sec:orgff3bd83}
Viruses modulate themselves, which make them particularly hard to deal
with as their DNA may change every so often to the bewilderment of the
immune system.

There are two ways by which this happens --- genetic Shift and Genetic
Drift. See
\href{KBhBIO101ViralGeneticModulationMutation.org}{KBhBIO101ViralGeneticModulationMutation}

\subsection{Retroviruses}
\label{sec:org62dc493}
Viruses are special types of viruses that not only infect people, but
also hijack cell DNA by inserting their own genetic code into them. They
are particularly terrible because they cause the infected cell and its
offsprings to inadventantly create more copies of the virus slowly as
daily \href{KBhBIO101CentralDogma.org}{KBhBIO101CentralDogma} happens.

See \href{KBhBIO101Retroviruses.org}{KBhBIO101Retroviruses}

\subsection{Viruses damaging host}
\label{sec:org405f040}
Viruses are terrible because they damage the infected host
cells/tissues, namely by\ldots{}

\begin{itemize}
\item Reducing gene expression capacity (hogging up the
\href{KBhBIO101CentralDogma.org}{KBhBIO101CentralDogma} channels to
achieve higher probability of replication)
\item Depleting cellular resources (needing to transcribe a thing that eat
you)
\item Causing cell lysis (to explode, which is bad)
\item Promoting tumorigenisis --- cancer (damaging promoters, among others)
\item Creating damaging immunological response (over-compensate to kill
viruses a la Ebola)
\end{itemize}

\subsection{Preventing Viruses}
\label{sec:org5561fdb}
See \href{KBhBIO101AntiViralDrugs.org}{KBhBIO101AntiViralDrugs}
\end{document}
