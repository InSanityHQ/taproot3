% Created 2021-09-11 Sat 09:35
% Intended LaTeX compiler: xelatex
\documentclass[letterpaper]{article}
\usepackage{graphicx}
\usepackage{grffile}
\usepackage{longtable}
\usepackage{wrapfig}
\usepackage{rotating}
\usepackage[normalem]{ulem}
\usepackage{amsmath}
\usepackage{textcomp}
\usepackage{amssymb}
\usepackage{capt-of}
\usepackage{hyperref}
\usepackage[margin=1in]{geometry}
\usepackage{fontspec}
\usepackage{indentfirst}
\setmainfont[ItalicFont = LiberationSans-Italic, BoldFont = LiberationSans-Bold, BoldItalicFont = LiberationSans-BoldItalic]{LiberationSans}
\newfontfamily\NHLight[ItalicFont = LiberationSansNarrow-Italic, BoldFont       = LiberationSansNarrow-Bold, BoldItalicFont = LiberationSansNarrow-BoldItalic]{LiberationSansNarrow}
\newcommand\textrmlf[1]{{\NHLight#1}}
\newcommand\textitlf[1]{{\NHLight\itshape#1}}
\let\textbflf\textrm
\newcommand\textulf[1]{{\NHLight\bfseries#1}}
\newcommand\textuitlf[1]{{\NHLight\bfseries\itshape#1}}
\usepackage{fancyhdr}
\pagestyle{fancy}
\usepackage{titlesec}
\usepackage{titling}
\makeatletter
\lhead{\textbf{\@title}}
\makeatother
\rhead{\textrmlf{Compiled} \today}
\lfoot{\theauthor\ \textbullet \ \textbf{2021-2022}}
\cfoot{}
\rfoot{\textrmlf{Page} \thepage}
\titleformat{\section} {\Large} {\textrmlf{\thesection} {|}} {0.3em} {\textbf}
\titleformat{\subsection} {\large} {\textrmlf{\thesubsection} {|}} {0.2em} {\textbf}
\titleformat{\subsubsection} {\large} {\textrmlf{\thesubsubsection} {|}} {0.1em} {\textbf}
\setlength{\parskip}{0.45em}
\renewcommand\maketitle{}
\author{Houjun Liu}
\date{\today}
\title{mRNA preprocessing}
\hypersetup{
 pdfauthor={Houjun Liu},
 pdftitle={mRNA preprocessing},
 pdfkeywords={},
 pdfsubject={},
 pdfcreator={Emacs 27.2 (Org mode 9.4.4)}, 
 pdflang={English}}
\begin{document}

\maketitle


\section{mRNA Pre-Processing}
\label{sec:org189e939}
Between Promoter and Terminator, \textbf{Exon} and \textbf{Intron} alternate. Exon is
coding, whereas Intron is non-coding and works as metadata.

After reading the intron, they are spliced out during mRNA processing =>
done by the "splicesome". The mRNA, after splicing, is "capped and
tailed" to mark pre-processing completion, at which point they leave the
nucleus + go to the ribosome.

\subsection{Slicing out the non-coding parts}
\label{sec:orge619bff}
\begin{itemize}
\item Begin by assembling helper proteins at intron-exon borders => "slicing
factors"
\item Other helping factor proteins come together and form the "splicesome"
to do the splicing
\item Splicesome splices by bringing exon ends together
\item After it's done, the splicesome disintergrates
\end{itemize}

\subsection{Marking for Maturity}
\label{sec:org647f981}
After the slicing is done, each finished mRNA is marked for maturity:

\begin{itemize}
\item 3' end => AAAAAA tail (using poly-adenine tailing enzyme)
\item 5' end => GGGGGG cap (using guanine-capping enzyme)
\end{itemize}
\end{document}
