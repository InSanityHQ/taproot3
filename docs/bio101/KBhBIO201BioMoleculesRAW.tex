% Created 2021-09-11 Sat 08:17
% Intended LaTeX compiler: xelatex
\documentclass[letterpaper]{article}
\usepackage{graphicx}
\usepackage{grffile}
\usepackage{longtable}
\usepackage{wrapfig}
\usepackage{rotating}
\usepackage[normalem]{ulem}
\usepackage{amsmath}
\usepackage{textcomp}
\usepackage{amssymb}
\usepackage{capt-of}
\usepackage{hyperref}
\usepackage[margin=1in]{geometry}
\usepackage{fontspec}
\usepackage{indentfirst}
\setmainfont[ItalicFont = LiberationSans-Italic, BoldFont = LiberationSans-Bold, BoldItalicFont = LiberationSans-BoldItalic]{LiberationSans}
\newfontfamily\NHLight[ItalicFont = LiberationSansNarrow-Italic, BoldFont       = LiberationSansNarrow-Bold, BoldItalicFont = LiberationSansNarrow-BoldItalic]{LiberationSansNarrow}
\newcommand\textrmlf[1]{{\NHLight#1}}
\newcommand\textitlf[1]{{\NHLight\itshape#1}}
\let\textbflf\textrm
\newcommand\textulf[1]{{\NHLight\bfseries#1}}
\newcommand\textuitlf[1]{{\NHLight\bfseries\itshape#1}}
\usepackage{fancyhdr}
\pagestyle{fancy}
\usepackage{titlesec}
\usepackage{titling}
\makeatletter
\lhead{\textbf{\@title}}
\makeatother
\rhead{\textrmlf{Compiled} \today}
\lfoot{\theauthor\ \textbullet \ \textbf{2021-2022}}
\cfoot{}
\rfoot{\textrmlf{Page} \thepage}
\titleformat{\section} {\Large} {\textrmlf{\thesection} {|}} {0.3em} {\textbf}
\titleformat{\subsection} {\large} {\textrmlf{\thesubsection} {|}} {0.2em} {\textbf}
\titleformat{\subsubsection} {\large} {\textrmlf{\thesubsubsection} {|}} {0.1em} {\textbf}
\setlength{\parskip}{0.45em}
\renewcommand\maketitle{}
\author{Houjun Liu}
\date{\today}
\title{Biomolecules RAW}
\hypersetup{
 pdfauthor={Houjun Liu},
 pdftitle={Biomolecules RAW},
 pdfkeywords={},
 pdfsubject={},
 pdfcreator={Emacs 27.2 (Org mode 9.4.4)}, 
 pdflang={English}}
\begin{document}

\maketitle


\section{Biomolecules quiz review, raw}
\label{sec:org27b383e}
\subsection{Carbohydrates}
\label{sec:org7a2098d}
\begin{itemize}
\item Set 1, carbs. See Luke De's video +
\href{KBhBIO101Carbs.org}{KBhBIO101Carbs}

\begin{itemize}
\item \emph{Glucose vs. fructose} --- both monosacharrides, one is a 6-carbon
ring and one is a 5-carbon ring
\item \emph{Mono vs. di. vs. polysaccharide} --- carbohydrates made out of a
single, double, and multiple monomer (single-unit) carbohydrates
\item \emph{Starch vs. glycogen vs. cellulose} --- lots of alpha glucose in
less branches, lots of alpha glucose in more branches, lots of beta
glucose in organized lattice respectively.

\begin{itemize}
\item Starch --- plant food reserve
\item Glycogen --- animal energy reserve
\item Cellulose --- cell wall in plants
\end{itemize}
\end{itemize}

\item Set 2, lipids. See Luke De's video +
\href{KBhBIO101Lipids.org}{KBhBIO101Lipids}

\begin{itemize}
\item \emph{Triglyceride vs. fatty acid vs. phosophilid} see
\href{KBhBIO101StructuresofCarbs.org}{KBhBIO101StructuresofCarbs}

\begin{itemize}
\item Glycerol => a fatty acid
\item Triglyceride => three of 'em above
\item Phospholipid => two fatty acid + phosphate head
\end{itemize}

\item \emph{Saturated vs unsaturated fatty acids} see also
\href{KBhBIO101StructuresofCarbs.org}{KBhBIO101StructuresofCarbs}

\begin{itemize}
\item Saturated Fats => no double bonds in the carbon chain of fatty
acids --- think! butter
\item Unsaturated Fats => double bonds in the carbon chain of fatty
acids --- think! olive oil
\end{itemize}
\end{itemize}

\item Identify functional groups

\begin{itemize}
\item Amino acid groups --- see
\href{KBhBIO101AminoAcids.org}{KBhBIO101AminoAcids}

\begin{itemize}
\item carboxyl/carboxylic acid --- H-O-C=O (left side of backbone)
\href{Screen Shot 2020-10-12 at 2.29.28 PM.org}{Screen Shot
2020-10-12 at 2.29.28 PM}
\item carbonyl --- C=O --- part of carboxyl
\item amide --- RC(=O)NR′R″ (frequently shown in side chains of amino
acids --- see
\href{https://en.wikipedia.org/wiki/Amide\#/media/File:Amide-general.png}{Amine})
\item amino/amine --- H3N+ (right side of backbone)
\item hydroxyl --- OH group. Need I say more?
\item ester --- take a carboxylic acid and replace the hydrogen with
anything else \#ASK. => What join fatty acid chains with the
glycerol to make trigrcyeride
\item ether --- R-O-R structure. => glycocidic bonds are formed by ether
bonds
\item alcohol group (H-O-R) as part of the carboxyl
\end{itemize}
\end{itemize}

\item Monomers vs Polymers
\href{KBhBIO101StructuresofCarbs.org}{KBhBIO101StructuresofCarbs}

\begin{itemize}
\item Monomer --- single molecule (such as a monosacchride) that could be
chained together to make polymers
\item Polymers --- complex molecues built from monomers
\item Building polymers --- dehydration reaction --- taking out water
molecules
\item Destructing polymers --- hydration reaction --- adding in water
molecules
\end{itemize}
\end{itemize}

\subsection{Cell Structures}
\label{sec:org11e31fa}
\begin{itemize}
\item Prokaryotic vs. Eukaryotic

\begin{itemize}
\item Prokaryotic cells --- often in single-cellular cells, has a cell
wall, and contained in capsules
\item Eukaryotic cells --- in multicellular cell elements, contains a
plasma membranes and nucleus
\end{itemize}

\item Compare and contrast a typical animal cell with a typical plant cell.
Be able to label diagrams of each. (See\ldots{} problem set 1)

\begin{itemize}
\item Animal Cell

\begin{itemize}
\item No cell wall
\item No chloroplast
\item Has Cytoplasm
\item Has Ribosomes
\item Has Mitochondria
\item No plastids --- organelle pigments
\item Has Cilla --- Hair-like items on the outer surface
\end{itemize}

\item Plant Cell

\begin{itemize}
\item Has cell wall
\item Has chloroplast --- photosynthesis
\item Has cytoplasm
\item Has Ribosomes
\item Has Mitochondria
\item Has plastics --- organelle pigments
\item Mostly has no Cilla
\end{itemize}
\end{itemize}

\item Endosymbiotic theory

\begin{itemize}
\item Endosymbiotic theory states that organelles within our current
eukaryotic cells --- the mitochondria and chloroplasts --- are
originally prokaryotic cells in their own right. This is because
they divide independently through binary fission, and also contains
circular DNA that is independent of the main cell itself.
\end{itemize}

\item Organizing organelles based on membranes \#ASK

\begin{itemize}
\item Used as a gauge to sort the evolutional history of cells
\item Membranous organelles --- possess own plasma => regulates own
macromolecure consumption, hormones, etc. Perhaps original
prokarotic cells

\begin{itemize}
\item Double membranes, evolved later

\begin{itemize}
\item Endoplasmic reticulum => forms the network of transferring
proteins and other elements
\item Golgi body/Gioli apparatus => packs, sorts, and modifies
proteins and other elements throughout the cell
\end{itemize}

\item Double membranes, prokarotic orginially

\begin{itemize}
\item Mitochrondria => store ATP and extract energy from ATP
\item Chloroplasts => Does photosynthesis
\end{itemize}

\item Single membranes => probably originally fragments of prokaryotic
cells

\begin{itemize}
\item Vesticles
\item Lysomoes => breaking stuff down and garbage dumps
\item Vacuoles => storing water, nutrients, waste
\end{itemize}
\end{itemize}

\item Non-membranous organelles --- does not posess own plasma => mostly
part of the cytoskeleton of a cell

\begin{itemize}
\item Ribosomes => protein synthesizer in the cell
\item Centrosome => forms flangella, cilla, and handles cells divisions
\item Plastids => creates colours displayed in the chromoplasts
\end{itemize}
\end{itemize}

\item Cell Components. Basicall all of these exist only in Eukareotic cells

\begin{itemize}
\item chloroplast and mitochondria

\begin{itemize}
\item Chloroplast --- found in plants + does photosynthesis
\item Mitochondria --- found in animals + store ATP and extract energy
from ATP
\end{itemize}

\item cell wall and plasma membrane

\begin{itemize}
\item Cell Wall --- found in plants => surround the cell: hard
\item Plasma membrane --- found in animals => surround the cell: soft
\href{KBhBIO101Lipids.org}{KBhBIO101Lipids}
\end{itemize}

\item rough endoplasmic reticulum (ER) and smooth ER

\begin{itemize}
\item Rough ER --- covered by ribosomes and folds
\href{KBhBIO101Proteins.org}{KBhBIO101Proteins}
\item Smooth ER --- not covered by ribosomes and makes
\href{KBhBIO101Lipids.org}{KBhBIO101Lipids}
\end{itemize}

\item cytosol, cytoplasm and cytoskeleton

\begin{itemize}
\item Cytosol => liquid found inside cells; the "cytoplasm" floats
within it
\item Cytoplasm => all the stuff within the cell
\item Cytoskeleton => complex network of proteins + fibres that organize
the rest of the cell
\end{itemize}

\item nucleus and nucleolus

\begin{itemize}
\item nucleus => centre of the cell, stores DNA
\item nucleolus => largest part of the nucleous that makes ribosomes
\end{itemize}

\item lysosomes and food vacuoles

\begin{itemize}
\item Lysosomes => vesticles that contains enzymes that breaks down
biomolecules
\item Food Vacoules => vesticels that stores food and other resources
\end{itemize}

\item cytoskeleton and microtubules

\begin{itemize}
\item Cytoskeleton => complex network of proteins + fibres that organize
the rest of the cell
\item Microtubulues => Polymers of tubulin protein that provides the
main structure of eukarotic cells
\end{itemize}

\item flagella and cilia

\begin{itemize}
\item Flagella => a bacteria's tail --- allow them to move and also act
as an sensory organ. longer than a cilla, and moves in sinosoidial
pattern.
\item Cilium => a cell's "hair" --- provides sensory and communications
functions. Motil cilla could move about to "grab" things, and
non-motile cilla can't move. more abundant that the flagella, and
moves in circular pattern if they do move, and moves in circular
pattern if they do move
\end{itemize}

\item Ribosomes and Golgi apparatus

\begin{itemize}
\item Ribosomes => synthesizes proteins
\item Golgi apparatus => packs, modifying, and moving proteins
\end{itemize}
\end{itemize}
\end{itemize}

\subsection{Plasma Membrane Structure + transport}
\label{sec:orga6239c5}
\begin{itemize}
\item Lipid structure and substructures:
\href{KBhBIO101Lipids.org}{KBhBIO101Lipids}
\item Functions of cell membrane

\begin{itemize}
\item Phosophilid structures
\href{KBhBIO101StructuresOfLipids.org}{KBhBIO101StructuresOfLipids}
\item Transmembrane proteins
\href{KbhBIO101CellTransport.org}{KbhBIO101CellTransport}
\item Hydrophobic + hydrophillic parts of a phosophilid
\href{KBhBIO101StructuresOfLipids.org}{KBhBIO101StructuresOfLipids}
\begin{itemize}
\item \href{KBhBIO101FluidMosaic.org}{KBhBIO101FluidMosaic}
\end{itemize}
\end{itemize}

\item Passive + active transport
\href{KbhBIO101CellTransport.org}{KbhBIO101CellTransport}
\item Cell transport process

\begin{itemize}
\item Simple diffusion => things just spread out from high concentration
to low concentrations
\item Passive diffusion => non-polar molecules needed "fall in" through
the phosolipid bi-layer
\item Facilitated diffusion => specific polar molecules go along the
gradient to get into the cell through transporter proteins. Osmosis
is the facilitated diffusion, just of water + auquaporin.
\item Phagocytosis => take a piece of the membrane with you to form a
vesticle to introduce large solid elements, recycling the membrane
after done --- "cell eating"
\item Pinocytosis => take a piece of the membrane with you to form a
vesticle to introduce large area of the "outside" in --- fluid and
solid and all, recycling the membrane after done --- "cell drinking"
\item Endocytosis => Phagocytosis + Pinocytosis
\item Extocytosis => opposite of endocytosis
\end{itemize}

\item Defining\ldots{}

\begin{itemize}
\item Isotonic => inside and outside have the same level of "osmolarity":
probablility for osmosis to happen through a semipermiable membrane
\item Hypertonic => inside has less osmolarity than the outside:
water/other elems will flow out of the cell
\item Hypotonic => outside has less osmolarity than the inside:
water/other elems will flow into the cell
\end{itemize}
\end{itemize}

\subsection{Proteins Structures and Function}
\label{sec:org8adcec1}
\begin{itemize}
\item Overall structure, monomers/building blocks, functions, and examples
of proteins => \href{KBhBIO101Proteins.org}{KBhBIO101Proteins}
\item "peptide" => a chain of amino-acids
\item Polymerization via dehydration

\begin{itemize}
\item Take two amino acids, take the H-O out of the alcahol, take the H
out of the Amine. Fill the hole with the other one
\end{itemize}

\item Protein structure

\begin{itemize}
\item Primary structure, secondary structure =>
\href{KBhBIO101Proteins.org}{KBhBIO101Proteins}
\item Amino acids, N \& C terminus =>
\href{KBhBIO101AminoAcids.org}{KBhBIO101AminoAcids}. N terminus
(Amine), C terminus (Carbolixic.)
\item Secondary structure --- H bonds between H-O, H-N
\item Tertiary structure => see the
\href{KBhBIO101Proteins.org}{KBhBIO101Proteins} articles
\end{itemize}

\item The functions of proteins are varied because the primary sequence can
be varied, effectively building any shape protein to do its specific
function
\item Form = function is the idea that the shape or form a protein takes
through the combination of primary, secondary, tertiary, or quaternary
structure determines how it will then function. Any changes to the
structure will have some impact on its function and the more the
structure is affected the more the function is likely to impacted
\item Functions => defense, movement, structure, transport, cell to cell
signaling, etc.
\end{itemize}

\subsection{Cell Structure}
\label{sec:orgc6ee271}
\begin{itemize}
\item Enzymes? \href{KBhBIO101Enzymes.org}{KBhBIO101Enzymes}
\end{itemize}

OK, so. Apparently Paul just answered the rest of his questions.

And I quote

““”

*Enzymes are catalysts. They speed reaction rates but do not affect the
change in free energy of the reaction (the difference in potential
energy between reactants and products).*

\begin{itemize}
\item Activation energy is the amount of kinetic energy required to reach
the transition state of a reaction.

\item Enzymes speed up a reaction by lowering the activation energy, often
with the help of cofactors or coenzymes.

\item \textbf{Enzymes lower the activation energy by some combination of\ldots{}}

\begin{itemize}
\item Orienting the reactions substrate(s) to promote more effective
collisions (and therefore reactions
\item Stressing or straining bonds to temporarily and/or slightly lower
the strength of attraction to allow the bond to break more easily
\item Involving amino acid R-groups or sidechains in creating the
transition state between reactants and products
\end{itemize}
\end{itemize}

Enzymes have active sites that bring substrates together and may change
shape to stabilize the transition state; known as Induced Fit upon
binding active site and slight change in enzyme shape.

Most enzymes are proteins, and thus their activity can be directly
influenced by modifications or environmental factors, such as
temperature and pH, that alter their three-dimensional structure.

Enzyme activity may be regulated/inhibited by molecules that compete
with substrates to occupy the active site (competitive inhibitor) or
alter enzyme shape so that substrates become unable to enter the active
site (non-competitive inhibitor).

““”
\end{document}
