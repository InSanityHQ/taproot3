% Created 2021-09-11 Sat 09:35
% Intended LaTeX compiler: xelatex
\documentclass[letterpaper]{article}
\usepackage{graphicx}
\usepackage{grffile}
\usepackage{longtable}
\usepackage{wrapfig}
\usepackage{rotating}
\usepackage[normalem]{ulem}
\usepackage{amsmath}
\usepackage{textcomp}
\usepackage{amssymb}
\usepackage{capt-of}
\usepackage{hyperref}
\usepackage[margin=1in]{geometry}
\usepackage{fontspec}
\usepackage{indentfirst}
\setmainfont[ItalicFont = LiberationSans-Italic, BoldFont = LiberationSans-Bold, BoldItalicFont = LiberationSans-BoldItalic]{LiberationSans}
\newfontfamily\NHLight[ItalicFont = LiberationSansNarrow-Italic, BoldFont       = LiberationSansNarrow-Bold, BoldItalicFont = LiberationSansNarrow-BoldItalic]{LiberationSansNarrow}
\newcommand\textrmlf[1]{{\NHLight#1}}
\newcommand\textitlf[1]{{\NHLight\itshape#1}}
\let\textbflf\textrm
\newcommand\textulf[1]{{\NHLight\bfseries#1}}
\newcommand\textuitlf[1]{{\NHLight\bfseries\itshape#1}}
\usepackage{fancyhdr}
\pagestyle{fancy}
\usepackage{titlesec}
\usepackage{titling}
\makeatletter
\lhead{\textbf{\@title}}
\makeatother
\rhead{\textrmlf{Compiled} \today}
\lfoot{\theauthor\ \textbullet \ \textbf{2021-2022}}
\cfoot{}
\rfoot{\textrmlf{Page} \thepage}
\titleformat{\section} {\Large} {\textrmlf{\thesection} {|}} {0.3em} {\textbf}
\titleformat{\subsection} {\large} {\textrmlf{\thesubsection} {|}} {0.2em} {\textbf}
\titleformat{\subsubsection} {\large} {\textrmlf{\thesubsubsection} {|}} {0.1em} {\textbf}
\setlength{\parskip}{0.45em}
\renewcommand\maketitle{}
\author{Exr0n}
\date{\today}
\title{Voltage in Cells}
\hypersetup{
 pdfauthor={Exr0n},
 pdftitle={Voltage in Cells},
 pdfkeywords={},
 pdfsubject={},
 pdfcreator={Emacs 27.2 (Org mode 9.4.4)}, 
 pdflang={English}}
\begin{document}

\maketitle
<T
\#flo \#ref \#disorganized \#incomplete

\section{Lipids}
\label{sec:org49e100e}

\#toexpand

\subsection{Self assembly}
\label{sec:org495c2ae}

\subsection{Construction}
\label{sec:org3b94b13}

\subsection{Fatty acids}
\label{sec:orga58304a}
\begin{itemize}
\item caboxylic acid
\item fatty part is a hydrocarbon
\item connected to the head by an ester linkage using a dehydration synthesis reaction
\item energy storage molecule
\end{itemize}

\subsubsection{Saturation vs unsaturated}
\label{sec:org15b4fb6}
\begin{itemize}
\item If it's saturated, then everything lines up nicely
\item unsaturated fatty acids have a kink (carbon doublebond)
\begin{itemize}
\item harder to pack and flow more smoothly
\end{itemize}
\end{itemize}

\subsection{Phospholipids}
\label{sec:orge605e0e}
\begin{itemize}
\item like normal lipids but with a mmore polar head
\item one saturated tail and one unsaturated tail
\item bilayers separate water from outside and inside
\end{itemize}

\subsubsection{Assembly}
\label{sec:org4242861}
\begin{itemize}
\item Liposome: bilayer, hydrophilic inside, layer of hydrophobic tails
\item Micelle: one layer, hydrophobic inside
\end{itemize}

\begin{enumerate}
\item Assembly structure depends on pH
\label{sec:org8613b6c}
\begin{itemize}
\item High pH = low protons available
\begin{itemize}
\item thus, the charged head is negatively ionized
\item Thus, the hydrophobic tails attract eachother by water exclusion and charged heads repel eachother
\end{itemize}
\item pH around 8.5 means half and half, which means that charged heads are likely to attract eachother
\end{itemize}
\end{enumerate}

\section{Voltage in Cells}
\label{sec:org8ca5145}
\end{document}
