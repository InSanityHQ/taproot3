% Created 2021-09-11 Sat 08:17
% Intended LaTeX compiler: xelatex
\documentclass[letterpaper]{article}
\usepackage{graphicx}
\usepackage{grffile}
\usepackage{longtable}
\usepackage{wrapfig}
\usepackage{rotating}
\usepackage[normalem]{ulem}
\usepackage{amsmath}
\usepackage{textcomp}
\usepackage{amssymb}
\usepackage{capt-of}
\usepackage{hyperref}
\usepackage[margin=1in]{geometry}
\usepackage{fontspec}
\usepackage{indentfirst}
\setmainfont[ItalicFont = LiberationSans-Italic, BoldFont = LiberationSans-Bold, BoldItalicFont = LiberationSans-BoldItalic]{LiberationSans}
\newfontfamily\NHLight[ItalicFont = LiberationSansNarrow-Italic, BoldFont       = LiberationSansNarrow-Bold, BoldItalicFont = LiberationSansNarrow-BoldItalic]{LiberationSansNarrow}
\newcommand\textrmlf[1]{{\NHLight#1}}
\newcommand\textitlf[1]{{\NHLight\itshape#1}}
\let\textbflf\textrm
\newcommand\textulf[1]{{\NHLight\bfseries#1}}
\newcommand\textuitlf[1]{{\NHLight\bfseries\itshape#1}}
\usepackage{fancyhdr}
\pagestyle{fancy}
\usepackage{titlesec}
\usepackage{titling}
\makeatletter
\lhead{\textbf{\@title}}
\makeatother
\rhead{\textrmlf{Compiled} \today}
\lfoot{\theauthor\ \textbullet \ \textbf{2021-2022}}
\cfoot{}
\rfoot{\textrmlf{Page} \thepage}
\titleformat{\section} {\Large} {\textrmlf{\thesection} {|}} {0.3em} {\textbf}
\titleformat{\subsection} {\large} {\textrmlf{\thesubsection} {|}} {0.2em} {\textbf}
\titleformat{\subsubsection} {\large} {\textrmlf{\thesubsubsection} {|}} {0.1em} {\textbf}
\setlength{\parskip}{0.45em}
\renewcommand\maketitle{}
\author{Huxley}
\date{\today}
\title{Genetic variation synthesis questions}
\hypersetup{
 pdfauthor={Huxley},
 pdftitle={Genetic variation synthesis questions},
 pdfkeywords={},
 pdfsubject={},
 pdfcreator={Emacs 27.2 (Org mode 9.4.4)}, 
 pdflang={English}}
\begin{document}

\maketitle
\#ref \#ret

\noindent\rule{\textwidth}{0.5pt}

\section{Questions}
\label{sec:orgc9383a4}
You may discuss the following questions in groups but please produce
your own individual responses here based on your own synthesis of the
concepts. These questions are based on today's learning and \textbf{will be
assessed} (reassessments are okay if needed). Submit your answers to
this assignment. 

\emph{Siblings from the same parents are related but not identical.}

\begin{enumerate}
\item What are all the mechanisms that create this genetic variation
between "full" siblings? Describe these processes in as much detail
as possible.

\item Do you expect there to be any genetic variation between identical
twins (from the same fertilized egg, which split into two separate
embryos early in development)? Explain your answer.
\end{enumerate}

\section{Answers!}
\label{sec:org180cf48}
\subsection{One}
\label{sec:orgb0f5dea}
\subsubsection{Crossing Over}
\label{sec:org1bdd3c5}
During meiosis 1, DNA segments are swapped between homologous
chromosomes. These homologs are aligned on the meiotic plates and
attached with the synaptonemal complex, where segments of each are
broken then recombined with the appropriate nucleotide sequence.
Mutations can also occur in the DNA synthesis that fills gaps created by
this process.

\subsubsection{Independent Assortment}
\label{sec:org64fa3b8}
During metaphase 1, homologs align randomly and independently to form
gametes. This results in significant variation of genetic information in
each daughter cell. Their are 2\textsuperscript{n} combinations of chromosomes, with n
being the number of unique chromosomes.

\subsubsection{S Phase}
\label{sec:orga2c27c5}
Cell division requires the replication of DNA. This replication occurs
during the S phase, where ribosomal errors that go undetected lead to
genetic variation.

\subsubsection{Environmental / Damage}
\label{sec:org065d10f}
Enviromental factors like UV rays or smoking can damage DNA, leading to
genetic variation. Errors can also result from the process of repairing
or replacing this damaged DNA. Environmental factors can also induce
epigenetic change.

\subsubsection{Viruses}
\label{sec:org973d6d2}
Viruses inject their own genetic information into its host's cells.
While this foreign genetic information may not be permanent, it is still
genetic variation.

\subsection{Two}
\label{sec:org75b1678}
Yes, I do. Identical twins occur when a single egg splits into multiple
embryos. Thus, at the very least, genetic variation arising from cell
division will occur, not to mention environmental factors and viruses.

\noindent\rule{\textwidth}{0.5pt}

\section{More notes!}
\label{sec:org9353378}
\subsection{Trait vs Phenotype}
\label{sec:orgfbbf359}
T: characteristics influenced by genes but can also have nurture
component P: Collection of traits

\subsection{Mutations}
\label{sec:org71a567b}
\begin{itemize}
\item point mutation: single base substitution

\begin{itemize}
\item silent mutation: no effect, doesn't impact codon sequence
\item missense: changes amino acid structure
\item nonsense: inserts a stop codon
\end{itemize}

\item frameshift: insertion/deletion of n amount of bases

\begin{itemize}
\item deleting two shifts the entire seqence to the right
\item break alot of things
\item so frameshift mutation != frameshift, and frameshift mutation
sometimes leads to frameshift
\end{itemize}

\item mutagens are like carcinogens for mutations
\end{itemize}

\subsubsection{Mutation Think Through}
\label{sec:org9073803}
thinking through mutations: - Can you think of scenarios in which the
insertion or deletion of bases in the above sequence would not result in
a frameshift? - deleting or inserting multiples of three that are not in
junctions - not true! will fix itself -- delete three means 1 and 2,
combine, back to three. - delete what would get frameshifted - delete
from the end - A silent mutation has no effect on protein sequence.
Could a silent mutation ever affect an organism's phenotype? Explain. -
no\ldots{} it shoudnt be able to - could be on some binding site that would
break? - mutates protein coding seqences - What functional predictions
would you make for a nonsense mutation that occurs very early vs. very
late in a gene's sequence? - very early would make it not get created,
middle would cause a strange protein, late would make little impact

\subsubsection{Large scale changes}
\label{sec:org9402305}
chromosonal reanrangrments are a thing. generally not called mutations
deletion, duplicatio, inversion, ect. of large sections

\subsubsection{Impact}
\label{sec:org03fd654}
\begin{itemize}
\item \textbf{Loss of function}

\begin{itemize}
\item complete loss of protein of function
\item reduction of function
\item -function
\end{itemize}

\item \textbf{Gain of function}

\begin{itemize}
\item increanse in function
\item new function
\item new expression time
\item +function but, most proteins are like links in a chain. Jehnna's
term is "pathway" which she seems to like
\end{itemize}
\end{itemize}

germline vs. somatic

\section{And the following questions}
\label{sec:orgbad19ab}
!\href{Pasted image 20210423165529.png.org}{Pasted image
20210423165529.png}

\begin{quote}
/Red blood cells have various carbohydrate molecules attached to
proteins on their surfaces (see diagram below). Human A-B-O blood
types are determined by the presence or absence of two particular
carbohydrate modifications, "A" and "B."/ /One gene with three main
alleles controls the A-B-O trait; it encodes a glycosyltransferase (an
enzyme that can attach carbohydrates to other molecules). The A and B
alleles both encode a functional enzyme, but each version of the
enzyme generates a different carbohydrate modification, "A" or "B."
The O allele encodes a non-functional enzyme./
\end{quote}

\subsubsection{What two alleles could a person with blood type A have? With type}
\label{sec:orgd924efc}
B? With type AB? With type O?
:CUSTOM\textsubscript{ID}: what-two-alleles-could-a-person-with-blood-type-a-have-with-type-b-with-type-ab-with-type-o

\begin{itemize}
\item \textbf{A}: \emph{AA, AO}
\item \textbf{B}: \emph{BB, BO}
\item \textbf{AB}: \emph{AB}
\item \textbf{O}: \emph{OO}
\end{itemize}

\subsubsection{If a person with type AB blood had a child with a type O person,}
\label{sec:org26ffaec}
what possible blood types could their child have? What would be the
likelihood of each type?
:CUSTOM\textsubscript{ID}: if-a-person-with-type-ab-blood-had-a-child-with-a-type-o-person-what-possible-blood-types-could-their-child-have-what-would-be-the-likelihood-of-each-type
A + B, O + O: - A + O -> A - B + O -> B

\textbf{50\% A, 50\% B}

\subsubsection{If the offspring from the previous question grew up to have a child}
\label{sec:org5f664c7}
with a type AB person, what blood types could their child potentially
have?
:CUSTOM\textsubscript{ID}: if-the-offspring-from-the-previous-question-grew-up-to-have-a-child-with-a-type-ab-person-what-blood-types-could-their-child-potentially-have
O+A||B, A+B: - O + A -> A - O + B -> B - A + A -> A - B + B -> B - A + B
-> A+B - B + A -> A+B

\textbf{1/3 A, 1/3 B, 1/3 A + B}

\section{More notes pt 2..}
\label{sec:orge91daa3}
\begin{itemize}
\item \textbf{differnet inheritence patterns}
\item mendelian: dominant vs recesive

\begin{itemize}
\item alleles, alt versions of traits, are responsible for varitations in
inherited traits
\end{itemize}

\item incomplete dominance

\begin{itemize}
\item both alles are visible, neither are completely dominant
\end{itemize}

\item codomidance

\begin{itemize}
\item both alleles are visible in distinguishible ways
\end{itemize}

\item polygnenic inheritance

\begin{itemize}
\item single phenotype determined the the addition of multiple
\end{itemize}

\item epistatis

\begin{itemize}
\item one gene alters another!
\end{itemize}

\item S-linked

\begin{itemize}
\item linked by sex
\end{itemize}
\end{itemize}
\end{document}
