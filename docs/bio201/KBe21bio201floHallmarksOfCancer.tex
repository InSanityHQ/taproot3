% Created 2021-09-11 Sat 08:18
% Intended LaTeX compiler: xelatex
\documentclass[letterpaper]{article}
\usepackage{graphicx}
\usepackage{grffile}
\usepackage{longtable}
\usepackage{wrapfig}
\usepackage{rotating}
\usepackage[normalem]{ulem}
\usepackage{amsmath}
\usepackage{textcomp}
\usepackage{amssymb}
\usepackage{capt-of}
\usepackage{hyperref}
\usepackage[margin=1in]{geometry}
\usepackage{fontspec}
\usepackage{indentfirst}
\setmainfont[ItalicFont = LiberationSans-Italic, BoldFont = LiberationSans-Bold, BoldItalicFont = LiberationSans-BoldItalic]{LiberationSans}
\newfontfamily\NHLight[ItalicFont = LiberationSansNarrow-Italic, BoldFont       = LiberationSansNarrow-Bold, BoldItalicFont = LiberationSansNarrow-BoldItalic]{LiberationSansNarrow}
\newcommand\textrmlf[1]{{\NHLight#1}}
\newcommand\textitlf[1]{{\NHLight\itshape#1}}
\let\textbflf\textrm
\newcommand\textulf[1]{{\NHLight\bfseries#1}}
\newcommand\textuitlf[1]{{\NHLight\bfseries\itshape#1}}
\usepackage{fancyhdr}
\pagestyle{fancy}
\usepackage{titlesec}
\usepackage{titling}
\makeatletter
\lhead{\textbf{\@title}}
\makeatother
\rhead{\textrmlf{Compiled} \today}
\lfoot{\theauthor\ \textbullet \ \textbf{2021-2022}}
\cfoot{}
\rfoot{\textrmlf{Page} \thepage}
\titleformat{\section} {\Large} {\textrmlf{\thesection} {|}} {0.3em} {\textbf}
\titleformat{\subsection} {\large} {\textrmlf{\thesubsection} {|}} {0.2em} {\textbf}
\titleformat{\subsubsection} {\large} {\textrmlf{\thesubsubsection} {|}} {0.1em} {\textbf}
\setlength{\parskip}{0.45em}
\renewcommand\maketitle{}
\author{Exr0n}
\date{\today}
\title{flo - hallmarks of cancer (2011) reading}
\hypersetup{
 pdfauthor={Exr0n},
 pdftitle={flo - hallmarks of cancer (2011) reading},
 pdfkeywords={},
 pdfsubject={},
 pdfcreator={Emacs 27.2 (Org mode 9.4.4)}, 
 pdflang={English}}
\begin{document}

\maketitle
\section{sources\hfill{}\textsc{source}}
\label{sec:orgd26793d}
\subsection{assignment: \url{https://nuevaschool.instructure.com/courses/3087/assignments/56036}}
\label{sec:orgd10f58d}
\subsection{reading: \href{KBsrcHallmarksOfCancer2011Reading.pdf}{Hallmarks of Cancer PDF}}
\label{sec:org5ade8eb}
\section{Flow}
\label{sec:org39a5b4f}
\subsection{Abstract}
\label{sec:org0caa958}
\subsubsection{hallmarks include}
\label{sec:org35ea45c}
\begin{enumerate}
\item sustaining proliferative signaling
\label{sec:orge05ed1d}
\item evading growth suppressors
\label{sec:orgecc1b55}
\item resisting cell death
\label{sec:orgfb06ef3}
\item enabling replicative immortality
\label{sec:org2cf4cf9}
\item inducing ingiogenesis
\label{sec:org2d5a6c3}
\item activating invasion and metastasis
\label{sec:orgecac7e1}
\end{enumerate}
\subsubsection{theese hallmarks are newer}
\label{sec:org9fc83ce}
\begin{enumerate}
\item reprogramming of energy metabolism
\label{sec:org24a7763}
\item evading immune destruction
\label{sec:orgc2e05d0}
\end{enumerate}
\subsubsection{underlying}
\label{sec:org53719e3}
\begin{enumerate}
\item genome instability
\label{sec:orge3fae32}
\begin{enumerate}
\item genetic diversity that expedites acquisition of hallmarks
\label{sec:orgd79bbaf}
\end{enumerate}
\item inflammation
\label{sec:orge0267c3}
\begin{enumerate}
\item "fosters multiple hallmark functions"
\label{sec:orgb6a91b2}
\end{enumerate}
\end{enumerate}
\subsection{Introduction}
\label{sec:orgf77ed58}
\subsubsection{Cancer cells evolve into cancer cells because they need to be cancer cells??}
\label{sec:org39b9b00}
\begin{enumerate}
\item {\bfseries\sffamily TODO} why do tumors have "the need \ldots{} to acquire the traints that enable them to become tumorigenic and ultimately malignant"?\hfill{}\textsc{question}
\label{sec:org8b9dde2}
\end{enumerate}
\subsubsection{tumors are not simple / idle 'insular masses of proliferating cancer cells'}
\label{sec:orgf016dbf}
\subsubsection{"recruited" normal cells (or 'stromal cells') are active parts of the tumor}
\label{sec:org7f30768}
\subsubsection{'the biology of tumors can no longer be understood simply by enumerating the traits of the cancer cells but instead must encompass the contributions of the "tumor microenvironment" to tumorigenesis.'}
\label{sec:orge8f365b}
\subsubsection{purpose is to consider new hallmarks that have been found or note that old ones weren't as general as we thought}
\label{sec:orgb992b02}
\subsection{section: 'An Emerging Hallmark: Evading Immune Destruction'}
\label{sec:orga320c9f}
\subsubsection{the immune system usually eradicates the 'formation and progression of incipient neoplasias, late-stage tumors, and micrometastases', so why not in these cancers?}
\label{sec:orgc295d9a}
\subsubsection{'long standing theory of immune surveillance' -> something went interesting}
\label{sec:org3e7c1cf}
\begin{enumerate}
\item 'cells and tissues are constantly monitored'
\label{sec:org63ddf90}
\item surveillance should elim cancer cells before they grow into tumors
\label{sec:orgb64a2e5}
\item thus, grown tumors have either hid from surveillance or limited the 'extent of immunological killing'
\label{sec:org42033af}
\end{enumerate}
\subsubsection{more cancer in immunocompromised individuals}
\label{sec:orgec9cabe}
\begin{enumerate}
\item but these are virus-induced cancers
\label{sec:org606fa53}
\begin{enumerate}
\item so helping these people = reducing viral infilltration
\label{sec:org2a6f3c4}
\end{enumerate}
\item other cancers still evade the immune system
\label{sec:orgd81c4b4}
\end{enumerate}
\subsubsection{'genetically engineered mice and clinical epidemeology suggest that the immune system' actually hurts cancer}
\label{sec:orgb1c83dc}
\begin{enumerate}
\item mice that are engineered to lack some immune parts got cancer faster/stronger/more
\label{sec:org4831b4f}
\begin{enumerate}
\item these guys are important in fighting cancer
\label{sec:orgf359846}
\begin{enumerate}
\item CD8\^{}+ cytotoxic T lymphocytes (CTLs)
\label{sec:orgc5fb3d3}
\item CD4\^{}+ T\textsubscript{h1} helper T cells
\label{sec:org66011aa}
\item natural killer (NK) cells
\label{sec:org336ebba}
\end{enumerate}
\item 'demonstrable increase in tumor incidence'
\label{sec:org649715a}
\item lacking multiple -> 'more susceptible to cancer development'
\label{sec:orgb469add}
\item 'both the innate and adaptive cellular arms of the immune system are able to contribute significantly to immune surveillance and thus tumor eradication'\hfill{}\textsc{conclusion}
\label{sec:org0e765e0}
\end{enumerate}
\end{enumerate}
\subsubsection{transplantation experiments}
\label{sec:org846c82b}
\begin{enumerate}
\item cancer cells from immunodeficient mice have a bad time in normal mice
\label{sec:org3ce0669}
\item cancer cells from normal mice can initiate tumors in both types of hosts
\label{sec:org8dd7c4e}
\item maybe some cancer cells are more easily detected and those would normally die in normal hosts but live in comprimised hosts, but when transplanted they meet a competent immune system and die\hfill{}\textsc{conclusion}
\label{sec:orge6610ee}
\item Open question: do some carcinogens tend to induce more/less immunogenic cancer cells?\hfill{}\textsc{nextstep}
\label{sec:org999e291}
\end{enumerate}
\subsubsection{the immune system probably includes antitumoral responses}
\label{sec:orge60ca26}
\begin{enumerate}
\item patients with colon and ovarian tumors who have lots of CTLs and NK cells have better prognosis
\label{sec:orga30364b}
\begin{enumerate}
\item evidence is not as strong for other cancers\hfill{}\textsc{nextstep}
\label{sec:orga203b2d}
\end{enumerate}
\item immunosupressed organ recievers got cancer from the donor
\label{sec:orgcb4836a}
\begin{enumerate}
\item suggests doner had immune system which held cancer down until organ was transplanted
\label{sec:org7ff4054}
\end{enumerate}
\end{enumerate}
\subsubsection{{\bfseries\sffamily TODO} 'still, the epidemiology of chronically immunosupressed patients does not indicate significantly increased incidences of the major forms of nonviral human cancer'}
\label{sec:orgcfe2d6f}
\subsubsection{{\bfseries\sffamily TODO} something about HIV patients who lack T and B cells and how they should still be able to fight cancer with NK cells and CTLs}
\label{sec:org0294f65}
\subsubsection{that was oversimplified as the tumor might also be actively supressing immune responses}
\label{sec:org0871485}
\begin{enumerate}
\item may 'paralyze infiltrating CTLs and NK cells by secreting TGF-\(\beta\) or other immunosuppressive factors'
\label{sec:orgc65c493}
\item 'more subtle mechinisms .. recruitment of inflamatory cells that are actively immunosuppressive'
\label{sec:orge145b93}
\begin{enumerate}
\item 'regulatory T cells (Tregs) and myeloid-derived suppressor cells (MDSCs)'
\label{sec:org6c298fa}
\end{enumerate}
\end{enumerate}
\subsubsection{it is so far unclear whether the immune system plays a large enough role to be considered a hallmark of cancer}
\label{sec:org3b9b983}
\section{Vocab}
\label{sec:org6da63f2}
\subsection{neoplastic disease}
\label{sec:orgc878dc5}
\subsubsection{anything that causes tumor growth (malignant or benign)}
\label{sec:org36760f3}
\subsection{ostensibly}
\label{sec:org2a81c9d}
\subsubsection{maybe 'technically'?'}
\label{sec:orgefbb487}
\subsection{tumor microenvironment}
\label{sec:org833045f}
\subsubsection{presumably inflammation, recruited normal cells, and other stuff that helps the tumor grow}
\label{sec:org1012b99}
\subsection{pathogenisis}
\label{sec:org40d4488}
\subsubsection{evolution of 'pathogen' (cancer)}
\label{sec:orgbdcaa0c}
\subsection{ancillary proposition}
\label{sec:orgea0309a}
\subsubsection{maybe the starting / base proposition}
\label{sec:orgf90445a}
\subsection{insular masses}
\label{sec:org8371c47}
\subsubsection{stagnant or something, simple}
\label{sec:org84b4c64}
\subsection{heterotypic interactions}
\label{sec:org307f921}
\subsubsection{many types of interactions}
\label{sec:org6867b09}
\subsection{tumorigenisis}
\label{sec:orge2a5bbb}
\subsubsection{the growth / development of a tumor?}
\label{sec:org18213a3}
\subsection{neoplasias}
\label{sec:org963e660}
\subsubsection{new uncontrolled growth of cells}
\label{sec:org8316ab3}
\subsection{micrometastases}
\label{sec:org8be11fd}
\subsubsection{clumps of cancer cells that spread around the body}
\label{sec:org52ba5b8}
\subsection{etiology}
\label{sec:org8c496d3}
\subsubsection{study of cancer?}
\label{sec:org863a3c7}
\subsection{immunogenic}
\label{sec:orgf65a1fe}
\subsubsection{easily detected by the immune system}
\label{sec:orgca634ce}
\subsection{immunoediting}
\label{sec:org7800494}
\subsubsection{"natural selection" by the immune system}
\label{sec:orgd13d8fd}
\subsection{prognosis}
\label{sec:orgd085c09}
\subsubsection{a prediction of the outcome of a cancer (or disease in general, apparently)}
\label{sec:orgc31b4b5}
\end{document}
