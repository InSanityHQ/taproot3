% Created 2021-09-12 Sun 22:49
% Intended LaTeX compiler: xelatex
\documentclass[letterpaper]{article}
\usepackage{graphicx}
\usepackage{grffile}
\usepackage{longtable}
\usepackage{wrapfig}
\usepackage{rotating}
\usepackage[normalem]{ulem}
\usepackage{amsmath}
\usepackage{textcomp}
\usepackage{amssymb}
\usepackage{capt-of}
\usepackage{hyperref}
\usepackage[margin=1in]{geometry}
\usepackage{fontspec}
\usepackage{indentfirst}
\setmainfont[ItalicFont = LiberationSans-Italic, BoldFont = LiberationSans-Bold, BoldItalicFont = LiberationSans-BoldItalic]{LiberationSans}
\newfontfamily\NHLight[ItalicFont = LiberationSansNarrow-Italic, BoldFont       = LiberationSansNarrow-Bold, BoldItalicFont = LiberationSansNarrow-BoldItalic]{LiberationSansNarrow}
\newcommand\textrmlf[1]{{\NHLight#1}}
\newcommand\textitlf[1]{{\NHLight\itshape#1}}
\let\textbflf\textrm
\newcommand\textulf[1]{{\NHLight\bfseries#1}}
\newcommand\textuitlf[1]{{\NHLight\bfseries\itshape#1}}
\usepackage{fancyhdr}
\pagestyle{fancy}
\usepackage{titlesec}
\usepackage{titling}
\makeatletter
\lhead{\textbf{\@title}}
\makeatother
\rhead{\textrmlf{Compiled} \today}
\lfoot{\theauthor\ \textbullet \ \textbf{2021-2022}}
\cfoot{}
\rfoot{\textrmlf{Page} \thepage}
\titleformat{\section} {\Large} {\textrmlf{\thesection} {|}} {0.3em} {\textbf}
\titleformat{\subsection} {\large} {\textrmlf{\thesubsection} {|}} {0.2em} {\textbf}
\titleformat{\subsubsection} {\large} {\textrmlf{\thesubsubsection} {|}} {0.1em} {\textbf}
\setlength{\parskip}{0.45em}
\renewcommand\maketitle{}
\author{Zachary Sayyah}
\date{\today}
\title{Watson Reading pages 1-3}
\hypersetup{
 pdfauthor={Zachary Sayyah},
 pdftitle={Watson Reading pages 1-3},
 pdfkeywords={},
 pdfsubject={},
 pdfcreator={Emacs 28.0.50 (Org mode 9.4.4)}, 
 pdflang={English}}
\begin{document}

\maketitle


\section{Annotation}
\label{sec:orgf75f766}
\href{https://google.com}{Linked Here}

\section{Defining the Scale}
\label{sec:orga857eb5}
\begin{itemize}
\item The author thinks that the words we use to describe bodies of
government are far too oversimplified

\begin{itemize}
\item They think that there are a lot of bodies that fall into the
category of being between completely independent, and being one with
other bodies.

\begin{itemize}
\item These two absolutes never occur in practice
\item It is convenient to divide up this spectrum to make states more
comparable into

\begin{itemize}
\item independence
\item hegemony
\item dominion
\item empire
\end{itemize}
\end{itemize}
\end{itemize}
\end{itemize}

\section{How States Move along that Scale}
\label{sec:org72b3906}
\begin{itemize}
\item In all nations there is a struggle between just the right amount of
freedom and the right amount of order

\begin{itemize}
\item The desire to autonomous and also the desire to be independent of a
greater body is usually the result of too many constraints and
commitments

\begin{itemize}
\item Such independence has drawbacks as it requires a nation to be
militarily and economically independent
\end{itemize}

\item An independent nation will eventually make voluntary commitments to
make easier the management of external affairs. This will move it
further up the spectrum and make it less independent.

\begin{itemize}
\item The more intertwined a state is, the less independent it is.
\item The freedom of a state to do as they please is always limited by
the pressure of interdependence.
\item These agreements/rules may start to be created by a hegemony

\begin{itemize}
\item A larger organization agreed upon by these states to help them
with external affairs.

\begin{itemize}
\item This organization often begins to be designed to give more
political power to a certain state or states
\end{itemize}

\item An acceptance of sorts is necessary for any hegemony to be
considered Suzerain
\end{itemize}

\item Further along the spectrum we have a dominion

\begin{itemize}
\item A dominion is essentially where there's an authority that to
some extent controls the internal affairs of other communities.

\begin{itemize}
\item They still retain their identity as separate states however
\end{itemize}
\end{itemize}

\item The most extreme case is an empire

\begin{itemize}
\item In an empire, the greater organization controls the affairs of
the smaller communities within it.
\end{itemize}

\item The amount that people believe in an authority is directly related
to its power.
\end{itemize}
\end{itemize}
\end{itemize}
\end{document}
