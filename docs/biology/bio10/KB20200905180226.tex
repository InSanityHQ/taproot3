% Created 2021-09-12 Sun 22:49
% Intended LaTeX compiler: xelatex
\documentclass[letterpaper]{article}
\usepackage{graphicx}
\usepackage{grffile}
\usepackage{longtable}
\usepackage{wrapfig}
\usepackage{rotating}
\usepackage[normalem]{ulem}
\usepackage{amsmath}
\usepackage{textcomp}
\usepackage{amssymb}
\usepackage{capt-of}
\usepackage{hyperref}
\usepackage[margin=1in]{geometry}
\usepackage{fontspec}
\usepackage{indentfirst}
\setmainfont[ItalicFont = LiberationSans-Italic, BoldFont = LiberationSans-Bold, BoldItalicFont = LiberationSans-BoldItalic]{LiberationSans}
\newfontfamily\NHLight[ItalicFont = LiberationSansNarrow-Italic, BoldFont       = LiberationSansNarrow-Bold, BoldItalicFont = LiberationSansNarrow-BoldItalic]{LiberationSansNarrow}
\newcommand\textrmlf[1]{{\NHLight#1}}
\newcommand\textitlf[1]{{\NHLight\itshape#1}}
\let\textbflf\textrm
\newcommand\textulf[1]{{\NHLight\bfseries#1}}
\newcommand\textuitlf[1]{{\NHLight\bfseries\itshape#1}}
\usepackage{fancyhdr}
\pagestyle{fancy}
\usepackage{titlesec}
\usepackage{titling}
\makeatletter
\lhead{\textbf{\@title}}
\makeatother
\rhead{\textrmlf{Compiled} \today}
\lfoot{\theauthor\ \textbullet \ \textbf{2021-2022}}
\cfoot{}
\rfoot{\textrmlf{Page} \thepage}
\titleformat{\section} {\Large} {\textrmlf{\thesection} {|}} {0.3em} {\textbf}
\titleformat{\subsection} {\large} {\textrmlf{\thesubsection} {|}} {0.2em} {\textbf}
\titleformat{\subsubsection} {\large} {\textrmlf{\thesubsubsection} {|}} {0.1em} {\textbf}
\setlength{\parskip}{0.45em}
\renewcommand\maketitle{}
\author{Zachary Sayyah}
\date{\today}
\title{Carbohydrates Tutorial (De)}
\hypersetup{
 pdfauthor={Zachary Sayyah},
 pdftitle={Carbohydrates Tutorial (De)},
 pdfkeywords={},
 pdfsubject={},
 pdfcreator={Emacs 28.0.50 (Org mode 9.4.4)}, 
 pdflang={English}}
\begin{document}

\maketitle


\section{Notes}
\label{sec:org68af8b7}
\begin{itemize}
\item Carbohydrates are found in all living things that we know of and are
critical to our functioning

\begin{itemize}
\item There is a lot of debate surrounding how much is good and how much
is bad for us
\end{itemize}

\item Carbohydrates usually have a long carbon chain with oxygen and
hydrogen branching off of it

\begin{itemize}
\item Usually they are tipped with hydrogen, or a hydroxel group
\end{itemize}

\item Can tend to mix with water well due to their polarity

\begin{itemize}
\item They also chain with each-other really well due to their polarity
which causes the long chains called polysaccarides
\item They; however, do not dissolve in water very well

\begin{itemize}
\item Large carbohydrates tend to dissolve in water well however

\begin{itemize}
\item Cellulose
\item Starch
\item Glycogen
\item other such things
\end{itemize}
\end{itemize}
\end{itemize}

\item Hexagonal drawings and edges without labels should be assumed to be
carbs unless said otherwise
\item These can be found on

\begin{itemize}
\item The cell membrane
\item In cell walls
\item Inside mitochondria
\end{itemize}

\item Pubmed is a very good source and we should go directly to the research
and read their conclusions instead of getting our scientific
information through the news
\end{itemize}
\end{document}
