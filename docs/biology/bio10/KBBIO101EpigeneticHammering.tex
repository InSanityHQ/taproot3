% Created 2021-09-12 Sun 22:49
% Intended LaTeX compiler: xelatex
\documentclass[letterpaper]{article}
\usepackage{graphicx}
\usepackage{grffile}
\usepackage{longtable}
\usepackage{wrapfig}
\usepackage{rotating}
\usepackage[normalem]{ulem}
\usepackage{amsmath}
\usepackage{textcomp}
\usepackage{amssymb}
\usepackage{capt-of}
\usepackage{hyperref}
\usepackage[margin=1in]{geometry}
\usepackage{fontspec}
\usepackage{indentfirst}
\setmainfont[ItalicFont = LiberationSans-Italic, BoldFont = LiberationSans-Bold, BoldItalicFont = LiberationSans-BoldItalic]{LiberationSans}
\newfontfamily\NHLight[ItalicFont = LiberationSansNarrow-Italic, BoldFont       = LiberationSansNarrow-Bold, BoldItalicFont = LiberationSansNarrow-BoldItalic]{LiberationSansNarrow}
\newcommand\textrmlf[1]{{\NHLight#1}}
\newcommand\textitlf[1]{{\NHLight\itshape#1}}
\let\textbflf\textrm
\newcommand\textulf[1]{{\NHLight\bfseries#1}}
\newcommand\textuitlf[1]{{\NHLight\bfseries\itshape#1}}
\usepackage{fancyhdr}
\pagestyle{fancy}
\usepackage{titlesec}
\usepackage{titling}
\makeatletter
\lhead{\textbf{\@title}}
\makeatother
\rhead{\textrmlf{Compiled} \today}
\lfoot{\theauthor\ \textbullet \ \textbf{2021-2022}}
\cfoot{}
\rfoot{\textrmlf{Page} \thepage}
\titleformat{\section} {\Large} {\textrmlf{\thesection} {|}} {0.3em} {\textbf}
\titleformat{\subsection} {\large} {\textrmlf{\thesubsection} {|}} {0.2em} {\textbf}
\titleformat{\subsubsection} {\large} {\textrmlf{\thesubsubsection} {|}} {0.1em} {\textbf}
\setlength{\parskip}{0.45em}
\renewcommand\maketitle{}
\date{\today}
\title{}
\hypersetup{
 pdfauthor={},
 pdftitle={},
 pdfkeywords={},
 pdfsubject={},
 pdfcreator={Emacs 28.0.50 (Org mode 9.4.4)}, 
 pdflang={English}}
\begin{document}

\begin{center}
\begin{tabular}{l}
title: Epigenetics: Hammering\\
author: Zachary Sayyah\\
course: BIO101\\
source: \href{KBBiologyMasterIndex.org}{KBBiologyMasterIndex}\\
\end{tabular}
\end{center}

\section{Notes}
\label{sec:orgf7c1f64}
\subsubsection{Genomes}
\label{sec:orgb0d431c}
\begin{itemize}
\item The Nucleus contains genes organized into two parts

\begin{itemize}
\item Each part is called a genome

\begin{itemize}
\item One is sourced from you mother and one from your father
\end{itemize}

\item Genomes are not the same
\end{itemize}

\item Breaking up the genome into pieces those are called chromosomes

\begin{itemize}
\item Humans have 23
\item Having two genomes makes us 2N
\end{itemize}

\item Each chromosome has a bunch of genes that are divided up into three
parts: the promoter (beginning), coding region (middle), and the
terminator (end)

\begin{itemize}
\item Each gene has enough information for a protein
\item Each genome has enough information to generate a human
\end{itemize}
\end{itemize}

\subsubsection{Epigenetics}
\label{sec:orgba09710}
\begin{itemize}
\item The epigenome is defined as the collection of DNA, RNA, proteins, and
their chemical modifications (generally altering gene expression)

\begin{itemize}
\item Epigenetic modifications are done by adding marks to the tails of
histones

\begin{itemize}
\item The addition of an acetyl group causes the tale to relax and
release DNA

\begin{itemize}
\item This increases transcription
\end{itemize}

\item Methyl groups can either increase or decrease that pattern of gene
expression depending

\begin{itemize}
\item putting this directly on DNA permanently shut it down
\end{itemize}
\end{itemize}
\end{itemize}

\item When the envoirnment of a cell changes it creates epigenetic
modifications

\begin{itemize}
\item This is also very useful to cancers as more than half of known
cancers contain mutations involved in regulation
\end{itemize}
\end{itemize}

\#\#\# DNA Packaging

\begin{itemize}
\item Packaging DNA starts with the assembly of a nucleosome via eight
separate histone protein sub units attaching to the DNA

\begin{itemize}
\item This creates a tight loop called the nucleosome
\end{itemize}

\item Multiple nucleosomes are coiled together and stacked on top of
eachother creating what is known as chromatin

\begin{itemize}
\item These are then looped and further packaged
\end{itemize}

\item These make tightly formed structures called chromosomes
\item DNA is in usually a less organized form during division
\end{itemize}
\end{document}
