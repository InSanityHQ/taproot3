% Created 2021-09-27 Mon 12:01
% Intended LaTeX compiler: xelatex
\documentclass[letterpaper]{article}
\usepackage{graphicx}
\usepackage{grffile}
\usepackage{longtable}
\usepackage{wrapfig}
\usepackage{rotating}
\usepackage[normalem]{ulem}
\usepackage{amsmath}
\usepackage{textcomp}
\usepackage{amssymb}
\usepackage{capt-of}
\usepackage{hyperref}
\setlength{\parindent}{0pt}
\usepackage[margin=1in]{geometry}
\usepackage{fontspec}
\usepackage{svg}
\usepackage{cancel}
\usepackage{indentfirst}
\setmainfont[ItalicFont = LiberationSans-Italic, BoldFont = LiberationSans-Bold, BoldItalicFont = LiberationSans-BoldItalic]{LiberationSans}
\newfontfamily\NHLight[ItalicFont = LiberationSansNarrow-Italic, BoldFont       = LiberationSansNarrow-Bold, BoldItalicFont = LiberationSansNarrow-BoldItalic]{LiberationSansNarrow}
\newcommand\textrmlf[1]{{\NHLight#1}}
\newcommand\textitlf[1]{{\NHLight\itshape#1}}
\let\textbflf\textrm
\newcommand\textulf[1]{{\NHLight\bfseries#1}}
\newcommand\textuitlf[1]{{\NHLight\bfseries\itshape#1}}
\usepackage{fancyhdr}
\pagestyle{fancy}
\usepackage{titlesec}
\usepackage{titling}
\makeatletter
\lhead{\textbf{\@title}}
\makeatother
\rhead{\textrmlf{Compiled} \today}
\lfoot{\theauthor\ \textbullet \ \textbf{2021-2022}}
\cfoot{}
\rfoot{\textrmlf{Page} \thepage}
\renewcommand{\tableofcontents}{}
\titleformat{\section} {\Large} {\textrmlf{\thesection} {|}} {0.3em} {\textbf}
\titleformat{\subsection} {\large} {\textrmlf{\thesubsection} {|}} {0.2em} {\textbf}
\titleformat{\subsubsection} {\large} {\textrmlf{\thesubsubsection} {|}} {0.1em} {\textbf}
\setlength{\parskip}{0.45em}
\renewcommand\maketitle{}
\author{Zachary Sayyah}
\date{\today}
\title{Biology Reading Protein Notes}
\hypersetup{
 pdfauthor={Zachary Sayyah},
 pdftitle={Biology Reading Protein Notes},
 pdfkeywords={},
 pdfsubject={},
 pdfcreator={Emacs 28.0.50 (Org mode 9.4.4)}, 
 pdflang={English}}
\begin{document}

\tableofcontents

\#ret \#flo

\noindent\rule{\textwidth}{0.5pt}

\section{Proteins}
\label{sec:org78ac866}
\subsection{Structures}
\label{sec:org202121c}
\begin{itemize}
\item Proteins account for 50\% of the dray mass of most cells
\item Enzymes are mostly proteins
\item Very structurally complex

\begin{itemize}
\item They are constructed from the same 20 sets of amino acids
\end{itemize}

\item A polypeptide is a polymer of amino acids

\begin{itemize}
\item A protein is made up of one or more polypeptides
\end{itemize}

\item A protein must serve a biological function to be a protein

\begin{itemize}
\item It also must be folded and coiled into a specific 3 dimensional
structure
\end{itemize}

\item There are many types of proteins

\begin{itemize}
\item Enzymatic proteins

\begin{itemize}
\item Selective acceleration of chemical reactions
\item Digestive enzymes are an example that catalyze the hydrolysis of
bonds in food
\end{itemize}

\item Defensive proteins

\begin{itemize}
\item Protect against disease
\item Antibodies are an example and inactivate as well as help destroy
viruses and bacteria
\end{itemize}

\item Storage proteins

\begin{itemize}
\item Storage of amino acids
\item Casein is an example which is in milk and is the major source of
amino acids for baby mammals
\end{itemize}

\item Transport proteins

\begin{itemize}
\item Transport of substances
\item Hemoglobin is an example that transports oxygen from the lungs to
other parts of the body.
\end{itemize}

\item Hormonal proteins

\begin{itemize}
\item COordination of an organism's activities
\item Insulin is an example as it causes other tissues to take up
glucose thus regulating the blood sugar concentration
\end{itemize}

\item Receptor proteins

\begin{itemize}
\item Response of cell to chemical stimuli
\item Responsible for stuff like detecting signaling molecules released
by other nerve cells
\end{itemize}

\item Contractile and motor proteins

\begin{itemize}
\item Meant for movement
\item Responsible for stuff like flagella
\end{itemize}

\item Structural proteins

\begin{itemize}
\item They are used as support
\item Keratin is an example
\end{itemize}
\end{itemize}
\end{itemize}

\#\#\# Amino Acids

\begin{itemize}
\item All amino acids share a common structure

\begin{itemize}
\item It is an organic molecule with both an amino group and a carboxyl
group

\begin{itemize}
\item An amino group is two Hydrogens bonded with a nitrogen and a
carboxyl group is an oxygen double bonded with a carbon and an OH
bonded with the same carbon
\end{itemize}

\item The side chain determines the unique characteristics of the
particular amino acid
\end{itemize}
\end{itemize}

\subsubsection{Protein Structure}
\label{sec:org21ac764}
\begin{itemize}
\item The specific function of a protein is a result of their shape
\item There is now easier sequencing of proteins, but originally it was very
difficult
\item The protein may spontaneously fold once constructed
\item Proteins share three superimposed levels of structure, known as
primary, secondary, and tertiary structure

\begin{itemize}
\item A fourth Quaternary structure arises once when a protein is made of
two or more polypeptide chains
\end{itemize}

\item Primary structure

\begin{itemize}
\item A sequence of amino acids
\end{itemize}

\item Secondary strucutre

\begin{itemize}
\item These are coils and folds formed by hydrogen bonds from partial
charges
\item An example would be a helix and pleated sheet
\end{itemize}

\item Tertiary structure

\begin{itemize}
\item It is the overall shape of a polypeptide resulting from interactions
between the side chains
\end{itemize}

\item Quaternary structure

\begin{itemize}
\item This is the overall protein structure resulting from the polypeptide
sub-units
\end{itemize}

\item Primary structure is very fundamental and even a slight change can
cause large consequences
\item Chemical conditions can alter a proteins shape

\begin{itemize}
\item Denaturing is when a protein becomes mishapen and therefore
biologically inactive
\end{itemize}

\item Many proteins have been sequenced
\item Misfolded proteins are the causes of many diseases
\item Crystallography is a technique for obtaining the 3d shape of a protein
\item Nucleic acids are made of polynucleotides which are made of
nucleotides
\end{itemize}
\end{document}
