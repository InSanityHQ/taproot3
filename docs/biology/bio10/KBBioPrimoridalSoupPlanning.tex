% Created 2021-09-12 Sun 22:48
% Intended LaTeX compiler: xelatex
\documentclass[letterpaper]{article}
\usepackage{graphicx}
\usepackage{grffile}
\usepackage{longtable}
\usepackage{wrapfig}
\usepackage{rotating}
\usepackage[normalem]{ulem}
\usepackage{amsmath}
\usepackage{textcomp}
\usepackage{amssymb}
\usepackage{capt-of}
\usepackage{hyperref}
\usepackage[margin=1in]{geometry}
\usepackage{fontspec}
\usepackage{indentfirst}
\setmainfont[ItalicFont = LiberationSans-Italic, BoldFont = LiberationSans-Bold, BoldItalicFont = LiberationSans-BoldItalic]{LiberationSans}
\newfontfamily\NHLight[ItalicFont = LiberationSansNarrow-Italic, BoldFont       = LiberationSansNarrow-Bold, BoldItalicFont = LiberationSansNarrow-BoldItalic]{LiberationSansNarrow}
\newcommand\textrmlf[1]{{\NHLight#1}}
\newcommand\textitlf[1]{{\NHLight\itshape#1}}
\let\textbflf\textrm
\newcommand\textulf[1]{{\NHLight\bfseries#1}}
\newcommand\textuitlf[1]{{\NHLight\bfseries\itshape#1}}
\usepackage{fancyhdr}
\pagestyle{fancy}
\usepackage{titlesec}
\usepackage{titling}
\makeatletter
\lhead{\textbf{\@title}}
\makeatother
\rhead{\textrmlf{Compiled} \today}
\lfoot{\theauthor\ \textbullet \ \textbf{2021-2022}}
\cfoot{}
\rfoot{\textrmlf{Page} \thepage}
\titleformat{\section} {\Large} {\textrmlf{\thesection} {|}} {0.3em} {\textbf}
\titleformat{\subsection} {\large} {\textrmlf{\thesubsection} {|}} {0.2em} {\textbf}
\titleformat{\subsubsection} {\large} {\textrmlf{\thesubsubsection} {|}} {0.1em} {\textbf}
\setlength{\parskip}{0.45em}
\renewcommand\maketitle{}
\author{Huxley}
\date{\today}
\title{Primordial soup vid planning doc}
\hypersetup{
 pdfauthor={Huxley},
 pdftitle={Primordial soup vid planning doc},
 pdfkeywords={},
 pdfsubject={},
 pdfcreator={Emacs 28.0.50 (Org mode 9.4.4)}, 
 pdflang={English}}
\begin{document}

\maketitle
\#ref \#ret \#disorganized \#incomplete

\noindent\rule{\textwidth}{0.5pt}

\section{Larger themes}
\label{sec:org58f490b}
Video? perhaps like Melody sheep?
\url{https://www.youtube.com/watch?v=ThDYazipjSI}

heavy blender

\section{General research}
\label{sec:org669ac06}
start with: How did life start?

content: theories of how life started,

explain miller-urey

end with we are not alone in the universe explanation:

life is inevitable

\subsection{deeper research}
\label{sec:orga82bf88}
\begin{itemize}
\item Volcanic clay

\begin{itemize}
\item In simulated ancient seawater, clay forms a hydrogel -- a mass of
microscopic spaces capable of soaking up liquids like a sponge.
\url{https://www.sciencedaily.com/releases/2013/11/131105132027.htm}
\item chemicals confined in those spaces could have carried out the
complex reactions that formed proteins, DNA and eventually all the
machinery that makes a living cell work. Clay hydrogels could have
confined and protected those chemical processes until the membrane
that surrounds living cells developed.
\item theorists have shown that cytoplasm -- the interior environment of a
cell -- behaves much like a hydrogel.
\item Unlike surfactants, lipids are difficult to synthesize. Surfactants
may transform into lipids. Apatite has been reported to be capable
of catalyzing the formation of a proto-lipid [58].
\url{https://www.intechopen.com/books/clay-minerals-in-nature-their-characterization-modification-and-application/role-of-clay-minerals-in-chemical-evolution-and-the-origin-of-life}
\item Clay minerals might function as a primordial cell [4]. When clay
minerals are deposited on the ocean floor (or dried), the particles
form a pile, enclosing small spaces (Figure 6). It is conceivable
that the small spaces behave like cells. Further, when clay minerals
are dispersed in water, bubbles form in water or the surface of
water, while the clay particles gather at the boundary between water
and air, as shown in Figure 7 [57]. In such a case, clay minerals
make a cell-like spherule.
\end{itemize}
\end{itemize}

clay deposits and a type of hydrogel with form sponge like pockets

these pockets could have acted as early cells

cytoplasm behaves a lot like a hydrogel

some of the common minerals found in clay can help the formation of
proto lipids

pockets in clay deposits would be the perfect environment for early cell
formation

\begin{itemize}
\item Deep sea vents

\begin{itemize}
\item Deep under the Earth's seas, there are vents where seawater comes
into contact with minerals from the planet's crust, reacting to
create a warm, alkaline (high on the pH scale) environment
containing hydrogen. The process creates mineral-rich chimneys with
alkaline and acidic fluids, providing a source of energy that
facilitates chemical reactions between hydrogen and carbon dioxide
to form increasingly complex organic compounds.
\url{https://www.sciencedaily.com/releases/2019/11/191104112437.htm\#:\~:text=Summary\%3A,vents}
rather than shallow pools.\&text=Some of the world's
oldest,originated in such underwater vents.
\item The researchers found that molecules with longer carbon chains
needed heat in order to form themselves into a vesicle (protocell).
An alkaline solution helped the fledgling vesicles keep their
electric charge. A saltwater environment also proved helpful, as the
fat molecules banded together more tightly in a salty fluid, forming
more stable vesicles
\item The researchers also point out that deep-sea hydrothermal vents are
not unique to Earth.
\item Authors of the new theory argue the environmental conditions in
porous hydrothermal vents --- where heated, mineral-laden seawater
spews from cracks in the ocean crust --- created a gradient in
positively charged protons that served as a "battery" to fuel the
creation of organic molecules and proto-cells. Later, primitive
cellular pumps gradually evolved the ability to use a different type
of gradient --- the difference in sodium particles inside and
outside the cell --- as a battery to power the construction of
complex molecules like proteins.
\url{https://www.livescience.com/26173-hydrothermal-vent-life-origins.html}
\item thriving on a chemical soup rich in hydrogen, carbon dioxide, and
sulfur, spewing from the geysers
\url{https://www.whoi.edu/press-room/news-release/study-tests-theory-that-life-originated-at-deep-sea-vents/}
\end{itemize}
\end{itemize}

warm, alkaline enviroment

alkaline and acidic fluids provide source of energy

creates a gradient that can be harnessed for energy

complex reactions between hydrogen and carbon dioxide

super rich in minerals

\begin{itemize}
\item tides of ponds

\begin{itemize}
\item Researchers report that shallow bodies of water, on the order of 10
centimeters deep, could have held high concentrations of what many
scientists believe to be a key ingredient for jump-starting life on
Earth: nitrogen.
\url{https://news.mit.edu/2019/earth-earliest-life-ponds-not-oceans-0412}
\item Atmospheric nitrogen consists of two nitrogen molecules, linked via
a strong triple bond, that can only be broken by an extremely
energetic event --- namely, lightning. "Lightning is like a really
intense bomb going off," Ranjan says. "It produces enough energy
that it breaks that triple bond in our atmospheric nitrogen gas, to
produce nitrogenous oxides that can then rain down into water
bodies."
\item In the ocean, ultraviolet light and dissolved iron would have made
nitrogenous oxides far less available for synthesizing living
organisms. In shallow ponds, however, life would have had a better
chance to take hold. That's mainly because ponds have much less
volume over which compounds can be diluted. As a result, nitrogenous
oxides would have built up to much higher concentrations in ponds.
\item In environments any deeper or larger, nitrogenous oxides would
simply have been too diluted, precluding any participation in
origin-of-life chemistry.
\item\relax [about rna] Having bonded in pairs at low tide, these newly formed
molecular strands would then dissociate at high tide, when salt
concentrations were reduced, providing what Lathe terms a
self-replicating system.
\url{https://www.scientificamerican.com/article/moon-life-tides/\#:\~:text=The}
ocean tides mirror life itself.\&text=Life emerged some 700
million,tides were much more extreme.
\end{itemize}

\item fixed nitrogen is essential for life

\item lighting

\item dissolved iron the oceans

\item high concentration of nitrogen in ponds

\item otherwise would be to diluted

\item rns would dissaocitate and reasociate at different tides

\item at home in the universe

\begin{itemize}
\item The chance that a single chemical is self replicating is very low
\item however, this chemical is much more likely to another chemical
\item as the number of chemicals and complexity increases, the likelihood
that these chemicals will become cyclic increases exponentially\\
\item life doesnt just become likely, it becomes inevitable.
\end{itemize}
\end{itemize}

\section{Outline}
\label{sec:orgb901e9c}
Life is amazing, incredibly complex, the question becomes, how did it
start?

Three theories:

volcanic clay deep sea vents tides of ponds

But there's more than just theories:

miller-urey experiment

TRANSITION IDEA: While the results of this experiment seem profound,
looking at this same scenario from a mathematical perspective reveals
startling new conclusions

At home in the universe

Life, is -- inevitable.

Sponsered by shabang studios!

\section{Script}
\label{sec:org024e37f}
Song: \url{https://artlist.io/song/12965/fallen}
\end{document}
