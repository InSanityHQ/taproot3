% Created 2021-09-11 Sat 16:40
% Intended LaTeX compiler: xelatex
\documentclass[letterpaper]{article}
\usepackage{graphicx}
\usepackage{grffile}
\usepackage{longtable}
\usepackage{wrapfig}
\usepackage{rotating}
\usepackage[normalem]{ulem}
\usepackage{amsmath}
\usepackage{textcomp}
\usepackage{amssymb}
\usepackage{capt-of}
\usepackage{hyperref}
\usepackage[margin=1in]{geometry}
\usepackage{fontspec}
\usepackage{indentfirst}
\setmainfont[ItalicFont = LiberationSans-Italic, BoldFont = LiberationSans-Bold, BoldItalicFont = LiberationSans-BoldItalic]{LiberationSans}
\newfontfamily\NHLight[ItalicFont = LiberationSansNarrow-Italic, BoldFont       = LiberationSansNarrow-Bold, BoldItalicFont = LiberationSansNarrow-BoldItalic]{LiberationSansNarrow}
\newcommand\textrmlf[1]{{\NHLight#1}}
\newcommand\textitlf[1]{{\NHLight\itshape#1}}
\let\textbflf\textrm
\newcommand\textulf[1]{{\NHLight\bfseries#1}}
\newcommand\textuitlf[1]{{\NHLight\bfseries\itshape#1}}
\usepackage{fancyhdr}
\pagestyle{fancy}
\usepackage{titlesec}
\usepackage{titling}
\makeatletter
\lhead{\textbf{\@title}}
\makeatother
\rhead{\textrmlf{Compiled} \today}
\lfoot{\theauthor\ \textbullet \ \textbf{2021-2022}}
\cfoot{}
\rfoot{\textrmlf{Page} \thepage}
\titleformat{\section} {\Large} {\textrmlf{\thesection} {|}} {0.3em} {\textbf}
\titleformat{\subsection} {\large} {\textrmlf{\thesubsection} {|}} {0.2em} {\textbf}
\titleformat{\subsubsection} {\large} {\textrmlf{\thesubsubsection} {|}} {0.1em} {\textbf}
\setlength{\parskip}{0.45em}
\renewcommand\maketitle{}
\author{Huxley Marvit}
\date{\today}
\title{Free Association Bio Review}
\hypersetup{
 pdfauthor={Huxley Marvit},
 pdftitle={Free Association Bio Review},
 pdfkeywords={},
 pdfsubject={},
 pdfcreator={Emacs 27.2 (Org mode 9.4.4)}, 
 pdflang={English}}
\begin{document}

\maketitle
\noindent\rule{\textwidth}{0.5pt}

Others:
\href{KBe2020bio101retFreeAssociation.org}{KBe2020bio101retFreeAssociation} -
Exr0n

Free associate: atoms, 2nd law of thermodynamics, energy,
electronegativity, polarity, types of chemical bonds, lipid(s),
carbohydrate(s), protein(s), amino acid(s), fold, enzyme.

\begin{itemize}
\item Atoms

\begin{itemize}
\item Exhibit electronegativity, which allows them to
\item Form molecules, which exhibits the property of polarity
\item Depending upon polarity, different types of bonds can be formed.
\item Covalent bonds are the bonds which peptides exhibit, the bond which
joins amino acids, which form proteins when chained together.
\item These proteins then fold, and the way they fold is critical to their
function.
\href{KBe2020bio101refProtienFoldingConflict.org}{KBe2020bio101refProtienFoldingConflict}
\item Some of these functions are turning carbohydrates into ATP.
\item ATP is biological energy (along with some lipids which form fats),
which is fundamental for reactions to occur.
\item A way of lowering the activation energy for reactions to take place
is with enzymes.
\item This increases reaction rate, just like heat does.
\item Each of these reactions contributes to the inevitable heat death of
the universe, due to the second law of thermodynamics
\end{itemize}
\end{itemize}

\section{Never done a free associate before, so lets see how this goes.}
\label{sec:org518d669}
\textbf{Atoms} Are the fundamental building blocks of the universe. Atoms make
up matter, and they are made of protons, neutrons, and electrons. The
number of protons within a given atom determines what element it is. The
neutrons bind said protons together. Electrons reside on rings (orbits)
around the nucleus of the atom, and are attracted to this nucleus. The
count and placement of these electrons, in part, determine the
\textbf{electronegativity} of an atom. Electronegativity is a measure of an
atoms attraction to shared electron pairs. It is fundamental in forming
\textbf{chemical bonds}. Bond types are dependent upon the difference in
electronegativity of the atoms bonding. A difference between 0.4 1.7 (if
I remember correctly) forms a polar covalent bond. Below 0.4 is
non-polar, and above 1.7 is ionic. These differences in
electronegativity also effect the \textbf{polarity} of the molecule they form.
Polarity is a very important concept, and effects how molecules interact
with one another. For example, this allows water to form hydrogen bonds,
and quite a bit more. Polarity determines whether a substance is
hydrophobic or hydrophilic, which determines the way \textbf{proteins} \textbf{fold}.
The way proteins fold determines the function of the protein -- one such
function is converting \textbf{carbohydrates} to ATP. ATP (along with \textbf{Lipids}
which is commonly used to hold energy in the form of fat) is the primary
\textbf{energy} carrying molecule of virtually all carbon based life forms.
Energy is required for chemical reactions to occur. As things always
seek their lowest energy form, reactants can get stuck in a "valley,"
where the energy required to initiate a reaction (the activation energy)
is higher than their current state. In order to surpass this barrier,
energy must be added to the reaction. One way of lowering this
activation energy is with \textbf{enzymes}. Enzymes are a type of bio-catalyst
which, as previously noted, helps facilitate reactions by lowering their
activation energy. This increases reaction rate, just like heat does.
every single one of these reactions contributes to the inevitable heat
death of the universe due to \textbf{The Second Law of Thermodynamics}, which
states that the entropy in an isolated system cannot decrease, and
instead only increases. Entropy is the \ldots{}. my ten minute timer just
went off.

****2nd law of thermodynamics

****Energy

****Electronegativity

****Polarity

****Types of chemical bonds

****Lipids

****Carbohydrates

****Protein

****Amino acids

****fold

****enzyme
\end{document}
