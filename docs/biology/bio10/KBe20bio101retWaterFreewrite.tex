% Created 2021-09-11 Sat 16:40
% Intended LaTeX compiler: xelatex
\documentclass[letterpaper]{article}
\usepackage{graphicx}
\usepackage{grffile}
\usepackage{longtable}
\usepackage{wrapfig}
\usepackage{rotating}
\usepackage[normalem]{ulem}
\usepackage{amsmath}
\usepackage{textcomp}
\usepackage{amssymb}
\usepackage{capt-of}
\usepackage{hyperref}
\usepackage[margin=1in]{geometry}
\usepackage{fontspec}
\usepackage{indentfirst}
\setmainfont[ItalicFont = LiberationSans-Italic, BoldFont = LiberationSans-Bold, BoldItalicFont = LiberationSans-BoldItalic]{LiberationSans}
\newfontfamily\NHLight[ItalicFont = LiberationSansNarrow-Italic, BoldFont       = LiberationSansNarrow-Bold, BoldItalicFont = LiberationSansNarrow-BoldItalic]{LiberationSansNarrow}
\newcommand\textrmlf[1]{{\NHLight#1}}
\newcommand\textitlf[1]{{\NHLight\itshape#1}}
\let\textbflf\textrm
\newcommand\textulf[1]{{\NHLight\bfseries#1}}
\newcommand\textuitlf[1]{{\NHLight\bfseries\itshape#1}}
\usepackage{fancyhdr}
\pagestyle{fancy}
\usepackage{titlesec}
\usepackage{titling}
\makeatletter
\lhead{\textbf{\@title}}
\makeatother
\rhead{\textrmlf{Compiled} \today}
\lfoot{\theauthor\ \textbullet \ \textbf{2021-2022}}
\cfoot{}
\rfoot{\textrmlf{Page} \thepage}
\titleformat{\section} {\Large} {\textrmlf{\thesection} {|}} {0.3em} {\textbf}
\titleformat{\subsection} {\large} {\textrmlf{\thesubsection} {|}} {0.2em} {\textbf}
\titleformat{\subsubsection} {\large} {\textrmlf{\thesubsubsection} {|}} {0.1em} {\textbf}
\setlength{\parskip}{0.45em}
\renewcommand\maketitle{}
\author{Exr0n}
\date{\today}
\title{Exr0n Water Freewrite}
\hypersetup{
 pdfauthor={Exr0n},
 pdftitle={Exr0n Water Freewrite},
 pdfkeywords={},
 pdfsubject={},
 pdfcreator={Emacs 27.2 (Org mode 9.4.4)}, 
 pdflang={English}}
\begin{document}

\maketitle
\href{https://docs.google.com/presentation/d/1tYwRk-rYtHwZPECztaiCxoiZd8ct2XSVFBEV3YTWC90/edit?usp=sharing}{Vidyo}

In the video, H2O and SO2 move very differently, where the water
molecules appear to be more energetic and strongly attracted to the
larger molecule while the sulfur dioxide molecules stay farther away.
This probably because of the different electrostatic interactions they
have with the other molecules. Water is quite polar, with oxygen and
hydrogen's electronegativity difference being 1.2, while oxygen and
sulfer's difference is only 0.6. I know that oxygen hydrogen bonds
create strong enough dipoles that those dipole dipole interactions have
a special name: hydrogen bonds. The video shows how hydrogen bonds
between the water molecule and the larger nucleotide? cause the water to
move around, while weaker dipole dipole bonds are what primarily affect
SO2 movement. The stronger hydrogen bonds are probably able to pull the
water molecules around more than the dipole dipole attractions between
the large molecule and the SO2. Water is important for life because the
dipoles mean that it can dissolve other polar things and ions, like salt
crystals. The fact that it's polar means that it can be controlled with
non-polar molecules, like how the cell membrane makes use of hydrophobic
phospholipid tails to mostly separate the inside and outside of a cell.
The fact that the cytoplasm and cell exterior is filled with water also
helps things float around and meet each other, where as an oil filled
cell would have a hard time transporting anything because there are no
charges to randomly pull things around. Basically, what I'm guessing is
that a polar liquid is a better "default" than a non-polar one because
the charges introduce more random forces that might speed up entropy and
reactions. I didn't get to the last question

\noindent\rule{\textwidth}{0.5pt}
\end{document}
