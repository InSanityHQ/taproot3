% Created 2021-09-11 Sat 16:41
% Intended LaTeX compiler: xelatex
\documentclass[letterpaper]{article}
\usepackage{graphicx}
\usepackage{grffile}
\usepackage{longtable}
\usepackage{wrapfig}
\usepackage{rotating}
\usepackage[normalem]{ulem}
\usepackage{amsmath}
\usepackage{textcomp}
\usepackage{amssymb}
\usepackage{capt-of}
\usepackage{hyperref}
\usepackage[margin=1in]{geometry}
\usepackage{fontspec}
\usepackage{indentfirst}
\setmainfont[ItalicFont = LiberationSans-Italic, BoldFont = LiberationSans-Bold, BoldItalicFont = LiberationSans-BoldItalic]{LiberationSans}
\newfontfamily\NHLight[ItalicFont = LiberationSansNarrow-Italic, BoldFont       = LiberationSansNarrow-Bold, BoldItalicFont = LiberationSansNarrow-BoldItalic]{LiberationSansNarrow}
\newcommand\textrmlf[1]{{\NHLight#1}}
\newcommand\textitlf[1]{{\NHLight\itshape#1}}
\let\textbflf\textrm
\newcommand\textulf[1]{{\NHLight\bfseries#1}}
\newcommand\textuitlf[1]{{\NHLight\bfseries\itshape#1}}
\usepackage{fancyhdr}
\pagestyle{fancy}
\usepackage{titlesec}
\usepackage{titling}
\makeatletter
\lhead{\textbf{\@title}}
\makeatother
\rhead{\textrmlf{Compiled} \today}
\lfoot{\theauthor\ \textbullet \ \textbf{2021-2022}}
\cfoot{}
\rfoot{\textrmlf{Page} \thepage}
\titleformat{\section} {\Large} {\textrmlf{\thesection} {|}} {0.3em} {\textbf}
\titleformat{\subsection} {\large} {\textrmlf{\thesubsection} {|}} {0.2em} {\textbf}
\titleformat{\subsubsection} {\large} {\textrmlf{\thesubsubsection} {|}} {0.1em} {\textbf}
\setlength{\parskip}{0.45em}
\renewcommand\maketitle{}
\author{Houjun Liu}
\date{\today}
\title{Cells}
\hypersetup{
 pdfauthor={Houjun Liu},
 pdftitle={Cells},
 pdfkeywords={},
 pdfsubject={},
 pdfcreator={Emacs 27.2 (Org mode 9.4.4)}, 
 pdflang={English}}
\begin{document}

\maketitle


\section{Cells}
\label{sec:org64911f5}
\subsection{The Two Major Cell Types}
\label{sec:org8c8d683}
\begin{itemize}
\item \textbf{Prokaryotic cells} --- often in single-cellular cells, has a cell
wall, and contained in capsules
\item \textbf{Eukaryotic cells} --- in multicellular cell elements, contains a
plasma membranes and nucleus
\end{itemize}

\subsection{Prokaryotic vs Eukaryotic Cells}
\label{sec:orgd929ef7}
\begin{center}
\begin{tabular}{lll}
Prokaryotic Cells & Both & Eukaryotic Cells\\
\hline
Cell wall & DNA & Plasma membrane\\
Capsule container & Cytoplasm & Nucleus\\
 & Ribosomes & Mitochondria\\
 & Membranes & \\
\end{tabular}
\end{center}

\subsection{Eukaryotic Cells, a deep dive}
\label{sec:orgda3a077}
\subsubsection{Plant and Animal Cells, Compare and Contrast}
\label{sec:orgc92bd2c}
\begin{center}
\begin{tabular}{ll}
Animal Cells & Plant Cells\\
\hline
Has soft plasma membrane & Has hard cell wall\\
No chloroplast & Has chloroplast to do photosynthesis\\
Has cytoplasm & Has cytoplasm\\
Has Ribosomes & Has Ribsonmes\\
Has mitochondria & Has mitochondria\\
No plastics & Has plastids --- organelles that form pigments\\
Has cilla --- hair like extrusions & Mostly no cilla\\
\end{tabular}
\end{center}

\subsubsection{Endosymbiotic theory}
\label{sec:org9ce43e2}
See \href{KBhBIO101Endosymbiotic.org}{KBhBIO101Endosymbiotic}

\subsubsection{Organelles in Eukaryotic Cells}
\label{sec:orgf503477}
See
\href{KBhBIO101EukaryoticOrganells.org}{KBhBIO101EukaryoticOrganells}

\subsection{Cell Membrains}
\label{sec:org59b1fc6}
Eukaryotes have a thin membrane layer that helps them regulate
nutrients, defend themselves, and control I/O. See
\href{KBhBIO101CellMembraines.org}{KBhBIO101CellMembraines}

\subsection{Cell Replication}
\label{sec:orge5efbca}
Eventually, at some point, cells need to replicate itself. This, of
course, is due to the fact that your body needs to grow. This intricate
process is dependent on
\href{KBhBIO101CentralDogma.org}{KBhBIO101CentralDogma}, specifically,
\href{KBhBIO101DNAReplication.org}{KBhBIO101DNAReplication}.

The timeschedule of each cell replicating is dependent on something
called "The Cell Cycle". See
\href{KBhBIO101CellLifecycle.org}{KBhBIO101CellLifecycle}
\end{document}
