% Created 2021-09-11 Sat 16:41
% Intended LaTeX compiler: xelatex
\documentclass[letterpaper]{article}
\usepackage{graphicx}
\usepackage{grffile}
\usepackage{longtable}
\usepackage{wrapfig}
\usepackage{rotating}
\usepackage[normalem]{ulem}
\usepackage{amsmath}
\usepackage{textcomp}
\usepackage{amssymb}
\usepackage{capt-of}
\usepackage{hyperref}
\usepackage[margin=1in]{geometry}
\usepackage{fontspec}
\usepackage{indentfirst}
\setmainfont[ItalicFont = LiberationSans-Italic, BoldFont = LiberationSans-Bold, BoldItalicFont = LiberationSans-BoldItalic]{LiberationSans}
\newfontfamily\NHLight[ItalicFont = LiberationSansNarrow-Italic, BoldFont       = LiberationSansNarrow-Bold, BoldItalicFont = LiberationSansNarrow-BoldItalic]{LiberationSansNarrow}
\newcommand\textrmlf[1]{{\NHLight#1}}
\newcommand\textitlf[1]{{\NHLight\itshape#1}}
\let\textbflf\textrm
\newcommand\textulf[1]{{\NHLight\bfseries#1}}
\newcommand\textuitlf[1]{{\NHLight\bfseries\itshape#1}}
\usepackage{fancyhdr}
\pagestyle{fancy}
\usepackage{titlesec}
\usepackage{titling}
\makeatletter
\lhead{\textbf{\@title}}
\makeatother
\rhead{\textrmlf{Compiled} \today}
\lfoot{\theauthor\ \textbullet \ \textbf{2021-2022}}
\cfoot{}
\rfoot{\textrmlf{Page} \thepage}
\titleformat{\section} {\Large} {\textrmlf{\thesection} {|}} {0.3em} {\textbf}
\titleformat{\subsection} {\large} {\textrmlf{\thesubsection} {|}} {0.2em} {\textbf}
\titleformat{\subsubsection} {\large} {\textrmlf{\thesubsubsection} {|}} {0.1em} {\textbf}
\setlength{\parskip}{0.45em}
\renewcommand\maketitle{}
\author{Houjun Liu}
\date{\today}
\title{DNA Transcription}
\hypersetup{
 pdfauthor={Houjun Liu},
 pdftitle={DNA Transcription},
 pdfkeywords={},
 pdfsubject={},
 pdfcreator={Emacs 27.2 (Org mode 9.4.4)}, 
 pdflang={English}}
\begin{document}

\maketitle


\section{DNA Transcription}
\label{sec:orgfd8c788}
The process of DNA transcription is done by the RNA Polymerase Enzyme.
DNA transcription begins by ripping apart hydrogen bonds using DNAse
enzyme, then the RNA polymerease reads one side (the "template strand",
a.k.a. noncoding "antisene" strand that runs from 3' to 5') of the
double helix, recognizing each nucleotide.

The point of transcription is to recognize the series of promoters that
code for a gene and copying them into the appropriately matching mRNA.

\definition{Gene}{information that successfully encodes a functional protein or a functional catalytic RNA}
=> "Promoter"s denotes beginning of a gene. "Terminator"s denotes the
end of gene.

\subsection{Starting Transcription}
\label{sec:org2c5d58f}
\begin{enumerate}
\item Series of utility "factors" proteins begin to assemble at the
promoter which signals transcription to call the attention of RNA
polymerase. One such signaling factors is the
\href{KBhBIO101TATABinding.org}{KBhBIO101TATABinding}.
\item RNA polyamerase binds to the Sigma Subunit => form a holoenzyme to
unwind DNA --- creates a \textbf{transcription bubble}
\item Sigma subunit informs the enzyme where to find a promoter (beginning
of binding)
\item "Enhancer" gene sequences help bind with activator proteins to help
attract RNA polymerase II
\end{enumerate}

\textbf{Promoters} Promoters are the signaling devices that mark the beginning
of a nucleotide in a gene. The strength of promoters could be modulated
to create different rates of transcription. Stronger promoters/enhancers
=> "enhance" "more." i.e. tumor viruses strengthen promoters for cell
growth

\subsection{Controlling Transcription}
\label{sec:org51e5dae}
(This applies only to promoters, \#disorganized, we have yet to get to
this process in Eukaryotes)

Between the promoter and the actual coding DNA, there is a region named
\emph{operator} that allows three types of regulatory molecules to bind to it
to alter how the gene is transcribed, namely:

\begin{itemize}
\item \textbf{Repressors}: proteins that suppress transcription
\item \textbf{Activators} are proteins that increase the transcription
\item \textbf{Inducers} catalyses repressors or activators --- making either a
strenthened activation or repression acting in conjunction with the
other regulator
\end{itemize}

\subsection{Actually transcribing}
\label{sec:org8af9f98}
The RNA Polymerase Enzyme starts at a promoter (typically found upstream
of the 5' start site) and ends at a terminator.

\begin{itemize}
\item A Box of TATAAT highlights transcription rate and the start site
\item TFIIA cofactor in RNA (polymerease?) recognizes TATAAT box, TFIIB
recognizes C/CG/CG/CGCCC upstream
\end{itemize}

The RNA ploymerease will pluck the correct corresponding nucleotide out
of the nucleus to form the antiparallel mRNA sequence.

\begin{itemize}
\item G->C
\item C->G
\item A->*U*
\item T->A
\end{itemize}

\subsection{Finishing Transcription}
\label{sec:orgdb96e06}
Transcription finishes at a gene \textbf{terminator}. This sequence will
signals the end of the gene sequence that codes for a protein.

\begin{itemize}
\item Two types in prokaryotes

\begin{itemize}
\item Rho-independent terminators --- roll back onto itself, causing the
RNA to terminate and mRNA to be release
\item Rho-dependent terminators --- activate cofactor named rho + unwind
the transcribed RNA-DNA hybrid
\end{itemize}

\item In Eukarotes

\begin{itemize}
\item Pol I genes --- transcription stopped through termination factor by
unwindng the transcribed RNA-DNA hybrid
\item Pol II genes --- don't stop until the end, but a polymerase has a
"cleavage" mechanism that clips the end out using a poly(A) tail
consensus sequence
\end{itemize}
\end{itemize}
\end{document}
