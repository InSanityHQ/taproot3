% Created 2021-09-12 Sun 22:49
% Intended LaTeX compiler: xelatex
\documentclass[letterpaper]{article}
\usepackage{graphicx}
\usepackage{grffile}
\usepackage{longtable}
\usepackage{wrapfig}
\usepackage{rotating}
\usepackage[normalem]{ulem}
\usepackage{amsmath}
\usepackage{textcomp}
\usepackage{amssymb}
\usepackage{capt-of}
\usepackage{hyperref}
\usepackage[margin=1in]{geometry}
\usepackage{fontspec}
\usepackage{indentfirst}
\setmainfont[ItalicFont = LiberationSans-Italic, BoldFont = LiberationSans-Bold, BoldItalicFont = LiberationSans-BoldItalic]{LiberationSans}
\newfontfamily\NHLight[ItalicFont = LiberationSansNarrow-Italic, BoldFont       = LiberationSansNarrow-Bold, BoldItalicFont = LiberationSansNarrow-BoldItalic]{LiberationSansNarrow}
\newcommand\textrmlf[1]{{\NHLight#1}}
\newcommand\textitlf[1]{{\NHLight\itshape#1}}
\let\textbflf\textrm
\newcommand\textulf[1]{{\NHLight\bfseries#1}}
\newcommand\textuitlf[1]{{\NHLight\bfseries\itshape#1}}
\usepackage{fancyhdr}
\pagestyle{fancy}
\usepackage{titlesec}
\usepackage{titling}
\makeatletter
\lhead{\textbf{\@title}}
\makeatother
\rhead{\textrmlf{Compiled} \today}
\lfoot{\theauthor\ \textbullet \ \textbf{2021-2022}}
\cfoot{}
\rfoot{\textrmlf{Page} \thepage}
\titleformat{\section} {\Large} {\textrmlf{\thesection} {|}} {0.3em} {\textbf}
\titleformat{\subsection} {\large} {\textrmlf{\thesubsection} {|}} {0.2em} {\textbf}
\titleformat{\subsubsection} {\large} {\textrmlf{\thesubsubsection} {|}} {0.1em} {\textbf}
\setlength{\parskip}{0.45em}
\renewcommand\maketitle{}
\author{Houjun Liu}
\date{\today}
\title{Biology Day 1}
\hypersetup{
 pdfauthor={Houjun Liu},
 pdftitle={Biology Day 1},
 pdfkeywords={},
 pdfsubject={},
 pdfcreator={Emacs 28.0.50 (Org mode 9.4.4)}, 
 pdflang={English}}
\begin{document}

\maketitle


\section{Biology Day 1}
\label{sec:org2ca7cbd}
\begin{enumerate}
\item What characteristics unite all life on earth (plants, animals, fungi,
bacteria (and archaea))? Describe these characteristics \textbf{in terms of
the biological macromolecules} you learned about in the fall.
\end{enumerate}

All life forms contain at least one cell that includes genomic
information stored as DNA --- a compound formed by nucleic acids, a form
of seperation from the outside --- through the use of a phosolipid
layer, and other constituent "cellular machinary" aimed at synthesizing
proteans --- created by synthesizing amino acids by the ribosome--- for
their functions; the functional proteins are all coded using
three-letter codons that code for specific amino acids to create the
proteins in question.

Each living cell metabolizes, and they all synthesize ATP, a large
molecule, used by all for energy to drive cellurar processses. All cells
reproduce to share their DNA.

\begin{enumerate}
\item The video revealed that all living things share a core set of 200-300
genes that have important functions in organisms (although the
sequences of the genes vary somewhat from organism to organism).
Based on what you learned in the fall, *predict what kinds of
proteins might be encoded by these core genes and share the reasoning
behind your ideas*.
\end{enumerate}

Such shared genes would probably drive the actions mentioned in question
1 --- especially on the front of DNA translation, transcription which
would synthesize proteins that then helps to create the specialised
parts of the cell.

Furthermore, the production of energy, with ATP being the shared
molecular storage thereof, will probably involve a similar pathway.

Hence, the genes perhaps encode the creation of mitocondria (for
energy), ribosomes and the basic machinary of creating m+tRNA (RNA
polymerease), and the foundations of cell wall/membranes (the
orientation of phospholipids) which are fundimentally common to all
cells.
\end{document}
