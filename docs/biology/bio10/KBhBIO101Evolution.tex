% Created 2021-09-12 Sun 22:48
% Intended LaTeX compiler: xelatex
\documentclass[letterpaper]{article}
\usepackage{graphicx}
\usepackage{grffile}
\usepackage{longtable}
\usepackage{wrapfig}
\usepackage{rotating}
\usepackage[normalem]{ulem}
\usepackage{amsmath}
\usepackage{textcomp}
\usepackage{amssymb}
\usepackage{capt-of}
\usepackage{hyperref}
\usepackage[margin=1in]{geometry}
\usepackage{fontspec}
\usepackage{indentfirst}
\setmainfont[ItalicFont = LiberationSans-Italic, BoldFont = LiberationSans-Bold, BoldItalicFont = LiberationSans-BoldItalic]{LiberationSans}
\newfontfamily\NHLight[ItalicFont = LiberationSansNarrow-Italic, BoldFont       = LiberationSansNarrow-Bold, BoldItalicFont = LiberationSansNarrow-BoldItalic]{LiberationSansNarrow}
\newcommand\textrmlf[1]{{\NHLight#1}}
\newcommand\textitlf[1]{{\NHLight\itshape#1}}
\let\textbflf\textrm
\newcommand\textulf[1]{{\NHLight\bfseries#1}}
\newcommand\textuitlf[1]{{\NHLight\bfseries\itshape#1}}
\usepackage{fancyhdr}
\pagestyle{fancy}
\usepackage{titlesec}
\usepackage{titling}
\makeatletter
\lhead{\textbf{\@title}}
\makeatother
\rhead{\textrmlf{Compiled} \today}
\lfoot{\theauthor\ \textbullet \ \textbf{2021-2022}}
\cfoot{}
\rfoot{\textrmlf{Page} \thepage}
\titleformat{\section} {\Large} {\textrmlf{\thesection} {|}} {0.3em} {\textbf}
\titleformat{\subsection} {\large} {\textrmlf{\thesubsection} {|}} {0.2em} {\textbf}
\titleformat{\subsubsection} {\large} {\textrmlf{\thesubsubsection} {|}} {0.1em} {\textbf}
\setlength{\parskip}{0.45em}
\renewcommand\maketitle{}
\author{Houjun Liu}
\date{\today}
\title{Evolution}
\hypersetup{
 pdfauthor={Houjun Liu},
 pdftitle={Evolution},
 pdfkeywords={},
 pdfsubject={},
 pdfcreator={Emacs 28.0.50 (Org mode 9.4.4)}, 
 pdflang={English}}
\begin{document}

\maketitle


\section{Evolution}
\label{sec:orga892550}
\emph{The unifying theory of all biology involving any change in the
heritable traits in a population over a long period of time.}

\textbf{Causes of of evolution} - different reproduction rates - Environmental
pressures - non-random mate choices - Migration

\textbf{Evidence for evolution} - Lab evidence of short-lifespan bacteria -
Fossels and and DNA evidence

\noindent\rule{\textwidth}{0.5pt}

\subsection{Begin by defining evolution}
\label{sec:orgea139ce}
=> Descend with modification

\textbf{Micro-evolution}: changes in alleal frequency within a population from
one generation to the next

\textbf{Macro-evolution}: descend of different special from a common ancestry
over much longer timescales

\emph{Remember: evolution happens over \textbf{deep time} --- much longer than your
monkey brain could feasibly preserved}

The size of civilization to now is about 10,000 years, which is 0.002
seconds if all history is 1 minute.

\noindent\rule{\textwidth}{0.5pt}

\subsubsection{DNA Evidence for evolution}
\label{sec:org470b740}
Comparing DNA between spcecis could show an idea of common ancestry.

\textbf{Evolution Experiment}

\begin{itemize}
\item Take bacteria
\item Introduce a filter/challenge (antibiotic)
\item Result: resistant bacterial is left, and they prosper
\end{itemize}

\subsubsection{Fossil Example}
\label{sec:org2f7016a}
\begin{itemize}
\item Analyzing fossils over time
\end{itemize}

\subsection{Origin of Life}
\label{sec:orgeff17cd}
(Before there was evolution)

\begin{itemize}
\item RNA world Hypothesis => RNA started self replicating and kabamm
\item Metabolism Evolution
\end{itemize}

The Miller-Agieri experiment: fundamental earth molecule + heats and
pressure => kabamm amino acids and DNA and other organic molecules.

\subsection{Common Ancestry}
\label{sec:orgbe400d2}
All life on earth is related by descent from a universal ancestor.

There is a certain ancestor LUCA --- which is the Last Universal Common
Ancestor.

\href{Pasted image 20210602134509.png.org}{Pasted image
20210602134509.png}

\subsection{Mechanisms of evolution}
\label{sec:org1d4f859}
\begin{itemize}
\item Natural Selection
\item Genetic drift
\item Gene flow
\item Variations/Mutations
\end{itemize}

\subsubsection{Natural Selection}
\label{sec:orgfe355da}
\begin{itemize}
\item Variatinon => for a certain trait, there are differences between
individuals
\item Heritability => differences that could be passed through generations
\item Reproductive advantage => ability to increase rate of
reproduction/competition
\end{itemize}

\emph{Natural selection could change allele frequencies in a particular
population over time.}

After a longer time, eventually, natural selection will make new
species.

\textbf{Sexual selection: a special case} \emph{The process of natural selection
acting on an organism's ability to access mates/fertilization.}

\emph{Direct Benefits} - Care, food, territory, etc.

\emph{Indirect Benefits} - Choosing of the most competent male - "Good genes"
of ornamentation (looking pretty is costly)

This could also produce harmful results (looking good also attracts
predators.)

\subsubsection{Genetic Drift + Gene Flow}
\label{sec:org49333bf}
\emph{Mechanisms of evolution without adaptation}

\textbf{Genetic Drift}

\begin{itemize}
\item Traits are not selected because they are beneficial against
environmental pressures
\item Allele frequencies change based on random chance or events
\end{itemize}

Random bottlenecks (like, colonization, a typhoon) cause the next
generation to randomly have a large alleal that's not at all
competitive.

\textbf{Gene Flow} Movement/migration of one individual with a dominante gene
over takes the others/change genetic makeup.

\begin{itemize}
\item \textbf{Genetic Drift}: one-way movement from larger population to
unestablished population causes (even recessive) genes to multiply
\item \textbf{Gene Flow}: potential for two-way movement in well-established
communities affecting population alleales (mostly dominant) by making
babies
\end{itemize}

\subsubsection{Mutations}
\label{sec:org4716661}
\href{KBhBIO101Mutations.org}{KBhBIO101Mutations}

\href{Pasted image 20210604104642.png.org}{Pasted image
20210604104642.png}

\subsubsection{Artificial Selection}
\label{sec:orga4c6a02}
A chuwawa + saint-benard mix.

\href{Pasted image 20210604104820.png.org}{Pasted image
20210604104820.png}

\subsection{Speciation}
\label{sec:org3e655aa}
When does many many mutations build up into one new species?

\begin{enumerate}
\item Establishing a barrier to gene flow (like, a large ocean)
\item Genetic divergence accumulation until reproductive seperation
\end{enumerate}

Variation => Natural Selection => Evolution => Speciation

\noindent\rule{\textwidth}{0.5pt}

\textbf{Fitness}: "how many offsprings can this organism reproduce and pass its
DNA to?" Evolution can take place when natural selection has occurred.

You could also create traits that's non benificial and gets weeded out.
\end{document}
