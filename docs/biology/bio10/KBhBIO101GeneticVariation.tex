% Created 2021-09-12 Sun 22:49
% Intended LaTeX compiler: xelatex
\documentclass[letterpaper]{article}
\usepackage{graphicx}
\usepackage{grffile}
\usepackage{longtable}
\usepackage{wrapfig}
\usepackage{rotating}
\usepackage[normalem]{ulem}
\usepackage{amsmath}
\usepackage{textcomp}
\usepackage{amssymb}
\usepackage{capt-of}
\usepackage{hyperref}
\usepackage[margin=1in]{geometry}
\usepackage{fontspec}
\usepackage{indentfirst}
\setmainfont[ItalicFont = LiberationSans-Italic, BoldFont = LiberationSans-Bold, BoldItalicFont = LiberationSans-BoldItalic]{LiberationSans}
\newfontfamily\NHLight[ItalicFont = LiberationSansNarrow-Italic, BoldFont       = LiberationSansNarrow-Bold, BoldItalicFont = LiberationSansNarrow-BoldItalic]{LiberationSansNarrow}
\newcommand\textrmlf[1]{{\NHLight#1}}
\newcommand\textitlf[1]{{\NHLight\itshape#1}}
\let\textbflf\textrm
\newcommand\textulf[1]{{\NHLight\bfseries#1}}
\newcommand\textuitlf[1]{{\NHLight\bfseries\itshape#1}}
\usepackage{fancyhdr}
\pagestyle{fancy}
\usepackage{titlesec}
\usepackage{titling}
\makeatletter
\lhead{\textbf{\@title}}
\makeatother
\rhead{\textrmlf{Compiled} \today}
\lfoot{\theauthor\ \textbullet \ \textbf{2021-2022}}
\cfoot{}
\rfoot{\textrmlf{Page} \thepage}
\titleformat{\section} {\Large} {\textrmlf{\thesection} {|}} {0.3em} {\textbf}
\titleformat{\subsection} {\large} {\textrmlf{\thesubsection} {|}} {0.2em} {\textbf}
\titleformat{\subsubsection} {\large} {\textrmlf{\thesubsubsection} {|}} {0.1em} {\textbf}
\setlength{\parskip}{0.45em}
\renewcommand\maketitle{}
\author{Houjun Liu}
\date{\today}
\title{Genetic Variation}
\hypersetup{
 pdfauthor={Houjun Liu},
 pdftitle={Genetic Variation},
 pdfkeywords={},
 pdfsubject={},
 pdfcreator={Emacs 28.0.50 (Org mode 9.4.4)}, 
 pdflang={English}}
\begin{document}

\maketitle


\section{Genetic Variation}
\label{sec:orgdf97eae}
There are two main ways by which genetic variation is introduced to a
cell cycle. Namely, \textbf{crossing over} and \textbf{independent assortment.}

\subsection{Crossing over}
\label{sec:orga81f8a6}
=> The process by which genetic information between \textbf{homologous}
(similar) chromasomes are exchanged.

During \href{KBhBIO101Meiosis.org}{KBhBIO101Meiosis}'s (critically,
ONLY meiosis') metaphase 1, homologous chromasomes pair with each other
and exchange like segments of their genetic material.

In this fashion, genetic variation is purposefully introduced into the
offspring to enable more competitive variation in downstream gametes.

\subsection{Independent Assortment}
\label{sec:org184964f}
During M1 and M2, which chromasomes/chromatids end up on which of the
four daughter cells is up to random chance based on which side of the
two spindles they are on.

Hence, this random combination of cells then create more variation in
how the genetic material of the grandparents are distributed amoungst
the daughter cells.

\subsection{Also, Mutations}
\label{sec:org5c6e3b4}
Cells could also just decide to mutate their DNAs. Which is rather
random but does introduce genetic variation. So
\href{KBhBIO101Mutations.org}{KBhBIO101Mutations}
\end{document}
