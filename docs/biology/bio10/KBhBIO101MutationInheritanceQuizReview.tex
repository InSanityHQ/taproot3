% Created 2021-09-11 Sat 16:41
% Intended LaTeX compiler: xelatex
\documentclass[letterpaper]{article}
\usepackage{graphicx}
\usepackage{grffile}
\usepackage{longtable}
\usepackage{wrapfig}
\usepackage{rotating}
\usepackage[normalem]{ulem}
\usepackage{amsmath}
\usepackage{textcomp}
\usepackage{amssymb}
\usepackage{capt-of}
\usepackage{hyperref}
\usepackage[margin=1in]{geometry}
\usepackage{fontspec}
\usepackage{indentfirst}
\setmainfont[ItalicFont = LiberationSans-Italic, BoldFont = LiberationSans-Bold, BoldItalicFont = LiberationSans-BoldItalic]{LiberationSans}
\newfontfamily\NHLight[ItalicFont = LiberationSansNarrow-Italic, BoldFont       = LiberationSansNarrow-Bold, BoldItalicFont = LiberationSansNarrow-BoldItalic]{LiberationSansNarrow}
\newcommand\textrmlf[1]{{\NHLight#1}}
\newcommand\textitlf[1]{{\NHLight\itshape#1}}
\let\textbflf\textrm
\newcommand\textulf[1]{{\NHLight\bfseries#1}}
\newcommand\textuitlf[1]{{\NHLight\bfseries\itshape#1}}
\usepackage{fancyhdr}
\pagestyle{fancy}
\usepackage{titlesec}
\usepackage{titling}
\makeatletter
\lhead{\textbf{\@title}}
\makeatother
\rhead{\textrmlf{Compiled} \today}
\lfoot{\theauthor\ \textbullet \ \textbf{2021-2022}}
\cfoot{}
\rfoot{\textrmlf{Page} \thepage}
\titleformat{\section} {\Large} {\textrmlf{\thesection} {|}} {0.3em} {\textbf}
\titleformat{\subsection} {\large} {\textrmlf{\thesubsection} {|}} {0.2em} {\textbf}
\titleformat{\subsubsection} {\large} {\textrmlf{\thesubsubsection} {|}} {0.1em} {\textbf}
\setlength{\parskip}{0.45em}
\renewcommand\maketitle{}
\author{Houjun Liu}
\date{\today}
\title{Mutation and Inheritance Quiz Review}
\hypersetup{
 pdfauthor={Houjun Liu},
 pdftitle={Mutation and Inheritance Quiz Review},
 pdfkeywords={},
 pdfsubject={},
 pdfcreator={Emacs 27.2 (Org mode 9.4.4)}, 
 pdflang={English}}
\begin{document}

\maketitle


\section{Mutation and Inheritance}
\label{sec:org1294635}
\subsection{Cell Division, Cell Cycle \& It's Regulation}
\label{sec:orge405416}
\textbf{Each cell lives and reproduced on a cycle; unsurprisingly, this is
called the \href{KBhBIO101CellLifecycle.org}{KBhBIO101CellLifecycle}!}

\begin{itemize}
\item These cell cycles create
\href{KBhBIO101GeneticVariation.org}{KBhBIO101GeneticVariation},
even in \href{KBhBIO101Mitosis.org}{KBhBIO101Mitosis}, because yes!,
in mitosis, there could be
\href{KBhBIO101Mutations.org}{KBhBIO101Mutations} which introduce
variation
\item However \href{KBhBIO101Mutations.org}{KBhBIO101Mutations} could
cause cancer if left unchecked, so we have
\href{KBhBIO101CellCycleRegulation.org}{KBhBIO101CellCycleRegulation}
to keep this cycle check.
\end{itemize}

*At the end of the cell cycle, a little bit of a thing happens where the
cell replicates (or makes offsprings, so not necessarily exact copies
of) itself. This bit of a thing's called
\href{KBhBIO101CellReproduction.org}{KBhBIO101CellReproduction}.*

\begin{itemize}
\item This reproduction process uses one of either
\href{KBhBIO101Mitosis.org}{KBhBIO101Mitosis} (exact copy, for
somatic cells (not sperm/egg) only) or
\item \href{KBhBIO101Meiosis.org}{KBhBIO101Meiosis} (half, randomly-mixed
genetic info, for gametes (sperm/egg) only).
\end{itemize}

\subsection{Genetics and Inheritance}
\label{sec:org93354d7}
\href{KBhBIO101GeneticVariation.org}{KBhBIO101GeneticVariation} is
like, really good. However, its woefully complicated and there are at
least 3 ways I think of that it happens.

DNA's sequence could vary by itself, and that will cause a
\href{KBhBIO101Mutations.org}{KBhBIO101Mutations}, which is actually
very rarely bad news bears and instead simply introduces genetic
variation if not doing nothing at all.

Organisms have different traits, and through
\href{KBhBIO101Meiosis.org}{KBhBIO101Meiosis} these traits are mixed.
But! which one of these traits are expressed (dad passed blue-eye, mom
passed red-eye, which one expressed?)? Well, find out at
\href{KBhBIO101Inheritance.org}{KBhBIO101Inheritance}.

Specifically, the mixture of a "heterozygous" alleals (different genes
from mother and father) will be determined by
\href{KBhBIO101GeneticInheritance.org}{KBhBIO101GeneticInheritance}.
\end{document}
