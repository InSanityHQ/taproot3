% Created 2021-09-27 Mon 12:01
% Intended LaTeX compiler: xelatex
\documentclass[letterpaper]{article}
\usepackage{graphicx}
\usepackage{grffile}
\usepackage{longtable}
\usepackage{wrapfig}
\usepackage{rotating}
\usepackage[normalem]{ulem}
\usepackage{amsmath}
\usepackage{textcomp}
\usepackage{amssymb}
\usepackage{capt-of}
\usepackage{hyperref}
\setlength{\parindent}{0pt}
\usepackage[margin=1in]{geometry}
\usepackage{fontspec}
\usepackage{svg}
\usepackage{cancel}
\usepackage{indentfirst}
\setmainfont[ItalicFont = LiberationSans-Italic, BoldFont = LiberationSans-Bold, BoldItalicFont = LiberationSans-BoldItalic]{LiberationSans}
\newfontfamily\NHLight[ItalicFont = LiberationSansNarrow-Italic, BoldFont       = LiberationSansNarrow-Bold, BoldItalicFont = LiberationSansNarrow-BoldItalic]{LiberationSansNarrow}
\newcommand\textrmlf[1]{{\NHLight#1}}
\newcommand\textitlf[1]{{\NHLight\itshape#1}}
\let\textbflf\textrm
\newcommand\textulf[1]{{\NHLight\bfseries#1}}
\newcommand\textuitlf[1]{{\NHLight\bfseries\itshape#1}}
\usepackage{fancyhdr}
\pagestyle{fancy}
\usepackage{titlesec}
\usepackage{titling}
\makeatletter
\lhead{\textbf{\@title}}
\makeatother
\rhead{\textrmlf{Compiled} \today}
\lfoot{\theauthor\ \textbullet \ \textbf{2021-2022}}
\cfoot{}
\rfoot{\textrmlf{Page} \thepage}
\renewcommand{\tableofcontents}{}
\titleformat{\section} {\Large} {\textrmlf{\thesection} {|}} {0.3em} {\textbf}
\titleformat{\subsection} {\large} {\textrmlf{\thesubsection} {|}} {0.2em} {\textbf}
\titleformat{\subsubsection} {\large} {\textrmlf{\thesubsubsection} {|}} {0.1em} {\textbf}
\setlength{\parskip}{0.45em}
\renewcommand\maketitle{}
\author{Houjun Liu}
\date{\today}
\title{Mutations}
\hypersetup{
 pdfauthor={Houjun Liu},
 pdftitle={Mutations},
 pdfkeywords={},
 pdfsubject={},
 pdfcreator={Emacs 28.0.50 (Org mode 9.4.4)}, 
 pdflang={English}}
\begin{document}

\tableofcontents



\section{Mutations}
\label{sec:orge3b338c}
Mutations are one way by which totally random, not controlled for, and
fully spontaneous genetic modifications happen to literally anywhere in
any cell's DNA during
\href{KBhBIO101CellReproduction.org}{KBhBIO101CellReproduction}.
Specifically, it involves an environmental factor or the sheer entropy
of things to directly, or indirectly (by causing/creating a oopsie
during \href{KBhBIO101DNAReplication.org}{KBhBIO101DNAReplication})
\emph{mutate} the resulting supposed-to-be-exact copy of DNA.

\begin{figure}[htbp]
\centering
\includegraphics[width=.9\linewidth]{./Pasted image 20210331134011.png}
\caption{Pasted image 20210331134011.png}
\end{figure}

Lot's of things cause mutations!
\begin{center}
\includegraphics[width=.9\linewidth]{Pasted image 20210423132309.png}
\end{center}

To figure out how mutations work, you first need to know how DNA looks
like, so here goes a\ldots{}

\noindent\rule{\textwidth}{0.5pt}

\emph{\textbf{Special Programming!} How does DNA work?}

There are two rought typos of codons on DNA, namely:

\begin{itemize}
\item \textbf{Pyrimides} - cytosine + thymine. Single ring. Which are usually
paired with\ldots{}
\item \textbf{Purines} - adenine + guanine. Double ring.
\end{itemize}

So if a mutation replaces adenine and guanine, it would has less of an
effect because a double ring is still matched with a single ring. But if
an adenine is replaced by thymine, we could have a bigger issue because
double-double ring is much longer than a traditional single/double
match.

Thank you for coming to this assembly. You could leave now. \textbf{*}

\subsection{Mutation Vocab}
\label{sec:org3820e7b}
\textbf{Trait}: characteristic of organism influnenced by its genes \& modified
by its enviroment

\textbf{Phenotype}: a collective subset of all the traits ("that looks
different from wild type") in an organism

\subsection{And now, an example}
\label{sec:org415f302}
\begin{center}
\includegraphics[width=.9\linewidth]{Pasted image 20210423131153.png}
\end{center}

Mutant hemoglobin could\ldots{} 1) with one mutation, cause a slight change
in the RBC but cause resistance to malaria 2) with two mutation, cause
sickle-cell.

Remember that DNA codes for proteins, so mutations in DNA will cause
different proteins BUT not necessarily different traits. In the case of
1-chromasome sickle-cell mutation, a protein is changed but the result
is not nocessarily a different RBC.

\subsection{Types of Mutations}
\label{sec:orgb8ae474}
There are many types/methods by which DNA mutate. These different types
dictate when they happen
(\href{KBhBIO101Meiosis.org}{KBhBIO101Meiosis} or
\href{KBhBIO101Mitosis.org}{KBhBIO101Mitosis}?), how severe they are,
and also their frequency. See\ldots{}
\href{KBhBIO101TypesOfMutations.org}{KBhBIO101TypesOfMutations}

\subsection{Impacts of mutations}
\label{sec:org41b92dd}
Mutations does one of two things, which are both pretty obvious: they
either cause a \textbf{loss of function} for the organism/cell or \textbf{gain of
function} for the organism/cell (you either loose something or gain
something\ldots{} duh).

\textbf{Loss of function mutations} - Complete loss of a proteins - Reduction
of a protein's ability to function

\textbf{Gain of function mutations} - Increase the function of a protein -
Aquire new protein function - Expression of protein in new location/time

\textbf{Neutral function} Does nothing :(

\subsection{Protein Pathways}
\label{sec:orgc991414}
Most DNA/proteins trigger in a pathway --- in that an environment factor
does not directly trigger a protein action; instead, a \emph{sequence} of
actions from the surface down happen and mutation in any of that
sequence of proteins may cause a difference in function.

For example, an growth hormone may attach to a receptor protein, which
triggers an "explosion" in KRAS protein, which then triggers cell
proliferation.

In a mutant KRAS case, however, the KRAS protein does not stop
triggering and forever triggers.

This is a case of a "gain of function" mutation that causes an abnormal
rapid cell cycle.
\end{document}
