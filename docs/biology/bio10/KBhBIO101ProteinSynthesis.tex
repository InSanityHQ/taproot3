% Created 2021-09-12 Sun 22:48
% Intended LaTeX compiler: xelatex
\documentclass[letterpaper]{article}
\usepackage{graphicx}
\usepackage{grffile}
\usepackage{longtable}
\usepackage{wrapfig}
\usepackage{rotating}
\usepackage[normalem]{ulem}
\usepackage{amsmath}
\usepackage{textcomp}
\usepackage{amssymb}
\usepackage{capt-of}
\usepackage{hyperref}
\usepackage[margin=1in]{geometry}
\usepackage{fontspec}
\usepackage{indentfirst}
\setmainfont[ItalicFont = LiberationSans-Italic, BoldFont = LiberationSans-Bold, BoldItalicFont = LiberationSans-BoldItalic]{LiberationSans}
\newfontfamily\NHLight[ItalicFont = LiberationSansNarrow-Italic, BoldFont       = LiberationSansNarrow-Bold, BoldItalicFont = LiberationSansNarrow-BoldItalic]{LiberationSansNarrow}
\newcommand\textrmlf[1]{{\NHLight#1}}
\newcommand\textitlf[1]{{\NHLight\itshape#1}}
\let\textbflf\textrm
\newcommand\textulf[1]{{\NHLight\bfseries#1}}
\newcommand\textuitlf[1]{{\NHLight\bfseries\itshape#1}}
\usepackage{fancyhdr}
\pagestyle{fancy}
\usepackage{titlesec}
\usepackage{titling}
\makeatletter
\lhead{\textbf{\@title}}
\makeatother
\rhead{\textrmlf{Compiled} \today}
\lfoot{\theauthor\ \textbullet \ \textbf{2021-2022}}
\cfoot{}
\rfoot{\textrmlf{Page} \thepage}
\titleformat{\section} {\Large} {\textrmlf{\thesection} {|}} {0.3em} {\textbf}
\titleformat{\subsection} {\large} {\textrmlf{\thesubsection} {|}} {0.2em} {\textbf}
\titleformat{\subsubsection} {\large} {\textrmlf{\thesubsubsection} {|}} {0.1em} {\textbf}
\setlength{\parskip}{0.45em}
\renewcommand\maketitle{}
\author{Houjun Liu}
\date{\today}
\title{Protein Synthesis}
\hypersetup{
 pdfauthor={Houjun Liu},
 pdftitle={Protein Synthesis},
 pdfkeywords={},
 pdfsubject={},
 pdfcreator={Emacs 28.0.50 (Org mode 9.4.4)}, 
 pdflang={English}}
\begin{document}

\maketitle


\section{Protein Synthesis}
\label{sec:orgadf7fd1}
Let's make a protein tegether!

\subsection{Before we begin, some background}
\label{sec:org614c9b0}
\textbf{Genetic Code} => "nucleotide code" found in the DNA that helps make
protein. There are two parts of this: translation and transcription.

\begin{itemize}
\item The process of \textbf{Transcription} involves taking the DNA, separating it,
and copying its corresponding pairs to RNA

\item The process of \textbf{Translation} involves taking the RNA and making
proteins.

\item \emph{Non-coding sequence}: metadata for DNA for the processors

\item \emph{Coding sequence}: DNA content for amino-acid production
\end{itemize}

Occasionally, the RNA is what we want to end up with, so then obviously
we no longer need the process of Translation.

\subsection{Transcription => converting DNA to mRNA}
\label{sec:orgd51179d}
The process of transcription is the process by which DNA is converted to
messenger RNA, a type of RNA that travels to the ribosome to create a
protein. This process is dependent on the enzyme \textbf{RNA Polymerease},
which is the primary driver that handles DNA transcription.

See \href{KBhBIO101DNATranscription.org}{KBhBIO101DNATranscription}

\subsection{mRNA processing => splicing mRNA}
\label{sec:org1e0b36d}
After the transcribed mRNA is finished, Eukaryotes only will need to go
through one additional process called "mRNA processing" that both remove
the non-protein-synthesizing Introns of the mRNA sequence, and mark the
mRNA for maturity.

Notably, \textbf{Prokaryotes does not do this!} Prokarotes' coding sequence
always makes a full protein, so we just start at promoter and end at
terminator and make a protein!

See \href{KBhBIO101mRNAPreprocessing.org}{KBhBIO101mRNAPreprocessing}

\subsubsection{Translation => RNA-directed polypeptide synthesis}
\label{sec:org1b4e81f}
And now, this is what we are here for. Now that we have a constructed
and mature mRNA, let's make a protein!

See \href{KBhBIO101Translation.org}{KBhBIO101Translation}
\end{document}
