% Created 2021-09-11 Sat 16:41
% Intended LaTeX compiler: xelatex
\documentclass[letterpaper]{article}
\usepackage{graphicx}
\usepackage{grffile}
\usepackage{longtable}
\usepackage{wrapfig}
\usepackage{rotating}
\usepackage[normalem]{ulem}
\usepackage{amsmath}
\usepackage{textcomp}
\usepackage{amssymb}
\usepackage{capt-of}
\usepackage{hyperref}
\usepackage[margin=1in]{geometry}
\usepackage{fontspec}
\usepackage{indentfirst}
\setmainfont[ItalicFont = LiberationSans-Italic, BoldFont = LiberationSans-Bold, BoldItalicFont = LiberationSans-BoldItalic]{LiberationSans}
\newfontfamily\NHLight[ItalicFont = LiberationSansNarrow-Italic, BoldFont       = LiberationSansNarrow-Bold, BoldItalicFont = LiberationSansNarrow-BoldItalic]{LiberationSansNarrow}
\newcommand\textrmlf[1]{{\NHLight#1}}
\newcommand\textitlf[1]{{\NHLight\itshape#1}}
\let\textbflf\textrm
\newcommand\textulf[1]{{\NHLight\bfseries#1}}
\newcommand\textuitlf[1]{{\NHLight\bfseries\itshape#1}}
\usepackage{fancyhdr}
\pagestyle{fancy}
\usepackage{titlesec}
\usepackage{titling}
\makeatletter
\lhead{\textbf{\@title}}
\makeatother
\rhead{\textrmlf{Compiled} \today}
\lfoot{\theauthor\ \textbullet \ \textbf{2021-2022}}
\cfoot{}
\rfoot{\textrmlf{Page} \thepage}
\titleformat{\section} {\Large} {\textrmlf{\thesection} {|}} {0.3em} {\textbf}
\titleformat{\subsection} {\large} {\textrmlf{\thesubsection} {|}} {0.2em} {\textbf}
\titleformat{\subsubsection} {\large} {\textrmlf{\thesubsubsection} {|}} {0.1em} {\textbf}
\setlength{\parskip}{0.45em}
\renewcommand\maketitle{}
\author{Houjun Liu}
\date{\today}
\title{What is Biology (biOLGO,gee)}
\hypersetup{
 pdfauthor={Houjun Liu},
 pdftitle={What is Biology (biOLGO,gee)},
 pdfkeywords={},
 pdfsubject={},
 pdfcreator={Emacs 27.2 (Org mode 9.4.4)}, 
 pdflang={English}}
\begin{document}

\maketitle
So, let's begin with a very easy concept (sarcasm)

\section{What is life?}
\label{sec:orgc64ee4d}
\textbf{When is someone dead?}

\begin{quote}
This is hard, so let's instead think about what a child?
\end{quote}

Biological Female + Sperm + relative safety + Lot's of Peanut Butter =>
Child.

But! We know that:

\begin{itemize}
\item The person does not consciously go "let's create a liver", for
instance
\item Instead\ldots{} She is following \textbf{self organization rules}

\begin{itemize}
\item Pieces self organize
\item Put some chemistry into some matter, and it will organize to create
a human being
\end{itemize}

\item Theses \textbf{self organization rules}\ldots{}

\begin{itemize}
\item Does not need to be taught
\item Is inate to system
\item Is on both the micro and macro level

\begin{itemize}
\item Micro => Cells functioning
\item Macro => Flock of birds
\end{itemize}
\end{itemize}
\end{itemize}

\subsection{Self Organization Rules}
\label{sec:orge824c90}
Simple structures causes intricate, complicated behavior.

See \href{KBhBIO101SelfOrganization.org}{KBhBIO101SelfOrganization},
self-organization.

\subsection{Types of Biology}
\label{sec:org17fca89}
\begin{itemize}
\item Translational
\item Academic
\item Textbook
\item Education
\end{itemize}

\subsubsection{Textbook Biology}
\label{sec:orgaa491af}
\begin{itemize}
\item Journaled and formalized
\item "Sum total of bio knowledge"
\end{itemize}

\subsubsection{Academic Biology}
\label{sec:orgd2dbfeb}
\begin{itemize}
\item Scientific pursue of information
\item Synthesis and discovery of information
\end{itemize}

\subsubsection{Translational Biology}
\label{sec:org274f67e}
\begin{itemize}
\item Research with outcome in mine
\item Trying to solve "a problem"
\end{itemize}

\subsubsection{Educational Biology}
\label{sec:org0cd1f09}
\begin{itemize}
\item Sharing and communicating biology
\item Luke De :heart: Edu Biology
\item Like\ldots{}

\begin{itemize}
\item Convincing people to have vaccines
\item Convincing mask wearing
\end{itemize}
\end{itemize}

Think: \textbf{how would I share this information}
\end{document}
