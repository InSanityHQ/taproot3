% Created 2021-09-11 Sat 16:41
% Intended LaTeX compiler: xelatex
\documentclass[letterpaper]{article}
\usepackage{graphicx}
\usepackage{grffile}
\usepackage{longtable}
\usepackage{wrapfig}
\usepackage{rotating}
\usepackage[normalem]{ulem}
\usepackage{amsmath}
\usepackage{textcomp}
\usepackage{amssymb}
\usepackage{capt-of}
\usepackage{hyperref}
\usepackage[margin=1in]{geometry}
\usepackage{fontspec}
\usepackage{indentfirst}
\setmainfont[ItalicFont = LiberationSans-Italic, BoldFont = LiberationSans-Bold, BoldItalicFont = LiberationSans-BoldItalic]{LiberationSans}
\newfontfamily\NHLight[ItalicFont = LiberationSansNarrow-Italic, BoldFont       = LiberationSansNarrow-Bold, BoldItalicFont = LiberationSansNarrow-BoldItalic]{LiberationSansNarrow}
\newcommand\textrmlf[1]{{\NHLight#1}}
\newcommand\textitlf[1]{{\NHLight\itshape#1}}
\let\textbflf\textrm
\newcommand\textulf[1]{{\NHLight\bfseries#1}}
\newcommand\textuitlf[1]{{\NHLight\bfseries\itshape#1}}
\usepackage{fancyhdr}
\pagestyle{fancy}
\usepackage{titlesec}
\usepackage{titling}
\makeatletter
\lhead{\textbf{\@title}}
\makeatother
\rhead{\textrmlf{Compiled} \today}
\lfoot{\theauthor\ \textbullet \ \textbf{2021-2022}}
\cfoot{}
\rfoot{\textrmlf{Page} \thepage}
\titleformat{\section} {\Large} {\textrmlf{\thesection} {|}} {0.3em} {\textbf}
\titleformat{\subsection} {\large} {\textrmlf{\thesubsection} {|}} {0.2em} {\textbf}
\titleformat{\subsubsection} {\large} {\textrmlf{\thesubsubsection} {|}} {0.1em} {\textbf}
\setlength{\parskip}{0.45em}
\renewcommand\maketitle{}
\author{Huxley}
\date{\today}
\title{PCR Problem Sheet}
\hypersetup{
 pdfauthor={Huxley},
 pdftitle={PCR Problem Sheet},
 pdfkeywords={},
 pdfsubject={},
 pdfcreator={Emacs 27.2 (Org mode 9.4.4)}, 
 pdflang={English}}
\begin{document}

\maketitle
\#ret

\noindent\rule{\textwidth}{0.5pt}

\section{Questions}
\label{sec:orgc97abe6}
\subsubsection{Describe what is happening during each cycle of the PCR:}
\label{sec:orgc7cf7a8}
\begin{enumerate}
\item \emph{Denaturation at approximately 95°C}

\begin{enumerate}
\item Denaturation splits the DNA, creating single-strands which act as
'templates.'
\end{enumerate}

\item \emph{Annealing at approximately 55°C}

\begin{enumerate}
\item Annealing allows the primers to bind to their respective sequences
on the earlier created 'templates.'
\end{enumerate}

\item \emph{Extension at approximately 72°C}

\begin{enumerate}
\item During the Extension phase, Taq polymerase creates new strands of
DNA by extending the primers.
\end{enumerate}
\end{enumerate}

\begin{enumerate}
\item In one or two sentences for each, explain why the following
\label{sec:org4265ff4}
mistakes would lead to a failed PCR reaction (assume 30 cycles of the
typical denaturation, annealing, and extension temperature sequence
unless otherwise noted):
:CUSTOM\textsubscript{ID}: in-one-or-two-sentences-for-each-explain-why-the-following-mistakes-would-lead-to-a-failed-pcr-reaction-assume-30-cycles-of-the-typical-denaturation-annealing-and-extension-temperature-sequence-unless-otherwise-noted

\begin{enumerate}
\item \emph{A human DNA polymerase was used rather than Taq DNA polymerase.} 1.
Taq DNA polymerase was isolated from temperature-tolerant bacteria,
and thus, it is thermostable. Human DNA polymerase is not, and would
be nonfunctional under the temperatures used in PCR.

\item \emph{Nucleotides were left out of the reaction.}

\begin{enumerate}
\item Nucleotides are the building blocks of DNA. Without them, the DNA
could not be synthesized.
\end{enumerate}

\item \emph{The denaturation phase temperature was set to 55°C.}

\begin{enumerate}
\item A temperature of 55°C is not sufficient to denature the DNA
strands. A temperature of \textasciitilde{}95°C is needed.
\end{enumerate}

\item \emph{The extension phase temperature was set to 4°C.}

\begin{enumerate}
\item Without a temperature of \textasciitilde{}72°C, Taq polymerase won't extend the
primers. Being sourced from bacteria used to very high
temperatures, the Taq polymerase most likely functions best under
said temperatures.
\end{enumerate}
\end{enumerate}
\end{enumerate}

\subsubsection{Luke set up his first PCR reaction recently.}
\label{sec:org1388e98}
\begin{quote}
After Luke's teacher ran his sample through the correct program on the
thermal cycler, she analyzed the results. Strangely, she noticed that
most of Luke's PCR product was \textbf{single-stranded} \textbf{rather than
double-stranded DNA}, and that his \textbf{total yield of PCR product was}
\textbf{lower than expected} (but he still had more material after
thermocycling than before). Luke said he got distracted by a classmate
while setting up the PCR, and might have left out one ingredient.
*What do you think Luke left out of his PCR reaction and why? Your
explanation should be linked to the strange results that the teacher
noticed.**
\end{quote}

Luke most likely left out either his forward primers or reverse primers.
This absence would lead to only one strand being a source of
replication, as only one strand would have the primers. Thus, many
single strands would form as the source of replication would not be able
to be replicated -- only its partner strand would be. Hence, Luke would
end up with mostly single stranded DNA, and less product than expected.
\end{document}
