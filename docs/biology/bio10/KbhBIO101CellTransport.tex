% Created 2021-09-12 Sun 22:49
% Intended LaTeX compiler: xelatex
\documentclass[letterpaper]{article}
\usepackage{graphicx}
\usepackage{grffile}
\usepackage{longtable}
\usepackage{wrapfig}
\usepackage{rotating}
\usepackage[normalem]{ulem}
\usepackage{amsmath}
\usepackage{textcomp}
\usepackage{amssymb}
\usepackage{capt-of}
\usepackage{hyperref}
\usepackage[margin=1in]{geometry}
\usepackage{fontspec}
\usepackage{indentfirst}
\setmainfont[ItalicFont = LiberationSans-Italic, BoldFont = LiberationSans-Bold, BoldItalicFont = LiberationSans-BoldItalic]{LiberationSans}
\newfontfamily\NHLight[ItalicFont = LiberationSansNarrow-Italic, BoldFont       = LiberationSansNarrow-Bold, BoldItalicFont = LiberationSansNarrow-BoldItalic]{LiberationSansNarrow}
\newcommand\textrmlf[1]{{\NHLight#1}}
\newcommand\textitlf[1]{{\NHLight\itshape#1}}
\let\textbflf\textrm
\newcommand\textulf[1]{{\NHLight\bfseries#1}}
\newcommand\textuitlf[1]{{\NHLight\bfseries\itshape#1}}
\usepackage{fancyhdr}
\pagestyle{fancy}
\usepackage{titlesec}
\usepackage{titling}
\makeatletter
\lhead{\textbf{\@title}}
\makeatother
\rhead{\textrmlf{Compiled} \today}
\lfoot{\theauthor\ \textbullet \ \textbf{2021-2022}}
\cfoot{}
\rfoot{\textrmlf{Page} \thepage}
\titleformat{\section} {\Large} {\textrmlf{\thesection} {|}} {0.3em} {\textbf}
\titleformat{\subsection} {\large} {\textrmlf{\thesubsection} {|}} {0.2em} {\textbf}
\titleformat{\subsubsection} {\large} {\textrmlf{\thesubsubsection} {|}} {0.1em} {\textbf}
\setlength{\parskip}{0.45em}
\renewcommand\maketitle{}
\author{Houjun Liu}
\date{\today}
\title{Cell Membranes Transports}
\hypersetup{
 pdfauthor={Houjun Liu},
 pdftitle={Cell Membranes Transports},
 pdfkeywords={},
 pdfsubject={},
 pdfcreator={Emacs 28.0.50 (Org mode 9.4.4)}, 
 pdflang={English}}
\begin{document}

\maketitle


\section{Cell Membrane Protean Transports}
\label{sec:org47d9998}
\subsection{Simple Diffusion}
\label{sec:orgf080178}
Due to the wonderful nature of \textbf{ENTROPY!!!}, things just tend to spread
out from high to low concentrations.

\subsection{Passive diffusion}
\label{sec:org41277d6}
\begin{itemize}
\item \textbf{Passive\ldots{} Passive Diffusion}: Non-Polar things simply "fall in" in
the direction of chemical gradient
\item \textbf{Facilitated Diffusion}: polar molecules selectively get through
protean channels
\item \textbf{Osmosis}: facilitated diffusion of water through a membrane
\end{itemize}

\noindent\rule{\textwidth}{0.5pt}

Quick ad break for some terms on Osmosis

\begin{itemize}
\item \textbf{Isotonic} => inside and outside have the same level of "osmolarity":
probablility for osmosis to happen through a semipermiable membrane
\item \textbf{Hypertonic} => inside has less osmolarity than the outside:
water/other elems will flow out of the cell
\item \textbf{Hypotonic} => outside has less osmolarity than the inside:
water/other elems will flow into the cell
\end{itemize}

\noindent\rule{\textwidth}{0.5pt}

\subsection{Active diffusion}
\label{sec:orgdb4afee}
ATP shepherds elements in

\subsection{Bulk transport}
\label{sec:org5c05100}
\begin{itemize}
\item \textbf{Phagocytosis} => take a piece of the membrane with you to form a
vesticle to introduce large solid elements, recycling the membrane
after done --- "cell eating"
\item \textbf{Pinocytosis} => take a piece of the membrane with you to form a
vesticle to introduce large area of the "outside" in --- fluid and
solid and all, recycling the membrane after done --- "cell drinking"
\item \textbf{Endocytosis} => Phagocytosis + Pinocytosis
\item \textbf{Extocytosis} => opposite of endocytosis
\end{itemize}
\end{document}
