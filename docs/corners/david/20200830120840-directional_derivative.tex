% Created 2021-09-27 Mon 11:51
% Intended LaTeX compiler: xelatex
\documentclass[letterpaper]{article}
\usepackage{graphicx}
\usepackage{grffile}
\usepackage{longtable}
\usepackage{wrapfig}
\usepackage{rotating}
\usepackage[normalem]{ulem}
\usepackage{amsmath}
\usepackage{textcomp}
\usepackage{amssymb}
\usepackage{capt-of}
\usepackage{hyperref}
\setlength{\parindent}{0pt}
\usepackage[margin=1in]{geometry}
\usepackage{fontspec}
\usepackage{svg}
\usepackage{cancel}
\usepackage{indentfirst}
\setmainfont[ItalicFont = LiberationSans-Italic, BoldFont = LiberationSans-Bold, BoldItalicFont = LiberationSans-BoldItalic]{LiberationSans}
\newfontfamily\NHLight[ItalicFont = LiberationSansNarrow-Italic, BoldFont       = LiberationSansNarrow-Bold, BoldItalicFont = LiberationSansNarrow-BoldItalic]{LiberationSansNarrow}
\newcommand\textrmlf[1]{{\NHLight#1}}
\newcommand\textitlf[1]{{\NHLight\itshape#1}}
\let\textbflf\textrm
\newcommand\textulf[1]{{\NHLight\bfseries#1}}
\newcommand\textuitlf[1]{{\NHLight\bfseries\itshape#1}}
\usepackage{fancyhdr}
\pagestyle{fancy}
\usepackage{titlesec}
\usepackage{titling}
\makeatletter
\lhead{\textbf{\@title}}
\makeatother
\rhead{\textrmlf{Compiled} \today}
\lfoot{\theauthor\ \textbullet \ \textbf{2021-2022}}
\cfoot{}
\rfoot{\textrmlf{Page} \thepage}
\renewcommand{\tableofcontents}{}
\titleformat{\section} {\Large} {\textrmlf{\thesection} {|}} {0.3em} {\textbf}
\titleformat{\subsection} {\large} {\textrmlf{\thesubsection} {|}} {0.2em} {\textbf}
\titleformat{\subsubsection} {\large} {\textrmlf{\thesubsubsection} {|}} {0.1em} {\textbf}
\setlength{\parskip}{0.45em}
\renewcommand\maketitle{}
\author{Taproot}
\date{\today}
\title{Directional Derivative}
\hypersetup{
 pdfauthor={Taproot},
 pdftitle={Directional Derivative},
 pdfkeywords={},
 pdfsubject={},
 pdfcreator={Emacs 28.0.50 (Org mode 9.4.4)}, 
 pdflang={English}}
\begin{document}

\tableofcontents


\section{Concept}
\label{sec:orge1c70f0}
For a function \(f(x,y)\) and a vector in its input space \(\vec{\textbf{v}}\), the directional derivative of \(f\) along \(\vec{\mathbf{v}}\) is the rate at which \(f\) changes as input moves along the vector. 

While it is represented by a number of symbols, \(\nabla_{\vec{\textbf{v}}}\) will be used to denote a directional derivative along \(\vec{\textbf{v}}\) for these notes.

They can be thought of as generalized \href{partial-derivatives.org}{partial derivatives} - as a partial derivative w/ respect to \(x\) tells us the amount a change in the input parallel to the \(x\) axis affects the output of the function, while this directional derivative describes how a change in any direction (as opposed to parallel to an axis) affects the change in the output in the function. 

\begin{note}
A partial derivative w/ respect to \(y\) can be thought of as a directional derivative along \(\vec{\mathbf{v}} = \widehat{\mathbf{j}}\) (so \(\frac{\partial f}{\partial y} = \nabla_{\widehat{\mathbf{j}}}\)).
\end{note}

\section{Computation}
\label{sec:org8ce978d}

Computing a directional derivative based on what we know so far is relatively simple. 

Take the example vector \(\vec{\mathbf{v}} = \left[\begin{matrix}2 \\ 3 \\ -1 \end{matrix}\right]\).

\(\nabla_{\vec{\textbf{v}}} f = 2\frac{\partial f}{\partial x} + 3\frac{\partial f}{\partial y} + (-1)\frac{\partial f}{\partial z}\)

This makes sense as \(\frac{\partial f}{\partial x}\) is the amount a change in the output of the function with a small change in \(x\), so a combination of each of the partial derivatives gives you change in the function for an arbitratry vector.

This also means that it can be computed via the \href{20200830000157-gradients.org}{gradient}: \(\nabla f \cdot \vec{\textbf{v}}\) as \(\nabla f\) is a vector of each of the partial derivatives and \(\vec{\textbf{v}}\) is a vector.

\section{Sources}
\label{sec:org98ee2b0}
\href{https://www.khanacademy.org/math/multivariable-calculus/multivariable-derivatives/partial-derivative-and-gradient-articles/a/directional-derivative-introduction}{This} describes basic directional derivatives nicely.
\end{document}
