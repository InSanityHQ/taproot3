% Created 2021-09-25 Sat 21:20
% Intended LaTeX compiler: xelatex
\documentclass[letterpaper]{article}
\usepackage{graphicx}
\usepackage{grffile}
\usepackage{longtable}
\usepackage{wrapfig}
\usepackage{rotating}
\usepackage[normalem]{ulem}
\usepackage{amsmath}
\usepackage{textcomp}
\usepackage{amssymb}
\usepackage{capt-of}
\usepackage{hyperref}
\setlength{\parindent}{0pt}
\usepackage[margin=1in]{geometry}
\usepackage{fontspec}
\usepackage{svg}
\usepackage{cancel}
\usepackage{indentfirst}
\setmainfont[ItalicFont = LiberationSans-Italic, BoldFont = LiberationSans-Bold, BoldItalicFont = LiberationSans-BoldItalic]{LiberationSans}
\newfontfamily\NHLight[ItalicFont = LiberationSansNarrow-Italic, BoldFont       = LiberationSansNarrow-Bold, BoldItalicFont = LiberationSansNarrow-BoldItalic]{LiberationSansNarrow}
\newcommand\textrmlf[1]{{\NHLight#1}}
\newcommand\textitlf[1]{{\NHLight\itshape#1}}
\let\textbflf\textrm
\newcommand\textulf[1]{{\NHLight\bfseries#1}}
\newcommand\textuitlf[1]{{\NHLight\bfseries\itshape#1}}
\usepackage{fancyhdr}
\pagestyle{fancy}
\usepackage{titlesec}
\usepackage{titling}
\makeatletter
\lhead{\textbf{\@title}}
\makeatother
\rhead{\textrmlf{Compiled} \today}
\lfoot{\theauthor\ \textbullet \ \textbf{2021-2022}}
\cfoot{}
\rfoot{\textrmlf{Page} \thepage}
\renewcommand{\tableofcontents}{}
\titleformat{\section} {\Large} {\textrmlf{\thesection} {|}} {0.3em} {\textbf}
\titleformat{\subsection} {\large} {\textrmlf{\thesubsection} {|}} {0.2em} {\textbf}
\titleformat{\subsubsection} {\large} {\textrmlf{\thesubsubsection} {|}} {0.1em} {\textbf}
\setlength{\parskip}{0.45em}
\renewcommand\maketitle{}
\author{Huxley Marvit}
\date{\today}
\title{Code of Conduct}
\hypersetup{
 pdfauthor={Huxley Marvit},
 pdftitle={Code of Conduct},
 pdfkeywords={},
 pdfsubject={},
 pdfcreator={Emacs 28.0.50 (Org mode 9.4.4)}, 
 pdflang={English}}
\begin{document}

\tableofcontents

\#ret

\noindent\rule{\textwidth}{0.5pt}

I'm not a believer in deontological ethics, so my code of conduct is as
follows:

\begin{itemize}
\item Try to minimize pain in the world
\item Don't act selfishly or be blinded by financial incentives and
therefore not follow the initial rule.
\end{itemize}

Feedback: /How do you know whether your actions will cause or minimize
pain? Consider the stories you read for the legal/ethical reflection,
Morris and Silk Road. Do you think that Morris and DPR failed to
consider whether their actions would cause pain, or did they not care,
or was it actually impossible to predict?/

My best guess with Morris is that he didn't expect to cause the pain he
did. Supposedly, a glitch was what caused many of then infected machines
to become "catatonic." As for DPR, I think he understood that he would
cause pain, but believed that his actions would bring more good into the
world than bad. As for knowing whether my actions will cause or minimize
pain, I don't. I simply have to take it as a case by case basis, using
pattern recognition to try and predict an outcome given a set of
actions. Of course, I can make a set of "rules" which will probably have
a decent success rate, but I don't want to be trapped by them like I see
so many others are. I have to try and operate in the realm of
ever-changing reality, not the realm of bureaucracy and blind process.
Of course, this will be a lot harder, but I've had a decent success rate
so far and I'm trying my best.
\end{document}
