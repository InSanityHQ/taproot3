% Created 2021-09-12 Sun 22:49
% Intended LaTeX compiler: xelatex
\documentclass[letterpaper]{article}
\usepackage{graphicx}
\usepackage{grffile}
\usepackage{longtable}
\usepackage{wrapfig}
\usepackage{rotating}
\usepackage[normalem]{ulem}
\usepackage{amsmath}
\usepackage{textcomp}
\usepackage{amssymb}
\usepackage{capt-of}
\usepackage{hyperref}
\usepackage[margin=1in]{geometry}
\usepackage{fontspec}
\usepackage{indentfirst}
\setmainfont[ItalicFont = LiberationSans-Italic, BoldFont = LiberationSans-Bold, BoldItalicFont = LiberationSans-BoldItalic]{LiberationSans}
\newfontfamily\NHLight[ItalicFont = LiberationSansNarrow-Italic, BoldFont       = LiberationSansNarrow-Bold, BoldItalicFont = LiberationSansNarrow-BoldItalic]{LiberationSansNarrow}
\newcommand\textrmlf[1]{{\NHLight#1}}
\newcommand\textitlf[1]{{\NHLight\itshape#1}}
\let\textbflf\textrm
\newcommand\textulf[1]{{\NHLight\bfseries#1}}
\newcommand\textuitlf[1]{{\NHLight\bfseries\itshape#1}}
\usepackage{fancyhdr}
\pagestyle{fancy}
\usepackage{titlesec}
\usepackage{titling}
\makeatletter
\lhead{\textbf{\@title}}
\makeatother
\rhead{\textrmlf{Compiled} \today}
\lfoot{\theauthor\ \textbullet \ \textbf{2021-2022}}
\cfoot{}
\rfoot{\textrmlf{Page} \thepage}
\titleformat{\section} {\Large} {\textrmlf{\thesection} {|}} {0.3em} {\textbf}
\titleformat{\subsection} {\large} {\textrmlf{\thesubsection} {|}} {0.2em} {\textbf}
\titleformat{\subsubsection} {\large} {\textrmlf{\thesubsubsection} {|}} {0.1em} {\textbf}
\setlength{\parskip}{0.45em}
\renewcommand\maketitle{}
\author{Huxley}
\date{\today}
\title{Validation Part B}
\hypersetup{
 pdfauthor={Huxley},
 pdftitle={Validation Part B},
 pdfkeywords={},
 pdfsubject={},
 pdfcreator={Emacs 28.0.50 (Org mode 9.4.4)}, 
 pdflang={English}}
\begin{document}

\maketitle
\noindent\rule{\textwidth}{0.5pt}

\section{Part A}
\label{sec:org3ce5207}
\begin{itemize}
\item What are the best and worst possible scores it could get?

\begin{itemize}
\item When using f1 score, the best possible score would be 1, and the
worst would be 0.
\end{itemize}

\item Why do you think the model is getting the score you're seeing?

\begin{itemize}
\item The score I'm seeing is 1. This is expected, as I am testing a
decision tree on the iris data it was trained on.
\end{itemize}

\item How do the scores between the training and testing sets compare?

\begin{itemize}
\item I would assume that the testing set score would have a lower f1
score than the training set, as the model is specifically fitted for
the training data.
\end{itemize}
\end{itemize}

\section{Part B}
\label{sec:orgd509709}
Please answer the following questions and briefly explain your answer:

\begin{itemize}
\item Thinking about the R\textsuperscript{2} metric used for evaluating regression, answer
the following questions:

\begin{itemize}
\item What is the best highest possible score you could get?

\begin{itemize}
\item The highest possible score is 1, representing a perfect fit. In
order to achieve an R\textsuperscript{2} of 1, the error must be 0.
\end{itemize}

\item If your model simply predicted the average value of the training set
no matter what the input was, what score would you get on a test set
whose average matched that of the training set?

\begin{itemize}
\item The score would be 0, as the denominator and numerator would be
equal. This would result in the equation 1-1, which equals 0.
\end{itemize}

\item What is the lowest score that you can get?

\begin{itemize}
\item Negative infinity.
\end{itemize}
\end{itemize}

\item When using accuracy to measure your model's performance on a
classification problem:

\begin{itemize}
\item What is the best possible score you could get?

\begin{itemize}
\item \begin{enumerate}
\item In a scenario with x samples, an entirely correct
classification would lead to x/x, which is 1.
\end{enumerate}
\end{itemize}

\item If your model always predicted the same class no matter what the
input, what score would you get on a test set where 85\% of the items
were in that class?

\begin{itemize}
\item 85\%
\end{itemize}
\end{itemize}

\item What is the worst possible score you can get on a dataset that only
has two classes?

\begin{itemize}
\item \begin{enumerate}
\item Every sample could be sorted incorrectly as long as there is more
than one class.
\end{enumerate}
\end{itemize}

\item A model gets a recall score of 0 for class A on a test set with
classes A, B, and C. If you take one of the test items that is in
class A and have this model predict what class it is, what will it
predict?

\begin{itemize}
\item A recall score of 0 for class A means that every item in class A was
labeled incorrectly. When given another sample in class A, the model
will predict either class B or class C.
\end{itemize}

\item A model gets a precision score of 1 for class A on a test set with
classes A, B, and C. If you take one of the test items that is in
class A and have your model predict what class it is, what will it
predict?

\begin{itemize}
\item The model could predict any of the three classes. A precision value
is only effected by true positive and false positive rate, meaning
that when given a sample in class A, predicting C or B would not
lower class A's precision.
\end{itemize}

\item If a model with classes A and B has an AUC score of 1 and you give it
an item from the test set that is in class A, what class will it
predict and what probability will it give for that class?

\begin{itemize}
\item It will predict class A, and give a probability of 100\%. This is
because an AUC score of 1 represents a perfect model.
\end{itemize}

\item If a model with classes A and B has an AUC score of 0 and you give it
an item from the test set that is in class A, what class will it
predict and what probability will it give for that class?

\begin{itemize}
\item This means that the model is always perfectly incorrect. If you were
to give it a sample from class B, it will always pick class A, and
vise versa. The model would still give a probability of 100\%.
\end{itemize}
\end{itemize}
\end{document}
