% Created 2021-09-11 Sat 09:36
% Intended LaTeX compiler: xelatex
\documentclass[letterpaper]{article}
\usepackage{graphicx}
\usepackage{grffile}
\usepackage{longtable}
\usepackage{wrapfig}
\usepackage{rotating}
\usepackage[normalem]{ulem}
\usepackage{amsmath}
\usepackage{textcomp}
\usepackage{amssymb}
\usepackage{capt-of}
\usepackage{hyperref}
\usepackage[margin=1in]{geometry}
\usepackage{fontspec}
\usepackage{indentfirst}
\setmainfont[ItalicFont = LiberationSans-Italic, BoldFont = LiberationSans-Bold, BoldItalicFont = LiberationSans-BoldItalic]{LiberationSans}
\newfontfamily\NHLight[ItalicFont = LiberationSansNarrow-Italic, BoldFont       = LiberationSansNarrow-Bold, BoldItalicFont = LiberationSansNarrow-BoldItalic]{LiberationSansNarrow}
\newcommand\textrmlf[1]{{\NHLight#1}}
\newcommand\textitlf[1]{{\NHLight\itshape#1}}
\let\textbflf\textrm
\newcommand\textulf[1]{{\NHLight\bfseries#1}}
\newcommand\textuitlf[1]{{\NHLight\bfseries\itshape#1}}
\usepackage{fancyhdr}
\pagestyle{fancy}
\usepackage{titlesec}
\usepackage{titling}
\makeatletter
\lhead{\textbf{\@title}}
\makeatother
\rhead{\textrmlf{Compiled} \today}
\lfoot{\theauthor\ \textbullet \ \textbf{2021-2022}}
\cfoot{}
\rfoot{\textrmlf{Page} \thepage}
\titleformat{\section} {\Large} {\textrmlf{\thesection} {|}} {0.3em} {\textbf}
\titleformat{\subsection} {\large} {\textrmlf{\thesubsection} {|}} {0.2em} {\textbf}
\titleformat{\subsubsection} {\large} {\textrmlf{\thesubsubsection} {|}} {0.1em} {\textbf}
\setlength{\parskip}{0.45em}
\renewcommand\maketitle{}
\author{Exr0n}
\date{\today}
\title{Bird dreaming Baobab Debrief}
\hypersetup{
 pdfauthor={Exr0n},
 pdftitle={Bird dreaming Baobab Debrief},
 pdfkeywords={},
 pdfsubject={},
 pdfcreator={Emacs 27.2 (Org mode 9.4.4)}, 
 pdflang={English}}
\begin{document}

\maketitle
\begin{itemize}
\item Mozambique central east Africa
\item early 17th century settled by Portuguese

\begin{itemize}
\item brought Catholosism
\end{itemize}

\item Breakout Questions:
\end{itemize}

\begin{verbatim}
1) What kind of symbolism does the bird man hold?  And what is the significance of his keeping of birds?

2) 
What effect or sentiment does Couto create by using vague pronouns to describe the Bird-Man, such as “that black”?


3) 
What is the symbolism of the baobab flower petals?


4) How does the white neighborhood’s perception of the birds change their opinion on the actual bird man?


5) 
In the last paragraph, while the tree is being burned, Couto’s verb choice doesn’t relate to burning at all, why does he do this?

6) Why exactly are the white people so afraid of the bird man? 

7) 
Do the white colonizers' rejection of the birds in the story reflect colonizers' rejection of traditional rituals and cultural practices of Africans?


8) 
How does the birdman escape prison?

    - Maybe the bird man represents a culture instead of a person?
    - So he would not have a name

9) 
What are the implications of the boy (Tiago) becoming the birdman?


10) 
Do the Portuguese see the birdman as a threat?


11) The Baobab is supposed to be able to commit suicide by fire, but the settlers end up burning the tree with one of their children inside. What is the significance of this?

12) Why did the author choose birds?
\end{verbatim}

Our focus: 1-3 1. > What kind of symbolism does the bird man hold? And
what is the significance of his keeping of birds? 1. Conscience of the
settlers, bird man is a metaphor for "deep down we are all human and
nice" 2. Innocent 3. Foreigner but also a native 4. He is like a bird on
the ground, that isn't his home anymore 5. Seen as "corrupting the
children" by existing 6. "The bird seller, by overstepping himself in
such a fashion, was leading the world towards other awareness."
Knowledge bad (pg 7.9) 2. > What effect or sentiment does Couto create
by using vague pronouns to describe the Bird-Man, such as "that
black"? 1. Nobody can understand what he's doing and why? Don't know too
much about him so uncertainty and misunderstanding is represented. 2.
Dehumanization tactic (not given a name) 3. > What is the symbolism of
the baobab flower petals? 1. Well being of the bird man / baobab tree,
red for blood. Bird man is replaced by the kid so the tree is okay
again? 1. Flowers were crushed by people coming to burn the tree
(pg 10) 4. General symbolism 1. Birds represent freedom 1. Bird man sold
birds in cages that "didn't even look like a prison", but the birds were
caged by the Portuguese once sold.

\begin{itemize}
\item \begin{quote}
Why doesn't the bird man have a name?
\end{quote}

\begin{itemize}
\item Maybe he is a metaphor or allegory, represents the entire culture
\item Maybe used as a dehumanization tactic
\end{itemize}

\item The bird man

\begin{itemize}
\item inspires fear
\item connects with creation stores

\begin{itemize}
\item Pied Piper

\begin{itemize}
\item Children all follow a figure who might be a danger
\item Strange man to children and animals
\end{itemize}

\item Genesis

\begin{itemize}
\item Adam and Eve name the animals, create the systems
\item VS Birdman's cages are flimsy
\end{itemize}
\end{itemize}

\item Magic

\begin{itemize}
\item Disappears from prison
\item Maybe came from a different place
\end{itemize}
\end{itemize}

\item Baobab tree (very symbolic)

\begin{itemize}
\item Tree of life
\item Looks like upside down roots

\begin{itemize}
\item Creates a fruit
\item can hold lots of water
\end{itemize}

\item Might've been the tree of knowledege
\item Portuguese name that means "may God protect"
\item Historic

\begin{itemize}
\item Predates drifted continents, humanity
\item Therefore baobab predates race and racism
\end{itemize}
\end{itemize}

\item Ending

\begin{itemize}
\item Burning of the tree
\item Tiago seems to become the tree?
\item Tiago becomes the bird man?

\begin{itemize}
\item Then is the original bird man even a native?
\item Future generations should or will recognize the importance of
native culture?
\item Author is white, but was born in Africa

\begin{itemize}
\item Children were innocent of the negative ideology, maybe Tiago
represents the author's generation
\item Maybe the children will grow up to respect the native culture
and be more connected to the land
\end{itemize}
\end{itemize}
\end{itemize}

\item Birds

\begin{itemize}
\item Played a role in freeing the bird man from prison?
\item Juxtaposing birds and children

\begin{itemize}
\item "the birds shouted and the children chirped"
\item "joyfulness was exchanged"
\end{itemize}

\item come from "deep within" Africa, which is where the colonizers want
to go

\begin{itemize}
\item Symbolic of the interior land
\item Symbolic of the order the colonizers want to impose

\begin{itemize}
\item Birds and children resist the new order
\end{itemize}
\end{itemize}

\item But why birds?

\begin{itemize}
\item All the birds here are unique and foreign to the settlers
\item A large variety of birds, can represent individuals and all the
people who are getting conquered
\item Associated words

\begin{itemize}
\item Joyful
\item Colorful
\item Flight / free spirited
\end{itemize}
\end{itemize}
\end{itemize}

\item Fire

\begin{itemize}
\item Punishment
\item Both destructive and life giving

\begin{itemize}
\item Pheonix

\begin{itemize}
\item Transformation
\item Rebirth
\item Tiago as the Pheonix
\end{itemize}
\end{itemize}
\end{itemize}

\item \begin{quote}


\begin{enumerate}
\item In the last paragraph, while the tree is being burned, Couto's
verb choice doesn't relate to burning at all, why does he do
this?
\end{enumerate}
\end{quote}

\begin{itemize}
\item "licked", "growing", "dreaming", "seduction of ash"
\end{itemize}
\end{itemize}

\noindent\rule{\textwidth}{0.5pt}
\end{document}
