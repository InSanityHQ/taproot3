% Created 2021-09-11 Sat 09:36
% Intended LaTeX compiler: xelatex
\documentclass[letterpaper]{article}
\usepackage{graphicx}
\usepackage{grffile}
\usepackage{longtable}
\usepackage{wrapfig}
\usepackage{rotating}
\usepackage[normalem]{ulem}
\usepackage{amsmath}
\usepackage{textcomp}
\usepackage{amssymb}
\usepackage{capt-of}
\usepackage{hyperref}
\usepackage[margin=1in]{geometry}
\usepackage{fontspec}
\usepackage{indentfirst}
\setmainfont[ItalicFont = LiberationSans-Italic, BoldFont = LiberationSans-Bold, BoldItalicFont = LiberationSans-BoldItalic]{LiberationSans}
\newfontfamily\NHLight[ItalicFont = LiberationSansNarrow-Italic, BoldFont       = LiberationSansNarrow-Bold, BoldItalicFont = LiberationSansNarrow-BoldItalic]{LiberationSansNarrow}
\newcommand\textrmlf[1]{{\NHLight#1}}
\newcommand\textitlf[1]{{\NHLight\itshape#1}}
\let\textbflf\textrm
\newcommand\textulf[1]{{\NHLight\bfseries#1}}
\newcommand\textuitlf[1]{{\NHLight\bfseries\itshape#1}}
\usepackage{fancyhdr}
\pagestyle{fancy}
\usepackage{titlesec}
\usepackage{titling}
\makeatletter
\lhead{\textbf{\@title}}
\makeatother
\rhead{\textrmlf{Compiled} \today}
\lfoot{\theauthor\ \textbullet \ \textbf{2021-2022}}
\cfoot{}
\rfoot{\textrmlf{Page} \thepage}
\titleformat{\section} {\Large} {\textrmlf{\thesection} {|}} {0.3em} {\textbf}
\titleformat{\subsection} {\large} {\textrmlf{\thesubsection} {|}} {0.2em} {\textbf}
\titleformat{\subsubsection} {\large} {\textrmlf{\thesubsubsection} {|}} {0.1em} {\textbf}
\setlength{\parskip}{0.45em}
\renewcommand\maketitle{}
\author{Huxley}
\date{\today}
\title{Heart of Darkness Essay}
\hypersetup{
 pdfauthor={Huxley},
 pdftitle={Heart of Darkness Essay},
 pdfkeywords={},
 pdfsubject={},
 pdfcreator={Emacs 27.2 (Org mode 9.4.4)}, 
 pdflang={English}}
\begin{document}

\maketitle
\noindent\rule{\textwidth}{0.5pt}

\section{This is fun!(?)}
\label{sec:org8093861}
\subsection{Prompt}
\label{sec:org229f14e}
\begin{verbatim}
Heart of Darkness Analytical Essay

English 10: Landscapes of the Self and Other

For your first literary analysis paper, you will be coming up with your own interpretive argument about some aspect of Heart of Darkness.

OPTION 1: Choose a recurring word, motif, pattern, or character

Choose a word, motif, pattern, or character that you’ve noticed throughout the book, and construct an analytical, argumentative essay around it. For example, you might want to look at specific moments where you see “light” and “dark” imagery. Or, maybe you want to look at every time the word “wilderness” shows up. Perhaps you want to analyze the role of women in the book, or the way that Marlow writes about the jungle. Again, if you choose a theme we’ve discussed in class, you need to go above and beyond what we’ve talked about—you need to add something new to the conversation. For this option, I recommend looking at a repetition or a character that/who seems to change throughout the book. If you look at a changing use of the same language or at the arc of a character, it will be easier for you to construct a forward-moving argument.

OPTION 2: Choose a moment in the text

Pick an excerpt of no more than 1/3-1/2 a page from the book, and construct an analytical, argumentative essay around it. To do this well you need to pay close attention to language: word choice, sentence structure, tone, other literary/rhetorical techniques, etc. If you choose a passage we’ve already discussed in class, you need to go above and beyond what we’ve talked about—you need to add something new to the conversation. Our recommendation is to choose an excerpt we haven’t treated in detail together. 

You will likely make connections to other parts of the text, particularly as you engage broader implications in your argument. However, the focus of your piece should be on your excerpt.

OPTION 3: Propose your own analytical adventure
\end{verbatim}

\subsection{Wait, thesis is due Thursday?}
\label{sec:orga6dc084}
\begin{itemize}
\item Topics

\begin{itemize}
\item Futility?

\begin{itemize}
\item Imagery of
\end{itemize}

\item Uselessness and contradiction

\begin{itemize}
\item Examples

\begin{itemize}
\item Blindfolded woman with torch
\item Bucket with hole in it
\item Doctor
\item Brickmaker
\item Giving biscuits to the dying black man
\end{itemize}

\item Whats the larger concept?
\end{itemize}
\end{itemize}
\end{itemize}

shoes\ldots{}?

\begin{enumerate}
\item Thesis: Conrad uses ironic imagery in the Heart of Darkness to
\label{sec:orge15ca32}
convey the dangers of placing independent and dissociated value in
ideas.
:CUSTOM\textsubscript{ID}: thesis-conrad-uses-ironic-imagery-in-the-heart-of-darkness-to-convey-the-dangers-of-placing-independent-and-dissociated-value-in-ideas.
our dangerous attachment to ideas.

the danger of false inherent value in ideas.

the dangers of placing inherent value in ideas

the dangers of placing independent / disassociated value in ideas

the unreliability of ideas.

"The ideas outrun the actual"

Can be applied to the European culture / colonialist movement at large.
\end{enumerate}

\subsection{Evidence bin}
\label{sec:orga745785}
\begin{itemize}
\item bucket

\begin{itemize}
\item "One evening a grass shed full of calico, cotton prints, beads, and
I don't know what else, burst into a blaze so suddenly that you
would have thought the earth had opened to let an avenging fire
consume all that trash. I was smoking my pipe quietly by my
dismantled steamer, and saw them all cutting capers in the light,
with their arms lifted high, when the stout man with moustaches came
tearing down to the river, a tin pail in his hand, assured me that
everybody was 'behaving splendidly, splendidly,' dipped about a
quart of water and tore back again. \emph{I noticed there was a hole in
the bottom of his pail.}"
\item "I strolled up. There was no hurry. You see the thing had gone off
like a box of matches. It had been hopeless from the very first. The
flame had leaped high, driven everybody back, lighted up
everything---and collapsed. The shed was already a heap of embers
glowing fiercely. A n[word] was being beaten near by. They said he
had caused the fire in some way;"
\end{itemize}

\item blindfold with torch

\begin{itemize}
\item “Then I noticed a small sketch in oils, on a panel, representing a
woman, draped and blindfolded, carrying a lighted torch. The
background was sombre---almost black. The movement of the woman was
stately, and the effect of the torchlight on the face was sinister.
\item (less important) "It arrested me, and he stood by civilly, holding
an empty half-pint champagne bottle (medical comforts) with the
candle stuck in it. To my question he said Mr. Kurtz had painted
this---in this very station more than a year ago---while waiting for
means to go to his trading post."
\end{itemize}

\item ship

\begin{itemize}
\item "Now and then a boat from the shore gave one a momentary contact
with reality [\ldots{}] For a time I would feel I belonged still to a
world of straightforward facts;"
\item "Once, I remember, we came upon a man-of-war anchored off the coast.
There wasn't even a shed there, and she was shelling the bush. It
appears the French had one of their wars going on thereabouts. Her
ensign dropped limp like a rag; the muzzles of the long six-inch
guns stuck out all over the low hull; the greasy, slimy swell swung
her up lazily and let her down, swaying her thin masts. In the empty
immensity of earth, sky, and water, there she was, incomprehensible,
firing into a continent. Pop, would go one of the six-inch guns; a
small flame would dart and vanish, a little white smoke would
disappear, a tiny projectile would give a feeble screech---and
nothing happened. Nothing could happen. There was a touch of
insanity in the proceeding, a sense of lugubrious drollery in the
sight; and it was not dissipated by somebody on board assuring me
earnestly there was a camp of natives---he called them
enemies!---hidden out of sight somewhere."
\end{itemize}
\end{itemize}

\subsection{Let's outline, shall we?}
\label{sec:orgf81c1ae}
TODO: change "idea" to something like "form over content" TODO: change
order to bucket, ship, torch

\begin{itemize}
\item Intro: Conrad uses ironic imagery in the Heart of Darkness to convey
the dangers of placing independent and dissociated value in ideas.

\item p1: bucket

\begin{itemize}
\item Bucket has a hole in it,

\begin{itemize}
\item not actually putting out the fire. It is an idea of a bucket
devoid of the context of the hole (real world)
\item wonderful metaphor for colonialism

\begin{itemize}
\item pretending to help, but not actually
\item goes and punishes a native,
\item justifying hurt and exploitation with the guise of the savior
\end{itemize}
\end{itemize}

\item To \emph{close} read:

\begin{itemize}
\item smoking my pipe quietly

\begin{itemize}
\item indifferent passive.

\begin{itemize}
\item about futility? passive mentioning being used to signal to the
broader theme?
\end{itemize}

\item fire contained vs fire uncontained

\begin{itemize}
\item 
\end{itemize}
\end{itemize}

\item "splendidly, splendidly"

\begin{itemize}
\item not about truth, about appearance.

\begin{itemize}
\item applies to conrad's view on European culture / colonialism
\end{itemize}
\end{itemize}
\end{itemize}

\item This action stems from placing inherent value in ideas, devoid of
context.
\item idea becomnes mental model not ties to reality
\end{itemize}

\item p2: torch

\begin{itemize}
\item the torch is being used simply for the sake of using a torch, for
the idea of using a torch
\item 'excuse' of vision, while truly being blinded.

\begin{itemize}
\item just like.. you guessed it! colonialism!
\end{itemize}

\item to close read:

\begin{itemize}
\item background was somber

\begin{itemize}
\item claims to be seeing the somberness, but is blind to the true
sadness and of their situation
\end{itemize}

\item movement was stately

\begin{itemize}
\item stately, like the state (doi)

\begin{itemize}
\item positions the woman as representing the colonists / europe
\end{itemize}

\item painted by kurtz pre-trip, reflects those views?
\item majestic, yet blinded. another bit of justaposition?
\end{itemize}

\item the effect of the torchlight was sinister

\begin{itemize}
\item again, kurtz pre-trip. The truth is sinister, scary.
\item corrupting the colonizer

\begin{itemize}
\item shown by kurtz rejecting the colonialist ways
\end{itemize}

\item dont see the torch, nor the effect of the torch
\end{itemize}
\end{itemize}
\end{itemize}

\item p3: ship

\begin{itemize}
\item fires at a supposed enemy, which doesn't exist (doesn't matter if it
exists, it's irrelevant)
\item uses words like "lazy" to represent how the reality of it doesn't
matter, only the idea
\item to close read:

\begin{itemize}
\item hidden out of sight somewhere

\begin{itemize}
\item justification
\end{itemize}

\item lazy, slimy, ect

\begin{itemize}
\item explained above, can be expanded upon easily
\end{itemize}

\item lugubrious drolly in the sight

\begin{itemize}
\item points out the sadness
\end{itemize}

\item feeble screech -- and nothing happened. Nothing could happen

\begin{itemize}
\item solidifying that it is just about the idea, not about reality
\end{itemize}

\item Momentary contact with reality

\begin{itemize}
\item yes, but!

\begin{itemize}
\item the ship is not in reality (doi)
\end{itemize}
\end{itemize}
\end{itemize}
\end{itemize}

\item conclusion:

\begin{itemize}
\item its all about the true nature of colonialism!

\begin{itemize}
\item that its a misplaced attempt to do what worked in one context that
doesnr work in another

\item \begin{quote}
mindless aplication of processs
\end{quote}
\end{itemize}
\end{itemize}

\item looser

\begin{itemize}
\item examples:

\begin{itemize}
\item blindfold with torch\\
\item bucket with hole
\item ship
\item order\ldots{}?
\end{itemize}

\item meta:

\begin{itemize}
\item colonialism
\end{itemize}
\end{itemize}
\end{itemize}

\section{Writing Thyme :sunglasses: (?)}
\label{sec:org7686c80}
\begin{itemize}
\item Intro: Conrad uses ironic imagery in the Heart of Darkness to convey
the dangers of placing independent and dissociated value in ideas.
\end{itemize}

\subsection{UNEDITED VERSION}
\label{sec:org6e760c5}
Converting ideas from the abstract to the real in order to achieve a
goal requires process. In fact, that's what processes are for. In Heart
of Darkness, Conrad explores the madness that occurs when processes
become ends in themselves - disconnected from the ideas and goals they
are intended to serve. This disconnection is inherent in colonialism,
where the goal is to transplant ideas - in this case, ways of doing
things and even ways of being - to unsuitable contexts. Conrad
repeatedly demonstrates that when processes are taken out of context,
they lead to absurd behaviors and outcomes - the very definition of
irony. In Heart of Darkness, he repeatedly uses ironic imagery of
decontextualized processes to warn the reader of the dangers they
create.

Through ironic imagery of a bucket, Conrad shows the danger of using
processes in the absence of context. Marlow describes one of the
evenings on his journey, in which a massive fire suddenly "burst into a
blaze" (citation). He details how a "stout man with moustaches came
tearing down to the river, a tin pail in his hand, assured me that
everybody was 'behaving splendidly, splendidly,' dipped about a quart of
water and tore back again. I noticed there was a hole in the bottom of
his pail" (citation). Of course, this man isn't truly helping put out
the fire -- after all, there is a hole in his bucket. Instead, he is
engaging mindlessly in the process of putting out a fire. The actual
reality of the fire is entirely irrelevant. The "stout man" is not
incorrect when he describes the situation as going "splendidly," because
splendid is defined as following the process. The actual state of
reality has no bearing upon whether or not something is going
"splendidly." This instance is the second time where this man has said
this line; in response to hearing that Marlow's steamer had sank, he
says his signature line then insists in agitation that Marlow must
follow the process and go "see the general manager at once" (citation).
This exact repetition of phrase------"splendidly, splendidly"------only
goes to show the man's further detachment from reality and adherence to
process. Marlow then goes on to describe that "A n[word] was being
beaten near by. They said he had caused the fire in some way;"
(citation). This situation is a microcosm of colonialism as a whole, in
which processes applied without context by the colonizer leads to the
hurt and exploitation of the colonized------justified under the guise of
a savior.

This concept does not only apply to singular people, but to larger
institutions and systems as well. Marlow notes that sometimes he gets
"momentary contact with reality." However, these were fleeting, as
"something would turn up to scare it away" (citation). He goes on to
describe one such thing, and speaks of a warship "In the empty immensity
of earth, sky, and water, there she was, incomprehensible, firing into a
continent" (citation). This warship resides in the "empty immensity,"
where, of course, there is nothing to fire at; the warship fires
anyways. Marlow goes on to describe the ships firing with words like
"feeble," "tiny," and "little," finishing by saying that "nothing
happened. Nothing could happen" (citation). The warship and the many who
keep it running are, in Marlow's words, "scar[ing] away" his contact
with reality. Not only is engaging in processes without context
supported by systems, but it is actively spread by them. When one lives
in a reality surrounded by those who do not, coexistence requires some
sort of conformity. When one does not live in reality, the processes
they partake in are unfalsifiable. This deadly combination is what
allows systems like these to spread so effectively, and what "scare[s]
away" Marlow's connection with reality. The members of the crew justify
their firing, claiming that there are enemies "hidden out of sight
somewhere" (citation). This justification could be confused with needing
ties with reality, and hence, the distinction between justification and
connection to reality must be drawn. Of course, there are no enemies.
This lack of need is what allows the firing to be "feeble." The fact
that there are no enemies is irrelevant. The justification has no ties
in reality, and its content is arbitrary. All that matters is that some
justification exists. This lack of connection to reality and ease of
spread contributes to the danger of engaging in processes without
context.

Conrad's ironic imagery of a torch serves as a metaphor for the
colonists, showing the effects of partaking in mindless processes.
Marlow notices a painting which he describes as "a small sketch in oils,
on a panel, representing a woman, draped and blindfolded, carrying a
lighted torch."(citation?) The juxtaposition and irony of a blindfolded
woman carrying a torch is immediately evident. Of course, a torch is
useless when blinded, just as a bucket is useless when it can't hold
water------both examples of following processes without the context of
reality. Marlow goes on to describe that "the background was
sombre---almost black. The movement of the woman was stately, and the
effect of the torchlight on the face was sinister." The use of the word
stately to describe the woman clearly eludes that she is meant to
represent the state. With this information, it can be inferred that the
torch represents that which the state------or more specifically, the
colonists------are bringing to the colonized. Not only are the colonists
blinded to what they are truly doing to the natives, but they are
blinded to what they are doing to themselves. Marlow describes the
"effect of the torchlight on the face" as sinister. Marlow is referring
specifically to "the face;" not the surroundings, and not \emph{her} face.
The effect of the torchlight applies only to "the face," as the
background is "almost black." This specific verbiage shows how the
colonists are themselves becoming sinister, being made sinister by what
they bring to the natives. The use of the word "the" instead of \emph{she} is
meant to generalize the statement. Another important distinction is that
the torch is not revealing the colonists to be sinister, but making them
so, hence the use of the word "effect." Not only is the torch futile and
useless as with the bucket, but it is actively making those who partake
in processes detached from reality------in this case,
colonialism------sinister.

These examples illustrate the effects of decontextualized processes.
They are useless in achieving the goals they are intended to, they lead
to violence, they are supported and propagated by systems and
institutions, and they make those who partake in them sinister. We would
like to think that we have overcome the horror present in colonialism,
that it is only something we read about in school. The truth is, the
root of colonialism -- what caused it to become so evil -- is not only
present in colonialism, but present in so much today. Processes
disconnected from reality is what allows institutions and systems to act
without morality or concern for reality, what allows systemic problems
to come into play, and what can turn even the kind hearted evil. The
best way to overcome these problems, is to first understand their root,
to re-attach these processes to reality.

\section{[[\url{https://docs.google.com/document/d/1onR1CQQjFge81saVBogQp7YOgZCGOrWpQ-6YIPkDTeE/edit?usp=sharing}][Final}
\label{sec:orgc386f35}
Version]]
:CUSTOM\textsubscript{ID}: final-version
\end{document}
