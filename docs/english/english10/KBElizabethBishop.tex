% Created 2021-09-27 Mon 12:02
% Intended LaTeX compiler: xelatex
\documentclass[letterpaper]{article}
\usepackage{graphicx}
\usepackage{grffile}
\usepackage{longtable}
\usepackage{wrapfig}
\usepackage{rotating}
\usepackage[normalem]{ulem}
\usepackage{amsmath}
\usepackage{textcomp}
\usepackage{amssymb}
\usepackage{capt-of}
\usepackage{hyperref}
\setlength{\parindent}{0pt}
\usepackage[margin=1in]{geometry}
\usepackage{fontspec}
\usepackage{svg}
\usepackage{cancel}
\usepackage{indentfirst}
\setmainfont[ItalicFont = LiberationSans-Italic, BoldFont = LiberationSans-Bold, BoldItalicFont = LiberationSans-BoldItalic]{LiberationSans}
\newfontfamily\NHLight[ItalicFont = LiberationSansNarrow-Italic, BoldFont       = LiberationSansNarrow-Bold, BoldItalicFont = LiberationSansNarrow-BoldItalic]{LiberationSansNarrow}
\newcommand\textrmlf[1]{{\NHLight#1}}
\newcommand\textitlf[1]{{\NHLight\itshape#1}}
\let\textbflf\textrm
\newcommand\textulf[1]{{\NHLight\bfseries#1}}
\newcommand\textuitlf[1]{{\NHLight\bfseries\itshape#1}}
\usepackage{fancyhdr}
\pagestyle{fancy}
\usepackage{titlesec}
\usepackage{titling}
\makeatletter
\lhead{\textbf{\@title}}
\makeatother
\rhead{\textrmlf{Compiled} \today}
\lfoot{\theauthor\ \textbullet \ \textbf{2021-2022}}
\cfoot{}
\rfoot{\textrmlf{Page} \thepage}
\renewcommand{\tableofcontents}{}
\titleformat{\section} {\Large} {\textrmlf{\thesection} {|}} {0.3em} {\textbf}
\titleformat{\subsection} {\large} {\textrmlf{\thesubsection} {|}} {0.2em} {\textbf}
\titleformat{\subsubsection} {\large} {\textrmlf{\thesubsubsection} {|}} {0.1em} {\textbf}
\setlength{\parskip}{0.45em}
\renewcommand\maketitle{}
\author{Huxley}
\date{\today}
\title{Elizabeth Bishop Poems}
\hypersetup{
 pdfauthor={Huxley},
 pdftitle={Elizabeth Bishop Poems},
 pdfkeywords={},
 pdfsubject={},
 pdfcreator={Emacs 28.0.50 (Org mode 9.4.4)}, 
 pdflang={English}}
\begin{document}

\tableofcontents

\noindent\rule{\textwidth}{0.5pt}

\#flo

\section{Note to Alexa:}
\label{sec:orgb2d0b3c}
These are my live notes (written in markdown) as I am going through the
reading. At the bottom is a brainstorm about possible topics for the
close reading paragraph.

\noindent\rule{\textwidth}{0.5pt}

\section{\(Going\ to\ the\ Bakery\)}
\label{sec:org1c3370d}
\begin{enumerate}
\item \emph{\(Elizabeth\ Bishop\)}
\label{sec:orgdc7e742}
Written in context of war and rationing and sickness

Perspective of moon

Describes pastries with humanoid features

\begin{quote}
perfect gibberish Contrast
\end{quote}

Also contrast between cheery bakery and morbid descriptions

Habit\ldots{}?

Cachaca = strong alcoholic beverage

\begin{itemize}
\item \begin{quote}
Terrific money \ldots{}?
\end{quote}

\begin{itemize}
\item Massive inflation rate increase ---
\href{https://en.wikipedia.org/wiki/Brazilian\_Miracle\#:\~:text=To reduce the dependency on,1968 to 34.55\%25 in 1974.}{Inflation
Rate}

\begin{itemize}
\item This perhaps led to people using their own form of currency, hence
\textbf{MY} terrific money

\item Also, government made false promises about indexation and debt

\begin{itemize}
\item Had the largest debt in the world

\begin{itemize}
\item This led to the 1970 energy crisis, and years of recession and
hyperinflation.
\end{itemize}
\end{itemize}
\end{itemize}
\end{itemize}
\end{itemize}

Doesn't acknowledge humanity through lack of true communication

Excuse -- gives money, yet still doesn't recognize humanity.

\begin{quote}
Invisible Side
\end{quote}

= Not recognized by the society or by the speaker

\begin{quote}
Dying, flaccid toy balloons
\end{quote}

= Dying promise. Figurative: Iridescence reflects idea?
\end{enumerate}

\subsection{\(Thesis/Brainstorming:\)}
\label{sec:orgfd3cc78}
\sout{The "black man" is not recognized by the speaker or the society as
human, only}

\sout{The speakers offhanded treatment of the "black man" represents the
colonialists relationship to the colonized}

\begin{enumerate}
\item The speakers off-handed treatment of the "black man" represents
\label{sec:org1208211}
the dismissal of the colonized persons humanity.
:CUSTOM\textsubscript{ID}: the-speakers-off-handed-treatment-of-the-black-man-represents-the-dismissal-of-the-colonized-persons-humanity.

\begin{quote}
Black, invisible side
\end{quote}

Not recognized by the society or the speaker

\begin{quote}
He speaks in perfect gibberish
\end{quote}

Speaks, but his words are viewed as holding no meaning.

Giving the money isn't actually helpful but a \emph{symbol} of helpfulness.
It's habit. The speaker is not engaging in true interaction, not
acknowledging the humanity of the "black man."

\begin{quote}
Say 'Good Night' from force of habit. Oh, mean habit! Not one word
more apt or bright? "Good Night" is completely incorrect. Said out of
habit, again, not acknowledging the humanity of the black man.
\end{quote}
\end{enumerate}

\item Theme: false promises
\label{sec:org5d2a501}
Reflected in flaccid balloons, government to people, and colonialists to
colonized

\#todo : ask Alexa how this relates and whether or not to incorporate it.

\begin{itemize}
\item Crazy Article

\begin{itemize}
\item Coup in 1964
\item Implemented all kinds of new monetary policies
\item Totally messed with the market
\item Prior to this, there was no less than 80 - 100\% inflation EVERY
YEAR.
\item This short story was written in the midst of all these false
promises
\item During inflation, very common to use other forms of currency (\emph{my}
money)

\begin{itemize}
\item cents might refer to US money
\end{itemize}
\end{itemize}
\end{itemize}

Author uses x and y to convey the overarching theme of false promises,
which was a prominent feeling in brazil due to (the coup)
\end{enumerate}

\subsection{\[Paragraph\ Begin.\]}
\label{sec:org6667aea}

\begin{enumerate}
\item Thesis:
\label{sec:org9690d76}
\begin{itemize}
\item Context

\begin{itemize}
\item Written right after a coup which led to massive legal changes in the
monetary system, totally messing with the market.
\item Large feeling of false promise

\begin{itemize}
\item "we are gonna create all these new laws, they are gonna make
everything better"
\end{itemize}
\end{itemize}

\item Bishop's Tools:

\begin{itemize}
\item Flaccid balloons -- symbol of false promises
\end{itemize}
\end{itemize}
\end{enumerate}

\item Outline:
\label{sec:org374a195}
\begin{itemize}
\item Topic Sentence / Thesis

\begin{itemize}
\item Elizabeth Bishop uses symbolism to convey the overarching themes of
false promises and deflated optimism, two immensely prominent
cultural feelings at the time of writing.
\item Elizabeth Bishop conveys the widespread cultural sentiment of
deflated optimism and unfulfilled promises through her vivid,
cynical imagery.
\end{itemize}

\item Context

\begin{itemize}
\item Coup - at the time\ldots{}

\begin{itemize}
\item Was in massive financial crisis, had a coup, implemented new
monetary laws and messed with the market, four years later still
in a financial crisis. AKA, unfulfilled promises, deflated
optimism.
\item 'This is perhaps most clearly referenced' by the \emph{my} money.

\begin{itemize}
\item Commonly used other forms of currency when times were bad
financially
\end{itemize}
\end{itemize}
\end{itemize}

\item Ex 1.

\begin{itemize}
\item Flaccid balloons

\begin{itemize}
\item \begin{quote}
The tin hides have the iridescence of dying, flaccid toy
balloons.
\end{quote}

\item Represents the once bright and cheery prospects of new leadership
and laws, now deflated.

\item Balloons are meant to be a promise of joy. (Childish hope) -
devaluing her hope after the fact by retrospectively calling it
childish
\end{itemize}
\end{itemize}

\item Ex. 2

\begin{itemize}
\item Pastries

\begin{itemize}
\item Uses sickly descriptions

\begin{itemize}
\item \begin{quote}
The gooey tarts are red and sore
\end{quote}

\item \begin{quote}
The loaves of bread lie like yellow-fever victims laid out in
a crowded ward.
\end{quote}
\end{itemize}

\item The happiness and hopefulness inherent in the concept of 'pasties'
\item The promise of joy brought by pastries has been broken, the
optimism completely sucked out of it.
\item hope has been deflated, rejected, defiled - it's now tainted and
repulsive, like illness. twisted, viscerally disgusting
\item Optimisn isn't just broken - it's repulsive. There's something
fundamentally repulsive about disease on a very basic level.- Hop
is forever tainted, sickened
\end{itemize}
\end{itemize}

\item Ex 3. false promises

\begin{itemize}
\item Good night
\item \texttt{OR}
\item False almond
\end{itemize}

\item Conclusion sentence\ldots{}?
\end{itemize}
\end{enumerate}

\section{\[Writing\ Time\]}
\label{sec:org06c6253}
\begin{enumerate}
\item \[:sunglasses:\]
\label{sec:orgc94a374}
Elizabeth Bishop conveys the widespread cultural sentiment of distrust
and deflated optimism through the vivid, cynical imagery in her poem
"Going to the Bakery." This poem was written during a time of massive
poverty in Brazil, roughly four years after a coup catalyzed massive
changes in monetary policy \{citation\}. The impact of these policies is
perhaps most clearly referenced when the speaker describes their
currency as "\emph{my} terrific money"\{citation\}. Notably, this is the only
place throughout the entire poem that the author uses italics for
emphasis. At the time, Brazil was in poverty; distrust towards the
government was rampant. Much like other times of economic and political
turmoil, it was common to use a form of currency not managed by the
legal system, hence Bishop's use of italics: "\emph{my} terrific money"
(citation). Bishop's distrust in the government is also reflected
earlier in the poem, where she describes her surroundings as having "the
iridescence of dying, flaccid, toy balloons" \{citation\}. Balloons are a
symbol of almost child-like hope -- a promise of joy. Here Bishop
describes them, however, as "dying" and "flaccid," just as the country's
sense of hope deflated while the new government failed to lift them out
of poverty \{citation\}. Bishop chooses to use toy balloons to represent
optimism, retrospectively calling it childish and naive. The speaker
then enters a bakery, describing "the gooey tarts" as "red and sore,"
the "loaves of bread\ldots{} like yellow-fever victims laid out in a crowded
ward" (citation). The promise of joy, and therefore hope, inherent in
pastries is not only deflated like the balloons, but also tainted and
repulsive on a visceral level. Hope is an illness; hope is \emph{sickly}.
Ironically, this outlook on hope simply maintains that which causes it;
without hope, one cannot achieve change, making it deadly to require
change to be hopeful.
\end{enumerate}
\end{document}
