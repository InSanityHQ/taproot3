% Created 2021-09-11 Sat 16:42
% Intended LaTeX compiler: xelatex
\documentclass[letterpaper]{article}
\usepackage{graphicx}
\usepackage{grffile}
\usepackage{longtable}
\usepackage{wrapfig}
\usepackage{rotating}
\usepackage[normalem]{ulem}
\usepackage{amsmath}
\usepackage{textcomp}
\usepackage{amssymb}
\usepackage{capt-of}
\usepackage{hyperref}
\usepackage[margin=1in]{geometry}
\usepackage{fontspec}
\usepackage{indentfirst}
\setmainfont[ItalicFont = LiberationSans-Italic, BoldFont = LiberationSans-Bold, BoldItalicFont = LiberationSans-BoldItalic]{LiberationSans}
\newfontfamily\NHLight[ItalicFont = LiberationSansNarrow-Italic, BoldFont       = LiberationSansNarrow-Bold, BoldItalicFont = LiberationSansNarrow-BoldItalic]{LiberationSansNarrow}
\newcommand\textrmlf[1]{{\NHLight#1}}
\newcommand\textitlf[1]{{\NHLight\itshape#1}}
\let\textbflf\textrm
\newcommand\textulf[1]{{\NHLight\bfseries#1}}
\newcommand\textuitlf[1]{{\NHLight\bfseries\itshape#1}}
\usepackage{fancyhdr}
\pagestyle{fancy}
\usepackage{titlesec}
\usepackage{titling}
\makeatletter
\lhead{\textbf{\@title}}
\makeatother
\rhead{\textrmlf{Compiled} \today}
\lfoot{\theauthor\ \textbullet \ \textbf{2021-2022}}
\cfoot{}
\rfoot{\textrmlf{Page} \thepage}
\titleformat{\section} {\Large} {\textrmlf{\thesection} {|}} {0.3em} {\textbf}
\titleformat{\subsection} {\large} {\textrmlf{\thesubsection} {|}} {0.2em} {\textbf}
\titleformat{\subsubsection} {\large} {\textrmlf{\thesubsubsection} {|}} {0.1em} {\textbf}
\setlength{\parskip}{0.45em}
\renewcommand\maketitle{}
\author{Exr0n}
\date{\today}
\title{Heart Of Darkness Part 1 debrief}
\hypersetup{
 pdfauthor={Exr0n},
 pdftitle={Heart Of Darkness Part 1 debrief},
 pdfkeywords={},
 pdfsubject={},
 pdfcreator={Emacs 27.2 (Org mode 9.4.4)}, 
 pdflang={English}}
\begin{document}

\maketitle
\#flo

\section{Breakouts}
\label{sec:org241efa2}
\begin{itemize}
\item \begin{quote}
What is the heart of darkness?
\end{quote}

\begin{itemize}
\item africa, congo river
\item savage = darkness
\item anthropomorphized the continent?
\end{itemize}

\item \begin{quote}
Where do we see scientific racism?
\end{quote}

\begin{itemize}
\item Phrenology, head and then mental state changes

\begin{itemize}
\item implies Africa is so horrible that its traumatic
\item size of skull used for racism
\end{itemize}

\item Described in parts instead of individuals

\begin{itemize}
\item Not a specific human, but rather masses of limbs
\end{itemize}
\end{itemize}

\item \begin{quote}
What narrative techniques does Conrad use to distance readers from
the plot?
\end{quote}

\begin{itemize}
\item Frame based story

\begin{itemize}
\item Modernism (don't know if what we are told is accurate)
\item Contrast with frankenstein frames: we always have quotes
\end{itemize}

\item Narrator disagrees with Marlow?
\item Past tense
\item Marlow has two modes: one that's dreamy and abstract for describing
the nature and the blacks, while the other is normal-ish for talking
to whites.
\item Jumpy narrative

\begin{itemize}
\item Doesn't always distinguish between thoughts and events
\item Passage of time isn't clear
\item Fog
\end{itemize}

\item No names other than Kurtz

\begin{itemize}
\item descriptions, but no other actual names
\end{itemize}

\item Marlow emotions

\begin{itemize}
\item Marlow stops talking when it gets too intense, pausing before
resuming
\end{itemize}
\end{itemize}

\item Racism

\begin{itemize}
\item So ingrained in the characters and society in the book that it feels
natural in the context of the book, but when talking about it in our
social context the whole thing feels very racist
\end{itemize}
\end{itemize}

\section{Discussion}
\label{sec:org86607b3}
\subsection{Map}
\label{sec:orgbc6dd33}
\subsection{Return to Africa = de-evolution}
\label{sec:org0a891da}
\begin{itemize}
\item Pg. 24 "i felt i was becoming scinetifically interesting"
\end{itemize}

\section{Breakouts again}
\label{sec:orge43f794}
\begin{itemize}
\item \begin{quote}
What do we know about Kurtz
\end{quote}

\begin{itemize}
\item Highly effective at collecting ivory
\item Seen as a genius or superhuman
\item Holds power and strength
\item In charge of a trading post
\item Local figurehead
\item Pioneer into the heart of darkness
\item People are upset that he's so good
\item Mostly heard about through rumors
\item Rumored sick
\end{itemize}

\item \begin{quote}
Painting pg. 29-30
\end{quote}

\begin{itemize}
\item Impressionist? Not too modern or surreal but not too many things
going on at once
\item European expansion into africa

\begin{itemize}
\item Blindfolded and carrying a torch into the darkness
\item blindfold - don't know whats going into
\item torch to civilize the natives
\end{itemize}

\item She's a woman?

\begin{itemize}
\item The position of a european women
\item He thinks they should be kept ignorant of the situation and of all
the darkness in the world
\item Women bring light but are kept ignorant of the darkness?
\item The company is willfully blinding itself to the dehumanization
\end{itemize}

\item darkness -> torchlight -> makes her sinister (light reflecting on
her face) although she was originally stately
\item Seems sinister to an outsider, but the woman doesn't think she's
sinister
\item Acting as a beacon to other people, blindly leading
\item Why did Kurtz paint it?

\begin{itemize}
\item Kurtz is a beacon?
\item Why does Kurtz see himself as blindfolded (juxtaposition with
other's description)
\item If he identifies with the painting, why did he paint a woman?
\end{itemize}

\item Draped?

\begin{itemize}
\item Not pants
\item Statue of liberty
\item Justice
\end{itemize}

\item Juxtoposition between light and dark
\end{itemize}

\item \begin{quote}
Religious Vocab
\end{quote}

\begin{itemize}
\item Painting
\item philanthropy
\item Saints
\item Devil/God knows

\begin{itemize}
\item Seen as blasphemy during the time
\end{itemize}

\item River has/is a devil/god
\item Devils of stuff that compel the whites
\item Inferno
\item River as snake
\item Other small references
\end{itemize}
\end{itemize}

\noindent\rule{\textwidth}{0.5pt}
\end{document}
