% Created 2021-09-12 Sun 22:49
% Intended LaTeX compiler: xelatex
\documentclass[letterpaper]{article}
\usepackage{graphicx}
\usepackage{grffile}
\usepackage{longtable}
\usepackage{wrapfig}
\usepackage{rotating}
\usepackage[normalem]{ulem}
\usepackage{amsmath}
\usepackage{textcomp}
\usepackage{amssymb}
\usepackage{capt-of}
\usepackage{hyperref}
\usepackage[margin=1in]{geometry}
\usepackage{fontspec}
\usepackage{indentfirst}
\setmainfont[ItalicFont = LiberationSans-Italic, BoldFont = LiberationSans-Bold, BoldItalicFont = LiberationSans-BoldItalic]{LiberationSans}
\newfontfamily\NHLight[ItalicFont = LiberationSansNarrow-Italic, BoldFont       = LiberationSansNarrow-Bold, BoldItalicFont = LiberationSansNarrow-BoldItalic]{LiberationSansNarrow}
\newcommand\textrmlf[1]{{\NHLight#1}}
\newcommand\textitlf[1]{{\NHLight\itshape#1}}
\let\textbflf\textrm
\newcommand\textulf[1]{{\NHLight\bfseries#1}}
\newcommand\textuitlf[1]{{\NHLight\bfseries\itshape#1}}
\usepackage{fancyhdr}
\pagestyle{fancy}
\usepackage{titlesec}
\usepackage{titling}
\makeatletter
\lhead{\textbf{\@title}}
\makeatother
\rhead{\textrmlf{Compiled} \today}
\lfoot{\theauthor\ \textbullet \ \textbf{2021-2022}}
\cfoot{}
\rfoot{\textrmlf{Page} \thepage}
\titleformat{\section} {\Large} {\textrmlf{\thesection} {|}} {0.3em} {\textbf}
\titleformat{\subsection} {\large} {\textrmlf{\thesubsection} {|}} {0.2em} {\textbf}
\titleformat{\subsubsection} {\large} {\textrmlf{\thesubsubsection} {|}} {0.1em} {\textbf}
\setlength{\parskip}{0.45em}
\renewcommand\maketitle{}
\author{Exr0n}
\date{\today}
\title{Close Reading Paragraph (Assessment 1)}
\hypersetup{
 pdfauthor={Exr0n},
 pdftitle={Close Reading Paragraph (Assessment 1)},
 pdfkeywords={},
 pdfsubject={},
 pdfcreator={Emacs 28.0.50 (Org mode 9.4.4)}, 
 pdflang={English}}
\begin{document}

\maketitle


\section{Assignment}
\label{sec:org79b7712}
\begin{quote}
\textbf{English 10, Assessment \#1: Close Reading Paragraph} Topic: You will
be writing on one of the short stories from our packet: "The I is
Never Alone" or "The Bird-Dreaming Baobab." (For extra challenge, you
may choose from the Elizabeth Bishop "challenge poems" at the end of
the packet). Choose an aspect of the short story or poem that will
enable close reading: a word pattern, a significant image, a literary
device, a repeated detail, etc. In your topic sentence and paragraph,
answer the following question: What does this aspect of the text
reveal about the story/poem's broader themes? This is a broad prompt,
and I am available to help you narrow it as you begin pre-writing.
This paragraph assesses the following template items:

\begin{itemize}
\item Understanding Literature: Form and Function
\item Close Reading and Argumentation
\item Structure and Mechanics
\item The Writer's Voice
\end{itemize}

\textbf{*} Important reminders:
:CUSTOM\textsubscript{ID}: important-reminders

\begin{itemize}
\item Length: One paragraph, so 250-350 words
\item Paragraph format: MLA style, double-spaced, 12-point font
\item In this paragraph, one of your primary goals is to demonstrate your
close reading skills. Close reading means unpacking the meaning of
individual words and phrases.
\item If you write about a pattern/image/etc. that we discussed in class,
you need to go above and beyond class discussion so that I can see
your own thinking.
\item Include page numbers for citations.
\end{itemize}
\end{quote}

\section{Planning}
\label{sec:org8b2596f}
\subsection{Text}
\label{sec:org68e34ea}
I wanted to use the pronouns based on narrorator vs settlers for the
bird seller in bird dreaming baobab, but I'm not sure what that "reveals
about the story/poem's broader themes". So, I will write about the
specificity of numbers in the I is never alone instead.

\subsection{Markup}
\label{sec:org2089624}
I realized that I didn't really want to write about the specificity of
numbers, because that felt too obvious after talking about it in class.
I noticed that the word "Siriak" appears in differing intensities
throughout the text, and decided to visualize it with a rolling average:
\url{https://github.com/Exr0nRandomProjects/exr0n20eng201retA1analysis}

\subsection{Thesis}
\label{sec:org71959bd}
The frequency of self reference is directly correlated with idleness,
suggesting that identity and reflection are in opposition.

\subsection{Evidence}
\label{sec:org18be2b9}
+Trough at beginning of paragraph 5 is when lands on the island,
shipwrecked. Peak halfway through paragraph 6 is when he catches and
teaches the parrots. Trough at paragraph 8 is when "he struggled to
remember that he also existed outside himself"+ < That was the old
graph, with stats before instead of around.

Half way through paragraph 2: Siriak realizes his companions are dead
and takes over the ship Spike at 6: Siriak captures parrots Dip at end
of 7: parrots call for Siriak (although he does nothing himself) Spike
at beginning of 9: "and so the weeks and months flew by"

\section{Outline}
\label{sec:orgae50ffb}
\subsection{Thesis}
\label{sec:org144af5a}
In Marcel Marien's \emph{The I Is Never Alone}, local frequency of references
to Siriak are directly correlated with idleness and reflection; this
reveals the interdependency between internal reflection and external
perception.

\subsection{Intro}
\label{sec:org970b182}
References to Siriak can be quantitatively analyzed by counting the
number of references in the surrounding words up to a radius away for
each word in the text. Using a radius of 30 words would result in a
graph as follows:
\href{https://github.com/Exr0nRandomProjects/exr0n20eng201retA1analysis/blob/master/process.py}{[[https://raw.githubusercontent.com/Exr0nRandomProjects/exr0n20eng201retA1analysis/master/chart.png}]]

\subsection{Body Points}
\label{sec:org2345313}
\sout{This chart visually spotlights the change in self-reference (TODO WC)
as the text progresses.} The spikes and troughs in the chart align with
moments of reflection and action in the story. For instance, the chart
shows a spike at the beginning of paragraph six--the first time "he
truly understood the depth of his solitude to which he was condemned".
Although Siriak is reflecting upon his own situation where there should
be no need to distinguish between himself and others, over twelve
percent of the nearby words refer to Siriak in second or third
person--from distinctly outside Siriak's inner mind and first person
solitude.

The antithesis further reinforces the entanglement of outer pronouns and
inner reflection: in the final third of paragraph seven, the parrots of
the isle begin to repeat Siriak's name. "As one might guess, the outcome
was everything Siriak might have hoped for, if not foreseen." Although
the Siriak's scheming cognition shows through, the author deliberately
minimizes the use of pronouns: Marien picks up the pace of
description--fewer words per action means pronouns can be carried
further and still flow--and uses complex, multi-phrase sentences--the
short, tacked on clauses removes the need for more Siriak-referential
pronouns.

\subsection{Conclusion}
\label{sec:org5c70f38}
Similarly, a spike in pronouns is observed at the end of paragraph
eight, where Siriak has an explicit identity crisis; and a dip in the
middle of paragraph nine, where Siriak turns to the physical outer world
and scatters mirrors about the island. This parity between external
pronouns and internal reflection shows that Siriak's identity is
primarily perceived through the lens of external pronouns, highlighting
the coupled nature of reflection and perception.

\section{[[\url{https://docs.google.com/document/d/1x\_QqfDP9v5V68n-RQ7RUQuva-EtGtTkm1hkB5KgRpuY/edit}][Compiled}
\label{sec:org5e63970}
Draft]]
:CUSTOM\textsubscript{ID}: compiled-draft

\noindent\rule{\textwidth}{0.5pt}
\end{document}
