% Created 2021-09-11 Sat 16:42
% Intended LaTeX compiler: xelatex
\documentclass[letterpaper]{article}
\usepackage{graphicx}
\usepackage{grffile}
\usepackage{longtable}
\usepackage{wrapfig}
\usepackage{rotating}
\usepackage[normalem]{ulem}
\usepackage{amsmath}
\usepackage{textcomp}
\usepackage{amssymb}
\usepackage{capt-of}
\usepackage{hyperref}
\usepackage[margin=1in]{geometry}
\usepackage{fontspec}
\usepackage{indentfirst}
\setmainfont[ItalicFont = LiberationSans-Italic, BoldFont = LiberationSans-Bold, BoldItalicFont = LiberationSans-BoldItalic]{LiberationSans}
\newfontfamily\NHLight[ItalicFont = LiberationSansNarrow-Italic, BoldFont       = LiberationSansNarrow-Bold, BoldItalicFont = LiberationSansNarrow-BoldItalic]{LiberationSansNarrow}
\newcommand\textrmlf[1]{{\NHLight#1}}
\newcommand\textitlf[1]{{\NHLight\itshape#1}}
\let\textbflf\textrm
\newcommand\textulf[1]{{\NHLight\bfseries#1}}
\newcommand\textuitlf[1]{{\NHLight\bfseries\itshape#1}}
\usepackage{fancyhdr}
\pagestyle{fancy}
\usepackage{titlesec}
\usepackage{titling}
\makeatletter
\lhead{\textbf{\@title}}
\makeatother
\rhead{\textrmlf{Compiled} \today}
\lfoot{\theauthor\ \textbullet \ \textbf{2021-2022}}
\cfoot{}
\rfoot{\textrmlf{Page} \thepage}
\titleformat{\section} {\Large} {\textrmlf{\thesection} {|}} {0.3em} {\textbf}
\titleformat{\subsection} {\large} {\textrmlf{\thesubsection} {|}} {0.2em} {\textbf}
\titleformat{\subsubsection} {\large} {\textrmlf{\thesubsubsection} {|}} {0.1em} {\textbf}
\setlength{\parskip}{0.45em}
\renewcommand\maketitle{}
\author{Houjun Liu}
\date{\today}
\title{Close Reading, Assessment 1 Synthesis}
\hypersetup{
 pdfauthor={Houjun Liu},
 pdftitle={Close Reading, Assessment 1 Synthesis},
 pdfkeywords={},
 pdfsubject={},
 pdfcreator={Emacs 27.2 (Org mode 9.4.4)}, 
 pdflang={English}}
\begin{document}

\maketitle


\section{Close Reading, Assessment 1}
\label{sec:org7208ab3}
\subsection{General Information}
\label{sec:org6c93187}
\begin{center}
\begin{tabular}{lll}
Due Date & Topic & Important Documents\\
\hline
Sep 8 by 1pm & You will be writing on one of the (non-definitional) texts from our packet: "The I is Never Alone," "The Bird-Dreaming Baobab," or the Elizabeth Bishop "challenge poems" at the end of the packet. & The Bird Dreaming Baobab\\
\end{tabular}
\end{center}

\subsection{Prompt}
\label{sec:org1154e83}
Choose an aspect of the short story or poem that will enable close
reading: a word pattern, a significant image, a literary device, a
repeated detail, etc. In your topic sentence and paragraph, answer the
following question:

\textbf{What does this aspect of the text reveal about the story/poem's broader
themes?}

\begin{itemize}
\item Length: One paragraph, so 250-350 words
\item Paragraph format: MLA style, double-spaced, 12-point font
\item In this paragraph, one of your primary goals is to demonstrate your
close reading skills.*Close reading means unpacking the meaning of
individual words and phrases.*
\item If you write about a pattern/image/etc. that we discussed in class,
you need to go above and beyond class discussion so that I can see
your own thinking. 
\item Include page numbers for citations and a works cited entry for your
selected story/poem at the end of the essay.
\end{itemize}

\subsection{Claim Synthesis}
\label{sec:orgd13d492}
\subsubsection{Development phase -- How and So-What}
\label{sec:org96c1ac1}
\begin{itemize}
\item \emph{The flowers are a symbol of the spirits of (??), and, by proxy, the
birdman}

\begin{itemize}
\item Evidence

\begin{itemize}
\item He even said so
\item His leaving caused it to turn red
\item His dying/disappearing/escaping caused it to turn white
\end{itemize}

\item So what, how is this theamatically important?
\end{itemize}

\item *The Mob-Mentality exhibited by the Portuguese marks a disorder with
the wild spirits --- of which the Baopap flower is an indicator ---
and the birdman is the communicator of that spirit.”

\begin{itemize}
\item Evidence

\begin{itemize}
\item He even said so
\item His leaving caused it to turn red
\item His dying/disappearing/escaping caused it to turn white
\item Crushing the spirit underfeet => harm to kid
\end{itemize}

\item Ok, so what is the CENTRAL LINE OF THEME
\end{itemize}

\item /The symbol of the flower in the BRB alerts the reader of the danger
of the mob-disregard for natural spirits --- the flowers being an
indicator of that spirit/
\end{itemize}

\subsubsection{Defluff}
\label{sec:org68573cb}
/The symbol of the flower in the BRB alerts the reader of the danger of
the mob-disregard for natural spirits --- the flowers being an indicator
of that spirit/

The flower is a symbol for the Natural, Wild spirit that the birdman is
communicating.

\begin{itemize}
\item He even said so
\item His capture brought color change => "The Spirit is in Flux"
\item Crushing the spirit underfeet brought harm to kid
\end{itemize}

Warning the readers of the dangers from mob-mentality + selfishness +
FOMO + caused bypassing of validation of assumptions.

Pause! Problem: how does one concisely describe Mob-Mentality,
Selfishness, and FOMO. => Advocating for own's interests.

*In the Bird-Dreaming Baobadadadbadabap, the author utilises the symbol
of the Baopapba's flower as an indicator of the Wild, Natural spirit
which the Birdman is a communicator of; through tracking the state of
the flowers through the Portugueses' capture of the Birdman, the author
warns the reader against the dangers of selfishly advocating for one's
own interests.*

\begin{itemize}
\item The Birdman literally said that "The Flowers is Where the Spirits
Dwell"
\item During the Birdman's capture, the spirits fell and is in flux =>
taking on the duty of communicating the pains of the communicator
(Birdman) when the communicator is silenced
\item The dying of the child only occurred after the Porchuguese stepped on
the flowers
\end{itemize}
\end{document}
