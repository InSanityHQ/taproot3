% Created 2021-09-11 Sat 16:42
% Intended LaTeX compiler: xelatex
\documentclass[letterpaper]{article}
\usepackage{graphicx}
\usepackage{grffile}
\usepackage{longtable}
\usepackage{wrapfig}
\usepackage{rotating}
\usepackage[normalem]{ulem}
\usepackage{amsmath}
\usepackage{textcomp}
\usepackage{amssymb}
\usepackage{capt-of}
\usepackage{hyperref}
\usepackage[margin=1in]{geometry}
\usepackage{fontspec}
\usepackage{indentfirst}
\setmainfont[ItalicFont = LiberationSans-Italic, BoldFont = LiberationSans-Bold, BoldItalicFont = LiberationSans-BoldItalic]{LiberationSans}
\newfontfamily\NHLight[ItalicFont = LiberationSansNarrow-Italic, BoldFont       = LiberationSansNarrow-Bold, BoldItalicFont = LiberationSansNarrow-BoldItalic]{LiberationSansNarrow}
\newcommand\textrmlf[1]{{\NHLight#1}}
\newcommand\textitlf[1]{{\NHLight\itshape#1}}
\let\textbflf\textrm
\newcommand\textulf[1]{{\NHLight\bfseries#1}}
\newcommand\textuitlf[1]{{\NHLight\bfseries\itshape#1}}
\usepackage{fancyhdr}
\pagestyle{fancy}
\usepackage{titlesec}
\usepackage{titling}
\makeatletter
\lhead{\textbf{\@title}}
\makeatother
\rhead{\textrmlf{Compiled} \today}
\lfoot{\theauthor\ \textbullet \ \textbf{2021-2022}}
\cfoot{}
\rfoot{\textrmlf{Page} \thepage}
\titleformat{\section} {\Large} {\textrmlf{\thesection} {|}} {0.3em} {\textbf}
\titleformat{\subsection} {\large} {\textrmlf{\thesubsection} {|}} {0.2em} {\textbf}
\titleformat{\subsubsection} {\large} {\textrmlf{\thesubsubsection} {|}} {0.1em} {\textbf}
\setlength{\parskip}{0.45em}
\renewcommand\maketitle{}
\author{Houjun Liu}
\date{\today}
\title{The Congo Free State}
\hypersetup{
 pdfauthor={Houjun Liu},
 pdftitle={The Congo Free State},
 pdfkeywords={},
 pdfsubject={},
 pdfcreator={Emacs 27.2 (Org mode 9.4.4)}, 
 pdflang={English}}
\begin{document}

\maketitle


\section{The Congo Free State, some Background}
\label{sec:org23fa424}
\subsection{Leopold's Conolisation}
\label{sec:org735ac11}
\begin{itemize}
\item Belgian emperor
\item Lead first European development efforts into the Congo Basin
\item Privatization of Public State allowed direct control

\begin{itemize}
\item Congo Free State was a privately held state under the seat of the
Belgian monarch, not the seat of the Belgian government
\item Used the guise of Philanthropy and spreading Christian message +
approved by big state governments (US, most European monarchies,
etc.)
\end{itemize}

\item Under Europe's transformation to parliament > monarch, Leopold used
Congo as a place where he could re-establish his absolute authority
\end{itemize}

\subsection{Forced Labour}
\label{sec:orgc2b6984}
"The second holocaust" => Rubber prices increases dramatically, leading
Congo rubber to be very valuable

\begin{itemize}
\item Held up females in villages and forced men to work to free their wives
\item Arm Cutting

\begin{itemize}
\item To show that there was not bullet wasted in the army, the Royal
Guard had to show the severed hand of each dead victim to his senior
officer. If someone missed, the officer would sometimes cut off
someone's hand to compensate.
\item To punish individuals who escaped or did not pour enough rubber,
arms were cut off too
\end{itemize}

\item And so, Birth Rates fell and famine ensued
\end{itemize}

\subsection{Effects of colonization}
\label{sec:org94e8640}
\begin{itemize}
\item Protests about the harsh working conditions
\item Between 10 and 23 million people died during \emph{Leopold's rule, from
1885 to 1908}.
\item Africans lost their right to land ownership anywhere except within
their own villages
\item Colonial administrators also kidnapped orphaned children from
communities and transported them to "child colonies" to work or train
as soldiers. Estimates suggest more than 50\% died there.
\item Villages were burned and leveled to make way for rubber plantations
\item Rubber quotas were set impossibly high, the punishment for not meeting
requirments was to have hands chopped off.
\item Small wars occured where villages attacked neighboring communities to
gather hands in order to escape the rubber quota.
\item Constant starvation, war, and disease led to large declines in
population
\item Millions of congolese were forced to harvest rubber, build railroads
and mine for ore
\end{itemize}

\subsection{Stats!}
\label{sec:org2c3dd6f}
\begin{itemize}
\item Population of the Congo state declined from 20 million => 8 million
after much widespread brutality in labor forces
\item estimates that the forced labor system led to the deaths of 50\% of the
population
\end{itemize}
\end{document}
