% Created 2021-09-12 Sun 22:49
% Intended LaTeX compiler: xelatex
\documentclass[letterpaper]{article}
\usepackage{graphicx}
\usepackage{grffile}
\usepackage{longtable}
\usepackage{wrapfig}
\usepackage{rotating}
\usepackage[normalem]{ulem}
\usepackage{amsmath}
\usepackage{textcomp}
\usepackage{amssymb}
\usepackage{capt-of}
\usepackage{hyperref}
\usepackage[margin=1in]{geometry}
\usepackage{fontspec}
\usepackage{indentfirst}
\setmainfont[ItalicFont = LiberationSans-Italic, BoldFont = LiberationSans-Bold, BoldItalicFont = LiberationSans-BoldItalic]{LiberationSans}
\newfontfamily\NHLight[ItalicFont = LiberationSansNarrow-Italic, BoldFont       = LiberationSansNarrow-Bold, BoldItalicFont = LiberationSansNarrow-BoldItalic]{LiberationSansNarrow}
\newcommand\textrmlf[1]{{\NHLight#1}}
\newcommand\textitlf[1]{{\NHLight\itshape#1}}
\let\textbflf\textrm
\newcommand\textulf[1]{{\NHLight\bfseries#1}}
\newcommand\textuitlf[1]{{\NHLight\bfseries\itshape#1}}
\usepackage{fancyhdr}
\pagestyle{fancy}
\usepackage{titlesec}
\usepackage{titling}
\makeatletter
\lhead{\textbf{\@title}}
\makeatother
\rhead{\textrmlf{Compiled} \today}
\lfoot{\theauthor\ \textbullet \ \textbf{2021-2022}}
\cfoot{}
\rfoot{\textrmlf{Page} \thepage}
\titleformat{\section} {\Large} {\textrmlf{\thesection} {|}} {0.3em} {\textbf}
\titleformat{\subsection} {\large} {\textrmlf{\thesubsection} {|}} {0.2em} {\textbf}
\titleformat{\subsubsection} {\large} {\textrmlf{\thesubsubsection} {|}} {0.1em} {\textbf}
\setlength{\parskip}{0.45em}
\renewcommand\maketitle{}
\author{Houjun Liu}
\date{\today}
\title{Thoughts on Imperialism + Colonialism}
\hypersetup{
 pdfauthor={Houjun Liu},
 pdftitle={Thoughts on Imperialism + Colonialism},
 pdfkeywords={},
 pdfsubject={},
 pdfcreator={Emacs 28.0.50 (Org mode 9.4.4)}, 
 pdflang={English}}
\begin{document}

\maketitle


\section{So, what \emph{is} Colonialism?}
\label{sec:orgac6e78f}
\begin{quote}
Colonialism is a practice of domination
\end{quote}

\begin{itemize}
\item Not very modern

\begin{itemize}
\item Acient Greek Mediterranean Colonies
\item Roman Europe Colonies.+ African Colonies
\item Moors
\item Ottomans
\end{itemize}

\item Time, space irrelevant, but TECHNOLOGY makes it easier
\item Colonialism, a synonym of imperialism?

\begin{itemize}
\item NO!
\item Similarities

\begin{itemize}
\item Both were forms of conquest
\item Both involve mostly European settlers
\end{itemize}

\item Differences

\begin{itemize}
\item Colonialism => Europeans take over + Bring their families

\begin{itemize}
\item Europeans go there
\item They integrate into and change the society
\end{itemize}

\item Imperialism => Europeans take over + Assimilate then leave

\begin{itemize}
\item Europeans go there
\item They setup their rule, and, (potentially) through local
councils, exercise power
\end{itemize}

\item Rapid Imperialism brings forth "Empires", but the U.S., a large
colonist, is a "Republic", for they bring the colonies under their
own care and bring their people there
\end{itemize}

\item Usually, Colonialism is a Consequence of Imperialism, as in\ldots{}

\begin{itemize}
\item People go there and establish their rule ("Imperialists")
\item When society is somewhat assimilated, bring their own people
("Colonists")
\item Profit!
\end{itemize}

\item i p.p1 \{margin: 0.0px 0.0px 0.0px 0.0px; font: 14.0px 'Times New
Roman'; color: \#000000\} span.s1 \{font: 15.0px 'Times New Roman'\}
span.s2 \{font: 12.0px 'Times New Roman'\} span.s3 \{font: 16.0px
'Times New Roman'\} span.s4 \{font: 13.0px 'Times New Roman'\} span.s5
\{font: 17.0px 'Times New Roman'\}
\end{itemize}
\end{itemize}

\begin{quote}
Neither imperialism nor colonialism is a simple act of accumulation
and acquisition. Both are supported and perhaps even impelled by
impressive ideological formations that include notions that certain
territories and people \emph{require} and beseech domination, as well as
forms of knowledge affiliated with domination: the vocabulary of
classic nineteenth-century imperial culture is plentiful with words
and concepts like "inferior" or "subject races," "subordinate
peoples," "dependency," "expansion," and "authority."
\end{quote}
\end{document}
