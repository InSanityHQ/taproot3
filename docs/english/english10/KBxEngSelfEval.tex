% Created 2021-09-11 Sat 16:42
% Intended LaTeX compiler: xelatex
\documentclass[letterpaper]{article}
\usepackage{graphicx}
\usepackage{grffile}
\usepackage{longtable}
\usepackage{wrapfig}
\usepackage{rotating}
\usepackage[normalem]{ulem}
\usepackage{amsmath}
\usepackage{textcomp}
\usepackage{amssymb}
\usepackage{capt-of}
\usepackage{hyperref}
\usepackage[margin=1in]{geometry}
\usepackage{fontspec}
\usepackage{indentfirst}
\setmainfont[ItalicFont = LiberationSans-Italic, BoldFont = LiberationSans-Bold, BoldItalicFont = LiberationSans-BoldItalic]{LiberationSans}
\newfontfamily\NHLight[ItalicFont = LiberationSansNarrow-Italic, BoldFont       = LiberationSansNarrow-Bold, BoldItalicFont = LiberationSansNarrow-BoldItalic]{LiberationSansNarrow}
\newcommand\textrmlf[1]{{\NHLight#1}}
\newcommand\textitlf[1]{{\NHLight\itshape#1}}
\let\textbflf\textrm
\newcommand\textulf[1]{{\NHLight\bfseries#1}}
\newcommand\textuitlf[1]{{\NHLight\bfseries\itshape#1}}
\usepackage{fancyhdr}
\pagestyle{fancy}
\usepackage{titlesec}
\usepackage{titling}
\makeatletter
\lhead{\textbf{\@title}}
\makeatother
\rhead{\textrmlf{Compiled} \today}
\lfoot{\theauthor\ \textbullet \ \textbf{2021-2022}}
\cfoot{}
\rfoot{\textrmlf{Page} \thepage}
\titleformat{\section} {\Large} {\textrmlf{\thesection} {|}} {0.3em} {\textbf}
\titleformat{\subsection} {\large} {\textrmlf{\thesubsection} {|}} {0.2em} {\textbf}
\titleformat{\subsubsection} {\large} {\textrmlf{\thesubsubsection} {|}} {0.1em} {\textbf}
\setlength{\parskip}{0.45em}
\renewcommand\maketitle{}
\author{Huxley}
\date{\today}
\title{English Self Eval}
\hypersetup{
 pdfauthor={Huxley},
 pdftitle={English Self Eval},
 pdfkeywords={},
 pdfsubject={},
 pdfcreator={Emacs 27.2 (Org mode 9.4.4)}, 
 pdflang={English}}
\begin{document}

\maketitle
\#ret

\noindent\rule{\textwidth}{0.5pt}

\begin{verbatim}
Please take your listening log from mid-semester conferences and re-read your reflection for this course. Copy and paste it into the web-based eval and write 150-250 words in response to that about your continued areas of strength and your progress towards your goals.

If students did not write a reflection and/or did not set a goal for their listening log, they were asked to respond to the following questions: 
\end{verbatim}

\begin{itemize}
\item What did you learn about writing this semester? How have you grown as
a writer?

\begin{itemize}
\item I've gained a lot of tacit knowledge about writing; I feel as if
there is less of a barrier between my thoughts and the words on the
(digital) paper.
\end{itemize}

\item Describe your writing process and how it has evolved over the
semester?

\begin{itemize}
\item For most work in English class, I generally start just by thinking
for a while. I create quite extensive outlines, starting with fluid
and undeveloped ideas; through research and more thinking, I move to
an outline describing my idea flow. I gather a "Quote Bin"
containing evidence, and begin writing. Throughout the course of the
semester, the time spent writing has shrank relative to the time
spent outlining.
\end{itemize}

\item Describe the progress you've made in your ability to understand a
text, to identify important moments, to annotate them, and to create
good questions to bring to class?

\begin{itemize}
\item This semester, I've tried to move from close reading to derive
possible meanings to close reading to understand the broader
concepts of the text. I have tried to not close read simply for the
sake of close reading, but rather to use close reading as a tool for
understanding the text at a broader level. This philosophy has
helped me do all the skills listed above.
\end{itemize}

\item What aspect of your performance in English class are you most proud
about this semester (writing process, discussion skills, close
reading, a specific assignment)?

\begin{itemize}
\item When in quarantine this long, it's hard not to just let the days
slip by without really experiencing or engaging in them. While it
sounds mundane, I'm proud of being mentally present to every English
class, and never just "sitting through" them.
\end{itemize}

\item What are your goals in English for next semester? Set at least two
specific goals and describe how you will meet them.

\begin{itemize}
\item I've been getting a lot of value out of group projects, just having
people to bounce ideas off of and learn from. This type of
interaction is much more limited nowadays; next semester, I would
like to engage in more of them.
\item For writing pieces, I often entirely change what I am writing about
multiple times because I'm not happy with the ideas in them. I end
up scrapping a lot of work and a lot of time. Next semester, I would
like to reduce this work and time lost.
\end{itemize}
\end{itemize}
\end{document}
