% Created 2021-09-27 Mon 12:02
% Intended LaTeX compiler: xelatex
\documentclass[letterpaper]{article}
\usepackage{graphicx}
\usepackage{grffile}
\usepackage{longtable}
\usepackage{wrapfig}
\usepackage{rotating}
\usepackage[normalem]{ulem}
\usepackage{amsmath}
\usepackage{textcomp}
\usepackage{amssymb}
\usepackage{capt-of}
\usepackage{hyperref}
\setlength{\parindent}{0pt}
\usepackage[margin=1in]{geometry}
\usepackage{fontspec}
\usepackage{svg}
\usepackage{cancel}
\usepackage{indentfirst}
\setmainfont[ItalicFont = LiberationSans-Italic, BoldFont = LiberationSans-Bold, BoldItalicFont = LiberationSans-BoldItalic]{LiberationSans}
\newfontfamily\NHLight[ItalicFont = LiberationSansNarrow-Italic, BoldFont       = LiberationSansNarrow-Bold, BoldItalicFont = LiberationSansNarrow-BoldItalic]{LiberationSansNarrow}
\newcommand\textrmlf[1]{{\NHLight#1}}
\newcommand\textitlf[1]{{\NHLight\itshape#1}}
\let\textbflf\textrm
\newcommand\textulf[1]{{\NHLight\bfseries#1}}
\newcommand\textuitlf[1]{{\NHLight\bfseries\itshape#1}}
\usepackage{fancyhdr}
\pagestyle{fancy}
\usepackage{titlesec}
\usepackage{titling}
\makeatletter
\lhead{\textbf{\@title}}
\makeatother
\rhead{\textrmlf{Compiled} \today}
\lfoot{\theauthor\ \textbullet \ \textbf{2021-2022}}
\cfoot{}
\rfoot{\textrmlf{Page} \thepage}
\renewcommand{\tableofcontents}{}
\titleformat{\section} {\Large} {\textrmlf{\thesection} {|}} {0.3em} {\textbf}
\titleformat{\subsection} {\large} {\textrmlf{\thesubsection} {|}} {0.2em} {\textbf}
\titleformat{\subsubsection} {\large} {\textrmlf{\thesubsubsection} {|}} {0.1em} {\textbf}
\setlength{\parskip}{0.45em}
\renewcommand\maketitle{}
\author{Huxley}
\date{\today}
\title{Pastiche Planning}
\hypersetup{
 pdfauthor={Huxley},
 pdftitle={Pastiche Planning},
 pdfkeywords={},
 pdfsubject={},
 pdfcreator={Emacs 28.0.50 (Org mode 9.4.4)}, 
 pdflang={English}}
\begin{document}

\tableofcontents

\#ref \#ret

\noindent\rule{\textwidth}{0.5pt}

\section{Prompt}
\label{sec:orgc918671}
\begin{verbatim}
Assignment guidelines

After reading and analyzing Kincaid’s book, you have a better understanding of rhetorical purpose and techniques. Using Kincaid as inspiration, and using at least three of her techniques (parentheses, tone, anaphora, personal pronouns, labyrinthine sentence, em-dashes, etc.), write a rhetorical pastiche on one of the prompts below. 

In addition to your pastiche, please provide a short reflection. You should explain your rhetorical purpose as well as the techniques and elements of style you are trying to imitate (this means you will need to reflect a bit on Kincaid’s techniques, so that you can explain what aspects of her writing you’ve been inspired by). Your reflection is a chance to explain what you want to accomplish, in case it doesn’t fully come through in your piece, as well as to demonstrate your understanding of rhetorical techniques.

Prompts

Conflicts in identity and place: In A Small Place Kincaid describes a conflicted relationship with identity, place, and the past, at times nostalgic, at times combative. Write about some aspect of your life—a part of your identity, a favorite place, your spoken/written languages, a relationship—that evokes contradiction and conflict for you. You may employ an implicit argument in your essay (like Kincaid’s), but you *must* have an explicit argument to articulate in your reflection. 

Responsible travel: Kincaid begins her text with a mockery of “you,” an assumed pasty-faced (read: white), European tourist visiting Antigua. Reflect on your position as a sometimes-tourist, privileged at least to some extent. Possible intended audiences include American friends and family unaware of the complicated power dynamics of traveling in non-Western and/or post-colonial countries, or the “locals” of said country: Antiguans, Peruvians, Costa Ricans, or others.

3. Solutions for issues based in poverty: Critiques of philanthropy often suggest it is motivated by a colonial mindset (“white savior complex,” “white volunteerism”). Yet the fact remains that poverty proliferates and many aid organizations work to alleviate illness, destitution, and inequity. Use the Poverty Action Lab site (https://www.povertyactionlab.org/ (Links to an external site.)) to suggest a solution for some *manageable* issue based in poverty. Note about navigating the site: You may find it useful to start using either “Regions” or “Sectors”; “evaluations” and “publications” motivate issues and provide solutions (policy proposals).
\end{verbatim}

\subsection{pt 1}
\label{sec:org212bfcf}
\begin{verbatim}
Provide details about the following:

1) What prompt have you selected? (There are three options) 

2) What rhetorical devices or stylistic choices are you drawn to from Kincaid's A Small Place? (you need to use three in your pastiche) 
    - Parentheses
    - 1st and 2nd person 
    - Criticism 
    - Sarcasm/Irony 
    - Repetition 
    - Contrast / Juxtaposition 
    - Long sentences (multiple clauses) 
    - Guilt/Shame 
    - Rhetorical Questions

3) To whom will you address your pastiche? 
\end{verbatim}

\subsection{Out. Lining.}
\label{sec:org580c048}
\texttt{const I = 'I'}

Main idea: address to Canadian immigration office

Identify crisis in that hate America but love America?

Crap on America, while saying "But it's my homeee :( (wwoe is me)"

\begin{itemize}
\item the bad.

\begin{itemize}
\item 
\end{itemize}

\item the good?

\item formal immigration document

\begin{itemize}
\item start with "imagine you move to America," second person

\begin{itemize}
\item talk about how truth gets destroyed due to fake news and such
\item talk about the obesity rates, and ect.
\item talk about the land of dreams
\item talk about the innovation, and how to be successful this is the
best place to do it.
\end{itemize}

\item now imagine you are me, and see that nothing's changed.
\item maybe: talk about how in modern times, I can escape the bad and
still have access to the good?
\end{itemize}
\end{itemize}

\subsection{Righting Thyme.}
\label{sec:org99b45c9}
(unedited)

Imagine you lived in America. Where the sun always shines (except when
it doesn't, but that doesn't matter), where the rain is always
refreshing, and where all dreams can come true. You look around and see
America, the land of dreams! --- Of freedom, of innovation, of equality,
ect. You notice the air is filled with hope (it's almost sickly sweet,
or maybe just sickly), and the eyes are filled with phones. You move on,
not thinking too much about it, because how could some phones ruin
(shatter your conception of) the magnificent, great, elitist land of
America? The land of success, where all the tech giants reside ---
Apple, Amazon, Google --- and you reside here to! You can't pass that up
for anything if you want to be successful like them. You watch the
entertainment --- the entertainment here is great! you think to
yourself; the entertainment is reality. Or maybe it isn't. Doesn't
matter. You talk with others in your gated bubble of a community
(inescapable) about the latest crazy thing that someone, somethem, did,
all the while patting yourself on the back for being an intellectual,
for being an informed citizen, for exercising your rights in this
wonderful functioning (functioning?) democracy (not disagreeing, of
course --- that would be horrendous --- only affirming beliefs here. But
are they beliefs?). You become united, defined, under your shared
"beliefs" with those in your bubble, (which you give a name, because the
beliefs themselves are irrelevant), then find connection, maybe tribal
connection, but connection nonetheless in being a part of whichever name
you have given your belief system, but it's all okay (of course),
because you are having conversations, you are reading the news, you
aren't like those them people with their lies and false beliefs, their
stubbornness and misinformation. You watch the news all day, all night,
because you have to be an informed citizen, (popcorn out of frame in
virtuous social media post), constantly being warned of fake news, of
the fake news, of the them news. How do you find this news? Oh, well,
that's simple: does it fall under your ideology? Yes? Great. Send it to
your friends (with an inspirational message). Does it not? Great. Send
it to your friends (with a laughing emoji or deep, sad message about the
state of this country / the news / the society / your mental state. The
news, the supposed reporters of reality, invade your society, your
conversation, your home, your dinner table, your mind. Funny how each
Marvel movie replaces the news for a short period. You watch it with
glee, (or watch it with hate, looking down on those who enjoy this type
of stuff instead of the educated, elitist media you consume) talk about
it with others in your bubble, then, when the novelty fades, you go back
to the news. What did you miss? Nothing; everything (doesn't matter, no
one else was talking about it anyways). You distract from your own
problems (or your own empty) by sending disgust, not hate (but hate),
towards those with a different name to their ideology --- you are
helping the world! Now you can keep wasting time and still feel
fulfilled. Maybe water the plant too, that will get you a bit
fulfillment, a bit more time you can waste without crumbling in your own
un-impactfulness. Sometimes, rarely, you 'converse' with another human
who disagrees with you (you aren't used to this; you don't like this;
this doesn't make you feel good, or included, or right) --- you leave,
writing them off as one of the evil disgusting, not hated (hated) them.
You are so much better! you think to (assure) yourself. You water the
plant again, grab some more popcorn, and open up your preffered media
that aligns itself with your ideology (or one that doesnt, if you want
to get some laughs / feel (project your) sad (ness)). Now imagine you're
me. The America I grew up in is making me a shell, and I won't have it
anymore. That is why I want to, why I must, move to a place of sanity, a
place where others share my idelogy.

\subsubsection{Reflection!}
\label{sec:orgd8401c4}
In my pastiche, I tried to replicate the style of the first section in
Jamaica Kincaid's \emph{A Small Place}. Stylistically, the primary way this
section differs from the rest of the book is through Kincaid's
overwhelming use of accusatory second person point of view. This
technique serves as a sort of transport mechanism for Kincaid's true
messages, usually delivered with sarcasm while being hidden in
parenthesis. I focused on this combination of techniques heavily in my
pastiche. Like Kincaid, I chose to write long and winding sentences to
force the readers to consider the text more deeply, and truly spend time
unpacking the meaning within it. These winding sentences also contribute
to a sense of immersion; while reading \emph{A Small Place}, I felt less as
if I was reading a text as an outsider and analyzing. Along with winding
sentences, I tried to achieve this immersion with my lack of paragraph
breaks and constant flow of text. I also filled my pastiche with
criticizing rhetorical questions, and ironic contradictions and
contrast. I didn't struggle with incorporating these last few techniques
as it seemed to flow naturally with the rest of Kincaid's stylistic
choices.
\end{document}
