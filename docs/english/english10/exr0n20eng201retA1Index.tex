% Created 2021-09-27 Mon 12:02
% Intended LaTeX compiler: xelatex
\documentclass[letterpaper]{article}
\usepackage{graphicx}
\usepackage{grffile}
\usepackage{longtable}
\usepackage{wrapfig}
\usepackage{rotating}
\usepackage[normalem]{ulem}
\usepackage{amsmath}
\usepackage{textcomp}
\usepackage{amssymb}
\usepackage{capt-of}
\usepackage{hyperref}
\setlength{\parindent}{0pt}
\usepackage[margin=1in]{geometry}
\usepackage{fontspec}
\usepackage{svg}
\usepackage{cancel}
\usepackage{indentfirst}
\setmainfont[ItalicFont = LiberationSans-Italic, BoldFont = LiberationSans-Bold, BoldItalicFont = LiberationSans-BoldItalic]{LiberationSans}
\newfontfamily\NHLight[ItalicFont = LiberationSansNarrow-Italic, BoldFont       = LiberationSansNarrow-Bold, BoldItalicFont = LiberationSansNarrow-BoldItalic]{LiberationSansNarrow}
\newcommand\textrmlf[1]{{\NHLight#1}}
\newcommand\textitlf[1]{{\NHLight\itshape#1}}
\let\textbflf\textrm
\newcommand\textulf[1]{{\NHLight\bfseries#1}}
\newcommand\textuitlf[1]{{\NHLight\bfseries\itshape#1}}
\usepackage{fancyhdr}
\pagestyle{fancy}
\usepackage{titlesec}
\usepackage{titling}
\makeatletter
\lhead{\textbf{\@title}}
\makeatother
\rhead{\textrmlf{Compiled} \today}
\lfoot{\theauthor\ \textbullet \ \textbf{2021-2022}}
\cfoot{}
\rfoot{\textrmlf{Page} \thepage}
\renewcommand{\tableofcontents}{}
\titleformat{\section} {\Large} {\textrmlf{\thesection} {|}} {0.3em} {\textbf}
\titleformat{\subsection} {\large} {\textrmlf{\thesubsection} {|}} {0.2em} {\textbf}
\titleformat{\subsubsection} {\large} {\textrmlf{\thesubsubsection} {|}} {0.1em} {\textbf}
\setlength{\parskip}{0.45em}
\renewcommand\maketitle{}
\author{Exr0n}
\date{\today}
\title{Close Reading Paragraph (Assessment 1)}
\hypersetup{
 pdfauthor={Exr0n},
 pdftitle={Close Reading Paragraph (Assessment 1)},
 pdfkeywords={},
 pdfsubject={},
 pdfcreator={Emacs 28.0.50 (Org mode 9.4.4)}, 
 pdflang={English}}
\begin{document}

\tableofcontents



\section{Assignment}
\label{sec:orgbccc92b}
\begin{quote}
\textbf{English 10, Assessment \#1: Close Reading Paragraph} Topic: You will
be writing on one of the short stories from our packet: "The I is
Never Alone" or "The Bird-Dreaming Baobab." (For extra challenge, you
may choose from the Elizabeth Bishop "challenge poems" at the end of
the packet). Choose an aspect of the short story or poem that will
enable close reading: a word pattern, a significant image, a literary
device, a repeated detail, etc. In your topic sentence and paragraph,
answer the following question: What does this aspect of the text
reveal about the story/poem's broader themes? This is a broad prompt,
and I am available to help you narrow it as you begin pre-writing.
This paragraph assesses the following template items:

\begin{itemize}
\item Understanding Literature: Form and Function
\item Close Reading and Argumentation
\item Structure and Mechanics
\item The Writer's Voice
\end{itemize}

\textbf{*} Important reminders:
:CUSTOM\textsubscript{ID}: important-reminders

\begin{itemize}
\item Length: One paragraph, so 250-350 words
\item Paragraph format: MLA style, double-spaced, 12-point font
\item In this paragraph, one of your primary goals is to demonstrate your
close reading skills. Close reading means unpacking the meaning of
individual words and phrases.
\item If you write about a pattern/image/etc. that we discussed in class,
you need to go above and beyond class discussion so that I can see
your own thinking.
\item Include page numbers for citations.
\end{itemize}
\end{quote}

\section{Planning}
\label{sec:org577aad4}
\subsection{Text}
\label{sec:orgbacb214}
I wanted to use the pronouns based on narrorator vs settlers for the
bird seller in bird dreaming baobab, but I'm not sure what that "reveals
about the story/poem's broader themes". So, I will write about the
specificity of numbers in the I is never alone instead.

\subsection{Markup}
\label{sec:org366b3d6}
I realized that I didn't really want to write about the specificity of
numbers, because that felt too obvious after talking about it in class.
I noticed that the word "Siriak" appears in differing intensities
throughout the text, and decided to visualize it with a rolling average:
\url{https://github.com/Exr0nRandomProjects/exr0n20eng201retA1analysis}

\subsection{Thesis}
\label{sec:org01df13d}
The frequency of self reference is directly correlated with idleness,
suggesting that identity and reflection are in opposition.

\subsection{Evidence}
\label{sec:orgd1b2812}
+Trough at beginning of paragraph 5 is when lands on the island,
shipwrecked. Peak halfway through paragraph 6 is when he catches and
teaches the parrots. Trough at paragraph 8 is when "he struggled to
remember that he also existed outside himself"+ < That was the old
graph, with stats before instead of around.

Half way through paragraph 2: Siriak realizes his companions are dead
and takes over the ship Spike at 6: Siriak captures parrots Dip at end
of 7: parrots call for Siriak (although he does nothing himself) Spike
at beginning of 9: "and so the weeks and months flew by"

\section{Outline}
\label{sec:orgc7a70e3}
\subsection{Thesis}
\label{sec:org2522e6f}
The local frequency of references to Siriak

\noindent\rule{\textwidth}{0.5pt}
\end{document}
