% Created 2021-09-11 Sat 16:42
% Intended LaTeX compiler: xelatex
\documentclass[letterpaper]{article}
\usepackage{graphicx}
\usepackage{grffile}
\usepackage{longtable}
\usepackage{wrapfig}
\usepackage{rotating}
\usepackage[normalem]{ulem}
\usepackage{amsmath}
\usepackage{textcomp}
\usepackage{amssymb}
\usepackage{capt-of}
\usepackage{hyperref}
\usepackage[margin=1in]{geometry}
\usepackage{fontspec}
\usepackage{indentfirst}
\setmainfont[ItalicFont = LiberationSans-Italic, BoldFont = LiberationSans-Bold, BoldItalicFont = LiberationSans-BoldItalic]{LiberationSans}
\newfontfamily\NHLight[ItalicFont = LiberationSansNarrow-Italic, BoldFont       = LiberationSansNarrow-Bold, BoldItalicFont = LiberationSansNarrow-BoldItalic]{LiberationSansNarrow}
\newcommand\textrmlf[1]{{\NHLight#1}}
\newcommand\textitlf[1]{{\NHLight\itshape#1}}
\let\textbflf\textrm
\newcommand\textulf[1]{{\NHLight\bfseries#1}}
\newcommand\textuitlf[1]{{\NHLight\bfseries\itshape#1}}
\usepackage{fancyhdr}
\pagestyle{fancy}
\usepackage{titlesec}
\usepackage{titling}
\makeatletter
\lhead{\textbf{\@title}}
\makeatother
\rhead{\textrmlf{Compiled} \today}
\lfoot{\theauthor\ \textbullet \ \textbf{2021-2022}}
\cfoot{}
\rfoot{\textrmlf{Page} \thepage}
\titleformat{\section} {\Large} {\textrmlf{\thesection} {|}} {0.3em} {\textbf}
\titleformat{\subsection} {\large} {\textrmlf{\thesubsection} {|}} {0.2em} {\textbf}
\titleformat{\subsubsection} {\large} {\textrmlf{\thesubsubsection} {|}} {0.1em} {\textbf}
\setlength{\parskip}{0.45em}
\renewcommand\maketitle{}
\author{Dylan Wallace}
\date{\today}
\title{There There Essay Outline}
\hypersetup{
 pdfauthor={Dylan Wallace},
 pdftitle={There There Essay Outline},
 pdfkeywords={},
 pdfsubject={},
 pdfcreator={Emacs 27.2 (Org mode 9.4.4)}, 
 pdflang={English}}
\begin{document}

\maketitle


\section{Prompt}
\label{sec:orgefb65a0}
\begin{verbatim}
Orange theorizes the concept of the “Urban Indian” in his Prologue (11). Define this concept and explain how one of the novel’s characters navigates this modern identity alongside his/her ancestral roots.
\end{verbatim}

\section{Outline}
\label{sec:orgc037169}
\begin{itemize}
\item Thesis (?)

\begin{itemize}
\item *Tony's identity as an Urban Indian is initially characterized by
his embracement of his urban upbringing and his Indian past, but as
the novel progresses, Tony starts exploring the Indian cultural side
of his identity.* (To be changed in the future)
\end{itemize}

\item What is an Urban Indian?

\begin{itemize}
\item "Urbanity" section in Prologue
\item Indians that were forced into an unfamiliar lifestyle and forced to
lose their identity in favor of the status quo identity
\item Attached to their cultural past and their history
\item This is going to be a hard concept to communicate exactly into words
in the essay
\end{itemize}

\item Talk about Tony Loneman and the Drome

\begin{itemize}
\item Face is a consequence of urban Indian life and its many hardships

\begin{itemize}
\item Disconnect from culture and socioeconomic pressure leads many
Urban Indians to alcohol addiction
\end{itemize}

\item Tony literally says that "the Drome" (i.e. Alcohol Poisoning) is "my
mom and why she drank, it's the way history lands on a face, and all
the ways I made it so far despite how it has fucked with me since
the day I found it there on the TV, staring back at me like a
fucking villain."

\begin{itemize}
\item In essence, the Drome is a metaphor (in Tony's, and in our,
minds,) of the burden that is being an Urban Indian.

\begin{itemize}
\item Drome causing Tony to be stupid => Stereotypes about Indians
being stupid???
\end{itemize}

\item However, Tony claims that he has overcame the Drome

\begin{itemize}
\item "all the ways I made it so far despite how it has fucked with me
since the day I found it there on the TV"
\end{itemize}
\end{itemize}
\end{itemize}

\item MF DOOM passage (May Remove)

\begin{itemize}
\item "And it helped because the Drome's what gives me my soul, and the
Drome is a face worn through."

\begin{itemize}
\item Tony has internalized the Drome as part of his identity (i.e. came
to terms with it)
\end{itemize}

\item MF DOOM background

\begin{itemize}
\item MF DOOM had many hardships (brother died, got dropped by his
record label, was homeless for a while)
\end{itemize}
\end{itemize}

\item Tony's disconnect from Indian culture

\begin{itemize}
\item Evident in robbing the Powwow

\begin{itemize}
\item Tony doesn't hesitate when Octavio tells him they will rob the
Powwow
\end{itemize}

\item Tony still feels the Indian culture deep down

\begin{itemize}
\item "I tightened my chin strap. I looked at my face. The Drome. I
didn't see it there. I saw an Indian. I saw a dancer."

\begin{itemize}
\item To Tony (and other Urban Indians), embracing Indian culture
(i.e. regalia) helps with overcoming the Drome (i.e. Urban
Indian burden).
\end{itemize}

\item Tony's actions in last chapter

\begin{itemize}
\item “”
\end{itemize}
\end{itemize}
\end{itemize}
\end{itemize}

\section{Essay}
\label{sec:org0363c80}
Tony is initially apathetic towards Indian culture in favor of the urban
culture he grew up in, and only interacts with his Indian heritage
through contemplating on the troubled past of his people.
\end{document}
