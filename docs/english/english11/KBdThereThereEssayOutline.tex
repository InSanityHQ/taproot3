% Created 2021-09-27 Mon 12:02
% Intended LaTeX compiler: xelatex
\documentclass[letterpaper]{article}
\usepackage{graphicx}
\usepackage{grffile}
\usepackage{longtable}
\usepackage{wrapfig}
\usepackage{rotating}
\usepackage[normalem]{ulem}
\usepackage{amsmath}
\usepackage{textcomp}
\usepackage{amssymb}
\usepackage{capt-of}
\usepackage{hyperref}
\setlength{\parindent}{0pt}
\usepackage[margin=1in]{geometry}
\usepackage{fontspec}
\usepackage{svg}
\usepackage{cancel}
\usepackage{indentfirst}
\setmainfont[ItalicFont = LiberationSans-Italic, BoldFont = LiberationSans-Bold, BoldItalicFont = LiberationSans-BoldItalic]{LiberationSans}
\newfontfamily\NHLight[ItalicFont = LiberationSansNarrow-Italic, BoldFont       = LiberationSansNarrow-Bold, BoldItalicFont = LiberationSansNarrow-BoldItalic]{LiberationSansNarrow}
\newcommand\textrmlf[1]{{\NHLight#1}}
\newcommand\textitlf[1]{{\NHLight\itshape#1}}
\let\textbflf\textrm
\newcommand\textulf[1]{{\NHLight\bfseries#1}}
\newcommand\textuitlf[1]{{\NHLight\bfseries\itshape#1}}
\usepackage{fancyhdr}
\pagestyle{fancy}
\usepackage{titlesec}
\usepackage{titling}
\makeatletter
\lhead{\textbf{\@title}}
\makeatother
\rhead{\textrmlf{Compiled} \today}
\lfoot{\theauthor\ \textbullet \ \textbf{2021-2022}}
\cfoot{}
\rfoot{\textrmlf{Page} \thepage}
\renewcommand{\tableofcontents}{}
\titleformat{\section} {\Large} {\textrmlf{\thesection} {|}} {0.3em} {\textbf}
\titleformat{\subsection} {\large} {\textrmlf{\thesubsection} {|}} {0.2em} {\textbf}
\titleformat{\subsubsection} {\large} {\textrmlf{\thesubsubsection} {|}} {0.1em} {\textbf}
\setlength{\parskip}{0.45em}
\renewcommand\maketitle{}
\author{Dylan Wallace}
\date{\today}
\title{There There Essay Outline}
\hypersetup{
 pdfauthor={Dylan Wallace},
 pdftitle={There There Essay Outline},
 pdfkeywords={},
 pdfsubject={},
 pdfcreator={Emacs 28.0.50 (Org mode 9.4.4)}, 
 pdflang={English}}
\begin{document}

\tableofcontents



\section{Prompt}
\label{sec:org4caa0c5}
\begin{verbatim}
Orange theorizes the concept of the “Urban Indian” in his Prologue (11). Define this concept and explain how one of the novel’s characters navigates this modern identity alongside his/her ancestral roots.
\end{verbatim}

\section{Outline}
\label{sec:org8c303f8}
\begin{itemize}
\item Thesis (?)

\begin{itemize}
\item *Tony's identity as an Urban Indian is initially characterized by
his embracement of his urban upbringing and his Indian past, but as
the novel progresses, Tony starts exploring the Indian cultural side
of his identity.* (To be changed in the future)
\end{itemize}

\item What is an Urban Indian?

\begin{itemize}
\item "Urbanity" section in Prologue
\item Indians that were forced into an unfamiliar lifestyle and forced to
lose their identity in favor of the status quo identity
\item Attached to their cultural past and their history
\item This is going to be a hard concept to communicate exactly into words
in the essay
\end{itemize}

\item Talk about Tony Loneman and the Drome

\begin{itemize}
\item Face is a consequence of urban Indian life and its many hardships

\begin{itemize}
\item Disconnect from culture and socioeconomic pressure leads many
Urban Indians to alcohol addiction
\end{itemize}

\item Tony literally says that "the Drome" (i.e. Alcohol Poisoning) is "my
mom and why she drank, it's the way history lands on a face, and all
the ways I made it so far despite how it has fucked with me since
the day I found it there on the TV, staring back at me like a
fucking villain."

\begin{itemize}
\item In essence, the Drome is a metaphor (in Tony's, and in our,
minds,) of the burden that is being an Urban Indian.

\begin{itemize}
\item Drome causing Tony to be stupid => Stereotypes about Indians
being stupid???
\end{itemize}

\item However, Tony claims that he has overcame the Drome

\begin{itemize}
\item "all the ways I made it so far despite how it has fucked with me
since the day I found it there on the TV"
\end{itemize}
\end{itemize}
\end{itemize}

\item MF DOOM passage (May Remove)

\begin{itemize}
\item "And it helped because the Drome's what gives me my soul, and the
Drome is a face worn through."

\begin{itemize}
\item Tony has internalized the Drome as part of his identity (i.e. came
to terms with it)
\end{itemize}

\item MF DOOM background

\begin{itemize}
\item MF DOOM had many hardships (brother died, got dropped by his
record label, was homeless for a while)
\end{itemize}
\end{itemize}

\item Tony's disconnect from Indian culture

\begin{itemize}
\item Evident in robbing the Powwow

\begin{itemize}
\item Tony doesn't hesitate when Octavio tells him they will rob the
Powwow
\end{itemize}

\item Tony still feels the Indian culture deep down

\begin{itemize}
\item "I tightened my chin strap. I looked at my face. The Drome. I
didn't see it there. I saw an Indian. I saw a dancer."

\begin{itemize}
\item To Tony (and other Urban Indians), embracing Indian culture
(i.e. regalia) helps with overcoming the Drome (i.e. Urban
Indian burden).
\end{itemize}

\item Tony's actions in last chapter

\begin{itemize}
\item “”
\end{itemize}
\end{itemize}
\end{itemize}
\end{itemize}

\section{Essay (1st draft)}
\label{sec:orgea03d0e}
\begin{verbatim}
[Hook + talk about Urban Indian]
Tony's identity as an Urban Indian is initially characterized by his embracement of his urban upbringing and his Indian past, but as the novel progresses, Tony starts exploring the Indian cultural side of his identity. This shift takes place mainly at a key point 
It is clear that initially Tony is cynical in regards to his Native American past. [Talk about Drome and Tony's view of it as metaphor for his Native American ancestory/history. Talk about how Tony views the Drome (i.e. his Native American ancestory) as being an obstacle that he had to overcome.][Elaborate.]
However, Tony's view of his ancestory shifts during [whenever it shifts.][Elaborate.]
This shift is significant because it represents . The shift is mirrored in several other main characters in There There, such as Orvil. The shift reflects the core message of the novel, being ___.
\end{verbatim}

\section{Essay (2nd draft)}
\label{sec:org5ba2193}
\begin{verbatim}
In his breakout novel There There, Orange tackles the question of what it means to be Native American in modern times while exploring the many problems that are present in urban Native American communities. This is done primarily through exploring the modern Native American experience of 10 Native Americans centered around the Oakland Powwow. In particular, Orange focuses on Tony, a young man who grapples with his identity as an Urban Indian. Tony's exploration of his Urban Indian identity is characterized by a conscious rejection of his Indian heritage, which is contrasted by an subconscious embracement of his ancestory.
The concept of the "Urban Indian" is a recurring and central theme throughout There There. Orange defines Urban Indians as being "the generation [of Native Americans] born in the city" (11), the descendents of the Native Americans who relocated to cities because of the Indian Relocation Act. Urban Indians grow up in the city, surrounded by the city, and as such, "belong to the city" (11) rather than to the reservations their ancestors came from. As a consequence of being formed as a Native American identity formed in cities, the Urban Indian identity is closely related to both Native American identity and urban identity, but it is not accurate to condense the Urban Indian identity as merely a synthesis of those two identities; Much like how African American identity is reflective of the many  hardships African Americans faced, Urban Indian identity is also tied heavily to the unique hardships Native Americans faced while navigating their urban environments.
No character's identity reflects this aspect of Urban Indian identity more than that of Tony. Tony's mother suffered with alcoholism, a problem widespread in urban Indian communities. As a result of her alcoholism, Tony was born with Fetal Alcohol Syndrome, which he refers to as the "Drome" (15). Although Tony credits the Drome for his street smarts, he also admits that the Drome also created many hardships for him, from making him "fail...the intelligence test" (16) to making him "look...like a monster" (19).
Tony's identity as an Urban Indian is closely tied to his perception of the Drome. Tony claims that "[t]he Drome is my mom and why she drank, it's the way history lands on a face, and all the ways I made it so far despite how it has fucked with me" (16). This passage is insightful in analyzing Tony's perception of his Urban Indian identity, in that it indicates that Tony views the Drome not only as a physical hardship that he has had to overcome over the years, but also as a metaphor for the burden of his ancestors' history that he has had to carry as an Urban Indian. When Tony claims that "people look at [him] and then look away when they see [him] see them see [him]" (16), he means that people look away both because of his appearance and because of the many negative stereotypes of Urban Indians. By viewing the "Drome" as a metaphor, it becomes clear that Tony views the Native American half of his Urban Indian identity as a "curse" (16). 
Despite his rejection of his Native American identity, Tony still displays signs of embracing that side of his Urban Indian identity subconsciously. We get a glimpse of this at the end of the first chapter, when Tony wears his Native American regelia in preparation to the robbery: "I looked at my face. The Drome. I didn't see it there. I saw an Indian." (26) Tony not seeing the "Drome" indicates that the action of putting the regelia on allowed Tony to temporarily ignore the negative aspects of his Native American identity and embrace his identity as an Urban Indian. 
As the main character of the first and last chapter, Tony's exploration of his identity as an Urban Indian is central to There There. Through his depiction of Tony's exploration of his own identity, Orange is able to communicate that there is a bit of Native American identity within every Urban Indian, no matter how much it is rejected.
\end{verbatim}
\end{document}
