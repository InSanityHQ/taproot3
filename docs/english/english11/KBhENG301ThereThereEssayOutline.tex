% Created 2021-09-12 Sun 22:49
% Intended LaTeX compiler: xelatex
\documentclass[letterpaper]{article}
\usepackage{graphicx}
\usepackage{grffile}
\usepackage{longtable}
\usepackage{wrapfig}
\usepackage{rotating}
\usepackage[normalem]{ulem}
\usepackage{amsmath}
\usepackage{textcomp}
\usepackage{amssymb}
\usepackage{capt-of}
\usepackage{hyperref}
\usepackage[margin=1in]{geometry}
\usepackage{fontspec}
\usepackage{indentfirst}
\setmainfont[ItalicFont = LiberationSans-Italic, BoldFont = LiberationSans-Bold, BoldItalicFont = LiberationSans-BoldItalic]{LiberationSans}
\newfontfamily\NHLight[ItalicFont = LiberationSansNarrow-Italic, BoldFont       = LiberationSansNarrow-Bold, BoldItalicFont = LiberationSansNarrow-BoldItalic]{LiberationSansNarrow}
\newcommand\textrmlf[1]{{\NHLight#1}}
\newcommand\textitlf[1]{{\NHLight\itshape#1}}
\let\textbflf\textrm
\newcommand\textulf[1]{{\NHLight\bfseries#1}}
\newcommand\textuitlf[1]{{\NHLight\bfseries\itshape#1}}
\usepackage{fancyhdr}
\pagestyle{fancy}
\usepackage{titlesec}
\usepackage{titling}
\makeatletter
\lhead{\textbf{\@title}}
\makeatother
\rhead{\textrmlf{Compiled} \today}
\lfoot{\theauthor\ \textbullet \ \textbf{2021-2022}}
\cfoot{}
\rfoot{\textrmlf{Page} \thepage}
\titleformat{\section} {\Large} {\textrmlf{\thesection} {|}} {0.3em} {\textbf}
\titleformat{\subsection} {\large} {\textrmlf{\thesubsection} {|}} {0.2em} {\textbf}
\titleformat{\subsubsection} {\large} {\textrmlf{\thesubsubsection} {|}} {0.1em} {\textbf}
\setlength{\parskip}{0.45em}
\renewcommand\maketitle{}
\author{Houjun Liu}
\date{\today}
\title{There There Essay Outline}
\hypersetup{
 pdfauthor={Houjun Liu},
 pdftitle={There There Essay Outline},
 pdfkeywords={},
 pdfsubject={},
 pdfcreator={Emacs 28.0.50 (Org mode 9.4.4)}, 
 pdflang={English}}
\begin{document}

\maketitle
\index{english!there there!There There Essay Outline}

\section{General Information}
\label{sec:org2b49bec}
\begin{center}
\begin{tabular}{lll}
Due Date & Topic & Important Documents\\
\hline
9/20 & Themes Prevalent in There There & There There by Tommy Orange\\
\end{tabular}
\end{center}

\section{Prompt}
\label{sec:org3f730fe}
One of\ldots{}

\begin{quote}
In his Prologue, Orange claims that "we are the memories we don’t remember, which live in us, which we feel" (10). Choose one character from the novel, and explain how he/she embodies, revises, or refutes this claim. Be sure to explain/define the quotation and how it connects to the character you are discussing.
\end{quote}

\begin{quote}
Choose one character from the novel and discuss why his or her narrative is 1st, 2nd, or 3rd person. What effect does the form of narrative have on how we read, understand, know the character?
\end{quote}

\section{Quotes Bin}
\label{sec:org69a4703}
\subsection{Raw Quotes}
\label{sec:org437a625}
\begin{itemize}
\item "For how many years had there been federally funded programs trying to prevent suicide with billboards and hotlines? It was no wonder it was getting worse. You can't sell life is OK when its not." (98) AA
\item "Jacquie's last relapse had not left burn holes in her life\ldots{}She was sober again, and ten days is the same as a year when you want to drink all the time." (99) AB
\item "But home for Jaquie and her sister was a locked station wagon in an empty parking lot. Home was a long ride on a bus\ldots{}.Home was the three of them anywhere safe for the night." (99) AC
\item "The night air was cool but didn't move" (100) AD
\item "She didn’t know how to swim. Mostly she just wanted to be in the water \ldots{} Mostly she just wanted to be in the water. To go under and open her eyes, look at her hands, watch the bubbles rise in that bluest light." (100) AF
\item "In this case Jaquie was the spider, and the minifridge was the web. Home was to drink. To drink was the trap." (101) AG
\item "She pressed her eyes into her knees and bursts of purple, black, green, and pink splotches bloomed there, behind her eyes, then slowly formed into images, then memories. She saw the big hole first. Then her daughter's emaciated body." (105) AJ
\item "The trister spider, Veho, her mom used to used to tell her and Opah about, he was always stealing eyes to see better \ldots{} the white man who came and made the old world watch with his eyes \ldots{} Until your eyes are drained and you can't see behind you and there's nothnig ahead, and the needle, the bottle, or the pipe is the only thing in sight that makes any sense." (106) AK
\item "She put a cough drop into her mouth so casually that you could tell she probably ate a lot of cough drops and smoked a lot of cigaretts, and never quite beat the caugh, but beat it enough while she was sucking on a cough drop, so ate them constantly" (109) AM
\item "There's an ache when you keep yourself from breathing. A relief when you come up for air. It was the same when you drank after telling yourself you wouldn't. Both broke at a point. Both gave and took." (116) AN
\item "Drinking had never been fun. It was kind of a solemn duty. It took the edge off, and it allowed her to say and do whatever she wanted without feeling bad about it." (152) AO
\item "She should watch the sunrise. How long had it been since she'd done that? Instead she closes the curtains and turns on the TV" (254) AR
\item "If she'd ever found spider legs in her leg, she probably would have ended it right there and then. She sudden feels so overwhelmed by all of it that she gets tired \ldots{} She feels grateful when it does, because most of the time her thoughts keep her up." (154) AP
\end{itemize}

\subsection{Quote Development}
\label{sec:org1926574}

\subsubsection{Society's drugs dulls pain (No I)}
\label{sec:orgbd7fe48}
\begin{itemize}
\item AO => drinking as something dulling and "solemn"
\end{itemize}

\begin{itemize}
\item AK => the white men draining the eyes (memories?) of natives to include only drugs or alcahol
\item AP => Thankful for spent indifference
\end{itemize}

\subsubsection{Pressures of society holds Jaquie down (No second person)}
\label{sec:org62df44e}
\begin{itemize}
\item AD => pain as a pressure manifest in the atmosphere holding Jaquie down, nonchalantly (cool, chill, invisible. Also it's freaking air)
\item AF => unlike the atmosphere, Jaquie want to see --- with her eyes --- the moving water (bubles rise) but Jaquie can't swim
\item AN => air as something addictive, that provides temporary relief that also takes something from you
\end{itemize}

\subsubsection{Wastepaper Basket}
\label{sec:org47d095c}
\begin{itemize}
\item AM => Jaqueline's actions reveal about her smoking habits, and it manifests visually through cough
\item AA => the system rejects the pain that drive one to suicide
\item AC => lack of static location of belonging and identity based on movement
\item AG => home is a trap, belonging is a trap
\item AR => systemic entertainment by the society (TV) favored by Jaquie over natural beauty
\item AJ => dark memories as something her body (knee) forced into her mind (eyes)
\end{itemize}



\section{Claim Synthesis}
\label{sec:org73014d5}

\begin{center}
\begin{tabular}{ll}
Symbol & Representation\\
\hline
Alcohol & Disenfranchisement\\
Eyes & Memory/Knowledge\\
Body & Heritage\\
Home & Tradition\\
Water & Freedom\\
Air & Restraint\\
\end{tabular}
\end{center}

\textbf{In Tommy Orange's novel \emph{There There}, the author's use of third-person perspective of Jacquie's character externalizes the systems in society which strip her of both a sense of identity and capacity to critique: creating a multifaceted grappling of the forces of oppression that plagues upon Native Americans in modern society.}
\end{document}
