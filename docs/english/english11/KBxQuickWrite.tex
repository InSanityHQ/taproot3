% Created 2021-09-11 Sat 16:42
% Intended LaTeX compiler: xelatex
\documentclass[letterpaper]{article}
\usepackage{graphicx}
\usepackage{grffile}
\usepackage{longtable}
\usepackage{wrapfig}
\usepackage{rotating}
\usepackage[normalem]{ulem}
\usepackage{amsmath}
\usepackage{textcomp}
\usepackage{amssymb}
\usepackage{capt-of}
\usepackage{hyperref}
\usepackage[margin=1in]{geometry}
\usepackage{fontspec}
\usepackage{indentfirst}
\setmainfont[ItalicFont = LiberationSans-Italic, BoldFont = LiberationSans-Bold, BoldItalicFont = LiberationSans-BoldItalic]{LiberationSans}
\newfontfamily\NHLight[ItalicFont = LiberationSansNarrow-Italic, BoldFont       = LiberationSansNarrow-Bold, BoldItalicFont = LiberationSansNarrow-BoldItalic]{LiberationSansNarrow}
\newcommand\textrmlf[1]{{\NHLight#1}}
\newcommand\textitlf[1]{{\NHLight\itshape#1}}
\let\textbflf\textrm
\newcommand\textulf[1]{{\NHLight\bfseries#1}}
\newcommand\textuitlf[1]{{\NHLight\bfseries\itshape#1}}
\usepackage{fancyhdr}
\pagestyle{fancy}
\usepackage{titlesec}
\usepackage{titling}
\makeatletter
\lhead{\textbf{\@title}}
\makeatother
\rhead{\textrmlf{Compiled} \today}
\lfoot{\theauthor\ \textbullet \ \textbf{2021-2022}}
\cfoot{}
\rfoot{\textrmlf{Page} \thepage}
\titleformat{\section} {\Large} {\textrmlf{\thesection} {|}} {0.3em} {\textbf}
\titleformat{\subsection} {\large} {\textrmlf{\thesubsection} {|}} {0.2em} {\textbf}
\titleformat{\subsubsection} {\large} {\textrmlf{\thesubsubsection} {|}} {0.1em} {\textbf}
\setlength{\parskip}{0.45em}
\renewcommand\maketitle{}
\author{Huxley Marvit}
\date{\today}
\title{Quick Write!}
\hypersetup{
 pdfauthor={Huxley Marvit},
 pdftitle={Quick Write!},
 pdfkeywords={},
 pdfsubject={},
 pdfcreator={Emacs 27.2 (Org mode 9.4.4)}, 
 pdflang={English}}
\begin{document}

\maketitle
\#flo \#ret \#disorganized \#inclass

\noindent\rule{\textwidth}{0.5pt}

\section{qw1}
\label{sec:org7db77c1}
Is it about beacon of hope even in dark times? As in, there is always
hope? Or is it darker, talking about how joyful activities are
disconnected from the events of the real world? Singing as escapism, and
joy as falsehood. Relates to There There\ldots{} the meetup was was the
celebration? Singing as community, or singing as communication, or
singing as art? Hope never dies, or hope never matters? Or hope is
crushed like in There There. Good example being Edwin if I remember
correctly. Singing as joyful, and dark times as the opposite, emotions
or states can exist in parallel? Talking about cultures existing in
parallel despite not matching there surroundings? \textbf{ABOUT} the dark times
-- is singing inherently joyful? Is joy or art itself corrupted by the
darkness? Is singing the way to deal with the dark times? "Also"
signing, so signing doesn't fit into the dark times. Thus, darkness can
be contained? About dark times yet not dark times.
\end{document}
