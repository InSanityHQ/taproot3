% Created 2021-09-20 Mon 22:20
% Intended LaTeX compiler: xelatex
\documentclass[letterpaper]{article}
\usepackage{graphicx}
\usepackage{grffile}
\usepackage{longtable}
\usepackage{wrapfig}
\usepackage{rotating}
\usepackage[normalem]{ulem}
\usepackage{amsmath}
\usepackage{textcomp}
\usepackage{amssymb}
\usepackage{capt-of}
\usepackage{hyperref}
\setlength{\parindent}{0pt}
\usepackage[margin=1in]{geometry}
\usepackage{fontspec}
\usepackage{svg}
\usepackage{cancel}
\usepackage{indentfirst}
\setmainfont[ItalicFont = LiberationSans-Italic, BoldFont = LiberationSans-Bold, BoldItalicFont = LiberationSans-BoldItalic]{LiberationSans}
\newfontfamily\NHLight[ItalicFont = LiberationSansNarrow-Italic, BoldFont       = LiberationSansNarrow-Bold, BoldItalicFont = LiberationSansNarrow-BoldItalic]{LiberationSansNarrow}
\newcommand\textrmlf[1]{{\NHLight#1}}
\newcommand\textitlf[1]{{\NHLight\itshape#1}}
\let\textbflf\textrm
\newcommand\textulf[1]{{\NHLight\bfseries#1}}
\newcommand\textuitlf[1]{{\NHLight\bfseries\itshape#1}}
\usepackage{fancyhdr}
\pagestyle{fancy}
\usepackage{titlesec}
\usepackage{titling}
\makeatletter
\lhead{\textbf{\@title}}
\makeatother
\rhead{\textrmlf{Compiled} \today}
\lfoot{\theauthor\ \textbullet \ \textbf{2021-2022}}
\cfoot{}
\rfoot{\textrmlf{Page} \thepage}
\titleformat{\section} {\Large} {\textrmlf{\thesection} {|}} {0.3em} {\textbf}
\titleformat{\subsection} {\large} {\textrmlf{\thesubsection} {|}} {0.2em} {\textbf}
\titleformat{\subsubsection} {\large} {\textrmlf{\thesubsubsection} {|}} {0.1em} {\textbf}
\setlength{\parskip}{0.45em}
\renewcommand\maketitle{}
\author{Huxley Marvit}
\date{\today}
\title{There There Essay Planning}
\hypersetup{
 pdfauthor={Huxley Marvit},
 pdftitle={There There Essay Planning},
 pdfkeywords={},
 pdfsubject={},
 pdfcreator={Emacs 28.0.50 (Org mode 9.4.4)}, 
 pdflang={English}}
\begin{document}

\maketitle
\#ret

\noindent\rule{\textwidth}{0.5pt}

\section{Ready? Go.}
\label{sec:orgb018c38}
\begin{verbatim}
1.  **Prompt**: In his Prologue, Orange claims that “we are the memories we don’t remember, which live in us, which we feel” (10). Choose one character from the novel, and explain how he/she embodies, revises, or refutes this claim. Be sure to explain/define the quotation and how it connects to the character you are discussing.
\end{verbatim}

rough thinking:
\href{KBxThereTherePromptsAndQuestions.org}{KBxThereTherePromptsAndQuestions}

\subsubsection{Idea}
\label{sec:orgc0d817d}
\begin{itemize}
\item rough: we are not the memories we don't remember, we live in the
memories we don't remember

\begin{itemize}
\item seems pedantic, but this small nuance has much broader implications

\begin{itemize}
\item you can reject reality (denial)

\begin{itemize}
\item even if you don't know how it got there, you know what it is and
can avoid it
\end{itemize}

\item you can change it (hope)
\item also, what was done to the ancestors (breadth)

\begin{itemize}
\item not just "we are what our ancestors did," you are what was done
to your ancestors
\end{itemize}
\end{itemize}
\end{itemize}
\end{itemize}

? "the sense that everything didnt come out"

\begin{enumerate}
\item mapping
\label{sec:org4bfb4d3}
edwin: - denial - internet obsession - second life - ? bill is both self
aware and in denial - constipation represents how when in denial, he is
stuck? cannot move forward?

\begin{itemize}
\item hope

\begin{itemize}
\item getting better, working on the powwow
\end{itemize}

\item breadth

\begin{itemize}
\item shooting? reflects the past, points out that it's now about just
about what you do, it's about what is done to you.
\end{itemize}
\end{itemize}

edwins arc maps to the revised point: "We live in the memories we don't
remember"
\end{enumerate}

\subsubsection{qoute bin}
\label{sec:orgdb6dfc6}
\begin{itemize}
\item \textbf{main}

\begin{itemize}
\item "But what we are is what our ancestors did. How they survived. We
are the memories we don't remember, which live in us, which we feel,
which make us sing and dance and pray the way we do, feelings from
memories that flare and bloom unexpectedly in our lives like blood
through a blanket from a wound made by a bullet fired by a man
shooting us in the back for our hair, for our hair, for our heads,
for a bounty, or just to get rid of us."
\end{itemize}

\item \textbf{denial}

\begin{itemize}
\item "but I dream of the internet"
\item "I was really into \emph{Second Life} for a while. I think I logged two
whole years there. And as I was growing, getting fatter in real
life, the Edwin Black I had in there, on there, I made him thinner,
and as I did less, he did more."
\item "The Edwin Black in there had a job and a girlfriend and his mom had
died tragically during childbirth. That Edwin Black was rained on
the reservation with his dad. The Edwin Black of my \emph{Second Life}
was proud. He had hope."
\item "When I moved back in with my mom, the door to my old room. to my
old life in that room, it opened up like a mouth and swallowed me."
\item "I read a lot and come away with nothing. This is how time skips."
\item "Remembering itself is becoming old-fashioned"
\item "And it reminds me how removed I am because of her."
\item "'Well, that's a pretty convenient theory for someone who spends
twenty hours a day leaning into their computer like ther're waiting
for a kiss,'"
\item 
\end{itemize}

\item \textbf{hope}

\begin{itemize}
\item "The trouble with believing is you have to believe that believing
will work, you have to believe in belief."
\item "I have to give up."
\item "I feel something not unlike hope"

\begin{itemize}
\item C: after connecting with dad, thinking about internship, life
starts moving.
\end{itemize}

\item "'At least he's got a job now. He's working. Every day. That's a
lot. For him. Please. I don't want to discourage him.'"
\item "Today means everything for them. The countless hours they out in.
All the different drum groups and vendors and dancers they had to
call and convince to come, that there was prize money to be had,
money to be made."
\item "But this means more than a job for Edwin at this point. This is a
new life."
\end{itemize}

\item \textbf{breadth}

\begin{itemize}
\item "'Whatever bro, my record keepers have it going down differently'"
\item "'That's their culture'"
\item "There's a gravity to it. A weight pulling him closer to Octavio,
who's now pointing his gun at Edwin and Blue. He's pointing at the
sage with the gun. He's calm about it. Calvin has his hand on his
gun through his shirt. Edwin crouches down to open the sage."
\item Shooting descriptors..
\item "When Blue pulls into Highland, Edwin is passed out. She'd been
telling him, yelling at him, screaming at him to stay awake. There
was probably a closer hospital, but she knew Highland. She keeps her
hand on the horn, to try to wake Edwin up and to get someone to come
out to help. She reaches her hand over and slaps Edwin a few times
on the cheek. Edwin shakes his head a little. “You gotta wake up,
Ed," Blue says. "We're here." He doesn't respond.”
\end{itemize}
\end{itemize}

\subsubsection{Outline}
\label{sec:orgd9429b2}
\begin{enumerate}
\item Intro:
\label{sec:orgc110d2e}
Not, we are the memories we don't remember, but we live in the memories
we don't remember. Ancestors -> ancestor experiences -> ancestor actions
-> shape world -> shape us ancestor experiences \& actions = memories we
don't remember in my model While it sounds pedantic, it has much broader
implications.

\item Denial
\label{sec:orgb27c1a7}
One cannot know what part of them stems from the memories we don't
remember. Orange talks about these parts being the culture, but that can
be rejected. One can't deny the memories because they are intangible,
but one can deny reality. Updated model allows for choice.

\item Hope
\label{sec:orgd006ded}
One can't change what happened in the past. But, one can change the
world, can change where they live. We are not \emph{just} the memories we
don't remember. We are also the memories we do remember!

\item Breadth
\label{sec:orge58c303}
Orange refers specifically to "what our ancestors did," but it's more
than that -- it's what was done to the ancestors. Everyone's memories
shaped the present. We can't forget or ignore the atrocities and their
impacts.

\item Conclusion
\label{sec:org9b317a1}
more powerful model? allows for choice? \#review

In his Prologue, Orange claims that "we are the memories we don't
remember, which live in us, which we feel" (10). \#\# Begin.

In the Prologue of \emph{There There} Tommy Orange writes "we are the
memories we don't remember" (10). He connects the past to the present by
claiming that what defines our identities are our ancestors' lived
experiences -- the "memories we don't remember." While these memories
are not ours, they are still the ones "which live in us, which we feel."
Our ancestors, and their experiences, live on in us just as they define
us. While Orange's claim is very insightful, a minor and seemingly
pedantic revision gives it \{much more power\} as is illustrated by Edwin
Black. The memories we don't remember do not directly shape us; instead,
they shape the world, which in turn shapes us. The state of the world is
the conduit through which our ancestors' experiences flow into our
identities. Instead of "we are the memories we don't remember," "we live
in the memories we don't remember." \%\%While this may seem \{like\} a
pedantic deconstruction, highlighting this difference \%\%

As demonstrated by Edwin Black, the updated claim allows for denial
while Orange's claim does not. The effect of our ancestors experiences
on us, while "we feel" them, are ineffable. We cannot describe or even
know what parts of us stem from the memories we don't remember, and
therefore we cannot deny them. And yet, Edwin does. At the start of his
arc he lives on the internet, he "dreams[s] of the internet" (cite). He
describes his time playing \emph{Second Life}, highlighting the disparity
between a false reality and the one he chooses to deny: "as I was
growing, getting fatter in real life, the Edwin Black I had in there, on
there, I made him thinner, and as I did less, he did more" (cite). Edwin
switches from "in there" to "on there" to represent the control his
character has over the false reality, something he believes he lacks in
the true one. He goes on to describe the hope that the false Edwin has,
the job and girlfriend and dad and proudness he has (cite). Edwin is
able to deny his ineffable identity by denying his effable reality, not
even getting to his ancestors memories. He chooses not to live in the
memories we don't remember and thus deny his identity, something not
possible within the bounds of Orange's original claim.

The past is set in stone -- unable to be changed -- but reality is not;
from the ability to mold reality and ourselves we get hope. Edwin begins
his story talking about belief. He explains that in order for believing
to work, "you have to believe in belief" itself (cite). One can't change
the past, but one can change the world, can change where they live. If
we are the past, "the memories we don't remember," then how can we
believe? Edwin starts off his story not believing: "I have to give up"
he narrates (cite). But once the reality Edwin lives in starts to
change, so does his belief. He connects with his dad, he gets work, and
he moves into the second phase of his arc. His life starts moving. He
ends his first chapter declaring "I feel something not unlike hope"
(cite). Orange's original claim doesn't allow us to believe in belief,
as we are the past. But we are not only the memories we don't remember,
we are also the memories we do remember. Edwin's reality changes, and he
changes his reality, just like his ancestors. That's why we can believe.
Because we live in the memories we don't remember instead of being the
memories we don't remember, we can have hope for something better by
being able to change reality instead of having to change the past.

The third and final phase of Edwin's arc demonstrates the third limit of
Orange's original claim: breadth. Edwin's job helping with the powwow
has moved him out of denial and into hope. The work is described as
meaning "more than a job for Edwin at this point. This is a new life"
(cite). Edwin is now changing the place he lives in -- and changing
himself -- creating a new life. He is exerting his control over the
world and trying his best to make the powwow go well, but of course, it
doesn't. Instead, Octavio points "his gun at Edwin and Blue" and Edwin
gets "shot--in the stomach" (cite). Directly prior to his claim about
memories, Orange refers specifically to Native Americans and writes
"what we are is what our ancestors did" (cite). But Orange ignores what
was done to them. Edwin's experience at the powwow mirrors the
experiences of the Native Americans suffering from the genocidal
atrocities committed by the early colonists -- culture and hope being
shattered by violence. When Edwin gets shot, it is not solely Edwin's
actions which shape his identity, but his shooters. We cannot forget or
ignore the atrocities committed by the colonists or their impacts. We
are not shaped by our ancestors, we are shaped by everyone's ancestors,
as we all live in one world.

By revising "we are the memories we don't remember" to "we live in the
memories we don't remember," we can accommodate denial, hope, and
breadth as is demonstrated by Edwin's arc throughout the novel. He
starts off denying his reality, primarily living in \emph{Second Life}, then
moves to being hopeful by shaping the world around him for the powwow,
and finally to being robbed and shot illustrating the breadth of what
effects him.
\end{enumerate}

\subsection{final version}
\label{sec:org4c01126}
\url{https://docs.google.com/document/d/1S-htkZDhHdHZh8CRe50VdCajlTc1U1b\_mHx6hasAlGs/edit}
\end{document}
