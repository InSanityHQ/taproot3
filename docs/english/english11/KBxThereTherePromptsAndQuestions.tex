% Created 2021-09-27 Mon 12:02
% Intended LaTeX compiler: xelatex
\documentclass[letterpaper]{article}
\usepackage{graphicx}
\usepackage{grffile}
\usepackage{longtable}
\usepackage{wrapfig}
\usepackage{rotating}
\usepackage[normalem]{ulem}
\usepackage{amsmath}
\usepackage{textcomp}
\usepackage{amssymb}
\usepackage{capt-of}
\usepackage{hyperref}
\setlength{\parindent}{0pt}
\usepackage[margin=1in]{geometry}
\usepackage{fontspec}
\usepackage{svg}
\usepackage{cancel}
\usepackage{indentfirst}
\setmainfont[ItalicFont = LiberationSans-Italic, BoldFont = LiberationSans-Bold, BoldItalicFont = LiberationSans-BoldItalic]{LiberationSans}
\newfontfamily\NHLight[ItalicFont = LiberationSansNarrow-Italic, BoldFont       = LiberationSansNarrow-Bold, BoldItalicFont = LiberationSansNarrow-BoldItalic]{LiberationSansNarrow}
\newcommand\textrmlf[1]{{\NHLight#1}}
\newcommand\textitlf[1]{{\NHLight\itshape#1}}
\let\textbflf\textrm
\newcommand\textulf[1]{{\NHLight\bfseries#1}}
\newcommand\textuitlf[1]{{\NHLight\bfseries\itshape#1}}
\usepackage{fancyhdr}
\pagestyle{fancy}
\usepackage{titlesec}
\usepackage{titling}
\makeatletter
\lhead{\textbf{\@title}}
\makeatother
\rhead{\textrmlf{Compiled} \today}
\lfoot{\theauthor\ \textbullet \ \textbf{2021-2022}}
\cfoot{}
\rfoot{\textrmlf{Page} \thepage}
\renewcommand{\tableofcontents}{}
\titleformat{\section} {\Large} {\textrmlf{\thesection} {|}} {0.3em} {\textbf}
\titleformat{\subsection} {\large} {\textrmlf{\thesubsection} {|}} {0.2em} {\textbf}
\titleformat{\subsubsection} {\large} {\textrmlf{\thesubsubsection} {|}} {0.1em} {\textbf}
\setlength{\parskip}{0.45em}
\renewcommand\maketitle{}
\date{\today}
\title{}
\hypersetup{
 pdfauthor={},
 pdftitle={},
 pdfkeywords={},
 pdfsubject={},
 pdfcreator={Emacs 28.0.50 (Org mode 9.4.4)}, 
 pdflang={English}}
\begin{document}

\tableofcontents

\#ref \#ret \#inclass

\noindent\rule{\textwidth}{0.5pt}

\section{Let's go?}
\label{sec:orge97866f}
\begin{verbatim}
**Directions:** Choose ONE of the following prompts and respond in a well-developed, organized, clear two- to three-page essay. Remember to include a bold yet concise thesis statement and to support it with ample evidence and analysis from throughout the text.

**Prompts**:

1.   In his Prologue, Orange claims that “we are the memories we don’t remember, which live in us, which we feel” (10). Choose one character from the novel, and explain how he/she embodies, revises, or refutes this claim. Be sure to explain/define the quotation and how it connects to the character you are discussing.
2.   Orange posits the concept of the “Urban Indian” in his Prologue (11). Define this concept and explain how one of the novel’s characters navigates this modern identity alongside his/her ancestral roots. 
3.   Discuss the Interlude section of the novel. What is its function in the book? How does it connect to the various narratives and the work as a whole?
4.   Choose one character from the novel and discuss why his or her narrative is 1st, 2nd, or 3rd person. What effect does the form of narrative have on how we read, understand, and know the character? 
5.   Create your own prompt (_requires a check-in with your teacher_). When designing your topic, please consider whether it is arguable in a paper of this length.
\end{verbatim}

\noindent\rule{\textwidth}{0.5pt}

there's gonna be a diagnostic essay but no assessed discussion

\subsubsection{mac and cheese time}
\label{sec:org7a55aa0}
has to do with our content, and aproach!

\begin{itemize}
\item mac and cheese activity was stylistic analysis
\item we were talking about what it means to be american \textasciitilde{}= what it means to
be mac and cheese? - rohan

\begin{itemize}
\item Q about def or identity
\item what is NOT

\begin{itemize}
\item ie powder on noodles is not mac and cheese
\end{itemize}
\end{itemize}

\item what is "proper"

\begin{itemize}
\item problematic concept!
\end{itemize}

\item lived experiences shape perceptions? - riyanna
\end{itemize}

\begin{verbatim}
mac and cheese is a text - *sarah*
\end{verbatim}

\begin{itemize}
\item exclusion and elitist

\item people in power vs majority vs\ldots{} consensus?

\item ooh, mia asks if there is merit to exclusion

\item i kinda dont want to take notes on this sorry

\item notes on beyonce?

\begin{itemize}
\item parental adivosry
\item i cant see man
\item i need glasses
\item wtf is happening
\item 
\end{itemize}
\end{itemize}

\noindent\rule{\textwidth}{0.5pt}

\subsubsection{ok back to it}
\label{sec:orgee18a68}
most intresting:

\begin{enumerate}
\item In his Prologue, Orange claims that "we are the memories we don't
remember, which live in us, which we feel" (10). Choose one character
from the novel, and explain how he/she embodies, revises, or refutes
this claim. Be sure to explain/define the quotation and how it
connects to the character you are discussing.
\end{enumerate}

\begin{verbatim}
are we?
\end{verbatim}

i disagree. I am not my ancestors actions? even though i am there genes?

memories we dont remeber -> ancestors experiences

experience propogation? ie. ancestor feels this which effects us in this
way? memories shape the way the current world is even if we dont know
the memories? we "feel" what is\ldots{} what is is a product of previous
actions which are memories that we dont own and thus dont remeber?

is this, we are \emph{only} the memories? we are \emph{partly} the memories?

actions vs past and culture "It's important that he dress like an
Indian, dance like an Indian, even if it is an act, even if he feels
like a fraud the whole time, because the only way to be Indian in this
world is to look and act like an Indian." -- defined by action not by
memory?

relates to the idea of authenticy? identity

really, it's about ways of addressing the past

\begin{verbatim}
are we defined by our ancestors actions and experiences? and if so, how? 
and then, how do we address or think about ourselvs in the context of the past?
\end{verbatim}
\end{document}
