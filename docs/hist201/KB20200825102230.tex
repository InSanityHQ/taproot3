% Created 2021-09-11 Sat 08:17
% Intended LaTeX compiler: xelatex
\documentclass[letterpaper]{article}
\usepackage{graphicx}
\usepackage{grffile}
\usepackage{longtable}
\usepackage{wrapfig}
\usepackage{rotating}
\usepackage[normalem]{ulem}
\usepackage{amsmath}
\usepackage{textcomp}
\usepackage{amssymb}
\usepackage{capt-of}
\usepackage{hyperref}
\usepackage[margin=1in]{geometry}
\usepackage{fontspec}
\usepackage{indentfirst}
\setmainfont[ItalicFont = LiberationSans-Italic, BoldFont = LiberationSans-Bold, BoldItalicFont = LiberationSans-BoldItalic]{LiberationSans}
\newfontfamily\NHLight[ItalicFont = LiberationSansNarrow-Italic, BoldFont       = LiberationSansNarrow-Bold, BoldItalicFont = LiberationSansNarrow-BoldItalic]{LiberationSansNarrow}
\newcommand\textrmlf[1]{{\NHLight#1}}
\newcommand\textitlf[1]{{\NHLight\itshape#1}}
\let\textbflf\textrm
\newcommand\textulf[1]{{\NHLight\bfseries#1}}
\newcommand\textuitlf[1]{{\NHLight\bfseries\itshape#1}}
\usepackage{fancyhdr}
\pagestyle{fancy}
\usepackage{titlesec}
\usepackage{titling}
\makeatletter
\lhead{\textbf{\@title}}
\makeatother
\rhead{\textrmlf{Compiled} \today}
\lfoot{\theauthor\ \textbullet \ \textbf{2021-2022}}
\cfoot{}
\rfoot{\textrmlf{Page} \thepage}
\titleformat{\section} {\Large} {\textrmlf{\thesection} {|}} {0.3em} {\textbf}
\titleformat{\subsection} {\large} {\textrmlf{\thesubsection} {|}} {0.2em} {\textbf}
\titleformat{\subsubsection} {\large} {\textrmlf{\thesubsubsection} {|}} {0.1em} {\textbf}
\setlength{\parskip}{0.45em}
\renewcommand\maketitle{}
\author{Zachary Sayyah}
\date{\today}
\title{Day 1 History Emily}
\hypersetup{
 pdfauthor={Zachary Sayyah},
 pdftitle={Day 1 History Emily},
 pdfkeywords={},
 pdfsubject={},
 pdfcreator={Emacs 27.2 (Org mode 9.4.4)}, 
 pdflang={English}}
\begin{document}

\maketitle


\section{Intro}
\label{sec:org33e5de4}
\subsection{About her}
\label{sec:org6e0131d}
\begin{itemize}
\item She majored in war and some history stuff

\begin{itemize}
\item Decided history wasn't enough to find out why people fight and kill
each-other
\end{itemize}
\end{itemize}

\subsection{Expectations}
\label{sec:org4255a2b}
\begin{itemize}
\item Readings are dense, so it isn't expected that you note and know every
single detail.

\begin{itemize}
\item Just make sure you're noting the big themes and ideas the author is
trying to communicate
\end{itemize}

\item She doesn't like talking alone for long amounts of time and prefers
that we unmute and shout out

\begin{itemize}
\item No need to raise hands to begin with.
\end{itemize}
\end{itemize}

\subsection{Course Focus}
\label{sec:orgc1c2b87}
\begin{itemize}
\item We will be starting with the 1500's because that is where most
scholars say is where the modern world system began

\begin{itemize}
\item We will start with the 4 most populous regions in the world
\end{itemize}

\item We will start with Hegemony and competition in the early modern world
\item Balance of power
\item Ideologies and technologies
\item Balance disrupted WW1 and it's aftermath
\item Balance disrupted pt2 Fascism Communism, WW2 and the new international
order (1920-1955)

\begin{itemize}
\item This is where we will be writing the major research paper of the
year
\item This is where we should be paying attention to what we are
interested in.
\end{itemize}

\item Bipolar times: Cold war, decolonization, and regional conflict
(1955-2000)
\end{itemize}

By the end of this course we should understand what lead us to the place
we are. Be critical of why stuff matters.

\subsection{Key Skills}
\label{sec:org6a3dcbb}
\begin{itemize}
\item Use primary and secondary source evidence to craft historical
arguments

\begin{itemize}
\item Seeking to uncover both the general and specific causes of
historical events.
\item Use and evaluate sources critically
\item This class will have us write a lot of essays

\begin{itemize}
\item Be able to construct and defend arguments with explanatory power

\begin{itemize}
\item Be able to explain why things happened the way they did and
figure out how this helps us understand stuff.
\end{itemize}

\item Try to come up with our own theories and explanations of how the
world works looking for patterns.
\item Express yourself in writing with clarity and logic
\end{itemize}

\item Look at different perspectives of sources
\end{itemize}
\end{itemize}

My in Class stuff dw about it Guidelines: - Military power - Relative
power of allies - Power over people of the country - Economic power -
Other countries economic dependence

Machevellie stuff - Fear and love of a government are comparable when it
comes to power over one's people. However, he argues that fear might be
even stronger.
\end{document}
