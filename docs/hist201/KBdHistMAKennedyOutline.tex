% Created 2021-09-11 Sat 09:36
% Intended LaTeX compiler: xelatex
\documentclass[letterpaper]{article}
\usepackage{graphicx}
\usepackage{grffile}
\usepackage{longtable}
\usepackage{wrapfig}
\usepackage{rotating}
\usepackage[normalem]{ulem}
\usepackage{amsmath}
\usepackage{textcomp}
\usepackage{amssymb}
\usepackage{capt-of}
\usepackage{hyperref}
\usepackage[margin=1in]{geometry}
\usepackage{fontspec}
\usepackage{indentfirst}
\setmainfont[ItalicFont = LiberationSans-Italic, BoldFont = LiberationSans-Bold, BoldItalicFont = LiberationSans-BoldItalic]{LiberationSans}
\newfontfamily\NHLight[ItalicFont = LiberationSansNarrow-Italic, BoldFont       = LiberationSansNarrow-Bold, BoldItalicFont = LiberationSansNarrow-BoldItalic]{LiberationSansNarrow}
\newcommand\textrmlf[1]{{\NHLight#1}}
\newcommand\textitlf[1]{{\NHLight\itshape#1}}
\let\textbflf\textrm
\newcommand\textulf[1]{{\NHLight\bfseries#1}}
\newcommand\textuitlf[1]{{\NHLight\bfseries\itshape#1}}
\usepackage{fancyhdr}
\pagestyle{fancy}
\usepackage{titlesec}
\usepackage{titling}
\makeatletter
\lhead{\textbf{\@title}}
\makeatother
\rhead{\textrmlf{Compiled} \today}
\lfoot{\theauthor\ \textbullet \ \textbf{2021-2022}}
\cfoot{}
\rfoot{\textrmlf{Page} \thepage}
\titleformat{\section} {\Large} {\textrmlf{\thesection} {|}} {0.3em} {\textbf}
\titleformat{\subsection} {\large} {\textrmlf{\thesubsection} {|}} {0.2em} {\textbf}
\titleformat{\subsubsection} {\large} {\textrmlf{\thesubsubsection} {|}} {0.1em} {\textbf}
\setlength{\parskip}{0.45em}
\renewcommand\maketitle{}
\author{Dylan}
\date{\today}
\title{Kennedy Essay Outline}
\hypersetup{
 pdfauthor={Dylan},
 pdftitle={Kennedy Essay Outline},
 pdfkeywords={},
 pdfsubject={},
 pdfcreator={Emacs 27.2 (Org mode 9.4.4)}, 
 pdflang={English}}
\begin{document}

\maketitle
\begin{verbatim}
In Chapter 1 of Rise and Fall of the Great Powers, Paul Kennedy sketches out an explanation of why the Ming Dynasty was, on the one hand powerful and prosperous, but ultimately was “a country which had turned in on itself” and subject to “steady relative decline.” Mann, in his chapter on the Ming trade, gives the reader a lot more detail on the nuances of Ming history in this period. Putting Kennedy and Mann into dialogue, does Kennedy’s argument still hold up? In your essay, argue for or against Kennedy’s argument using the details of Ming history analyzed by Mann.
\end{verbatim}

\section{Kennedy}
\label{sec:org3830f65}
\begin{itemize}
\item Kennedy's argument is that the downfall of the Ming Dynasty was caused
by the "conservativism of the Confucian bureaucracy"

\begin{itemize}
\item Several key actions

\begin{itemize}
\item Banned Seafaring

\begin{itemize}
\item Loss of opportunity
\end{itemize}

\item Backwards thinking

\begin{itemize}
\item Looked towards the past
\end{itemize}

\item Distrust towards merchants

\begin{itemize}
\item Hurt the economy
\end{itemize}
\end{itemize}

\item Result

\begin{itemize}
\item Lack of investment (domestic) into development, rather into land
\item Merchants suffered due to no backing from government (who
controled economy)
\end{itemize}
\end{itemize}

\item Some of Kennedy's arguments can be supported by Mann

\begin{itemize}
\item Zheng's explorations were very lucrative for Ming because it allowed
them to show their presence in areas that they had little physical
contact

\begin{itemize}
\item Sri Lanka, Sumatra, etc
\item Banning of it and other expeditions was troubling because it meant
that the Ming would fall behind in its \emph{projection of power}
\end{itemize}

\item The banning of foreign trade encouraged piracy, eventually leading
to pirate-controlled enclaves

\begin{itemize}
\item Weakened government control
\item Tribute payments were still allowed, but only applied to the
government

\begin{itemize}
\item Used as a front for the government to trade with foreign nations

\begin{itemize}
\item "A front for international commerce" (Mann, 127)
\end{itemize}
\end{itemize}
\end{itemize}

\item Introduction of fiat currency led to inflation, which in turn led to
the decline of the economy
\end{itemize}
\end{itemize}
\end{document}
