% Created 2021-09-11 Sat 08:17
% Intended LaTeX compiler: xelatex
\documentclass[letterpaper]{article}
\usepackage{graphicx}
\usepackage{grffile}
\usepackage{longtable}
\usepackage{wrapfig}
\usepackage{rotating}
\usepackage[normalem]{ulem}
\usepackage{amsmath}
\usepackage{textcomp}
\usepackage{amssymb}
\usepackage{capt-of}
\usepackage{hyperref}
\usepackage[margin=1in]{geometry}
\usepackage{fontspec}
\usepackage{indentfirst}
\setmainfont[ItalicFont = LiberationSans-Italic, BoldFont = LiberationSans-Bold, BoldItalicFont = LiberationSans-BoldItalic]{LiberationSans}
\newfontfamily\NHLight[ItalicFont = LiberationSansNarrow-Italic, BoldFont       = LiberationSansNarrow-Bold, BoldItalicFont = LiberationSansNarrow-BoldItalic]{LiberationSansNarrow}
\newcommand\textrmlf[1]{{\NHLight#1}}
\newcommand\textitlf[1]{{\NHLight\itshape#1}}
\let\textbflf\textrm
\newcommand\textulf[1]{{\NHLight\bfseries#1}}
\newcommand\textuitlf[1]{{\NHLight\bfseries\itshape#1}}
\usepackage{fancyhdr}
\pagestyle{fancy}
\usepackage{titlesec}
\usepackage{titling}
\makeatletter
\lhead{\textbf{\@title}}
\makeatother
\rhead{\textrmlf{Compiled} \today}
\lfoot{\theauthor\ \textbullet \ \textbf{2021-2022}}
\cfoot{}
\rfoot{\textrmlf{Page} \thepage}
\titleformat{\section} {\Large} {\textrmlf{\thesection} {|}} {0.3em} {\textbf}
\titleformat{\subsection} {\large} {\textrmlf{\thesubsection} {|}} {0.2em} {\textbf}
\titleformat{\subsubsection} {\large} {\textrmlf{\thesubsubsection} {|}} {0.1em} {\textbf}
\setlength{\parskip}{0.45em}
\renewcommand\maketitle{}
\author{Exr0n}
\date{\today}
\title{Causes of WWI Essay Outline}
\hypersetup{
 pdfauthor={Exr0n},
 pdftitle={Causes of WWI Essay Outline},
 pdfkeywords={},
 pdfsubject={},
 pdfcreator={Emacs 27.2 (Org mode 9.4.4)}, 
 pdflang={English}}
\begin{document}

\maketitle
\section{prompt}
\label{sec:org453eef9}
\begin{quote}
The political scientist Kenneth Waltz argues that the causes of war can be analyzed at three different levels: the individual human level, the state level, and the international system level. Those who view things from the first level believe that war is best explained by “selfishness,” “misdirected aggressive impulses,” or “stupidity” within the human psyche.

Those who favor the second level believe there are hostile or aggressive or revisionist states who, because of their form of government or other domestic issues, behave in a warlike manner while other states simply want to keep the peace (the status quo).

Those who favor the third level believe that the international system itself, because it is an anarchy with “no system of law enforceable” between states, and in which each state acts according to its own interest and reserves the right to use force to achieve its aims, makes war inevitable.

Analyze World War 1 according to one (or a blend) of these levels of analysis. Which best explains the general causes of the war as well as the specific sequence of events (\textbf{including events that prolonged the war beyond the initial outbreak})?

Essays should cite from the Palmer reading, and, if you want to aim for the exemplary evidence standard, any of these \href{https://drive.google.com/drive/folders/1KTggTDz3Yl7fT9MxwG4l25qMPNyiUioe?usp=sharing}{primary sources as well}.


Other Submission guidelines: 3 pages, size 12 font, double-spaced. Citations should be in-line and formatted as (Authorname Pagenumber) i.e. (Kennedy 12). Include a Works Cited page in MLA format for the secondary sources. Primary sources do not need to be included in the Works Cited page, but their authorship and date and other relevant information should be introduced in your text when you cite them.

Tips: See the essay rubric guide below for questions to ask yourself as you write and revise.
\href{https://docs.google.com/document/d/1cHuvVjKQbwUmRgRh2qbgk76dbMBoOcCgaBAasiznj6U/edit?usp=sharing}{History essay rubric guide}
\end{quote}
\section{standards targets}
\label{sec:orga8f4c75}
\subsection{{\bfseries\sffamily TODO} knowledge of history: reference specific events, places, dates, and people with a clear sense of chronology}
\label{sec:org10458db}
\subsection{{\bfseries\sffamily TODO} understanding patterns: define an array of historical trends \{ religious?, political, social, economic, cultural \}}
\label{sec:orgd721d65}
\subsection{{\bfseries\sffamily TODO} understanding patterns: show how the trends affect each other}
\label{sec:orgb4edb62}
\subsection{{\bfseries\sffamily TODO} argument and argumentation: address the prompt, explain events with a nuanced and precise sense of cause and effect}
\label{sec:org61eda8a}
\subsection{{\bfseries\sffamily TODO} argument connects to broader trends and specific moments}
\label{sec:org98bc723}
\subsection{{\bfseries\sffamily TODO} needs a "so what" to demonstrate it's relevance}
\label{sec:orgffc1fb2}
\subsection{{\bfseries\sffamily TODO} use of evidence: use specific evidence from the 'widest array of sources' to support points in every paragraph}
\label{sec:org4620d3c}
\subsection{written expression: use precise terminology and express nuanced thoughts, make clear intro/conclusion, and body paragraph structure clarified in thesis}
\label{sec:orgb788702}
\subsection{{\bfseries\sffamily TODO} \textbf{events} that prolonged the war after it's outbreak}
\label{sec:org45854fb}

\section{evidence}
\label{sec:org9759cbc}

\subsection{primary sources}
\label{sec:org8cfde17}

\subsubsection{German historian Heinrich von Treitschke (1834-1896) glorified warfare quoted in Politics (1899-1900) (14 years before the war)}
\label{sec:org5922bc6}

\begin{enumerate}
\item war is the only way out for 'an afflicted people'
\label{sec:org1cd0b24}

\item forgo the ego and join the greater good (greatness of war)
\label{sec:org3dfedc7}

\item those who appeal to peace / Christianity are cowards ('the leader should wield the sword')
\label{sec:orgcdab22e}

\item (peace is reactionary -> bad) -> (banishing war -> banishing progress)
\label{sec:org402817f}
\end{enumerate}

\subsubsection{German general and influential military writer Friedrich von Bernhardi (1849-1930) in \emph{Germany and the Next War} (1911)}
\label{sec:org2685df2}

\begin{enumerate}
\item 'war is the father of all things', concept of war being necessary for all advancement of society
\label{sec:orgf69306b}

\item concept of 'the mighty must do what the mighty must' (international anarchy)
\label{sec:org6b81d18}

\item flourishing nations need more land and thus 'conquest becomes a law of necessity'
\label{sec:orgb26d9bd}

\item 'the right of conquest is universally acknowledged'\ldots{} 'right to annexation'
\label{sec:org19f1716}

\item 'might is at once the supreme right', basically war is necessary, correct, and natural
\label{sec:org2415649}

\item conclusion: exclusion of war 'must be demonstrably untenable'
\label{sec:orgf5d6620}
\end{enumerate}

\subsubsection{French writer Ronald Dorgeles (1885-1973) recalls the mood in [Paris at the outbreak of the war}
\label{sec:orge2fd1e5}

\begin{enumerate}
\item stunned -> 'What? War, was it? Well, then, let's go!'
\label{sec:orgadaf2dc}

\item people were ready and excited, 'but this time it was better than a song'
\label{sec:orgb049b94}

\item excited by seeing cavalry and foot soldiers marching off to battle
\label{sec:org7f9ccf0}

\item even the socialist workers 'seeing their old dreams of peace crumble' would cry 'To Berlin!' (even they are pro-war)
\label{sec:org4fc25d9}

\item 'Frenchmen' national identity brought people of socioeconomic and political diversity together
\label{sec:org71da14a}

\item hindsight: was the fight and death worth it if 50 years later everyone was friendly
\label{sec:org6211f15}
\end{enumerate}

\subsubsection{poems from dudes in the trenches}
\label{sec:org2b51a6f}

\begin{enumerate}
\item it sucks
\label{sec:org0ea4929}
\end{enumerate}

\subsection{the book dude}
\label{sec:org447bbfd}

\subsubsection{level 1}
\label{sec:org4a0a2d4}

\begin{enumerate}
\item ethnically diverse citizens (serbian nationalists, among others) want to not be part of austria hungary
\label{sec:org646cd4e}
\end{enumerate}

\subsubsection{level 2}
\label{sec:orgb9ddfb5}

\begin{enumerate}
\item germany rose up, making france and russia concerned (is this innevitable)
\label{sec:org89cba52}

\item leaders expected war to come, so that may have made it more innevitable
\label{sec:org3ca74dc}
\end{enumerate}

\subsubsection{level 3}
\label{sec:org7d49446}

\begin{enumerate}
\item game theoretic prisoner's dilemma style cost matrix
\label{sec:orga49fb01}

\item no common power to hold states accountable to the both-defend policy
\label{sec:org85f75f6}

\item security dilemma
\label{sec:org3037861}
\end{enumerate}

\section{outline}
\label{sec:org764c3d3}
\subsection{{\bfseries\sffamily TODO} Intro}
\label{sec:org413c568}
\subsection{{\bfseries\sffamily TODO} Thesis}
\label{sec:org73d038b}
\subsubsection{\textbf{e}}
\label{sec:org782244c}
\subsection{BP1: citizens start the war}
\label{sec:org6300323}
\subsubsection{germany wants a spot in the sun}
\label{sec:org2fec31c}
The rapid economic development of Germany [cite tables] in [years] inflated the German national identity and instilled fear in its neighbors.
Spot in the sun [cite palmer, authorship, year]
\subsubsection{primary sources}
\label{sec:org7300d97}
These writers further inflated the national identity and glorified warfare, creating a populous that is itching to fight.
\subsubsection{security dilemma => people think war is coming (level 3 influence)}
\label{sec:orgab12fc7}
Troops are amassed on both sides in a vicious cycle, an example of the so called 'security dilemma' (level 3 mechanisms catalyze the war)

\subsubsection{bad news bears in the Balkans (sets it off)}
\label{sec:org5afdf9c}

\subsection{{\bfseries\sffamily TODO} BP2: governments keep the war}
\label{sec:org424b910}
\subsubsection{mutinies}
\label{sec:orge25f104}
\subsubsection{government ideas}
\label{sec:org380abaf}
\subsection{{\bfseries\sffamily TODO} BP3: greater power could've stopped the reaction at any time}
\label{sec:org67eeb9d}
\subsubsection{{\bfseries\sffamily TODO} germany was scared of US involvement}
\label{sec:orgb98e6b0}
\subsubsection{{\bfseries\sffamily TODO} MAD would change the reward matrix (cult of the offensive -> cult of the defensive)}
\label{sec:org74e436d}
\subsection{{\bfseries\sffamily TODO} Conclusion}
\label{sec:org41ab25b}

\section{outline2 barley boogaloo}
\label{sec:org1b70d93}

\subsection{intro}
\label{sec:orgd93778a}
At the turn of the twentieth century, Europe was locked in an arms race caused by international political and economic incentives. As tensions grew, cultural strifes inevitably intensified and ultimately sparked the war.

\subsection{{\bfseries\sffamily DONE} thesis}
\label{sec:orgc0c8db6}
Although a lack of enforcement of international order and ballooning militaries both incentivized and enabled WWI, the necessary spark was provided by individual civilian interests.

\subsection{{\bfseries\sffamily DONE} BP1: level 3 security dilemma/cult of the offensive puts everyone on edge}
\label{sec:orgef26322}
Reinforcing international incentives such as the security dilemma and cult of the offensive put each of the international powers on edge, bringing the European powers closer to war.

\subsubsection{industrialization -> bigger militaries}
\label{sec:orgbd3a2eb}
As a united Germany industrialized, both its population and industrial might grew to rival the French and British powers of the time. For instance, in 1880--nine years after Germany was officially unified--the German empire produced only 8.5\% of the world's manufacturing output while Britain produce 22.9\% of it. By 1913, deep into the security dilemma and one year before the war, Germany had surpassed British production and nearly doubled that of France's (Kennedy, Table 18).

Contries tend to grow their military as they industrialize, if only for defensive purposes. As Germany doubled it's military population over three decades to challenge century-long British and French domination, nearby countries grew wary. As surpassed power and a failing empire, France and Russia grew wary of the newfound power between them. They allied with Britain in 1904 and 1907 respectively for fear of a coming war.

\subsubsection{the security dilemma}
\label{sec:org18cfef2}
As countries formed aliances and grew their militaries, opposing parties were forced to keep up in the arms race. This so called "security dilemma" doubled the number of military and navel personel worldwide in the 30 years between the German unification and the war, and nearly tripled the global warship tonnage (Kennedy, Table 19-20). A level two perspcetive would explain this aggression with Germany's expansionistic ideals, but even Britain's liberal parliamentary democracy quadrupled it's naval tonnage.

\subsubsection{cult of the offensive}
\label{sec:org7824bdf}
Leaders at the time believed that preempting war would allow a fast and decisive victory. Even simplifying the outcomes to two countries and four possibilites, where each country either attacks or defends, greedy actors will choose to preempt war. As a result, each country prepared to invade it's neighboors, and European tensions grew.

\subsubsection{mutually assured destruction}
\label{sec:orgb7f2cf1}
In fact, had a there been an international disencentive such as Mutually Assured Destruction, the relative ordering of possibilites and therefore the cost matrix would've prevented all out war between such parties. For example, in a nuclear scenerio where any attacked country can retaliate with their own warheads, the utility of each scenerio would be ordered as follows:

\begin{center}
\begin{tabular}{rll}
Utility & Our Actions & Their Actions\\
\hline
4 & Defend & Defend\\
3 & Attack & Defend\\
2 & Attack & Attack\\
1 & Defend & Attack\\
\end{tabular}
\end{center}

And in a two party system,

\begin{center}
\begin{tabular}{lll}
Top,Left & Attack & Defend\\
\hline
Attack & 2, 2 & 1, 3\\
Defend & 3, 1 & 4, 4\\
\end{tabular}
\end{center}

Although it may seem less risky for any given party to attack, the utility of both defend increases as weapons get stronger until both parties opt for a defense strategy under MAD. Modern mutually assured destruction has so far prevented all out war, and a lack of such disincentives made war more likely in the early 1900s.

\subsection{{\bfseries\sffamily TODO} BP2: upset / war-hungry people are required to spark the war}
\label{sec:org3fe3441}

As a side effect of this global militarization, the populous glorified and anticipated war. This level three influence on the level one psyche inflamed nationalist ideals across Europe and primed a now-ticking explosive.

Popular works from the years leading up to the war describe how natural and necessary war is.
For instance, German general and influential military writer Friedrich von Bernhardi (1849-1930) wrote in \emph{Germany and the Next War} (1911) that "War is a biological necessity of the first importance," and "The right of conquest is universally acknowledged." (CITE)
As both a high-ranking general and a best-selling author, Bernhardi was in a unique position to influence the public opinion about war. His aggressive stance is not surprising given his military background, and his work was instrumental to priming Germany for battle. A nation cannot go to war without the support of the populous, as the citizens at large provide the troops, taxes, and labor to sustain warfare. Such vehement arguments swayed public opinion and opened the possibility of large-scale battle.

A level two viewpoint may counter that Germany was naturally expansionist, but similar widespread sentiment in France suggests government structure and ideology were not a sufficient influence on public opinion. French writer Ronald Dorgeles (1885-1973) recalls the mood in Paris at the outbreak of war, writing "Suddenly a heroic wind lifted their heads. What? War, was it? Well then, let's go!" (CITE)
The French parliamentary constitutional government had been weakened by civil unrest and would hardly have been able to force a uncooperative populous to war, but even the left-wing activists agreed in August of 1914 to refrain from calling strikes during the duration of the war in the Union Sacrée or Sacred Union. (CITE needed? :question: \url{https://journals.sagepub.com/doi/abs/10.1177/026569147800800402?journalCode=ehqa}) Thus, French actions could not have been primarily governmental influence, and such countries went to war due to level three influences on public opinion.

An exclusively level one viewpoint may counter that German writers like Heinrich von Treitschke had been espousing and glorifying war decades before the rapid German industrialization beginning in 1970. However, the shift was more recent in other countries. For instance, Dorgeles notes the ideological one-eighty that socialist workers take upon hearing of war. "seeing their old dreams of peace crumble, [socialism workers] would stream out into the boulevards \ldots{} [but] they would cry 'To Berlin!,' not 'Down with war!'" (CITE) Although Germany's actions may be a result of it's level two structure, the level three influence on level one psyche is required to explain the actions of other states.

\subsubsection{serbs in austria want to make yugoslavia (also bosnia?)}
\label{sec:org78f6a3b}

As countries militarized and nationalist views grew, would ethnic and religious divisions intensify until something inevitably sparked war.
In the case of WWI, the weakest link was the religious divide in Austria-Hungary. Over the course of a number of "Balkan crises," the Eastern Orthodox Serbs and Bosnians in southern Austria-Hungary grew discontent with the Roman Catholic Dual Monarchy that ruled the Habsburg empire--soon to be Austria-Hungary. As the Ottoman empire declined, the Serbs marked Bosnia as their own and were infuriated when Austria annexed Bosnia in 1908. When a the Balkan wars saw Austria cut Serbia off from the sea, Serbs both independent and Austrian grew exasperated and desperate. (Reader page 6 CITE)
This chain of events was driven by recent level three influences: the ongoing security-dilemma-induced arms race had Germany's neighbors scrambling for land and power. States and citizens alike were expecting war, and looking to gain as much of an upper hand as possible before it broke out.

level 3 -> level 1: individual actions cause stuff
try to cite something external about expectations causing stuffs

These level three influences also shifted the general psyche to become more war-like, pushing a few individuals near the top of the bell curve past a critical point.
Cultural cracks plauge numerous states, but these

\subsection{{\bfseries\sffamily TODO} conclusion}
\label{sec:orgc47326a}
As power dynamics shifted around the turn of the twentieth century, the defined scarcity of state goals--such as the British ambition of having the largest navy--set off a chain of events that led ultimately and innevitably to global war. Without a change of level three incentives, such as a global mediator or mutually assured destruction, shifting power dynamics and the cult of the offensive will lead inescapably to security-dilemma-induced arms race and growing tensions which cause nationalist viewponts and breed rash individuals. Thus, international disincentives like mutually assured destruction are key to keeping political and economic incentives from inflaming ideological divides and causing warfare.

\section{todos}
\label{sec:org5e92444}

\subsection{{\bfseries\sffamily DONE} read \href{https://drive.google.com/drive/folders/1KTggTDz3Yl7fT9MxwG4l25qMPNyiUioe?usp=sharing}{primary sources}.}
\label{sec:org6043acf}

\subsection{{\bfseries\sffamily DONE} review evidence/notes}
\label{sec:org3059314}


\subsection{questions 5 april 2021}
\label{sec:orgb63d5b7}
\subsubsection{{\bfseries\sffamily DONE} how much of bad news bears in the balkans is needed\hfill{}\textsc{question}}
\label{sec:org67e4ff5}
\begin{enumerate}
\item bosnia
\label{sec:org7330cba}
\end{enumerate}
\subsubsection{{\bfseries\sffamily DONE} can I cite external sources / tables n stuff\hfill{}\textsc{question}}
\label{sec:orge78a93c}
\begin{enumerate}
\item yes
\label{sec:org558f032}
\end{enumerate}
\subsubsection{{\bfseries\sffamily DONE} how to cite stuff in class?\hfill{}\textsc{question}}
\label{sec:orge8e71cb}
\begin{enumerate}
\item people just trying to outlast others -- need explicit or analyze from shipping?
\label{sec:orga71ac7c}
\item idea of war on others turf
\label{sec:orge7b7870}
\end{enumerate}
\subsubsection{{\bfseries\sffamily DONE} what are you looking for in the intro / conclusion?\hfill{}\textsc{question}}
\label{sec:orge7ce4c5}
\begin{enumerate}
\item intro - need background? or just thesis? avoid fluff?
\label{sec:org36ae570}
\item conclusion: what do you know now? why is this cool? whats not obvious from the thesis?
\label{sec:org482e06e}
\end{enumerate}
\subsection{questions 7 april 2021}
\label{sec:orgb347fed}
\subsubsection{{\bfseries\sffamily DONE} what trends are there?}
\label{sec:org90472a1}
just connect stuff. no set list of trends. link different domains
\subsubsection{{\bfseries\sffamily DONE} need to talk about post war stuff?}
\label{sec:orgf9d3671}
\subsubsection{{\bfseries\sffamily DONE} check thesis? does it work for body paragraphs?}
\label{sec:org7277ab6}
is last paragraph level 2 or level 1? phrase as a 'all human nature thing', imagine how it would happen in liberal democracies
\subsubsection{{\bfseries\sffamily DONE} length?}
\label{sec:org9f4bc6f}
\subsubsection{{\bfseries\sffamily DONE} so what?}
\label{sec:org2c3a876}

\subsection{{\bfseries\sffamily DONE} come up with general frame}
\label{sec:org026f5ee}

\subsection{{\bfseries\sffamily DONE} come up with argument, body paragraphs}
\label{sec:org87affff}

\subsection{{\bfseries\sffamily TODO} \{ outline, write, edit \} for standards}
\label{sec:org1e9422b}
\end{document}
