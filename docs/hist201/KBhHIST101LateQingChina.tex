% Created 2021-09-11 Sat 09:36
% Intended LaTeX compiler: xelatex
\documentclass[letterpaper]{article}
\usepackage{graphicx}
\usepackage{grffile}
\usepackage{longtable}
\usepackage{wrapfig}
\usepackage{rotating}
\usepackage[normalem]{ulem}
\usepackage{amsmath}
\usepackage{textcomp}
\usepackage{amssymb}
\usepackage{capt-of}
\usepackage{hyperref}
\usepackage[margin=1in]{geometry}
\usepackage{fontspec}
\usepackage{indentfirst}
\setmainfont[ItalicFont = LiberationSans-Italic, BoldFont = LiberationSans-Bold, BoldItalicFont = LiberationSans-BoldItalic]{LiberationSans}
\newfontfamily\NHLight[ItalicFont = LiberationSansNarrow-Italic, BoldFont       = LiberationSansNarrow-Bold, BoldItalicFont = LiberationSansNarrow-BoldItalic]{LiberationSansNarrow}
\newcommand\textrmlf[1]{{\NHLight#1}}
\newcommand\textitlf[1]{{\NHLight\itshape#1}}
\let\textbflf\textrm
\newcommand\textulf[1]{{\NHLight\bfseries#1}}
\newcommand\textuitlf[1]{{\NHLight\bfseries\itshape#1}}
\usepackage{fancyhdr}
\pagestyle{fancy}
\usepackage{titlesec}
\usepackage{titling}
\makeatletter
\lhead{\textbf{\@title}}
\makeatother
\rhead{\textrmlf{Compiled} \today}
\lfoot{\theauthor\ \textbullet \ \textbf{2021-2022}}
\cfoot{}
\rfoot{\textrmlf{Page} \thepage}
\titleformat{\section} {\Large} {\textrmlf{\thesection} {|}} {0.3em} {\textbf}
\titleformat{\subsection} {\large} {\textrmlf{\thesubsection} {|}} {0.2em} {\textbf}
\titleformat{\subsubsection} {\large} {\textrmlf{\thesubsubsection} {|}} {0.1em} {\textbf}
\setlength{\parskip}{0.45em}
\renewcommand\maketitle{}
\author{Houjun Liu}
\date{\today}
\title{Late Qing China}
\hypersetup{
 pdfauthor={Houjun Liu},
 pdftitle={Late Qing China},
 pdfkeywords={},
 pdfsubject={},
 pdfcreator={Emacs 27.2 (Org mode 9.4.4)}, 
 pdflang={English}}
\begin{document}

\maketitle


\section{Late Qing China}
\label{sec:org1741ee9}
\#flo

\textbf{The Qing dynasty's failure is centered upon the early Industrial Rev.}

\subsection{Failures of the Qing court}
\label{sec:org33badb9}
\begin{itemize}
\item Considered the British as a lowly subject and dismissed their claims
to increase trade and communication
\item Created a high trade deficit through the oblivious to trade through
Confucian philosophy
\end{itemize}

\subsection{The Opium Wars}
\label{sec:orgaa6453f}
\begin{itemize}
\item The British used opium as compensation for the trade deficit between
Britian=>China and China=>Britian
\item When Chinese officials began controlling the trade of opionum, it was
both to late and also was handled with mixed results due to difference
in opionion.
\item When sudden enforcement of trade occured, the British felt like it was
an insult to the British crown and proceeded to wage war against the
navally-weak China.
\item After loosing the opionum war, the Qing court had to agree to a series
of supplicating agreements that ended with hurtful consiquences and
CLAIM without realizing destroyed the foreign policies of China.
\end{itemize}

\subsection{Beginnings of Unrest}
\label{sec:org18ca790}
\begin{itemize}
\item The Taiping Movement

\begin{itemize}
\item Taiping movement threatened the Qing government's (a.k.a. British)
trading of opium, but supported the spead of Christianity (albeit
the flavor where the emperor is Jesus' yonger brother.)
\item Power struggle in the Taiping regieme lead to internal collapse a
few years later.
\item Manchu weakness forced them to give the Chinese-Chinese army more
power to quell the movement
\end{itemize}
\end{itemize}

*In 1858, Anglo-French forces invaded Beijing, storming the summer
palace, took over the Chinese tax system, and eventually basically
established the Qing rule as a colony of the west.*

\begin{itemize}
\item Kidnapped Chinese people to serve as indentured servants in the west.
\item Confusion officials called for "self-strengthening", usually to little
results due to the argricultural-dependent Qing state.
\end{itemize}

\subsection{Japan vs. China}
\label{sec:org32d9e39}
\begin{itemize}
\item The Chinese succeded Taiwan to Japan
\item Lead the Western nations to fear for the collapse of the profitable
Qing dynasty
\item In turn, the \textbf{Scramble for consessions} occured where contries
fervantly attempted to establish special trading licenses.
\item The uninvolved US issued "open door notes" to all contries, calling on
opening China as a free trade zone.
\end{itemize}

\subsection{The Empress Dowagers vs. Herself vs. The Europeans}
\label{sec:orgfe7c556}
\begin{itemize}
\item The empress dowager's mismanagement of funds are CLAIM a symptom of
the Qing court weakness.
\item Kang Youwei urged the emperor to issue many edicts of westernization,
but was quickly crushed by the empress dowager cixi.
\item Conservatives seized the control of the Qing court, which resulted in
more anger and mutiny throughout the country (but, interestingly,
against western regions.)
\end{itemize}

\subsubsection{Boxer Rebellion and the Final Countdown}
\label{sec:org128909c}
\begin{itemize}
\item The Boxer rebellion urged the whole country to dispose of any
foreigners there may be for they believed that the foreigners were the
root of Qing dynasty's problems.
\item The empress eventually supported their decision, causing the \textbf{8-nation
army} to invade and her to fleed to the countryside. After this, the
e.d. decided to support westernization.
\item Urge for the adoption of constitutional monarchy created centers of
opposition to the Qing imperial system.
\item Sun Zhongshan, after being found out to promote the overthrowing of
the Chinese governement, fled to Japan and established his concept of
revolution.
\item Qiu Jin --- a woman revolutionary who studied in Japan and went back
to China for the revolutionary cause. She was later executed for
treasion.
\end{itemize}

\section{CN1202020}
\label{sec:orgecf5ccd}
\textbf{China vs. Imperialism}

The Dynastic Cycle\ldots{}

\begin{enumerate}
\item New Empire

\begin{itemize}
\item Peace
\item Order
\end{itemize}

\item Old Empire

\begin{itemize}
\item Corruption
\item Incompetence
\item Downfaall
\end{itemize}

\item Conflict

\begin{itemize}
\item Natural Disasters
\item Invasions
\end{itemize}
\end{enumerate}

Repeat!

The British acquired a large amount of tea, silk, and vermillion from
China, but imported basically only Opium.

\begin{itemize}
\item Opium => unproductivity

\begin{itemize}
\item Weakened the productivity of China
\item Offset the trade deficit with Britian
\end{itemize}
\end{itemize}

\textbf{Opium is technically illegal}, so the whole operation is largely
smuggling.

\textbf{Taiping Rebellion}

\begin{itemize}
\item CLAIM @sushu: much more scary than even the Opium war

\begin{itemize}
\item Took a large part of the country
\item Got control of a large part of grain and wheat production
\end{itemize}

\item 洪秀全 --- Taiping Rebellion

\begin{itemize}
\item Thinks that he is Jesus' old brother

\begin{itemize}
\item Had a fever dream of visiting "father"
\item After hearing about Christian philosophy, ret-cons the dream into
\end{itemize}

\item Is a failed scholar who did not work for the government
\item Harnessed peasant discontent + cut South China away
\end{itemize}

\item To rapidly quell the Taiping rebellion\ldots{}

\begin{itemize}
\item Granted more army and power to the provincial government
\item Decentralized control, which caused problems down the line
\end{itemize}
\end{itemize}
\end{document}
