% Created 2021-09-11 Sat 09:36
% Intended LaTeX compiler: xelatex
\documentclass[letterpaper]{article}
\usepackage{graphicx}
\usepackage{grffile}
\usepackage{longtable}
\usepackage{wrapfig}
\usepackage{rotating}
\usepackage[normalem]{ulem}
\usepackage{amsmath}
\usepackage{textcomp}
\usepackage{amssymb}
\usepackage{capt-of}
\usepackage{hyperref}
\usepackage[margin=1in]{geometry}
\usepackage{fontspec}
\usepackage{indentfirst}
\setmainfont[ItalicFont = LiberationSans-Italic, BoldFont = LiberationSans-Bold, BoldItalicFont = LiberationSans-BoldItalic]{LiberationSans}
\newfontfamily\NHLight[ItalicFont = LiberationSansNarrow-Italic, BoldFont       = LiberationSansNarrow-Bold, BoldItalicFont = LiberationSansNarrow-BoldItalic]{LiberationSansNarrow}
\newcommand\textrmlf[1]{{\NHLight#1}}
\newcommand\textitlf[1]{{\NHLight\itshape#1}}
\let\textbflf\textrm
\newcommand\textulf[1]{{\NHLight\bfseries#1}}
\newcommand\textuitlf[1]{{\NHLight\bfseries\itshape#1}}
\usepackage{fancyhdr}
\pagestyle{fancy}
\usepackage{titlesec}
\usepackage{titling}
\makeatletter
\lhead{\textbf{\@title}}
\makeatother
\rhead{\textrmlf{Compiled} \today}
\lfoot{\theauthor\ \textbullet \ \textbf{2021-2022}}
\cfoot{}
\rfoot{\textrmlf{Page} \thepage}
\titleformat{\section} {\Large} {\textrmlf{\thesection} {|}} {0.3em} {\textbf}
\titleformat{\subsection} {\large} {\textrmlf{\thesubsection} {|}} {0.2em} {\textbf}
\titleformat{\subsubsection} {\large} {\textrmlf{\thesubsubsection} {|}} {0.1em} {\textbf}
\setlength{\parskip}{0.45em}
\renewcommand\maketitle{}
\author{Houjun Liu}
\date{\today}
\title{Communism}
\hypersetup{
 pdfauthor={Houjun Liu},
 pdftitle={Communism},
 pdfkeywords={},
 pdfsubject={},
 pdfcreator={Emacs 27.2 (Org mode 9.4.4)}, 
 pdflang={English}}
\begin{document}

\maketitle


\section{Communism (!!!)}
\label{sec:org1bf9bdd}
Communism is a derivative branch of socialism --- that society should be
built bottom-up with resources shared --- which is in itself a form of
\href{KBhHIST201Nationalism.org}{KBhHIST201Nationalism}

\subsection{The Communist Manifesto}
\label{sec:orgacfb0e3}
\begin{itemize}
\item Written by two German exiles (Karl Marx + Friedrich Engels)
\item Called for worldwide \emph{Worker's Revolution} => overthrow capitalism +
establish fully public property
\item Marked the emergence of socialism that, CLAIM, would become a powerful
force for change in Europe
\end{itemize}

\subsubsection{Spread of socialism}
\label{sec:orgd6858f5}
\begin{itemize}
\item By 1883, Marxist societies spread all over Europe
\item Russia was seized in the 1917 communist revolutions => First Marxist
country started (USSR)
\end{itemize}

\subsection{Maxist Ideologies}
\label{sec:orgefb637d}
CLAIM: a product of the Englignment and in fact has been around for a
while.

\begin{itemize}
\item Reflect Enlightenment scientific, historical, and human condition
believes
\item Inspired by the French Rev + "Liberte, egalite, fratenete"
\end{itemize}

\subsubsection{The Guy}
\label{sec:org3430a59}
\begin{itemize}
\item Born in 1818 in Prussia to Jewish parents

\item father was a lawyer + studied enlightenment philosophers such as Kant
and Voltaire

\item Advocate for Prussian Constitutionalism

\item Educated in Germany + recieved Dr. of Phil at Jena

\item Edited the \emph{Rheininsche Zeitung}, but was shut down due to its
outspoken nature

\item Exiled to Paris with wife + met Friedrich Engels, but both were later
exiled from France and so they moved to Brussles
\end{itemize}

\textbf{Marx's Short Trip Around the World}

\begin{itemize}
\item Marx returned to Prussia temporarily + took a more moderate stance
advocating for democracy
\item When 1848's Summer brought the King of Prussia to cut back on the
democracy, Marx became more radical
\item Eventually, he was banished again --- first to France, and
subsequently London
\end{itemize}

Only involve in activism was the People's Spring; lived in poverty +
subsisted on bread and potateos.

\textbf{Death}

Suffered from "chronic mental depression" for he saw little hope for the
proletariat Advocated for a war to overthrow Russia, conservatism, etc.
This, of course, did happen, but that's called WWI

\subsubsection{The Manifesto}
\label{sec:orgf24de7c}
Created after the Marx and Engels joined the \emph{Communist League},
promoting a more extreme sect of socialism.

Published the Manifesto in Jan 1848, a 23 page phamplet outlining the
sect's vision.

The opening lines of the document claimed that a "specter" of Communism
is "Haunting" Europe, and that the traditional \emph{Holy Alliance} (pope,
Russian tsar, and ministers of Austria and France) were aiming to rid of
the specter

\textbf{Historical Materialism, Class Conflict, and Proletarian Revolution} is
at the heart of Marxist theory. That "history should not be understood
as \ldots{} [that] of great individuals but of social classes and their
struggles."

Class struggle w.r.t. capitalism is that between the bourgeoisie (owners
of capital) and the proletariat (workers/producers of capital.) Claims +
calls to action a overthrowing of the bourgeoisie by the proletariat to
create an egalitarian and classless society

Marx and Engels thought the Paris' June Days as an imminent sign of
revolution.

\subsection{The First International}
\label{sec:org7157265}
A antecedent of what became the Communist Party of the Soviet Union.

The organization rapidly expanded and grew to about 800,000 adherents by
1849, but CLAIM: failed due to the infighting caused by yet another
Paris Revolution + increased acceptance to reform instead of revolution.

\subsubsection{F.I. Failures}
\label{sec:orgcd9a7b3}
Marx and Engels thinks that a "dictatorship of the proletariat" was the
first step towards full-on communism, making them support the Paris
Commune + its violence. But! the First International contradicted their
opinion leading to internal debate.

\textbf{English Reform Bill} and the other systems of reform around the time
gave renewed hope on the possibility of reform.

\subsection{Marxist Theory, at a glance}
\label{sec:org8117824}
Marx think that Industrial Rev-brought machines makes people's lives
worse b/c is performs the division of labor and hence is used as an
additional instrument of the capitalist. By leveraging machines + wages,
capitalists are getting richer and richer though making the poor poorer.
"Instead of selling a product, the laborer must sell part of his own
being as property."

As machines create undifferentiated labor, the price of individual
dropped and productivity increased.

In general, Marx believed that the \emph{means of subsistence} should be
regularly provided + covered by average income.

Most important work are in the \emph{Communist Manifesto} and \emph{Das Kapital}
=> provided a "scientific" approach to history + economics that reflects
19th century lit trends w/ the development of realism.

Believes that there are branches of socialism that was too Utopian and
less "scientific" as his.

\subsubsection{Historical Materialism}
\label{sec:org68ba950}
That all things have a material basis, and so does historical
development. Believed that history could be analyzed through the
analysis of the \emph{means of production} of a place.

\textbf{Means of Production}

\begin{itemize}
\item Land in a feudal society
\item Factories in a capitalistic society
\end{itemize}

Believes that the owners of the \emph{means of production} is also the
dominators of society.

\subsubsection{Societal Substructures}
\label{sec:org1453bc0}
The owners of the \emph{means of production} control systems in society
called \emph{societal substructures}. Among these substructures are economy,
political system, relationships, culture, religion, and, this is cool,
\textbf{human freaking consciousness}. In other words, all these things are
controlled by those who are in control of the means of production.

According to the duo, all of these substructures are instruments that
the dominant uses to keep the lower classes in their place.

\subsubsection{Historical Pathways}
\label{sec:orgf8ec157}
Marx thinks that every society is traveling along predetermined
pathways, that societies\ldots{}

\begin{itemize}
\item Begin with a communal stage
\item Move into slavery (with master-slave-dominance)
\item Then into feudalism (with lord-subject-dominance)
\item and into capitalism (with owner-worker dominance)
\item and finally into communism (yay!)
\end{itemize}

Each of these pathways creates its \emph{dialectic} --- that each states
sowed the seeds of the next and of its own destruction. \textbf{Class
consciousness} develop in a community overtime --- where each lower
class realize that they have nothing to gain and resort to revolution.

Based on this believe, Marx thinks that capitalism is on its way out in
places like England, France, and Germany.

\subsubsection{Economic determinism}
\label{sec:org82fa699}
If you could get ahold of the economy, you control much of other things
in society.

\subsection{Communistic Utopia}
\label{sec:org4e17b2f}
\begin{quote}
when the workers own the means of pro- duction, the entire economic
substructure will collapse and re-form, as will the superstructure of
society. Social classes will disappear
\end{quote}

Believes that capitalism has provided the background onto which
everyone's basic needs could be satisfied if even distribution of goods
were supplied.

Creates a meritocracy of sorts where earnings are distributed based on
abilities, needs, and contribution. => each person contributes their
specialization and get what they need generally in return.

Believes that, once implemented, human nature will change no longer
yearn for capitalistic shows of wealth (say, a yacht) and instead will
foster cooperation and solidarity.

Through this, poverty will also be gone. And so apparently crime, greed,
and competition will too "wither away" according to Engels. This would
later contribute to the lack of necessity for states, and eventually
we'd be a whole happy planet of happy people.

\subsection{In Soviet Russia and elsewhere}
\label{sec:org6ec95b6}
Pre-1905 Russia had little political freedom. Russian exiles in
Switzerland took advantage of communistic ideals and formed the basis of
Marxism in Russia --- an attempted repositioning of Russia to fix its
despotic problems.

WWI weakened the Russian state, and, in 1917, the Communist \emph{Bolshevik}
lead by Lenin seized power.

Lenin established a new set of protocols based on Marxist ideas named
Marxism-Leninism --- uniting the Communist factions until collapse
in 1991.

WWII weakened Chinese, Eastern European, NK, South Asia, and Cubans
states, and they all decided to give communism a go to various degrees
of success.

\subsection{CN 11182020}
\label{sec:org81d77fd}
\#disorganized \#flo

\begin{itemize}
\item Gig working is cheaper, until machines happened:

\begin{itemize}
\item Looms => Power Loom --- machines could no longer fit in a "cottage"
\item Transitioned from local economy to being consolidated in cities
\end{itemize}

\item After the establishment of minimum work age (10), progressive
schooling occurred => "what are we to do with non-working kids with
working parents?"

\begin{itemize}
\item Elementary school designed to create a national citizen
\end{itemize}

\item IR: \textbf{the market is dictated by the machine}
\item Marx: repositioning perspective around labor
\end{itemize}

\begin{quote}
No ethical consumption in capitalism
\end{quote}
\end{document}
