% Created 2021-09-11 Sat 08:18
% Intended LaTeX compiler: xelatex
\documentclass[letterpaper]{article}
\usepackage{graphicx}
\usepackage{grffile}
\usepackage{longtable}
\usepackage{wrapfig}
\usepackage{rotating}
\usepackage[normalem]{ulem}
\usepackage{amsmath}
\usepackage{textcomp}
\usepackage{amssymb}
\usepackage{capt-of}
\usepackage{hyperref}
\usepackage[margin=1in]{geometry}
\usepackage{fontspec}
\usepackage{indentfirst}
\setmainfont[ItalicFont = LiberationSans-Italic, BoldFont = LiberationSans-Bold, BoldItalicFont = LiberationSans-BoldItalic]{LiberationSans}
\newfontfamily\NHLight[ItalicFont = LiberationSansNarrow-Italic, BoldFont       = LiberationSansNarrow-Bold, BoldItalicFont = LiberationSansNarrow-BoldItalic]{LiberationSansNarrow}
\newcommand\textrmlf[1]{{\NHLight#1}}
\newcommand\textitlf[1]{{\NHLight\itshape#1}}
\let\textbflf\textrm
\newcommand\textulf[1]{{\NHLight\bfseries#1}}
\newcommand\textuitlf[1]{{\NHLight\bfseries\itshape#1}}
\usepackage{fancyhdr}
\pagestyle{fancy}
\usepackage{titlesec}
\usepackage{titling}
\makeatletter
\lhead{\textbf{\@title}}
\makeatother
\rhead{\textrmlf{Compiled} \today}
\lfoot{\theauthor\ \textbullet \ \textbf{2021-2022}}
\cfoot{}
\rfoot{\textrmlf{Page} \thepage}
\titleformat{\section} {\Large} {\textrmlf{\thesection} {|}} {0.3em} {\textbf}
\titleformat{\subsection} {\large} {\textrmlf{\thesubsection} {|}} {0.2em} {\textbf}
\titleformat{\subsubsection} {\large} {\textrmlf{\thesubsubsection} {|}} {0.1em} {\textbf}
\setlength{\parskip}{0.45em}
\renewcommand\maketitle{}
\author{Houjun Liu}
\date{\today}
\title{Holy Roman Empire in the 1500s}
\hypersetup{
 pdfauthor={Houjun Liu},
 pdftitle={Holy Roman Empire in the 1500s},
 pdfkeywords={},
 pdfsubject={},
 pdfcreator={Emacs 27.2 (Org mode 9.4.4)}, 
 pdflang={English}}
\begin{document}

\maketitle


\section{The Holy Roman Empire}
\label{sec:orgc6548d1}
Attempted to adopt the model of European universality => one church, one
emperor.

But\ldots{} They didn't! Because\ldots{}

\subsection{Reasons for non-universality}
\label{sec:org7d2211e}
@\href{KBhHIST201Kissinger.org}{KBhHIST201Kissinger}

\begin{enumerate}
\item Lack of transportation and communication systems made tying large
countries together difficult
\item HRE had separation between church and state, which makes the
authority less authoritative
\item Pope + emperor constantly fought

\begin{itemize}
\item Need constitution to settle
\item Enabled fuetal rulers to enhance autonomy
\item Hasburg dynasty + combination with Spanish royalty => Über powerful
HRE => Almost centralized nation
\item Yet, eventually weakened pope brought end to religious
universality, which brought end to that centralized Europe idea
\end{itemize}
\end{enumerate}

Meanwhile, in France\ldots{}
\href{KBhHIST201RaisonDeEtat.org}{KBhHIST201RaisonDeEtat}

\subsection{Counter-Reformation}
\label{sec:orge687986}
A process of "revving Catholic universality".

See
\href{KBhHIST201CounterReformation.org}{KBhHIST201CounterReformation}

\subsection{Emperor Ferdinand II}
\label{sec:orge576b0a}
Practiced the\ldots{} well\ldots{} opposite of
\href{KBhHIST201RaisonDeEtat.org}{KBhHIST201RaisonDeEtat} => Religion
\begin{itemize}
\item Morality > state interest

\item Refused to treaty with Muslim Turks + Protestant Swedes
\item "Less concerned with the Empire's welfare than that of the will of God
\ldots{} The state existed to serve the religion \ldots{} for Ferdinand" --
\href{KBhHIST201Kissinger.org}{KBhHIST201Kissinger}
\item \textbf{Edict of reinstatution} => Demanded land taken by protestants since
1555 be returned
\end{itemize}
\end{document}
