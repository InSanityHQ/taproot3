% Created 2021-09-11 Sat 09:36
% Intended LaTeX compiler: xelatex
\documentclass[letterpaper]{article}
\usepackage{graphicx}
\usepackage{grffile}
\usepackage{longtable}
\usepackage{wrapfig}
\usepackage{rotating}
\usepackage[normalem]{ulem}
\usepackage{amsmath}
\usepackage{textcomp}
\usepackage{amssymb}
\usepackage{capt-of}
\usepackage{hyperref}
\usepackage[margin=1in]{geometry}
\usepackage{fontspec}
\usepackage{indentfirst}
\setmainfont[ItalicFont = LiberationSans-Italic, BoldFont = LiberationSans-Bold, BoldItalicFont = LiberationSans-BoldItalic]{LiberationSans}
\newfontfamily\NHLight[ItalicFont = LiberationSansNarrow-Italic, BoldFont       = LiberationSansNarrow-Bold, BoldItalicFont = LiberationSansNarrow-BoldItalic]{LiberationSansNarrow}
\newcommand\textrmlf[1]{{\NHLight#1}}
\newcommand\textitlf[1]{{\NHLight\itshape#1}}
\let\textbflf\textrm
\newcommand\textulf[1]{{\NHLight\bfseries#1}}
\newcommand\textuitlf[1]{{\NHLight\bfseries\itshape#1}}
\usepackage{fancyhdr}
\pagestyle{fancy}
\usepackage{titlesec}
\usepackage{titling}
\makeatletter
\lhead{\textbf{\@title}}
\makeatother
\rhead{\textrmlf{Compiled} \today}
\lfoot{\theauthor\ \textbullet \ \textbf{2021-2022}}
\cfoot{}
\rfoot{\textrmlf{Page} \thepage}
\titleformat{\section} {\Large} {\textrmlf{\thesection} {|}} {0.3em} {\textbf}
\titleformat{\subsection} {\large} {\textrmlf{\thesubsection} {|}} {0.2em} {\textbf}
\titleformat{\subsubsection} {\large} {\textrmlf{\thesubsubsection} {|}} {0.1em} {\textbf}
\setlength{\parskip}{0.45em}
\renewcommand\maketitle{}
\author{Houjun Liu}
\date{\today}
\title{Homogenocene}
\hypersetup{
 pdfauthor={Houjun Liu},
 pdftitle={Homogenocene},
 pdfkeywords={},
 pdfsubject={},
 pdfcreator={Emacs 27.2 (Org mode 9.4.4)}, 
 pdflang={English}}
\begin{document}

\maketitle
\textbf{Before we start, notes pre-globalization}

\begin{itemize}
\item Spanish royals actively prevented trade with China \#why
\item Current day, emphasis was placed around those in native American
regions who were anti-Spanish, yet a large majority of the individuals
who really brought globalization were Spanish
\end{itemize}

\section{Effects of the Homogenoscene}
\label{sec:org9b56c5e}
\subsection{Silver\ldots{}}
\label{sec:org8dd1589}
\begin{itemize}
\item (Most?) Important asset\\
\item Got cheapened as soon as the Americas overflowed with it, leading to
lot's of silver problems
\href{KBhHIST201ProblemsWithSilver.org}{KBhHIST201ProblemsWithSilver}
\end{itemize}

\subsection{The Little Ice Age}
\label{sec:org5454b8b}
See \href{KBhHIST201LittleIceAge.org}{KBhHIST201LittleIceAge} The
Little Ice Age

\subsection{Capital Movement}
\label{sec:orge7dae50}
\subsubsection{Original, pre-globalized capitals}
\label{sec:org43e89aa}
\begin{itemize}
\item All located within the tropics

\begin{itemize}
\item Beijing
\item Viayanagar (Hindu capital) -- Cairo
\end{itemize}

\item Decidedly not western\\
\item Decidedly hot (maybe, but Beijing was definitely not hot \#why)
\end{itemize}

\subsubsection{Globalized Capitals}
\label{sec:orgd634c5a}
\begin{itemize}
\item In the north, mostly

\begin{itemize}
\item New York\\
\item Chicago\\
\item Tokyo\\
\item Manchester
\item London
\end{itemize}

\item Much more westiner
\item \ldots{}and much colder
\end{itemize}

\subsection{Food for thought}
\label{sec:org509d924}
\begin{itemize}
\item What are some ways that the world was linked together?\\
\item Trace the effects of a commodity around the world.

\begin{itemize}
\item Silver
\href{KBhHIST201ProblemsWithSilver.org}{KBhHIST201ProblemsWithSilver}
\item Disease
\item Cold weather
\href{KBhHIST201LittleIceAge.org}{KBhHIST201LittleIceAge}
\item Slaves
\item Vegetables
\item Horses
\item Wars
\item Wood
\item Corn
\end{itemize}

\item What does the homogenocene mean to Mann?

\begin{itemize}
\item Changes to state position?
\item Shift of power?
\end{itemize}

\item What was the extent and the limits of Spanish power during this
period?
\end{itemize}

\noindent\rule{\textwidth}{0.5pt}

\textbf{Look out!}: interesting to see when population \emph{drops///booms} instead
of climbing/falling \emph{slowly}.

For instance, take Mexico City

\begin{itemize}
\item Was really "on top of it"
\item But did not experience rapid growth
\end{itemize}

\section{Maps}
\label{sec:org7870f5f}
Guiding question: \textbf{what tensions and contrasts do you see between the
different maps?}

\uline{Look for where the boundaries don't match.\_}

\href{https://taproot.shabang.cf/2020HIST201/Maps.pdf}{Them Maps}

\begin{itemize}
\item When looking at GDP , remember to analyze GDP vs. Population
\item Places worth interest

\begin{itemize}
\item Tokyo
\item Shanghai
\item Gulf of Oman
\end{itemize}

\item Places to perhaps focus

\begin{itemize}
\item Transportation w.r.t economy
\item Trans-boundary Rivers w.r.t. conflicts
\item River control w.r.t. control of power
\item Religious boundaries w.r.t. sectarian violence
\end{itemize}
\end{itemize}
\end{document}
