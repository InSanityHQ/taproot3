% Created 2021-09-11 Sat 09:36
% Intended LaTeX compiler: xelatex
\documentclass[letterpaper]{article}
\usepackage{graphicx}
\usepackage{grffile}
\usepackage{longtable}
\usepackage{wrapfig}
\usepackage{rotating}
\usepackage[normalem]{ulem}
\usepackage{amsmath}
\usepackage{textcomp}
\usepackage{amssymb}
\usepackage{capt-of}
\usepackage{hyperref}
\usepackage[margin=1in]{geometry}
\usepackage{fontspec}
\usepackage{indentfirst}
\setmainfont[ItalicFont = LiberationSans-Italic, BoldFont = LiberationSans-Bold, BoldItalicFont = LiberationSans-BoldItalic]{LiberationSans}
\newfontfamily\NHLight[ItalicFont = LiberationSansNarrow-Italic, BoldFont       = LiberationSansNarrow-Bold, BoldItalicFont = LiberationSansNarrow-BoldItalic]{LiberationSansNarrow}
\newcommand\textrmlf[1]{{\NHLight#1}}
\newcommand\textitlf[1]{{\NHLight\itshape#1}}
\let\textbflf\textrm
\newcommand\textulf[1]{{\NHLight\bfseries#1}}
\newcommand\textuitlf[1]{{\NHLight\bfseries\itshape#1}}
\usepackage{fancyhdr}
\pagestyle{fancy}
\usepackage{titlesec}
\usepackage{titling}
\makeatletter
\lhead{\textbf{\@title}}
\makeatother
\rhead{\textrmlf{Compiled} \today}
\lfoot{\theauthor\ \textbullet \ \textbf{2021-2022}}
\cfoot{}
\rfoot{\textrmlf{Page} \thepage}
\titleformat{\section} {\Large} {\textrmlf{\thesection} {|}} {0.3em} {\textbf}
\titleformat{\subsection} {\large} {\textrmlf{\thesubsection} {|}} {0.2em} {\textbf}
\titleformat{\subsubsection} {\large} {\textrmlf{\thesubsubsection} {|}} {0.1em} {\textbf}
\setlength{\parskip}{0.45em}
\renewcommand\maketitle{}
\author{Houjun Liu}
\date{\today}
\title{Unit 3 Essay (Outline)}
\hypersetup{
 pdfauthor={Houjun Liu},
 pdftitle={Unit 3 Essay (Outline)},
 pdfkeywords={},
 pdfsubject={},
 pdfcreator={Emacs 27.2 (Org mode 9.4.4)}, 
 pdflang={English}}
\begin{document}

\maketitle


\section{Industrial Revolution Essay (Outline)}
\label{sec:org91e0c18}
*For a nation-state to achieve global success and dominance in the 19th
century, they must leverage industrial modernization: bringing the
increased fighting power of an industrialized modern military, the
economic benefits of adopting and weaponizing free trade, and the
centralized political control offered by top-down civic nationalism.*

\subsection{Modern, industrialized military brings increased fighting power}
\label{sec:org8ac1dfa}
England's unification efforts allowed the country to gather a huge
competitive advantage over other, less-developed nation-states.

New industrialized technology made the military much better: "the
advanced technology of steam engines and machine-made tools gave Europe
decisive economic and military advantages." (Kennedy Ch. 4)

This advantage manifests itself in two core ways:

\begin{enumerate}
\item Modern military is much more economically and physically efficient
\item Modern military is stronger and could easily fight over other
advanced yet unindustrialized military
\end{enumerate}

"Despite a steady reduction in its own numbers after 1815, the Royal
Navy was at some times probably as powerful as the next three or four
navies in actual fighting power." (Kennedy Ch. 4). Due to to Britian's
steady technological advancement, the British Royal Navy was able to use
its advanced technology --- of which it was one the first to adopt ---
to offset its reduced numbers.

In addition to increased efficiency, the advanced technologies adopted
by the British Royal Navy also allowed it to conquer even traditionally
powerful armies with ease.

"The steam-driven gunboat meant that European sea power, already supreme
in open waters, could be extended inland. \ldots{} The ironclad \emph{Nemesis} \ldots{}
was a disaster for the defending Chinese forces, which were easily
brushed aside." (Kennedy Ch. 4) Without the technologies of the steam
engine --- the hallmark of industrialization, it would be impossible for
the British military to be able to so efficiently threaten the Chinese
rule.

Both of these factors would not exist because without the advent of the
advanced military technologies brought by industrialization.

\subsection{State support and weaponization of free-trade and private production}
\label{sec:org8170356}
brought wealth and economic power
:CUSTOM\textsubscript{ID}: state-support-and-weaponization-of-free-trade-and-private-production-brought-wealth-and-economic-power
In the 19th century, states began the "erosion of tariff barriers \ldots{}
[and] the widespread propergation of ideas about free trade and
incarnation." (Kennedy Ch. 4) These devices allowed economy to flow
freely throughout the world economy, and allowed the novel technologies
of industrialization that sprung up in Europe to spread rapidly.

European states actively leveraged their newly-advantageous position in
global trade to bring further economic wealth to them. The systems of
Imperialism were in part created to propagate European ideas of free
trade to more difficult markets like that of Asia or Africa: "European
states sought 'sheltered markets' free from such restrictions to trade
and found them in the colonies they established." (Mason Ch. 8 )

This forcible induction of state-sponsored "free trade" had significant
advantages for European nations such as Britian. Although opium trade to
China was officially managed by the private British East India Co.,
Britian did not hesitate to leverage their army to force trading to
continue: when Canton commisoner Lin Zexu threatened the Crown's opium
trade, "Britain sent an expeditionary military force \ldots{} [that] the
Chinese had no naval forces capable of defeating." (Ropp Late Qing).

The legacy Chinese forces had little capability to defend themselves
against the newly-strengthened British forces as per aforementioned, and
hence was forced to continue trade --- a lucrative business for the
British that not only offsetted their trade deficit with the Qing court,
but made an additional \$1 Million in profit. Arguably, the sustenance of
the opium trade in late Qing China is also one of the reasons that the
administration eventually collapsed

\subsection{Promotion of Civically Nationalistic Governance}
\label{sec:org18f2bde}
During the 19th century, two major European players --- Germany and
Italy --- were created from the advent of civic nationalistic ideals.
These new nations rapidly became modernized, developed, and rose to be a
world player in global power balance.

Throughout Europe, "Statesmen \ldots{} were using warfare and civic
nationalism to forge powerful new nation-states" (Mason Ch. 7). The
control offered by nationalistic ideals offered unique,
counter-balancing forces that created an opportunity for the ruling
class to both efficiently control subjects and ensure the integrity of
the nation.

Although it is not a strictly nationalistic speech, the Iron and Blood
address by the Prussian chancellor Otto von Bismarck rings with deeply
nationalistic sentiments: "the position of Prussia in Germany \ldots{} will
be determined \ldots{} by iron and blood" --- rallying the citizens to
powerful action under the image of "Germans": a \emph{national identity}
previously separated from statehood.

In this maneuver, Bismarck created the German confederation --- an act
that strengthened the already-dominant position of Prussia for dominance
in their European region.

\noindent\rule{\textwidth}{0.5pt}

\section{Paragraph on Nationalism}
\label{sec:orgd9f9c71}
In addition to the technical and economic prowess brought by
industrialization, the 19th century also engendered a new wave of
governance that rapidly enabled the growth of new, powerful nations.
Throughout Europe, "Statesmen \ldots{} were using warfare and civic
nationalism to forge new nation-states" (Mason Ch. 7) like that of
Germany and Italy --- creating soon-to-be dominant players on world
power balance. The process Civic nationalism is hugely effective partly
due to the duality of control it offers. Firstly, nationalism in itself
ensures the unity of a nation-state for it secures national uniformity
not under forced rule but under a shared identity: "a nation is a group
of people with a common culture, a sense of identity, and political
aspirations \ldots{} [it] requires the psychological element of identity".
Indeed, a sense of togetherness and common goal could be created without
having to resort to sheer physical and political leverage --- creating a
perfect space for the governance style of civic nationalism. "Civic
nationalism [is] directed from the top" (Mason Ch. 7), allowing seat of
power to directly implement actions in the uniform populous
psychologically primed to support actions that would further the
nation's shared aspirations. This feature of civic nationalism expedites
policy decisions --- allowing powerful nations to be created with
reasonably rapid pace. Conversely, states not following generally civic
nationalistic governance may experience "separatism": attempts of
enforcing \emph{civic} control without the establishment (or, in
multiculturalistic states, the possibility) of nationalism. When faced
with European competition, the Ottoman empire established a campaign of
"Defensive Developmentalism", where a combination of "military reform
[and] \ldots{} discipline and coordinat[tion of] their population" (Gelvin
Ch. 5) --- actions that enforce direct civic control --- lead to revolts
and unrest between classes and groups of conflicting interest to
eventually orchestrated the breakup of the Ottoman Empire (reference to
Mason Ch. 7). In Civically Nationalistic states, this would not take
place. Chancellor Otto von Bismark of Prussia, when unifying Germany,
leveraged the same civic control as the Ottomans but did so under a
nationalistic lens: claiming that "the position of Prussia in Germany
\ldots{} will be determined \ldots{} by iron and blood" --- creating a cry that
rallies the citizens to action under the image of "Germans", a \emph{national
identity} previously separated from statehood. Bismark's powerful call
to action solidified the Prussian involvement in the creation and
dominance in the German state --- a nation with influence over global
balance of power even today. Through its combined effect of control and
unity, the adoption of civic nationalism is a powerful strategy that
ensured the rapid political success of states in the 19th century.
\end{document}
