% Created 2021-09-11 Sat 08:18
% Intended LaTeX compiler: xelatex
\documentclass[letterpaper]{article}
\usepackage{graphicx}
\usepackage{grffile}
\usepackage{longtable}
\usepackage{wrapfig}
\usepackage{rotating}
\usepackage[normalem]{ulem}
\usepackage{amsmath}
\usepackage{textcomp}
\usepackage{amssymb}
\usepackage{capt-of}
\usepackage{hyperref}
\usepackage[margin=1in]{geometry}
\usepackage{fontspec}
\usepackage{indentfirst}
\setmainfont[ItalicFont = LiberationSans-Italic, BoldFont = LiberationSans-Bold, BoldItalicFont = LiberationSans-BoldItalic]{LiberationSans}
\newfontfamily\NHLight[ItalicFont = LiberationSansNarrow-Italic, BoldFont       = LiberationSansNarrow-Bold, BoldItalicFont = LiberationSansNarrow-BoldItalic]{LiberationSansNarrow}
\newcommand\textrmlf[1]{{\NHLight#1}}
\newcommand\textitlf[1]{{\NHLight\itshape#1}}
\let\textbflf\textrm
\newcommand\textulf[1]{{\NHLight\bfseries#1}}
\newcommand\textuitlf[1]{{\NHLight\bfseries\itshape#1}}
\usepackage{fancyhdr}
\pagestyle{fancy}
\usepackage{titlesec}
\usepackage{titling}
\makeatletter
\lhead{\textbf{\@title}}
\makeatother
\rhead{\textrmlf{Compiled} \today}
\lfoot{\theauthor\ \textbullet \ \textbf{2021-2022}}
\cfoot{}
\rfoot{\textrmlf{Page} \thepage}
\titleformat{\section} {\Large} {\textrmlf{\thesection} {|}} {0.3em} {\textbf}
\titleformat{\subsection} {\large} {\textrmlf{\thesubsection} {|}} {0.2em} {\textbf}
\titleformat{\subsubsection} {\large} {\textrmlf{\thesubsubsection} {|}} {0.1em} {\textbf}
\setlength{\parskip}{0.45em}
\renewcommand\maketitle{}
\author{Huxley}
\date{\today}
\title{Ropp on China}
\hypersetup{
 pdfauthor={Huxley},
 pdftitle={Ropp on China},
 pdfkeywords={},
 pdfsubject={},
 pdfcreator={Emacs 27.2 (Org mode 9.4.4)}, 
 pdflang={English}}
\begin{document}

\maketitle
\#flo \#ref \#disorganized

\noindent\rule{\textwidth}{0.5pt}

\section{On China, Ropp. (63 - 70)}
\label{sec:org1f7f734}
\subsection{Fall + aftermath of the Qing Empire (1800 - 1920)}
\label{sec:orgfcda429}
\begin{itemize}
\item European powers competing for world dom through trade and warfare

\item chinese trade sucked

\begin{itemize}
\item British traders became frustrated
\item But they liked the chinese products alot and were buying alot of
them
\end{itemize}

\item Merchants ranked low in confucian vlue system

\item trade to the Qing gov was not for generationg wealth but rather more
like a charity granted to the "barbarians"

\begin{itemize}
\item In exchange, they would pay respects to the son of heaven and his
court?
\end{itemize}

\item Led to bad trade conditions for the british merchants

\item denied british requests for better env, emporer said

\begin{itemize}
\item \begin{quote}
standard emperor's command to his lowly subjects: "Tremblingly
obey and show no negligence!"
\end{quote}

\item fun.
\end{itemize}

\item \begin{quote}
The emperor's condescending attitude reflected how little he under
stood the power realities of the world at the end of the eighteenth
cen- tury.
\end{quote}

\item Opium got added to the mix,

\begin{itemize}
\item changed the flow from basicly soley china -> britin (tea) to britin
-> chin (opium)
\end{itemize}

\item Opium addiction went crazy

\item Totally killed trade surplus

\item Chinese gov destroyed a bunch of opium, britian sent warships

\item Totally chrushed chinese forces

\item led to the british taking control over the policies, and increased
opium trade a bunch

\item bunch of rebelions occured, see pg. 66.5

\item Japan beat the Qing in a war, got taiwan?

\begin{itemize}
\item Becuase of this, other governments backed off on China out of fear
that it would collapse
\end{itemize}

\item Also led to (lots of) civil unrest

\item Heavy westernization occured

\item reformers got crushed, conservatives seized control of the court

\item bunch o' dates and stuff about revolutionaries

\item Look how it ended on pg. 70!
\end{itemize}
\end{document}
