% Created 2021-09-11 Sat 09:36
% Intended LaTeX compiler: xelatex
\documentclass[letterpaper]{article}
\usepackage{graphicx}
\usepackage{grffile}
\usepackage{longtable}
\usepackage{wrapfig}
\usepackage{rotating}
\usepackage[normalem]{ulem}
\usepackage{amsmath}
\usepackage{textcomp}
\usepackage{amssymb}
\usepackage{capt-of}
\usepackage{hyperref}
\usepackage[margin=1in]{geometry}
\usepackage{fontspec}
\usepackage{indentfirst}
\setmainfont[ItalicFont = LiberationSans-Italic, BoldFont = LiberationSans-Bold, BoldItalicFont = LiberationSans-BoldItalic]{LiberationSans}
\newfontfamily\NHLight[ItalicFont = LiberationSansNarrow-Italic, BoldFont       = LiberationSansNarrow-Bold, BoldItalicFont = LiberationSansNarrow-BoldItalic]{LiberationSansNarrow}
\newcommand\textrmlf[1]{{\NHLight#1}}
\newcommand\textitlf[1]{{\NHLight\itshape#1}}
\let\textbflf\textrm
\newcommand\textulf[1]{{\NHLight\bfseries#1}}
\newcommand\textuitlf[1]{{\NHLight\bfseries\itshape#1}}
\usepackage{fancyhdr}
\pagestyle{fancy}
\usepackage{titlesec}
\usepackage{titling}
\makeatletter
\lhead{\textbf{\@title}}
\makeatother
\rhead{\textrmlf{Compiled} \today}
\lfoot{\theauthor\ \textbullet \ \textbf{2021-2022}}
\cfoot{}
\rfoot{\textrmlf{Page} \thepage}
\titleformat{\section} {\Large} {\textrmlf{\thesection} {|}} {0.3em} {\textbf}
\titleformat{\subsection} {\large} {\textrmlf{\thesubsection} {|}} {0.2em} {\textbf}
\titleformat{\subsubsection} {\large} {\textrmlf{\thesubsubsection} {|}} {0.1em} {\textbf}
\setlength{\parskip}{0.45em}
\renewcommand\maketitle{}
\author{Huxley}
\date{\today}
\title{Unit One Essay Planning}
\hypersetup{
 pdfauthor={Huxley},
 pdftitle={Unit One Essay Planning},
 pdfkeywords={},
 pdfsubject={},
 pdfcreator={Emacs 27.2 (Org mode 9.4.4)}, 
 pdflang={English}}
\begin{document}

\maketitle
\noindent\rule{\textwidth}{0.5pt}

\section{Here we go again.}
\label{sec:orgb576ac3}
\subsection{Prompt}
\label{sec:org4dcdc7e}
\begin{verbatim}
Essay option 1: In the early modern period, three of the four major power centers of the world unified under multi-ethnic, multi-religious empires (the Qing in East Asia, the Mughals in South Asia, and the Ottomans in the Middle East), while Europe remained politically fragmented into multiple, competitive independent states. Why was no European power able to unify their local state system the way the Asian land empires did? In a well-organized essay, answer this question by comparing Europe to any two of the above empires. Your essay should cite evidence from a variety of secondary sources from the unit pdf. A strong essay will use the sources as evidence for both specific historical events that demonstrate your thesis as well as for broader concepts and trends. All sources from the Unit 1 reader are fair game, including what we read before the last essay. 

Essentially, why coudnt they centralize? Compare and contrast essay. 

Essay option 2: According to Charles Tilly’s “bellicist” theory of state formation, states form to protect a territory from external threats. In the process of doing this (war-making), they develop the tools to eliminate internal rivals as well (state-making), and in order to fund these endeavours, they develop institutions like taxation to raise revenue from their territory (extraction). In order to maximize the wealth available for extraction and prevent property owners from needing private armies to protect themselves, the state establishes rule of law to protect property rights (protection). Using this theory as a starting point, compare state formation in two or three of the four regions we have studied (Europe, India, China, and the Middle East/North Africa): your goal should be to highlight an interesting contrast and pose a causal explanation for this contrast. Your essay should cite evidence from a broad variety of secondary sources from the unit pdf. A strong essay will use the sources as evidence for both specific historical events that demonstrate your thesis as well as for broader concepts and trends. All sources from the Unit 1 reader are fair game, including what we read before the last essay.

\end{verbatim}

\subsection{Possible topics}
\label{sec:orgb708d63}
\begin{itemize}
\item Broa
\item Bellicist theory is \textbf{INCOMPLETE} (duhn duhn dunnnnn!)

\begin{itemize}
\item Bellicist theory is about war, and discounts other forms of danger.
Should be danger (and maybe some accompanying changes instead of
war.

\begin{itemize}
\item Three paragraphs, three examples where other forms of danger
required state-making
\item End with some conclusion about the concept of models / theories?
\end{itemize}

\item To think about: what other things require state-making besides
war-making?
\item Easy essay if I can get evidence
\end{itemize}

\item How does Bellicist theory incorporate trade?

\begin{itemize}
\item Is this managed by the state?
\item Only talks about interactions with other states if they are war.
\item Does trade require state-made organization?
\end{itemize}
\end{itemize}

\texttt{=<1AM-jots!>=}

\begin{enumerate}
\item Bellicist theory is incomplete
\label{sec:orge6d0368}
\begin{itemize}
\item Other forms of danger besides war require state-making

\begin{itemize}
\item Plauge

\begin{itemize}
\item Look at ottoman orthodox vs european new ways
\end{itemize}

\item Ideological

\begin{itemize}
\item Religon?

\begin{itemize}
\item Religon is effective way to control people, but religion can get
threatened.
\end{itemize}
\end{itemize}

\item Income / profit

\begin{itemize}
\item Silver inflation from Spain and stuff
\end{itemize}
\end{itemize}
\end{itemize}

Bellicist theory: war making doesnt directly lead to statemaking. War
making leads to need for statemaking which leads to statemaking or
collapse.

This is an imporant distinction that must be made \{becuaseeeeeee\}

it allows us to say things that threaten the state lead to state making,
effectively :)

The nesscecity for the state leads to its creating. Ie. if the state
gets hit, and its power is decreased, the cause of the hit can still be
counted as an equivalent to war-making.

\texttt{=</1AM-jots!>=}
\end{enumerate}

\subsection{True Planing, Begin.}
\label{sec:org68450ec}
\subsubsection{Evidence bin}
\label{sec:orgbdee5fc}
\begin{itemize}
\item Plague

\begin{itemize}
\item "Contemptuous of European ideas and prac· tices, the Turks declined
to adopt newer methods for containing plagues; consequently, their
populations suffered more from severe epidemics. In one truly
amazing fit of obscurantism, a force of janissar- ies destroyed a
state observatory in 1580, alleging that it had caused a plague."
Kennedy, 12
\item There was little improvement in communica- tions, and no machinery
for assistance in the event of famine, flood, and plague-which were,
of course, fairly regular occurrences. \emph{kenndey, 13}
\item 
\end{itemize}

\item Inflation

\begin{itemize}
\item In the late sixteenth century, inflation caused by a floodof cheap
silver from the New World (see Environ- ment and Technology: Metal
Currency and Inflation), affectedmany of the remaining landholders,
who col- lected taxes according to legally fixed rates. Some saw
theirpurchasing power decline so much that they could notreport for
military service. \emph{Bulliet, 491}

\item In the six- teenth and seventeenth centuries, however, precious
metal poured into Spain from silver and gold mines in the New World,
but there was no increase in the availability of goods and services.
The resulting infla-tion triggered a "price revolut ion" in Europe-a
gen- eral tripling of prices between 1500 and 1650. In Paris in 1650
the price of wheat and hay was fifteen times higher than the price
had been in 1500. \emph{Bulliet, 494}

\item As a result, the country faced the unsolvable problem of finding
money to pay the army and bureauc- racy. \emph{Bulliet, 500}

\item First, governments through- out Eurasia had attempted to displace
the warlords and tame the aristocracies that had provided services
to the crown by building armies and bureaucracies loyal to the
central government alone. Building these armies and bureaucracies
was expensive, and soldiers and bureaucrats had to be paid.
Inflation raised the costs of maintaining them in the manner to
which they had grown accustomed. \emph{gelvin, 33-34}

\item States spen t an enormous amount of money to sustain their emp
loyees. In Persia, for examp le, an estimated 38 percent of the
state's expendi tures went to the army. Another 41 percent went to
the imperial harem, the royal family, and royal attendants. States
competed with the private sector for resources, and this drove up
prices. \emph{gelvin, 35}

\item This complicated situation resulted in revolts that devastated
Anatolia between 1590 and 1610. Former landholding cavalrymen,
short-term soldiers released at the end of a campaign, peasants
overburdened by emer- gencytaxes, and even impoverished students of
religion formed bands of marauders. \emph{bulliet, 491}
\end{itemize}

\item Ideological

\begin{itemize}
\item Europeans engaged in numerous conflicts pitting Catholics against
Protestants. \emph{roberts, 45}
\item Assuredly, the effort to renew the church and make it holy had an
enormous and immediate impact upon politics. Sustained and serious
effort to make human life conform to God's will as revealed in the
Bible changed men's minds and altered their behavior. Wholesale
violence sanctified by dogmatic differences quickly erupted,
\emph{mcneil, 310}
\item The act of eliminating internal rival forces and insurgents from
within its own territories. \emph{systems and states, 11}
\end{itemize}
\end{itemize}

\subsubsection{Outline}
\label{sec:org85f83fb}
\begin{enumerate}
\item Thesis: Bellicist theory
\label{sec:org31a7fa5}
\begin{itemize}
\item Intro

\begin{itemize}
\item Bellicist theory is incomplete

\begin{itemize}
\item Bellicist theory is\ldots{}.
\item War making should be replaced with "need"

\begin{itemize}
\item Many things replace need, such as x y and z.
\item Tilly leaves out the obvious, which is that failure to do this
leads to crash

\begin{itemize}
\item Allows proving things in the negative
\end{itemize}
\end{itemize}
\end{itemize}
\end{itemize}

\item P1

\begin{itemize}
\item Plague

\begin{itemize}
\item Ottomans kept old ways of dealing with plague instead of adopting
new European ones which led to allot of plagues

\begin{itemize}
\item Dealing with plague required organization that needed to be
dealt with by the government
\item The Ottomans lack of state making (rejecting their orthodox
beliefs..?) aided their downfall.
\item preventing this would require state-making
\end{itemize}
\end{itemize}
\end{itemize}

\item P2

\begin{itemize}
\item Inflation

\begin{itemize}
\item Spain created a massive influx of silver which led to massive
inflation

\begin{itemize}
\item people hired by the government can no longer be paid off
\item Causes efforts at autonomy
\item Which requires state-making to crush
\end{itemize}
\end{itemize}
\end{itemize}

\item P3

\begin{itemize}
\item Ideological

\begin{itemize}
\item States control their people with religion
\item when a new religion is introduced, that is a threat to the states
control
\item so they perform "statemaking," and kill the members of the new
religion

\begin{itemize}
\item ex. the protestant revolution

\begin{itemize}
\item protestant ideas threatened the power of the state
\item state decided to statemake-ify them
\end{itemize}
\end{itemize}
\end{itemize}
\end{itemize}
\end{itemize}
\end{enumerate}

\subsection{Writing time.}
\label{sec:org3549817}
Charles Tilly is famously quoted as stating "War makes states and states
make war" (Systems and States, 10). He lays out a theory of state
making, claiming that a state has four primary functions: war making,
protection, extraction, and of course, state making. Simply put, war
making stems from the need to protect a territory, which then leads to
extraction, protection, and state making. Tilly leaves out the obvious,
which is that failure to achieve one of these functions would lead to a
crash. War making does not lead to state making if one loses the war.
\{diagram\} Tilly dubbed this idea "Bellicist Theory," bellicist meaning
one who advocates for war. However, the error in this theory is in the
namesake itself. Tilly overgeneralizes\{wc\}, ignoring many other causes
that lead to state making in the same way that war making does. To amend
this, 'need' should be in place of war making. \{diagram\} For war making,
this need would be the need for protection against an enemy army, or the
need for resources as acquired through war making. Throughout the course
of this essay, I will examine places in which this amended Bellicist
theory functions and normal Bellicist theory does not.

The Bellicist theory remains consistent when war making is replaced with
plague, a biological attack. Effective mitigation of plague requires a
level of organization higher than a single person can achieve. Instead,
it requires protection as provided by the state, perhaps taking the form
of sanitation mandates or information as opposed to the formation of an
army. This protection would require extraction, which, with the same
logic as originally applied with war making, would lead to state making.
An example of this occurrence--in the negative--can be found in the late
1500s during the fall of the Ottoman empire. Kennedy reports,
"Contemptuous of European ideas and practices, the Turks declined to
adopt newer methods for containing plagues;" (citation). Due to this
lack of new method adoption, the Ottomans did not follow the modified
Bellicist theory \{wc\}, and hence, "suffered more from severe epidemics"
(citation). Instead, the Ottomans opted for methods such as destroying
state observatories, "alleging that it had caused a plague" (citation).
This failure to achieve the primary functions of the state, as described
by Tilly, ultimately aided in the downfall of the Ottoman Empire.

Just like physical attack and biological attack, income attack\{wc,
intent?\} also leads to state making. Income attack can manifest in many
ways, one of which is inflation. When inflation occurred\{tense?\}, many
countries whose "soldiers and bureaucrats had to be paid," "faced the
unsolvable problem of finding money to pay the army and
bureaucracy"(Citation, gelvin, bulliet). During the late sixteenth
century, a flood of silver from Spain caused massive inflation, leading
to the problem described above. This event has been dubbed the "price
revolution[, with] a general tripling of prices between 1500 and
1650"(citation bulliet). This price revolution forced the Ottoman Empire
to implement "emergency taxes," or, in other words, extraction
(citation). In the Ottoman Empire, this inflation was so extreme that
soldiers, peasants, bureaucrats, "and even impoverished students of
religion formed bands of marauders"(citation bulliet). These marauders
turned to internal plundering, and, expectedly, the government attempted
to eliminate them. This process lines up exactly with Tilly's precise
definition of state making: "The act of eliminating internal rival
forces and insurgents from within its own territories" (citation). This
elimination was of course an act of protection, and thus, it can be seen
that income attack can be a replacement for war making while keeping
Bellicist theory consistent.

A final example is ideological attack. A way states control their people
is through a shared, commonly theological, ideology. When a competing
ideology is introduced, this threatens the state's control, and thus,
the state attempts to crush them. This process is seen time and time
again throughout history, but perhaps one of the most notable examples
is the Protestant Reformation. In the early 1500s, Catholicism was
threatened by a new ideology, Protestantism. This ideological threat
"had an enormous and immediate impact upon politics," and the state
quickly decided to "pit[\ldots{}] Catholics against Protestants"
(citation)(citation). This ideological threat led to the state
attempting to eliminate rivals within its territory, which, again, is
Tilly's precise definition of state making. The original threat was not
physical; there was no army the state had to protect itself against, and
yet, the process laid out in Bellicist theory still occurred. \#\#
Commence Google Docs Edit With broken formatting :sunglasses:
\href{https://docs.google.com/document/d/1bhHnfqcC083889JImMnP\_sD7S8siIN8k4C3tW6fxGfQ/edit?usp=sharing}{Google
Docs Essay}

\subsubsection{The Amended Bellicist Theory}
\label{sec:org9d0c08a}
Charles Tilly is famously quoted as stating "War makes states and states
make war" (Systems and States 10). He lays out his theory of state
making, claiming that a state has four primary functions: war making,
protection, extraction, and of course, state making. Simply put, war
making stems from the need to protect a territory, which then leads to
the other three functions. Tilly leaves out the obvious, which is that
failure to achieve one of these functions would lead to a crash. War
making does not lead to state making if one loses the war. \{diagram\}
Tilly dubbed this idea "Bellicist Theory," bellicist meaning one who
advocates for war. However, the limits of this theory are in its name
itself. While incredibly insightful, Tilly takes an overly narrow view,
implying that war making is the only cause that leads to state making.
This viewpoint ignores many other causes. Replacing 'war making' with
'address need' appropriately amends this model. \{diagram\} For war
making, this need would be the need for protection against an enemy
army, or the need for resources as acquired through war making.
Throughout the course of this essay, I will examine places in which this
amended Bellicist theory applies to the historical record while normal
Bellicist theory does not. When war making is replaced with plague, the
underlying ideas in the Bellicist theory remain valid. Effective
mitigation of contagious disease requires a level of organization beyond
that which a single person can achieve. Instead, it requires protection
as provided by the state, perhaps taking the form of sanitation mandates
or information as opposed to the formation of an army. This protection
would require extraction, which, with the same logic as originally
applied with war making, would lead to state making. The fall of the
Ottoman empire in the late 1500s demonstrates the explanatory power of
the amended Bellicist theory---in the negative. Kennedy reports,
"Contemptuous of European ideas and practices, the Turks declined to
adopt newer methods for containing plagues." By "declin[ing] to adopt
newer methods," the Ottomans failed to follow the amended Bellicist
theory, and hence, "suffered more from severe epidemics." They failed to
address the need. Instead, the Ottomans opted for methods such as
destroying state observatories, "alleging that it had caused a plague"
(Kennedy 12). This failure to achieve the primary functions of the state
ultimately contributed to the downfall of the Ottoman Empire. Much like
physical attack and biological attack, income attack can also lead to
state making. Income attack can manifest in many ways, one of which is
inflation. During the late sixteenth century, a flood of silver from
Spain caused massive inflation. With the currency devalued, many
countries whose "soldiers and bureaucrats had to be paid," faced "the
unsolvable problem of finding money to pay the army and bureaucracy"
(Gelvin 33-34, Bulliet 500). This event has been dubbed the "price
revolution[, with] a general tripling of prices between 1500 and 1650"
(Bulliet 494). The price revolution forced the Ottoman Empire to
implement "emergency taxes," or, in Tilly's terminology, extraction. In
the Ottoman Empire, this inflation was so extreme that soldiers,
peasants, bureaucrats, "and even impoverished students of religion
formed bands of marauders" (Bulliet 491). These marauders turned to
internal plundering, and, as might be expected, the government attempted
to eliminate them. This process lines up exactly with Tilly's precise
definition of state making: "The act of eliminating internal rival
forces and insurgents from within its own territories" (Systems and
States 11). This elimination was of course an act of protection. Thus,
it can be seen that the effects of an income attack can map to war
making in the amended Bellicist theory. A final example of a limitation
in Tilly's Bellicist theory---that is addressed by the amended Bellicist
theory---is the ideological attack. One way states control their people
is through a shared, typically theological, ideology. When a competing
ideology is introduced, this threatens the state's control. Consistent
with the amended Bellicist theory, the state attempts to crush them.
This process is seen time and time again throughout history, but perhaps
one of the most notable examples is the Protestant Reformation. In the
early 1500s, Catholicism was threatened by a new ideology,
Protestantism. This ideological threat "had an enormous and immediate
impact upon politics," and the state quickly decided to "pit[\ldots{}]
Catholics against Protestants" (McNeil 310, Roberts 45). This
ideological threat led to the state attempting to eliminate rivals
within its territory, which, again, is Tilly's precise definition of
state making. The original threat was not physical; there was no army
the state had to protect itself against. Tilly's theory does not address
ideological attack, and yet the process laid out in Bellicist theory
still occurred.\\
Tilly's theory is deeply insightful, and has phenomenal explanatory
power. But, in its efforts to explain the relationship of war making to
other state functions, it limits its focus. War making, influential as
it is, is a special case of a broader phenomenon. By amending Tilly's
theory and replacing war making with 'address need,' Tilly's insights
can be preserved and expanded. We have shown several examples where
amended Bellicist theory applies and Bellicist theory does not,
including plague, inflation, and ideological attack. Of course, these
are only a select few cases where amended Bellicist theory can be
applied. Powerful theories often start small, and are expanded over time
as their true capability is discovered. This theory may be such a case.

Works Cited:

"The Rise of the Western World." The Rise and Fall of the Great Powers:
Economic Change and Military Conflict from 1500-2000, by Paul M.
Kennedy, William Collins, 2017.

"The Middle East and the Modern World System." The Modern Middle East: A
History, by James L. Gelvin, Oxford University Press, 2020, pp. 33--44.

"Southwest Asia and the Indian Ocean, 1500-1750." The Earth and Its
Peoples: a Global History, by Richard W. Bulliet, Wadsworth, Cengage
Learning, 2011, pp. 484--508.

"Europe's Self-Transformation." A World History, by William H. McNeill,
Peking University Press, 2008, pp. 309--326.

Khakpour, Arta. Systems and States. August, 2020,
docs.google.com/presentation/d/1vJnwwlECaFQkPPlls\textsubscript{B68Ujhqd4zRIZwixMch}\textsubscript{TnO}=g/.
Google Slides Presentation.
\end{document}
