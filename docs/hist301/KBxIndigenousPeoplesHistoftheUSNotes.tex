% Created 2021-09-11 Sat 08:17
% Intended LaTeX compiler: xelatex
\documentclass[letterpaper]{article}
\usepackage{graphicx}
\usepackage{grffile}
\usepackage{longtable}
\usepackage{wrapfig}
\usepackage{rotating}
\usepackage[normalem]{ulem}
\usepackage{amsmath}
\usepackage{textcomp}
\usepackage{amssymb}
\usepackage{capt-of}
\usepackage{hyperref}
\usepackage[margin=1in]{geometry}
\usepackage{fontspec}
\usepackage{indentfirst}
\setmainfont[ItalicFont = LiberationSans-Italic, BoldFont = LiberationSans-Bold, BoldItalicFont = LiberationSans-BoldItalic]{LiberationSans}
\newfontfamily\NHLight[ItalicFont = LiberationSansNarrow-Italic, BoldFont       = LiberationSansNarrow-Bold, BoldItalicFont = LiberationSansNarrow-BoldItalic]{LiberationSansNarrow}
\newcommand\textrmlf[1]{{\NHLight#1}}
\newcommand\textitlf[1]{{\NHLight\itshape#1}}
\let\textbflf\textrm
\newcommand\textulf[1]{{\NHLight\bfseries#1}}
\newcommand\textuitlf[1]{{\NHLight\bfseries\itshape#1}}
\usepackage{fancyhdr}
\pagestyle{fancy}
\usepackage{titlesec}
\usepackage{titling}
\makeatletter
\lhead{\textbf{\@title}}
\makeatother
\rhead{\textrmlf{Compiled} \today}
\lfoot{\theauthor\ \textbullet \ \textbf{2021-2022}}
\cfoot{}
\rfoot{\textrmlf{Page} \thepage}
\titleformat{\section} {\Large} {\textrmlf{\thesection} {|}} {0.3em} {\textbf}
\titleformat{\subsection} {\large} {\textrmlf{\thesubsection} {|}} {0.2em} {\textbf}
\titleformat{\subsubsection} {\large} {\textrmlf{\thesubsubsection} {|}} {0.1em} {\textbf}
\setlength{\parskip}{0.45em}
\renewcommand\maketitle{}
\author{Huxley Marvit}
\date{\today}
\title{Notes on Indigenous Peoples History}
\hypersetup{
 pdfauthor={Huxley Marvit},
 pdftitle={Notes on Indigenous Peoples History},
 pdfkeywords={},
 pdfsubject={},
 pdfcreator={Emacs 27.2 (Org mode 9.4.4)}, 
 pdflang={English}}
\begin{document}

\maketitle
\#flo \#disorganized \#inclass

\noindent\rule{\textwidth}{0.5pt}

\section{Nice way to start the mornin}
\label{sec:orgd8acc3a}
\href{https://docs.google.com/presentation/d/1XgOKmrGAzfc7-x1gEXsQXQIsvbNFkZhkPzw66FktRng/edit\#slide=id.ged7f61d18c\_0\_0}{slides}

\subsubsection{skimming strat!}
\label{sec:orge281397}
everyrone should have one isnt expecting to read every page, epsec in
college experiment with skimming no need to mem every page etc.

tom does rec and outlines sections / types of argument

\begin{verbatim}
try automating?
\end{verbatim}

\subsubsection{questions}
\label{sec:org595a094}
\begin{verbatim}
1.  What is “The Myth” of US history that Dunbar-Ortiz seeks to challenge?

2.  Why does the history of indigenous peoples not fit into the framework of multiculturalism?

3.  How does Dunbar-Ortiz’s approach to the history of the United States compare with Hannah-Jones?

4.  What is the difference between an “origin narrative” and a history of the founding of the United States? What role does each play in our understanding of the country?
\end{verbatim}

\begin{enumerate}
\item hannah-jones still has american exceptionalism, dunbar says it's all
bad?

\begin{enumerate}
\item multiculturalism -> mixed salad, distinct

\begin{enumerate}
\item dunbar argues that natives got assimilated
\item hannah jones argues that natives drove that which makes america
exceptional
\end{enumerate}
\end{enumerate}
\end{enumerate}

\begin{verbatim}
every author has a "project," an agenda
\end{verbatim}

the history of american bends towrds freedom - \textasciitilde{}obama no, it's cus of
brutal fighting, argues dunbar and friends

\begin{verbatim}
we've kind of been looking at things from the perspective of the left a little bit
\end{verbatim}

"america, love it or leave it"

\subsubsection{A better anti-racism, article by Coleman Hughes}
\label{sec:org3db0afd}
\begin{itemize}
\item types on anti-racism

\begin{itemize}
\item MLK

\begin{itemize}
\item race is an insignificant atribute
\end{itemize}

\item race-consuinsess

\begin{itemize}
\item opressed groups need more recognition, repsect, resources

\begin{itemize}
\item not about racial harmony
\end{itemize}
\end{itemize}
\end{itemize}
\end{itemize}
\end{document}
