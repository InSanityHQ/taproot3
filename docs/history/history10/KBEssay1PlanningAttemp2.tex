% Created 2021-09-11 Sat 16:39
% Intended LaTeX compiler: xelatex
\documentclass[letterpaper]{article}
\usepackage{graphicx}
\usepackage{grffile}
\usepackage{longtable}
\usepackage{wrapfig}
\usepackage{rotating}
\usepackage[normalem]{ulem}
\usepackage{amsmath}
\usepackage{textcomp}
\usepackage{amssymb}
\usepackage{capt-of}
\usepackage{hyperref}
\usepackage[margin=1in]{geometry}
\usepackage{fontspec}
\usepackage{indentfirst}
\setmainfont[ItalicFont = LiberationSans-Italic, BoldFont = LiberationSans-Bold, BoldItalicFont = LiberationSans-BoldItalic]{LiberationSans}
\newfontfamily\NHLight[ItalicFont = LiberationSansNarrow-Italic, BoldFont       = LiberationSansNarrow-Bold, BoldItalicFont = LiberationSansNarrow-BoldItalic]{LiberationSansNarrow}
\newcommand\textrmlf[1]{{\NHLight#1}}
\newcommand\textitlf[1]{{\NHLight\itshape#1}}
\let\textbflf\textrm
\newcommand\textulf[1]{{\NHLight\bfseries#1}}
\newcommand\textuitlf[1]{{\NHLight\bfseries\itshape#1}}
\usepackage{fancyhdr}
\pagestyle{fancy}
\usepackage{titlesec}
\usepackage{titling}
\makeatletter
\lhead{\textbf{\@title}}
\makeatother
\rhead{\textrmlf{Compiled} \today}
\lfoot{\theauthor\ \textbullet \ \textbf{2021-2022}}
\cfoot{}
\rfoot{\textrmlf{Page} \thepage}
\titleformat{\section} {\Large} {\textrmlf{\thesection} {|}} {0.3em} {\textbf}
\titleformat{\subsection} {\large} {\textrmlf{\thesubsection} {|}} {0.2em} {\textbf}
\titleformat{\subsubsection} {\large} {\textrmlf{\thesubsubsection} {|}} {0.1em} {\textbf}
\setlength{\parskip}{0.45em}
\renewcommand\maketitle{}
\author{Huxley}
\date{\today}
\title{Essay One Planning Attempt Two}
\hypersetup{
 pdfauthor={Huxley},
 pdftitle={Essay One Planning Attempt Two},
 pdfkeywords={},
 pdfsubject={},
 pdfcreator={Emacs 27.2 (Org mode 9.4.4)}, 
 pdflang={English}}
\begin{document}

\maketitle
\noindent\rule{\textwidth}{0.5pt}

\section{\[EOPA2\]}
\label{sec:orga837ed1}
\subsubsection{\[This\ time\ hopefully\ not\ as\ much\ as\ a\ dumpster\ fire.\]}
\label{sec:org116e665}
\begin{itemize}
\item Options:

\begin{itemize}
\item Prove one reading self inconsistent
\item Prove one reading inconsistent with the primary sources
\item Find a deeper, fundamental disagreement and point it out.
\end{itemize}

\item Primary source notes:

\begin{itemize}
\item Ambassador (of holy roman empire to the ottoman empire) memior's

\begin{itemize}
\item \begin{quote}
No distinction is attached to birth among the Turks
\end{quote}

\item Merit based system

\item Says that this is the reason that the Turks are successful in
their undertakings.
\end{itemize}

\item Ottoman sultan's letter to leader of Safavid Persia to justify war

\begin{itemize}
\item Describes his titles and parents
\item Basically says that they don't follow the Quran and now they are
going to war against them
\end{itemize}

\item Elite Court-born Ottoman travelogue for educated ottomans

\begin{itemize}
\item Says that the ottomans sultan created the gun-foundry which
Bayazit II enlarged
\item Struggled in war against the Holy roman empire for 36 years, way
longer than all other wars
\item Says the Romans had great artillery, but Sultan Suleyman was able
to overtake them

\begin{itemize}
\item \begin{quote}
by recruiting gunners and artillerymen from all countries with
the offer of rich rewards
\end{quote}
\end{itemize}

\item Destroyed the old gun foundry and replaced it with a new one
\item Viewed as a testament to human strength and intelligence
\end{itemize}

\item British diplomat analysis / survey of ottoman empire

\begin{itemize}
\item Says that the Turks were once formidable not because of numbers
but because of their "military and civil institutions, far
surpassing those of their opponents"

\item \begin{quote}
Conquest was to them a passion
\end{quote}

\item Says that the turks are seditous

\item \begin{quote}
Mob assembled rather than an army levied
\end{quote}

\item Says they have a bad navy
\end{itemize}
\end{itemize}
\end{itemize}

\begin{itemize}
\item Others Notes quick sum

\begin{itemize}
\item Why the ottomans succeeded

\begin{itemize}
\item Control of silk road
\item Landmass
\item Strong Military power
\end{itemize}

\item Fall

\begin{itemize}
\item Over-expanded

\begin{itemize}
\item Centralized
\end{itemize}

\item Switched to an "Iron Fist" management style of crushing
dissidents, encouraging the Persians to ally with the Europeans to
crush the Ottomans -- Jack
\item Government

\begin{itemize}
\item became to Despotic, orthodox, conservative, bureaucratic
\item Internal Plundering by the government

\begin{itemize}
\item High taxes, bribery, property seizures, ect.
\end{itemize}
\end{itemize}
\end{itemize}
\end{itemize}
\end{itemize}

\begin{itemize}
\item Kennedy

\begin{itemize}
\item Rise / Strengths

\begin{itemize}
\item says that the ottomans threats and wars seemed part of an coherent
grand strategy and the Europeans were disjointed and sporadic \{p4\}
\item Early 16th century china turned in on itself, but the ottomans did
not. In middle staged of expansion
\item Ottomans were the greatest muslim threat to Europe becuase of
their army and their superior seige train. \{p9\}
\item Applied pressure to europe \{p9\}
\item Had a great navel power, won a bunch of battles, raided a bunch of
places with their navy.
\item Had an offical fath, culture, and language over an area greater
than the romans.
\item Were way more advanced in tech and culture
\item Large tolererance of other races led to influx of talented people
\{p10\}
\end{itemize}

\item Fall / Weaknesses

\begin{itemize}
\item Eventually turned inward

\item Hard for army to expand due to immense cost

\item Ottoman imperialism wasn't that profitable

\item second half of 16th century, showed signs of "strategical
over-extenstion" \{p11\}

\item Shi'ite kingdom was prepared to ally with the Europeans against
the Ottomans

\item Needed good leadership, but after 1566, there was 13 incompetent
Sultans in a row.

\item Centralized, despotic, "orthodix in its attitude towards
initiative, disent, and commerce"

\item \begin{quote}
An idiot sultan could paralyze the Ottoman empire in the way
that a pope or Holy Roman emperor could never do for all Europe.
\{p12\}
\end{quote}

\item \begin{quote}
Without clear directives from above, the arteries of the
bureaucracy hardened, preferring conservatism to change, and
stifling innovation.
\end{quote}

\item Poverty -> internal plundering

\begin{itemize}
\item Lack of expanstion and hence riches combined with the "vast rise
in prices" caused janissaries to "turn to internal plunder"
\item Merchants and entrapanuers were met with unpredictable tax rates
and "outright seuizure of property"
\item Soldiers raded peasants land, peasants also turned to
plundering, eveerything went downhill.
\end{itemize}

\item Shi'ite religions made officials crack down on free thought

\begin{itemize}
\item Printing press was forbidden

\item \begin{quote}
Economic notions remained primitive
\end{quote}

\begin{itemize}
\item Imports desired, but exports were forbidden
\item Didn't like innovation or rise of capitalism
\item Religions didn't like traders.
\end{itemize}

\item Kept old methods of dealing with plagues, and suffered from more
epidemics due to it.

\item \begin{quote}
Their armed services had become, indeed, a bastion of
conservatism.
\end{quote}
\end{itemize}
\end{itemize}

\item Main Ideas:

\begin{itemize}
\item Infighting

\begin{itemize}
\item expansion

\begin{itemize}
\item Iron first tactic of crushing others led to them uniting
against the ottomans
\item Harder to keep expanding, imperialism was no longer profitable
\item Needed good leadership / new direction, but heavy
centralization allowed for a single "idiot" leader to stall
the empire completely. This happened thirteen times in a row.
\item This made bureaucracy harden, which led to a culture of
conservatism
\item Lack of income lead to infighting and plundering
\item Threat from Shi'ite religions led to cracking down on free
thought

\begin{itemize}
\item Stifled innovation + income\\
\item More plagues
\end{itemize}
\end{itemize}
\end{itemize}
\end{itemize}
\end{itemize}

\item Bulliet

\begin{itemize}
\item \begin{quote}
These periods of change reveal the problems faced by huge, land
based empires around the world
\end{quote}

\item Rise

\begin{itemize}
\item Grew because of: \{486\}

\begin{itemize}
\item The shrewdness of its founders and their descendents
\item Control of a strategic link between Europe and asia
\item Army that took advantage of the traditional skills of the
turkish cavalryman presented by gunpowder and christian prisoner
of war
\end{itemize}

\item Navy was helpful\ldots{}? Had a weak navy\ldots{}?\\
\item Late 1400s, got christian slaves to use as a valuable resource
\item Taxed male children for warriors \{p489\}
\end{itemize}

\item Fall

\begin{itemize}
\item Crisis of the military state

\begin{itemize}
\item Newer tech -> greater importance of cannons and light weight
fire arms

\item late 16th century, influx of silver led to inflation,
landholders couldn't report for military duty \{490\}

\item Canvalrymen reduced / put out of buisness, replaced with
janissary corps.

\item Also scholars suffered from reduced income\\

\item Cannot fundementaly alter tax system due to religous law.

\item Government recruited short term soldies which were out of money
when the campaign ended

\item \begin{quote}
Former landholding cavalrymen, short-term soldiers released at
the end of a campaign, peasants overburdened by emergency
taxes, and even impoverished students of religion formed bands
of marauders.
\end{quote}
\end{itemize}

\item Economic change and growing weakness

\begin{itemize}
\item Kept sultans confined to the palace so they woudnt start coups

\begin{itemize}
\item led to them not being experienced with the real world
\end{itemize}

\item Janissaries used their increased power to make privliges in
their corps hereditary
\end{itemize}

\item Inflation due to a massive influx of silver hit people with fixed
incomes hard

\begin{itemize}
\item Such as, cavalrymen holding land grants
\item Students on fixed scholarships \{\textasciitilde{}493\}
\end{itemize}

\item Army was weakening, clear by the middle of the 18th century

\item Trade agreements led to the Europeans dominating the Ottomans in
seaborne trade \{494\}

\item Tulip period\ldots{}?

\item Central governments weakness allowed smaller leaders to fragment
the nation.

\item \begin{quote}
Although no region declared full independence, the sultan's
power was slipping away to the advantage of a broad array of
lower officials and upstart chieftains in all parts of the
empire while the Ottoman economy was reorienting itself toward
Europe.
\end{quote}
\end{itemize}

\item Main Ideas

\begin{itemize}
\item Inflation

\begin{itemize}
\item Influx of silver led to soldiers and students with fixed
salaries starving
\item Only mention of conservatism (probs): coudn't fundamentally
change tax system due to religious law
\end{itemize}

\item Formed bands of marauders
\item Trade agreements allowed the Europeans to dominate in seaborne
Trade
\item Central government allowed smaller leaders to fragment the nation
\end{itemize}
\end{itemize}
\end{itemize}

\begin{enumerate}
\item Disagreements
\label{sec:org9b89359}
\begin{itemize}
\item Bulliet doesnt mention stress caused by nations united agaisnt the
ottomans
\item Bul doesnt talk about overexpanstion
\item Bul doesnt talk about beurocracy or culture of orthodoxy
\item Bul doesnt talk about cracking down on free thought and innovation
\item Kennedy glossed over inflation
\item Kennedy doesnt mention trade agreements
\item Kennedy doesnt talk about fragmentation of the nation
\end{itemize}
\end{enumerate}
\end{document}
