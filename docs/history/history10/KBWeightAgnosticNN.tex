% Created 2021-09-27 Mon 12:00
% Intended LaTeX compiler: xelatex
\documentclass[letterpaper]{article}
\usepackage{graphicx}
\usepackage{grffile}
\usepackage{longtable}
\usepackage{wrapfig}
\usepackage{rotating}
\usepackage[normalem]{ulem}
\usepackage{amsmath}
\usepackage{textcomp}
\usepackage{amssymb}
\usepackage{capt-of}
\usepackage{hyperref}
\setlength{\parindent}{0pt}
\usepackage[margin=1in]{geometry}
\usepackage{fontspec}
\usepackage{svg}
\usepackage{cancel}
\usepackage{indentfirst}
\setmainfont[ItalicFont = LiberationSans-Italic, BoldFont = LiberationSans-Bold, BoldItalicFont = LiberationSans-BoldItalic]{LiberationSans}
\newfontfamily\NHLight[ItalicFont = LiberationSansNarrow-Italic, BoldFont       = LiberationSansNarrow-Bold, BoldItalicFont = LiberationSansNarrow-BoldItalic]{LiberationSansNarrow}
\newcommand\textrmlf[1]{{\NHLight#1}}
\newcommand\textitlf[1]{{\NHLight\itshape#1}}
\let\textbflf\textrm
\newcommand\textulf[1]{{\NHLight\bfseries#1}}
\newcommand\textuitlf[1]{{\NHLight\bfseries\itshape#1}}
\usepackage{fancyhdr}
\pagestyle{fancy}
\usepackage{titlesec}
\usepackage{titling}
\makeatletter
\lhead{\textbf{\@title}}
\makeatother
\rhead{\textrmlf{Compiled} \today}
\lfoot{\theauthor\ \textbullet \ \textbf{2021-2022}}
\cfoot{}
\rfoot{\textrmlf{Page} \thepage}
\renewcommand{\tableofcontents}{}
\titleformat{\section} {\Large} {\textrmlf{\thesection} {|}} {0.3em} {\textbf}
\titleformat{\subsection} {\large} {\textrmlf{\thesubsection} {|}} {0.2em} {\textbf}
\titleformat{\subsubsection} {\large} {\textrmlf{\thesubsubsection} {|}} {0.1em} {\textbf}
\setlength{\parskip}{0.45em}
\renewcommand\maketitle{}
\author{Huxley}
\date{\today}
\title{Weight Agnostic Neural Networks}
\hypersetup{
 pdfauthor={Huxley},
 pdftitle={Weight Agnostic Neural Networks},
 pdfkeywords={},
 pdfsubject={},
 pdfcreator={Emacs 28.0.50 (Org mode 9.4.4)}, 
 pdflang={English}}
\begin{document}

\tableofcontents

\#ref \#ret

\noindent\rule{\textwidth}{0.5pt}

\section{Weight Agnostic Neural Networks}
\label{sec:org53d7881}
\emph{Or WANNN, for short.}

My personal note's on
\href{https://towardsdatascience.com/weight-agnostic-neural-networks-fce8120ee829}{this
article}, and also
\href{https://ai.googleblog.com/2019/08/exploring-weight-agnostic-neural.html}{this
article}.

\noindent\rule{\textwidth}{0.5pt}

Animals can perform tasks when they are born without prior experience to
the world. If the brain is pre-wired, then learning new from experience
would cause a loss of the old skill. What gives?

WANNs can perform tasks regardless of the weights in its connections by
operating off of a pre-made structure.

Also, finding structures with inductive bias is hard and slow!

\subsection{NEAT}
\label{sec:org9f609fe}
\emph{NeuroEvolution of Augmented Topologies}

Genetic algorithm in which mutations are done by changing the
\textbf{structure} of the network.

\subsection{Back to WANN}
\label{sec:orgeeb6068}
Can generalize the network to work with a range of weight values?

Instead of changing connection weights, they

\begin{itemize}
\item add connections,
\item add weight,
\item change activation functions.
\end{itemize}

\begin{quote}
Networks in which the structure enables the task to be completed, not
the weights, can be developed.
\end{quote}

\subsubsection{Finding WANNs}
\label{sec:orga73bf3c}
Start with a small amount of network architectures then use NEAT on
them. With this system of growth and training, easier and more efficient
to optimize models across a wide range of input values.

Also optimizes for less complexity in the network.

Is general, but not as good at the individual scenarios as normal
networks.

\textbf{Proposed: WANN as starting point, then run normal training on the
network}

Also allows for much easier training as the structure generally has a
lot less nodes as it is specialized for a certain task.

Analogous to how animals learn.

Can also copy WANN network, and individually train, then use them in
collections for different input values?

\subsection{So?}
\label{sec:org6f1babb}
WANNs make models more interpretable, as their solutions or logic is
encoded directly into their structure.

More general, and deals better with varying inputs.

Also allows us to encode 'intellegence' from the creation of of the
network

Can be used to find 'building blocks,' sort of like automating the
finding of revolutionary structures like CNNs.
\end{document}
