% Created 2021-09-11 Sat 16:40
% Intended LaTeX compiler: xelatex
\documentclass[letterpaper]{article}
\usepackage{graphicx}
\usepackage{grffile}
\usepackage{longtable}
\usepackage{wrapfig}
\usepackage{rotating}
\usepackage[normalem]{ulem}
\usepackage{amsmath}
\usepackage{textcomp}
\usepackage{amssymb}
\usepackage{capt-of}
\usepackage{hyperref}
\usepackage[margin=1in]{geometry}
\usepackage{fontspec}
\usepackage{indentfirst}
\setmainfont[ItalicFont = LiberationSans-Italic, BoldFont = LiberationSans-Bold, BoldItalicFont = LiberationSans-BoldItalic]{LiberationSans}
\newfontfamily\NHLight[ItalicFont = LiberationSansNarrow-Italic, BoldFont       = LiberationSansNarrow-Bold, BoldItalicFont = LiberationSansNarrow-BoldItalic]{LiberationSansNarrow}
\newcommand\textrmlf[1]{{\NHLight#1}}
\newcommand\textitlf[1]{{\NHLight\itshape#1}}
\let\textbflf\textrm
\newcommand\textulf[1]{{\NHLight\bfseries#1}}
\newcommand\textuitlf[1]{{\NHLight\bfseries\itshape#1}}
\usepackage{fancyhdr}
\pagestyle{fancy}
\usepackage{titlesec}
\usepackage{titling}
\makeatletter
\lhead{\textbf{\@title}}
\makeatother
\rhead{\textrmlf{Compiled} \today}
\lfoot{\theauthor\ \textbullet \ \textbf{2021-2022}}
\cfoot{}
\rfoot{\textrmlf{Page} \thepage}
\titleformat{\section} {\Large} {\textrmlf{\thesection} {|}} {0.3em} {\textbf}
\titleformat{\subsection} {\large} {\textrmlf{\thesubsection} {|}} {0.2em} {\textbf}
\titleformat{\subsubsection} {\large} {\textrmlf{\thesubsubsection} {|}} {0.1em} {\textbf}
\setlength{\parskip}{0.45em}
\renewcommand\maketitle{}
\author{Exr0n}
\date{\today}
\title{Unit 1 Essay Planning - Exr0n}
\hypersetup{
 pdfauthor={Exr0n},
 pdftitle={Unit 1 Essay Planning - Exr0n},
 pdfkeywords={},
 pdfsubject={},
 pdfcreator={Emacs 27.2 (Org mode 9.4.4)}, 
 pdflang={English}}
\begin{document}

\maketitle
\#flo \#disorganized \#incomplete

\section{Prompt}
\label{sec:org1bfe38d}
\section{Planning}
\label{sec:orgfc8108c}
\begin{itemize}
\item bulliet, maybe a little bit of gelvin for ottoman silver

\item gelvin talks about silver trade a little, 17th century collapse was
because silver

\item mann talks about silver on spain

\item why did china react the way they did and how did it affect them?

\item as you look at differing responses, might pull in how silver trade
relates to creatinog of trade based vs territory based empires

\begin{itemize}
\item or similarities and differences and those become the paragraphs
\item maybe everyone reacted differently and some were screwwed less
because silver but more blank (centralization, conservatism)
\end{itemize}

\item ways different

\begin{itemize}
\item differnt trade treaties
\item maybe silver made different impacts on their economic system
\end{itemize}

\item next steps

\begin{itemize}
\item think about how ottoman vs ming silver reaction differed
\end{itemize}
\end{itemize}

\section{Literally All the Things}
\label{sec:org42c2c1b}
\subsection{Ottomans}
\label{sec:org9c6b554}
\subsubsection{Ottoman military replaced cavalry with Janissaries to save money and}
\label{sec:orge1b43ad}
be more effective, but it backfired
:CUSTOM\textsubscript{ID}: ottoman-military-replaced-cavalry-with-janissaries-to-save-money-and-be-more-effective-but-it-backfired
\#source bulliet - Ottoman military was originally made of cavalry and
Janissaries - Cavalry were paid with land, while Janissaries were paid
with money. Janissaries were also more effective under devshirme - to
regain land, government squeezed out cavalry - Inflation from south
american silver also caused fixed currency peoples to struggle - all
this led to revolt (cavalry + suddenly poor people) - Janissaries took
advantage to make "Janissication" hereditary and marry/start
businesses - Revolts + Ended up costing more -> government struggling

\subsubsection{Ottomans didn't produce their own metals, so it all got traded away}
\label{sec:org4676f72}
\#source bulliet - European silver created a wave of inflation, where
european traders had more silver than ottoman internal merchants -
Safavid Iran also needed precious metals so they traded with the
ottomans for it, which meant the ottomans had even hard money - Some use
of copper coins which were inflation resistant?

\subsubsection{"penetration of European merchant capital"}
\label{sec:org7b423f7}
\#source cleveland - ottoman merchants traded raw materials for european
manufactured products - benefited merchants, hurt government revenues -
lack of raw materials caused inflation - This meant government couldn't
pay military

\subsubsection{Capitulations}
\label{sec:org6322380}
\#source cleveland - treaties originally meant to encourage international
trade - later exploited by european merchants with stronger military to
back them up

\begin{enumerate}
\item First Capitulation
\label{sec:orgdf0308c}
\begin{itemize}
\item Signed with france in 1536

\begin{itemize}
\item french merchants could trade with low taxes, etc in ottoman ports
\item merchants would be punished under french law
\end{itemize}

\item See example: \#source primary-franc-ottoman-treaty
\end{itemize}
\end{enumerate}

\subsubsection{Government weakening}
\label{sec:org82352e5}
\#source cleveland - Caused by inflation - government workers were on a
fixed wage - when in inflation happened, they were not paid enough -
were more likely to take bribes and "other forms of corruption" - lead
to declining military -> embarrassing treaties -> Treaty of Küchük
Kay-narja (1774) -> Russia meddling with ottoman politics -
"technological advantage lost"

\subsection{Ming}
\label{sec:org405fbe3}
\subsubsection{Smuggled Silver}
\label{sec:org55825dc}
\#source mann homogenocene - The conflict - Spanish government wanted all
silver in spain - trading silver and gold in china was more profitable
than anywhere else - disagreement - Official sources say 25\% silver was
smuggled to china - Historians assumed no more 10\% - New research argues
that smuggling took maybe half of silver to china - implications - was
the economy driven by european expansion or chinese demand? \#\#\# Money
Troubles \#source mann ming trade - unreliable state currency lead to
merchants using hunks of silver - bronze coins too cheap and not enough
to go around - paper money easily inflated - new emperors kept outlawing
old coins and creating new ones that immediately destabilized -
merchants started using small silver ingots

\begin{enumerate}
\item Silver as money
\label{sec:org652369e}
\begin{itemize}
\item Bowls 1-4 inches in diameter
\item purity certified by "kanyinshi"

\begin{itemize}
\item who regularly cheated all parties
\end{itemize}

\item money not issued by government

\begin{itemize}
\item privatized money
\item very not conservative
\end{itemize}

\item basically all transactions + taxes silver by 1570
\end{itemize}

\item Lack of Silver
\label{sec:orge770e08}
\begin{itemize}
\item Chinese silver mines ran dry, needed source from other places

\begin{itemize}
\item japan, kinda
\item wokou, traded a bit of silver for many goods (much silk and
porcelain)
\end{itemize}

\item wokou silver -> business people -> taxes -> government -> military ->
attacks on wokou

\begin{itemize}
\item "The ming government was at war with its own money supply"
\end{itemize}

\item Thus, the government had to allow the Fujianese traders

\begin{itemize}
\item they went around, and found boats of money in the Philippines
\end{itemize}
\end{itemize}
\end{enumerate}

\subsubsection{silver trade primary sources}
\label{sec:orgca32c35}
\#source silver trade dbq - Doc2: lots of silver was traded through the
Philippines - Doc3: silver is scarce (economic deflation) - Doc4:
portuguese trade silver for goods in China, goods for silver in japan -
Doc5: more strict economics with silver currency - Doc6: spanish silver
production - \textbf{silver mine discovered in 1545} - by spanish records,
326000000 coins - \textbf{does not include smuggling!} - Doc7: ming official
asks to repeal trade ban for much profit - Doc8: England trades china
silver for "luxary items" that are of no use, but can't stop because
someone else would take over trade and just sell to them at higher
price.

\subsection{Kennedy Bashing on People}
\label{sec:org938a202}
\subsubsection{Ming}
\label{sec:orgabcbdc4}
\begin{itemize}
\item "sheer conservatism of the Confucian bureaucracy"
\item "did not reassess" when people appeared off the coast
\item "dislike of commerce and private capital"
\item restricted technology (canals, clocks, printing)
\item spent money on land and education instead of technology
\end{itemize}

\subsubsection{Ottomans}
\label{sec:org6bf86b5}
\begin{itemize}
\item "falter, to turn inward, and to lose the chance of world domination"
\item strategical overextension
\item many enemy states around (after crazy fast expansion)
\item incompetent sultans in succession

\begin{itemize}
\item "an idiot sultan could paralyze the ottoman empire in the way that a
pope or holy roman emperor could never do for all europe"
\end{itemize}

\item lack of expansion after 1550 caused Janissaries to turn to internal
plunder
\item merchants (many foreign) may be seized
\end{itemize}

\subsubsection{Similarities in Kennedy's eyes}
\label{sec:org68a9de0}
\begin{itemize}
\item conservative
\item stopped military expansion/exploration
\item incompetent leaders
\item doesn't like merchants
\end{itemize}

\section{Outlining}
\label{sec:org0cb0f95}
\subsection{Thesis Ideas}
\label{sec:org8d11d2c}
"kennedy said that the ming and the ottomans suffered the same downfall,
and while they did both ultamately struggle due to spanish silver
inflation and european traders, the inflationary loop started with
emperors in ming china while the ottomans just kinda got stomped +
janissaries weren't vere patriotic"

Although the economies of both the Ottoman and Ming empires suffered due
to spiraling inflation and European trade, their misfortunes were not as
similar as Kennedy suggests: the Ottomans' overstretched military was
undermined by Europeans trading silver while the Mings' internal
inflation spiral forced trade with and ultimately destruction by
Europeans.

\subsection{Body 1 (choice A)}
\label{sec:org1ec4998}
\subsubsection{Topic}
\label{sec:org6492663}
Kennedy said the ming and the ottomans suffered the same downfall due to
centralization and economic troubles.

\subsubsection{Evidence}
\label{sec:orgfa4857a}
\begin{itemize}
\item "[the ottomans] were to falter [\ldots{}] strikingly similar Ming decline"
(Kennedy 11)
\item "The system as a whole, like that of Ming China, increasingly suffered
from some of the defects of being centralized, despotic, and severely
orthodox in it's attitude toward initiative dissent, and commerce."
(Kennedy 11)
\item "dislike trade" similarities

\begin{itemize}
\item "Merchants ant entrepreneurs (nearly all of whom were foreigners),
who earlier had been encouraged, now found themselves subject to
unpredictable taxes and outright seizures of property" (Kennedy 12)
\item "The mandarins [had] a suspicion of trader" (Kennedy 8)
\item "[The mandarins] dislike of commerce and private capital [\ldots{}]"
(Kennedy 8)
\end{itemize}
\end{itemize}

\subsection{Body 1 (choice B)}
\label{sec:org25d7b5d}
\subsubsection{Topic}
\label{sec:org2706f1a}
Both the Ming and Ottoman empires suffered from economies weakened by
instability and revolt caused by inflation.

\subsubsection{Evidence}
\label{sec:orgf5afabe}
\begin{itemize}
\item ”
\end{itemize}

\subsection{Body 2}
\label{sec:org791034c}
\subsubsection{Topic}
\label{sec:orgfd186e4}
\subsubsection{Evidence}
\label{sec:orgbd37619}
\subsection{Body 3}
\label{sec:orgfd678ba}
\subsubsection{Topic}
\label{sec:orgec629e4}
\subsubsection{Evidence}
\label{sec:org629c51b}
\section{Sources}
\label{sec:orgd169212}
\begin{itemize}
\item Bulliet Ottomans
\item Mann ming trade
\item Mann homogenocene
\end{itemize}

\noindent\rule{\textwidth}{0.5pt}
\end{document}
