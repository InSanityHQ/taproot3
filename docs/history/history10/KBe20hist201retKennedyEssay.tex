% Created 2021-09-27 Mon 12:00
% Intended LaTeX compiler: xelatex
\documentclass[letterpaper]{article}
\usepackage{graphicx}
\usepackage{grffile}
\usepackage{longtable}
\usepackage{wrapfig}
\usepackage{rotating}
\usepackage[normalem]{ulem}
\usepackage{amsmath}
\usepackage{textcomp}
\usepackage{amssymb}
\usepackage{capt-of}
\usepackage{hyperref}
\setlength{\parindent}{0pt}
\usepackage[margin=1in]{geometry}
\usepackage{fontspec}
\usepackage{svg}
\usepackage{cancel}
\usepackage{indentfirst}
\setmainfont[ItalicFont = LiberationSans-Italic, BoldFont = LiberationSans-Bold, BoldItalicFont = LiberationSans-BoldItalic]{LiberationSans}
\newfontfamily\NHLight[ItalicFont = LiberationSansNarrow-Italic, BoldFont       = LiberationSansNarrow-Bold, BoldItalicFont = LiberationSansNarrow-BoldItalic]{LiberationSansNarrow}
\newcommand\textrmlf[1]{{\NHLight#1}}
\newcommand\textitlf[1]{{\NHLight\itshape#1}}
\let\textbflf\textrm
\newcommand\textulf[1]{{\NHLight\bfseries#1}}
\newcommand\textuitlf[1]{{\NHLight\bfseries\itshape#1}}
\usepackage{fancyhdr}
\pagestyle{fancy}
\usepackage{titlesec}
\usepackage{titling}
\makeatletter
\lhead{\textbf{\@title}}
\makeatother
\rhead{\textrmlf{Compiled} \today}
\lfoot{\theauthor\ \textbullet \ \textbf{2021-2022}}
\cfoot{}
\rfoot{\textrmlf{Page} \thepage}
\renewcommand{\tableofcontents}{}
\titleformat{\section} {\Large} {\textrmlf{\thesection} {|}} {0.3em} {\textbf}
\titleformat{\subsection} {\large} {\textrmlf{\thesubsection} {|}} {0.2em} {\textbf}
\titleformat{\subsubsection} {\large} {\textrmlf{\thesubsubsection} {|}} {0.1em} {\textbf}
\setlength{\parskip}{0.45em}
\renewcommand\maketitle{}
\author{Exr0n}
\date{\today}
\title{Kennedy Essay Planning}
\hypersetup{
 pdfauthor={Exr0n},
 pdftitle={Kennedy Essay Planning},
 pdfkeywords={},
 pdfsubject={},
 pdfcreator={Emacs 28.0.50 (Org mode 9.4.4)}, 
 pdflang={English}}
\begin{document}

\tableofcontents

\#ret

\section{Prompt}
\label{sec:orgff69c94}
\begin{verbatim}
Essay 1: Kennedy and Mann on Ming Decline



Directions: In Chapter 1 of Rise and Fall of the Great Powers, Paul Kennedy sketches out an explanation of why the Ming Dynasty was, on the one hand powerful and prosperous, but ultimately was “a country which had turned in on itself” and subject to “steady relative decline.” Mann, in his chapter on the Ming trade, gives the reader a lot more detail on the nuances of Ming history in this period. Putting Kennedy and Mann into dialogue, does Kennedy’s argument still hold up? In your essay, argue for or against Kennedy’s argument using the details of Ming history analyzed by Mann.



A strong essay will clearly describe Kennedy’s argument and link it to specific pieces of supporting or challenging evidence from Mann.   



Additional sources: In addition to Kennedy and Mann, you may OPTIONALLY use and cite from any of the primary sources below, as well as the more extensive versions in the Drive. This will give you a chance to demonstrate the “exemplary” level of the Use of Evidence standard (which requires a “diverse variety of sources”), but is not required for any other standard. When using these sources, keep their authorship and context in mind. 


Citations: Direct quotations as well as paraphrasing from any of the sources should be cited with a simple footnote citation with author and page number (if available). Such as: (Kennedy 23) or (Chelebi).

Length: 350-750 words

Format: 12 pt font, double spaced, double-sided if possible, with name on at least the first page.



Wang Xijue, Ming dynasty court official, report to the emperor, 1593.

The venerable elders of my home district explain that the reason grain is cheap despite poor harvests in recent years is due entirely to the scarcity of silver coin.  The national government requires silver for taxes but disburses little silver in its expenditures.  As the price of grain falls, tillers of the soil receive lower returns on their labors, and thus less land is put into cultivation.



Huang Zongxi, late Ming dynasty scholar who fought against eunuch rule of the court and the Manchu invasion, writes about the need for a prime minister in “Waiting for the Dawn”, after the Manchu conquest in 1662.

The origin of misrule under the Ming lay in the abolition of the prime ministership by [the Ming founder] Gao Huangdi. […] It may be argued that in recent times matters of state have been discussed in cabinet, which actually amounted to having prime ministers, even though nominally there were no prime ministers.  But this is not so.  The job of those who handled matters in the cabinet has been to draft comments of approval and disapproval just like court clerks. […] 

I believe that those with the actual power of prime ministers today are the palace menials [eunuchs]. Final authority always rests with someone, and the palace menials, seeing the executive functions of the prime minister fall to the ground, undischarged by anyone, have seized the opportunity to establish numerous regulations, extend the scope of their control, and take over from the prime minister the power of life and death.



Zhang Han (1510-1593), was a Ming official who writes Songchuang Meng Yu (松窗夢語) during his retirement.  This is from a chapter, “On Trade.”

As to the foreign trade on the northwestern frontier and the foreign sea trade in the southeast, if we compare their advantages and disadvantages with respect to our nation’s wealth and the people’s well-being, we will discover that they are as different as black and white. But those who are in charge of state economic matters know only the benefits of the Northwest trade, ignoring the benefits of the sea trade. How can they be so blind?

[Northwestern] Foreigners are recalcitrant and their greed knows no bounds. […] I do not think our present trade with them will ensure us a century of peace.

As to the foreigners in the southeast, their goods are useful to us just as ours are to them. To use what one has to exchange for what one does not have is what trade is all about. Moreover, these foreigners trade with China under the name of tributary contributions. That means China’s authority is established and the foreigners are submissive. Even if the gifts we grant them are great and the tribute they send us is small, our expense is still less than one ten-thousandth of the benefit we gain from trading with them.



The Salt and Iron Debates from 81 AD documented Confucian scholars’ critique of the government’s trade monopolies. This represents the traditional Confucian stance on trade.

The Confucian learned men: The purpose of merchants is circulation and the purpose of artisans is making tools. These matters should not become a major concern of the government.

At present the government ignores what people have and exacts what they lack. The common people then must sell their products cheaply to satisfy the demands of the government. […]

The government officers busy themselves with gaining control of the market and cornering commodities. With the commodities cornered, prices soar and merchants make private deals and speculate. The officers connive with the cunning merchants who are hoarding commodities against future need. Quick traders and unscrupulous officials buy when goods are cheap in order to make high profits. Where is the balance in this standard?



Also in the Google Drive:

Cook_ZhengHe.pdf goes into a bit more detail about Zheng He’s journeys
Brook_Zhang Han article.pdf is a few pages from an article about Zhang Han’s document that discusses various theories for Ming trade policies
Ropp_Ming.pdf has more historical detail about the Ming dynasty
Ropp_Qing.pdf has Qing dynasty events if you want to use it to contrast with Ming
Zhang Han-Ming Trade.pdf is a fuller primary source with his critique of Ming trade policies


Getting started:

Lay out Kennedy’s argument, and decide which aspects you agree with, disagree with, or want to complicate.  What evidence from Mann, et al, can support you? 

Kennedy’s argument is _______, and he is wrong in ____, ____, and _____
Kennedy’s argument is _______, and can be supported by ____, ____, ____
Kennedy’s argument is _______, and while he is right in _______, he is wrong in _____ and can be more nuanced in ______
\end{verbatim}

\section{Outline}
\label{sec:orga142bcb}
\subsection{Kennedy's Argument}
\label{sec:orgb563331}
\begin{quote}
There was, to be sure, a plausible strategical reason for this
decision. The northern frontiers of the empire were again under some
pressure from the Mongols, and it may have seemed prudent to
concentrate military resources in this more vulnerable area. Under
such circumstances a large navy was an expensive luxury, and in any
case, the attempted Chinese expansion southward into Annam (Vietnam)
was proving fruitless and costly. Yet this quite valid reasoning does
not appear to have been reconsidered when the disadvantages of naval
retrenchment later became clear: within a century or so, the Chinese
coastline and even cities on the Yangtze were being attacked by
Japanese pirates, but there was no serious rebuilding of an imperial
navy. Even the repeated appearance of Portuguese vessels off the China
coast did not force a reassessment. Defense on land was all that was
required, the mandarins reasoned, for had not all maritime trade by
Chinese subjects been forbidden in any case? Apart from the costs and
other disincentives involved, therefore, a key element in China's
retreat was the sheer conservatism of the Confucian bureaucracy---a
conservatism heightened in the Ming period by resentment at the
changes earlier forced upon them by the Mongols. In this "Restoration"
atmosphere, the all-important officialdom concerned to preserve and
recapture the past, not to create a brighter future based upon
overseas expansion and commerce. According to the Confucian code,
warfare itself was a deplorable activity and armed forces were made
necessary only by the fear of barbarian attacks or internal revolts.
The mandarins' dislike of the army (and the navy) was accompanied by a
suspicion of the trader. The accumulation of private capital, the
practice of buying cheap and selling dear, the ostentation of the
nouveau riche merchant, all offended the elite, scholarly
bureaucrats---almost as much as they aroused the resentments of the
toiling masses. While not wishing to bring the entire market economy
to a halt, the mandarins often intervened against individual merchants
by confiscating their property or banning their business. Foreign
trade by Chinese subjects must have seemed even more dubious to
mandarin eyes, simply because it was less under their control.
\end{quote}

\begin{itemize}
\item Wrong

\begin{itemize}
\item "a key element in China's retreat was the sheer conservatism of the
Confucian bureaucracy"
\end{itemize}

\item Nuance

\begin{itemize}
\item "The accumulation of private capital, the practice of buying cheap
and selling dear, the ostentation of the nouveau riche merchant, all
offended the elite"
\item "Foreign trade by Chinese subjects must have seemed even more
dubious to mandarin eyes, simply because it was less under their
control."
\end{itemize}

\item Right?

\begin{itemize}
\item "warfare itself was a deplorable activity and armed forces were made
necessary only by the fear of barbarian attacks or internal
revolts."”
\end{itemize}
\end{itemize}

\subsection{Meet Sushu}
\label{sec:org592ca0e}
\begin{verbatim}
- needs more nuance: accumulation of private capital —> Mann doesn’t see foreign trade as dubious - disagree: conservative confucianism - agree: govt doesn’t want to fight - conservatism = not wanting to contact the outside world, but primary source Zhang Han shows that trade is good - true that govt doesn’t want accumulation of private capital 
From Sushu Xia to Everyone: (3:35 PM)
 - not correct: government did want to trade with foreigners
 - what about Mann’s thing about trade ban, smugglers, etc? government wanted tributary trade relationships to better control trade, didn’t want commoners to trade 
From Sushu Xia to Everyone: (3:37 PM)
 —> government wanting to control/centralize trade —> isn’t that confucian conservatism? - confucian conservatism isn’t about ending overseas trade, it’s about controlling overseas trade???? 80% agree with Kennedy? 
From Sushu Xia to Everyone: (3:41 PM)
 - disagree: “ming china is less vigorous/enterprising” and general notion of decline - disagree with cultural characterization 
From Sushu Xia to Everyone: (3:43 PM)
 - Kennedy is mostly right re: conservative confucian forces in the govt, but it’s about centralization and control, and not some inherent lack of vigor or innovation 
From Sushu Xia to Everyone: (3:44 PM)
 - the merchant pirates were quite vigorous in their pursuit of trade.   they were being stopped by the govt but still adamant for trade - Brook - central bureaucracy vs. local merchants/trade 
From Sushu Xia to Everyone: (3:46 PM)
 Kennedy is mostly right re: conservative confucian forces in the govt, but they banned trade not because they wanted to turn away from the world (Zhang Han), but because they wanted to control trade (Mann) 
From Sushu Xia to Everyone: (3:48 PM)
1) Kennedy is mostly right re: conservative confucian (govt didn’t want commoners to trade) 2) it’s not cultural vigor stuff or turning away from the world (Zhang Han, smuggling) 3) it’s because desire to control trade (Mann, brook, salt iron debates)
\end{verbatim}

Although John Kennedy correctly traces the Ming government's distaste
for ( \#todo-exr0n word choice: commoner trade) to conservative
Confucianism, he incorrectly attributes the Ming slashing of foreign
trade to a lack cultural vigor--this was instead an attempt to control
and profit.

As Kennedy notes, the Ming government banned foreign trade due to the
Confucian conservative values of government control and opposition to
foreign influence. Kennedy claims "a key element in China's retreat was
the sheer conservatism of the Confucian bureaucracy" and Mann tactfully
elaborates "all contact with the world outside was supposed to be
supervised by Beijing" (Kennedy 7, Mann 127). Although Kennedy doesn't
provide much detail on what he means by "sheer conservatism", Mann
supplement's Kennedy's argument by explaining Beijing's plan to limit
and supervise trade relations with the outside world. All sources agree
that on a high level, Confucianism's conservative values played a major
role in China's restrictive trade policies.

However, Kennedy inaccurately suggests that Confucian China stopped
trade due to a dislike of commerce and cultural complacency. Kennedy
writes "[Foreign trade] must have seemed even more dubious to mandarin
eyes", whose imprecise wording and questionable support heavily
contrasts the rest of the text; He further milks this dubious claim by
generalizing it to a "dislike of commerce" (Kennedy 8). A more nuanced
take notes counters that Beijing's slashing on private trade with
foreign parties is a means of consolidating power and centralizing
profit, instead of due to a generic distaste of trade (Mann 126). In
fact, the benefits of foreign trade were known and considered--a retired
Ming official writes "our expense is still less than one ten-thousandth
of the benefit we gain from trading with them" (Zhang Han). That the
author was retired and still actively writing about the importance of
trade with foreign powers shows that trade had not been a niche concept
but rather a generally considered and debated topic within the mandarin
court. The Chinese did not decimate public trade because they found it
distasteful--the advantages of trade were known and leveraged and the
private ban was a part of their plan. Kennedy also writes that "Ming
China was a much less vigorous and enterprising land" and gives the
impression that some cultural complacency is at fault (Kennedy 8).
However, Kennedy again assumes the claim and uses it as a transition--a
deeper analysis reveals that trade-oriented merchants joined forces with
pirates to ensure their business. At one point, they fought off the
imperial forces sent to pacify them and later invited illegal foreign
smugglers onto Chinese land (Mann 133).

Contrary to Kennedy's simplification, the Ming government banned foreign
trade to control and profit international relations through tributary
interactions. Mann elaborates on an important detail: the Chinese
allowed foreign nations to pay tribute to the emperor, who would give
them small gifts in return. The submissive nation would often be allowed
to sell extra goods to the masses, which essentially turned the scheme
into an outlet for foreign trade under strict government control (Mann
127). Internally, these visitations were thought of as assertions of
dominance because the only way foreigners could trade with China was
under the name of tributary contributions (Zhang Han). Ultimately Ming
China banned private trade with foreign parties not due to a Confucian
dislike of commerce or cultural lack of vigor, but rather as a tool to
consolidate government power and make a profit. Although historians
concur that aspects of Ming China's Confucian philosophy played a major
role in the decline of foreign interaction, Kennedy's assumption
\#todo-exr0n WC) that this was due to a dislike of trade and cultural
triumphancy is debunked by realizing the political and economic
motivations--the Ming did not ban trade but rather reframed and
restricted it for governmental advantage.

\begin{itemize}
\item "But as long as they were content to remain at the margin of
production and gather what wealth they could from the peasants, their
political subordination and their economic security were
simultaneously assured." (Brook 185)
\end{itemize}

\section{Export}
\label{sec:orge64ddc6}
\href{https://docs.google.com/document/d/1w\_NYtVPbDXAY8JT6Rc8dzNQ4TCRxdU73UmQFtB\_sRPw/edit}{Link}
\#\#\# Thesis Although John Kennedy correctly traces the Ming government's
distaste for ( \#todo-exr0n word choice: commoner trade) to conservative
Confucianism, he incorrectly attributes the Ming slashing of foreign
trade to a lack cultural vigor--this was instead an attempt to control
and profit.

\subsection{Body 1}
\label{sec:orgc6cd71e}
As Kennedy notes, the Ming government banned foreign trade for due to
the Confucian conservative values of government control and opposition
to foreign influence. - "a key element in China's retreat was the sheer
conservatism of the Confucian bureaucracy" (Kennedy 7) - "With a few
exceptions, all contact with the world outside was supposed to be
supervised by Beijing." (Mann 127)

\subsection{Body 2}
\label{sec:org6500a66}
However, Kennedy inaccurately suggests that Confucian China stopped
trade due to political and economic complacency and a dislike of
commerce. - Kennedy: "This dislike of commerce" (Kennedy 8) - "Beijing's
prohibition on private trade has less to do with an abhorrence of trade
than a desire to control it for the dynasty's benefit" (Mann 126) - "our
expense is still less than one ten-thousandth of the benefit we gain
from trading with them" (Zhang Han) - Not complacent: Trade oriented
pirate merchants beat back three hundred imperial soldiers in 1557 and
some even invited three thousand Japanese and Portuguese smugglers to
camp on Chinese territory. (Mann 133)

\subsection{Body 3}
\label{sec:orgc2736d7}
Contrary to Kennedy's simplification, the Ming government banned foreign
trade to control and profit international relations through tributary
interactions. - "the ban-and-tribute scheme for what it was: a way for
the government to control international commerce". (Mann 127) -
"Moreover, these foreigners trade with China under the name of tributary
contributions. That means China's authority is established and the
foreigners are submissive." (Zhang Han) - "But as long as they were
content to remain at the margin of production and gather what wealth
they could from the peasants, their political subordination and their
economic security were simultaneously assured." (Brook 185)

\noindent\rule{\textwidth}{0.5pt}
\end{document}
