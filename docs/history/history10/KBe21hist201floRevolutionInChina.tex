% Created 2021-09-27 Mon 11:52
% Intended LaTeX compiler: xelatex
\documentclass[letterpaper]{article}
\usepackage{graphicx}
\usepackage{grffile}
\usepackage{longtable}
\usepackage{wrapfig}
\usepackage{rotating}
\usepackage[normalem]{ulem}
\usepackage{amsmath}
\usepackage{textcomp}
\usepackage{amssymb}
\usepackage{capt-of}
\usepackage{hyperref}
\setlength{\parindent}{0pt}
\usepackage[margin=1in]{geometry}
\usepackage{fontspec}
\usepackage{svg}
\usepackage{cancel}
\usepackage{indentfirst}
\setmainfont[ItalicFont = LiberationSans-Italic, BoldFont = LiberationSans-Bold, BoldItalicFont = LiberationSans-BoldItalic]{LiberationSans}
\newfontfamily\NHLight[ItalicFont = LiberationSansNarrow-Italic, BoldFont       = LiberationSansNarrow-Bold, BoldItalicFont = LiberationSansNarrow-BoldItalic]{LiberationSansNarrow}
\newcommand\textrmlf[1]{{\NHLight#1}}
\newcommand\textitlf[1]{{\NHLight\itshape#1}}
\let\textbflf\textrm
\newcommand\textulf[1]{{\NHLight\bfseries#1}}
\newcommand\textuitlf[1]{{\NHLight\bfseries\itshape#1}}
\usepackage{fancyhdr}
\pagestyle{fancy}
\usepackage{titlesec}
\usepackage{titling}
\makeatletter
\lhead{\textbf{\@title}}
\makeatother
\rhead{\textrmlf{Compiled} \today}
\lfoot{\theauthor\ \textbullet \ \textbf{2021-2022}}
\cfoot{}
\rfoot{\textrmlf{Page} \thepage}
\renewcommand{\tableofcontents}{}
\titleformat{\section} {\Large} {\textrmlf{\thesection} {|}} {0.3em} {\textbf}
\titleformat{\subsection} {\large} {\textrmlf{\thesubsection} {|}} {0.2em} {\textbf}
\titleformat{\subsubsection} {\large} {\textrmlf{\thesubsubsection} {|}} {0.1em} {\textbf}
\setlength{\parskip}{0.45em}
\renewcommand\maketitle{}
\author{Taproot}
\date{\today}
\title{Revolution in China Stearns}
\hypersetup{
 pdfauthor={Taproot},
 pdftitle={Revolution in China Stearns},
 pdfkeywords={},
 pdfsubject={},
 pdfcreator={Emacs 28.0.50 (Org mode 9.4.4)}, 
 pdflang={English}}
\begin{document}

\tableofcontents

\section{overview}
\label{sec:orgd9275cf}
\subsection{fall of Qing}
\label{sec:orga357112}
\subsubsection{Manchu 12-yo boy emperor Puyi abducted in 1912}
\label{sec:org6fe2664}
\subsubsection{power vacuum competed for by 'regional warlords, loose alliance of students, politicians, secret societes, japanese, communist movement'}
\label{sec:org1f61322}
\subsubsection{internal divisions and foreign influences paved the way for the ult victory of Mao Zedong\hfill{}\textsc{claim}}
\label{sec:org11bf415}
\subsection{warlord cliques}
\label{sec:org262764f}
\subsubsection{dominated chinese politics for next 3 decades}
\label{sec:orgb748ec6}
\subsubsection{most powerful clique was in north china headed by Yuan Shikai}
\label{sec:org069a34d}
\subsubsection{Shanghai and Canton made second power center, favored politicians like Sun Yat-sen}
\label{sec:org9237798}
\subsection{university students, teachers, and intellectuals}
\label{sec:org27dbae0}
\subsubsection{played critical roles in civilization but were defenseless in force}
\label{sec:org14ab9dc}
\subsection{secret societies}
\label{sec:org3ebdf8c}
\subsubsection{envisioned restoration of monarchical rule under a Chinese dynasty}
\label{sec:orgbbdcd80}
\subsection{western power intervention and japan}
\label{sec:org4baad20}
\subsubsection{wanted to capitalize on power vacuum}
\label{sec:org59ddb90}
\subsubsection{japan from mid-1890s to 1945 (end of WWI)}
\label{sec:orgc063bb7}
\section{may fourth movement}
\label{sec:orgf06d015}
\subsection{Sun Yat-sen headed Revolutionary Alliance---loose combo of anti-Qing political groups that started the 1911 revolt}
\label{sec:org43a3a22}
\subsection{claimed mandate of heaven but warlords had true power}
\label{sec:orgce16453}
\subsection{set up a Parliament and elected cabinets but had minimal actual effect}
\label{sec:org20e41e6}
\subsection{Sun Yat-sen resigned in favor of Yuan Shikai in 1912 (northern warlord)}
\label{sec:orgdf39c17}
\subsubsection{Yuan Shikai pretended to be democratic but built up military}
\label{sec:org31a72c2}
\subsubsection{few years later, used military and assassinations to remove opposition}
\label{sec:org662ab7f}
\subsection{Japan and WWI}
\label{sec:orga59225e}
\subsubsection{japan took german concessions in China after WWI}
\label{sec:org2b4b366}
\subsubsection{gave Yuan the 21 demands in early 1915, which would reduce china to a 'dependant protectorate'}
\label{sec:org60b0d09}
\subsubsection{Yuan was indecisive and instead rallied support for himself instead of responding}
\label{sec:orgd3a99fd}
\begin{enumerate}
\item a rival warlord took his support by being hostile to japan and Yuan resigned in 1916
\label{sec:org95a34bd}
\end{enumerate}
\subsubsection{after the war (1919), japan won german concessions}
\label{sec:orga72ef42}
\begin{enumerate}
\item this made the students upset -> protests and mass boycotts
\label{sec:orgd239e87}
\end{enumerate}
\subsection{democracy and individualism popular in urban youth}
\label{sec:orgcfa21d3}
\subsubsection{democratic thinkers toured china}
\label{sec:org7e4086a}
\subsubsection{novel by Ba Jin depicts boy ignoring arranged marriage}
\label{sec:org5484aaf}
\subsubsection{however, elections and stuff didnt work because warlords were in control}
\label{sec:org952c67a}
\begin{enumerate}
\item so they decided more radical action was needed
\label{sec:org10efe32}
\end{enumerate}
\subsection{Bolshevik victory in Russia}
\label{sec:orgbfb8b92}
\subsubsection{chinese seriosuly considered marxism}
\label{sec:org7ec1c81}
\subsubsection{Li Dazhao decided to interpret marxism for china's situation}
\label{sec:orga3ba894}
\begin{enumerate}
\item he saw the pheasants as the vanguard of urban change
\label{sec:org448714e}
\end{enumerate}
\subsubsection{all chinese as proletarian, and bourgeois was the industrialized West (unification)}
\label{sec:orgcc483ca}
\subsection{marxist study club (including Mao Zedong)}
\label{sec:org2ff1568}
\subsubsection{also believed in authoritarian state that intervened helpfully in many aspects of life}
\label{sec:orge24ab35}
\subsection{summer of 1921}
\label{sec:orgbcd0d9b}
\subsubsection{a handful of marxist leaders from different parts of China met secretly in Shanghai}
\label{sec:org0f68bcd}
\subsubsection{Communist party of China born}
\label{sec:org1673f78}
\subsubsection{few supporters but provided new ideology over confucianism}
\label{sec:orgf16bba6}
\section{Seizure of Power by the Guomindang (nationalist party, Sun Yat-sen)}
\label{sec:orgc6b5a4c}
\subsection{promised international and domestic change, but only implemented international change}
\label{sec:org7cc939f}
\subsubsection{pushed foreigners out but didnt implement land reform which is what the pheasants cared about}
\label{sec:orged8caf5}
\subsection{slowly forged alliances with 'key social groups' and built an army in south of china}
\label{sec:orgfd2b3a9}
\subsection{nationalists used communists as major link to peasants and urban workers}
\label{sec:org109f3e5}
\subsection{also asked soviets for help}
\label{sec:org95acb59}
\subsection{soviet military academy}
\label{sec:orgd1898fa}
\subsubsection{first headed by Chiang Kai-shek who didnt like the communists}
\label{sec:org7880172}
\subsubsection{but he had to wait for the army to be trained}
\label{sec:orgff8a035}
\subsection{after Sun yat-sen dies in 1925, Chaing kai-shek captures and bribes warlords}
\label{sec:org9694064}
\subsubsection{becomes the head of a warlord hierarchy, essentially controlling china}
\label{sec:orge1c0c35}
\section{mao and the peasant option}
\label{sec:org50a1b7e}
\subsection{mao background}
\label{sec:org4c7fe21}
\subsubsection{father was a prosperous peasant, but mao rebelled early}
\label{sec:org679a002}
\subsubsection{believed revolution was violent and peasants needed to use force to overthrow landlords}
\label{sec:org18697f6}
\subsection{after Chaing seized control, he massacared communists in Shanghai in 1927}
\label{sec:org9276fe0}
\subsection{a later attack to communists in south central china caused Mao to spearhead a long march}
\label{sec:org5e28c83}
\subsubsection{90k followers in 1934, thousands of miles to the more remote northwest}
\label{sec:org6ecfa01}
\subsubsection{created a new communist center}
\label{sec:org1c735f9}
\subsection{long march solidified Mao's leadership of the Chinese communist party, but japanese eroded Chaing's power structure}
\label{sec:org64da44e}
\section{global great depression}
\label{sec:orgf6d9db9}
\subsection{a decade after WWI, caused many international crises}
\label{sec:org21e3e00}
\subsection{caused by problems in economic systems and reliance on cheap raw goods}
\label{sec:orgfc3778f}
\subsection{causation}
\label{sec:orgadc94ef}
\subsubsection{food overproduction drove down prices}
\label{sec:org76ba74a}
\begin{enumerate}
\item high prices during the war led to overconfident loans
\label{sec:org04a9a78}
\item runaway spiral of loans from the US to european countries + postwar inflation?
\label{sec:org06db47f}
\item optimized exploitation of colonies to produce coffee sugar rubber production worsened the same cycle
\label{sec:orgbd588ef}
\end{enumerate}
\subsubsection{poor leadership}
\label{sec:org6025263}
\begin{enumerate}
\item leaders were interested in their own debts being paid than facilitating balanced econ growth
\label{sec:org2bad43e}
\item protectionism reduced market opportunities and made it worse
\label{sec:orgf8b1798}
\end{enumerate}
\subsection{the formal advent of the Depression (October 1929)}
\label{sec:org88cee85}
\subsubsection{us stock market crash brings everything down}
\label{sec:orgdfb1570}
\begin{enumerate}
\item bc the US gave out so many loans
\label{sec:orgc0d392f}
\item people trying to cut losses made things worse
\label{sec:org71aa87f}
\item thus lower production levels and rising unemployment
\label{sec:org4c85d20}
\item reinforcing cycle from 1929 to 1933, even France and Italy drawn into vortex by 1931
\label{sec:org8fbee7c}
\end{enumerate}
\subsection{comparison}
\label{sec:org45dc556}
\subsubsection{first great depression of the industrial age}
\label{sec:orged7cdf2}
\subsubsection{also it was in the mess of the world wars which made it worse}
\label{sec:org632b926}
\subsubsection{and it lasted nearly a decade}
\label{sec:orgb015a15}
\subsection{social impacts}
\label{sec:org55eaf3d}
\subsubsection{created fears of loss of earnings or work}
\label{sec:orgecf91bd}
\subsubsection{confused family and gender roles}
\label{sec:org24111b8}
\subsubsection{economic hesitancy caused recessions through 1939}
\label{sec:org9f528c5}
\subsection{non-western impacts}
\label{sec:orgdd15fa6}
\subsubsection{export economies got rekt (latain america, japan)}
\label{sec:org497b5ca}
\subsection{western responses to the great depression}
\label{sec:org1427695}

\subsubsection{government tariffs, spending cuts, and inflation fear worsened the problem}
\label{sec:org5d2289e}
\subsubsection{people turned to radicalism both left and right}
\label{sec:org7ea9114}
\subsubsection{generally, parliaments were either frozen in indecision or straight up overthrown}
\label{sec:org46b998a}
\subsubsection{eg. france got frozen}
\label{sec:orgdbab0d8}
\begin{enumerate}
\item parliment reacted sluggishly, popular front (comprised of liberal, socialist, and communist parties) won election in 1936
\label{sec:org2744d33}
\item but it too got second-thoughty  and fell in 1938, but everything was really a standstill
\label{sec:org84742ea}
\end{enumerate}
\subsubsection{scandinavian states were somewhat socialist and neared a welfare state which was decent}
\label{sec:org54fb2f9}
\subsubsection{british innovation (television) improved some sectors but not enough}
\label{sec:orgefbd346}
\subsection{the new deal}
\label{sec:org4e29bd0}
\subsubsection{herbert hoover did european-like stuff that went poorly}
\label{sec:org08bfdbf}
\subsubsection{franklin roosevelt and the new deal (1930s) offered more direct aid}
\label{sec:org84b64be}
\begin{enumerate}
\item public works projects increased employment
\label{sec:orgd04b7f8}
\item social security system to form a social baseline
\label{sec:orgdc14e8c}
\item economic planning also helped
\label{sec:orga818961}
\item led to economic growth which was good, and restored faith in the US system, prevented more radical movements
\label{sec:org6e15e02}
\end{enumerate}
\subsection{militarization of japan}
\label{sec:orgd3377da}
\subsubsection{authoritarian military rule took over japan and had conquered Chinese Manchuria w/o civilian government backing by 1931}
\label{sec:orga9c370f}
\subsubsection{many parties and ideas, generally anti-western}
\label{sec:orga2388b7}
\subsubsection{in May 1932, army officers murdered the prime minister. they also put down another coup attempt in 1936, but generally prime ministers became increasingly militaristic}
\label{sec:org7ba0d9c}
\subsubsection{in 1937, a small scuffle turned into a large scale battle when japanese generals decided to defeat China's armies to prevent future trouble}
\label{sec:org4745f34}
\subsubsection{by 1938 Japan controlled a large-ish empire. military leaders and economic leaders liked this and pressed for more conquest in WWII}
\label{sec:org3475953}
\subsubsection{civilians were like brUh the entire time but the military was in control}
\label{sec:org7cb5205}
\subsection{japanese industrialization and recovery}
\label{sec:org9cb64fb}
\subsubsection{depression hit hard, but active government policies helped}
\label{sec:orge9934ee}
\begin{enumerate}
\item increased spending to boost employment and generally boosted economy
\label{sec:org4dfb2b6}
\item export boom and "virtual elimination of unemployment by 1936" after the low point in 1931
\label{sec:org95db06e}
\item lots of rapid industrialization stats
\label{sec:org24bd039}
\end{enumerate}
\subsubsection{distinctive industrial policies}
\label{sec:orgd27fc1e}
\begin{enumerate}
\item mass patriotism, group loyalt, lifetime contracts
\label{sec:org95f674f}
\end{enumerate}
\subsubsection{by 1937, japan was looking industrialized, self sustainable, and generally good}
\label{sec:org12cd1ef}
\end{document}
