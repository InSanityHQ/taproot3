% Created 2021-09-27 Mon 12:00
% Intended LaTeX compiler: xelatex
\documentclass[letterpaper]{article}
\usepackage{graphicx}
\usepackage{grffile}
\usepackage{longtable}
\usepackage{wrapfig}
\usepackage{rotating}
\usepackage[normalem]{ulem}
\usepackage{amsmath}
\usepackage{textcomp}
\usepackage{amssymb}
\usepackage{capt-of}
\usepackage{hyperref}
\setlength{\parindent}{0pt}
\usepackage[margin=1in]{geometry}
\usepackage{fontspec}
\usepackage{svg}
\usepackage{cancel}
\usepackage{indentfirst}
\setmainfont[ItalicFont = LiberationSans-Italic, BoldFont = LiberationSans-Bold, BoldItalicFont = LiberationSans-BoldItalic]{LiberationSans}
\newfontfamily\NHLight[ItalicFont = LiberationSansNarrow-Italic, BoldFont       = LiberationSansNarrow-Bold, BoldItalicFont = LiberationSansNarrow-BoldItalic]{LiberationSansNarrow}
\newcommand\textrmlf[1]{{\NHLight#1}}
\newcommand\textitlf[1]{{\NHLight\itshape#1}}
\let\textbflf\textrm
\newcommand\textulf[1]{{\NHLight\bfseries#1}}
\newcommand\textuitlf[1]{{\NHLight\bfseries\itshape#1}}
\usepackage{fancyhdr}
\pagestyle{fancy}
\usepackage{titlesec}
\usepackage{titling}
\makeatletter
\lhead{\textbf{\@title}}
\makeatother
\rhead{\textrmlf{Compiled} \today}
\lfoot{\theauthor\ \textbullet \ \textbf{2021-2022}}
\cfoot{}
\rfoot{\textrmlf{Page} \thepage}
\renewcommand{\tableofcontents}{}
\titleformat{\section} {\Large} {\textrmlf{\thesection} {|}} {0.3em} {\textbf}
\titleformat{\subsection} {\large} {\textrmlf{\thesubsection} {|}} {0.2em} {\textbf}
\titleformat{\subsubsection} {\large} {\textrmlf{\thesubsubsection} {|}} {0.1em} {\textbf}
\setlength{\parskip}{0.45em}
\renewcommand\maketitle{}
\author{Houjun Liu}
\date{\today}
\title{Bulliet British Raj}
\hypersetup{
 pdfauthor={Houjun Liu},
 pdftitle={Bulliet British Raj},
 pdfkeywords={},
 pdfsubject={},
 pdfcreator={Emacs 28.0.50 (Org mode 9.4.4)}, 
 pdflang={English}}
\begin{document}

\tableofcontents



\section{Bulliet, British Raj}
\label{sec:org9c12f69}
\begin{itemize}
\item India became politically fragmented
\item EIC became dominant in Bengal though the avenging of Black Hole of
Calcatta + overthrowing of nawab
\item South Asia became the hotbed of colonialism

\begin{itemize}
\item Through a series of negotiations, persuation, and installation of
Nawabs, EIC gained strategic control over much of India
\item Goals of british raj: remake india through British admin, British
social reform, British economic development, and British Technology.
\item EIC Indian rule was inconsistent because CLAIM this is complicated!
\item EIC control --- double edge sword. Created anglicized jobs, but
destroyed original artisanship + handywork.
\end{itemize}

\item Goals:

\begin{enumerate}
\item Create efficient government
\item Disarm princes
\item Christanizationificiation
\item Revamp land holding claims
\end{enumerate}

\item Brought also British celebrations and expansion of social class,
property ownership, etc. to bolster EIC control.

\begin{itemize}
\item Religious and tradition conflicted with Britian's requirement for
solders.
\item Localized rebellions became empowered by the Indian British army
committing mutiny due to religious conflict.
\item Woman's rights went to the dumps, so did the poor's rights. CLAIM:
this was due to the brits trying to establish a sense of superority.
\end{itemize}

\item British admin change after 1858 rev\ldots{}

\begin{enumerate}
\item British men held high-level administrative posts and controled
indian offocials. CLAIM: this is complicated! Not a mutiny --- more
soldiers and people; not a revolution --- there was nothing they
are representing themselves with.

\item Lots of brits in the ICS, because they are damn racist.
\item Britian consolidated rule after rebellion + centralized gov't

\item Britian invested in Indian infrastructure and industry. (which
promoted job growth)

\item =>Used elaborate shows of wealth to maintain a sense of authority.

\item British government also promoted introduction of high tech ASAP to
India

\begin{itemize}
\item Railroad allowed india to become more connected + intergrated
while promoting british commerse \& investment.
\item Trains also brought cholora, but it was quickly mitigated by
installing filtrated water.
\end{itemize}
\end{enumerate}

\item => CLAIM: indian nationalism spawn from the establishment of British
India.

\begin{itemize}
\item Nationalism spawn from the needs to reform the Indian society +
improve education.
\item Rammohum Roy established schools promoting indian secular
nationalistic ideals while promoting social reforms to India.
\item New nationalists came from the middle class --- CLAIM propped up by
increase in trade caused by Brits.
\end{itemize}
\end{itemize}
\end{document}
