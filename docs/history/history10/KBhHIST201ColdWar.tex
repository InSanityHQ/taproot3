% Created 2021-09-12 Sun 22:48
% Intended LaTeX compiler: xelatex
\documentclass[letterpaper]{article}
\usepackage{graphicx}
\usepackage{grffile}
\usepackage{longtable}
\usepackage{wrapfig}
\usepackage{rotating}
\usepackage[normalem]{ulem}
\usepackage{amsmath}
\usepackage{textcomp}
\usepackage{amssymb}
\usepackage{capt-of}
\usepackage{hyperref}
\usepackage[margin=1in]{geometry}
\usepackage{fontspec}
\usepackage{indentfirst}
\setmainfont[ItalicFont = LiberationSans-Italic, BoldFont = LiberationSans-Bold, BoldItalicFont = LiberationSans-BoldItalic]{LiberationSans}
\newfontfamily\NHLight[ItalicFont = LiberationSansNarrow-Italic, BoldFont       = LiberationSansNarrow-Bold, BoldItalicFont = LiberationSansNarrow-BoldItalic]{LiberationSansNarrow}
\newcommand\textrmlf[1]{{\NHLight#1}}
\newcommand\textitlf[1]{{\NHLight\itshape#1}}
\let\textbflf\textrm
\newcommand\textulf[1]{{\NHLight\bfseries#1}}
\newcommand\textuitlf[1]{{\NHLight\bfseries\itshape#1}}
\usepackage{fancyhdr}
\pagestyle{fancy}
\usepackage{titlesec}
\usepackage{titling}
\makeatletter
\lhead{\textbf{\@title}}
\makeatother
\rhead{\textrmlf{Compiled} \today}
\lfoot{\theauthor\ \textbullet \ \textbf{2021-2022}}
\cfoot{}
\rfoot{\textrmlf{Page} \thepage}
\titleformat{\section} {\Large} {\textrmlf{\thesection} {|}} {0.3em} {\textbf}
\titleformat{\subsection} {\large} {\textrmlf{\thesubsection} {|}} {0.2em} {\textbf}
\titleformat{\subsubsection} {\large} {\textrmlf{\thesubsubsection} {|}} {0.1em} {\textbf}
\setlength{\parskip}{0.45em}
\renewcommand\maketitle{}
\author{Houjun Liu}
\date{\today}
\title{The Mid 20th Century Crisis}
\hypersetup{
 pdfauthor={Houjun Liu},
 pdftitle={The Mid 20th Century Crisis},
 pdfkeywords={},
 pdfsubject={},
 pdfcreator={Emacs 28.0.50 (Org mode 9.4.4)}, 
 pdflang={English}}
\begin{document}

\maketitle


\section{WWII and the Rise of the Cold War Tensions}
\label{sec:org856c887}
\subsection{Causes of WWII: a brainstorm}
\label{sec:org1fb145d}
\begin{itemize}
\item The failure of the treaty of versailies
\item WWI not being as hard and did not destroy the prospects of german
nationalism
\item The economic downfall caused by the previous war
\item The rise of the brutal systems of structure throughout the world ---
i.e. strong alt-right nationalism
\item The ruthless desire to peace
\item The destroying of present systems of the balance of power by
higher-level fighting tools
\end{itemize}

\subsection{How not to get a bonus Hitler}
\label{sec:orgefbda4d}
\begin{itemize}
\item Weaken individual control
\item Operate under the shared assumption of peacekeeping and democracy
\item Find shared goals and ideals to operate upon
\item Set clear guidelines for treaties and consiquences
\item Build up collective force to be able to enforce treties (up and not
limited to the threat of nuclear armegetton)
\end{itemize}

\textbf{Under the assumption of global cooperation} - Weaking individual
economic control

\textbf{Strategies to prevent} - Diminished economic freedom - Co-operation
across countries - GIVE PEOPLE NUKES!!!!!! (mutually assured
destruction)

Postwar world's rules

\begin{itemize}
\item Constructing economic systems and global market
\item "If you depend on a country for trade, you won't nuke them."
\item => Really, markets are cultivated. The plants are growing
independently, but the gardener is responsible for cultivating the
garden and preventing weeds
\end{itemize}

\subsubsection{Kensian Capitalism}
\label{sec:org86f6aab}
\begin{itemize}
\item Capitalism is not self-regulating
\item Capitalism need an external structure for making it work well
\end{itemize}

=> In economic downturns, the govrenment should put money in
circulation, in upward economy, the economy should start regulating
economy

\subsubsection{Neoliberalistic Capitalism}
\label{sec:orgdd88ab9}
\begin{itemize}
\item Capitalism is self-correcting
\item Inflation is the thing you should fear
\end{itemize}

=> Goverments will worsen inflation, which is the boogieman, so
capitalism needs guardrails against inflation, but generally they will
self-correct and so leave them to self-correct

\subsection{Global Market: Bretton Woods Agreement}
\label{sec:orgbde5b8e}
\href{Pasted image 20210224140335.png.org}{Pasted image
20210224140335.png}

\textbf{United Nations}: facilatete diplomatic exchange

IMF = create and loan out short-term trade deficits and regulate
exchanges

World Bank = roll infrastructure loans to countries for the long
economic development

To standardize, everything is pegged against the dollar, which is pegged
against \$36/oz.

This post-war order allows effective stability and balance

\noindent\rule{\textwidth}{0.5pt}

\subsection{Postwar World}
\label{sec:org6fbb232}
If you have two powers, they almost always result in some form of
conflict.

"The attitutes around the constrsuction of the post-war world lead to
the need for a global hegemone that ensures stability." And, at that
moment, the United States is the target to achieve that hegemony.

The United States posted the most likely hegemone, but becaues the
soviet union refused to take place in that hegemone, they became their
own system

Allie meetings at Tehran, Yalta, and Postdam

\begin{itemize}
\item Germany will be occupied by US, UK, France, and USSR
\item Korea will be occupied by the US and USSR for the moment, but it will
eventually be turned over
\end{itemize}

The \textbf{united nations} failed to fufill its final goals: became a forum
instead of a dicipline body

\href{Pasted image 20210224142636.png.org}{Pasted image
20210224142636.png}

Features of the USSR: that the leadership have unchecked power to act
without the support

\textbf{Both sides believe that the other side will collapse under its own
weight.}

Used to keep out western influence

\begin{itemize}
\item To keep control of their own populatino
\item To hasten the collapse of western democracy
\end{itemize}

The civil rights movements picked up heat in the united states: that the
collective realization that the USSR attempted to foment racial anger in
the United States.

Moderate-left group excluded communists, but they were the most "brave"
to stand up to societal injustices

\section{Globalization + End of CW}
\label{sec:orgfa4e627}
\begin{itemize}
\item Attitudes of capitalism different than that of the past
\item Transformation leads to reconfiguration of capitalism
\end{itemize}

There were general optimizism regarding post-cold war globalization. But
nowadays "globalization" is more cynical.

Fukayama's dialectical model of history: that Liberalism is the endpoint
for all government. => for Karl Marx, its the promise that communism
will rid of class troubles. For Fukuyama, its the promise that
liberalism will rid of ideological confilicts.

\subsection{Globalization + Liberalism}
\label{sec:org01760f2}
Globalization => increased attetion about the relationshiyp btween
nation states and the markets

Neoliberalism => a politicial project to promote pro-market supply-side
policies; by giving people more market access, one is able to alleviate
the conditions of poverty. "Markets Fundimentalism": markets and only
markets are the agents of individual liberalty and social progress.
Market economy as the absoulte

IMF + others tried to push forward the change to neolibral capitalism

\noindent\rule{\textwidth}{0.5pt}

Progress requires destruction and pain: society can't progress without
accepting the suffering that progress produces.
\end{document}
