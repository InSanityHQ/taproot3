% Created 2021-09-12 Sun 22:48
% Intended LaTeX compiler: xelatex
\documentclass[letterpaper]{article}
\usepackage{graphicx}
\usepackage{grffile}
\usepackage{longtable}
\usepackage{wrapfig}
\usepackage{rotating}
\usepackage[normalem]{ulem}
\usepackage{amsmath}
\usepackage{textcomp}
\usepackage{amssymb}
\usepackage{capt-of}
\usepackage{hyperref}
\usepackage[margin=1in]{geometry}
\usepackage{fontspec}
\usepackage{indentfirst}
\setmainfont[ItalicFont = LiberationSans-Italic, BoldFont = LiberationSans-Bold, BoldItalicFont = LiberationSans-BoldItalic]{LiberationSans}
\newfontfamily\NHLight[ItalicFont = LiberationSansNarrow-Italic, BoldFont       = LiberationSansNarrow-Bold, BoldItalicFont = LiberationSansNarrow-BoldItalic]{LiberationSansNarrow}
\newcommand\textrmlf[1]{{\NHLight#1}}
\newcommand\textitlf[1]{{\NHLight\itshape#1}}
\let\textbflf\textrm
\newcommand\textulf[1]{{\NHLight\bfseries#1}}
\newcommand\textuitlf[1]{{\NHLight\bfseries\itshape#1}}
\usepackage{fancyhdr}
\pagestyle{fancy}
\usepackage{titlesec}
\usepackage{titling}
\makeatletter
\lhead{\textbf{\@title}}
\makeatother
\rhead{\textrmlf{Compiled} \today}
\lfoot{\theauthor\ \textbullet \ \textbf{2021-2022}}
\cfoot{}
\rfoot{\textrmlf{Page} \thepage}
\titleformat{\section} {\Large} {\textrmlf{\thesection} {|}} {0.3em} {\textbf}
\titleformat{\subsection} {\large} {\textrmlf{\thesubsection} {|}} {0.2em} {\textbf}
\titleformat{\subsubsection} {\large} {\textrmlf{\thesubsubsection} {|}} {0.1em} {\textbf}
\setlength{\parskip}{0.45em}
\renewcommand\maketitle{}
\author{Houjun Liu}
\date{\today}
\title{Cold War Essay Planning Jack}
\hypersetup{
 pdfauthor={Houjun Liu},
 pdftitle={Cold War Essay Planning Jack},
 pdfkeywords={},
 pdfsubject={},
 pdfcreator={Emacs 28.0.50 (Org mode 9.4.4)}, 
 pdflang={English}}
\begin{document}

\maketitle


\section{Cold War Research Paper}
\label{sec:org5c7f061}
\subsection{General Information}
\label{sec:orgcf4887a}
\begin{center}
\begin{tabular}{lll}
Due Date & Topic & Important Documents\\
\hline
3/17/21 & Cold War in the Developing World & JSTOR, Palmer 23\\
\end{tabular}
\end{center}

\subsection{Prompt}
\label{sec:org87ecec5}
New accounts push past a bifurcated world to present the Cold War as a
triangular struggle between the two great powers and developing nations
to show how those involved in decolonization struggles adapted
strategies shaped by the Cold War dynamic.

Your task is to explore this triangular relationship within a single
case study. Your task is to analyze how your sources suggest a way to
capture this triangular cold war dynamic.

In a \textasciitilde{}4 page paper, evaluate how a global perspective for the Cold War
helps to understand events in your case study.

**How did the Cold War system shape development and/or decolonization in
your region?

\begin{html}
<!--You might ask how the “cold war lens” led Soviet and American policy makers to pursue counterproductive policies. How leaders and revolutionaries took advantage of the Cold War dynamic to advance their own interests, playing one power off of another? You might even identify policies and strategies that were separate from Cold War imperatives--What is the relation, for example, of populism, Arab Nationalism and/or Pan-Africanism to Cold War priorities?-->
\end{html}

\subsection{Quotes Bin}
\label{sec:org86f95a8}
\begin{itemize}
\item "Beijing and Paris moved from military and political antagonism over
the Indochina conflict in 1946-1954 and over the Algerian liberation
struggle in 1954-1962 toward a partial alignment of views." AA
\item "It was the shared uneasiness over the deteriorating situation in
Vietnam and over the Limited Test Ban Treaty (LTBT), signed by the
United States, the United Kingdom, and the Soviet Union in Moscow in
early August 1963, that helped to usher in the process of
recognition." AB
\item A five point reason for establishment of relationships as argued by
historian historian Garret Martin L1

\begin{enumerate}
\item "First, the end of the Algerian War in the spring of 1962 removed a
major point of Sino-French conflict." L1A
\item "Second, the Cuban missile crisis in October ofthe same year
revealed that neither superpower wanted global war. In turn, this
provided new opportunities for medium powers to follow independent
policies." L1B
\item "Third, Sino-French recognition was a natural follow-up to de
Gaulle's attempts to break Anglo-American dominance in the West in
early 1963 through the Franco-German treaty." L1C
\item "Fourth, de Gaulle perceived the LTBT of August 1963 as an attack
on French global interests." L1D
\item "Finally, the political instability in South Vietnam in 1963
brought China and France to agree in their criticism of Ngo Dinh
Diem's regime." L1E
\end{enumerate}

\item "The shared nuclear isolation and the partial agreement on Vietnam in
August 1963 by Beijing and Paris symbolized a new stage in the
trajectory oftheir mutual relationship from antagonism toward mutual
recognition" AC
\item "From the day the PRC was founded in the fall of 1949, it had striven
for universal diplomatic recognition. Not only did this policy express
the old nationalist aim ofrestoring China to its former glory and
greatness in international relations; it also conveyed the aspiration
ofthe Communist leaders to obtain recognition from non-Communist
states, based on the “principles of equality"” AD
\item "The PRC explicitly warned the Soviet Union against assuming legal
“responsibilities in lieu of China. Even now, decades after the
signing of the LTBT, Chinese publications still claim that the treaty
deprived the PRC ofthe right to develop and test nuclear weapons." AE
\item "President John F. Kennedy, who feared the Chinese acquisition
ofnuclear weapons in 1962-1963, was at times more interested in the
proliferation aspects of the treaty than in the banning of testing in
multiple environments" AF
\item "De Gaulle was highly critical of the LTBT. In reality, however,
France was not affected by the accord s provisions. The underground
tests it was conducting in Algeria were permitted by the
Soviet-British-American pact as the only way to test nuclear weapons.
But the French president saw the trilateral agreement as yet another
Yalta-style deal among the three most powerful countries on earth,
concluded behind the backs ofless powerful states, and designed to
preserve the three great powers' preeminent status in international
relations" AE
\item "Even if France and the PRC disagreed on the future course with regard
to Vietnam, they both believed in the necessity of U.S. political and
military withdrawal from the country" AF
\item "The desire for recognition by the outside world also included the
wish to join the UN and, as the ultimate goal, to take up the
permanent seat ofthe ROC on the UN Security Council." AG
\item "The PRC took pains to express its interest in better relations
through some of its highest officials present at the Laos conference
in Geneva." AH
\item "The French ambassador had the strong impression that the PRC wanted
to give the mutual relationship a “new importance." The Chinese had
extended a hand, and they believed it was now up to the French to
accept it.” AI
\item "[Mao's] statement signified a departure from his earlier thinking on
intermediate zones, which envisioned a Europe divided in halves
belonging to either the socialist or the capitalist camp. But what did
Maos new intermediate zone in Europe really mean? He claimed that the
European countries were “unhappy with the U.S. and the Soviet Union."
AJ
\item "As the LTBT negotiations in Moscow drew to a close---the concurrent
Sino-Soviet party-to-party talks in that city, which were intended to
bring about ideological reconciliation, had already ended in failure"
AK
\item The French Foreign Ministry Asia Department Communique: "Asia
Department ofthe French Foreign Ministry prepared a report in early
February 1961 advocating rapprochement with Communist China---for
reasons of principle and practicality." L2

\begin{itemize}
\item "The Chinese Communists had held de facto power over mainland China
for more than ten years. The notion that the ROC in Taiwan \ldots{} was
the legal representative of the Chinese people \ldots{} had long before
turned out to be a fiction." L2A
\item "Since the end of the Indochina War in 1954---and despite the
Algerian war---France and the PRC \ldots{} no longer had any reason to be
antagonistic toward each other." L2B
\item "Because many newly independent French colonies in Africa were
gravitating toward recognition ofmainland China, France had to
appear to lead them in recognizing the PRC, otherwise it would lose
influence in Africa." L2C
\item "Finally, Communist China had become the most important French
trading partner in East Asia." L2D
\end{itemize}

\item "He explicitly remarked that the LTBT did not prevent the PRC or
France from continuing nuclear testing. Expecting an imminent Chinese
nuclear test, de Gaulle elaborated that once China had obtained
nuclear capacities it would become a major actor in international
relations." AL
\item "Zhou concluded that the Sino-Soviet split and the temporary
U.S.-Soviet rapprochement on nuclear issues had created the basis for
a major rearrangement in international relations, which would
logically include Sino-French rapprochement." AM
\item So why did Zhou decide this was a good idea? L3

\begin{itemize}
\item They "believed that rapprochement with the most important European
continental power, France, would progressively lead to increased
political and economic relations with other West European countries,
which in turn would break the U.S. economic blockade ofthe PRC while
simultaneously increasing China's international status." L3A
\item "Sino-French rapprochement would be beneficial in isolating and
opposing U.S. imperialism on a global scale." L3B
\item "Finally, China could align with France's policy of national
independence designed to break the superpower monopoly in
international relations." L3C
\end{itemize}

\item "As de Gaulle saw it, France was willing to work for the admission
ofthe PRC to the UN and to scale down French- Taiwanese relations to
an informal level, but it was not willing to offer a French initiative
to break relations with the ROC." AN
\item "De Gaulle wanted both to proceed cautiously and to portray France as
a great power unwilling to sub- mit to the dictates of any other
country." AO
\item "Zhou lauded France's decision to grant Algeria independence the year
before and commented on the fact that both France and China, as
aspiring nuclear powers, opposed the LTBT." AP
\item "Zhou attempted to bring up the One-China principle, whereas Faure
demanded the exclusion ofthe issue from this preliminary session." AQ
\item "Faure insisted that de Gaulle would never take the initiative in
breaking relations, so he suggested that France and the PRC simply
exchange ambassadors, while the French govern- ment would downgrade
the ROC representation in Paris to a level to be specified later." AR
\item "The situation in Indochina had deteriorated further; the relations
ofGreat Britain and the United States---France's two competitors for
global influence---with the Southeast Asian countries had worsened in
parallel."” AU
\item "De Gaulle parted from Faure with the comment that he would go through
with French recognition of the PRC if his conversations in Washington
did not change U.S. views\ldots{}. deGaulle's encounters with Johnson did
not go well." AV
\item "DeGaulle instructed deBeaumarchais \ldots{} only"to define a procedure:
/the simpler the better/”” AW
\item "Zhou proposed a fourth procedure in case France was not able to agree
to the One-China prin ciple. In that case, Li should state the Chinese
position on the issue but not insist on the inclusion ofa sentence to
that effect in any draft joint communique. \ldots{} de Beaumarchais
insisted that France would not agree to any reference to the One-China
principle in the joint communique. Li, following the instructions he
had received, dropped the issue" AX
\item "if deBeaumarchais insisted on France maintaining a Two-China policy,
Li was supposed to state clearly that such a position would mean a
departure from earlier French positions and a sign ofdisrespect to the
PRC, but Zhou's instructions did \emph{not} call for a termination
ofnegotiations in such a case" AY
\item "De Gaulle's insistence on a simple text left the agreement vague with
regard to the ROC on Taiwan." AZ
\item So why did deGaull think this was such a good idea? L4

\begin{itemize}
\item "He stated that China was the world's largest country, that it was
no longer under the control ofthe Soviet Union, that its Communist
government was a fact of life, and that all “political realities" of
Asia” L4A
\item "Thus, he continued, it would be absolutely “impossible" to envision
a solution to the problems ofEast Asia without China” L4B
\item "de Gaulle also stressed that recognition of the PRC entailed
approval of neither its institutions nor its policies" L4C
\end{itemize}
\end{itemize}

\subsection{Claim Synthesis}
\label{sec:org4787fd8}
\subsubsection{The Claim}
\label{sec:orgb83dc19}
\begin{enumerate}
\item Scholarly View
\label{sec:org9e1c748}
\begin{itemize}
\item AA France and China moved away from military political conflict and
towards view alignment
\end{itemize}

\item Shared View
\label{sec:org170acd6}
\begin{itemize}
\item AB shared uneasiness over LTBT that triggered french recognition
\item AF although PRC and France disagreed on what happens to vietnam, they
both hated the US boogieman meddling with Vietnam
\end{itemize}

\item French View
\label{sec:orgcafeb8b}
\textbf{Value-Based Decisions}

\begin{itemize}
\item The French approached Sino-French relations cautiously, not wanting to
give in too much, and AO portray France as a great, unwilling to
submit power
\item For instance, when

\begin{itemize}
\item AQ Zhou tried to bring up One-China principle, and Faure said: No.
\end{itemize}

\item is a scene where French were willing to risk inacceptance over their
formal diplomatic relations

\begin{itemize}
\item AR Faure maintained that deGaulle would never directly break RoC
relations, and suggest waiting for RoC to break diplomatic contact
\item AZ So, DeGaulle left it vague what they are going to do about 2
china policy
\end{itemize}
\end{itemize}

w.r.t. "Two China Policy" --- a Chinese values-based demand --- the
French clearly pushed back and because there is no interest in them
giving up RoC relations, did not actively use it as a barganing chip.

\textbf{Realism-Based Decisions}

\begin{itemize}
\item AQ Zhou tried to bring up One-China principle, and Faure said: No.
\item AR Faure maintained that deGaulle would never directly break RoC
relations, and suggest waiting for RoC to break diplomatic contact
\item AV last straw deGaulle tried to convince Johnson to do something about
diplomacy with China. Johnson did not listen. So, deGaulle did
diplomacy with China (connection BA)
\item AN deGaull thinks that the French is willing to help PRC go to UN and
scale Franco-RoC relations down but not directly offer a breaking of
Franco-RoC relations
\end{itemize}

However, given things that are realist tenants to what the French could
do to help china (UN), the French did not push back.

\textbf{Value-Based Reflections}

\begin{itemize}
\item AE The actualy LTBT doesen't really matter to the French, but its more
about deGaulle not wanting to be the dog of British-American-Soviet
relations
\item L4C deGaulle also stressed that PRC recognition is not an approval of
it policies/institutios
\item L2C the French need to set an example in approaching China, otherwise
china is going to approach the Indep. French colonies first, which
would be awkward
\item AL deGaull know that LTBT did not prevent PRC/France from trying
Kaboom; so, expecting China to try blowing things up soon, deGaull
reasoned that China will soon become a superpower once they could make
things go Kaboom.
\end{itemize}

The French wanted to help China as a way of advacing its own
nationalistic goals in the department of international leadreship and
representation.

\textbf{Realism-Based Reflections}

\begin{itemize}
\item AI the french ambassidor had the impression that "The Chinese had
extended a hand, and they believed it was now up to the French to
accept it."”
\item L1B cuban missle crisis showed that american/russia did not want to
make world go kaboom, so medium states feel better to persue
independent policies
\item L4A deGaull thinks that China's large economy no longer controlled by
USSR is a fact of life, so \#dealwithit
\item L4B deGaull reasoned that it would be impossible to solve the East
Asia problem w/o China
\end{itemize}

The French also see that China being in asia as a fact of life, so they
have to \#dealwithit; might as well do so whilst it is so fresh and China
is open to it.

\item Chinese View
\label{sec:orgea65159}
\textbf{Value-Based Decisions}

Conspicuously missing.

\textbf{Realism-Based Decisions}

\begin{itemize}
\item AE PRC did not want USSR to represent China wrt LTBT b/c of
discontent; to this day PRC claim LTBT deprived them of nuclear
development rights
\item AK the reconciliation between Moscow and Beijing did a die after LTBT
negotiations in Moscow (connection BA)
\item L3A Zhou believed that reappoarchment with France will eveuntally
cause US to stop blocading PRC + increase Chineses status
\item AX Zhou decided that if the French was really not up to the one-china
thing in the comminque, it is ok to just drop it. the French was no
ok, so they dropped it!
\item AY Zhou was even open to negotiatinos even if deBeaumarchais (deGaull
\begin{itemize}
\item friends) wanted to maintain a Two-China policy
\end{itemize}
\end{itemize}

China wanted to use realistic strategies that may not necessarily align
with its political goals to advance its goals of getting a "spot in the
light" ideologically.

\textbf{Value-Based Reflections}

\begin{itemize}
\item AD China drove for diplomatic recognition from capitalists from day
one aiming to restore nationalist glory and the "principles of
equality"
\item AJ Mao departed from his earlier thinking of Europe in two \textbf{a la cold
war} and moved to a European countries is unhappy with both US and
Soviet stance (connect CA)
\item L3C china could align with French's National Independence to break
superpower monopoly
\item L3B Sino-French corroporation will oppose US + imperialism
\end{itemize}

China reasoned that the Cold War will probably splinter Europe and cause
smaller European countries to resist superpower monopoly. Hence,
partnership with France will break the imperialistic and capitalistic
systems.

\textbf{Realism-Based Reflections} - AM Zhou Enlai concluded that he
Sino-Soviet split + US Soviet Nuclear Reapporachment (\textbf{product of the
cold war??}) means that international BoP will change soon, so might as
well build relations with France (connect CA)

Realistically, China believed that the cold war's BOP shift will offer
them a chance to have some skin in the game.

More better question: how does the Cold War influence the Franco-Chinese
balance of Ideloogy vs. Pragmatism?

\noindent\rule{\textwidth}{0.5pt}
\end{enumerate}

\subsection{The Claim}
\label{sec:org6906932}
\begin{html}
<!--\textbf{\textbf{The Cold War gave an opportunity for the capitalist French Republic and the communist PRC an opportunity of cross-aisle collaboration: leveraging shared gains under political pragmatism and rallying plans via a sense of nationalistic pride whilst resisting fundimental changes of values and morals.}}-->
\end{html}

*Cold war dynamics of the 1960s created a diplomatically timely
opportunity for the development of Sino-French relationships for both
states to assert their nationalism-fueled need for increased
international recognition; to garner heightened recognition, both had to
leverage diplomatic pragmatism in balancing their values with their
realities to ultimately achieve their nationalistic goals of added
recognition.*

\emph{Finally, define the results of the athour's focus on the sidelines of
the Cold War. Engage with the projects of the author / their choices.}

\begin{itemize}
\item Cold war create an opportuntity => /foregoround it with how the author
is defining the cold war moment. The author is conciously decenter the
cold war in order to compare China with France/

\item Sino french leverage opportunity with pragmatic ways

\item Nationalism as a end whilst value maintainnence as a means to that end

\item The cold war created the ideal circumstance for sino-french
collaboration

\begin{itemize}
\item On the Chinese side, they saw the Cold War as an opportunity of
growth and assertion of national identity admidst European chaos:

\begin{itemize}
\item AM Zhou Enlai concluded that he Sino-Soviet split + US Soviet
Nuclear Reapporachment --- a product of the cold war --- means
that international BoP will change soon, so might as well build
relations with France (connect CA)
\item AJ Mao departed from his earlier thinking of Europe in two \textbf{a la
cold war} and moved to a European countries is unhappy with both
US and Soviet stance (connect CA)
\item AE PRC did not want USSR to represent China wrt LTBT b/c of
discontent; to this day PRC claim LTBT deprived them of nuclear
development rights
\end{itemize}

\item On the French side, the combination of the the cuban missle crisis +
LTBT gave the french a carrot and a stick to persue idependent
dipolmacy from major European powers

\begin{itemize}
\item \emph{The carrot}: L1B cuban missle crisis showed that american/russia
did not want to make world go kaboom, so medium states feel better
to persue independent policies
\item \emph{The stick}: AE The actualy LTBT doesen't really matter to the
French, but its more about deGaulle not wanting to be the dog of
British-American-Soviet relations
\end{itemize}
\end{itemize}

\item Both countries reasoned the reconsiliation on practical grounds

\begin{itemize}
\item The French reasoned the necessesity of Sino-French relationships
based on the fact that China will be a major player at least in the
East whether it liked it or not, so\ldots{}

\begin{itemize}
\item AL deGaull know that LTBT did not prevent PRC/France from trying
Kaboom; so, expecting China to try blowing things up soon, deGaull
reasoned that China will soon become a superpower once they could
make things go Kaboom.
\item L4B deGaull reasoned that it would be impossible to solve the East
Asia problem w/o China
\item L2C the French need to set an example in approaching China,
otherwise china is going to approach the Indep. French colonies
first, which would be awkward
\end{itemize}

\item The Chinese approached it from a need of international recognition,
and potentially the advance of the 1 china policy.

\begin{itemize}
\item AK the reconciliation between Moscow and Beijing did a die after
LTBT negotiations in Moscow (connection BA)
\item L3A Zhou believed that reappoarchment with France will eveuntally
cause US to stop blocading PRC + increase Chineses status
\item L3B Sino-French corroporation will oppose US + imperialism
\item L3C china could align with French's National Independence to break
superpower monopoly
\end{itemize}
\end{itemize}

\item Both parties approached the negotiations cautiously; but the Chinese
were willing to give up ideological grounds for practical gain while
the French guarded their ideologial bottom line

\begin{itemize}
\item The French approached Sino-French relations cautiously, not wanting
to give in too much, and AO portray France as a great, unwilling to
submit power
\item L4C deGaulle also stressed that PRC recognition is not an approval
of it policies/institutios
\item AX Zhou decided that if the French was really not up to the
one-china thing in the comminque, it is ok to just drop it. the
French was no ok, so they dropped it!
\item AY Zhou was even open to negotiatinos even if deBeaumarchais
(deGaull + friends) wanted to maintain a Two-China policy
\end{itemize}
\end{itemize}

\begin{enumerate}
\item Discussions with Tom
\label{sec:orga340c5c}
Asserting a French power to reaffirm the "specialness" of France =>
nationalism! Not ideology.

Maintainence of ideological commitments as a means to National
preeminence => ideology is trans-national

\begin{itemize}
\item France maintains ideological commitments to not appear as capitulating
\item Disentangle nationalism from idelogy
\end{itemize}

"Why is it important for the French to maintain their conservatism?"

\begin{enumerate}
\item In the 1960s, CW dynamics created an timely opportunity for china +
us to assert national position on the international stage
\item To do so, however, countries had to balance values and pragmatism in
order to ultimately support their goal of doing so.
\end{enumerate}

\noindent\rule{\textwidth}{0.5pt}

\noindent\rule{\textwidth}{0.5pt}

How does the Chinese people get sold on the idea of establishing
relationships with the French? Was it authoritarianism?

How does nationalism inform the decision between deGaulle and Mao?
deGaulle's wish to play/dominate European politics + Mao's wish to
establish "principles of equality" => both sentiments of Nationalism.

Perhaps both states employed \emph{realpolitik} treatment to politics and to
advanced their state goals.

Cold war stabilshed a balance between pragmatism and ideology, and
reality does/does not matter?

Revolutionary seniments used and the sense of "fellow revolutionary"
mentality as a tool to persue a realist sense of partnership
\end{enumerate}

\subsubsection{Garbagish}
\label{sec:org31a311d}
\begin{itemize}
\item "The nature of the documents \ldots{} does not provide a comprehensive view
into top-level decision-making. Most of that kind of documentation is
stored in the Central Party Archive in Beijing, which is closed to
foreign researchers."
\item "Although de Gaulle's proposal was welcomed in Laos and Cambodia, the
two Vietnams suspected the French president of trying to regain French
influence. The United States did not take the announcement well
either, in part because de Gaulle had assured Kennedy in 1961 that he
would keep his disagreement on Vietnam to himself."
\item "Whereas the establishment ofSino-French diplomatic relations remained
stuck in the Cold War and blocked by the Algerian war, trade relations
de- veloped following the collapse of Sino-Soviet economic relations
in mid-1960."
\item "Isolated in the Arab world, Egypt recipro- cated with a gesture to
the Chinese: it recognized both North Korea and North Vietnam in early
October."
\item "explaining that the French government considered the time “ripe" to
recognize the Chinese government.”
\item "The goal of Zhou's seven-week tour to ten Middle Eastern and African
countries \ldots{} was to improve bilateral contacts but also, according to
the memoir of Zhou's interpreter, to gain support for UN membership."
\item "During the meal on 20 August, Faure asked point-blank for an
invitation to visit China in October for the purpose ofmeeting Chinese
leaders and “visiting peoples communes."”
\item "France had not been against recognition in principle, but de Gaulle's
decision to reach out to the PRC happened against the background ofthe
evolving Vietnam crisis, albeit \emph{before} the introduction ofmartial
law in South Vietnam on 20 August and the French president's call for
neutralization nine days later."
\item "Given that the PRC had challenged the Soviet Union since the late
1950s, the Chinese foreign minister lauded France for standing up to
the United States, the head ofthe capitalist camp. He also ex- pressed
his desire to see a strong France---one that would restrain the United
States and West Germany in Europe and help to contain the United
States and Japan in Asia"
\item "[Mao's] analysis hinted at one seemingly obvious outcome: the advent
of multipolarity and a rearrangement of international relations."
\item "The report thus suggested using the “Cuban method" of making a public
agreement on recognition in principle, accompanied by a secret
agreement regarding the nature of diplomatic relations.”
\item "Faure replied that the circumstances seemed to have changed in recent
times in favor ofrecognition: the Algeria problem had been resolved,
China was in dire straits after the Sino-Soviet split, and France had
already succeeded in marking out a stance in international rela- tions
that was independent from the United States---particularly on
Vietnam."
\item "Playing on de Gaulle's anti-American attitudes, Faure asserted that
the Chinese leaders believed that only three non-Communist powers in
the world \ldots{} had some sympathy for the PRC \ldots{} although the Chinese
leaders were convinced that only France had “escaped the American
feudalization."”
\end{itemize}

\subsubsection{The Dump}
\label{sec:org09b00cc}
\begin{itemize}
\item AH China actively brough up better relations with France at Laos
conference
\item AF JFK wanted Chinese nuclear proliferation more than test banning
\item AC nuclear isolation + partial agreement on Vietnam developed
sino-french relations
\item AG chinese desire of recognition also included inclusion to the UN
\item L2 The French Foreign Ministry established a communique regarding the
principle and practical reasons of chinese reappoarchment

\begin{itemize}
\item L2A the notion that RoC was legally China is just not true
\item L2B the French had no practical reason to hate on China
\item L2D also, trade is cool! And china does asia trade goodly.
\end{itemize}

\item AP Zhou commended French decision to make Algeria independent + also
believed both countries hated LTBT
\item AU indonesia deteriated. further. GB and US relations in Southeast
asia worsened
\item AW deGaulle wanted Occam's Razor Chinese Relations
\item L1 a five point argument of why relationships established

\begin{itemize}
\item L1A End of the Algerian war eased tensions
\item L1C Sino-French recognition is a follow up to deGaulle's
de-angloamerican dominance campaign
\item L1D deGaulle felt LTBT was an attach on French interests
\item L1E south Vietnam instability became triggering point for shared
political agreement against Ngo Dinh Diem
\end{itemize}
\end{itemize}

\noindent\rule{\textwidth}{0.5pt}

There is always
\href{https://wp.ucla.edu/wp-content/uploads/2016/01/UWC\_handouts\_What-How-So-What-Thesis-revised-5-4-15-RZ.pdf}{UCLA
Writing Lab}
\end{document}
