% Created 2021-09-27 Mon 12:00
% Intended LaTeX compiler: xelatex
\documentclass[letterpaper]{article}
\usepackage{graphicx}
\usepackage{grffile}
\usepackage{longtable}
\usepackage{wrapfig}
\usepackage{rotating}
\usepackage[normalem]{ulem}
\usepackage{amsmath}
\usepackage{textcomp}
\usepackage{amssymb}
\usepackage{capt-of}
\usepackage{hyperref}
\setlength{\parindent}{0pt}
\usepackage[margin=1in]{geometry}
\usepackage{fontspec}
\usepackage{svg}
\usepackage{cancel}
\usepackage{indentfirst}
\setmainfont[ItalicFont = LiberationSans-Italic, BoldFont = LiberationSans-Bold, BoldItalicFont = LiberationSans-BoldItalic]{LiberationSans}
\newfontfamily\NHLight[ItalicFont = LiberationSansNarrow-Italic, BoldFont       = LiberationSansNarrow-Bold, BoldItalicFont = LiberationSansNarrow-BoldItalic]{LiberationSansNarrow}
\newcommand\textrmlf[1]{{\NHLight#1}}
\newcommand\textitlf[1]{{\NHLight\itshape#1}}
\let\textbflf\textrm
\newcommand\textulf[1]{{\NHLight\bfseries#1}}
\newcommand\textuitlf[1]{{\NHLight\bfseries\itshape#1}}
\usepackage{fancyhdr}
\pagestyle{fancy}
\usepackage{titlesec}
\usepackage{titling}
\makeatletter
\lhead{\textbf{\@title}}
\makeatother
\rhead{\textrmlf{Compiled} \today}
\lfoot{\theauthor\ \textbullet \ \textbf{2021-2022}}
\cfoot{}
\rfoot{\textrmlf{Page} \thepage}
\renewcommand{\tableofcontents}{}
\titleformat{\section} {\Large} {\textrmlf{\thesection} {|}} {0.3em} {\textbf}
\titleformat{\subsection} {\large} {\textrmlf{\thesubsection} {|}} {0.2em} {\textbf}
\titleformat{\subsubsection} {\large} {\textrmlf{\thesubsubsection} {|}} {0.1em} {\textbf}
\setlength{\parskip}{0.45em}
\renewcommand\maketitle{}
\author{Houjun Liu}
\date{\today}
\title{Gelvyn Chapter 5}
\hypersetup{
 pdfauthor={Houjun Liu},
 pdftitle={Gelvyn Chapter 5},
 pdfkeywords={},
 pdfsubject={},
 pdfcreator={Emacs 28.0.50 (Org mode 9.4.4)}, 
 pdflang={English}}
\begin{document}

\tableofcontents



\section{Gelvin Chapter 5}
\label{sec:org8761a2e}
\begin{itemize}
\item States fought over the emergent control after the crisis of the 17th
century.
\item Employed the strategy of \textbf{Defensive Developmentalism.}
\end{itemize}

Defensive developmentalism: develop your country as a defense against
others

\begin{quote}
"Developing because Europe is happening to you."
\end{quote}

\subsection{Defensive Developmentalism}
\label{sec:org724d3db}
\textbf{Denfense developmentalism: centralizing authority by making goverment
more efficient and managing resource better.}

\begin{enumerate}
\item Millitary reform to consolidate power.
\item Control and coordinate population and resource
\item Decipline population to become agents of State.

\item Eliminated cash farming and augmented admin.
\item DD encouraged the process of monopolization and direct contro
\end{enumerate}

\subsubsection{Problems in DD}
\label{sec:org33b0dad}
\begin{itemize}
\item Local suspicious caused difficulty in implementing centralized control

\begin{itemize}
\item implimentation of government plans is difficult due to the emergence
of local resistance.
\item New class of the educated under the new system championed the
improved inclusion in governance. Many framed this plea as the plea
for constitutional governance
\end{itemize}

\item the middle eastern implimentation of DD had focus on developing
indurstry to support the DD processes such as the army => free trade
is opposite DD, which caused paradoxical effects.
\item Centralization helped very little people except for the central
government, making it resisted very widely.
\end{itemize}

\subsection{Ottomans and DD}
\label{sec:org3737ea0}
\begin{itemize}
\item Ottoman DD happened in two periods.

\begin{itemize}
\item Per. 1: tanzimat --- the "liberal" period where constitutionalism
was briefly trialed
\item Per. 2: promotion of direct control by the sultan.
\item Two phases' change was a change from bottom-up nationalism to
top-down nationalism.
\end{itemize}

\item Process of DD impliementation

\begin{itemize}
\item Ottomans first attempted economic control. That didn't go really
well either --- lack of monies and control.

\begin{itemize}
\item After doing everything wrong, the Ottomans conceded to build an
open economy and be connected
\end{itemize}

\item The widespread nature of the ottomans made it diffucult to have a
central point of control.
\item Millitary reform was tried again, creating European-style New Corps.
This was used to eliminate the janissaries.
\item System of equality actually promoted more discord between
communites.
\item The efforts of bottom-up nationalisation was disliked by everyone
except the hindus, b/c the Muslims felt that it hindered the
dominance of the muslimes while the Christians wanted to avoid the
widespread conscription
\end{itemize}
\end{itemize}

\section{CN12092020}
\label{sec:org6086fbb}
\begin{itemize}
\item Nationalism begins taking place
\item Ottomans decline

\begin{itemize}
\item Traditionalist values + religios schooling
\item Defensive Developmentalizm

\begin{itemize}
\item Tanzimat Reform (1839)

\begin{itemize}
\item Lead by Sultan Abdulmecid
\item Industrialization, tax reform, abolist millet system
\end{itemize}

\item Young Ottomans (1876)

\begin{itemize}
\item Write and passed a constitution
\item The Sultan immediately abolistes it
\end{itemize}

\item Hamidian Reforms (1990)

\begin{itemize}
\item Railroads, telegraphs, universities
\item CRackdown on dissent/Armenians
\end{itemize}
\end{itemize}
\end{itemize}
\end{itemize}
\end{document}
