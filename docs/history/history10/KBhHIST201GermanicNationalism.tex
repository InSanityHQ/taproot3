% Created 2021-09-11 Sat 16:40
% Intended LaTeX compiler: xelatex
\documentclass[letterpaper]{article}
\usepackage{graphicx}
\usepackage{grffile}
\usepackage{longtable}
\usepackage{wrapfig}
\usepackage{rotating}
\usepackage[normalem]{ulem}
\usepackage{amsmath}
\usepackage{textcomp}
\usepackage{amssymb}
\usepackage{capt-of}
\usepackage{hyperref}
\usepackage[margin=1in]{geometry}
\usepackage{fontspec}
\usepackage{indentfirst}
\setmainfont[ItalicFont = LiberationSans-Italic, BoldFont = LiberationSans-Bold, BoldItalicFont = LiberationSans-BoldItalic]{LiberationSans}
\newfontfamily\NHLight[ItalicFont = LiberationSansNarrow-Italic, BoldFont       = LiberationSansNarrow-Bold, BoldItalicFont = LiberationSansNarrow-BoldItalic]{LiberationSansNarrow}
\newcommand\textrmlf[1]{{\NHLight#1}}
\newcommand\textitlf[1]{{\NHLight\itshape#1}}
\let\textbflf\textrm
\newcommand\textulf[1]{{\NHLight\bfseries#1}}
\newcommand\textuitlf[1]{{\NHLight\bfseries\itshape#1}}
\usepackage{fancyhdr}
\pagestyle{fancy}
\usepackage{titlesec}
\usepackage{titling}
\makeatletter
\lhead{\textbf{\@title}}
\makeatother
\rhead{\textrmlf{Compiled} \today}
\lfoot{\theauthor\ \textbullet \ \textbf{2021-2022}}
\cfoot{}
\rfoot{\textrmlf{Page} \thepage}
\titleformat{\section} {\Large} {\textrmlf{\thesection} {|}} {0.3em} {\textbf}
\titleformat{\subsection} {\large} {\textrmlf{\thesubsection} {|}} {0.2em} {\textbf}
\titleformat{\subsubsection} {\large} {\textrmlf{\thesubsubsection} {|}} {0.1em} {\textbf}
\setlength{\parskip}{0.45em}
\renewcommand\maketitle{}
\author{Houjun liu}
\date{\today}
\title{Germanic Nationalism}
\hypersetup{
 pdfauthor={Houjun liu},
 pdftitle={Germanic Nationalism},
 pdfkeywords={},
 pdfsubject={},
 pdfcreator={Emacs 27.2 (Org mode 9.4.4)}, 
 pdflang={English}}
\begin{document}

\maketitle


\section{Germanic Nationalism}
\label{sec:org1a75e9c}
CLAIM: Conservative establishments of Europe was threatened by 1840s
\href{KBhHIST201Nationalism.org}{KBhHIST201Nationalism}

Strong figures in \textbf{Germany and Italy} created nation states from a
top-down approach using
\href{KBhHIST201IRonWarfare.org}{KBhHIST201IRonWarfare}.

=> Post-German Unification wars, Germany became the largest+strongest
state in Europe

\begin{itemize}
\item \textbf{Otto von Bismarck} lead Prusisa to wage war on neighboring states to
consolidate German territories
\item Bismarck wanted to strengthen the position of Prussia => made the
"Iron and Blood" speech that called for absolute top-down unity of
Germany

\begin{itemize}
\item Advocated for the creation of a German confederation but w/o Austria
=> achieved by small, short wars against decisive land
\item Wanted to remove Austria from the German region so that Prussia
would be the only one that shapes German business

\begin{itemize}
\item Fought the \textbf{Seven Weeks' War} => very, very, very quick
\item CLAIM: won because of INDUSTIRALIZAION! (I suppose
\href{KBhHIST201IRonWarfare.org}{KBhHIST201IRonWarfare}) and the
new military technologies of logistics and better guns
\end{itemize}
\end{itemize}

\item After weakening Austria, the North German Confederation was formed w/
a new parliament and broad suffrage
\end{itemize}
\end{document}
