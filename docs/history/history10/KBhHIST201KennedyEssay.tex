% Created 2021-09-11 Sat 16:40
% Intended LaTeX compiler: xelatex
\documentclass[letterpaper]{article}
\usepackage{graphicx}
\usepackage{grffile}
\usepackage{longtable}
\usepackage{wrapfig}
\usepackage{rotating}
\usepackage[normalem]{ulem}
\usepackage{amsmath}
\usepackage{textcomp}
\usepackage{amssymb}
\usepackage{capt-of}
\usepackage{hyperref}
\usepackage[margin=1in]{geometry}
\usepackage{fontspec}
\usepackage{indentfirst}
\setmainfont[ItalicFont = LiberationSans-Italic, BoldFont = LiberationSans-Bold, BoldItalicFont = LiberationSans-BoldItalic]{LiberationSans}
\newfontfamily\NHLight[ItalicFont = LiberationSansNarrow-Italic, BoldFont       = LiberationSansNarrow-Bold, BoldItalicFont = LiberationSansNarrow-BoldItalic]{LiberationSansNarrow}
\newcommand\textrmlf[1]{{\NHLight#1}}
\newcommand\textitlf[1]{{\NHLight\itshape#1}}
\let\textbflf\textrm
\newcommand\textulf[1]{{\NHLight\bfseries#1}}
\newcommand\textuitlf[1]{{\NHLight\bfseries\itshape#1}}
\usepackage{fancyhdr}
\pagestyle{fancy}
\usepackage{titlesec}
\usepackage{titling}
\makeatletter
\lhead{\textbf{\@title}}
\makeatother
\rhead{\textrmlf{Compiled} \today}
\lfoot{\theauthor\ \textbullet \ \textbf{2021-2022}}
\cfoot{}
\rfoot{\textrmlf{Page} \thepage}
\titleformat{\section} {\Large} {\textrmlf{\thesection} {|}} {0.3em} {\textbf}
\titleformat{\subsection} {\large} {\textrmlf{\thesubsection} {|}} {0.2em} {\textbf}
\titleformat{\subsubsection} {\large} {\textrmlf{\thesubsubsection} {|}} {0.1em} {\textbf}
\setlength{\parskip}{0.45em}
\renewcommand\maketitle{}
\author{Houjun Liu}
\date{\today}
\title{Kennedy Essay}
\hypersetup{
 pdfauthor={Houjun Liu},
 pdftitle={Kennedy Essay},
 pdfkeywords={},
 pdfsubject={},
 pdfcreator={Emacs 27.2 (Org mode 9.4.4)}, 
 pdflang={English}}
\begin{document}

\maketitle


\section{Essay 1: how does Kennedy Hold Up?}
\label{sec:orgba8c369}
\subsection{General Information}
\label{sec:org85aaba9}
\begin{center}
\begin{tabular}{lll}
Due Date & Topic & Important Documents\\
\hline
Wednesday by 11:59pm & Throwing Shade on Kennedy & Kennedy, chapter1.\\
\end{tabular}
\end{center}

\subsection{Prompt}
\label{sec:org271ac16}
In Chapter 1 of \emph{Rise and Fall of the Great Powers}, Paul Kennedy
sketches out an explanation of why the Ming Dynasty was, on the one hand
powerful and prosperous, but ultimately was "a country which had turned
in on itself" and subject to "steady relative decline." Mann, in his
chapter on the Ming trade, gives the reader a lot more detail on the
nuances of Ming history in this period. Putting Kennedy and Mann into
dialogue, does Kennedy's argument still hold up? In your essay, argue
for or against Kennedy's argument using the details of Ming history
analyzed by Mann.

\href{KBhHIST201ChinasDeclineWRTZhengHe.org}{KBhHIST201ChinasDeclineWRTZhengHe}
China's Decline w.r.t. Zheng He

\subsection{Kennedy Analysis}
\label{sec:org117164f}
\begin{quote}
China had decided to turn its back on the world [after Zheng He's
expedition of 1433]\ldots{} There was \ldots{} a plausible strategical reason
for this decision. The northern frontiers of the empire were again
under some pressure from the Mongols \ldots{} Yet [stopping maritime
activity] \ldots{} does not appear to have been reconsidered when the
disadvantages \ldots{} became clear: within a century or so, the Chinese
coastline and even cities on the Yangtze were being attacked by
Japanese pirates\ldots{} Therefore, a key element in China's retreat was
the sheer conservatism of the Confucian bureaucracy \ldots{} In [the
Ming's] "Restoration" atmosphere, the all-important officialdom was
concerned to preserve and recapture the past, not to create a brighter
future.
\end{quote}

\textbf{Things that need to be true}

\begin{enumerate}
\item Japanese pirate's attack could be resolved by opening overseas trade
\item Pressure from the Mongols warrents a reconsidering
\item The past means complete shutdown and against reopening
\end{enumerate}

\textbf{To prove that Kennedy is stupid, we need to prove\ldots{}}

\begin{enumerate}
\item Reopening trade, especially with Zheng He ships, \sout{does not actually
stop costal rebels} does not actually help in the long run
\item \sout{Mongols (and perhaps others?) are actually really bad to warrant a
full, permanent, diversion of resources} Mongol war is, although
important, only a cover for a bigger scene of infighting, so could
not be easily, simply "reconsidered"
\item Trade and maritime trade happening in the past w/ Confucian
government.
\end{enumerate}

\subsection{Claim Synthesis}
\label{sec:org1f11137}
\textbf{Reopening trade does not stop help in the long run}

\begin{itemize}
\item China actually tried this!
@\href{KBhHIST201MannMing.org}{KBhHIST201MannMing}

\begin{itemize}
\item Can't really control rebels from the coast, as Kennedy admitted
\item Coastal commissioner of Fu Jian threw in the towel after trying
everything he could think of
\item Wrote home (Beijing) to trade again, leading to a reliance to Silver
\end{itemize}
\end{itemize}

\href{KBhHIST201ChinasDeclineWRTReopening.org}{KBhHIST201ChinasDeclineWRTReopening}
China's Decline w.r.t Reopening

\begin{quote}
The government reversed course not only because it recognized its
inability to stop smuggling, or because it had begun to appreciate how
much Fujian's populace depended on trade. Beijing had come to realize
that the nation desperately needed the merchants' most important good:
silver.
\end{quote}

\begin{itemize}
\item After some kurfluffle with local currency surrounding
\href{KBhHIST201ChinasDeclineWRTCurrency.org}{KBhHIST201ChinasDeclineWRTCurrency}
China's Decline w.r.t. Currency, became reliant on silver

\begin{itemize}
\item Official currency to unstable due to hyper in/de flation
\item Relied on in-kind and silver payment as stable income store
\end{itemize}

\item As trade is happening, we got a little thing called
\href{KBhHIST201ProblemsWithSilver.org}{KBhHIST201ProblemsWithSilver}
Problems with Silver

\begin{itemize}
\item Large silver store found in the Americas Potosí mines
@\href{KBhHIST201HomogenosceneLN.org}{KBhHIST201HomogenosceneLN}
\item Causing a larger devaluing of the thing
\end{itemize}
\end{itemize}

\begin{quote}
In 1642, so much silver has been produced that its value is falling
even as the mines slacken.
\end{quote}

=> Which, caused a failing of the Chinese economy

\begin{quote}
Like the Spanish king, the Ming emperor backs his military ventures
with Spanish silver, which his subjects must use to pay their taxes.
When the value of silver falls, the government runs out of money.
\end{quote}

So, it doesn't really matter trade/no trade; pirates/no pirates ---
there exists a fundamental problem with the Chinese economy's basis on a
currency so easily influenced by the central government that's still
learning how a globalized economy works.

\textbf{Mongols are pretty darn bad, but ultimately only a scapegote for a
bigger problem}

\begin{itemize}
\item Larger military threats from the Mongols were also paired with party
\emph{infighting}

\begin{itemize}
\item Zheng He became sacrifice for
infighting\href{KBhHIST201ChinasDeclineWRTZhengHe.org}{KBhHIST201ChinasDeclineWRTZhengHe}

\begin{itemize}
\item 朱棣's son aligned with faction opposing him
\item So, he canceled the voyages as a show of strength
\end{itemize}
\end{itemize}
\end{itemize}

\begin{quote}
They had become a target in political infighting---one bureaucratic
faction championing them, another trying to take down the first by
decrying their expense \ldots{} Yongle's son and successor aligned with the
faction that opposed his father's policies.
\end{quote}

Indeed, Mongols are a perennial problem.

\begin{itemize}
\item Mongols never really left

\begin{itemize}
\item Strong milltary holds lead to extended fights
\item Eventually, we ended up with the Manchus leading the country
\end{itemize}
\end{itemize}

\begin{quote}
The Manchus were descended from the Jurchen Jin dynasty rulers who had
taken north China during Song times. The Manchu forces were the most
capable and well-disciplined in the empire.
\end{quote}

\begin{itemize}
\item Mongol attacks
\item And later on Manchu attacks
\item Elevate
\end{itemize}

But! It's not even like that's the problem, anyway.

The mongol attacks, as admitted by Kennedy, are pretty bad. However,
they \emph{never really left}, so its not a one-time problem that could, as
per the language of Kennedy, simply be "reconsidered" after its, as
inferred by Kennedy, \emph{delt with.}

Furthermore, saying that the voyages' costs are too high are a mask for
a bigger (or, in some respects, smaller?) problem --- that its simply
political tribalism within the Ming court decrying the prices of these
attacks trying to hide the tribalism itself; so\ldots{} this would no be
solved even after the Mongols are gone --- they will simply find
something else to blame or the emperor will just\ldots{} die and switch
faction.

\textbf{Conservative Confucian government ≠ No Trade!}

Han Salt and Iron Debate => "good ol' times" according to Kennedy, no?
81 AD

\begin{itemize}
\item Confucians actually arguing \emph{against} central control and monopolies
\end{itemize}

\begin{quote}
The purpose of merchants is circulation and the purpose of artisans is
making tools. These matters should not become a major concern of the
government.
\end{quote}

\begin{itemize}
\item Kennedy indeed \emph{misinterpreted} the Confucian view

\begin{itemize}
\item \emph{Governments} should not be \textbf{concerned} with trade
\item Natural trade --- without an emphasis on profit from the government
--- should neither be encouraged nor disencouraged
\item Does not mean monopoly and limit
\end{itemize}
\end{itemize}

\begin{quote}
When profit is not emphasized, civilization flourishes and the customs
of the people improve. \ldots{} At present the government ignores what
people have and exacts what they lack. The common people then must
sell their products cheaply to satisfy the demands of the government.
\ldots{} Quick traders and unscrupulous officials buy when goods are cheap
in order to make high profits.
\end{quote}

\begin{itemize}
\item Confucian tenant is not to limit trade, Kennedy Style, instead, focus
on limitation of market forces

\begin{itemize}
\item Additional restrictions on trade causes citizens to pander to the
trader + causing hyperdeflation => this actually happened!

\begin{itemize}
\item China closed trade after Zheng He, smugglers became rampant
\item Even Kennedy acknowledged this\ldots{}

\begin{itemize}
\item Fujian actively invited smugglers to sustain trade
\end{itemize}
\end{itemize}
\end{itemize}
\end{itemize}

\begin{quote}
Fujian depended on the sea \ldots{} when international trade was officially
banned, Fujianese found themselves in an uncomfortable
position---there was nothing for them on land \ldots{} [So] Fujianese
traders invited three thousand Japanese and Portuguese smugglers to
reoccupy the former Dutch base at Wu Island.
\end{quote}

\begin{itemize}
\item Market economy, and \emph{not morality} takes over, against Confucian
principles
\end{itemize}

Kennedy incorrectly interpreted the viewpoint of trade from the
Confucian perspective. Through the Salt and Iron debates, Confucian
scholars indeed is trying to convince a struggling Han dynasty \emph{from}
imposing a strict trade limitation. This is because the central
Confucian tenant of \emph{morality} warrents merchants to stay in their place
--- promoting circulation --- and not driving market economies. Through
imposing a trade ban, the Confucian philosophers rightly believe that it
would cause a hyperdeflation --- causing merchants to have more
opportunity to "wrongly" prosper.

\subsection{Defluffifying}
\label{sec:orgb0908c8}
Again, here's Kennedy:

\begin{quote}
China had decided to turn its back on the world [after Zheng He's
expedition of 1433]\ldots{} There was \ldots{} a plausible strategical reason
for this decision. The northern frontiers of the empire were again
under some pressure from the Mongols \ldots{} Yet [stopping maritime
activity] \ldots{} does not appear to have been reconsidered when the
disadvantages \ldots{} became clear: within a century or so, the Chinese
coastline and even cities on the Yangtze were being attacked by
Japanese pirates\ldots{} Therefore, a key element in China's retreat was
the sheer conservatism of the Confucian bureaucracy \ldots{} In [the
Ming's] "Restoration" atmosphere, the all-important officialdom was
concerned to preserve and recapture the past, not to create a brighter
future.
\end{quote}

/In Kennedy's claim of China relying on conservative Confucianism being
the driving force behind China's closing of its maritime activities, the
author fails to recognize the difficulty of sustaining trade in a
falling economy, political infighting blamed on strengthening rebels,
and in fact misinterpreates the Confucian philosophies of trade ---
creating a faulty argument/

\begin{itemize}
\item Economic reopening can't save China in the long run in a failing
economy, making reopening less effective as claimed
\item Mongols are threat that never was "resolved", and hence here used as a
scapegoat for bigger political infighting
\item Confucianism embraced trade as a means to allow goods circulation, so
emulating traditional Confucian philosophy does not mean shutting
trade down
\end{itemize}

Now, defluffify by re-writing the three points + so what in as little
words as possible.

\begin{html}
<!--
    \textbf{\textbf{In Robert Kennedy's claim of the traditionalist Confucian reasoning behind the halting of maritime activities by the Ming government after Zheng He's voyage of 1433, Kennedy cast a faulty argument that ignores both the ineffectiveness of reopening trade in a failing global economy and the Ming court's usage of Mongolian invasions as scapegoat for political infighting; through his argument, Kennedy also misinterpreted the tenet of Confucian philosophy that favors market circulation and advises governments not to encourage price racketeering as a complete ideological ban on trade.}}
-->
\end{html}

*In Kennedy's claim arguing that the conservative Confucian views caused
Ming's closing of maritime trade, he presents a faulty argument that
ignores both the ineffectiveness of unnecessarily reopening maritime
trade and the discord in the infighting-plagued Ming government; through
his argument, Kennedy also misinterpreted Confucian philosophy central
to his argument for it, instead of an ideological ban on trade, actually
favors market circulation and advises governments not to encourage price
racketeering.*

\begin{itemize}
\item Explain these things actually say about Kennedy => "We should not
listen to Kennedy" => push more towards why the rest of Kennedy's
Europe argument is faulty
\item Or, push an alternative reasoning
\item Thesis too specific
\end{itemize}

\href{KBhHIST201EssayBackup.org}{KBhHIST201EssayBackup}

\noindent\rule{\textwidth}{0.5pt}

There is always
\href{https://wp.ucla.edu/wp-content/uploads/2016/01/UWC\_handouts\_What-How-So-What-Thesis-revised-5-4-15-RZ.pdf}{UCLA
Writing Lab}
\end{document}
