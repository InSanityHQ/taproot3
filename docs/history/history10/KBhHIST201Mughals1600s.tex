% Created 2021-09-27 Mon 12:00
% Intended LaTeX compiler: xelatex
\documentclass[letterpaper]{article}
\usepackage{graphicx}
\usepackage{grffile}
\usepackage{longtable}
\usepackage{wrapfig}
\usepackage{rotating}
\usepackage[normalem]{ulem}
\usepackage{amsmath}
\usepackage{textcomp}
\usepackage{amssymb}
\usepackage{capt-of}
\usepackage{hyperref}
\setlength{\parindent}{0pt}
\usepackage[margin=1in]{geometry}
\usepackage{fontspec}
\usepackage{svg}
\usepackage{cancel}
\usepackage{indentfirst}
\setmainfont[ItalicFont = LiberationSans-Italic, BoldFont = LiberationSans-Bold, BoldItalicFont = LiberationSans-BoldItalic]{LiberationSans}
\newfontfamily\NHLight[ItalicFont = LiberationSansNarrow-Italic, BoldFont       = LiberationSansNarrow-Bold, BoldItalicFont = LiberationSansNarrow-BoldItalic]{LiberationSansNarrow}
\newcommand\textrmlf[1]{{\NHLight#1}}
\newcommand\textitlf[1]{{\NHLight\itshape#1}}
\let\textbflf\textrm
\newcommand\textulf[1]{{\NHLight\bfseries#1}}
\newcommand\textuitlf[1]{{\NHLight\bfseries\itshape#1}}
\usepackage{fancyhdr}
\pagestyle{fancy}
\usepackage{titlesec}
\usepackage{titling}
\makeatletter
\lhead{\textbf{\@title}}
\makeatother
\rhead{\textrmlf{Compiled} \today}
\lfoot{\theauthor\ \textbullet \ \textbf{2021-2022}}
\cfoot{}
\rfoot{\textrmlf{Page} \thepage}
\renewcommand{\tableofcontents}{}
\titleformat{\section} {\Large} {\textrmlf{\thesection} {|}} {0.3em} {\textbf}
\titleformat{\subsection} {\large} {\textrmlf{\thesubsection} {|}} {0.2em} {\textbf}
\titleformat{\subsubsection} {\large} {\textrmlf{\thesubsubsection} {|}} {0.1em} {\textbf}
\setlength{\parskip}{0.45em}
\renewcommand\maketitle{}
\author{Houjun Liu}
\date{\today}
\title{Mughals in the 1600s}
\hypersetup{
 pdfauthor={Houjun Liu},
 pdftitle={Mughals in the 1600s},
 pdfkeywords={},
 pdfsubject={},
 pdfcreator={Emacs 28.0.50 (Org mode 9.4.4)}, 
 pdflang={English}}
\begin{document}

\tableofcontents



\section{Mughals in the 1600s}
\label{sec:org01db6e3}
\#flo \#disorganized

\begin{itemize}
\item Akbar's rule of the Mughals

\begin{itemize}
\item Tried to align his subject's interests to the Mughal's interest
\item Goal was to maintain adequate compensation and preventing officials
form unjust enrichment by overtaxing the peasants who could then not
work on government projects

\begin{itemize}
\item Directed administration to award land revenue salary instead of
assigning land
\item Value of government official based on cost w.r.t. operating
military men

\begin{itemize}
\item a local commander = 500 men
\item a provincial government = 5000 men
\end{itemize}

\item First model of separation of powers between government and
military
\item Prevented financial corruptiona

\begin{itemize}
\item Made constant transfers and deferrals
\item Prevented passing on of wealth to offspring
\end{itemize}

\item His minister, Todar Mal, made tax collection proportional to value
generation

\begin{itemize}
\item So bad crop year could pay less tax
\item Prevented overluxuriation and benefitted peseants
\end{itemize}
\end{itemize}

\item Favored appointement of native born over foreign --- due to pledges
of loyalty to Mughal state: promoting religious indignity
\end{itemize}

\item CLAIM: Melded together Mughal and indigenous elites

\begin{itemize}
\item Encouraged intermarriage
\item Reformes aimed at selling Mughal to other people
\item All official appointments are treated as gifts from the emperor
\item Gave grants to Muslim and Hindu cultural institutions
\item Supported the arts and sciences
\end{itemize}

\item Akbar made empire popular by pimping it up

\begin{itemize}
\item Improved living quaters
\item Regulation of school
\item System of laws
\item Challenged the patriarchy and improved the role of woman
\item Discouraged child marriages and encouraged remarrying of widowns
\end{itemize}

\item Intended all of his social reform to support his object of
\emph{sulh-i-kul} => universal harmony. (Not a fan of \emph{raison d'etat, I
see})

\item The Porchuguese

\begin{itemize}
\item Acted as middlemen between Venitians, Arabs, and Turks

\begin{itemize}
\item Traded spices and cotton
\item Served as foundation of Western medicines
\item De Gama's Explorations

\begin{itemize}
\item Invaded port of good hope in Africa

\begin{itemize}
\item Disguised as Muslim traders
\item When about to be kicked out, fought and burned the city
\end{itemize}

\item Zamorin also got scammed by de Gama too

\begin{itemize}
\item Convinced that he was a pirate
\item Did not drive off, and permitted to trade
\item Kidnapped some local fishermen for crew along the way
\end{itemize}
\end{itemize}

\item Evenutally, setup a larger network of trade
\end{itemize}
\end{itemize}

\item Mughals saw the Porchuguese, and wanted to curtail them

\begin{itemize}
\item Resorted to a model of compromise => “Gave free passage to ships in
exchange for pilgrims on their way to Mecca
\item Porchuguese, Dutch, and English collectively tried te “interfere in
international shiping

\begin{itemize}
\item Seisure of a ship by practicing Hindu
\end{itemize}

\item Mughals eventually partnered with english and dutch to try to
curtail the porchuguese and create competiton
\item English and Dutch both adopted the porchuguese model
\item Mugals tried to create strategic partnership with European cultures

\begin{itemize}
\item Ordered christian symbolism to be painted
\item Europeans impressed with Mugal style that Mogul became associated
with power
\end{itemize}
\end{itemize}

\item Aurangzeb's rule

\begin{itemize}
\item Orthodox muslim

\begin{itemize}
\item Took religious values over tradition
\item Dismaltiled Mughal's multicultralism

\begin{itemize}
\item Banned music and dance
\item Enforced islam codes of pubilc conduct via censors
\item Halt constructions of new Hindu temples
\item Attack established structuers
\item Reimposed the jizya payment in leu of state service that is
demanded from non-muslims
\item Enforced system of jizya payments that had to be done while
chanting about inferoity
\item Opposed appointing hindus to highest ranks
\end{itemize}

\item Hindus and other non-muslim cultural icons lamented this
\item Shah Janan's army campaigns increased tax revenue to meet higher
expectations

\begin{itemize}
\item Hindu agricultural exploration fell hardest
\item Shivaji contradicted the Mughal court

\begin{itemize}
\item Which means, he got quickly struck down
\item Escaped the court, and went to the Marathas
\item 1674-1680 started invading Mughals in gurilla campaigns
\item Which, is a self-deprecating loop --- causing Auranzb to
invest even MORE moneyon fighting

\begin{itemize}
\item Created the Marathas empire
\end{itemize}
\end{itemize}

\item Muhammed Akbar opposed his fathers rules

\begin{itemize}
\item Fled to Arabia
\item Tortured and killed son of Shiviaji
\end{itemize}
\end{itemize}

\item Also started a struggle with the Briting East India Company

\begin{itemize}
\item Started complaining of higher taxes
\item Interpreted their license to say that they would only need to
pay taxes at major international ports
\item In defense, the company declared war against the Mughals
\item The Mughals retailated by destroying corporate stations
\item Eventually forced negotiations to sink Muslim ships bound for
Mecca
\item Eventually forced back into trade negotiations after a larger
fine
\item This incident humiliated both the Mughals and the company ---
displeasing European directors
\end{itemize}

\item Fights of independence broke out amoung the Marathas and the
Hindu-predominant north between 1674-1680
\end{itemize}

\item In the end, Aurangzeb ended his life noting "I don't know who I am,
nor what I have been doing"
\item Marathas won great parts of Mughal territory

\begin{itemize}
\item Empire's rulers force to pay tribute to Marathas
\item Others paid a largely symbolic to the Mughals
\end{itemize}
\end{itemize}
\end{itemize}

\noindent\rule{\textwidth}{0.5pt}

\#disorganized \#flo

\begin{itemize}
\item India is very hard to rule in a very centralized way

\begin{itemize}
\item Deccan plateu to the south

\begin{itemize}
\item Hard to conquer
\item Have to re-conquer because people did not respect his rule
\end{itemize}

\item Gangatic plain to the north

\begin{itemize}
\item Easy to conquer
\end{itemize}

\item Rajputs

\begin{itemize}
\item Small city-states with isolated rules
\item With independent principalities
\end{itemize}
\end{itemize}
\end{itemize}

And now, we are comparing palaces?

\begin{itemize}
\item Chende constructed to match people's religions

\begin{itemize}
\item Instead of meeting people in the impressively Han seat of power,
meet at a more relaxed place
\item Both asserted power in the main palace and appeased ethnicity in the
summer palace
\end{itemize}

\item Agra constructed to be a mix of religions

\begin{itemize}
\item Instead of meeting people in the strictly Muslim seat of power, meet
at a more\ldots{}. fortified place
\item Both asserted power in to Tokata and appeased ethnicity in the
summer palace
\end{itemize}
\end{itemize}
\end{document}
