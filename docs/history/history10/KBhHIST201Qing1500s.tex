% Created 2021-09-27 Mon 12:00
% Intended LaTeX compiler: xelatex
\documentclass[letterpaper]{article}
\usepackage{graphicx}
\usepackage{grffile}
\usepackage{longtable}
\usepackage{wrapfig}
\usepackage{rotating}
\usepackage[normalem]{ulem}
\usepackage{amsmath}
\usepackage{textcomp}
\usepackage{amssymb}
\usepackage{capt-of}
\usepackage{hyperref}
\setlength{\parindent}{0pt}
\usepackage[margin=1in]{geometry}
\usepackage{fontspec}
\usepackage{svg}
\usepackage{cancel}
\usepackage{indentfirst}
\setmainfont[ItalicFont = LiberationSans-Italic, BoldFont = LiberationSans-Bold, BoldItalicFont = LiberationSans-BoldItalic]{LiberationSans}
\newfontfamily\NHLight[ItalicFont = LiberationSansNarrow-Italic, BoldFont       = LiberationSansNarrow-Bold, BoldItalicFont = LiberationSansNarrow-BoldItalic]{LiberationSansNarrow}
\newcommand\textrmlf[1]{{\NHLight#1}}
\newcommand\textitlf[1]{{\NHLight\itshape#1}}
\let\textbflf\textrm
\newcommand\textulf[1]{{\NHLight\bfseries#1}}
\newcommand\textuitlf[1]{{\NHLight\bfseries\itshape#1}}
\usepackage{fancyhdr}
\pagestyle{fancy}
\usepackage{titlesec}
\usepackage{titling}
\makeatletter
\lhead{\textbf{\@title}}
\makeatother
\rhead{\textrmlf{Compiled} \today}
\lfoot{\theauthor\ \textbullet \ \textbf{2021-2022}}
\cfoot{}
\rfoot{\textrmlf{Page} \thepage}
\renewcommand{\tableofcontents}{}
\titleformat{\section} {\Large} {\textrmlf{\thesection} {|}} {0.3em} {\textbf}
\titleformat{\subsection} {\large} {\textrmlf{\thesubsection} {|}} {0.2em} {\textbf}
\titleformat{\subsubsection} {\large} {\textrmlf{\thesubsubsection} {|}} {0.1em} {\textbf}
\setlength{\parskip}{0.45em}
\renewcommand\maketitle{}
\author{Houjun Liu}
\date{\today}
\title{Manchus}
\hypersetup{
 pdfauthor={Houjun Liu},
 pdftitle={Manchus},
 pdfkeywords={},
 pdfsubject={},
 pdfcreator={Emacs 28.0.50 (Org mode 9.4.4)}, 
 pdflang={English}}
\begin{document}

\tableofcontents



\section{Qing}
\label{sec:org56be547}
\#flo \#disorganized

\begin{itemize}
\item The Manchus!

\begin{itemize}
\item Powerful army took China

\begin{itemize}
\item Professional military organized under 8 banners
\item Took Beijing
\item Restored order
\item Proclaimed that the mandate passed to them
\end{itemize}

\item Assured that Chinese culture would continue, but those who resisted
are punished

\begin{itemize}
\item Yangzhou refused to surrender
\item So Manchus took the city and instantiated the purge
\end{itemize}
\end{itemize}

\item Hairstyle submission

\begin{itemize}
\item Forced Chinese men to submit to a Manchu hairstyle
\item Present symbol of Manchu rule
\item Took a whole generation to solidify rule
\end{itemize}

\item Three great emperors

\begin{itemize}
\item Kangxi Emperor

\begin{itemize}
\item One of the most effective rulers of China
\item Held the throne for 60 years
\item Financials

\begin{itemize}
\item Froze tax assessment in 1712
\item Made tax increase no longer a threat
\end{itemize}

\item Regions

\begin{itemize}
\item Extended the empire northward + establish borders with Korea +
Russia
\item Lead campaigns against Mongols and occupied Tibet
\end{itemize}

\item CLAIM: why he was great

\begin{itemize}
\item Great guy

\begin{itemize}
\item Dilligent
\item Good judge of character + warrented honest answers
\item Did not fight Ming loyalists as long as they break no laws
\end{itemize}

\item Promoted liberal arts

\begin{itemize}
\item Held examinations to promote scholars
\item Patronized art, philosophy, and poetry
\item Interested in Western learning

\begin{itemize}
\item Learned through Jesuit missionaries
\item Jesuits saw worship as a ceremony and not rites
\item However, was not fully accepted by the Emperor after the
early 18th century
\end{itemize}
\end{itemize}
\end{itemize}
\end{itemize}

\item Yongzheng Emperor

\begin{itemize}
\item More guarded and suspicious than Kangxi
\item Anti-corruption efforts

\begin{itemize}
\item Expanded secret memorial system
\item A new tax reform that prevented tax evasion
\end{itemize}
\end{itemize}

\item Qianlong Emperor

\begin{itemize}
\item Reigned for 60 years
\item Emulated Kangxi

\begin{itemize}
\item Intensified Qing involvement in Tibet
\item Expanded into Turkestan
\item Patron of culture and arts

\begin{itemize}
\item Compiled collection of Chinese work
\item Supressed anti-Manchu, anti-Confucion, and heretics by burning
them
\end{itemize}
\end{itemize}
\end{itemize}
\end{itemize}

\item Extended Chinese model of leadership + united the Chinese Mongols
Uighurs and Tibetans
\item 18s Century

\begin{itemize}
\item Happy times
\item Prosperous and peaceful
\item Conservatively confusion
\item Two great novels written
\end{itemize}

\item Beginning of decline

\begin{itemize}
\item Governmert did not keep pace with rapid population growth
\item Qianlong became fond of his bodyguards, who embezzled silver

\begin{itemize}
\item CLAIM: this is an early sign of decline
\end{itemize}

\item Continuous military campains eventually lead to near bankrupcy
\end{itemize}
\end{itemize}
\end{document}
