% Created 2021-09-12 Sun 22:48
% Intended LaTeX compiler: xelatex
\documentclass[letterpaper]{article}
\usepackage{graphicx}
\usepackage{grffile}
\usepackage{longtable}
\usepackage{wrapfig}
\usepackage{rotating}
\usepackage[normalem]{ulem}
\usepackage{amsmath}
\usepackage{textcomp}
\usepackage{amssymb}
\usepackage{capt-of}
\usepackage{hyperref}
\usepackage[margin=1in]{geometry}
\usepackage{fontspec}
\usepackage{indentfirst}
\setmainfont[ItalicFont = LiberationSans-Italic, BoldFont = LiberationSans-Bold, BoldItalicFont = LiberationSans-BoldItalic]{LiberationSans}
\newfontfamily\NHLight[ItalicFont = LiberationSansNarrow-Italic, BoldFont       = LiberationSansNarrow-Bold, BoldItalicFont = LiberationSansNarrow-BoldItalic]{LiberationSansNarrow}
\newcommand\textrmlf[1]{{\NHLight#1}}
\newcommand\textitlf[1]{{\NHLight\itshape#1}}
\let\textbflf\textrm
\newcommand\textulf[1]{{\NHLight\bfseries#1}}
\newcommand\textuitlf[1]{{\NHLight\bfseries\itshape#1}}
\usepackage{fancyhdr}
\pagestyle{fancy}
\usepackage{titlesec}
\usepackage{titling}
\makeatletter
\lhead{\textbf{\@title}}
\makeatother
\rhead{\textrmlf{Compiled} \today}
\lfoot{\theauthor\ \textbullet \ \textbf{2021-2022}}
\cfoot{}
\rfoot{\textrmlf{Page} \thepage}
\titleformat{\section} {\Large} {\textrmlf{\thesection} {|}} {0.3em} {\textbf}
\titleformat{\subsection} {\large} {\textrmlf{\thesubsection} {|}} {0.2em} {\textbf}
\titleformat{\subsubsection} {\large} {\textrmlf{\thesubsubsection} {|}} {0.1em} {\textbf}
\setlength{\parskip}{0.45em}
\renewcommand\maketitle{}
\author{Houjun Liu}
\date{\today}
\title{Raison d'etat}
\hypersetup{
 pdfauthor={Houjun Liu},
 pdftitle={Raison d'etat},
 pdfkeywords={},
 pdfsubject={},
 pdfcreator={Emacs 28.0.50 (Org mode 9.4.4)}, 
 pdflang={English}}
\begin{document}

\maketitle


\section{Raison d'etat}
\label{sec:orga3a0e4d}
\begin{quote}
Each state depended on the other. The well being of the state
justified whatever means were employed to further it. The national
interest supplanted the medieval notion of a universal morality. ---
KBhHIST201Kissinger
\end{quote}

A method of "sensible government" that promises to set aside personal
ideological differences for the betterment of the country as a whole.

This makes politics non-secular, which means\ldots{} Nonsecular wars less
violent than holy wars because CLAIM
\href{KBhHIST201Kissinger.org}{KBhHIST201Kissinger}: they did not
involve emotion

\subsection{Exhibit A: France!}
\label{sec:org06eb603}
\href{KBhHIST201FrenchRichelieuAndRaisonDeEtat.org}{KBhHIST201FrenchRichelieuAndRaisonDeEtat}

\subsection{Exhibit B: Federick the Great}
\label{sec:org1177a9b}
CLAIM @\href{KBhHIST201Kissinger.org}{KBhHIST201Kissinger}: Federick
the Great's decision to invade Silesia was pure strategy move

\begin{enumerate}
\item Conquest made Prussia a "\emph{bona-fide} Great Power"
\item Prussia joined by France, Spain, etc. in war of 1740-1748
\item In 1756-1763, switched sides
\end{enumerate}

CLAIM @\href{KBhHIST201Kissinger.org}{KBhHIST201Kissinger}: the
side-switching was a pure result of calculations of benefit

\subsection{Failure and Overextension}
\label{sec:org27c6d0a}
CLAIM @\href{KBhHIST201Kissinger.org}{KBhHIST201Kissinger} --- too
much power without morality is no good, for instance

\subsubsection{Exhibit B: Still France!}
\label{sec:orgb0a8763}
\begin{itemize}
\item \textbf{Louis XIV}, under the guidance of
\href{KBhHIST201FrenchRichelieuAndRaisonDeEtat.org}{KBhHIST201FrenchRichelieuAndRaisonDeEtat},
gone trigger happy on the expansion
\item Ultimately, this is detrimental

\begin{itemize}
\item When most states starts being fully rational and not at all moral,
this becomes less fun
\item If no one else is expanding, a country will keep taking advantage of
others, which\ldots{} does not make you a lot of friends
\end{itemize}
\end{itemize}

\begin{quote}
Under Raison d'etat, "The stronger would seek to dominate, and the
weaker would resist by forming coalitions to augment their individual
strengths"
\end{quote}
\end{document}
