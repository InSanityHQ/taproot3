% Created 2021-09-12 Sun 22:48
% Intended LaTeX compiler: xelatex
\documentclass[letterpaper]{article}
\usepackage{graphicx}
\usepackage{grffile}
\usepackage{longtable}
\usepackage{wrapfig}
\usepackage{rotating}
\usepackage[normalem]{ulem}
\usepackage{amsmath}
\usepackage{textcomp}
\usepackage{amssymb}
\usepackage{capt-of}
\usepackage{hyperref}
\usepackage[margin=1in]{geometry}
\usepackage{fontspec}
\usepackage{indentfirst}
\setmainfont[ItalicFont = LiberationSans-Italic, BoldFont = LiberationSans-Bold, BoldItalicFont = LiberationSans-BoldItalic]{LiberationSans}
\newfontfamily\NHLight[ItalicFont = LiberationSansNarrow-Italic, BoldFont       = LiberationSansNarrow-Bold, BoldItalicFont = LiberationSansNarrow-BoldItalic]{LiberationSansNarrow}
\newcommand\textrmlf[1]{{\NHLight#1}}
\newcommand\textitlf[1]{{\NHLight\itshape#1}}
\let\textbflf\textrm
\newcommand\textulf[1]{{\NHLight\bfseries#1}}
\newcommand\textuitlf[1]{{\NHLight\bfseries\itshape#1}}
\usepackage{fancyhdr}
\pagestyle{fancy}
\usepackage{titlesec}
\usepackage{titling}
\makeatletter
\lhead{\textbf{\@title}}
\makeatother
\rhead{\textrmlf{Compiled} \today}
\lfoot{\theauthor\ \textbullet \ \textbf{2021-2022}}
\cfoot{}
\rfoot{\textrmlf{Page} \thepage}
\titleformat{\section} {\Large} {\textrmlf{\thesection} {|}} {0.3em} {\textbf}
\titleformat{\subsection} {\large} {\textrmlf{\thesubsection} {|}} {0.2em} {\textbf}
\titleformat{\subsubsection} {\large} {\textrmlf{\thesubsubsection} {|}} {0.1em} {\textbf}
\setlength{\parskip}{0.45em}
\renewcommand\maketitle{}
\author{Houjun Liu}
\date{\today}
\title{Unit 2 French Revolution Essay}
\hypersetup{
 pdfauthor={Houjun Liu},
 pdftitle={Unit 2 French Revolution Essay},
 pdfkeywords={},
 pdfsubject={},
 pdfcreator={Emacs 28.0.50 (Org mode 9.4.4)}, 
 pdflang={English}}
\begin{document}

\maketitle


\section{Unit 2 Essay}
\label{sec:org2cdac09}
\subsection{General Information}
\label{sec:orgb1706cf}
\begin{center}
\begin{tabular}{lll}
Due Date & Topic & Important Documents\\
\hline
Nov 9th, a Monday & (ski-pie-di-scraper) BOP BOP BOP & Henry the Kissinger, Mason, Roberts, Surprise Kennedy, Surprise Trauttmann\\
\end{tabular}
\end{center}

\subsection{Prompt}
\label{sec:org53ce25a}
 In the end, did the European balance of power succeed in its goal to,
as Kissinger puts it, limit conflict and produce the "best possible
outcome" from flawed human nature? Or did it magnify conflict and
increase the likelihood of global war?

\#\# Quotable Quotes

\begin{itemize}
\item Kissinger

\begin{itemize}
\item "If the circumstances were reversed, we could equally be pro-German
and anti-French." A1
\item "William was perfectly willing to negotiate with Louis XIV when he
felt the balance of power could best be served by doing so." A2
\item "Agreeing on the importance of the balance of power did not,
however, still British disputes about the best strategy to implement
the policy. \ldots{} The Whigs argued that GB should engage only when
balance threatened, and only long enough to remove threat; Tories
argued that GB's duty was to \emph{shape} the balance of power." A3
\item CLAIM: "Of course, in the end a balance of power always comes about
de facto when several states interact." A4
\item "Power is too difficult to assess, and the willingness to vindicate
it too various, to permit treating it as a reliable guide to
international order. \ldots{} The balance of power inhibits the capacity
to overthrow the international order" A5
\item "Every king consoled himself with the thought that strengthening his
own rule was the greatest possible contribution to the general
peace, and left it to the ubiquitous invisible hand to justify his
exertions - without limiting his ambitions." A6
\item "But Louis XN gained no peace. of mind from security; he saw in it
an opportunity for conquest. In his overzealous pursuit of raison
d'etat, LouisXN alarmed the rest of Europe and brought together an
anti-French coalition which, in the end, thwarted his design." A7
\item "Thereby, France lost the advantage of having adversaries
constrained by moral considerations \ldots{} Once all states played by
the same rules, gains became much more difficult to achieve" A8
\end{itemize}

\item Roberts

\begin{itemize}
\item "In the next four centuries, Christianity was often to have
disastrous effects. Confident in the possession of the true
religion, Europeans were impatient and contemptuous of the values
and achievements of the peoples and civilizations they disturbed."
A9
\item "Greed quickly led to the abuse of power, to domination and
exploitation by force. In the end this led to great crimes - though
they were often committed unconsciously." A10
\item "Europeans could usually exact what they wanted in the end because
of a technical superiority which exaggerated the power of their tiny
numbers and for a few centuries turned the balance against the great
historic agglomerations of population and civilization." A11
\item "The obscuring of the Company's primary commercial role was not good
for business. It also gave its employees even greater opportunities
to feather their own nests. \ldots{} the British government hoped as
fervently as the Company to avoid being dragged any further into the
role of imperial power in India." A12
\item "The huge and growing Caribbean market for slaves and imported
European goods was added to that already offered by a Spanish empire
increasingly unable to defend its economic monopoly. This fixed the
role of the West Indies in the relationships of the European powers
for the next century. \ldots{} They were long a prey to disorder" A13
\item "Gradually, the great powers fought out their disputes until they
arrived at acceptable agreements, but this was to take a long time."
A14
\item "All the colonial powers had, by the eighteenth century, been able
to extract some economic profit from their colonies" A15
\item "Europeans did not just conquer; they exterminated local cultures
and peoples and replaced them with their own" A16
\item "Older cultures were to be cut off from populating the new worlds or
setting their mark on them" A17
\item "The French were Great Britain's most dangerous potential
competitors, but their government was always likely to be distracted
by its European continental commitments." A18
\item "The British lacked missionary zeal \ldots{} they had no immediate urge
to interfere with the native custom or institution" A19
\item "Before its outbreak, there had in fact been no remission of
fighting in India, even while France and Great Britain were
officially at peace after 1748." A20
\item "The possession of a station at Calcutta placed them at the door to
that part of India which was potentially the richest prize - Bengal
and the lower Ganges valley." A21
\end{itemize}

\item Mason

\begin{itemize}
\item "A long struggle between Parliament and the Stuart kings [in
England?] and essentially replaced the absolute monarchy with a
constitutional monarchy" A22
\item "'Sovereign power resides in my person alone \ldots{}. It is from me
alone that my policies take their existence and their authority'"
A23
\item "European monarchies had consciously pursued a policy of the balance
of power, a system of shifting international alliances that
prevented any one country from becoming too powerful" A24
\item "Wars were fought not so much for ideology or nationalism but to
maintain the balance of power \ldots{} The victor did not want to crush
the vanquished" A25
\item "Napoleon formed mass armies and led them into other countries to
spread the ideas of the Revolution and to enhance his own power" A26
\item "The French Revolution and the Napoleonic wars had unleashed forces
that would shake the foundations of European society." A27
\item "Napoleon had changed the nature of warfare in Europe by
conscripting huge armies and infusing them with a commitment to
fight for France and for 'liberty, equality, and fraternity,' the
slogan of the Revolution." A28
\item "The victor did not want to crush the vanquished, as this would
upset the balance; in any case, the defeated state might be a future
ally." A29
\item "Louis XVIII issued a constitutional charter that incorporated many
of the changes that had entered into French life and society since
1789" A30
\item "An inefficient system of taxation made it difficult for the
monarchy to raise the money it needed. \ldots{} Between 1726 and 1789,
the cost of living increased by 62 percent, whereas wages rose by
only 25 percent." A31
\item "Napoleon's military fortunes began to wane \ldots{} The allied armies
pressed on, entered Paris, and forced Napoleon to abdicate" A32
\item "By giving free rein to individual greed and the private
accumulation of wealth, the"invisible hand” of the market would
benefit society in the end” A37
\end{itemize}

\item Our Surprise Friend Kennedy

\begin{itemize}
\item Europe had always been politically fragmented, despite even the best
efforts of the Romans A33 => Europe is politically fragmented
\item For this political diversity Europe had largely to thank its
geography. There were no enormous plains over which an empire of
horse men could impose its swift dominion; nor were there broad and
fertile zones \ldots{} providing the food for masses of toiling and
easily conquerable peasants A34 => Europe is A33 because of its
unique geography
\item The fact was that in Europe there were always some princes and local
lords willing to tolerate merchants and their ways even when others
plundered and expelled them A35 => Europe had mixed variety of local
ideals that often conflicted with each other
\end{itemize}

\item Trauttmann, another big surprise

\begin{itemize}
\item The Company had in fact aimed at territorial dominion, following the
Dutch mode A36 => In India, England (the Mediator!) elected to
control land instead of doing balance-of-power
\end{itemize}

\item Balance of power only encourages fighting when provoked

\item Entropy, so people provoke

\item But ultimately its more stable than old world empires

\item It encourages fighting, but that's unavoidable

\begin{itemize}
\item Competing resources
\item Competing land
\item Competing trade
\end{itemize}

\item BOP naturally create conflict

\item But these conflicts are still better/necessary than normal war

\item BP 2

\begin{itemize}
\item Napolean
\item England, too + their overseas
\end{itemize}

\item Better BP3 =>

\item Perhaps Roman context?

\item France gave up on India after it lost the balance of Power
\end{itemize}

\subsection{Idea Snippets Gathering}
\label{sec:org7dc80be}
\begin{itemize}
\item Unquoted

\begin{itemize}
\item old France got power from the King alone A23
\item European's colonization had strategic advantage A17
\item HUN? Spain's cat-and-mouse game with defending its monopoly fixed
the contry's stance in the Indeas A13
\item Napoleon changed wars into a war of ideology A28
\item Britian's aquisition of Calcutta train station was beneficial A21
\item Good outcome for future collab: when interest is served, working
with adversary is possible A2. Did not want to crush enemy in hopes
of not hurting future collab A29.
\end{itemize}

\item wars are fought for BOP A25

\begin{itemize}
\item Europe's technical superiority turned the balance against historical
civilzations A11
\item every king believes that acting in his self intrest strengthened
general peace A6
\item greed lead to unconcious crimes A10
\item too many countries Raisoning may lead to countries unrestrained by
morality A8
\item Before the 7 years war, Britian and France still fought even
nominally at peace A20
\item all the colonial powers had, by the eighteenth century, been able to
extract some economic profit from their colonies A15
\end{itemize}

\item balance of power whether ye wanted or not A4

\begin{itemize}
\item No agreement? Fight! Eventually there will be one. A14
\item Europeans intentionally set up goverment to prevent one from being
too powerful A24
\item Napoleon being napolean

\begin{itemize}
\item France is looking pretty grim A31
\item napolean tried to spread ideas of the revolution A26
\item French Rev shaked europe A27
\item Napoleon got Balance-of-powered A32
\item Bloodshead or whatever, but eventually we got there and got a
better policy A30
\end{itemize}
\end{itemize}

\item Its advantageous to do business instead of doing brute territorialism
A12

\begin{itemize}
\item The britons did not want to assimilate in the Indeas A19
\item when interest is served, working with adversary is possible A2. Did
not want to crush enemy in hopes of not hurting future collab A29
\item Being contemptuous of the values of other societies bring disasters
A9
\item power in itself can't be used to gauge international order, but bop
inhibits capacity to overthrow intl order A5
\end{itemize}
\end{itemize}

\subsection{Claim Synthesis}
\label{sec:org0742ef1}
/Although the European balance of power does lead to frequent
competitive conflict --- like that between France and England --- it
ultimately is both naturally unavoidable and the best outcome for
generally peaceful international policy/

\begin{itemize}
\item To BOP conflict is needed

\begin{itemize}
\item We got to our current Euro-centric world b/c European nations,
through leveraging its technical superiority, upset the balance of
power in acient countries A11
\item They did they b/c our good friend Raison de etat. Every king
believes that acting in his self intrest strengthened general peace
A6, but greed lead to unconcious crimes A10
\item Indeed, too many countries Raisoning may lead to countries
unrestrained by morality A8 to achieve the bestest they could
\item But to actually achieve "the best", we need to constantly adjust the
balance:

\begin{itemize}
\item Before the 7 years war, Britian and France still fought even
nominally at peace A20
\item all the colonial powers had, by the eighteenth century, been able
to extract some economic profit from their colonies A15
\end{itemize}
\end{itemize}

\item Why so much conflict? => conflict = balance readjustment

\begin{itemize}
\item No agreement? Fight! Eventually there will be one. A14
\item Because europeans intentionally set up goverment to prevent one from
being too powerful A24
\item What adjustments? Tracing Napolean:

\begin{itemize}
\item France was baaad: A31
\item The Enlightment happpened (need quote?) Which seeks to BOP

\begin{itemize}
\item French Rev shaked europe A27
\item napolean tried to spread ideas of the revolution A26
\item How did he fade? He got balance-of-powered A32 again b/c he was
being too good at this emperor-not-an-emperor thing.
\end{itemize}
\end{itemize}
\end{itemize}

\item It's fine, because its natural

\begin{itemize}
\item Napoleon example Net positive? Hard to tell. But! At that time, this
was natural and inevitable: we got from Louis to Louis, but it is
from A31 Louis to A30 better Louis
\item Competition in French and English colony in India often set aside
for mainland competition A18, why? because of colonized land having
A34 unique resources that make them easier to control

\begin{itemize}
\item A33 Europe has always been geographically fragmented, so one has
to keep fighting to maintain balance
\item A35: in Europe, no unified ideas and hence constant fighting, but
not in conquered places, which were mostly (and easily) controlled
by one group (Mughals, India. Ming/Qing, China)
\end{itemize}

\item In India, England (the Mediator!) elected to control land instead of
doing balance-of-power: A36
\end{itemize}
\end{itemize}

\begin{html}
<!--* England was always the mediator (QUOTENEEDED)-->
\end{html}

\subsection{Defluffifying}
\label{sec:org78b732b}
/Although the European balance of power does lead to frequent
competitive conflict --- like that between France and England --- it
ultimately is both naturally unavoidable and evolved to fit European
power dynamics/

\begin{itemize}
\item European BOP lead to frequent conflict
\item BOP is naturally unavoidable
\item it is uniquely European, making global BOP-caused war unlikely
\end{itemize}

So what? SO WHAT

Now, defluffify by re-writing the three points + so what in as little
words as possible.

*Although the European balance of power does lead to frequent
competitive conflict through the constant adjustment of balance, it is
naturally unavoidable due to the unique situation in Europe and
ultimately the best outcome for a generally peaceful and democratic
international policy*

\begin{enumerate}
\item Inevitable

\item Natural and Unavoidable in Europe

\item Ultimately lead to spead of liberalism/democracy

\item Conflicts happen + happens more under BOP

\item B/c BOP has to readjust

\item It's fine, b/c its natural
\end{enumerate}

Balance of power \emph{did} make global

\noindent\rule{\textwidth}{0.5pt}

There is always
\href{https://wp.ucla.edu/wp-content/uploads/2016/01/UWC\_handouts\_What-How-So-What-Thesis-revised-5-4-15-RZ.pdf}{UCLA
Writing Lab}
\end{document}
