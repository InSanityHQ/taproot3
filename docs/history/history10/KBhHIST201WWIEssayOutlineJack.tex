% Created 2021-09-11 Sat 16:40
% Intended LaTeX compiler: xelatex
\documentclass[letterpaper]{article}
\usepackage{graphicx}
\usepackage{grffile}
\usepackage{longtable}
\usepackage{wrapfig}
\usepackage{rotating}
\usepackage[normalem]{ulem}
\usepackage{amsmath}
\usepackage{textcomp}
\usepackage{amssymb}
\usepackage{capt-of}
\usepackage{hyperref}
\usepackage[margin=1in]{geometry}
\usepackage{fontspec}
\usepackage{indentfirst}
\setmainfont[ItalicFont = LiberationSans-Italic, BoldFont = LiberationSans-Bold, BoldItalicFont = LiberationSans-BoldItalic]{LiberationSans}
\newfontfamily\NHLight[ItalicFont = LiberationSansNarrow-Italic, BoldFont       = LiberationSansNarrow-Bold, BoldItalicFont = LiberationSansNarrow-BoldItalic]{LiberationSansNarrow}
\newcommand\textrmlf[1]{{\NHLight#1}}
\newcommand\textitlf[1]{{\NHLight\itshape#1}}
\let\textbflf\textrm
\newcommand\textulf[1]{{\NHLight\bfseries#1}}
\newcommand\textuitlf[1]{{\NHLight\bfseries\itshape#1}}
\usepackage{fancyhdr}
\pagestyle{fancy}
\usepackage{titlesec}
\usepackage{titling}
\makeatletter
\lhead{\textbf{\@title}}
\makeatother
\rhead{\textrmlf{Compiled} \today}
\lfoot{\theauthor\ \textbullet \ \textbf{2021-2022}}
\cfoot{}
\rfoot{\textrmlf{Page} \thepage}
\titleformat{\section} {\Large} {\textrmlf{\thesection} {|}} {0.3em} {\textbf}
\titleformat{\subsection} {\large} {\textrmlf{\thesubsection} {|}} {0.2em} {\textbf}
\titleformat{\subsubsection} {\large} {\textrmlf{\thesubsubsection} {|}} {0.1em} {\textbf}
\setlength{\parskip}{0.45em}
\renewcommand\maketitle{}
\author{Houjun Liu}
\date{\today}
\title{History Essay Planning}
\hypersetup{
 pdfauthor={Houjun Liu},
 pdftitle={History Essay Planning},
 pdfkeywords={},
 pdfsubject={},
 pdfcreator={Emacs 27.2 (Org mode 9.4.4)}, 
 pdflang={English}}
\begin{document}

\maketitle


\section{Essay Template}
\label{sec:org2235634}
\subsection{General Information}
\label{sec:org08760d9}
\begin{center}
\begin{tabular}{lll}
Due Date & Topic & Important Documents\\
\hline
Monday, Jan 28th & Causes of WWI & Basically Palmer ch. 17 only??\\
\end{tabular}
\end{center}

\subsection{Prompt}
\label{sec:org89274e9}
The political scientist Kenneth Waltz argues that the causes of war can
be analyzed at three different levels: the individual human level, the
state level, and the international system level. Those who view things
from the first level believe that war is best explained by
"selfishness," "misdirected aggressive impulses," or "stupidity" within
the human psyche. Those who favor the second level believe there are
hostile or aggressive or revisionist states who, because of their form
of government or other domestic issues, behave in a warlike manner while
other states simply want to keep the peace (the status quo). Those who
favor the third level believe that the international system itself,
because it is an anarchy with "no system of law enforceable" between
states, and in which each state acts according to its own interest and
reserves the right to use force to achieve its aims, makes war
inevitable.

*Analyze World War 1 according to one (or a blend) of these levels of
analysis. Make an argument that combines an explanation of the general
causes of the war with the specific sequence of events (including events
that prolonged the war beyond the initial outbreak).*

\subsection{Thought Bucket}
\label{sec:org86a4f48}
\begin{itemize}
\item Formation of alliances?
\item The wish for peace is a cause or an effect?
\item Goverments
\end{itemize}

Self-interested \sout{coorporation} compromises with other parties causes
war.

\subsection{Quotes bin}
\label{sec:orgd369c54}
\begin{html}
<!-- - "Young Turks, whose long agitation against Abdul Hamid has been noted, managed in that year to carry through a revolution. They obliged the sultan to restore the liberal parliamentary constitution of 1876." => Turkish revolutionaries weakened the structure of the Ottoman Empire, which weakened its governmental capacities  -->
\end{html}

\begin{itemize}
\item "He formed a military alliance with Austria-Hungary, to which Italy
was added in 1882. Bismarck signed a 'reinsurance' treaty with Russia
also. Since Russia and Austria were enemies (because ofthe Balkans),
to be allied to both at the same time took considerable diplomatic
finesse." AA
\item "They had long prided themselves on a 'splendid isolation,' going
their own way, disdaining the kind of dependency that alliance with
others always brings" AB
\item "The older Triple Alliance faced a newer Triple Entente, the latter
somewhat the looser, since the British refused to make any formal
military commitments." AC
\item "In 1904 the British and French governments agreed to forget Fashoda
and the accumulated bad feeling of the preceding 25 years. \ldots{} There
was no specific alliance; neither side said what it would do in the
event ofwar; it was only a close understanding, an entente cordiale."
AD
\item "They would call an international conference, at which Russia would
favor Austrian annexation of Bosnia, and Austria would support the
opening of the Straits to Russian warships. Austria, without waiting
for a conference, proclaimed the annexation ofBosnia without more
ado\ldots{} Isvolsky was never able to realize his plans for
Constantinople. His partners in the Triple Entente, Britain and
France, refused to back him." AE
\item "An agreement of the great powers, to keep the peace, conjured up an
independent kingdom in Albania. This confirmed the Austrian policy,
kept Serbia from the sea, and aroused vehement outcries in both Serbia
and Russia. But Russia again backed down. Serbian expansionism was
again frustrated and inflamed." AF
\item "Arthur Zim- mermann, dispatched a telegram to the German minister at
Mexico City, telling him what to say to the Mexican president. He was
to say that if the United States went to war with Germany. Germany
would form an alliance with Mexico and if possible Japan, enabling
Mexico to get back its 'lost territories.' \ldots{} Zimmermann's telegram
was intercepted and decoded by the British, and passed on to them to
Washington. Printed in the newspapers, it shocked public opinion in
the United States." AG
\item "By the end of 1917 the submarine was no more than a nuisance. For the
Germans the great plan produced the anticipated penalty without the
reward---its net result was only to add America to their enemies." AH
\item "The Germans, issuing their famous 'blank check,' encouraged the
Austrians to be firm. The Austrians, thus reassured, dispatched a
drastic ultimatum to Serbia" AI
\item "France, terrified at the possibility ofbeing some day caught alone in
a warwith Germany and determined to keep Russia as an ally at any
cost, in effect gave a blank check to Russia." AJ
\item "The first victim ofthe First World War, among governments, was the
Russian empire" AK
\end{itemize}

\subsection{Claim Synthesis}
\label{sec:orgc36f93c}
\emph{Compromises for peace will inflame tensions}

\begin{itemize}
\item AA Bismarck tried to double ensure his country by forming two-sided
alliances that failed
\item AF the creation of the independent Albanian kingdom meant to keep
peace further inflamed tensions
\end{itemize}

\emph{Preemptive use of absolute force will lead to the victimization of the
user}

\begin{itemize}
\item AI Germans encouraged austria-hungry to use full force, bringing war
\item AJ Russians got also issued a blank check for Russia to fight, which
actually lead Russia to be victimized during the war (AK)
\end{itemize}

\emph{Subversive deals with opposition will cause systemic collapse}

\begin{itemize}
\item AE Capitulation between Austria and Russia ended up screwing Russia
over
\item AG attempts to coax Mexico for security + to end the stalemate
foreshadowed the involvement of the US in the war
\end{itemize}

(um\ldots{} Raison freaking detat again! Or maybe
\href{KBhHIST201Realism.org}{KBhHIST201Realism})

*Instead of leveraging a more passive strategy of defense to maintain
peace, European nations in the early 19th century chose a more active
role for national security though compromising peacekeeping agreements,
preemptive use of absolute force, and subversive deals with opposition
that resulted in the disruption of preset balances of power and lead to
World War I.*

\noindent\rule{\textwidth}{0.5pt}

There is always
\href{https://wp.ucla.edu/wp-content/uploads/2016/01/UWC\_handouts\_What-How-So-What-Thesis-revised-5-4-15-RZ.pdf}{UCLA
Writing Lab}
\end{document}
