% Created 2021-09-12 Sun 22:48
% Intended LaTeX compiler: xelatex
\documentclass[letterpaper]{article}
\usepackage{graphicx}
\usepackage{grffile}
\usepackage{longtable}
\usepackage{wrapfig}
\usepackage{rotating}
\usepackage[normalem]{ulem}
\usepackage{amsmath}
\usepackage{textcomp}
\usepackage{amssymb}
\usepackage{capt-of}
\usepackage{hyperref}
\usepackage[margin=1in]{geometry}
\usepackage{fontspec}
\usepackage{indentfirst}
\setmainfont[ItalicFont = LiberationSans-Italic, BoldFont = LiberationSans-Bold, BoldItalicFont = LiberationSans-BoldItalic]{LiberationSans}
\newfontfamily\NHLight[ItalicFont = LiberationSansNarrow-Italic, BoldFont       = LiberationSansNarrow-Bold, BoldItalicFont = LiberationSansNarrow-BoldItalic]{LiberationSansNarrow}
\newcommand\textrmlf[1]{{\NHLight#1}}
\newcommand\textitlf[1]{{\NHLight\itshape#1}}
\let\textbflf\textrm
\newcommand\textulf[1]{{\NHLight\bfseries#1}}
\newcommand\textuitlf[1]{{\NHLight\bfseries\itshape#1}}
\usepackage{fancyhdr}
\pagestyle{fancy}
\usepackage{titlesec}
\usepackage{titling}
\makeatletter
\lhead{\textbf{\@title}}
\makeatother
\rhead{\textrmlf{Compiled} \today}
\lfoot{\theauthor\ \textbullet \ \textbf{2021-2022}}
\cfoot{}
\rfoot{\textrmlf{Page} \thepage}
\titleformat{\section} {\Large} {\textrmlf{\thesection} {|}} {0.3em} {\textbf}
\titleformat{\subsection} {\large} {\textrmlf{\thesubsection} {|}} {0.2em} {\textbf}
\titleformat{\subsubsection} {\large} {\textrmlf{\thesubsubsection} {|}} {0.1em} {\textbf}
\setlength{\parskip}{0.45em}
\renewcommand\maketitle{}
\author{Huxley}
\date{\today}
\title{Feedback Responses}
\hypersetup{
 pdfauthor={Huxley},
 pdftitle={Feedback Responses},
 pdfkeywords={},
 pdfsubject={},
 pdfcreator={Emacs 28.0.50 (Org mode 9.4.4)}, 
 pdflang={English}}
\begin{document}

\maketitle
\#ret

\noindent\rule{\textwidth}{0.5pt}

On the exploratory notebook, I agree with your answers on exercise 0,
although I think you could be more clear on your answer to problem 2.
Why do repeated values imply optimal values? Are the values returned by
image 2 actually optimal? You successfully found two images that produce
positive values on exercise 1, and you also made new filters that
behaves as you predicted, producing a positive value on the gradient
image. I suspect, though, that this is not quite what you were trying to
do --- it might be interesting to come back to this one and think more
about how you would detect a gradient. On the tuning exercise, you
successfully increased accuracy to 81\%, and I liked that you included
all your code for both the models that didn't work that well, as well as
the ones that did. You provided a brief summary of your results as to
what worked better or worse, but I would challenge you to think more
deeply about this: why do you think these are true? Are there
experiments you could run that would prove some of your suppositions to
be true? For example, you say that more convolutional layers might make
things worse because we only have two epochs. How would you test to see
if this is in fact the reason?

As for problem 2 with the repeated values, intuitively speaking, given
that each group of values is repeating some number as opposed to
repeating the same collection of different numbers their must be one
optimal group which repeats the highest number. As for the gradients,
without either large kernels or some type of memory or hidden cell
state, I don't know how to make filters that would check for consistent
gradients between sections. I would love to know the answer to this.

To test my hypothesis for my model, I could always run different types
of models for more than 2 epochs and see how they perform. Less layers
might also work better because of overfitting, or simply because the
features we are trying to pick up aren't that detailed. Having larger
filters then smaller filters makes sense for this problem, because we
want to look for the larger features initially before processing them
further with the smaller filters.

\noindent\rule{\textwidth}{0.5pt}

Good work, Huxley. I agree with all of your answers. Here are a few
additional questions for your consideration: 1-2. Would there be any
negative rewards? 1-3. If your input is a single 2d grid, how does the
game know whether a given object is moving left or right? That is, if
the current state contains a row that is [car] [frog], how does the game
distinguish between the car about to hit the frog, and the car passing
by harmlessly? 1-5. I like this idea. How do you know what the maximum
possible score is for a given state? 2-6. Tell me more about data
harvesting. What ethical concerns does this present? 3-3. I think this
will work fine, though I wonder: why only normalize rating and not other
interval inputs, such as number of reviews? 3-4. I assume the F in FNN
stands for fully-connected? If so, that will work, but is there
something simpler you could try first? 4-5. Any reason why you picked
f-score instead of precision, recall, or accuracy? 4-6. Any privacy
issues here?

For 1-2, there could be negative rewards for negative actions, like the
frog dying or hitting some kind of obstacle. In 1-3, we could either
have the RNN handle it, or pass in extra states ourselves like
directionality or velocity. 1-5, I am not to knowledgeable on Frogger
mechanics but I do believe it is a completable game with each action
giving some number of possible points. Thus, we can find the maximum
scores. We could also set a point goal for our agent to achieve, and use
that as our maximum. 3-3, those other inputs should most likely be
normalized as well. 3-4, a type of simple regression model could work as
well. 4-5, accuracy is used when true positives/negatives are valued
more than false positives/negatives, which is not what we want in this
scenario. F-score combines recall and precision with a harmonic mean,
producing a result generally though of as better from what I have seen.
4-6, people don't want their medical info being seen. As for anonymity,
privacy-preservation techniques could be employed, but those are often
flawed.
\end{document}
