% Created 2021-09-12 Sun 22:48
% Intended LaTeX compiler: xelatex
\documentclass[letterpaper]{article}
\usepackage{graphicx}
\usepackage{grffile}
\usepackage{longtable}
\usepackage{wrapfig}
\usepackage{rotating}
\usepackage[normalem]{ulem}
\usepackage{amsmath}
\usepackage{textcomp}
\usepackage{amssymb}
\usepackage{capt-of}
\usepackage{hyperref}
\usepackage[margin=1in]{geometry}
\usepackage{fontspec}
\usepackage{indentfirst}
\setmainfont[ItalicFont = LiberationSans-Italic, BoldFont = LiberationSans-Bold, BoldItalicFont = LiberationSans-BoldItalic]{LiberationSans}
\newfontfamily\NHLight[ItalicFont = LiberationSansNarrow-Italic, BoldFont       = LiberationSansNarrow-Bold, BoldItalicFont = LiberationSansNarrow-BoldItalic]{LiberationSansNarrow}
\newcommand\textrmlf[1]{{\NHLight#1}}
\newcommand\textitlf[1]{{\NHLight\itshape#1}}
\let\textbflf\textrm
\newcommand\textulf[1]{{\NHLight\bfseries#1}}
\newcommand\textuitlf[1]{{\NHLight\bfseries\itshape#1}}
\usepackage{fancyhdr}
\pagestyle{fancy}
\usepackage{titlesec}
\usepackage{titling}
\makeatletter
\lhead{\textbf{\@title}}
\makeatother
\rhead{\textrmlf{Compiled} \today}
\lfoot{\theauthor\ \textbullet \ \textbf{2021-2022}}
\cfoot{}
\rfoot{\textrmlf{Page} \thepage}
\titleformat{\section} {\Large} {\textrmlf{\thesection} {|}} {0.3em} {\textbf}
\titleformat{\subsection} {\large} {\textrmlf{\thesubsection} {|}} {0.2em} {\textbf}
\titleformat{\subsubsection} {\large} {\textrmlf{\thesubsubsection} {|}} {0.1em} {\textbf}
\setlength{\parskip}{0.45em}
\renewcommand\maketitle{}
\author{Huxley}
\date{\today}
\title{Unit 3 Essay Planning}
\hypersetup{
 pdfauthor={Huxley},
 pdftitle={Unit 3 Essay Planning},
 pdfkeywords={},
 pdfsubject={},
 pdfcreator={Emacs 28.0.50 (Org mode 9.4.4)}, 
 pdflang={English}}
\begin{document}

\maketitle
\#flo \#ret \#disorganized \#incomplete

\noindent\rule{\textwidth}{0.5pt}

\section{UNIT. THREE.}
\label{sec:orgf2c8290}
\subsection{prompt}
\label{sec:orgc651507}
\begin{verbatim}
Unit 3 Essay: 19th Century Transformations

Essay option 1: The political, economic, and technological transformations of the 19th century drastically changed the global power balance. In some cases, they made already strong states dramatically more powerful: witness Great Britain. In other cases, they paved the way for new, rising powers such as Japan, Germany, and Italy. And in still other cases, such as Egypt, India, China, Southeast Asia and Africa, they led to foreign intervention, imperialism and the loss of independence. What factor or factors most strongly determined which states would increase or decline in power in this time period (1815 to roughly 1900)? Answer this question with a thesis that addresses several, diverse examples from across the world. Tip: rather than accumulating a long, complex list of unrelated factors, simplify and pare down your thesis, saving the complexity for your examples.

Essay option 2:  According to Charles Tilly’s “bellicist” theory of state formation, states form to protect a territory from external threats. In the process of doing this (war-making), they develop the tools to eliminate internal rivals as well (state-making), and in order to fund these endeavours, they develop institutions like taxation to raise revenue from their territory (extraction). In order to maximize the wealth available for extraction and prevent property owners from needing private armies to protect themselves, the state establishes rule of law to protect property rights (protection). Insufficient state capacity in one of these domains is thus likely to affect the others. The 19th century is an especially interesting era in which to see this theory in action, as it witnesses the rise and fall of a number of new and old states. Using this theory as a starting point, compare and contrast state formation in two or three 19th century states we’ve studied this unit. You should have a thesis which points out a meaningful trend, disparity or hypothesis. 

Essay option 3: The post-1815 world was born from a massive ideological upheaval: The French Revolution and the subsequent spreading of liberal and nationalist ideals throughout an increasingly interconnected world. In the century to follow, nationalism and liberalism vied with each other as well as with other ideologies such as conservatism and socialism/communism. Blends of these existed as well. In an essay drawing upon specific examples from across the world, argue which of these ideologies was most transformative or important to understanding the events of the 19th century. Tip: the contrast between ideologies can help make your argument be more debatable and have more power-- show, for example, why liberalism had more of an impact than nationalism alone, or vice versa.

Note: Essays (either option) should cite from a wide variety of sources from the Unit 3 Reader. 

Other Submission guidelines: 3-4 pages, size 12 font, double-spaced. Citations should be in-line and formatted as (Authorname Pagenumber) i.e. (Kennedy 12). Include a Works Cited page in MLA format. 

Tips: See the essay rubric guide below for questions to ask yourself as you write and revise. 
History essay rubric guide
\end{verbatim}

prompt 1:

about types of change?

more nuanced version of: conservative bad, innovation good

look at conservatives in china, say that cuased them to fall

look at britian and their innovation, led to their rise

say that change and innovation is not nesasarrily good, look at
communism

bad: about striking the balance

good: some sort of categorical diff -- changing tech vs human nature?
use vs users?

Need to fufill: - Uses diverse variety of evidence to support
arguments - Creates an argument with strong explanatory power and
relevance to both broader historical trends and specific moments in
history

Japan was the model of modernization

interconection

huge amount of change

major driving factor was increase in interconnectivuity

reaction to itc

interconnecivity: increase in exanchge - goods and ideas

interconectivity inherently means a decrease in autonomy

\subsection{Planning}
\label{sec:orga5f4a2e}
\subsubsection{Quoute bin}
\label{sec:orgb75999b}
\begin{itemize}
\item interconnecitivty ran rampant

\begin{itemize}
\item Prussian suc- cess was due in large measure to the application of
new technologies to logistics and warfare: the new breech-loading
"needle gun" (which could be fired from the prone position) and the
use of the railroad and the tele- graph to move and coordinate
troops and supplies. - mason
\item he empi re was so widesp read that it was difficu lt for the power
of the central government to radiate out throug h the provinces,
even with the use of nineteenth· century technologies such as
telegraphs and railroads. - gelvin
\item The telegraph rapidly became as conventional a presence in the
nineteenth century as the cell phone is today. And like other
nineteenth century inventions -s teamboats and railroads and the
Gatling gun - the telegraph proved to be an indispensable tool of
imperialism - gelvin
\end{itemize}

\item britain embraced interconnectivity (embraced/made profit off of trade,
had a strong navy)

\begin{itemize}
\item For over a century before 1815, of course, the Royal Navy had
usually been the largest in the world. - kennedy
\item Th.e first was the steady and then (after the 1840s) spectacular
growthof an integrated global economy, which drew ever more re-
gionsinto a transoceanic and transcontinental trading and financial
networkcentered upon western Europe, and in particular upon Great
Britain. - kennedy
\item The British compelled China to sign a one-hundred-year lease over
Hong Kong, which became one of the most important commercial and
trading centers in Asia. - mason
\item The European states initially relied primarily on chartered trading
com- panies to explore and develop colonial areas, expecting that
the resulting colonies would essentially pay for themselves. - mason
\end{itemize}

\item japan 54! 57!

\begin{itemize}
\item started as closed off

\begin{itemize}
\item In 1853, the American Commodore Perry forced his way with a fleet
of naval vessels into Yedo Bay. insisted upon landing, and
demanded of the Japanese government that it en gage in commercial
relations with the United States and other Western powers. In the
next year the Japanese began to comply, and in 1867 an internal
revolution took place, of which the most conspicuous consequence
was a rapid westernizing of Japanese life and institu tions. But
if it looked as if the country had been "opened" by Westerners,
actually Japan had exploded from within. - palmer
\item For over two centuries Japan had followed a program of
self-imposed isolation. No Japa nese was allowed to leave the
islands or even to build a ship large enough to navigate the high
seas. No foreigner, except for handfuls of Dutch and Chinese, was
allowed to enter. Japan remained a sealed book to the West. -
palmer
\end{itemize}

\item opened up

\begin{itemize}
\item In the summer of 1898, Kang Youwei, a brilliant Confucian scholar
who admired Japan for its rapid adoption of Western institutions
and industrialization, - ropp
\item The lords of Choshu and Satsuma now concluded that the only way to
deal with the West was to adopt the military and technical
equipment of the West itself. - palmer
\item The Meiji era (1868-1912) was the great era of the westernization
of Japan. Japan turned into a modern nation-state. - palmer
\item Foreign trade, almost literally zero in 1854, was valued at \$200
million a year by the end of the century. - palmer
\end{itemize}

\item took stuff and became powerful

\begin{itemize}
\item A navy, modeled on the British, followed somewhat later. - palmer
\item newly western,zecl army and navy, - palmer
\item The Russians sent their Baltic fleet around three continents to
the Far East, but to the world's amazement the Rus sian fleet was
met and destroyed at Tsushima Strait by the new and untested navy
of Japan. - palmer
\end{itemize}
\end{itemize}

\item china resisted interconnevtivity

\begin{itemize}
\item here were Confucian officials in the late nineteenth century who
called for "self-strengthening," learning from the West, and who
began to build modern weapons, steamships, railroads and telegraph
lines. But the Qing Empire was a vast, poor, mostly agricultural and
overpopulated territory with a small, weak gov ernment, and the
modernization efforts were confined to tiny coastal areas that had
little impact inland. - ropp

\item \begin{itemize}
\item 
\end{itemize}

\item In the summer of 1898, Kang Youwei, a brilliant Confucian scholar
who admired Japan for its rapid adoption of Western institutions and
industrialization,

\item liked japans industrialization (see above) then "ordered the reform
movement crushed." - ropp

\item The Qing court remained largely ignorant of these processes. In the
late eighteenth century, British traders came to feel increas- ingly
frustrated with problems in the China trade. - ropp

\item Qing government saw inter national trade not as a way to generate
new wealth but as a privilege granted to less-developed "barbarians"
in exchange for their paying respects to the Son of Heaven and his
court. British merchants were allowed to trade only at the
southeastern seaport of.Guangzhou (known in the West as Canton),
where they were confined to a few warehouses and allowed to reside
only temporarily to load and unload their ships. - ropp

\item British mer chants would be expelled if they tried to come ashore
anywhere other than Guangzhou and concluding with a standard
emperor's command to his lowly subjects: "Tremblingly obey and show
no negligence!"1 - ropp

\item China's ship of state had fallen into serious disrepair. “She may,
perhaps, not sink outright; she may drift some time as a wreck, and
will then be dashed to pieces on the shores; but she can never be
rebuilt on the bottom - ropp
\end{itemize}

\item china got fricked

\begin{itemize}
\item In 1894-1895, fighting over influence in Korea, Japanese troops
quickly and soundly defeated Qing forces. - ropp
\item In September 1899, John Hay, America's secretary of state, issued a
series of "Open Door Notes" to Britain, France, Ger many, Russia,
Italy, and Japan, calling on all foreign powers in China to allow
free trade in all spheres of influence. - ropp
\item Great Britain sent an expedition ary force of sixteen warships, four
armed steamers, twenty-seven transport ships, and one troop ship to
China in 1840, with a total of 4,000 British troops. The Chinese had
no naval forces capable of defeating such a force and little
comprehension of how deadly serious the British government was in
its determination to force the opium trade to continue and to
grow. - ropp
\end{itemize}

\item conclustion?

\begin{itemize}
\item The moral was clear. Everywhere leaders of subjugated peoples con
cluded, from the Japanese precedent, that they must bring Western
science - palmer
\end{itemize}
\end{itemize}

\subsubsection{Outline}
\label{sec:org1ea78a3}
\begin{enumerate}
\item title: freedom, or else.
\label{sec:org227e8c2}
\item THESIS: the major deciding factor in a states change in power in
\label{sec:orgf386239}
the 19th century was its willingness to embrace interconnectivity.
:CUSTOM\textsubscript{ID}: thesis-the-major-deciding-factor-in-a-states-change-in-power-in-the-19th-century-was-its-willingness-to-embrace-interconnectivity.

\begin{itemize}
\item interconectivity ran rampant

\begin{itemize}
\item industiral rev led to shrinking world
\item embracing or rejecting interconnectivity was the major deciding
factor in determining a nation's change in power
\end{itemize}

\item Britain embraced and leveraged this

\begin{itemize}
\item used interconnectivity (trade) for profit, and became much more
powerful
\item navy is a tool for interconnectivity
\end{itemize}

\item Japan flipped and accepted

\begin{itemize}
\item started as closed off
\item americans showed up
\item opened japan
\item took ideas and goods
\item japan became more powerful
\end{itemize}

\item China resisted interconnectivity for fear of loss of autonomy and lost
control

\begin{itemize}
\item china tried to resist free trade (one major way inteconncection
manifests)
\item crumbled
\end{itemize}

\item conc: the power of connection is greater than any nation state
\end{itemize}
\end{enumerate}

\subsubsection{Kinda wanna be more than friendsssss (unedited version)}
\label{sec:orgb18af11}
The Industrial Revolution gave birth to a massively shrinking world. It
brought the invention of railroads, steamboats, and the telegraph, which
"rapidly became as conventional a presence in the nineteenth century as
the cell phone is today" (Gelvin 80). These inventions allowed for the
rapid travel of goods and information the likes of which had never been
seen before. The Industrial Revolution led to a sort of co-revolution
which continues to this day: the interconnectivity revolution.
Interconnectivity, in this context, can be defined as the exchange of
goods and ideas and the infrastructure facilitating this exchange. All
of a sudden, the world was becoming massively interconnected, not only
dramatically increasing the pool of ideas that any given state had
access to, but exponentially increasing the ways these ideas could
interact with one another. As a consequence, the 19th century was
fraught with massive changes in power. The major factor that determined
which side of the power shift a state would be on was its willingness to
embrace interconnectivity.

Britain heavily embraced interconnectivity, leveraging it for power and
profit. Trade, one of Britain's main sources of income, is one of the
primary manifestations of interconnectivity. Britain heavily promoted
trade, even leveraging its power to take territory from other states,
creating "important commercial and trading centers" (Mason 98). They
focused on developing a powerful navy, which became "the largest in the
world" (Kennedy 152). This navy effectively served as an infrastructure
for propagating interconnectivity, as seafaring was the primary means of
transferring goods and ideas between states. They used this navy to
enforce free trade and manage trade routes, further promoting
interconnectivity and eventually leading to the "spectacular growth of
an integrated global economy [\ldots{}] and financial network centered [\ldots{}]
upon Great Britain" (Kennedy 143). Of course, Britain benefited
massively off of this newly created integrated financial network, which,
along with it's goods, would spread ideas. Expanding and rising in power
immensely, Britain truly was one of the driving forces behind this
interconnectivity revolution.

After centuries of isolation, Japan accepted the interconnectivity
revolution and prospered because of it. During Japan's period of self
isolation, "no Japanese was allowed to leave the islands or even to
build a ship large enough to navigate the high seas. No foreigner [\ldots{}]
was allowed to enter" (Palmer 543). Japan had little power, little
influence, and was the antithesis of an interconnected state. This
isolation continued until 1853, when "the American Commodore Perry
forced his way with a fleet of naval vessels into Yedo Bay," initiating
Japan into the interconnectivity revolution (Palmer 543). Commodore
Perry demanded that the Japanese engage in trade with the Western
powers, and "the Japanese began to comply [\ldots{}] in 1867 an internal
revolution took place, of which the most conspicuous consequence was a
rapid westernizing of Japanese life and institutions" (Palmer 543). In
what is now called the Meiji era, spanning from 1868 to 1912, Japan
experienced a complete shift from isolation to being the model of a
modern state. Trade, "almost literally zero in 1854, was valued at \$200
million a year by the end of the century" (Palmer 548). Japan embraced
interconnectivity, exchanging and adopting ideas along with goods. This
exchange led to a massive increase in Japan's power, and they formed a
"newly westernized army and navy" which was "modeled on the British"
(Palmer 547). This increase in power allowed Japan to defeat the
Russians and the Chinese, much "to the world's amazement" (Palmer 654).
Japan, beginning as the very opposite of an interconnected state,
accepted interconnectivity, and rapidly gained the ideas and goods which
ultimately led to a massive increase in their power.

China, starting with isolation similar to Japan's, resisted the
interconnectivity revolution and suffered because of it. Initially,
China resisted free trade. They saw trade itself as a sort of charity, a
privilege "granted to less-developed 'barbarians' in exchange for their
paying respects to the Son of Heaven and his court" (Ropp 102). They
also heavily restricted trade, allowing it only in specific ports and
confining traders to but "a few warehouses [\ldots{}] to reside only
temporarily to load and unload their ships." (Ropp 102). This outlook on
trade --- and the policy that followed --- led to the inevitable and
increasing frustration of the British traders. Not only did China resist
the exchange of goods, but they also resisted the exchange of ideas ---
two key components in interconnectivity. In 1898, "a brilliant Confucian
scholar who admired Japan for its rapid adoption of Western institutions
and industrialization" tried to do the same to China (Ropp 110). He
issued a myriad of edicts, all based upon the idea of sending China
through a "crash program of industrialization and Westernization" (Ropp
110). Conservatives grew alarmed, and the reform movement was easily and
soundly crushed. Interconnectivity inherently brings with it a loss in
autonomy and tradition, and China's unwillingness to sacrifice these led
to it's downfall. Of course, Britain eventually released their navy upon
China. Unlike the Japanese who embraced the interconnectivity revolution
and built a powerful military, "the Chinese had no naval forces" and
were almost instantly defeated (Ropp 105). The British mandated free
trade by force, and China massively decreased in power. China was now an
unwilling part of the interconnectivity revolution.

Not only does interconnectivity allows for broader access to ideas and
goods, but it allows for new ideas to be created that would otherwise be
impossible. The power of the interconnectivity revolution was greater
than any state, and acceptance of this historical trend was required for
success. The interconnectivity revolution has only continued, the world
still shrinking day by day. In modern times --- with the invention of
the internet and such --- it is much easier to see the value of
interconnectivity, making refuting its importance almost futile.
Sometimes we must sacrifice in the face of global trends

interconnectivity is more important and evident in modern day

Works Cited:

Palmer, R R, Joel Colton, and Lloyd S. Kramer. \emph{A History of the Modern
World,} 2000

Ropp, Paul Stanley. \emph{China in World History.} Oxford University
Press, 2012.

Mason, David. \emph{A Concise History of Modern Europe: Liberty, Equality,
Solidarity}. Plymouth, Rowman and Littlefield.

Kennedy, Paul M. 1988. \emph{The rise and fall of the great powers.} New
York, NY: Random House.

Gelvin, James L. 2011. \emph{The modern Middle East : a history.} New York: :
Oxford University Press.
\end{document}
