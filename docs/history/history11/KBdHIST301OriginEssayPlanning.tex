% Created 2021-09-21 Tue 22:33
% Intended LaTeX compiler: xelatex
\documentclass[letterpaper]{article}
\usepackage{graphicx}
\usepackage{grffile}
\usepackage{longtable}
\usepackage{wrapfig}
\usepackage{rotating}
\usepackage[normalem]{ulem}
\usepackage{amsmath}
\usepackage{textcomp}
\usepackage{amssymb}
\usepackage{capt-of}
\usepackage{hyperref}
\setlength{\parindent}{0pt}
\usepackage[margin=1in]{geometry}
\usepackage{fontspec}
\usepackage{svg}
\usepackage{cancel}
\usepackage{indentfirst}
\setmainfont[ItalicFont = LiberationSans-Italic, BoldFont = LiberationSans-Bold, BoldItalicFont = LiberationSans-BoldItalic]{LiberationSans}
\newfontfamily\NHLight[ItalicFont = LiberationSansNarrow-Italic, BoldFont       = LiberationSansNarrow-Bold, BoldItalicFont = LiberationSansNarrow-BoldItalic]{LiberationSansNarrow}
\newcommand\textrmlf[1]{{\NHLight#1}}
\newcommand\textitlf[1]{{\NHLight\itshape#1}}
\let\textbflf\textrm
\newcommand\textulf[1]{{\NHLight\bfseries#1}}
\newcommand\textuitlf[1]{{\NHLight\bfseries\itshape#1}}
\usepackage{fancyhdr}
\pagestyle{fancy}
\usepackage{titlesec}
\usepackage{titling}
\makeatletter
\lhead{\textbf{\@title}}
\makeatother
\rhead{\textrmlf{Compiled} \today}
\lfoot{\theauthor\ \textbullet \ \textbf{2021-2022}}
\cfoot{}
\rfoot{\textrmlf{Page} \thepage}
\titleformat{\section} {\Large} {\textrmlf{\thesection} {|}} {0.3em} {\textbf}
\titleformat{\subsection} {\large} {\textrmlf{\thesubsection} {|}} {0.2em} {\textbf}
\titleformat{\subsubsection} {\large} {\textrmlf{\thesubsubsection} {|}} {0.1em} {\textbf}
\setlength{\parskip}{0.45em}
\renewcommand\maketitle{}
\author{Dylan Wallace}
\date{\today}
\title{HIST301 Origin Essay Planning}
\hypersetup{
 pdfauthor={Dylan Wallace},
 pdftitle={HIST301 Origin Essay Planning},
 pdfkeywords={},
 pdfsubject={},
 pdfcreator={Emacs 28.0.50 (Org mode 9.4.4)}, 
 pdflang={English}}
\begin{document}

\maketitle


\section{Prompt}
\label{sec:orge655fed}
\begin{verbatim}
The telling of American history has become one of the focal points of conflicts over our national culture,
particularly the relationship between ideas of American exceptionalism and stories of race and conquest in 
US history. What is the value of a shared national narrative and what dangers does crafting such a narrative 
present? Why does the telling of US history generate such controversy?

In a 2-3 page, double spaced essay, evaluate how the authors we’ve read so far approach the telling of 
American history and make an argument for what you see as the key considerations in constructing a narrative
of early American history. Historical narratives, by their nature, are created through choices of what to 
include and what not, what to emphasize and what to relegate to the margins. In making your argument, include 
an explicit engagement with the sources we’ve worked with so far. You might, for example, engage with the 
controversy over the 1619 project, with Dunbar-Ortiz’s criticism of multiculturalism or with Richter’s 
geographical positioning of history.
\end{verbatim}

\section{Outline}
\label{sec:org18d136a}
\subsubsection{Main idea}
\label{sec:org9388928}
There isn't a clear consensus as to what (lens/approaches) topics to
discuss among scholars in regards to early American history (before
1776). Some scholars write mainly about the economic aspect of
pre-independence history, whereas others may write mainly about the
culture of the many groups that inhabited early North America. . .many
other approaches. However, the two topics are not mutually exclusive,
and in fact are causal: economics influence culture, and culture
influences economics. Despite this fact, many examples of historical
literature focus on only one aspect out of the two intertwined topics.
Many times this division is done for the sake of briefness, as there are
only a certain number of pages that can go in a book until it becomes
impractical. However, dividing these two topics can hinder our
understanding of the complex relationships between them.

Socio-economic lens. The socio-economic lens is effective for the
analysis of early American hsitory because the cylical relationship
between a society's economics and social forces sheds light on the. .
American Paradox - value of looking at both economic and social forces
(Author's approach) Facing East - ''

\subsubsection{Outline (1st Draft)}
\label{sec:org4987ed9}
\begin{enumerate}
\item Intro thing - present thesis
\item Flaws of some of the readings (i.e. they are incomplete)

\begin{enumerate}
\item Pilgrims and Puritans

\begin{enumerate}
\item Author focuses on Pilgrim and Puritan culture and their
differences
\item Fails to address economic motivations
\end{enumerate}

\item 1619

\begin{enumerate}
\item Author focuses on cultural and social aspect of slavery
\item Fails to address economic factors
\end{enumerate}

\item All in All

\begin{enumerate}
\item When looking at history through purely a cultural lens, it is
impossible to understand the true cocktail of motivations that
led to the many events
\item It is impossible to be "right" about history without looking at
history through both lenses
\end{enumerate}
\end{enumerate}

\item Good Analysis: Facing East from Indian Country and The American
Paradox

\begin{enumerate}
\item Facing East

\begin{enumerate}
\item Talks mostly about how Europeans interacted with Indians
\item Economic => Cultural

\begin{enumerate}
\item Indians using European tools for ceremonies
\end{enumerate}

\item Cultural => Economic

\begin{enumerate}
\item Furs perceived as being luxurious led to cooperation between
Indians and Europeans and cash flow towards the Americas
\end{enumerate}
\end{enumerate}

\item American Paradox
\end{enumerate}

\item Conclusion: Why is this analysis style important?

\begin{enumerate}
\item More specifically, how does the "new view" of culture and
economics being merged impact American history as a subject and
how does that have an impact on current day politics/social
dynamics/etc.?
\end{enumerate}
\end{enumerate}

\subsubsection{Outline (2nd Draft) + Notes}
\label{sec:org6340fd2}
\begin{itemize}
\item The socio-economic lens is effective for the analysis of early
American hsitory because the cylical relationship between a society's
economics and social forces sheds light on the causes of the many
events.

\item Intro thing - present thesis
\item Early North American Pre-independence History and Facing East

\begin{itemize}
\item Social and Economic factors are able to be separated, but viewing
the events through either lens will not give the full picture
\item Example: Native American--European Trading

\begin{itemize}
\item Fur Trading

\begin{itemize}
\item Some Native American tribes prospered because of their
geography and environment, whereas others were left broke. This
created power balances. (economic impacts social)
\end{itemize}

\item Metal blades

\begin{itemize}
\item Very valuable to Native Americans due to their improved
durability in comparison to Native American tools, and allowed
for more efficient work (economic impacts social)
\end{itemize}

\item Guns

\begin{itemize}
\item Very powerful in inter-tribe warfare, were highly sought after
(both economic impacts social and social impacts economic)
\end{itemize}
\end{itemize}
\end{itemize}

\item Later North American Pre-independence History and American Paradox

\begin{itemize}
\item Social and Economic factors cannot be separated; they are so
intertwined that it is impossible to categorize events as being
purely economic or social
\item Example: Master-Servant Dynamic + Slaves

\begin{itemize}
\item Civil unrest due to unemployment in Britain leads to import of
workers (acting as servants) to America
\item Fear of similar unrest in America from large worker immigration
leads to slave imports and reliance, as well as more rights for
former servants
\end{itemize}
\end{itemize}

\item 
\end{itemize}

\section{Essay (1st Draft)}
\label{sec:org59355fa}
\begin{verbatim}
The socio-economic lens is effective for the analysis of early American hsitory because the cylical relationship between a society's economics and social forces sheds light on the causes of the many events. (TEMP)
\end{verbatim}
\end{document}
