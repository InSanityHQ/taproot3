% Created 2021-09-27 Mon 12:01
% Intended LaTeX compiler: xelatex
\documentclass[letterpaper]{article}
\usepackage{graphicx}
\usepackage{grffile}
\usepackage{longtable}
\usepackage{wrapfig}
\usepackage{rotating}
\usepackage[normalem]{ulem}
\usepackage{amsmath}
\usepackage{textcomp}
\usepackage{amssymb}
\usepackage{capt-of}
\usepackage{hyperref}
\setlength{\parindent}{0pt}
\usepackage[margin=1in]{geometry}
\usepackage{fontspec}
\usepackage{svg}
\usepackage{cancel}
\usepackage{indentfirst}
\setmainfont[ItalicFont = LiberationSans-Italic, BoldFont = LiberationSans-Bold, BoldItalicFont = LiberationSans-BoldItalic]{LiberationSans}
\newfontfamily\NHLight[ItalicFont = LiberationSansNarrow-Italic, BoldFont       = LiberationSansNarrow-Bold, BoldItalicFont = LiberationSansNarrow-BoldItalic]{LiberationSansNarrow}
\newcommand\textrmlf[1]{{\NHLight#1}}
\newcommand\textitlf[1]{{\NHLight\itshape#1}}
\let\textbflf\textrm
\newcommand\textulf[1]{{\NHLight\bfseries#1}}
\newcommand\textuitlf[1]{{\NHLight\bfseries\itshape#1}}
\usepackage{fancyhdr}
\pagestyle{fancy}
\usepackage{titlesec}
\usepackage{titling}
\makeatletter
\lhead{\textbf{\@title}}
\makeatother
\rhead{\textrmlf{Compiled} \today}
\lfoot{\theauthor\ \textbullet \ \textbf{2021-2022}}
\cfoot{}
\rfoot{\textrmlf{Page} \thepage}
\renewcommand{\tableofcontents}{}
\titleformat{\section} {\Large} {\textrmlf{\thesection} {|}} {0.3em} {\textbf}
\titleformat{\subsection} {\large} {\textrmlf{\thesubsection} {|}} {0.2em} {\textbf}
\titleformat{\subsubsection} {\large} {\textrmlf{\thesubsubsection} {|}} {0.1em} {\textbf}
\setlength{\parskip}{0.45em}
\renewcommand\maketitle{}
\author{Dylan Wallace}
\date{\today}
\title{Pilgrims and Puritans}
\hypersetup{
 pdfauthor={Dylan Wallace},
 pdftitle={Pilgrims and Puritans},
 pdfkeywords={},
 pdfsubject={},
 pdfcreator={Emacs 28.0.50 (Org mode 9.4.4)}, 
 pdflang={English}}
\begin{document}

\tableofcontents

\#flo

\section{Intro}
\label{sec:org333ed5f}
\begin{itemize}
\item Pilgrims weren't entirely religious fanatics
\item Difference between Pilgrims and Puritans

\begin{itemize}
\item \textbf{Pilgrims}: Mayflower, 1620
\item \textbf{Puritans}: Arbella, 1630
\end{itemize}

\item Only from 1691 with merging of Massachusetts did Puritans and Pilgrims
become single settlement
\item Judeo-Christian narrative of escaping slavery to a promised land

\begin{itemize}
\item Religious narrative turned cultural myth of America
\item America was thought of as a Utopia ever since the Renaissance in
Europe
\end{itemize}

\item The American Genesis myth

\begin{itemize}
\item America created as Christian Utopia
\item Dominant from 1930's to 1980's in discourse
\item Waned in 1980's after 60's and 70's movements
\end{itemize}

\item Europeans thought America was poggers

\begin{itemize}
\item Wasn't really based on anything
\item Didn't find any utopias or anything so decided to build their own
\end{itemize}

\item Bunch of religious separatists moved to America
\end{itemize}

\section{Puritans}
\label{sec:orge45ac6c}
\begin{itemize}
\item Puritans were protestants and believed in individual process for
enlightenment
\item Initially thought they would come to new world and find the promised
land
\item Piquot War => Sign of God's mercy?
\item America and utopianism go hand and hand
\item Jeremiad literature

\begin{itemize}
\item Started by Puritans
\item List complaints about community, then prompts change and presents
incentives
\end{itemize}
\end{itemize}

\section{Pilgrims}
\label{sec:orgc7778b0}
\begin{itemize}
\item Were more merciful with Indians
\item Saw English-speaking Indian dude as sign from God
\item Bradford dude wrote a lot about colony history
\end{itemize}
\end{document}
