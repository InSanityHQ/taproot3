% Created 2021-10-17 Sun 17:27
% Intended LaTeX compiler: xelatex
\documentclass[letterpaper]{article}
\usepackage{graphicx}
\usepackage{grffile}
\usepackage{longtable}
\usepackage{wrapfig}
\usepackage{rotating}
\usepackage[normalem]{ulem}
\usepackage{amsmath}
\usepackage{textcomp}
\usepackage{amssymb}
\usepackage{capt-of}
\usepackage{hyperref}
\setlength{\parindent}{0pt}
\usepackage[margin=1in]{geometry}
\usepackage{fontspec}
\usepackage{svg}
\usepackage{tikz}
\usepackage{cancel}
\usepackage{pgfplots}
\usepackage{indentfirst}
\setmainfont[ItalicFont = LiberationSans-Italic, BoldFont = LiberationSans-Bold, BoldItalicFont = LiberationSans-BoldItalic]{LiberationSans}
\newfontfamily\NHLight[ItalicFont = LiberationSansNarrow-Italic, BoldFont       = LiberationSansNarrow-Bold, BoldItalicFont = LiberationSansNarrow-BoldItalic]{LiberationSansNarrow}
\newcommand\textrmlf[1]{{\NHLight#1}}
\newcommand\textitlf[1]{{\NHLight\itshape#1}}
\let\textbflf\textrm
\newcommand\textulf[1]{{\NHLight\bfseries#1}}
\newcommand\textuitlf[1]{{\NHLight\bfseries\itshape#1}}
\usepackage{fancyhdr}
\usepackage{csquotes}
\pagestyle{fancy}
\usepackage{titlesec}
\usepackage{titling}
\makeatletter
\lhead{\textbf{\@title}}
\makeatother
\rhead{\textrmlf{Compiled} \today}
\lfoot{\theauthor\ \textbullet \ \textbf{2021-2022}}
\cfoot{}
\rfoot{\textrmlf{Page} \thepage}
\renewcommand{\tableofcontents}{}
\titleformat{\section} {\Large} {\textrmlf{\thesection} {|}} {0.3em} {\textbf}
\titleformat{\subsection} {\large} {\textrmlf{\thesubsection} {|}} {0.2em} {\textbf}
\titleformat{\subsubsection} {\large} {\textrmlf{\thesubsubsection} {|}} {0.1em} {\textbf}
\setlength{\parskip}{0.45em}
\renewcommand\maketitle{}
\author{Huxley Marvit}
\date{\today}
\title{American Revolution Historiography Essay Prep}
\hypersetup{
 pdfauthor={Huxley Marvit},
 pdftitle={American Revolution Historiography Essay Prep},
 pdfkeywords={},
 pdfsubject={},
 pdfcreator={Emacs 28.0.50 (Org mode 9.4.4)}, 
 pdflang={English}}
\begin{document}

\tableofcontents

\#flo \#ret \#disorganized \#hw

\noindent\rule{\textwidth}{0.5pt}

\begin{verbatim}
title: prompt
American Revolution Historiography Essay

In a paper approximately 4 pages in length, write a synthesis essay using at least two of the American revolution articles. The goal of the paper is to write about the historiography of the American revolution, synthesizing different interpretations to make your own argument about the causes of the American revolution. 

Synthesis writing is not that much different from what you did in 10th grade where you incorporated multiple sources while answering questions posed in the prompt. You are not summarizing the texts, comparing/contrasting, or making a judgement about how one source stacks up against another. Instead, you are using the sources to develop and advance your own idea about the revolution. 

Your synthesis might take one of two general tracks (though I am open to discussing additional ideas as well):

1.  Historiography: How might the sources help us to think about how histories of the revolution are told. You might think about the types of arguments the authors are making: do they construct a top-down or bottom-up narrative? Do they highlight political, economic, cultural, or ideological factors in their explanation of the events leading up to the revolution? What are the implications of these organizational and analytical decisions? Your paper might resemble a “state of the field” highlighting the insights and lingering questions from our accumulative understanding of the revolution. You might also highlight what is “at stake” in these arguments about the revolution

2.  Historical synthesis: Using the evidence and insights from the authors, what kinds of conclusions might you draw about the revolution. You might look at different causal forces to talk about the radical potential or radical results of the revolution (or lack thereof). You might develop connections between the different analytical perspectives offered by the authors--for example, how might we look at ideological roots, elite anxiety, and popular participation as being constituent parts of a larger whole? Does the revolution represent an advancement in democracy or a continuation of elite rule? What sort of ideas, fears, and interests do the authors point to in describing the motivations of the different actors involved in the revolution?


To assemble the material for your essay, you should feel free to utilize your notes from the presentations of other readings, consult with your classmates, and/or read parts of the other readings while preparing your essay. While your thesis should of course be your own original work, it is always a good idea to talk about your ideas with others, including your classmates.
\end{verbatim}

what was the cause of the american revoltion?

thinking:

historians that we read think about causality wrong

proximal and distal causality?

infinite chain of causality

cause -> cause -> effect

multiple causes can lead to multiple effects which in turn have effects

there is no single cause! it fits into this graph-network of causes and
effects

essay: fit the causes of the american revolution into this graph
framework of cause -> effect

for example, being able to unite over shared rejection of goods might be
a proximal cause, but it is not \emph{the} cause of the revolution.

big idea: \textbf{NEW MODEL OF CAUSALITY} this is, historiography.

\begin{itemize}
\item summaries of the readings:

\begin{itemize}
\item Holton: forced founders

\begin{itemize}
\item economic situations!

\begin{itemize}
\item each chapter about how lack of independence has negative
economic impacts
\end{itemize}

\item elites had to do stuff because of their debt?
\end{itemize}

\item Wood: The Radicalism of the American Revolution

\begin{itemize}
\item united by values
\item popular belief shaped by lived experience

\begin{itemize}
\item this is what drove the revolution
\end{itemize}
\end{itemize}

\item Bailyn: The Ideological Origins of the American Revolution

\begin{itemize}
\item british political radicals?
\item enslaving america via taxes?
\end{itemize}

\item Breen: The Marketplace of Revolution

\begin{itemize}
\item social resources allowed people to unite across colonies

\begin{itemize}
\item shared rejection of goods!
\end{itemize}
\end{itemize}

\item Linebaugh and Rediker: Many Headed Hydra

\begin{itemize}
\item Not about great men, it's really about

\begin{itemize}
\item sailors, slaves, mobs
\end{itemize}

\item marxist! rise of the proletariat!
\end{itemize}
\end{itemize}
\end{itemize}

[forced founders + ->

bailyn -> [wood, holton] ->

linebaugh: breen: [wood, holton] : bailyn <- time

\begin{itemize}
\item many different people united.

\begin{itemize}
\item how did they unite?
\end{itemize}

\item people could united their ideals through shared rejection of goods

\begin{itemize}
\item why did they have the ideals?
\end{itemize}

\item shared lived experience -- in debt

\begin{itemize}
\item why did they have that lived experience -- why were they in debt?
\end{itemize}

\item taxes!

\begin{itemize}
\item why did they have taxes?
\end{itemize}
\end{itemize}

the complex nature of causality

outline: - thesis - causality is complex and historians pick a small
piece that matches a ideological agenda -

with this updated model, the "cause" of the revolution becomes
multifactorial \{graph\}

proximal - distal

given complexity, how do we think about causality? exponential
reweighting

C -> c --> c --> c -> E 1 2 3 4 \^{} this c

them: [c, c, c] -> me: c -> c -> c

luck:

inp -> [] -> oup

[] = (divine) luck = [god*random]

A

A -> -> -> B e = c

\begin{verse}
e != 0\\
\end{verse}

\textbf{primary cause is flawed}

history = [] -> oup [] ??

( \{ ) \}

() \{\}

models: god random (luck) primary cause too interdependant

theology -> modernism -> postmoderism!

groups:

\begin{itemize}
\item theological

\begin{itemize}
\item by god
\end{itemize}

\item modernism

\begin{itemize}
\item by cause
\end{itemize}

\item postmodernism

\begin{itemize}
\item no cause
\end{itemize}

\item transcendent-modernism

\begin{itemize}
\item graph
\end{itemize}
\end{itemize}

objective truth != exist useful truth = exist postmodern -> useful truth
!= exist

\subsection{Thesis!}
\label{sec:orgee5e1b3}
models of causality are based on

historians look for causes.

causes historians find

the concept of the cause of the american revolution has evolved

perceived cause has co-evolved with changes in the view of causality
itself -- it has gone from theological causality, to modernistic
causality, postmodernistic

*the frameworks in which we situate causes have evolved just as the
assigned causes themselves have changed. we can understand these
frameworks -- how they have shifted, and even where they should go -- by
looking at them from a higher level of abstraction: the evolving view of
causality itself.*

\subsection{Outlining}
\label{sec:org6efe298}
\begin{itemize}
\item intro: models of causality

\begin{itemize}
\item the frameworks in which we situate causes have evolved just as the
assigned causes themselves have changed.
\item we can understand these frameworks -- how they have shifted, and
even where they should go -- by looking at them from a higher level
of abstraction: the evolving view of causality itself.
\end{itemize}

\item theological causality

\begin{itemize}
\item the american revolution was caused by god / divine intervention, and
we can't hope to understand it
\end{itemize}

\item modernistic causality

\begin{itemize}
\item the american revolution was caused by this cause, and this lets us
understand it
\end{itemize}

\item postmodernistic causality

\begin{itemize}
\item the american revolution had no cause, and we cannot understand it
\end{itemize}

\item transcendent-modernism

\begin{itemize}
\item the american revolution had a graph of causes, and while we may not
be able to understand it, we can still learn from it
\end{itemize}

\item conclusion
\end{itemize}

\subsection{Evidencing}
\label{sec:org965347a}
\begin{itemize}
\item theological

\begin{itemize}
\item breen, end of page 7.

\begin{itemize}
\item "divine luck defying close analysis"
\end{itemize}

\item declaration of independence!

\begin{itemize}
\item natures god
\item \url{https://www.washingtonpost.com/news/volokh-conspiracy/wp/2015/07/05/the-declaration-of-independence-and-god/}
\end{itemize}
\end{itemize}

\item modernism

\begin{itemize}
\item breen!
\item holton!
\item bailyn!
\item rettiker!
\end{itemize}

\item postmodernistic

\begin{itemize}
\item the existance of all these different intepretations?
\end{itemize}

\item transcendent-modernism

\begin{itemize}
\item my thing! with the graph!
\end{itemize}
\end{itemize}

\subsection{finer}
\label{sec:orga603867}
\begin{itemize}
\item intro: models of causality

\begin{itemize}
\item the frameworks in which we situate causes have evolved just as the
assigned causes themselves have changed.
\item we can understand these frameworks -- how they have shifted, and
even where they should go -- by looking at them from a higher level
of abstraction: the evolving view of causality itself.
\end{itemize}

\item theological causality

\begin{itemize}
\item the american revolution was caused by god / divine intervention, and
we shoudn't try to understand it

\begin{itemize}
\item breen, constitution
\item also includes just luck that we shouldnt try to understand

\begin{itemize}
\item 13 clocks striking at the same time
\end{itemize}
\end{itemize}

\item asks: how do we please god?
\end{itemize}

\item modernistic causality

\begin{itemize}
\item the american revolution was caused by this cause, and this lets us
understand it

\begin{itemize}
\item while the actual causes vary, this model is the most prominent

\begin{itemize}
\item causes vary from x to y

\begin{itemize}
\item from holton to rediker
\end{itemize}
\end{itemize}

\item distinction from theological with god being the cause, is that
modernism claims to be able to understand from their cause
\end{itemize}

\item asks: what were the true causes?
\end{itemize}

\item postmodernistic causality

\begin{itemize}
\item the american revolution had no cause, and we cannot understand it
\item this is the next logical step, from such a wide pool of perspectives
\item posits that no lessons can be learned; where transcendent-modernism
differs
\item asks: nothing.
\end{itemize}

\item transcendent-modernism

\begin{itemize}
\item the american revolution had a graph of causes, and while we may not
be able to understand it, we can still learn from it
\item while modernisism argues that there is a solution to the problem of
history, postmodernism argues there isnt,

\begin{itemize}
\item transcendent modernism argues we don't need one.
\end{itemize}

\item model causality as a graph with infinitesimal granularity

\begin{itemize}
\item the graph!
\end{itemize}

\item how did the factors interact, and what can we learn from them?
\end{itemize}

\item conclusion

\begin{itemize}
\item ?? maybe,

\begin{itemize}
\item all models of dealing with a "black wall" (instead of black box)
\item dont worry
\item heres the solution
\item there isnt a solution
\item there doesnt need to be a solution
\end{itemize}
\end{itemize}
\end{itemize}

\subsection{Quote bin}
\label{sec:orgf4c84ab}
\subsection{fridge}
\label{sec:org80b0bef}
as the very framework in which we situate causes has evolved just as the
assigned causes themselves have changed.

\section{Writing. Begin.}
\label{sec:org55e95c2}
Grappling with causality is inherent in the historical analysis process.
This process is predicated on the concept of cause and effect;
constructed histories take an effect, assign a cause, then explain the
connection between them. Even with a single effect, say, the American
revolution, these histories take many different forms and evolve as time
goes on. They have shifted from arguments involving divine right,
through the "great men" narrative, and into economic and cultural
explanations. These are all assigned causes to the American revolution,
the evolution of which is a history in and of itself. However, just as
these assigned causes evolve over time, the frameworks these causes
reside in have coevolved. The very way we think about causality has
changed, allowing us to ask and consider new questions which are only
allowed to emerge in certain frameworks of causality. We can understand
these frameworks -- how they have shifted, what their limits are, and
even where they should go -- by looking at them from a higher level of
abstraction: the evolving view of causality itself.

One of the first frameworks took the form of theological causality, a
framework which most modern historians reject. This framework's causes
stem from the divine: god caused the American revolution to happen, and
we shoudn't try to understand it. The modern historian Breen points to
this framework as something that should be avoided in his 2004 book \emph{The
Marketplace of Revolution}. He calls the histories emerging from this
framework

"a kind of divine blessing defying close analysis"

Causes in this framework stem from the divine, and thus, we should not
try to understandf

When applied to the American revolution,
\end{document}
