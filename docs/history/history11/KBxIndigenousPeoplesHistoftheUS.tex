% Created 2021-09-11 Sat 16:40
% Intended LaTeX compiler: xelatex
\documentclass[letterpaper]{article}
\usepackage{graphicx}
\usepackage{grffile}
\usepackage{longtable}
\usepackage{wrapfig}
\usepackage{rotating}
\usepackage[normalem]{ulem}
\usepackage{amsmath}
\usepackage{textcomp}
\usepackage{amssymb}
\usepackage{capt-of}
\usepackage{hyperref}
\usepackage[margin=1in]{geometry}
\usepackage{fontspec}
\usepackage{indentfirst}
\setmainfont[ItalicFont = LiberationSans-Italic, BoldFont = LiberationSans-Bold, BoldItalicFont = LiberationSans-BoldItalic]{LiberationSans}
\newfontfamily\NHLight[ItalicFont = LiberationSansNarrow-Italic, BoldFont       = LiberationSansNarrow-Bold, BoldItalicFont = LiberationSansNarrow-BoldItalic]{LiberationSansNarrow}
\newcommand\textrmlf[1]{{\NHLight#1}}
\newcommand\textitlf[1]{{\NHLight\itshape#1}}
\let\textbflf\textrm
\newcommand\textulf[1]{{\NHLight\bfseries#1}}
\newcommand\textuitlf[1]{{\NHLight\bfseries\itshape#1}}
\usepackage{fancyhdr}
\pagestyle{fancy}
\usepackage{titlesec}
\usepackage{titling}
\makeatletter
\lhead{\textbf{\@title}}
\makeatother
\rhead{\textrmlf{Compiled} \today}
\lfoot{\theauthor\ \textbullet \ \textbf{2021-2022}}
\cfoot{}
\rfoot{\textrmlf{Page} \thepage}
\titleformat{\section} {\Large} {\textrmlf{\thesection} {|}} {0.3em} {\textbf}
\titleformat{\subsection} {\large} {\textrmlf{\thesubsection} {|}} {0.2em} {\textbf}
\titleformat{\subsubsection} {\large} {\textrmlf{\thesubsubsection} {|}} {0.1em} {\textbf}
\setlength{\parskip}{0.45em}
\renewcommand\maketitle{}
\author{Huxley Marvit}
\date{\today}
\title{Dunbar\textsubscript{Ortiz}\textsubscript{intro} to Indigenous peoples history}
\hypersetup{
 pdfauthor={Huxley Marvit},
 pdftitle={Dunbar\textsubscript{Ortiz}\textsubscript{intro} to Indigenous peoples history},
 pdfkeywords={},
 pdfsubject={},
 pdfcreator={Emacs 27.2 (Org mode 9.4.4)}, 
 pdflang={English}}
\begin{document}

\maketitle
\#flo \#ref \#disorganized \#incomplete \#hw

\noindent\rule{\textwidth}{0.5pt}

\section{Begin.}
\label{sec:org810dedc}
Author: historian, read a lot of stuff, said that none of them gave him
the perspective he got from experience had a rough childhood

\begin{verbatim}
*We are here to educate, not forgive. We are here to enlighten, not accuse.* - Willie Johns
\end{verbatim}

\begin{itemize}
\item people did bad things

\begin{itemize}
\item learning and knowing the hist is "both a nessesity and a
responsibility to the ancestors and descendatns of all parties."
\end{itemize}

\item "everything in US history is about the land" what?? is it? what about
laws, or the concepts, or the market??

\begin{itemize}
\item could make a "everything stems from" arg but that's awfully
reductionistic
\end{itemize}

\item argues that reconciliation is not visible in modern day

\begin{itemize}
\item but arnt we getting better?
\item says not even in utopian dreams -- people dont want it to be better?
\end{itemize}

\item \begin{quote}
\textbf{how might ackknowloging the reality of US history work to transform
society? That is the central question this book pursues.}
\end{quote}

\item original narratives were puritan settlers had a covenant with god to
take the land

\item postmodernist studies called for \emph{agency} "under the guise of
indivudual and collective emporwment" blaming the bad of colonialism
on the natives

\item post-civil-rights multiculturalism means no indigenous communites?

\item all meant to disguise the fact that the existence of america is based
on the looting of the entire continent

\begin{itemize}
\item wait isnt that how land works?
\end{itemize}
\end{itemize}

\begin{verbatim}
**Man·i·fest Des·tin·y**

_noun_

1.  the 19th-century doctrine or belief that the expansion of the US throughout the American continents was both justified and inevitable.
\end{verbatim}

\begin{itemize}
\item multiculturalism: manifest destiny won
\item def of modern genocide: terminate survival as peoples

\begin{itemize}
\item premodern: extreme violence without the goal of extincshtion
\end{itemize}

\item united states is a result of the colonial process
\end{itemize}

\begin{verbatim}
if there was no one here, how different would it be?
would america not be successful?
\end{verbatim}

\begin{itemize}
\item natives survived and hold their hist
\item argues that postmodernism and multiculturalism emerged from
neocolonialism?
\item fundemental prob: absence of colonial framework?
\item says, US did economic penetration of native societis, and made them
economicaly dependent

\begin{itemize}
\item imbalance of trade
\item incorporated natives into spheres of influence
\item controlled them with christian missionaries and alchohal?
\end{itemize}

\item eatin food\ldots{}.
\end{itemize}

\href{KBxNotesonHannahJones.org}{KBxNotesonHannahJones}
\end{document}
