% Created 2021-09-11 Sat 16:41
% Intended LaTeX compiler: xelatex
\documentclass[letterpaper]{article}
\usepackage{graphicx}
\usepackage{grffile}
\usepackage{longtable}
\usepackage{wrapfig}
\usepackage{rotating}
\usepackage[normalem]{ulem}
\usepackage{amsmath}
\usepackage{textcomp}
\usepackage{amssymb}
\usepackage{capt-of}
\usepackage{hyperref}
\usepackage[margin=1in]{geometry}
\usepackage{fontspec}
\usepackage{indentfirst}
\setmainfont[ItalicFont = LiberationSans-Italic, BoldFont = LiberationSans-Bold, BoldItalicFont = LiberationSans-BoldItalic]{LiberationSans}
\newfontfamily\NHLight[ItalicFont = LiberationSansNarrow-Italic, BoldFont       = LiberationSansNarrow-Bold, BoldItalicFont = LiberationSansNarrow-BoldItalic]{LiberationSansNarrow}
\newcommand\textrmlf[1]{{\NHLight#1}}
\newcommand\textitlf[1]{{\NHLight\itshape#1}}
\let\textbflf\textrm
\newcommand\textulf[1]{{\NHLight\bfseries#1}}
\newcommand\textuitlf[1]{{\NHLight\bfseries\itshape#1}}
\usepackage{fancyhdr}
\pagestyle{fancy}
\usepackage{titlesec}
\usepackage{titling}
\makeatletter
\lhead{\textbf{\@title}}
\makeatother
\rhead{\textrmlf{Compiled} \today}
\lfoot{\theauthor\ \textbullet \ \textbf{2021-2022}}
\cfoot{}
\rfoot{\textrmlf{Page} \thepage}
\titleformat{\section} {\Large} {\textrmlf{\thesection} {|}} {0.3em} {\textbf}
\titleformat{\subsection} {\large} {\textrmlf{\thesubsection} {|}} {0.2em} {\textbf}
\titleformat{\subsubsection} {\large} {\textrmlf{\thesubsubsection} {|}} {0.1em} {\textbf}
\setlength{\parskip}{0.45em}
\renewcommand\maketitle{}
\author{Taproot}
\date{\today}
\title{Taproot}
\hypersetup{
 pdfauthor={Taproot},
 pdftitle={Taproot},
 pdfkeywords={},
 pdfsubject={},
 pdfcreator={Emacs 27.2 (Org mode 9.4.4)}, 
 pdflang={English}}
\begin{document}

\maketitle

\section{Welcome}
\label{sec:org749a7a6}
Howdy 👋, welcome to Taproot. Take a look around, either in person or this \href{https://taproot3.sanity.gq}{Handy Web Portal}

\section{Take a Looksie}
\label{sec:org434484a}

\section{Philosophy}
\label{sec:org1db14da}
Zettelkasten, maybe. But basically, create a repository of knowledge that should be easy to refer back to and effective for relearning things.
We strive to create atomic, self contained notes that link to other references. Think a more granular Wikipedia.

\section{Structure}
\label{sec:orgf7a29bc}
At the moment, Taproot is organized by the course that each concept falls into.
The project was started with Zettelkasten style IDs prefixed with "KB", but was soon moved to semantic naming.

\section{Epilogue}
\label{sec:org5763f58}

Thanks for stopping by!

A "taproot" is a "large, central, and dominant root from which other roots sprout laterally" \href{https://en.wikipedia.org/wiki/Taproot}{See Wikipedia}. Also, it's a food!
\end{document}
