% Created 2021-09-17 Fri 23:15
% Intended LaTeX compiler: xelatex
\documentclass[letterpaper]{article}
\usepackage{graphicx}
\usepackage{grffile}
\usepackage{longtable}
\usepackage{wrapfig}
\usepackage{rotating}
\usepackage[normalem]{ulem}
\usepackage{amsmath}
\usepackage{textcomp}
\usepackage{amssymb}
\usepackage{capt-of}
\usepackage{hyperref}
\setlength{\parindent}{0pt}
\usepackage[margin=1in]{geometry}
\usepackage{fontspec}
\usepackage{svg}
\usepackage{indentfirst}
\setmainfont[ItalicFont = LiberationSans-Italic, BoldFont = LiberationSans-Bold, BoldItalicFont = LiberationSans-BoldItalic]{LiberationSans}
\newfontfamily\NHLight[ItalicFont = LiberationSansNarrow-Italic, BoldFont       = LiberationSansNarrow-Bold, BoldItalicFont = LiberationSansNarrow-BoldItalic]{LiberationSansNarrow}
\newcommand\textrmlf[1]{{\NHLight#1}}
\newcommand\textitlf[1]{{\NHLight\itshape#1}}
\let\textbflf\textrm
\newcommand\textulf[1]{{\NHLight\bfseries#1}}
\newcommand\textuitlf[1]{{\NHLight\bfseries\itshape#1}}
\usepackage{fancyhdr}
\pagestyle{fancy}
\usepackage{titlesec}
\usepackage{titling}
\makeatletter
\lhead{\textbf{\@title}}
\makeatother
\rhead{\textrmlf{Compiled} \today}
\lfoot{\theauthor\ \textbullet \ \textbf{2021-2022}}
\cfoot{}
\rfoot{\textrmlf{Page} \thepage}
\titleformat{\section} {\Large} {\textrmlf{\thesection} {|}} {0.3em} {\textbf}
\titleformat{\subsection} {\large} {\textrmlf{\thesubsection} {|}} {0.2em} {\textbf}
\titleformat{\subsubsection} {\large} {\textrmlf{\thesubsubsection} {|}} {0.1em} {\textbf}
\setlength{\parskip}{0.45em}
\renewcommand\maketitle{}
\author{Huxley}
\date{\today}
\title{Gender and Biology ISOS Reading}
\hypersetup{
 pdfauthor={Huxley},
 pdftitle={Gender and Biology ISOS Reading},
 pdfkeywords={},
 pdfsubject={},
 pdfcreator={Emacs 28.0.50 (Org mode 9.4.4)}, 
 pdflang={English}}
\begin{document}

\maketitle
\#flo

\section{\[Gender\ and\ the\ Biological\ Sciences\]}
\label{sec:org3ad9d12}
\begin{quote}
Feminist critiques of science provide fertile ground for any
investigation of the ways in which social influences may shape the
content of science.
\end{quote}

Oh jeez, this is gonna be a fun discussion

Argue that sexist bio is an excuse for the suppression of women.

\textbf{Kathleen's Hierarchy of sciences:} Physics -> Biology -> Social
sciences (or reversed, it's unclear)

Argues that feminist critiques of "the social sciences are dismissed out
of hand by philosophers of science on the ground that the social
sciences arn't science anyways," then says: "it is, however, not quite
so easy to dismiss biology as pseudo science." Despite the fact that she
\emph{just drew a distinction between biology and social sciences.}

Calls case study of outdated model of fertilization process -- says that
sexism has blinded scientists to the truth, the previous model being the
"Sleeping Beauty/Prince Charming" model. > The egg waits patiently [\ldots{}]
the sperm heroically battles [\ldots{}] (yatta yatta)

this model "appeared as early as 1795." In 1895, electron microscopy on
sea urchins revealed that the egg grows “finger like projections
\href{....org}{\ldots{}} but this has been largely ignored until recently.

The original model appeared 100 years earlier, of course it was widely
know in 1895. The new model barely changed the previous model. What does
"largely ignored" even mean? To my knowledge, this isn't a very hotly
debated or discussed topic. Is that what she means by ignoring it?

She goes on to say that the new theory is still controversial, and they
don't know if it's entirely correct!

Of course people stuck with teaching the old model then! And the
electron microscope was invented in 1931! She claims this was discovered
using electron microscopy in 1895!

\begin{enumerate}
\item wtaf is this reading?
\label{sec:orga8ef033}
Gives examples of clearly incorrect biological explanations of why men
are more intelligent that women form specific scientists, and uses them
to generalize all of male biologists. \emph{Nice.}

Drawings that updated skeletons to be gender specific were "favored for
just that (the fact that they reflected cultural ideals of masculinity)
over drawings that were in some sense more accurate."

What exactly does "in some sense more accurate" mean?

\begin{quote}
A science based upon women would be an improvement over the current
science, according to standpoint epistemology.
\end{quote}

Essentially argues that between two choices, scientific theory cannot
guarantee that one is true. Yeah, that's the point.

Argues that since theories are based upon one another, and since
original theories were created in deeply sexist times, the sexist
theories persevere regardless of the current state of cultural sexism.

This is reasonable, but we scientists do backtrack significantly when
new evidence is presented. It's not as if these sexists theories have
tainted modern theories forever.
\end{enumerate}

\subsection{Reflection}
\label{sec:orgd1741e6}
Well, that was certainly interesting.

\subsubsection{Discussion Point:}
\label{sec:org678c1d6}
I wonder what the author means by "in some sense more accurate" when
referring to the conflicting skeletal models on page 198.

\noindent\rule{\textwidth}{0.5pt}

All I have to say about this reading. Holy hell.
\end{document}
