% Created 2021-09-12 Sun 22:49
% Intended LaTeX compiler: xelatex
\documentclass[letterpaper]{article}
\usepackage{graphicx}
\usepackage{grffile}
\usepackage{longtable}
\usepackage{wrapfig}
\usepackage{rotating}
\usepackage[normalem]{ulem}
\usepackage{amsmath}
\usepackage{textcomp}
\usepackage{amssymb}
\usepackage{capt-of}
\usepackage{hyperref}
\usepackage[margin=1in]{geometry}
\usepackage{fontspec}
\usepackage{indentfirst}
\setmainfont[ItalicFont = LiberationSans-Italic, BoldFont = LiberationSans-Bold, BoldItalicFont = LiberationSans-BoldItalic]{LiberationSans}
\newfontfamily\NHLight[ItalicFont = LiberationSansNarrow-Italic, BoldFont       = LiberationSansNarrow-Bold, BoldItalicFont = LiberationSansNarrow-BoldItalic]{LiberationSansNarrow}
\newcommand\textrmlf[1]{{\NHLight#1}}
\newcommand\textitlf[1]{{\NHLight\itshape#1}}
\let\textbflf\textrm
\newcommand\textulf[1]{{\NHLight\bfseries#1}}
\newcommand\textuitlf[1]{{\NHLight\bfseries\itshape#1}}
\usepackage{fancyhdr}
\pagestyle{fancy}
\usepackage{titlesec}
\usepackage{titling}
\makeatletter
\lhead{\textbf{\@title}}
\makeatother
\rhead{\textrmlf{Compiled} \today}
\lfoot{\theauthor\ \textbullet \ \textbf{2021-2022}}
\cfoot{}
\rfoot{\textrmlf{Page} \thepage}
\titleformat{\section} {\Large} {\textrmlf{\thesection} {|}} {0.3em} {\textbf}
\titleformat{\subsection} {\large} {\textrmlf{\thesubsection} {|}} {0.2em} {\textbf}
\titleformat{\subsubsection} {\large} {\textrmlf{\thesubsubsection} {|}} {0.1em} {\textbf}
\setlength{\parskip}{0.45em}
\renewcommand\maketitle{}
\author{Huxley}
\date{\today}
\title{Patterns of Discovery Notes}
\hypersetup{
 pdfauthor={Huxley},
 pdftitle={Patterns of Discovery Notes},
 pdfkeywords={},
 pdfsubject={},
 pdfcreator={Emacs 28.0.50 (Org mode 9.4.4)}, 
 pdflang={English}}
\begin{document}

\maketitle
\noindent\rule{\textwidth}{0.5pt}

\#flo \# \[Patterns\ of\ Discovery\]

Observe two different things even though their > eyesight is normal and
they are visually aware of the same object

\begin{verbatim}
Thought as a filter? 



Opinion effects perception

One see's amoeba as single celled, the other as one celled 


> seeing is an experience. A retinal reaction is only a physical state - a photo-chemical excitation. 

*People* see, not eyes. Eyes are **blind.** 


Optical illusions can be viewed differently, and are not skewed by opinion. 



> The infant and the layman can see: they are not blind. But they cannot see what the physicist sees; they are blind to what he sees. 


This argues that perception (not interpretation) is effected by knowledge.

> seeing is a 'theory-laden' undertaking. 



Disability to prove / disprove via experiment 

\end{verbatim}

\begin{verbatim}
## $$Discussion\ Point:$$ 
\end{verbatim}

```

I may be interpreting this paper incorrectly, but it seems to me that
they are defining 'seeing' as a synonym to 'perception.' If not, where
does the line between 'seeing' and 'perception' fall? If so, what is the
point of their definition of 'seeing?' Doesn't defining it this way just
reduce meaning? Isn't the point of having both the words 'seeing' and
'perception' to be able to draw a distinction between them?

Also, on page 15, the author practically equates seeing to knowledge. I
wonder if they have ever read Theaetetus\ldots{}
\end{document}
