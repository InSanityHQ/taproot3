% Created 2021-09-17 Fri 23:16
% Intended LaTeX compiler: xelatex
\documentclass[letterpaper]{article}
\usepackage{graphicx}
\usepackage{grffile}
\usepackage{longtable}
\usepackage{wrapfig}
\usepackage{rotating}
\usepackage[normalem]{ulem}
\usepackage{amsmath}
\usepackage{textcomp}
\usepackage{amssymb}
\usepackage{capt-of}
\usepackage{hyperref}
\setlength{\parindent}{0pt}
\usepackage[margin=1in]{geometry}
\usepackage{fontspec}
\usepackage{svg}
\usepackage{indentfirst}
\setmainfont[ItalicFont = LiberationSans-Italic, BoldFont = LiberationSans-Bold, BoldItalicFont = LiberationSans-BoldItalic]{LiberationSans}
\newfontfamily\NHLight[ItalicFont = LiberationSansNarrow-Italic, BoldFont       = LiberationSansNarrow-Bold, BoldItalicFont = LiberationSansNarrow-BoldItalic]{LiberationSansNarrow}
\newcommand\textrmlf[1]{{\NHLight#1}}
\newcommand\textitlf[1]{{\NHLight\itshape#1}}
\let\textbflf\textrm
\newcommand\textulf[1]{{\NHLight\bfseries#1}}
\newcommand\textuitlf[1]{{\NHLight\bfseries\itshape#1}}
\usepackage{fancyhdr}
\pagestyle{fancy}
\usepackage{titlesec}
\usepackage{titling}
\makeatletter
\lhead{\textbf{\@title}}
\makeatother
\rhead{\textrmlf{Compiled} \today}
\lfoot{\theauthor\ \textbullet \ \textbf{2021-2022}}
\cfoot{}
\rfoot{\textrmlf{Page} \thepage}
\titleformat{\section} {\Large} {\textrmlf{\thesection} {|}} {0.3em} {\textbf}
\titleformat{\subsection} {\large} {\textrmlf{\thesubsection} {|}} {0.2em} {\textbf}
\titleformat{\subsubsection} {\large} {\textrmlf{\thesubsubsection} {|}} {0.1em} {\textbf}
\setlength{\parskip}{0.45em}
\renewcommand\maketitle{}
\author{Dylan}
\date{\today}
\title{Gender and Biology Reading}
\hypersetup{
 pdfauthor={Dylan},
 pdftitle={Gender and Biology Reading},
 pdfkeywords={},
 pdfsubject={},
 pdfcreator={Emacs 28.0.50 (Org mode 9.4.4)}, 
 pdflang={English}}
\begin{document}

\maketitle


\section{Hoo Boy!}
\label{sec:org05d555b}
This is gonna be fun\ldots{} See
\href{KBGenderAndBioISOS.org}{KBGenderAndBioISOS} for Huxley's take

\section{Gist}
\label{sec:org003af5d}
\begin{itemize}
\item Biology is mainly dominated by men

\begin{itemize}
\item Therefore, it is \emph{biased} towards men
\item "Feminist" hypotheses have been dismissed
\item "Social Sciences" ignored for "not being science"
\end{itemize}

\item Culture has had a large impact on the sciences

\begin{itemize}
\item Critical Theory lmao
\end{itemize}

\item Feminists have tried to reverse this trend

\begin{itemize}
\item Bring attention to the domination of men!
\item Postmodernist feminism: Feelings over facts

\begin{itemize}
\item Ben Shapiro Seething
\end{itemize}
\end{itemize}

\item Scientific method and procedure are "tainted" with bias and
sociological factors

\begin{itemize}
\item Let's not do science anymore :)
\item For and against

\begin{itemize}
\item David Bloor: \emph{Ceteris Paribus}, people favor theories that appeal
to them on a sociological level
\item Larry Laudain: There aren't many good theories to choose from and
there's good reason to choose one over the other

\begin{itemize}
\item Premise of David Bloor's point is incorrect
\item "Oh, but how are those good theories chosen??"
\end{itemize}
\end{itemize}
\end{itemize}

\item We have to be diverse when picking theories and hypotheses or else we
won't find change
\end{itemize}

\section{Thoughts}
\label{sec:orgfb3a12a}
\begin{itemize}
\item The author frequently makes brash claims about certain real world
theories, like female skeletons and brains, but never presents any
evidence for those theories.
\item Although I could see the intent behind writing this article, I think
that it could have been summed up as "Scientific results could be
biased because humans find them."
\item Cherry-picking evidence? There are many examples of females performing
better at certain tasks, but the author never mentions these.
\item What is the point anyways? We already established in class that it is
impossible to know anything for sure, and therefore science is just a
bogus explanation of what we see. It is just as rigorous as religion
at explaining the world, because, while it may seem to be more
rigorous by using "objective" evidence, all objective evidence is
merely information that has been collected and processed subjectively
by humans and presented as objective fact. In fact, as we can't prove
the existence of anything other than ourselves, we cannot trust other
"people" for presenting accurate information. /Science is only useful
if we suspend our disbelief and start believing in certain things like
objective truth./
\end{itemize}

\section{Discussion Points}
\label{sec:org2895597}
\begin{itemize}
\item How does this reading tie in with what Socrates had to say?
\end{itemize}
\end{document}
