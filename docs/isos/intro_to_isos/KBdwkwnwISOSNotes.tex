% Created 2021-09-11 Sat 16:42
% Intended LaTeX compiler: xelatex
\documentclass[letterpaper]{article}
\usepackage{graphicx}
\usepackage{grffile}
\usepackage{longtable}
\usepackage{wrapfig}
\usepackage{rotating}
\usepackage[normalem]{ulem}
\usepackage{amsmath}
\usepackage{textcomp}
\usepackage{amssymb}
\usepackage{capt-of}
\usepackage{hyperref}
\usepackage[margin=1in]{geometry}
\usepackage{fontspec}
\usepackage{indentfirst}
\setmainfont[ItalicFont = LiberationSans-Italic, BoldFont = LiberationSans-Bold, BoldItalicFont = LiberationSans-BoldItalic]{LiberationSans}
\newfontfamily\NHLight[ItalicFont = LiberationSansNarrow-Italic, BoldFont       = LiberationSansNarrow-Bold, BoldItalicFont = LiberationSansNarrow-BoldItalic]{LiberationSansNarrow}
\newcommand\textrmlf[1]{{\NHLight#1}}
\newcommand\textitlf[1]{{\NHLight\itshape#1}}
\let\textbflf\textrm
\newcommand\textulf[1]{{\NHLight\bfseries#1}}
\newcommand\textuitlf[1]{{\NHLight\bfseries\itshape#1}}
\usepackage{fancyhdr}
\pagestyle{fancy}
\usepackage{titlesec}
\usepackage{titling}
\makeatletter
\lhead{\textbf{\@title}}
\makeatother
\rhead{\textrmlf{Compiled} \today}
\lfoot{\theauthor\ \textbullet \ \textbf{2021-2022}}
\cfoot{}
\rfoot{\textrmlf{Page} \thepage}
\titleformat{\section} {\Large} {\textrmlf{\thesection} {|}} {0.3em} {\textbf}
\titleformat{\subsection} {\large} {\textrmlf{\thesubsection} {|}} {0.2em} {\textbf}
\titleformat{\subsubsection} {\large} {\textrmlf{\thesubsubsection} {|}} {0.1em} {\textbf}
\setlength{\parskip}{0.45em}
\renewcommand\maketitle{}
\author{Huxley}
\date{\today}
\title{How do we know were not wrong}
\hypersetup{
 pdfauthor={Huxley},
 pdftitle={How do we know were not wrong},
 pdfkeywords={},
 pdfsubject={},
 pdfcreator={Emacs 27.2 (Org mode 9.4.4)}, 
 pdflang={English}}
\begin{document}

\maketitle
\#flo \#disorganized

\noindent\rule{\textwidth}{0.5pt}

\section{So, tell me, how do we know?}
\label{sec:org6fe402f}
So scientists believe that climate change is real, and the American
people are misinformed.

\textbf{This reading will ask: "Might the scientific consensus be wrong?"}

Been wrong before, might be wrong again. How do we know?

\subsection{What is consensus and how do we find it?}
\label{sec:org073b717}
Interconnection of discoveries, hard for outsiders to understand.

No consensus on tempo and mode of climate change

\subsection{Perceived disagreement}
\label{sec:org755f0f8}
Uncertainty about the future is conflated with uncertainty about current
knowledge

Lack of communication from scientists

Culture of not communicating to the masses

\texttt{=in science, the question cannot be if they might be mistaken, but
whether there is a reason to think they are mistaken=} wait no\ldots{}

Hard to falsifie models of future. What they do is see how well it can
produce past events.

\subsection{Discussion Point}
\label{sec:org8f916cd}
I knew climate change wasn't real! I knew it all along!

This reading really didn't seem to be that novel or interesting to me.
Perhaps the most interesting bit was the part about applying the
'scientific method' to models of the future. Continued tweaking does not
guarantee good results, only the right results given a specific
scenario. Continued tweaking, in one scenario, can account for missing
something broader, and provide the correct result while being entirely
wrong. It's quite fascinating to think about applying 'old' ways of
science to things that are only possible with modern day technology, and
deal with new emerging problems due to our changing capabilities.
\end{document}
