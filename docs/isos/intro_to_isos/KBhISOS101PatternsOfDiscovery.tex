% Created 2021-09-11 Sat 16:42
% Intended LaTeX compiler: xelatex
\documentclass[letterpaper]{article}
\usepackage{graphicx}
\usepackage{grffile}
\usepackage{longtable}
\usepackage{wrapfig}
\usepackage{rotating}
\usepackage[normalem]{ulem}
\usepackage{amsmath}
\usepackage{textcomp}
\usepackage{amssymb}
\usepackage{capt-of}
\usepackage{hyperref}
\usepackage[margin=1in]{geometry}
\usepackage{fontspec}
\usepackage{indentfirst}
\setmainfont[ItalicFont = LiberationSans-Italic, BoldFont = LiberationSans-Bold, BoldItalicFont = LiberationSans-BoldItalic]{LiberationSans}
\newfontfamily\NHLight[ItalicFont = LiberationSansNarrow-Italic, BoldFont       = LiberationSansNarrow-Bold, BoldItalicFont = LiberationSansNarrow-BoldItalic]{LiberationSansNarrow}
\newcommand\textrmlf[1]{{\NHLight#1}}
\newcommand\textitlf[1]{{\NHLight\itshape#1}}
\let\textbflf\textrm
\newcommand\textulf[1]{{\NHLight\bfseries#1}}
\newcommand\textuitlf[1]{{\NHLight\bfseries\itshape#1}}
\usepackage{fancyhdr}
\pagestyle{fancy}
\usepackage{titlesec}
\usepackage{titling}
\makeatletter
\lhead{\textbf{\@title}}
\makeatother
\rhead{\textrmlf{Compiled} \today}
\lfoot{\theauthor\ \textbullet \ \textbf{2021-2022}}
\cfoot{}
\rfoot{\textrmlf{Page} \thepage}
\titleformat{\section} {\Large} {\textrmlf{\thesection} {|}} {0.3em} {\textbf}
\titleformat{\subsection} {\large} {\textrmlf{\thesubsection} {|}} {0.2em} {\textbf}
\titleformat{\subsubsection} {\large} {\textrmlf{\thesubsubsection} {|}} {0.1em} {\textbf}
\setlength{\parskip}{0.45em}
\renewcommand\maketitle{}
\author{Houjun Liu}
\date{\today}
\title{Patterns of Discovery}
\hypersetup{
 pdfauthor={Houjun Liu},
 pdftitle={Patterns of Discovery},
 pdfkeywords={},
 pdfsubject={},
 pdfcreator={Emacs 27.2 (Org mode 9.4.4)}, 
 pdflang={English}}
\begin{document}

\maketitle


\section{Patterns of Discovery}
\label{sec:org6c478d6}
Importance lies in how \emph{observed} data affects \emph{interpreted} data.

\begin{itemize}
\item Two people may see the same thing, but they interprete wildly
different results
\item Different means and systems of thinking affect data molding and
interpretation
\end{itemize}

Seeing => usually implies both the photochemical reaction and the
interpretation thereof, but difference usually not appreciated.

\begin{itemize}
\item Process between viewing and interpretation not often appreciated
\item People cannot see without interpreting --- \emph{"one simply does not soak
up an optical patter and then clamp an interpretation on it."}
\end{itemize}

Existence of two perspectives possible (\emph{duh}) even when presented to
the same image.

\subsection{Interpretation is hard and faulty}
\label{sec:org0524bd1}
\begin{itemize}
\item Knowledge of whats illustrated is needed to have illustration be
communicated
\item Explanation and knowledge affects or induces interpretation
\item One could be "blind" to a geometric image if background knowledge not
introduced
\end{itemize}

Seeing: \emph{observation of x \ldots{} shaped by prior knowledge of x}.
\end{document}
