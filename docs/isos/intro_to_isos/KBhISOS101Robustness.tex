% Created 2021-09-11 Sat 16:41
% Intended LaTeX compiler: xelatex
\documentclass[letterpaper]{article}
\usepackage{graphicx}
\usepackage{grffile}
\usepackage{longtable}
\usepackage{wrapfig}
\usepackage{rotating}
\usepackage[normalem]{ulem}
\usepackage{amsmath}
\usepackage{textcomp}
\usepackage{amssymb}
\usepackage{capt-of}
\usepackage{hyperref}
\usepackage[margin=1in]{geometry}
\usepackage{fontspec}
\usepackage{indentfirst}
\setmainfont[ItalicFont = LiberationSans-Italic, BoldFont = LiberationSans-Bold, BoldItalicFont = LiberationSans-BoldItalic]{LiberationSans}
\newfontfamily\NHLight[ItalicFont = LiberationSansNarrow-Italic, BoldFont       = LiberationSansNarrow-Bold, BoldItalicFont = LiberationSansNarrow-BoldItalic]{LiberationSansNarrow}
\newcommand\textrmlf[1]{{\NHLight#1}}
\newcommand\textitlf[1]{{\NHLight\itshape#1}}
\let\textbflf\textrm
\newcommand\textulf[1]{{\NHLight\bfseries#1}}
\newcommand\textuitlf[1]{{\NHLight\bfseries\itshape#1}}
\usepackage{fancyhdr}
\pagestyle{fancy}
\usepackage{titlesec}
\usepackage{titling}
\makeatletter
\lhead{\textbf{\@title}}
\makeatother
\rhead{\textrmlf{Compiled} \today}
\lfoot{\theauthor\ \textbullet \ \textbf{2021-2022}}
\cfoot{}
\rfoot{\textrmlf{Page} \thepage}
\titleformat{\section} {\Large} {\textrmlf{\thesection} {|}} {0.3em} {\textbf}
\titleformat{\subsection} {\large} {\textrmlf{\thesubsection} {|}} {0.2em} {\textbf}
\titleformat{\subsubsection} {\large} {\textrmlf{\thesubsubsection} {|}} {0.1em} {\textbf}
\setlength{\parskip}{0.45em}
\renewcommand\maketitle{}
\author{Houjun Liu}
\date{\today}
\title{Robustness}
\hypersetup{
 pdfauthor={Houjun Liu},
 pdftitle={Robustness},
 pdfkeywords={},
 pdfsubject={},
 pdfcreator={Emacs 27.2 (Org mode 9.4.4)}, 
 pdflang={English}}
\begin{document}

\maketitle


\section{Robustness}
\label{sec:orgb5d42e6}
\#flo \#disorganized

\begin{itemize}
\item Aristotle as seed of idea triangulation

\begin{itemize}
\item The idea of confirming a phenomenon through multiple ways of
observation
\item Not a thing that people talk about much, but present in many
philosophies
\end{itemize}

\item Robustness Analysis

\begin{itemize}
\item Based on concept of triangulation
\item Basic steps

\begin{enumerate}
\item Analyze a variety of independent derivations

\begin{itemize}
\item This could mean a lot of things, like

\begin{itemize}
\item Different senses of the same thing
\item Different procidures to sense the same thing
\item Different assumptions to verify the same thing
\item Different tests of the same thing
\end{itemize}
\end{itemize}

\item Look for identical conclusions from these different derivations
\item Analyze the scope and conditions from which these derivations
exist
\item Analyze any failures of the invariance
\end{enumerate}

\item If, under step 4, there be things that are invariant and within the
margin of falure, the analysis is "robust"
\end{itemize}

\item Common theme across all types of robust analysis

\begin{itemize}
\item Distinction between the material and the unmaterial
\item Each verification process is independent
\item Robustness evaluated on the basis of "changeability" --- that is, if
under different circumstances, theories are unmutating, they are
more robusta
\end{itemize}

\item Robustness prevents the "weakest link problem"

\begin{itemize}
\item With multiple derivations under different assumptions, problems
could be spotted independently
\item Thus, if one point in one senario theory breaks down, you either
notice it very quickly or the theory is not entirely disproven
although less robust
\item If one arm is simply weakened, still the others could support the
theory and the special case could further lead to scientific
discovery
\end{itemize}

\item Failures of robustness analysis --- "illusions of robustness"

\begin{itemize}
\item Supposedly independent tests acutally dependent
\item For instance, IQ tests are not actually quite that independent of
social factors
\item Not very easy to detect underlying causes of dependence

\begin{itemize}
\item Factors could be reinforcing
\item Each may hide the others being actually dependen
\end{itemize}
\end{itemize}
\end{itemize}
\end{document}
