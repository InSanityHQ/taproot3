% Created 2021-09-11 Sat 16:42
% Intended LaTeX compiler: xelatex
\documentclass[letterpaper]{article}
\usepackage{graphicx}
\usepackage{grffile}
\usepackage{longtable}
\usepackage{wrapfig}
\usepackage{rotating}
\usepackage[normalem]{ulem}
\usepackage{amsmath}
\usepackage{textcomp}
\usepackage{amssymb}
\usepackage{capt-of}
\usepackage{hyperref}
\usepackage[margin=1in]{geometry}
\usepackage{fontspec}
\usepackage{indentfirst}
\setmainfont[ItalicFont = LiberationSans-Italic, BoldFont = LiberationSans-Bold, BoldItalicFont = LiberationSans-BoldItalic]{LiberationSans}
\newfontfamily\NHLight[ItalicFont = LiberationSansNarrow-Italic, BoldFont       = LiberationSansNarrow-Bold, BoldItalicFont = LiberationSansNarrow-BoldItalic]{LiberationSansNarrow}
\newcommand\textrmlf[1]{{\NHLight#1}}
\newcommand\textitlf[1]{{\NHLight\itshape#1}}
\let\textbflf\textrm
\newcommand\textulf[1]{{\NHLight\bfseries#1}}
\newcommand\textuitlf[1]{{\NHLight\bfseries\itshape#1}}
\usepackage{fancyhdr}
\pagestyle{fancy}
\usepackage{titlesec}
\usepackage{titling}
\makeatletter
\lhead{\textbf{\@title}}
\makeatother
\rhead{\textrmlf{Compiled} \today}
\lfoot{\theauthor\ \textbullet \ \textbf{2021-2022}}
\cfoot{}
\rfoot{\textrmlf{Page} \thepage}
\titleformat{\section} {\Large} {\textrmlf{\thesection} {|}} {0.3em} {\textbf}
\titleformat{\subsection} {\large} {\textrmlf{\thesubsection} {|}} {0.2em} {\textbf}
\titleformat{\subsubsection} {\large} {\textrmlf{\thesubsubsection} {|}} {0.1em} {\textbf}
\setlength{\parskip}{0.45em}
\renewcommand\maketitle{}
\author{Houjun Liu}
\date{\today}
\title{Thaetetus Selections}
\hypersetup{
 pdfauthor={Houjun Liu},
 pdftitle={Thaetetus Selections},
 pdfkeywords={},
 pdfsubject={},
 pdfcreator={Emacs 27.2 (Org mode 9.4.4)}, 
 pdflang={English}}
\begin{document}

\maketitle


\section{Face verification}
\label{sec:org6d4647a}
\begin{itemize}
\item That if Theodorus claims that Theaetetus' face and Socrates' face
looks alike, Theaetetus ought to verify it agaist Theodorus' skill
\item Otherwise, one should not trust him with something he is unskilled of
\end{itemize}

\section{What is knowledge?}
\label{sec:org4824444}
\begin{itemize}
\item Knowledge is (not) wisdom?
\item And also, it's difficult to define knowledge, anyways
\item It's easy to define a point to an instance of knowledge, but difficult
to get at what knowledge \emph{is} itself
\item Simply perceiving something is also not knowledge

\begin{itemize}
\item Perception varies person-to-person
\item Is not necessarily a "stable" metric

\begin{itemize}
\item Different people have different views
\item Crazy people are not muses
\end{itemize}

\item Knowledge, in one case, could be contestable
\item Classification of things are often done as comparison
\item Knowledge is correction judgemen
\end{itemize}
\end{itemize}

\begin{quote}
Then knowledge is to be found not in the experiences but in the
process of reasoning about them; it is here, seemingly, not in the
experiences, that it is possible to grasp being and truth.
\end{quote}

\begin{itemize}
\item So\ldots{} True judgments with evidence to support it => knowledge
\end{itemize}

\subsection{Definitions of Knowledge}
\label{sec:orgfe1a87d}
\subsubsection{Perception}
\label{sec:orgb5d0c5c}
\begin{itemize}
\item Objection:

\begin{itemize}
\item Crazy people + dreams
\item If people mis-see things then it's not quite knowledge
\end{itemize}
\end{itemize}

\subsubsection{True judgement is knowledge}
\label{sec:orge1b339a}
\begin{itemize}
\item Objection

\begin{itemize}
\item to decide whether judgement is true comes from knowledge and
therefore this is a circular definition
\end{itemize}
\end{itemize}

\subsubsection{A true judgement with an account}
\label{sec:org4124fd1}
\begin{itemize}
\item account: an argument or line of reasoning

\begin{itemize}
\item Objection: same as above
\end{itemize}
\end{itemize}
\end{document}
