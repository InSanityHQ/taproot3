% Created 2021-09-11 Sat 09:35
% Intended LaTeX compiler: xelatex
\documentclass[letterpaper]{article}
\usepackage{graphicx}
\usepackage{grffile}
\usepackage{longtable}
\usepackage{wrapfig}
\usepackage{rotating}
\usepackage[normalem]{ulem}
\usepackage{amsmath}
\usepackage{textcomp}
\usepackage{amssymb}
\usepackage{capt-of}
\usepackage{hyperref}
\usepackage[margin=1in]{geometry}
\usepackage{fontspec}
\usepackage{indentfirst}
\setmainfont[ItalicFont = LiberationSans-Italic, BoldFont = LiberationSans-Bold, BoldItalicFont = LiberationSans-BoldItalic]{LiberationSans}
\newfontfamily\NHLight[ItalicFont = LiberationSansNarrow-Italic, BoldFont       = LiberationSansNarrow-Bold, BoldItalicFont = LiberationSansNarrow-BoldItalic]{LiberationSansNarrow}
\newcommand\textrmlf[1]{{\NHLight#1}}
\newcommand\textitlf[1]{{\NHLight\itshape#1}}
\let\textbflf\textrm
\newcommand\textulf[1]{{\NHLight\bfseries#1}}
\newcommand\textuitlf[1]{{\NHLight\bfseries\itshape#1}}
\usepackage{fancyhdr}
\pagestyle{fancy}
\usepackage{titlesec}
\usepackage{titling}
\makeatletter
\lhead{\textbf{\@title}}
\makeatother
\rhead{\textrmlf{Compiled} \today}
\lfoot{\theauthor\ \textbullet \ \textbf{2021-2022}}
\cfoot{}
\rfoot{\textrmlf{Page} \thepage}
\titleformat{\section} {\Large} {\textrmlf{\thesection} {|}} {0.3em} {\textbf}
\titleformat{\subsection} {\large} {\textrmlf{\thesubsection} {|}} {0.2em} {\textbf}
\titleformat{\subsubsection} {\large} {\textrmlf{\thesubsubsection} {|}} {0.1em} {\textbf}
\setlength{\parskip}{0.45em}
\renewcommand\maketitle{}
\author{Zachary Sayyah}
\date{\today}
\title{Patterns of Discovery}
\hypersetup{
 pdfauthor={Zachary Sayyah},
 pdftitle={Patterns of Discovery},
 pdfkeywords={},
 pdfsubject={},
 pdfcreator={Emacs 27.2 (Org mode 9.4.4)}, 
 pdflang={English}}
\begin{document}

\maketitle


\section{Notes}
\label{sec:org36462a0}
\begin{itemize}
\item Two people can study the same thing and come up with different results
due to a small difference in procedure

\begin{itemize}
\item This does not mean that any observation is incorrect, but rather
that they are making incorrect assumptions based upon them
\item They could also both hold part of the picture without either
understanding why both are true
\end{itemize}

\item Some scientists might even see the same data, but derive different
conclusions
\item There is a difference between experiences and physical states

\begin{itemize}
\item People, not their eyeballs see and we can have differences there.
\end{itemize}

\item In reference to different interpretations of what we see, we have the
same visual data that if asked to be reproduced will be roughly the
same, but too the observer is interpreted differently

\begin{itemize}
\item Usually context will give us clues as to how to interpret visual
data

\begin{itemize}
\item We don't interpret things randomly, but rather with our given
context, or in other words our bias
\item Seeing is not the only thing in visual experience as our brain
will automatically put our visual input into context and help us
understand what we are seeing around us
\end{itemize}

\item People without context can see, but they can't interpret what the
people with context can
\end{itemize}
\end{itemize}
\end{document}
