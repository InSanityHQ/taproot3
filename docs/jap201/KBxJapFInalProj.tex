% Created 2021-09-11 Sat 09:35
% Intended LaTeX compiler: xelatex
\documentclass[letterpaper]{article}
\usepackage{graphicx}
\usepackage{grffile}
\usepackage{longtable}
\usepackage{wrapfig}
\usepackage{rotating}
\usepackage[normalem]{ulem}
\usepackage{amsmath}
\usepackage{textcomp}
\usepackage{amssymb}
\usepackage{capt-of}
\usepackage{hyperref}
\usepackage[margin=1in]{geometry}
\usepackage{fontspec}
\usepackage{indentfirst}
\setmainfont[ItalicFont = LiberationSans-Italic, BoldFont = LiberationSans-Bold, BoldItalicFont = LiberationSans-BoldItalic]{LiberationSans}
\newfontfamily\NHLight[ItalicFont = LiberationSansNarrow-Italic, BoldFont       = LiberationSansNarrow-Bold, BoldItalicFont = LiberationSansNarrow-BoldItalic]{LiberationSansNarrow}
\newcommand\textrmlf[1]{{\NHLight#1}}
\newcommand\textitlf[1]{{\NHLight\itshape#1}}
\let\textbflf\textrm
\newcommand\textulf[1]{{\NHLight\bfseries#1}}
\newcommand\textuitlf[1]{{\NHLight\bfseries\itshape#1}}
\usepackage{fancyhdr}
\pagestyle{fancy}
\usepackage{titlesec}
\usepackage{titling}
\makeatletter
\lhead{\textbf{\@title}}
\makeatother
\rhead{\textrmlf{Compiled} \today}
\lfoot{\theauthor\ \textbullet \ \textbf{2021-2022}}
\cfoot{}
\rfoot{\textrmlf{Page} \thepage}
\titleformat{\section} {\Large} {\textrmlf{\thesection} {|}} {0.3em} {\textbf}
\titleformat{\subsection} {\large} {\textrmlf{\thesubsection} {|}} {0.2em} {\textbf}
\titleformat{\subsubsection} {\large} {\textrmlf{\thesubsubsection} {|}} {0.1em} {\textbf}
\setlength{\parskip}{0.45em}
\renewcommand\maketitle{}
\author{Huxley}
\date{\today}
\title{Japanese Final Project}
\hypersetup{
 pdfauthor={Huxley},
 pdftitle={Japanese Final Project},
 pdfkeywords={},
 pdfsubject={},
 pdfcreator={Emacs 27.2 (Org mode 9.4.4)}, 
 pdflang={English}}
\begin{document}

\maketitle
\#ref \#ret

\noindent\rule{\textwidth}{0.5pt}

location: \textbf{別府}

Spent most of my time reading through the examples and consolidating
research.

filtered research to include:

\begin{itemize}
\item Geographical or climatic features

\begin{itemize}
\item hot springs!
\item covered in mist
\item beautiful nature nearby
\end{itemize}

\item Famous places (名所、めいしょ) or products (名物、めいぶつ)

\begin{itemize}
\item many types of hot springs
\item amazing park
\item the "Hells of Beppu"
\item monkey park
\end{itemize}

\item What you can see, do, and/or eat there

\begin{itemize}
\item onsen food

\begin{itemize}
\item onsen egg
\end{itemize}

\item hot springs, of course, but also hot sand baths and steam baths
\item and also, some theme parks
\end{itemize}

\item Why you like it or recommend it

\begin{itemize}
\item hot springs are very cool
\item food is very good
\end{itemize}

\item Any other "fun facts"

\begin{itemize}
\item over 100 hot springs!
\item produces more hot spring water than any other resort
\end{itemize}
\end{itemize}

\section{Planning}
\label{sec:orgf1d4562}
to include:

potential form, \textasciitilde{}shi \textasciitilde{}shi form, \textasciitilde{}sou desu, \textasciitilde{}nara, giving and receiving
verbs, giving advice, volitional form, sentence modifiers or relative
clauses

\subsubsection{Eng Outline}
\label{sec:orgf6c3957}
\begin{enumerate}
\item All of japan looks wonderful, but next time I go, I want to go to
beppu.
\item Beppu is very beautiful, but its main attraction is its hot springs
\item It has some of the most beautiful hot springs in the world, and it
produces more hot spring water than any other resort!
\item Beppu has 8 hot spring sources, called the hells of beppu.
\item In beppu their are over 2000 hot springs, so their is a kind of hot
spring for everybody.
\item Their are bathing hot springs, mud hot springs, sand hot springs,
and viewing hot springs.
\item These hot springs are only for viewing, not bathing.
\item Their is also very famous hot spring food.
\item This food is cooked using the hot springs.
\item Here is an onsen egg, a very famous type of onsen food.
\item If you get tired of hot springs, you can check out the monkey park.
\item The monkey park has around 1500 wild japanese macaques.
\item Beppu also has wonderful firework shows, normally in late july.
\item In beppu, their is something for everyone.
\item Please come!
\end{enumerate}

\subsubsection{Japanese time}
\label{sec:org8eaf06f}
\begin{enumerate}
\item 日本はどこでもきれいですが、こんどは、別府にいきたいんです。
\item 別府はとてもきれいですが、一番有名なのは、温泉です。
\item 別府はせっかいじゅう一番きれいな温泉がたくさんあります。
\item 別府の地獄というゆうめいな温泉があります。
\item にせんいじょうの温泉があって、いろいろなしゅるいがあります。
\item ふつうの温泉がありますし、どろとすなの温泉もあります。
\item いくつかの温泉がみるのみで、ちかづかないでください。
\item 食べものもとてもゆうめいです。
\item とくべつな温泉蒸し料理があります。
\item これはゆうめいな温泉卵です。
\item とてもおいしいですよ。
\item 温泉にあきたら、高崎山自然動物園にいって、さるを見てください。
\item 千五百ぐらいのさるがいます。
\item また、別府のはなびたいかいもとてもゆうめいです。しちがつのすえにあります。
\item ぜひ別府にきてください。
\end{enumerate}

\url{https://docs.google.com/document/d/1NJTz4ov\_3QtUYiPY4YtGkTQ5EadnAj6\_HR32jy5GXAc/edit?ts=60b6cfd7}

\subsubsection{updated!}
\label{sec:orgf1062ec}
日本はどこでもきれいですが、今回(こんかい、this
time)は、別府というところにいきたいんです。
別府は日本の南のほうにあります。
別府はとてもきれいですが、一番有名なのは、温泉です。
別府にせかいじゅうの一番きれいな温泉がたくさんあります。
別府の地獄めぐりというゆうめいな温泉があります。
にせんいじょうの温泉があって、いろいろなしゅるいがあります。
ふつうの温泉があるし、どろとすなの温泉もあります。
いくつかの温泉は見てもいいですが、入って(はいって)はいけません。
あつすぎるんですから。 別府の食べものもとてもゆうめいです。
とくべつな温泉の蒸し料理があります。 これは日本人の大好きな温泉卵です。
温泉卵というのは温泉の水の中にゆでる卵です。 とてもおいしいですよ。
温泉にあきたら、高崎山自然動物園にいって、さるを見たらどうですか。
ここにさるが千五百匹(びき)もいます。
また、別府のはなびたいかいもとてもゆうめいです。
しちがつのおわりにあります。 みなさん、ぜひ別府にきてください。

にっぽん は どこ でも きれい です が、 こんかいは、 べっぷ ,という,
ところに, いきたい ん です。 べっぷ は, にっぽん の, みなみ の ほう に,
あります。 べっぷ は, とても,きれい です が、 いちばん ,ゆうめいな の
は、 おんせん です。 べっぷ に, せ かいじゅう の, いちばん, きれいな,
おんせん が, たくさん, あります。 べっぷ の, じごくめ
ぐり,という,ゆうめいな ,おんせん が ,あります。 にせん,い じ ょう の,
おんせん が, あって、 いろいろ な,しゅるいが,あります。 ふつう のm
おんせん もm ある しm ど ろ とm す な のm おんせん もm あります。

いくつ か のm おんせん はm みて もm いい です がm はいって
はmいけません。 あつすぎる ん です から。 べっぷ のm たべもの もm とても
mゆう めい です。 とくべつ なm おんせん の mむし りょうり がm あります。

これ はm にっぽんじん のm だいすきな mおんせん たまごです。

おんせん たまご mという の はm おんせん の mみず のm なかにm ゆでる
mたまご です。 とても おいしい です よ。

おんせん にm あきたらm たかさきやま-しぜん-どうぶつえんにm いってm さる
をm みたら mどう です か。

ここ にm さる がm せん ご ひゃく-ひきもm います。

またm べっぷ のm はな びmたいかいもmとてもmゆうめいです。

し ち が つ のm お わり にm あります。

みなさん、 ぜひm べっぷ にm きて mください。

\subsubsection{Vocab List}
\label{sec:orge07604a}
\begin{itemize}
\item こんど: next time 1

\item せかいじゅう: around the world 3

\item ゆうめい: famous 4

\item じごく: hell 4

\item ちかづく: to get close 6

\item しゅるい: types 5

\item どろ: mud 7

\item すな: sand 7

\item むしりょうり: steamed food 8

\item とくべつ: special 9

\item あきる: get tired of 13

\item すえ: end (of the month) 13

\item はなびたいかい: firework display 15

\item ぜひ: by all means 17

\item こんど: next time

\item せかいじゅう: around the world

\item ゆうめい: famous

\item じごく: hell

\item ちかづく: to get close

\item しゅるい: types

\item どろ: mud

\item すな: sand

\item むしりょうり: steamed food

\item とくべつ: special

\item あきる: get tired of

\item すえ: end (of the month)

\item はなびたいかい: firework display

\item ぜひ: by all means
\end{itemize}

\subsubsection{hiragana!}
\label{sec:orga735f36}
1 . にっぽん は どこ でも きれい です が、 こんど は、 べっぷ に
いきたい ん です。 2 . べっぷ は とても きれい です が、 いちばん
ゆうめいな の は、 おんせん です。 3 . べっぷ は せっかいじゅう いちばん
きれいな おんせん が たくさん あります。 4 . べっぷ の じごく という
ゆう めい な おんせん が あります。 5 . に せん いじ ょう の おんせん が
あって、 いろいろ なし ゅるいがあります。 6 . ふつう の おんせん が
あります し、 ど ろ と す な の おんせん も あります。 7 . いくつ か の
おんせん が みる のみ で、 ちかづかないで ください。 8 . たべもの も
とても ゆう めい です。 9 . とくべつ な おんせん むし りょうり が
あります。 10 . これ は ゆう めい な おんせん たまご です。 11 . とても
おいしい です よ。 12 . おんせん に あきたら、 たかさきやま しぜん
どうぶつえん に いって、 さる を みて ください。 13 . せん ご ひゃく
ぐらい の さる が います。 14 . また、 べっぷ の はな
びたいかいもとてもゆうめいです。 し ち が つ の すえ に あります。 15 .
ぜひ べっぷ に きて ください。
\end{document}
