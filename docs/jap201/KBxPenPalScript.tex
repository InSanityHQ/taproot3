% Created 2021-09-11 Sat 08:17
% Intended LaTeX compiler: xelatex
\documentclass[letterpaper]{article}
\usepackage{graphicx}
\usepackage{grffile}
\usepackage{longtable}
\usepackage{wrapfig}
\usepackage{rotating}
\usepackage[normalem]{ulem}
\usepackage{amsmath}
\usepackage{textcomp}
\usepackage{amssymb}
\usepackage{capt-of}
\usepackage{hyperref}
\usepackage[margin=1in]{geometry}
\usepackage{fontspec}
\usepackage{indentfirst}
\setmainfont[ItalicFont = LiberationSans-Italic, BoldFont = LiberationSans-Bold, BoldItalicFont = LiberationSans-BoldItalic]{LiberationSans}
\newfontfamily\NHLight[ItalicFont = LiberationSansNarrow-Italic, BoldFont       = LiberationSansNarrow-Bold, BoldItalicFont = LiberationSansNarrow-BoldItalic]{LiberationSansNarrow}
\newcommand\textrmlf[1]{{\NHLight#1}}
\newcommand\textitlf[1]{{\NHLight\itshape#1}}
\let\textbflf\textrm
\newcommand\textulf[1]{{\NHLight\bfseries#1}}
\newcommand\textuitlf[1]{{\NHLight\bfseries\itshape#1}}
\usepackage{fancyhdr}
\pagestyle{fancy}
\usepackage{titlesec}
\usepackage{titling}
\makeatletter
\lhead{\textbf{\@title}}
\makeatother
\rhead{\textrmlf{Compiled} \today}
\lfoot{\theauthor\ \textbullet \ \textbf{2021-2022}}
\cfoot{}
\rfoot{\textrmlf{Page} \thepage}
\titleformat{\section} {\Large} {\textrmlf{\thesection} {|}} {0.3em} {\textbf}
\titleformat{\subsection} {\large} {\textrmlf{\thesubsection} {|}} {0.2em} {\textbf}
\titleformat{\subsubsection} {\large} {\textrmlf{\thesubsubsection} {|}} {0.1em} {\textbf}
\setlength{\parskip}{0.45em}
\renewcommand\maketitle{}
\author{Go Away}
\date{\today}
\title{Don't}
\hypersetup{
 pdfauthor={Go Away},
 pdftitle={Don't},
 pdfkeywords={},
 pdfsubject={},
 pdfcreator={Emacs 27.2 (Org mode 9.4.4)}, 
 pdflang={English}}
\begin{document}

\maketitle
\#ref \#ret

\noindent\rule{\textwidth}{0.5pt}

\begin{itemize}
\item Your name

\item age/grade

\item What's your favorite food/precious item/hobby/favorite subject/sport

\item What you are looking for your future pen pal e.g. types of friends
\end{itemize}

For example,
「やさしくて、おもしろいともだちをさがしています。」(さがす=look for)

\begin{itemize}
\item Share a little bit about you
\end{itemize}

For example,
「私は、書いたり、詩(し=poem)をつくったりするのが好きなタイプです。」

\begin{itemize}
\item You'll end yourself introduction in Japanese with following options:
\end{itemize}

\#2: I'll look forward to making friends with you.
=ともだちになれることをたのしみにしています。

\#3: I'll look forward to doing pen pal with you.
=ペンパルできることをたのしみにしています。

hajimemashite!

My name is Huxley!

I am 15, and I am in my second year of high school

My favorite hobby is coding

I'm looking for an interesting and kind friend.

I'm the type who likes to work on projects and talk with friends.

ペンパルできることをたのしみにしています。

\noindent\rule{\textwidth}{0.5pt}

はじめまして。

ぼくのなまえはハクスリーです。

じゅうごさいです。こうこうにねんせいです。

しゅみはコーディングです。

おもしろくて、やさしいともだちをさがしています。

ぼくは,
プロジェクトにとりこんだり、ともだちとはなしをしたりするのがすきなタイプです。

ペンパルできることをたのしみにしています。
\end{document}
