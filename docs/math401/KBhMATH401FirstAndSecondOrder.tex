% Created 2021-09-11 Sat 08:16
% Intended LaTeX compiler: xelatex
\documentclass[letterpaper]{article}
\usepackage{graphicx}
\usepackage{grffile}
\usepackage{longtable}
\usepackage{wrapfig}
\usepackage{rotating}
\usepackage[normalem]{ulem}
\usepackage{amsmath}
\usepackage{textcomp}
\usepackage{amssymb}
\usepackage{capt-of}
\usepackage{hyperref}
\usepackage[margin=1in]{geometry}
\usepackage{fontspec}
\usepackage{indentfirst}
\setmainfont[ItalicFont = LiberationSans-Italic, BoldFont = LiberationSans-Bold, BoldItalicFont = LiberationSans-BoldItalic]{LiberationSans}
\newfontfamily\NHLight[ItalicFont = LiberationSansNarrow-Italic, BoldFont       = LiberationSansNarrow-Bold, BoldItalicFont = LiberationSansNarrow-BoldItalic]{LiberationSansNarrow}
\newcommand\textrmlf[1]{{\NHLight#1}}
\newcommand\textitlf[1]{{\NHLight\itshape#1}}
\let\textbflf\textrm
\newcommand\textulf[1]{{\NHLight\bfseries#1}}
\newcommand\textuitlf[1]{{\NHLight\bfseries\itshape#1}}
\usepackage{fancyhdr}
\pagestyle{fancy}
\usepackage{titlesec}
\usepackage{titling}
\makeatletter
\lhead{\textbf{\@title}}
\makeatother
\rhead{\textrmlf{Compiled} \today}
\lfoot{\theauthor\ \textbullet \ \textbf{2021-2022}}
\cfoot{}
\rfoot{\textrmlf{Page} \thepage}
\titleformat{\section} {\Large} {\textrmlf{\thesection} {|}} {0.3em} {\textbf}
\titleformat{\subsection} {\large} {\textrmlf{\thesubsection} {|}} {0.2em} {\textbf}
\titleformat{\subsubsection} {\large} {\textrmlf{\thesubsubsection} {|}} {0.1em} {\textbf}
\setlength{\parskip}{0.45em}
\renewcommand\maketitle{}
\author{Houjun Liu}
\date{\today}
\title{First and Second Order Differences}
\hypersetup{
 pdfauthor={Houjun Liu},
 pdftitle={First and Second Order Differences},
 pdfkeywords={},
 pdfsubject={},
 pdfcreator={Emacs 27.2 (Org mode 9.4.4)}, 
 pdflang={English}}
\begin{document}

\maketitle


\section{First order difference}
\label{sec:org8fb74dc}
\(\frac{\Delta y}{\Delta x}\) (Average slope of the function between two
points)

When the first order difference > 0 --- as x increases, f(x) \emph{increases}

When first order difference > 0 --- as x increases, f(x) \emph{decreases}

\section{Second order differences}
\label{sec:org410f7a0}
\(\frac{\frac{\Delta y}{\Delta x}}{\Delta x}\) (The "acceleration" of
the function)

\emph{This is related to the concavity of the graph!!}

When the second order difference > 0 --- the graph should be concave up

\begin{quote}
"As soon as the yoyo is rolled, it is accelerating upwards. First, the
acceleration works to slow the downwards velocity. Then, it actually
flips the velocity up."
\end{quote}

When first order difference > 0 --- as x increases, f(x) \emph{decreases}

\begin{quote}
"As soon as the ball is tossed, it is accelerating downwards. First,
the acceleration works to slow the upwards velocity. Then, it actually
flips the velocity down."
\end{quote}

\section{Linear Functions}
\label{sec:org047168c}
\begin{itemize}
\item No change in slope
\item And hence, 0 second order difference
\item and so there is no concavity
\end{itemize}

\section{Log Functions}
\label{sec:org12f0d6b}
They are inverse to expotential functions.

Recall that, for the base graph \(y=log(x)\):

\begin{enumerate}
\item Domain should be (0, \(\infty\))
\item Range is (-\(\infty\), \(\infty\))
\item As \(x \to 0^+\), \(y \to -\infty\)
\item x-int at \(y=0\).
\end{enumerate}

These, of course, are flipped for its sister, inverse function
\(y=10^x\)

\begin{enumerate}
\item Range should be (0, \(\infty\))
\item Domain is (-\(\infty\), \(\infty\))
\item As \(x\to-\infty\), \(y\to0\)
\item y-int at \(x=0\).
\end{enumerate}
\end{document}
