% Created 2021-09-11 Sat 09:35
% Intended LaTeX compiler: xelatex
\documentclass[letterpaper]{article}
\usepackage{graphicx}
\usepackage{grffile}
\usepackage{longtable}
\usepackage{wrapfig}
\usepackage{rotating}
\usepackage[normalem]{ulem}
\usepackage{amsmath}
\usepackage{textcomp}
\usepackage{amssymb}
\usepackage{capt-of}
\usepackage{hyperref}
\usepackage[margin=1in]{geometry}
\usepackage{fontspec}
\usepackage{indentfirst}
\setmainfont[ItalicFont = LiberationSans-Italic, BoldFont = LiberationSans-Bold, BoldItalicFont = LiberationSans-BoldItalic]{LiberationSans}
\newfontfamily\NHLight[ItalicFont = LiberationSansNarrow-Italic, BoldFont       = LiberationSansNarrow-Bold, BoldItalicFont = LiberationSansNarrow-BoldItalic]{LiberationSansNarrow}
\newcommand\textrmlf[1]{{\NHLight#1}}
\newcommand\textitlf[1]{{\NHLight\itshape#1}}
\let\textbflf\textrm
\newcommand\textulf[1]{{\NHLight\bfseries#1}}
\newcommand\textuitlf[1]{{\NHLight\bfseries\itshape#1}}
\usepackage{fancyhdr}
\pagestyle{fancy}
\usepackage{titlesec}
\usepackage{titling}
\makeatletter
\lhead{\textbf{\@title}}
\makeatother
\rhead{\textrmlf{Compiled} \today}
\lfoot{\theauthor\ \textbullet \ \textbf{2021-2022}}
\cfoot{}
\rfoot{\textrmlf{Page} \thepage}
\titleformat{\section} {\Large} {\textrmlf{\thesection} {|}} {0.3em} {\textbf}
\titleformat{\subsection} {\large} {\textrmlf{\thesubsection} {|}} {0.2em} {\textbf}
\titleformat{\subsubsection} {\large} {\textrmlf{\thesubsubsection} {|}} {0.1em} {\textbf}
\setlength{\parskip}{0.45em}
\renewcommand\maketitle{}
\author{Houjun Liu}
\date{\today}
\title{Solving Limits with Elimination}
\hypersetup{
 pdfauthor={Houjun Liu},
 pdftitle={Solving Limits with Elimination},
 pdfkeywords={},
 pdfsubject={},
 pdfcreator={Emacs 27.2 (Org mode 9.4.4)}, 
 pdflang={English}}
\begin{document}

\maketitle


\section{Solving Limits with Elimination}
\label{sec:orga9e1b1c}
With solving limits via elimination, we are tipically analyzing a
rational function that needs factoring of a term out of the polynomials
on the top and/or the bottom to get out of the indeterminate form
\((\frac{0}{0})\).

\begin{itemize}
\item Try factoring both the top and bottom

\begin{itemize}
\item \((x\pm1)\)
\item \((x\pm2)\)
\end{itemize}

\item Rationalize all of the square roots
\end{itemize}

Tip for picking factors

\noindent\rule{\textwidth}{0.5pt}

\textbf{Tip!} If you plug in a value to an expression, and out pops 0, that
value is a \textbf{zero} of the expression. It is helpful like this

Factor: \((x^6-1)\)

As you could see, plugging \(x=1\) yields \(0\), meaning that \((x-1)\)
is a \textbf{zero} of \((x^6-1)\), and hence would be able to be factored out.

To factor it out, either do synthetic division or long division.

\noindent\rule{\textwidth}{0.5pt}

Let's do a problem solve for \(\lim_{x\to 2} \frac{(x^2-4)}{(x-2)}\)

\begin{enumerate}
\item First, notice the fact this function will have a hole at \(x=2\).
This is especially important because after we simplify we will loose
this hole.
\item Ok, now let's simply.
\(\frac{(x^2-4)}{(x-2)} = \frac{(x+2)(\cancel{(x-2)})}{(\cancel{x-2})} = (x+2)\)
\item Great! So, we know that this function behaves linearly with simply a
hole at 2.
\item Doing the double-sided limits\ldots{}

\begin{itemize}
\item Evaluating \(\lim_{x\to2^+}\), the value will be \(4\) because
\(2+2=4\).
\item Evaluating\\
\end{itemize}
\end{enumerate}

Here's another one! \(\lim_{x\to0} \frac{\sqrt{x+4}-2}{x}\)

\begin{enumerate}
\item First, notice that if we are going to solve this problem, we have to
divide the top thing (\(\sqrt{x+4}-2\)) by \(x\), somehow
\item The only thing we could do here is rationalize the top by multiplying
the whole faction by a fancy one
\(\frac{\sqrt{x+4}+2}{\sqrt{x+4}+2}\).
\item This results in \(\frac{x+4-4}{x\times(\sqrt{x+4}+2)}\), which
simplifies to \(\frac{\cancel{x}}{\cancel{x}\times(\sqrt{x+4}+2)}\)
\item Plugging in \(x=0\), you get \(\frac{1}{4}\).
\end{enumerate}

\textbf{If there is no factors, you got yourself a vertical asymtote. Refer to
\#missing \#disorganized for solution!}
\end{document}
