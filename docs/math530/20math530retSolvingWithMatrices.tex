% Created 2021-09-11 Sat 08:18
% Intended LaTeX compiler: xelatex
\documentclass[letterpaper]{article}
\usepackage{graphicx}
\usepackage{grffile}
\usepackage{longtable}
\usepackage{wrapfig}
\usepackage{rotating}
\usepackage[normalem]{ulem}
\usepackage{amsmath}
\usepackage{textcomp}
\usepackage{amssymb}
\usepackage{capt-of}
\usepackage{hyperref}
\usepackage[margin=1in]{geometry}
\usepackage{fontspec}
\usepackage{indentfirst}
\setmainfont[ItalicFont = LiberationSans-Italic, BoldFont = LiberationSans-Bold, BoldItalicFont = LiberationSans-BoldItalic]{LiberationSans}
\newfontfamily\NHLight[ItalicFont = LiberationSansNarrow-Italic, BoldFont       = LiberationSansNarrow-Bold, BoldItalicFont = LiberationSansNarrow-BoldItalic]{LiberationSansNarrow}
\newcommand\textrmlf[1]{{\NHLight#1}}
\newcommand\textitlf[1]{{\NHLight\itshape#1}}
\let\textbflf\textrm
\newcommand\textulf[1]{{\NHLight\bfseries#1}}
\newcommand\textuitlf[1]{{\NHLight\bfseries\itshape#1}}
\usepackage{fancyhdr}
\pagestyle{fancy}
\usepackage{titlesec}
\usepackage{titling}
\makeatletter
\lhead{\textbf{\@title}}
\makeatother
\rhead{\textrmlf{Compiled} \today}
\lfoot{\theauthor\ \textbullet \ \textbf{2021-2022}}
\cfoot{}
\rfoot{\textrmlf{Page} \thepage}
\titleformat{\section} {\Large} {\textrmlf{\thesection} {|}} {0.3em} {\textbf}
\titleformat{\subsection} {\large} {\textrmlf{\thesubsection} {|}} {0.2em} {\textbf}
\titleformat{\subsubsection} {\large} {\textrmlf{\thesubsubsection} {|}} {0.1em} {\textbf}
\setlength{\parskip}{0.45em}
\renewcommand\maketitle{}
\author{Albert Huang}
\date{\today}
\title{Solving with matrices}
\hypersetup{
 pdfauthor={Albert Huang},
 pdftitle={Solving with matrices},
 pdfkeywords={},
 pdfsubject={},
 pdfcreator={Emacs 27.2 (Org mode 9.4.4)}, 
 pdflang={English}}
\begin{document}

\maketitle
\#ret

\begin{enumerate}
\item \begin{quote}
Suppose A = \(\begin{pmatrix} 1 & 3\\ 2 & -1 \end{pmatrix}\) and B
= \(\begin{pmatrix} 0 & -1\\ 2 & 1 \end{pmatrix}\). Multiply \(AB\)
and \(BA\). What do you notice???
\end{quote}
\end{enumerate}

Nothing. Matrix multiplication is not commutative.

\begin{enumerate}
\item \begin{quote}
Use matrices to solve the system:
\(\begin{aligned}2x-y=3\\x+3y=5\end{aligned}\)
\end{quote}
\end{enumerate}

$\backslash$[
\begin{aligned}
&&\left[\begin{matrix}2&-1\\1&3\end{matrix}\right]
\left[\begin{matrix}x\\y\end{matrix}\right]&=
&\left[\begin{matrix}3\\5\end{matrix}\right]
\\
&\left[\begin{matrix}3&0\\0&1\end{matrix}\right]
&\left[\begin{matrix}2&-1\\1&3\end{matrix}\right]
\left[\begin{matrix}x\\y\end{matrix}\right]&=
&\left[\begin{matrix}3&0\\0&1\end{matrix}\right]
\left[\begin{matrix}3\\5\end{matrix}\right]
\\
\left[\begin{matrix}1&1\\0&1\end{matrix}\right]
&\left[\begin{matrix}3&0\\0&1\end{matrix}\right]
&\left[\begin{matrix}2&-1\\1&3\end{matrix}\right]
\left[\begin{matrix}x\\y\end{matrix}\right]&=
\left[\begin{matrix}1&1\\0&1\end{matrix}\right]
&\left[\begin{matrix}3&0\\0&1\end{matrix}\right]
\left[\begin{matrix}3\\5\end{matrix}\right]
\\
\left[\begin{matrix}1&1\\0&1\end{matrix}\right]
&&\left[\begin{matrix}6&-3\\1&3\end{matrix}\right]
\left[\begin{matrix}x\\y\end{matrix}\right]&=
\left[\begin{matrix}1&1\\0&1\end{matrix}\right]
&\left[\begin{matrix}9\\5\end{matrix}\right]
\\
&&\left[\begin{matrix}7&0\\1&3\end{matrix}\right]
\left[\begin{matrix}x\\y\end{matrix}\right]&=
&\left[\begin{matrix}14\\5\end{matrix}\right]
\\
&&\left[\begin{matrix}7x\\x+3y\end{matrix}\right]&=
&\left[\begin{matrix}14\\5\end{matrix}\right]
\\
&\left[\begin{matrix}\frac{1}{7}&0\\0&1\end{matrix}\right]
&\left[\begin{matrix}7x\\x+3y\end{matrix}\right]&=
&\left[\begin{matrix}\frac{1}{7}&0\\0&1\end{matrix}\right]
\left[\begin{matrix}14\\5\end{matrix}\right]
\\
&&\left[\begin{matrix}x\\x+3y\end{matrix}\right]&=
&\left[\begin{matrix}2\\5\end{matrix}\right]
\end{aligned}
$\backslash$] $\backslash$[
\begin{aligned}
x = 2
\\
x + 3 y = 5
\end{aligned}
$\backslash$] I'm not sure how to solve the rest of it with matrices, so I'll just
do it normally: $\backslash$[
\begin{aligned}
x &= 2\\
x + 3y &= 5\\
2 + 3y &= 5\\
3y &= 3\\
y &= 1\\
\end{aligned}
$\backslash$] 3. > Do 2x2 matrices form a group under > a. addition? > b.
multiplication?

See \href{KBe2020math530refGroups.org}{KBe2020math530refGroups} I'll
assume that our matrices have real numbers in them.

\begin{center}
\begin{tabular}{llllll}
Operation  Property & Closed & Identity & Inverse & Associative? & Final\\
\hline
Addition & Yes & \(e=\left[\begin{matrix}0&0\\0&0\end{matrix}\right]\) & \(\left[\begin{matrix}a&b\\c&d\end{matrix}\right] + \left[\begin{matrix}-a&-b\\-c&-d\end{matrix}\right]=e\) & "Inherits from addition" & Yes\\
Multiplication & Yes & \(e=\left[\begin{matrix}1&0\\0&1\end{matrix}\right]\) & Maybe? & Yes, see below & Undecided\\
\end{tabular}
\end{center}

Associativity of 2x2 matrices under multiplication: $\backslash$[
\begin{aligned}
\left(\left[\begin{matrix}a&b\\c&d\end{matrix}\right]
\left[\begin{matrix}e&f\\g&h\end{matrix}\right]\right)
\left[\begin{matrix}i&j\\k&l\end{matrix}\right]
= 
\left[\begin{matrix}ae+bg&af+bh \\ ce+dg&cf+dh \end{matrix}\right]
\left[\begin{matrix}i&j\\k&l\end{matrix}\right]
\\=
\left[\begin{matrix}aei+bgi+afk+bhk&aej+bgj+afl+bhl\\cei+dgi+cfk+dhk&cej+dgj+cfl+dhl\end{matrix}\right]
\\=
\left[\begin{matrix}a&b\\c&d\end{matrix}\right]
\left[\begin{matrix}ei+fk&ej+fl\\gi+hk&gj+hl\end{matrix}\right]
=
\left[\begin{matrix}a&b\\c&d\end{matrix}\right]
\left(\left[\begin{matrix}e&f\\g&h\end{matrix}\right]
\left[\begin{matrix}i&j\\k&l\end{matrix}\right]\right)
\end{aligned}
$\backslash$]

I can't figure out if 2x2 matrices have multiplicative inverses\ldots{} I
tried to work it out using algebra but kept proving identities. Not sure
what the right way to go about this is\ldots{}

I spent far too long on this assignment (1.6h), so I probably won't
spend as much time LaTeXing my homework in the future.
\end{document}
