% Created 2021-09-11 Sat 09:36
% Intended LaTeX compiler: xelatex
\documentclass[letterpaper]{article}
\usepackage{graphicx}
\usepackage{grffile}
\usepackage{longtable}
\usepackage{wrapfig}
\usepackage{rotating}
\usepackage[normalem]{ulem}
\usepackage{amsmath}
\usepackage{textcomp}
\usepackage{amssymb}
\usepackage{capt-of}
\usepackage{hyperref}
\usepackage[margin=1in]{geometry}
\usepackage{fontspec}
\usepackage{indentfirst}
\setmainfont[ItalicFont = LiberationSans-Italic, BoldFont = LiberationSans-Bold, BoldItalicFont = LiberationSans-BoldItalic]{LiberationSans}
\newfontfamily\NHLight[ItalicFont = LiberationSansNarrow-Italic, BoldFont       = LiberationSansNarrow-Bold, BoldItalicFont = LiberationSansNarrow-BoldItalic]{LiberationSansNarrow}
\newcommand\textrmlf[1]{{\NHLight#1}}
\newcommand\textitlf[1]{{\NHLight\itshape#1}}
\let\textbflf\textrm
\newcommand\textulf[1]{{\NHLight\bfseries#1}}
\newcommand\textuitlf[1]{{\NHLight\bfseries\itshape#1}}
\usepackage{fancyhdr}
\pagestyle{fancy}
\usepackage{titlesec}
\usepackage{titling}
\makeatletter
\lhead{\textbf{\@title}}
\makeatother
\rhead{\textrmlf{Compiled} \today}
\lfoot{\theauthor\ \textbullet \ \textbf{2021-2022}}
\cfoot{}
\rfoot{\textrmlf{Page} \thepage}
\titleformat{\section} {\Large} {\textrmlf{\thesection} {|}} {0.3em} {\textbf}
\titleformat{\subsection} {\large} {\textrmlf{\thesubsection} {|}} {0.2em} {\textbf}
\titleformat{\subsubsection} {\large} {\textrmlf{\thesubsubsection} {|}} {0.1em} {\textbf}
\setlength{\parskip}{0.45em}
\renewcommand\maketitle{}
\author{Exr0n}
\date{\today}
\title{LinAlg Flow}
\hypersetup{
 pdfauthor={Exr0n},
 pdftitle={LinAlg Flow},
 pdfkeywords={},
 pdfsubject={},
 pdfcreator={Emacs 27.2 (Org mode 9.4.4)}, 
 pdflang={English}}
\begin{document}

\maketitle


\section{Looking forward}
\label{sec:org17275b0}
\begin{itemize}
\item Will use canvas's discussion board in the future.
\item Assume matrices have real numbers
\end{itemize}

\section{Solving with Matrices}
\label{sec:org0ffd731}
\begin{itemize}
\item Elementary matrices (like
\(\left[\begin{matrix}1 &-2 \\ 0 &1\end{matrix}\right]\))
\item Steps walk through

\begin{itemize}
\item Start with \(\left[\begin{matrix}a&b\\d&e\end{matrix}\right]\) (the
coefficient matrix).
\item You want to get somewhere such that
\(\left[\begin{matrix}1x\\0y\end{matrix}\right] = \left[\begin{matrix}c\\f\end{matrix}\right]\)
\item And ultimately
\(\left[\begin{matrix}1&0\\0&1\end{matrix}\right]\left[\begin{matrix}x\\y\end{matrix}\right]=\left[\begin{matrix}{ans}_x\\{ans}_y\end{matrix}\right]\)
\item \href{srcD3SolveWithMatricies.png.org}{srcD3SolveWithMatricies.png}
\end{itemize}
\end{itemize}

\section{Matrix Inverse Formula}
\label{sec:org7106e39}
\begin{itemize}
\item I should technically know this already.
\end{itemize}

\subsection{Derivation}
\label{sec:orgb10fb50}
$\backslash$[
\left[\begin{matrix}a\&b$\backslash$\c\&d\end{matrix}\right]
\left[\begin{matrix}w\&x$\backslash$\y\&z\end{matrix}\right]
\left[\begin{matrix}aw+by\&ax+bz $\backslash$\ cw+dy\&cx+dz\end{matrix}\right]
$\backslash$\\(\therefore\)\\
\begin{split}
aw + by = 1\\
cw + dy = 0\\
ax + bz = 0\\
cx + dz = 1\\
\end{split}
$\backslash$]

\begin{itemize}
\item There's two 2 variable equations.
\href{srcIdentityMatrixFormula.png.org}{srcIdentityMatrixFormula.png}
\end{itemize}

\section{Matrix Operations}
\label{sec:org2336f82}
\begin{itemize}
\item If we have a set of objects that are almost groups in under both
addition and multiplication, then it's called a field

\begin{itemize}
\item 2x2 Matrices aren't quite close enough on the multiplication (too
many no inverses) but we can work with other sizes. \#\#\# Vector
Products
\end{itemize}

\item Matrices of dimension \(n\)x\(1\)
\item What multiplications on vectors are "nice"?

\begin{itemize}
\item Transpose the first (left) one and multiply normally, then squish
2x2 into 2x1
\item Cross product
\item Element wise (is closed)
\item Take every element and multiply them all together, and then
duplicate?

\begin{itemize}
\item No, no identity
\end{itemize}

\item Any one to one mapping?

\begin{itemize}
\item No, identity doesn't work if it's on the left.
\end{itemize}
\end{itemize}
\end{itemize}

\noindent\rule{\textwidth}{0.5pt}
\end{document}
