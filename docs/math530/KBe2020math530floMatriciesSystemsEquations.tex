% Created 2021-09-11 Sat 08:18
% Intended LaTeX compiler: xelatex
\documentclass[letterpaper]{article}
\usepackage{graphicx}
\usepackage{grffile}
\usepackage{longtable}
\usepackage{wrapfig}
\usepackage{rotating}
\usepackage[normalem]{ulem}
\usepackage{amsmath}
\usepackage{textcomp}
\usepackage{amssymb}
\usepackage{capt-of}
\usepackage{hyperref}
\usepackage[margin=1in]{geometry}
\usepackage{fontspec}
\usepackage{indentfirst}
\setmainfont[ItalicFont = LiberationSans-Italic, BoldFont = LiberationSans-Bold, BoldItalicFont = LiberationSans-BoldItalic]{LiberationSans}
\newfontfamily\NHLight[ItalicFont = LiberationSansNarrow-Italic, BoldFont       = LiberationSansNarrow-Bold, BoldItalicFont = LiberationSansNarrow-BoldItalic]{LiberationSansNarrow}
\newcommand\textrmlf[1]{{\NHLight#1}}
\newcommand\textitlf[1]{{\NHLight\itshape#1}}
\let\textbflf\textrm
\newcommand\textulf[1]{{\NHLight\bfseries#1}}
\newcommand\textuitlf[1]{{\NHLight\bfseries\itshape#1}}
\usepackage{fancyhdr}
\pagestyle{fancy}
\usepackage{titlesec}
\usepackage{titling}
\makeatletter
\lhead{\textbf{\@title}}
\makeatother
\rhead{\textrmlf{Compiled} \today}
\lfoot{\theauthor\ \textbullet \ \textbf{2021-2022}}
\cfoot{}
\rfoot{\textrmlf{Page} \thepage}
\titleformat{\section} {\Large} {\textrmlf{\thesection} {|}} {0.3em} {\textbf}
\titleformat{\subsection} {\large} {\textrmlf{\thesubsection} {|}} {0.2em} {\textbf}
\titleformat{\subsubsection} {\large} {\textrmlf{\thesubsubsection} {|}} {0.1em} {\textbf}
\setlength{\parskip}{0.45em}
\renewcommand\maketitle{}
\author{Exr0n}
\date{\today}
\title{Matrices as systems of equations}
\hypersetup{
 pdfauthor={Exr0n},
 pdftitle={Matrices as systems of equations},
 pdfkeywords={},
 pdfsubject={},
 pdfcreator={Emacs 27.2 (Org mode 9.4.4)}, 
 pdflang={English}}
\begin{document}

\maketitle
Linear combination aka elimination method

\begin{align}
2x &+ 3y &= 5 \\
 x &+  y &= 1
\end{align}

is equivalent to

\[
\left[\begin{matrix}
2 &3 \\
1 &1
\end{matrix}\right]
\left[\begin{matrix}x\\y\end{matrix}\right]
=
\left[\begin{matrix}5\\1\end{matrix}\right]
\].

We want to multiply the bottom equation by \(-2\) when solving with the
elimination method normally, so we might expect to multiply by the
identity matrix but with the "bottom row selector" modified:

\(\left[\begin{matrix}1&0\\0&-2\end{matrix}\right]\).

\[
\left[\begin{matrix} 1 &0 \\ 0 &-2\end{matrix}\right]
\left[\begin{matrix} 2 &3 \\ 1 &1 \end{matrix}\right]
\left[\begin{matrix} x \\ y \end{matrix}\right]
\]

=

\[
\left[\begin{matrix} 1 &0 \\ 0 &-2\end{matrix}\right]
\left[\begin{matrix} 5 \\ 1\end{matrix}\right]
\]

And then, to add the bottom to the top we can use
\(\left[\begin{matrix}1&1\\0&1\end{matrix}\right]\).

\href{KBe2020math530srcMatriciesAsEquationsIntro.png.org}{KBe2020math530srcMatriciesAsEquationsIntro.png}
\href{KBe2020math530floMatrixMultiplyToSolve.png.org}{KBe2020math530floMatrixMultiplyToSolve.png}

\noindent\rule{\textwidth}{0.5pt}
\end{document}
