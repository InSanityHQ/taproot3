% Created 2021-09-11 Sat 08:18
% Intended LaTeX compiler: xelatex
\documentclass[letterpaper]{article}
\usepackage{graphicx}
\usepackage{grffile}
\usepackage{longtable}
\usepackage{wrapfig}
\usepackage{rotating}
\usepackage[normalem]{ulem}
\usepackage{amsmath}
\usepackage{textcomp}
\usepackage{amssymb}
\usepackage{capt-of}
\usepackage{hyperref}
\usepackage[margin=1in]{geometry}
\usepackage{fontspec}
\usepackage{indentfirst}
\setmainfont[ItalicFont = LiberationSans-Italic, BoldFont = LiberationSans-Bold, BoldItalicFont = LiberationSans-BoldItalic]{LiberationSans}
\newfontfamily\NHLight[ItalicFont = LiberationSansNarrow-Italic, BoldFont       = LiberationSansNarrow-Bold, BoldItalicFont = LiberationSansNarrow-BoldItalic]{LiberationSansNarrow}
\newcommand\textrmlf[1]{{\NHLight#1}}
\newcommand\textitlf[1]{{\NHLight\itshape#1}}
\let\textbflf\textrm
\newcommand\textulf[1]{{\NHLight\bfseries#1}}
\newcommand\textuitlf[1]{{\NHLight\bfseries\itshape#1}}
\usepackage{fancyhdr}
\pagestyle{fancy}
\usepackage{titlesec}
\usepackage{titling}
\makeatletter
\lhead{\textbf{\@title}}
\makeatother
\rhead{\textrmlf{Compiled} \today}
\lfoot{\theauthor\ \textbullet \ \textbf{2021-2022}}
\cfoot{}
\rfoot{\textrmlf{Page} \thepage}
\titleformat{\section} {\Large} {\textrmlf{\thesection} {|}} {0.3em} {\textbf}
\titleformat{\subsection} {\large} {\textrmlf{\thesubsection} {|}} {0.2em} {\textbf}
\titleformat{\subsubsection} {\large} {\textrmlf{\thesubsubsection} {|}} {0.1em} {\textbf}
\setlength{\parskip}{0.45em}
\renewcommand\maketitle{}
\author{Exr0n}
\date{\today}
\title{Numbers}
\hypersetup{
 pdfauthor={Exr0n},
 pdftitle={Numbers},
 pdfkeywords={},
 pdfsubject={},
 pdfcreator={Emacs 27.2 (Org mode 9.4.4)}, 
 pdflang={English}}
\begin{document}

\maketitle
\begin{itemize}
\item Discussion Results: what is a number?

\begin{itemize}
\item Something about group theory

\begin{itemize}
\item This is more like a way of telling us how to use numbers, not
really a good definition.
\item Set up bounds to define things
\end{itemize}

\item Different classes (natural, real, imaginary)
\item Where do you draw the boundaries between objects?
\item A way to quantify the nature of living and reality
\end{itemize}

\item Number Systems

\begin{itemize}
\item We want them to be desirable and group-like
\item Types

\begin{itemize}
\item \textbf{Natural Numbers}

\begin{itemize}
\item Integers greater than zero
\end{itemize}

\item Whole Numbers

\begin{itemize}
\item Natural Numbers + 0
\item 0 is the hole.
\end{itemize}

\item Integers

\begin{itemize}
\item \{ \ldots{}, -2, -1, 0, 1, 2, \ldots{} \}
\item Good for algebra, we'll see later
\end{itemize}

\item Rationals

\begin{itemize}
\item Like \(\frac{1}{2}\).
\item A ratio/fraction/quotient of integers
\end{itemize}

\item \textbf{Real}

\begin{itemize}
\item Like \(\pi\)
\item A number on the number line

\begin{itemize}
\item A number that can be a distance to something.
\item A good enough definition that isn't "real analysis"
\end{itemize}
\end{itemize}

\item \textbf{Complex Numbers}

\begin{itemize}
\item Like \(5i\)
\item There will be many complex numbers

\begin{itemize}
\item Matrices with complex numbers can be different from real
numbers
\end{itemize}

\item Complex plane
\end{itemize}

\item Hamaltonian numbers music video? \#curiosity
\end{itemize}

\item Why do we want more numbers?

\begin{itemize}
\item Why Zero?

\begin{itemize}
\item Additive identity

\begin{itemize}
\item Zero vector = identity vector
\item Frame of reference, starting point, nice and neutral
\end{itemize}
\end{itemize}

\item Zintegers?

\begin{itemize}
\item Why negatives?

\begin{itemize}
\item So you can make zero
\item Undo each other, undo a \(+5\)
\item Inverse

\begin{itemize}
\item \(-a\) and \(a\) are additive \emph{inverses}
\end{itemize}
\end{itemize}
\end{itemize}
\end{itemize}

\item That's all we need to get to a group:
\href{KBe2020math530refGroups.org}{KBe2020math530refGroups}
\end{itemize}
\end{itemize}

\noindent\rule{\textwidth}{0.5pt}
\end{document}
