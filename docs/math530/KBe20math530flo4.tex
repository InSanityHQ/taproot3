% Created 2021-09-11 Sat 09:36
% Intended LaTeX compiler: xelatex
\documentclass[letterpaper]{article}
\usepackage{graphicx}
\usepackage{grffile}
\usepackage{longtable}
\usepackage{wrapfig}
\usepackage{rotating}
\usepackage[normalem]{ulem}
\usepackage{amsmath}
\usepackage{textcomp}
\usepackage{amssymb}
\usepackage{capt-of}
\usepackage{hyperref}
\usepackage[margin=1in]{geometry}
\usepackage{fontspec}
\usepackage{indentfirst}
\setmainfont[ItalicFont = LiberationSans-Italic, BoldFont = LiberationSans-Bold, BoldItalicFont = LiberationSans-BoldItalic]{LiberationSans}
\newfontfamily\NHLight[ItalicFont = LiberationSansNarrow-Italic, BoldFont       = LiberationSansNarrow-Bold, BoldItalicFont = LiberationSansNarrow-BoldItalic]{LiberationSansNarrow}
\newcommand\textrmlf[1]{{\NHLight#1}}
\newcommand\textitlf[1]{{\NHLight\itshape#1}}
\let\textbflf\textrm
\newcommand\textulf[1]{{\NHLight\bfseries#1}}
\newcommand\textuitlf[1]{{\NHLight\bfseries\itshape#1}}
\usepackage{fancyhdr}
\pagestyle{fancy}
\usepackage{titlesec}
\usepackage{titling}
\makeatletter
\lhead{\textbf{\@title}}
\makeatother
\rhead{\textrmlf{Compiled} \today}
\lfoot{\theauthor\ \textbullet \ \textbf{2021-2022}}
\cfoot{}
\rfoot{\textrmlf{Page} \thepage}
\titleformat{\section} {\Large} {\textrmlf{\thesection} {|}} {0.3em} {\textbf}
\titleformat{\subsection} {\large} {\textrmlf{\thesubsection} {|}} {0.2em} {\textbf}
\titleformat{\subsubsection} {\large} {\textrmlf{\thesubsubsection} {|}} {0.1em} {\textbf}
\setlength{\parskip}{0.45em}
\renewcommand\maketitle{}
\author{Exr0n}
\date{\today}
\title{Linear Algebra class Flo 4}
\hypersetup{
 pdfauthor={Exr0n},
 pdftitle={Linear Algebra class Flo 4},
 pdfkeywords={},
 pdfsubject={},
 pdfcreator={Emacs 27.2 (Org mode 9.4.4)}, 
 pdflang={English}}
\begin{document}

\maketitle
\begin{itemize}
\item Vector spaces and fields are like groups

\begin{itemize}
\item With 2 operations
\end{itemize}

\item Vector

\begin{itemize}
\item direction and magnitude
\item numbers with an order

\begin{itemize}
\item list = ordered set
\item \(N\)x\(1\) matrix
\end{itemize}

\item A vector is not just an \(N\)x\(1\) matrix. \textbf{A vector exists in a
vector space}

\begin{itemize}
\item might be full of physics vectors, matrices, or polynomials
\end{itemize}
\end{itemize}

\item Field

\begin{itemize}
\item Addition and multiplication might be different

\begin{itemize}
\item might be related to normal addition/multiplication
\item might by any binary operation
\item Addition is "primary" operation, multiplication is "secondary"

\begin{itemize}
\item addition is really good (more group like)
\item multiplication needs to exclude the additive identity (because
it can't have an inverse)
\end{itemize}

\item questions

\begin{itemize}
\item multiplication is repeated addition?

\begin{itemize}
\item not necessarily
\end{itemize}

\item binary expressions?
\item associative?

\begin{itemize}
\item both yes
\end{itemize}
\end{itemize}

\item 1.3 demonstrates that the complex numbers are a field

\begin{itemize}
\item commutativity
\item associativity
\item identities
\item additive inverse
\item multiplicative inverse except additive identity
\item distributive
\end{itemize}
\end{itemize}

\item usually means Reals or Complex

\begin{itemize}
\item integers mod 3 are a field

\begin{itemize}
\item \#bonushw show integers mod 3 are a field
\end{itemize}
\end{itemize}

\item higher dimensions

\begin{itemize}
\item \(R^2\) is a Cartesian plane, \(R^4\) is a space
\end{itemize}

\item operations

\begin{itemize}
\item addition is really nice (element wise)
\item scalar multiplication is easy enough
\item vector vector multiplication is hard to find
\end{itemize}
\end{itemize}

\item two square roots of \(i\)

\begin{itemize}
\item fundamental theorem of algebra

\begin{itemize}
\item (important)
\end{itemize}

\item So, \(i\) should have two square roots
\item Powers of \(i\) go in a circle (90 degrees rotation)

\begin{itemize}
\item Complex number rotation gives a preferred direction
\item So that's why the quadrants are numbered in that direction
\end{itemize}

\item One can be found geometrically
\href{20math530srcSquareRootI.png.org}{20math530srcSquareRootI.png}
\item We could also try it algebraically

\begin{itemize}
\item \((a+bi)^2=i=a^2-b^2+2abi\)
\item so \(a^2-b^2 = 0\) and \(2ab = 1\)
\end{itemize}
\end{itemize}

\item dot product

\begin{itemize}
\item How much of \(\vec{A}\) is in the direction of \(\vec{B}\)
multiplied by the magnitude of \(\vec{B}\)
\item \(\vec{A} \cdot \vec{B} = |A||B| cos \theta\)

\begin{itemize}
\item \#bonushw prove that \^{}\^{}
\end{itemize}

\item Identity: \(\frac{\vec{A}\cdot\vec{B}}{|A||B|} = cos \theta\)
\end{itemize}

\item cross product

\begin{itemize}
\item only works on 3x1 matrices
\item steps

\begin{itemize}
\item determinant
\item \(i\), \(j\), \(k\) are the unit vectors
\item \[\begin{bmatrix}2\\1\\0\end{bmatrix}\begin{bmatrix}1\\2\\-1\end{bmatrix} =
        \left|\begin{bmatrix}i &j &k\\2 &1 &0\\1 &2 &-1\end{bmatrix}\right| = i\left|\begin{bmatrix}1&0\\2&-1\end{bmatrix}\right|-j\left|\begin{bmatrix}2&0\\1&-1\end{bmatrix}\right| + k\left|\begin{bmatrix}2&1\\1&2\end{bmatrix}\right| = \begin{bmatrix}-1\\2\\3\end{bmatrix}\]
\end{itemize}
\end{itemize}

\item dropping zero: \(0a = (0+0)a = 0a+0a \Rightarrow 0a = 0\), so the
additive identity can't have a multiplicative inverse (everything
multiplied it will just be the additive identity)

\begin{itemize}
\item \href{20math530srcFieldsMultiplyCannotBeGroup.png.org}{20math530srcFieldsMultiplyCannotBeGroup.png}
\end{itemize}

\item determinant

\begin{itemize}
\item measures the "size" of a matrix, denoted absolute value (relevant to
inverse of a matrix multiplication)
\end{itemize}

\item \#todo \#exr0n \#future prove identities are unique
\end{itemize}

\noindent\rule{\textwidth}{0.5pt}
\end{document}
