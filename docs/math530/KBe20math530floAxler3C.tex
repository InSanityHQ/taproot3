% Created 2021-09-11 Sat 09:36
% Intended LaTeX compiler: xelatex
\documentclass[letterpaper]{article}
\usepackage{graphicx}
\usepackage{grffile}
\usepackage{longtable}
\usepackage{wrapfig}
\usepackage{rotating}
\usepackage[normalem]{ulem}
\usepackage{amsmath}
\usepackage{textcomp}
\usepackage{amssymb}
\usepackage{capt-of}
\usepackage{hyperref}
\usepackage[margin=1in]{geometry}
\usepackage{fontspec}
\usepackage{indentfirst}
\setmainfont[ItalicFont = LiberationSans-Italic, BoldFont = LiberationSans-Bold, BoldItalicFont = LiberationSans-BoldItalic]{LiberationSans}
\newfontfamily\NHLight[ItalicFont = LiberationSansNarrow-Italic, BoldFont       = LiberationSansNarrow-Bold, BoldItalicFont = LiberationSansNarrow-BoldItalic]{LiberationSansNarrow}
\newcommand\textrmlf[1]{{\NHLight#1}}
\newcommand\textitlf[1]{{\NHLight\itshape#1}}
\let\textbflf\textrm
\newcommand\textulf[1]{{\NHLight\bfseries#1}}
\newcommand\textuitlf[1]{{\NHLight\bfseries\itshape#1}}
\usepackage{fancyhdr}
\pagestyle{fancy}
\usepackage{titlesec}
\usepackage{titling}
\makeatletter
\lhead{\textbf{\@title}}
\makeatother
\rhead{\textrmlf{Compiled} \today}
\lfoot{\theauthor\ \textbullet \ \textbf{2021-2022}}
\cfoot{}
\rfoot{\textrmlf{Page} \thepage}
\titleformat{\section} {\Large} {\textrmlf{\thesection} {|}} {0.3em} {\textbf}
\titleformat{\subsection} {\large} {\textrmlf{\thesubsection} {|}} {0.2em} {\textbf}
\titleformat{\subsubsection} {\large} {\textrmlf{\thesubsubsection} {|}} {0.1em} {\textbf}
\setlength{\parskip}{0.45em}
\renewcommand\maketitle{}
\author{Exr0n}
\date{\today}
\title{Axler 3.C Flow because CanVaS PoStInG is cool}
\hypersetup{
 pdfauthor={Exr0n},
 pdftitle={Axler 3.C Flow because CanVaS PoStInG is cool},
 pdfkeywords={},
 pdfsubject={},
 pdfcreator={Emacs 27.2 (Org mode 9.4.4)}, 
 pdflang={English}}
\begin{document}

\maketitle
\section{Axler3.30 \#def matrix \(A_{j,k}\)\hfill{}\textsc{def}}
\label{sec:org8467094}
A \(m\time n\) matrix is a rectangle of numbers with \(m\) rows and \(n\) columns. And other stuff you would expect
\section{Axler3.32 \#def matrix of a linear map, \(\mathcal M(T)\)\hfill{}\textsc{def}}
\label{sec:org7a96924}
\begin{quote}
Suppose \(T \in \mathcal L(V, W)\) and \(v_1, \ldots, v_n\) is a basis of \(V\) and \(w_1, \ldots, w_m\) is a basis of \(W\). The \emph{matrix of \(T\)} with respect to these bases is the \(m\times n\) matrix \(\mathcal M\left(\mathcal T, \left(v_1, \ldots, v_n\right), \left(w_1, \ldots, w_m\right)\right)\) whose entries \(A_{j,k}\) are defined by
$$Tv_k = A_{1, k}w_1 + \cdots + A_{m, k}w_m$$.
\end{quote}
Note that for each output \(Tv_k\) is a linear combination of a column.
\section{Algebra things}
\label{sec:orge22817f}
\subsection{Axler3.35 \#def Matrix Sum\hfill{}\textsc{def}}
\label{sec:org54369aa}
Pointwise addition, pretty straight forward. \textbf{Only works on matrices of the same size!}
\subsection{Axler 3.36 The matrix sum of linear maps}
\label{sec:orgf2313e1}
Basically matrices that are linear maps also satisfie additivity of linear maps (Given \(S, T \in \mathcal L(V, W), \mathcal M(S) + \mathcal M(T) = \mathcal M(S+T)\))
\subsection{Axler3.37 and Axler3.38 (same for scalar multiplication)}
\label{sec:org21e66f9}
Its the same for scalar multiplication, yay
\section{Notation Axler3.39 \(\mathbb F^{m,n}\)\hfill{}\textsc{notation}}
\label{sec:orgabb148f}
\(F^{m, n}\) is the set of all \(m\times n\) matrices with entries in \(\mathbb F\).
\section{Axler3.40 \(\text{dim }\mathbb F^{m,n} = mn\)}
\label{sec:orgf226542}
\(\mathbb F^{m,n}\) is itself a vector space with dimension \(mn\). (Each basis vector being a matrix with a single one at \(i, j\) for each pair of \(i, j\))?
\section{Axler3.44 \(A_{j,\cdot}\), \(A_{\cdot, k}\)}
\label{sec:org224415a}
The dot just means "everything in that row/column".
\section{Axler3.49 Column of matri product equal matrix times column}
\label{sec:org595bc35}
For \(m\times n\) matrix \(A\) and \(n\times p\) matrix \(C\), $$(AC)_{\cdot, k} = AC_{\cdot, k}$$.
\section{And many other ways to think about matrix multplication}
\label{sec:org2a5ee00}
\end{document}
