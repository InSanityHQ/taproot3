% Created 2021-09-11 Sat 08:18
% Intended LaTeX compiler: xelatex
\documentclass[letterpaper]{article}
\usepackage{graphicx}
\usepackage{grffile}
\usepackage{longtable}
\usepackage{wrapfig}
\usepackage{rotating}
\usepackage[normalem]{ulem}
\usepackage{amsmath}
\usepackage{textcomp}
\usepackage{amssymb}
\usepackage{capt-of}
\usepackage{hyperref}
\usepackage[margin=1in]{geometry}
\usepackage{fontspec}
\usepackage{indentfirst}
\setmainfont[ItalicFont = LiberationSans-Italic, BoldFont = LiberationSans-Bold, BoldItalicFont = LiberationSans-BoldItalic]{LiberationSans}
\newfontfamily\NHLight[ItalicFont = LiberationSansNarrow-Italic, BoldFont       = LiberationSansNarrow-Bold, BoldItalicFont = LiberationSansNarrow-BoldItalic]{LiberationSansNarrow}
\newcommand\textrmlf[1]{{\NHLight#1}}
\newcommand\textitlf[1]{{\NHLight\itshape#1}}
\let\textbflf\textrm
\newcommand\textulf[1]{{\NHLight\bfseries#1}}
\newcommand\textuitlf[1]{{\NHLight\bfseries\itshape#1}}
\usepackage{fancyhdr}
\pagestyle{fancy}
\usepackage{titlesec}
\usepackage{titling}
\makeatletter
\lhead{\textbf{\@title}}
\makeatother
\rhead{\textrmlf{Compiled} \today}
\lfoot{\theauthor\ \textbullet \ \textbf{2021-2022}}
\cfoot{}
\rfoot{\textrmlf{Page} \thepage}
\titleformat{\section} {\Large} {\textrmlf{\thesection} {|}} {0.3em} {\textbf}
\titleformat{\subsection} {\large} {\textrmlf{\thesubsection} {|}} {0.2em} {\textbf}
\titleformat{\subsubsection} {\large} {\textrmlf{\thesubsubsection} {|}} {0.1em} {\textbf}
\setlength{\parskip}{0.45em}
\renewcommand\maketitle{}
\author{Exr0n}
\date{\today}
\title{}
\hypersetup{
 pdfauthor={Exr0n},
 pdftitle={},
 pdfkeywords={},
 pdfsubject={},
 pdfcreator={Emacs 27.2 (Org mode 9.4.4)}, 
 pdflang={English}}
\begin{document}

\section{eigenvalues}
\label{sec:org056ee47}
eigenvalue: multiplied by a scalar?
a subspace that, when put through a linear map, only gets scaled.

\[ Tv = \lambda v \]

Where \(v \neq 0\). (we ignore it because its no fun to send zero to zero, and bc the span is empty).

\textbf{T must be an operator!} Otherwise the matrix sizes don't work out when subtracting \(\lambda I\).

where \(v\) is the eigenvector and \(\lambda\) is the eigenvalue. The equation is often rewritten as:

\[
  \begin{aligned}
  Tv - \lambda v &= 0
  Tv - \lambda Iv &= 0
  (T-\lambda I) v &= 0
  \end{aligned} \]

We want \(T-\lambda I\) to be singular, because we want the null space to include \(v\).
So we subtract \(\lambda\) from the \(I\) diagonal of \(T\) and then find values of \(v\) which are equal to zero?

now this can be factored and roots can be found. also it's an operator.

\subsection{Axler 5.6 equivalent conditions}
\label{sec:orgde6755f}
Only when \(V\) is finite dimensional!
\subsubsection{\(T-\lambda I\) is not injective, because both \(v, 0\) are in the null space.}
\label{sec:org3c4e45d}
\subsubsection{\(T-\lambda I\) is also not surjective or invertible bc finite dim operator.}
\label{sec:org196d734}

\section{an example}
\label{sec:org6c7ce82}
Given the matrix \(\begin{pmatrix}3&1\\0&2\end{pmatrix}\), find the eigenvalues and eigenvectors.

Now that we have that other fomulation, we can just subtract \(\lambda I\) from \(T\) to get
\[ \begin{pmatrix}3-\lambda &1\\0&2-lambda \end{pmatrix} \]

Then, we just need to find whether it is non-invertible aka singular aka determinant.

\[ (3-\lambda)(2-\lambda) = 0 \]
The solutions are \(\lambda = 2 \text{ or } 3\). These are the eigenvalues.

Now just plug in \(\lambda\) and find the null space using RREF. The null space for \(\lambda = 3\) has null space \(\text{span}(x, 0)\), so we just pick one of those vectors (ex. \((1, 0)\)) to be the eigenvector.


\subsection{review of terms}
\label{sec:org3860a0e}

\subsubsection{\(span(1, 0)\) is an invariant subspace. (also whatever you get for \(\lambda = 2\)}
\label{sec:org8703233}

\subsubsection{any vector in an invariant subspace is an eigenvector}
\label{sec:org453c3de}

\subsubsection{the eigenvalues are \(2, 3\)}
\label{sec:org7d0b675}

\subsection{general idea}
\label{sec:orgc35a34c}
the point of eigenvectors is to figure out where other vectors go by looking at pieces that only get streched or shrunk.

\section{depends on}
\label{sec:orgd2a984d}
\subsection{finding roots is helpful}
\label{sec:org908ee6c}
\end{document}
