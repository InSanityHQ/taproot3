% Created 2021-09-11 Sat 09:36
% Intended LaTeX compiler: xelatex
\documentclass[letterpaper]{article}
\usepackage{graphicx}
\usepackage{grffile}
\usepackage{longtable}
\usepackage{wrapfig}
\usepackage{rotating}
\usepackage[normalem]{ulem}
\usepackage{amsmath}
\usepackage{textcomp}
\usepackage{amssymb}
\usepackage{capt-of}
\usepackage{hyperref}
\usepackage[margin=1in]{geometry}
\usepackage{fontspec}
\usepackage{indentfirst}
\setmainfont[ItalicFont = LiberationSans-Italic, BoldFont = LiberationSans-Bold, BoldItalicFont = LiberationSans-BoldItalic]{LiberationSans}
\newfontfamily\NHLight[ItalicFont = LiberationSansNarrow-Italic, BoldFont       = LiberationSansNarrow-Bold, BoldItalicFont = LiberationSansNarrow-BoldItalic]{LiberationSansNarrow}
\newcommand\textrmlf[1]{{\NHLight#1}}
\newcommand\textitlf[1]{{\NHLight\itshape#1}}
\let\textbflf\textrm
\newcommand\textulf[1]{{\NHLight\bfseries#1}}
\newcommand\textuitlf[1]{{\NHLight\bfseries\itshape#1}}
\usepackage{fancyhdr}
\pagestyle{fancy}
\usepackage{titlesec}
\usepackage{titling}
\makeatletter
\lhead{\textbf{\@title}}
\makeatother
\rhead{\textrmlf{Compiled} \today}
\lfoot{\theauthor\ \textbullet \ \textbf{2021-2022}}
\cfoot{}
\rfoot{\textrmlf{Page} \thepage}
\titleformat{\section} {\Large} {\textrmlf{\thesection} {|}} {0.3em} {\textbf}
\titleformat{\subsection} {\large} {\textrmlf{\thesubsection} {|}} {0.2em} {\textbf}
\titleformat{\subsubsection} {\large} {\textrmlf{\thesubsubsection} {|}} {0.1em} {\textbf}
\setlength{\parskip}{0.45em}
\renewcommand\maketitle{}
\author{Exr0n}
\date{\today}
\title{Linear Algebra Dimensions (Axler 2.C)}
\hypersetup{
 pdfauthor={Exr0n},
 pdftitle={Linear Algebra Dimensions (Axler 2.C)},
 pdfkeywords={},
 pdfsubject={},
 pdfcreator={Emacs 27.2 (Org mode 9.4.4)}, 
 pdflang={English}}
\begin{document}

\maketitle
\#source Axler2.C

\#ref \#disorganized \#incomplete

\section{\#definition dimension}
\label{sec:org4e91cc6}
\begin{quote}
The dimension of \(V\) (denoted \(\text{dim }V\) is the length of a
basis in \(V\) - This relies on Axler2.35: Basis length does not
depend on the basis Any two bases of a finite-dimensional vector space
have the same length
\end{quote}

\subsection{Results}
\label{sec:org919014d}
\subsection{Axler2.38 Dimension of a subspace}
\label{sec:org1aa9717}
\begin{quote}
If \(V\) is finite-dimensional and \(U\) is a subspace of \(V\), then
\(\text{dim }U \le \text{dim }V\) - Intuitive. The basis of a subspace
is shorter than the basis of the original vecspace, because otherwise
it's span would be larger than the original vecspace (because bases
are linearly independent + len lin-indep \(\le\) len span).
\end{quote}

\subsection{Axler2.39 Linearly independent list of the right length is a basis}
\label{sec:org92a7b57}
\begin{quote}
All linearly independent lists of the length \(\text{dim }V\) are
bases. - Intuitive. If it's already linearly independent meaning each
element brings "new information", then if there's that many elements
then there should be that much information.
\end{quote}

\subsection{Axler2.43 Dimension of a sum}
\label{sec:orge48c633}
\begin{quote}
If \(U_1\) and \(U_2\) are subspaces of a finite dimensional vecspace,
then
\[\text{dim}(U_1+U_2) = \text{dim }U_1 + \text{dim }U_2 - \text{dim}(U_1\cap U_2)\] -
This inducts into something analogous to PIE!
\href{KBrefPrincipleInclusionExclusion.org}{KBrefPrincipleInclusionExclusion}
\end{quote}

\noindent\rule{\textwidth}{0.5pt}
\end{document}
