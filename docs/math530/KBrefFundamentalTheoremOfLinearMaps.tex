% Created 2021-09-11 Sat 09:36
% Intended LaTeX compiler: xelatex
\documentclass[letterpaper]{article}
\usepackage{graphicx}
\usepackage{grffile}
\usepackage{longtable}
\usepackage{wrapfig}
\usepackage{rotating}
\usepackage[normalem]{ulem}
\usepackage{amsmath}
\usepackage{textcomp}
\usepackage{amssymb}
\usepackage{capt-of}
\usepackage{hyperref}
\usepackage[margin=1in]{geometry}
\usepackage{fontspec}
\usepackage{indentfirst}
\setmainfont[ItalicFont = LiberationSans-Italic, BoldFont = LiberationSans-Bold, BoldItalicFont = LiberationSans-BoldItalic]{LiberationSans}
\newfontfamily\NHLight[ItalicFont = LiberationSansNarrow-Italic, BoldFont       = LiberationSansNarrow-Bold, BoldItalicFont = LiberationSansNarrow-BoldItalic]{LiberationSansNarrow}
\newcommand\textrmlf[1]{{\NHLight#1}}
\newcommand\textitlf[1]{{\NHLight\itshape#1}}
\let\textbflf\textrm
\newcommand\textulf[1]{{\NHLight\bfseries#1}}
\newcommand\textuitlf[1]{{\NHLight\bfseries\itshape#1}}
\usepackage{fancyhdr}
\pagestyle{fancy}
\usepackage{titlesec}
\usepackage{titling}
\makeatletter
\lhead{\textbf{\@title}}
\makeatother
\rhead{\textrmlf{Compiled} \today}
\lfoot{\theauthor\ \textbullet \ \textbf{2021-2022}}
\cfoot{}
\rfoot{\textrmlf{Page} \thepage}
\titleformat{\section} {\Large} {\textrmlf{\thesection} {|}} {0.3em} {\textbf}
\titleformat{\subsection} {\large} {\textrmlf{\thesubsection} {|}} {0.2em} {\textbf}
\titleformat{\subsubsection} {\large} {\textrmlf{\thesubsubsection} {|}} {0.1em} {\textbf}
\setlength{\parskip}{0.45em}
\renewcommand\maketitle{}
\author{Exr0n}
\date{\today}
\title{Fundamental Theorem of Linear Maps!}
\hypersetup{
 pdfauthor={Exr0n},
 pdftitle={Fundamental Theorem of Linear Maps!},
 pdfkeywords={},
 pdfsubject={},
 pdfcreator={Emacs 27.2 (Org mode 9.4.4)}, 
 pdflang={English}}
\begin{document}

\maketitle
\section{Intuition}
\label{sec:org8db8196}
In a linear map \(T : V\to W\), the dimension of the domain \(V\) is amount of stuff that you throw away (null space) \textbf{plus} the amount of stuff that does not get thrown away (the column space).
\sout{If \(T\) is a map from \(V\) to \(W\), then the dimension of the source map \(V\) is the dimension of the null space (everything in \(V\) that \(T\) takes to 0) plus the dimension of the range (all possible things taken to by \(T\))}
\section{\#definition Fundamental Theorem of Linear Maps\hfill{}\textsc{def}}
\label{sec:orgd5f86d3}
\begin{quote}
Suppose \(V\) is finite-dimensional and \$T \(\in\) \mathcal L(V, W). Then \text{range }T is finite-dimensional and
$$ \text{dim }V = \text{dim null }T + \text{dim range} T $$
\end{quote}
\subsection{See also \href{KBrefDimension.org}{dimension}, \href{KBrefNullSpace.org}{null space}, and \href{KBrefFunctionRange.org}{range of a function}.}
\label{sec:org5e1ca79}
\section{AKA: Rank Nullity Theorem}
\label{sec:org5ec604d}
This is a subset of the Fundamental Theorem of Linear Algebra (\#todo-expand)
\end{document}
