% Created 2021-09-27 Mon 12:03
% Intended LaTeX compiler: xelatex
\documentclass[letterpaper]{article}
\usepackage{graphicx}
\usepackage{grffile}
\usepackage{longtable}
\usepackage{wrapfig}
\usepackage{rotating}
\usepackage[normalem]{ulem}
\usepackage{amsmath}
\usepackage{textcomp}
\usepackage{amssymb}
\usepackage{capt-of}
\usepackage{hyperref}
\setlength{\parindent}{0pt}
\usepackage[margin=1in]{geometry}
\usepackage{fontspec}
\usepackage{svg}
\usepackage{cancel}
\usepackage{indentfirst}
\setmainfont[ItalicFont = LiberationSans-Italic, BoldFont = LiberationSans-Bold, BoldItalicFont = LiberationSans-BoldItalic]{LiberationSans}
\newfontfamily\NHLight[ItalicFont = LiberationSansNarrow-Italic, BoldFont       = LiberationSansNarrow-Bold, BoldItalicFont = LiberationSansNarrow-BoldItalic]{LiberationSansNarrow}
\newcommand\textrmlf[1]{{\NHLight#1}}
\newcommand\textitlf[1]{{\NHLight\itshape#1}}
\let\textbflf\textrm
\newcommand\textulf[1]{{\NHLight\bfseries#1}}
\newcommand\textuitlf[1]{{\NHLight\bfseries\itshape#1}}
\usepackage{fancyhdr}
\pagestyle{fancy}
\usepackage{titlesec}
\usepackage{titling}
\makeatletter
\lhead{\textbf{\@title}}
\makeatother
\rhead{\textrmlf{Compiled} \today}
\lfoot{\theauthor\ \textbullet \ \textbf{2021-2022}}
\cfoot{}
\rfoot{\textrmlf{Page} \thepage}
\renewcommand{\tableofcontents}{}
\titleformat{\section} {\Large} {\textrmlf{\thesection} {|}} {0.3em} {\textbf}
\titleformat{\subsection} {\large} {\textrmlf{\thesubsection} {|}} {0.2em} {\textbf}
\titleformat{\subsubsection} {\large} {\textrmlf{\thesubsubsection} {|}} {0.1em} {\textbf}
\setlength{\parskip}{0.45em}
\renewcommand\maketitle{}
\author{Exr0n}
\date{\today}
\title{Practice multiplying matrices and identifying groups}
\hypersetup{
 pdfauthor={Exr0n},
 pdftitle={Practice multiplying matrices and identifying groups},
 pdfkeywords={},
 pdfsubject={},
 pdfcreator={Emacs 28.0.50 (Org mode 9.4.4)}, 
 pdflang={English}}
\begin{document}

\tableofcontents

\begin{quote}
What sizes of matrix can you add? When can't you add matrices?
\end{quote}

Matrices of the same dimensions (because we do it element wise). Maybe
you can add a vector to a matrix if the number of rows is equal to the
dimensionality of the vector.

\begin{quote}
What sizes of matrix can you multiply? When can't you multiply
matrices?
\end{quote}

Multiply: \(N\times M\) * \(M\times K\) => \(N\times K\).

\begin{quote}
Multiply $\backslash$[
\begin{bmatrix} 
3 & 0 \\
0 & 1 
\end{bmatrix},
\begin{bmatrix} 
1 & 0 \\
0 & -2 
\end{bmatrix},
\begin{bmatrix} 
1 & 1 \\
0 & 1 
\end{bmatrix},
\begin{bmatrix} 
0 & 1 \\
-1 & 0 
\end{bmatrix}
$\backslash$] by vectors in \(\mathbb{R}^2\) (for example, you could multiply by
\(\begin{bmatrix} 0\\ 0 \end{bmatrix}\) or
\(\begin{bmatrix} 1\\ -2 \end{bmatrix}\)).
\end{quote}

\begin{quote}
Can you characterize the transformations you get by multiplying (lots
of vectors) by each of these matrices?
\end{quote}

\begin{center}
\begin{tabular}{ll}
Action & Matrix\\
\hline
Identity & \(\begin{bmatrix} 1 \\ 1 \end{bmatrix}\)\\
Select left column & \(\begin{bmatrix} 1 \\ 0 \end{bmatrix}\)\\
Select right column & \(\begin{bmatrix} 0 \\ 1 \end{bmatrix}\)\\
Treat as expression (linear combination/transformation?)* & \(\begin{bmatrix} a \\ b \end{bmatrix}\)\\
\end{tabular}
\end{center}

*I'm not sure what linear combinations/transformations are, but I think
this is somehow related? Anyways, it takes each row \(i\) and returns
\(\sigma A_{i,j} * B_{j}\)

\begin{quote}
Which of the number systems we discussed today form a group under
addition? Under multiplication?
\end{quote}

Source: \href{KBe2020math530refGroups.org}{KBe2020math530refGroups}

\begin{center}
\begin{tabular}{lll}
Number System & Multiplication & Addition\\
\hline
Natural Numbers & No inverse & No identity\\
Whole Numbers & No inverse & No inverse\\
Integers & No inverse & Yes\\
Rationals & Yes & Yes\\
Reals & Yes & Yes\\
Complex Numbers & Yes & Yes\\
\end{tabular}
\end{center}

\noindent\rule{\textwidth}{0.5pt}
\end{document}
