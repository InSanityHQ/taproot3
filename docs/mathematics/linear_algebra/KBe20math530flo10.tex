% Created 2021-09-11 Sat 16:43
% Intended LaTeX compiler: xelatex
\documentclass[letterpaper]{article}
\usepackage{graphicx}
\usepackage{grffile}
\usepackage{longtable}
\usepackage{wrapfig}
\usepackage{rotating}
\usepackage[normalem]{ulem}
\usepackage{amsmath}
\usepackage{textcomp}
\usepackage{amssymb}
\usepackage{capt-of}
\usepackage{hyperref}
\usepackage[margin=1in]{geometry}
\usepackage{fontspec}
\usepackage{indentfirst}
\setmainfont[ItalicFont = LiberationSans-Italic, BoldFont = LiberationSans-Bold, BoldItalicFont = LiberationSans-BoldItalic]{LiberationSans}
\newfontfamily\NHLight[ItalicFont = LiberationSansNarrow-Italic, BoldFont       = LiberationSansNarrow-Bold, BoldItalicFont = LiberationSansNarrow-BoldItalic]{LiberationSansNarrow}
\newcommand\textrmlf[1]{{\NHLight#1}}
\newcommand\textitlf[1]{{\NHLight\itshape#1}}
\let\textbflf\textrm
\newcommand\textulf[1]{{\NHLight\bfseries#1}}
\newcommand\textuitlf[1]{{\NHLight\bfseries\itshape#1}}
\usepackage{fancyhdr}
\pagestyle{fancy}
\usepackage{titlesec}
\usepackage{titling}
\makeatletter
\lhead{\textbf{\@title}}
\makeatother
\rhead{\textrmlf{Compiled} \today}
\lfoot{\theauthor\ \textbullet \ \textbf{2021-2022}}
\cfoot{}
\rfoot{\textrmlf{Page} \thepage}
\titleformat{\section} {\Large} {\textrmlf{\thesection} {|}} {0.3em} {\textbf}
\titleformat{\subsection} {\large} {\textrmlf{\thesubsection} {|}} {0.2em} {\textbf}
\titleformat{\subsubsection} {\large} {\textrmlf{\thesubsubsection} {|}} {0.1em} {\textbf}
\setlength{\parskip}{0.45em}
\renewcommand\maketitle{}
\author{Exr0n}
\date{\today}
\title{Flo 10}
\hypersetup{
 pdfauthor={Exr0n},
 pdftitle={Flo 10},
 pdfkeywords={},
 pdfsubject={},
 pdfcreator={Emacs 27.2 (Org mode 9.4.4)}, 
 pdflang={English}}
\begin{document}

\maketitle
\#flo

\section{Span}
\label{sec:org9e0e11c}
\subsection{Smallest/largest containing subspaces}
\label{sec:orgb8da55c}
\begin{itemize}
\item Spans are not the largest vector space that contains the given vectors
\href{Pasted image 20200924131215.png.org}{Pasted image
20200924131215.png}
\item The span of that vector is a line. It's a subspace. But it's not the
biggest, because there's also R\textsuperscript{2}
\end{itemize}

\subsection{Spans tend to be infinite}
\label{sec:org85a8203}
\begin{itemize}
\item Usually a span has infinitely many vectors (unless you're in a weird
field (modulo) or have the zero span)
\item In the span of just one vector, you can multiply by any scalar which
there tends to be infinite of \href{Pasted image 20200924131215.png.org}{Pasted image 20200924131215.png}
\item The span of that vector is a line. It's a subspace. But it's not the
biggest, because there's also R\textsuperscript{2}
\item It only won't be infinite if your span is the span of \(()\) (empty
list)
\end{itemize}

\subsection{Given a linearly independent set of vectors, would the span equal to}
\label{sec:org3f074c7}
the vector space?
:CUSTOM\textsubscript{ID}: given-a-linearly-independent-set-of-vectors-would-the-span-equal-to-the-vector-space

\begin{itemize}
\item No? It's unclear which vector space is being referred to.
\end{itemize}

\subsection{Span of vectors (example 2.6)}
\label{sec:org4d29d61}
\begin{itemize}
\item When it's two vectors, you'd expect the span to be a 2d plane unless
the vectors are parallel

\begin{itemize}
\item In other words, if they are linear combinations or scalar multiples
of one another
\item A linear combination on one other vector is the same as a scalar
multiple
\item in 2space they have to not be colinear, in 3space they have to not
be coplanar.
\item They have to be linearly independent
\end{itemize}

\item That probably generalizes to higher and lower dimensions
\end{itemize}

\subsection{Adding a vector doesn't make the span smaller}
\label{sec:org843ff12}
\begin{itemize}
\item Because you can just do what you had originally and make it's
coefficient zero
\end{itemize}

\subsection{Size of spans/subspaces}
\label{sec:org32bb55d}
\begin{itemize}
\item You can't really just count the number of vectors, because say a line
and a plane both have infinite points
\item But we still want a plane to be larger than a line and a space to be
larger than a plane
\item So one way we compare is to say \(A\) is larger than \(B\) if \(B\) is
strictly contained within \(A\)
\item something like "dimensionality", maybe the minimum number of vectors
needed for their span to be equal to the space
\end{itemize}

\section{2.7 Span is the smallest containing subspace}
\label{sec:org1210fb1}
\begin{itemize}
\item First the proof shows that the span is a subspace
\item Then, because the span only neds to contain each vector and be a
subspace, any subspace containing those vectors will at least contain
the span.
\end{itemize}

\section{Linear Dependence}
\label{sec:org4cbd075}
\begin{itemize}
\item When one of the vectors provides no "new information" aka can be
constructed by a linear combination of vectors you already had
\item It's a property of a set of vectors, not just one vector. A single
vector is always linearly independent on its own, because there's
nothing else to depend on.
\item The span of the zero vector \((0)\) is linearly dependent on itself,
and you already don't really get anything. So we usually talk about it
as a span of no vectors \(()\)
\end{itemize}

\section{Rotation matrices}
\label{sec:orgc712621}
\begin{itemize}
\item Find a formula
\item Prove the formula
\item maybe draw a picture
\item \href{KBE2020math501floMatriciesAsTransformations.org}{KBE2020math501floMatriciesAsTransformations}
\end{itemize}

\noindent\rule{\textwidth}{0.5pt}
\end{document}
