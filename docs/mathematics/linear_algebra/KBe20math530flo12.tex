% Created 2021-09-12 Sun 22:49
% Intended LaTeX compiler: xelatex
\documentclass[letterpaper]{article}
\usepackage{graphicx}
\usepackage{grffile}
\usepackage{longtable}
\usepackage{wrapfig}
\usepackage{rotating}
\usepackage[normalem]{ulem}
\usepackage{amsmath}
\usepackage{textcomp}
\usepackage{amssymb}
\usepackage{capt-of}
\usepackage{hyperref}
\usepackage[margin=1in]{geometry}
\usepackage{fontspec}
\usepackage{indentfirst}
\setmainfont[ItalicFont = LiberationSans-Italic, BoldFont = LiberationSans-Bold, BoldItalicFont = LiberationSans-BoldItalic]{LiberationSans}
\newfontfamily\NHLight[ItalicFont = LiberationSansNarrow-Italic, BoldFont       = LiberationSansNarrow-Bold, BoldItalicFont = LiberationSansNarrow-BoldItalic]{LiberationSansNarrow}
\newcommand\textrmlf[1]{{\NHLight#1}}
\newcommand\textitlf[1]{{\NHLight\itshape#1}}
\let\textbflf\textrm
\newcommand\textulf[1]{{\NHLight\bfseries#1}}
\newcommand\textuitlf[1]{{\NHLight\bfseries\itshape#1}}
\usepackage{fancyhdr}
\pagestyle{fancy}
\usepackage{titlesec}
\usepackage{titling}
\makeatletter
\lhead{\textbf{\@title}}
\makeatother
\rhead{\textrmlf{Compiled} \today}
\lfoot{\theauthor\ \textbullet \ \textbf{2021-2022}}
\cfoot{}
\rfoot{\textrmlf{Page} \thepage}
\titleformat{\section} {\Large} {\textrmlf{\thesection} {|}} {0.3em} {\textbf}
\titleformat{\subsection} {\large} {\textrmlf{\thesubsection} {|}} {0.2em} {\textbf}
\titleformat{\subsubsection} {\large} {\textrmlf{\thesubsubsection} {|}} {0.1em} {\textbf}
\setlength{\parskip}{0.45em}
\renewcommand\maketitle{}
\author{Exr0n}
\date{\today}
\title{Exr0n}
\hypersetup{
 pdfauthor={Exr0n},
 pdftitle={Exr0n},
 pdfkeywords={},
 pdfsubject={},
 pdfcreator={Emacs 28.0.50 (Org mode 9.4.4)}, 
 pdflang={English}}
\begin{document}

\maketitle


\section{\#exercise 2.A.17}
\label{sec:org7418d37}
\begin{itemize}
\item All polynomials have \((x+2)\) as a factor, and therefore can be
written in the form \((x+2)f_j(x)\) where \(f_j(x)\) has degree at
most \(m-1\).
\item Because the \(z^0, z^1, ..., z^{m-1}\) is a spanning list of
\(P_m-1(F)\), the spanning list of \(P_{m-1}(F)\) is of length \(m\).
\item The original list had \(m+1\) elements, so by Axler 2.23 the list
cannot be linearly independent.
\item We can therefore find a non-trivial combination that equals zero, and
can thus find a non-trivial combination of the original list by
multiplying each vector by \((x-2)\).
\end{itemize}

\section{Elementary Matrices}
\label{sec:orgad0c646}
\#incomplete

\subsection{Things you can do}
\label{sec:orgfdad5d5}
\begin{itemize}
\item Multiply a row by a nonzero scalar
\item Add two rows
\item Switch the ordering of the rows
\end{itemize}

The matrices that correspond to these operations are what we call
\#definition elementary matrices.

This includes the identity matrix (multiply by the scalar 1).

\noindent\rule{\textwidth}{0.5pt}
\end{document}
