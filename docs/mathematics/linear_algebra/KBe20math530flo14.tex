% Created 2021-09-27 Mon 12:03
% Intended LaTeX compiler: xelatex
\documentclass[letterpaper]{article}
\usepackage{graphicx}
\usepackage{grffile}
\usepackage{longtable}
\usepackage{wrapfig}
\usepackage{rotating}
\usepackage[normalem]{ulem}
\usepackage{amsmath}
\usepackage{textcomp}
\usepackage{amssymb}
\usepackage{capt-of}
\usepackage{hyperref}
\setlength{\parindent}{0pt}
\usepackage[margin=1in]{geometry}
\usepackage{fontspec}
\usepackage{svg}
\usepackage{cancel}
\usepackage{indentfirst}
\setmainfont[ItalicFont = LiberationSans-Italic, BoldFont = LiberationSans-Bold, BoldItalicFont = LiberationSans-BoldItalic]{LiberationSans}
\newfontfamily\NHLight[ItalicFont = LiberationSansNarrow-Italic, BoldFont       = LiberationSansNarrow-Bold, BoldItalicFont = LiberationSansNarrow-BoldItalic]{LiberationSansNarrow}
\newcommand\textrmlf[1]{{\NHLight#1}}
\newcommand\textitlf[1]{{\NHLight\itshape#1}}
\let\textbflf\textrm
\newcommand\textulf[1]{{\NHLight\bfseries#1}}
\newcommand\textuitlf[1]{{\NHLight\bfseries\itshape#1}}
\usepackage{fancyhdr}
\pagestyle{fancy}
\usepackage{titlesec}
\usepackage{titling}
\makeatletter
\lhead{\textbf{\@title}}
\makeatother
\rhead{\textrmlf{Compiled} \today}
\lfoot{\theauthor\ \textbullet \ \textbf{2021-2022}}
\cfoot{}
\rfoot{\textrmlf{Page} \thepage}
\renewcommand{\tableofcontents}{}
\titleformat{\section} {\Large} {\textrmlf{\thesection} {|}} {0.3em} {\textbf}
\titleformat{\subsection} {\large} {\textrmlf{\thesubsection} {|}} {0.2em} {\textbf}
\titleformat{\subsubsection} {\large} {\textrmlf{\thesubsubsection} {|}} {0.1em} {\textbf}
\setlength{\parskip}{0.45em}
\renewcommand\maketitle{}
\author{Exr0n}
\date{\today}
\title{LinAlg flo 14}
\hypersetup{
 pdfauthor={Exr0n},
 pdftitle={LinAlg flo 14},
 pdfkeywords={},
 pdfsubject={},
 pdfcreator={Emacs 28.0.50 (Org mode 9.4.4)}, 
 pdflang={English}}
\begin{document}

\tableofcontents

\#flo \#disorganized \#incomplete

\section{Administrative bits}
\label{sec:org2b82910}
\begin{itemize}
\item Will present problems from 2.B and/or 2.C next week
\item Mini quiz, stop yourself after an hour
\item and give your subconscious a chance to think about things
\item \textbf{No need to say "clearly", "obviously", "evidently"}
\end{itemize}

\section{\#icr Axler2.C}
\label{sec:orgebc5fe4}
\#source Axler Linear Algebra Done Right 2.C \#\# Polynomials are vectors -
because you can add and scale them and they are kind of nice in general

\subsection{The box under 2.38}
\label{sec:orgc0d0d8e}
\begin{itemize}
\item You can't understand a vector space just by knowing the vectors inside

\begin{itemize}
\item you also need to know the field that you are in
\item See 2.A ex5
\end{itemize}

\item The field that you are over changes your dimension: usually we think
of \(\mathbb{C}\) as a vector space over \(\mathbb{R}\), but in this
class we think of it as over \(\mathbb{C}\), which means
\(\text{dim }\mathbb{C} = 1\)
\end{itemize}

\subsection{Axler2.41}
\label{sec:org956ed3b}
\begin{itemize}
\item It's my question! See
\href{KBe20math530floQuestions.org}{KBe20math530floQuestions}
\end{itemize}

\subsection{Axler2.42}
\label{sec:orgc6d1a59}
\begin{itemize}
\item \#tip If it's a spanning list that's the right length, then it's a
basis and therefore linearly independent.
\item If it's a linearly independent list and it's the right length, then
it's a basis and therefore spanning.
\end{itemize}

\subsection{Axler2.43 Dimension of a Sum}
\label{sec:org4975357}
\subsubsection{An Example}
\label{sec:org0bac76a}
\begin{itemize}
\item If you have two planes in 3 space, and they intersect at exactly one
line, then you can't just add the dimension of the two planes (2+2 = 4
which is more than 3 space can allow).

\begin{itemize}
\item If the planes are parallel, and both subspaces, then we know they
both go through the origin and thus are the same plane.
\end{itemize}
\end{itemize}

\subsubsection{Some tips}
\label{sec:org3ae4c97}
\begin{itemize}
\item Usually easiest to get a basis of a subspace by building on instead of
taking out

\begin{itemize}
\item for example if you have a slanty plane in 3 space, and you start
with standard basis, then you won't even get the slanty plane.
\end{itemize}
\end{itemize}

\subsubsection{The span is \(U_1+U_2\)}
\label{sec:orgdd731b5}
\begin{itemize}
\item Because it's a double containment

\begin{itemize}
\item \(span \subset U_1+U_2\)
\item \(v \in span \implies v = a_1u_1 + \ldots + a_mb_m + b_1v_1 + \dots\)
\item For all \(u\). in the span, you can write it as something in \(U_1\)
\begin{itemize}
\item something in \(U_2\)
\end{itemize}
\end{itemize}
\end{itemize}

\noindent\rule{\textwidth}{0.5pt}
\end{document}
