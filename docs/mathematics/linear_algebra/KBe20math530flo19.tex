% Created 2021-09-12 Sun 22:49
% Intended LaTeX compiler: xelatex
\documentclass[letterpaper]{article}
\usepackage{graphicx}
\usepackage{grffile}
\usepackage{longtable}
\usepackage{wrapfig}
\usepackage{rotating}
\usepackage[normalem]{ulem}
\usepackage{amsmath}
\usepackage{textcomp}
\usepackage{amssymb}
\usepackage{capt-of}
\usepackage{hyperref}
\usepackage[margin=1in]{geometry}
\usepackage{fontspec}
\usepackage{indentfirst}
\setmainfont[ItalicFont = LiberationSans-Italic, BoldFont = LiberationSans-Bold, BoldItalicFont = LiberationSans-BoldItalic]{LiberationSans}
\newfontfamily\NHLight[ItalicFont = LiberationSansNarrow-Italic, BoldFont       = LiberationSansNarrow-Bold, BoldItalicFont = LiberationSansNarrow-BoldItalic]{LiberationSansNarrow}
\newcommand\textrmlf[1]{{\NHLight#1}}
\newcommand\textitlf[1]{{\NHLight\itshape#1}}
\let\textbflf\textrm
\newcommand\textulf[1]{{\NHLight\bfseries#1}}
\newcommand\textuitlf[1]{{\NHLight\bfseries\itshape#1}}
\usepackage{fancyhdr}
\pagestyle{fancy}
\usepackage{titlesec}
\usepackage{titling}
\makeatletter
\lhead{\textbf{\@title}}
\makeatother
\rhead{\textrmlf{Compiled} \today}
\lfoot{\theauthor\ \textbullet \ \textbf{2021-2022}}
\cfoot{}
\rfoot{\textrmlf{Page} \thepage}
\titleformat{\section} {\Large} {\textrmlf{\thesection} {|}} {0.3em} {\textbf}
\titleformat{\subsection} {\large} {\textrmlf{\thesubsection} {|}} {0.2em} {\textbf}
\titleformat{\subsubsection} {\large} {\textrmlf{\thesubsubsection} {|}} {0.1em} {\textbf}
\setlength{\parskip}{0.45em}
\renewcommand\maketitle{}
\author{Exr0n}
\date{\today}
\title{Lin Alg flo 19}
\hypersetup{
 pdfauthor={Exr0n},
 pdftitle={Lin Alg flo 19},
 pdfkeywords={},
 pdfsubject={},
 pdfcreator={Emacs 28.0.50 (Org mode 9.4.4)}, 
 pdflang={English}}
\begin{document}

\maketitle
\section{Broader vector spaces}
\label{sec:org13aa69e}
\begin{itemize}
\item Doesn't have to be physics vectors
\item maybe it's like matrices
\item or linear maps themselves
\end{itemize}
\section{The Linear Map 0}
\label{sec:org73c2ce8}
A linear map \(S = 0\) is a map where \(Su = 0 \forall u\).
\section{Axler 3.A ex7 (w/ Vienna + Mason)}
\label{sec:org47c25ba}
Let \(w = Tv\).

\subsection{If \(v = 0\) then}
\label{sec:org5cd3f62}
$$Tv = 0$$
By Axler 3.11 (Maps take 0 to 0). Thus, \(\lambda\) can be anything in \(\mathbb F\).

\subsection{Otherwise,}
\label{sec:org67e62a0}
\(\frac{1}{v} \in \mathbb F\) because the field has multiplicative inverses for all elements except 0.
$$
   Tv = w = \left( w \frac{1}{v} \right)v
   $$
Let \(\lambda = w \frac{1}{v}\), then
$$ \lambda v = w \frac{1}{v} v = w $$
which is in \(\mathbb F\) because \(w, \frac{1}{v} \in \mathbb F\) and fields are closed under multiplication.

\section{Axler 3.A ex10 (w/ Vienna + Mason)}
\label{sec:org9ed1fd4}
The additivity of a linear map \(T\) requires \(T(u+v) = Tu + Tv\). Because \(U \subset V, U \neq V\), there must be some element \(v \in V\) yet \(v \notin U\).

For some element \(u \in U\),
$$Tu + Tv = Su + 0 = Su$$
Yet \(u+v \notin U\) because if it were, then \((u+v)+(-v) = v\) would be in \(U\). Thus,
$$T(u+v) = 0$$

Because for some \(u\) \(Su\neq 0\), additivity does not hold over \(T\) and thus the map is not linear.
\end{document}
