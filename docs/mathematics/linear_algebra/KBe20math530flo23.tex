% Created 2021-09-27 Mon 11:52
% Intended LaTeX compiler: xelatex
\documentclass[letterpaper]{article}
\usepackage{graphicx}
\usepackage{grffile}
\usepackage{longtable}
\usepackage{wrapfig}
\usepackage{rotating}
\usepackage[normalem]{ulem}
\usepackage{amsmath}
\usepackage{textcomp}
\usepackage{amssymb}
\usepackage{capt-of}
\usepackage{hyperref}
\setlength{\parindent}{0pt}
\usepackage[margin=1in]{geometry}
\usepackage{fontspec}
\usepackage{svg}
\usepackage{cancel}
\usepackage{indentfirst}
\setmainfont[ItalicFont = LiberationSans-Italic, BoldFont = LiberationSans-Bold, BoldItalicFont = LiberationSans-BoldItalic]{LiberationSans}
\newfontfamily\NHLight[ItalicFont = LiberationSansNarrow-Italic, BoldFont       = LiberationSansNarrow-Bold, BoldItalicFont = LiberationSansNarrow-BoldItalic]{LiberationSansNarrow}
\newcommand\textrmlf[1]{{\NHLight#1}}
\newcommand\textitlf[1]{{\NHLight\itshape#1}}
\let\textbflf\textrm
\newcommand\textulf[1]{{\NHLight\bfseries#1}}
\newcommand\textuitlf[1]{{\NHLight\bfseries\itshape#1}}
\usepackage{fancyhdr}
\pagestyle{fancy}
\usepackage{titlesec}
\usepackage{titling}
\makeatletter
\lhead{\textbf{\@title}}
\makeatother
\rhead{\textrmlf{Compiled} \today}
\lfoot{\theauthor\ \textbullet \ \textbf{2021-2022}}
\cfoot{}
\rfoot{\textrmlf{Page} \thepage}
\renewcommand{\tableofcontents}{}
\titleformat{\section} {\Large} {\textrmlf{\thesection} {|}} {0.3em} {\textbf}
\titleformat{\subsection} {\large} {\textrmlf{\thesubsection} {|}} {0.2em} {\textbf}
\titleformat{\subsubsection} {\large} {\textrmlf{\thesubsubsection} {|}} {0.1em} {\textbf}
\setlength{\parskip}{0.45em}
\renewcommand\maketitle{}
\author{Exr0n}
\date{\today}
\title{Flo 23 (19 Nov 2020, skipped a bunch of proofs)}
\hypersetup{
 pdfauthor={Exr0n},
 pdftitle={Flo 23 (19 Nov 2020, skipped a bunch of proofs)},
 pdfkeywords={},
 pdfsubject={},
 pdfcreator={Emacs 28.0.50 (Org mode 9.4.4)}, 
 pdflang={English}}
\begin{document}

\tableofcontents


\section{Row Reduced Echelon Form}
\label{sec:orge29a932}
Null space is the same (because algebra).
Then turn it into a system of equations and use those equations to find the null space.

\section{Factoring a vector}
\label{sec:org9911c62}
Say we have \(\begin{pmatrix}-2x_3-4x_4\\-4x_3-7x_4\\x_3\\x_4\end{pmatrix}\).
Then you can write it as the linear combination $$\begin{pmatrix}-2x_3\\-4x_3\\x_3\\0\end{pmatrix}+\begin{pmatrix}-4x_4\\-7x_4\\0\\x_4\end{pmatrix} = x_3\begin{pmatrix}-2\\-4\\1\\0\end{pmatrix}+x_4\begin{pmatrix}-4\\-7\\0\\1\end{pmatrix}$$

\section{\#icr 3.C\hfill{}\textsc{icr}}
\label{sec:orgac7eccd}

\subsection{Matrix Definition}
\label{sec:org0c1babf}
Old news (but lots of subscripts)

\subsection{Making a matrix from a map}
\label{sec:orgbcdfd9a}
Based on maps being uniquely determined

\subsection{Matrix addition and scalar multiplication}
\label{sec:orgdc3021d}
Not news

\subsection{The matrix for the derivative map (finite)}
\label{sec:orge78ba80}
$$T \in \mathcal L\left(\mathcal P_5\left(\mathbb R\right), \mathcal P_4\left(\mathbb R\right)\right)$$
Start with standard bases: \(\mathcal P_5 \rightarrow 1, x, x^2, x^3, x^4, x^5\), \(\mathcal P_4 \rightarrow 1, x, x^2, x^3, x^4\)
Now lets define the map:
$$\begin{align*}
   T 1 &= 0\\
   T x &= 1\\
   T x^2 &= 2x\\
   T x^3 &= 3x^2\\
   T x^4 &= 4x^3\\
   T x^5 &= 5x^4\\
   \end{align*}$$

And then we write each output as a linear combo of the basis of \(\mathcal P_4\) then we can define the matrix:

$$ \begin{pmatrix}
   0&1&0&0&0&0\\
   0&0&2&0&0&0\\
   0&0&0&3&0&0\\
   0&0&0&0&4&0\\
   0&0&0&0&0&5\\
   \end{pmatrix} \begin{pmatrix}
   a_0\\a_1\\a_2\\a_3\\a_4\\a_5
   \end{pmatrix} $$

Note that the matrix is \(5\times 6\) because we are going from dimension \(6 \to 5\) (and the second dimension gets "consumed" in the multiplication)

\subsection{Axler3.40 dimension of the matrix vector space\hfill{}\textsc{icr}}
\label{sec:org25f9b20}
Put a one in every location which forms a basis.

\subsection{Axler3.49 column of matrix product equals matrix times column\hfill{}\textsc{icr}}
\label{sec:org33c2f08}
Makes sense if you draw it out.. basically a column in the product \(AC\) will have used all of \(A\) but only the one column in \(C\).
$$(AC)_{\cdot, k} = A(C_{\cdot, k})$$ and $$(AC)_{j, \cdot} = (A_{j, \cdot})C$$
\end{document}
