% Created 2021-09-12 Sun 22:50
% Intended LaTeX compiler: xelatex
\documentclass[letterpaper]{article}
\usepackage{graphicx}
\usepackage{grffile}
\usepackage{longtable}
\usepackage{wrapfig}
\usepackage{rotating}
\usepackage[normalem]{ulem}
\usepackage{amsmath}
\usepackage{textcomp}
\usepackage{amssymb}
\usepackage{capt-of}
\usepackage{hyperref}
\usepackage[margin=1in]{geometry}
\usepackage{fontspec}
\usepackage{indentfirst}
\setmainfont[ItalicFont = LiberationSans-Italic, BoldFont = LiberationSans-Bold, BoldItalicFont = LiberationSans-BoldItalic]{LiberationSans}
\newfontfamily\NHLight[ItalicFont = LiberationSansNarrow-Italic, BoldFont       = LiberationSansNarrow-Bold, BoldItalicFont = LiberationSansNarrow-BoldItalic]{LiberationSansNarrow}
\newcommand\textrmlf[1]{{\NHLight#1}}
\newcommand\textitlf[1]{{\NHLight\itshape#1}}
\let\textbflf\textrm
\newcommand\textulf[1]{{\NHLight\bfseries#1}}
\newcommand\textuitlf[1]{{\NHLight\bfseries\itshape#1}}
\usepackage{fancyhdr}
\pagestyle{fancy}
\usepackage{titlesec}
\usepackage{titling}
\makeatletter
\lhead{\textbf{\@title}}
\makeatother
\rhead{\textrmlf{Compiled} \today}
\lfoot{\theauthor\ \textbullet \ \textbf{2021-2022}}
\cfoot{}
\rfoot{\textrmlf{Page} \thepage}
\titleformat{\section} {\Large} {\textrmlf{\thesection} {|}} {0.3em} {\textbf}
\titleformat{\subsection} {\large} {\textrmlf{\thesubsection} {|}} {0.2em} {\textbf}
\titleformat{\subsubsection} {\large} {\textrmlf{\thesubsubsection} {|}} {0.1em} {\textbf}
\setlength{\parskip}{0.45em}
\renewcommand\maketitle{}
\author{Exr0n}
\date{\today}
\title{Linear Algebra Questions}
\hypersetup{
 pdfauthor={Exr0n},
 pdftitle={Linear Algebra Questions},
 pdfkeywords={},
 pdfsubject={},
 pdfcreator={Emacs 28.0.50 (Org mode 9.4.4)}, 
 pdflang={English}}
\begin{document}

\maketitle


\section{Readings}
\label{sec:org76d8406}
\begin{itemize}
\item Axler 2.A

\begin{itemize}
\item Under "Linear Independence", what is the whole thing about
subtracting equations and "if the only way to do this is the obvious
way"? pg.32
\item Linear independence feels somewhat okay, but everything past linear
dependence lost me.
\end{itemize}

\item Axler 2.C

\begin{itemize}
\item Under example 2.41, near the end, why can't \(\text{dim }U\) not
equal 4? Why must you be able to expand it by at least one element?

\begin{itemize}
\item Maybe because there are elements in \(\mathcal{P}_m(\mathbb{R})\)
that aren't in \(U\), so the basis of \(U\) must be a different
length from the basis of \(V\) (else \(U\) would equal \(V\) and
all elements of \(V\) would be in \(U\) by 2.39)
\item We can shove \(f(x) = x\) into the basis of \(U\) and it will
still be linearly independent (because \(f\) was not in \(U\)), so
\(\text{dim }U\) must be less than 4.
\end{itemize}
\end{itemize}
\end{itemize}

\noindent\rule{\textwidth}{0.5pt}
\end{document}
