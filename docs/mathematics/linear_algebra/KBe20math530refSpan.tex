% Created 2021-09-12 Sun 22:49
% Intended LaTeX compiler: xelatex
\documentclass[letterpaper]{article}
\usepackage{graphicx}
\usepackage{grffile}
\usepackage{longtable}
\usepackage{wrapfig}
\usepackage{rotating}
\usepackage[normalem]{ulem}
\usepackage{amsmath}
\usepackage{textcomp}
\usepackage{amssymb}
\usepackage{capt-of}
\usepackage{hyperref}
\usepackage[margin=1in]{geometry}
\usepackage{fontspec}
\usepackage{indentfirst}
\setmainfont[ItalicFont = LiberationSans-Italic, BoldFont = LiberationSans-Bold, BoldItalicFont = LiberationSans-BoldItalic]{LiberationSans}
\newfontfamily\NHLight[ItalicFont = LiberationSansNarrow-Italic, BoldFont       = LiberationSansNarrow-Bold, BoldItalicFont = LiberationSansNarrow-BoldItalic]{LiberationSansNarrow}
\newcommand\textrmlf[1]{{\NHLight#1}}
\newcommand\textitlf[1]{{\NHLight\itshape#1}}
\let\textbflf\textrm
\newcommand\textulf[1]{{\NHLight\bfseries#1}}
\newcommand\textuitlf[1]{{\NHLight\bfseries\itshape#1}}
\usepackage{fancyhdr}
\pagestyle{fancy}
\usepackage{titlesec}
\usepackage{titling}
\makeatletter
\lhead{\textbf{\@title}}
\makeatother
\rhead{\textrmlf{Compiled} \today}
\lfoot{\theauthor\ \textbullet \ \textbf{2021-2022}}
\cfoot{}
\rfoot{\textrmlf{Page} \thepage}
\titleformat{\section} {\Large} {\textrmlf{\thesection} {|}} {0.3em} {\textbf}
\titleformat{\subsection} {\large} {\textrmlf{\thesubsection} {|}} {0.2em} {\textbf}
\titleformat{\subsubsection} {\large} {\textrmlf{\thesubsubsection} {|}} {0.1em} {\textbf}
\setlength{\parskip}{0.45em}
\renewcommand\maketitle{}
\author{Exr0n}
\date{\today}
\title{Span of vector lists}
\hypersetup{
 pdfauthor={Exr0n},
 pdftitle={Span of vector lists},
 pdfkeywords={},
 pdfsubject={},
 pdfcreator={Emacs 28.0.50 (Org mode 9.4.4)}, 
 pdflang={English}}
\begin{document}

\maketitle
\#source Axler2.A

\section{\#definition span}
\label{sec:orgf5e7ce2}
\begin{quote}
The set of all linear combinations of a list of vectors
\(v_1, ..., v_m\) in \(V\) is called the span of \(v_1, ..., v_m\),
denoted \(\text{span}(v_1,...,v_m)\):
\[\text{span}(v_1,...,v_m) = {a_1v_1 + ... + a_mv_m | a_1, ..., a_m \in F}\]
And the span of an empty list \(()\) is \({0}\) - This is just to make
Axler2.C work out nicely
(\href{KBeRefLinAlgDimension.org}{KBeRefLinAlgDimension})
\end{quote}

\section{Properties}
\label{sec:orgbb39794}
\begin{itemize}
\item The span is the smallest containing subspace

\begin{itemize}
\item \begin{quote}
The span of a list of vectors in \(V\) is the smallest subspace of
\(V\) containing all the vectors in the list.
\end{quote}
\end{itemize}
\end{itemize}

\subsection{\#definition spans}
\label{sec:org977d6ff}
\begin{quote}
If \(\text{span}(v_1,...,v_m) = V\), then \(v_1, ..., v_m\) \textbf{\emph{spans}}
\(V\)
\end{quote}

\section{Examples}
\label{sec:orgccbfca1}
\subsection{Axler 2.9}
\label{sec:org5ed7d1c}
\begin{quote}
Suppose \(n\) is a positive integer. Show that
\((1, 0, ..., 0), (0, 1, 0, ..., 0), ..., (0, ..., 0, 1)\) spans
\(F^n\). - Basically, if a list of vectors spans a vector space then
linear combinations of those vectors (almost like colloquial
polynomials of those vectors) can form each vector in the space. - In
this case, the vector space \(F^n\) is a list of vectors in \(F\), and
having the \(1\) in each slot is enough to, when scalar multiplied
with \(a \in F\), get all possibilities of \(F^n\). - I need to wrap
my head around this some more.
\end{quote}

\noindent\rule{\textwidth}{0.5pt}
\end{document}
