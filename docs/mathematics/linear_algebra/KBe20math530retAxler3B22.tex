% Created 2021-09-12 Sun 22:49
% Intended LaTeX compiler: xelatex
\documentclass[letterpaper]{article}
\usepackage{graphicx}
\usepackage{grffile}
\usepackage{longtable}
\usepackage{wrapfig}
\usepackage{rotating}
\usepackage[normalem]{ulem}
\usepackage{amsmath}
\usepackage{textcomp}
\usepackage{amssymb}
\usepackage{capt-of}
\usepackage{hyperref}
\usepackage[margin=1in]{geometry}
\usepackage{fontspec}
\usepackage{indentfirst}
\setmainfont[ItalicFont = LiberationSans-Italic, BoldFont = LiberationSans-Bold, BoldItalicFont = LiberationSans-BoldItalic]{LiberationSans}
\newfontfamily\NHLight[ItalicFont = LiberationSansNarrow-Italic, BoldFont       = LiberationSansNarrow-Bold, BoldItalicFont = LiberationSansNarrow-BoldItalic]{LiberationSansNarrow}
\newcommand\textrmlf[1]{{\NHLight#1}}
\newcommand\textitlf[1]{{\NHLight\itshape#1}}
\let\textbflf\textrm
\newcommand\textulf[1]{{\NHLight\bfseries#1}}
\newcommand\textuitlf[1]{{\NHLight\bfseries\itshape#1}}
\usepackage{fancyhdr}
\pagestyle{fancy}
\usepackage{titlesec}
\usepackage{titling}
\makeatletter
\lhead{\textbf{\@title}}
\makeatother
\rhead{\textrmlf{Compiled} \today}
\lfoot{\theauthor\ \textbullet \ \textbf{2021-2022}}
\cfoot{}
\rfoot{\textrmlf{Page} \thepage}
\titleformat{\section} {\Large} {\textrmlf{\thesection} {|}} {0.3em} {\textbf}
\titleformat{\subsection} {\large} {\textrmlf{\thesubsection} {|}} {0.2em} {\textbf}
\titleformat{\subsubsection} {\large} {\textrmlf{\thesubsubsection} {|}} {0.1em} {\textbf}
\setlength{\parskip}{0.45em}
\renewcommand\maketitle{}
\author{Exr0n}
\date{\today}
\title{Axler 3.B Exercise 22}
\hypersetup{
 pdfauthor={Exr0n},
 pdftitle={Axler 3.B Exercise 22},
 pdfkeywords={},
 pdfsubject={},
 pdfcreator={Emacs 28.0.50 (Org mode 9.4.4)}, 
 pdflang={English}}
\begin{document}

\maketitle
\section{Problem}
\label{sec:orgd3f6e3c}
\begin{quote}
Suppose \(U\) and \(V\) are finite-dimensional vector spaces and \(S \in \mathcal L(V ,w)\) and \(T \in \mathcal L(U, V)\). Prove that
$$\text{dim null }ST \leq \text{dim null }S + \text{dim null }T.$$
\end{quote}
\section{Proof}
\label{sec:orgff9405d}
All vectors \(v \in \text{null }ST\) must have been nulled by \(T\) or \(S\), and therefore either it must be in \(\text{null T}\) or \(Tv\) in \(\text{range }T \cap \text{null }S\). Notationally,
$$\text{null }ST = \text{null }T \cup \{v : Tv \in \left(\text{range }T \cap \text{null }S\right)\}$$
Note that because this union is equal to \(\text{null }ST\), it is a vector space.
Because no vector can be in both \(\text{null }T\) and \(\{v : Tv \in \left(\text{range }T \cap \text{null }S\right)\}\), the dimension of the union is
$$\text{dim null }ST = \text{dim null }T + \text{dim }\left(\{v : Tv \in \left(\text{range }T \cap \text{null }S\right)\}\right)$$
Every value of \(w\) that satisfies \(w \in \left(\text{range }T \cap \text{null }S\right)\) will the output of \(Tv\) for some \(v\), because the range is defined as all the outputs of \(Tv\).
$$\text{dim null }ST = \text{dim null }T + \text{dim }\left(\text{range }T \cap \text{null }S\right)$$
An intersection can only make the dimension of a set smaller, so \(\text{dim }\left(\text{range }T \cap \text{null }S\right) \leq \text{dim null }S\) and
$$\text{dim range }ST \leq \text{dim null }S, \text{dim null }T$$
\end{document}
