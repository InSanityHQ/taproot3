% Created 2021-09-11 Sat 16:42
% Intended LaTeX compiler: xelatex
\documentclass[letterpaper]{article}
\usepackage{graphicx}
\usepackage{grffile}
\usepackage{longtable}
\usepackage{wrapfig}
\usepackage{rotating}
\usepackage[normalem]{ulem}
\usepackage{amsmath}
\usepackage{textcomp}
\usepackage{amssymb}
\usepackage{capt-of}
\usepackage{hyperref}
\usepackage[margin=1in]{geometry}
\usepackage{fontspec}
\usepackage{indentfirst}
\setmainfont[ItalicFont = LiberationSans-Italic, BoldFont = LiberationSans-Bold, BoldItalicFont = LiberationSans-BoldItalic]{LiberationSans}
\newfontfamily\NHLight[ItalicFont = LiberationSansNarrow-Italic, BoldFont       = LiberationSansNarrow-Bold, BoldItalicFont = LiberationSansNarrow-BoldItalic]{LiberationSansNarrow}
\newcommand\textrmlf[1]{{\NHLight#1}}
\newcommand\textitlf[1]{{\NHLight\itshape#1}}
\let\textbflf\textrm
\newcommand\textulf[1]{{\NHLight\bfseries#1}}
\newcommand\textuitlf[1]{{\NHLight\bfseries\itshape#1}}
\usepackage{fancyhdr}
\pagestyle{fancy}
\usepackage{titlesec}
\usepackage{titling}
\makeatletter
\lhead{\textbf{\@title}}
\makeatother
\rhead{\textrmlf{Compiled} \today}
\lfoot{\theauthor\ \textbullet \ \textbf{2021-2022}}
\cfoot{}
\rfoot{\textrmlf{Page} \thepage}
\titleformat{\section} {\Large} {\textrmlf{\thesection} {|}} {0.3em} {\textbf}
\titleformat{\subsection} {\large} {\textrmlf{\thesubsection} {|}} {0.2em} {\textbf}
\titleformat{\subsubsection} {\large} {\textrmlf{\thesubsubsection} {|}} {0.1em} {\textbf}
\setlength{\parskip}{0.45em}
\renewcommand\maketitle{}
\author{Exr0n}
\date{\today}
\title{Length of spanning list is greater equal to length of linearly independent list}
\hypersetup{
 pdfauthor={Exr0n},
 pdftitle={Length of spanning list is greater equal to length of linearly independent list},
 pdfkeywords={},
 pdfsubject={},
 pdfcreator={Emacs 27.2 (Org mode 9.4.4)}, 
 pdflang={English}}
\begin{document}

\maketitle
\section{Lemma}
\label{sec:orgba64203}

\begin{quote}
The length of a linearly indpendent list is less than or equal to the length of a spanning list over some vector space \(V\).
\end{quote}

\section{Intermediate Result: Span of a linearly independent extension of a linearly independent list has more elements than the span of the original list.}
\label{sec:orgf997bd1}
\subsection{Lemma}
\label{sec:org983aa12}
  Given a linearly independent list \(v = v_1, \ldots, v_k\) where each vector \(v_1, \ldots, v_k \in V\) and another vector \(v_{k+1}\) which is linearly independent with \(v\), show that
$$\text{span}\left(v_1, \ldots, v_k, v_{k+1}\right)$$
contains elements that are not in
$$\text{span}\left(v_1, \ldots, v_k\right)$$
TODO: This needs to show that a longer list will have a larger span, not just an extended one.

\subsection{Proof}
\label{sec:org1ce32cc}
   Because \(v_{k+1}\) is linearly independent with \(v\), it cannot be written as a linear combination of elements in \(v\). Thus,
$$v_{k+1} \notin \text{span}\left(v_1, \ldots, v_k\right)$$
However, \(v_{k+1}\) must be in the span of the extended list, because we can write \(v_{k+1}\) as
$$0v_1 + 0v_2 + \ldots + 0v_k + 1v_{k+1}$$
Thus, the extended list contains atleast one element that the original did not.
\section{Proof}
\label{sec:org067a36e}
Given a spanning list \(u = u_1, \ldots, u_j\) and a linearly independent list \(v = v_1, \ldots, v_k\), show that the \(|u| \ge |v|\). The Linear Dependence Lemma states that while \(u\) is linearly dependent, it is possible to remove some vector \(u_i\) from \(u\) such that the span stays the same. Thus, there exists a linearly independent list \(b\) that has the same span as \(u\), aka that also spans \(V\). Because this list can be obtained by removing elements from \(u\), \(|b| \le |u|\).

The linearly independent list \(v\) must be shorter than or equal to \(b\) in length, because otherwise, \(\text{span }v\) would have more elements than \(\text{span }b\) by the intermediate result. Thus, \(|v| \le |b| \le |u|\).
\end{document}
