% Created 2021-09-12 Sun 22:49
% Intended LaTeX compiler: xelatex
\documentclass[letterpaper]{article}
\usepackage{graphicx}
\usepackage{grffile}
\usepackage{longtable}
\usepackage{wrapfig}
\usepackage{rotating}
\usepackage[normalem]{ulem}
\usepackage{amsmath}
\usepackage{textcomp}
\usepackage{amssymb}
\usepackage{capt-of}
\usepackage{hyperref}
\usepackage[margin=1in]{geometry}
\usepackage{fontspec}
\usepackage{indentfirst}
\setmainfont[ItalicFont = LiberationSans-Italic, BoldFont = LiberationSans-Bold, BoldItalicFont = LiberationSans-BoldItalic]{LiberationSans}
\newfontfamily\NHLight[ItalicFont = LiberationSansNarrow-Italic, BoldFont       = LiberationSansNarrow-Bold, BoldItalicFont = LiberationSansNarrow-BoldItalic]{LiberationSansNarrow}
\newcommand\textrmlf[1]{{\NHLight#1}}
\newcommand\textitlf[1]{{\NHLight\itshape#1}}
\let\textbflf\textrm
\newcommand\textulf[1]{{\NHLight\bfseries#1}}
\newcommand\textuitlf[1]{{\NHLight\bfseries\itshape#1}}
\usepackage{fancyhdr}
\pagestyle{fancy}
\usepackage{titlesec}
\usepackage{titling}
\makeatletter
\lhead{\textbf{\@title}}
\makeatother
\rhead{\textrmlf{Compiled} \today}
\lfoot{\theauthor\ \textbullet \ \textbf{2021-2022}}
\cfoot{}
\rfoot{\textrmlf{Page} \thepage}
\titleformat{\section} {\Large} {\textrmlf{\thesection} {|}} {0.3em} {\textbf}
\titleformat{\subsection} {\large} {\textrmlf{\thesubsection} {|}} {0.2em} {\textbf}
\titleformat{\subsubsection} {\large} {\textrmlf{\thesubsubsection} {|}} {0.1em} {\textbf}
\setlength{\parskip}{0.45em}
\renewcommand\maketitle{}
\author{Taproot}
\date{\today}
\title{Axler 7.A ex 2}
\hypersetup{
 pdfauthor={Taproot},
 pdftitle={Axler 7.A ex 2},
 pdfkeywords={},
 pdfsubject={},
 pdfcreator={Emacs 28.0.50 (Org mode 9.4.4)}, 
 pdflang={English}}
\begin{document}

\maketitle
\begin{quote}
Suppose \(T \in  \mathcal{L}(V)\) and \(\lambda \in \mathbb{F}\). Prove that \(\lambda\) is an eigenvalue of \(T\) iff \(\overline{\lambda}\) is an eigenvalue of \(T^*\).
\end{quote}

Given \(\lambda\) is an eigenvalue of \(T\), show that \(\overline{\lambda}\) is an eigenvalue of \(T^*\). This will imply both directions, since \(\lambda = \overline{\overline{\lambda}}\) and \(T = T^{*^*}\)

Suppose \(\mathcal{M}(T)\) is the matrix of \(T\) wrt some orthonormal basis. Then, the matrix \(\mathcal{M}(T^*)\) of \(T^*\) wrt the same orthonormal basis will equal the conjugate transpose of \(\mathcal{M}(T)\).

Eigenvalues lie on the diagonal of a matrix, so the conjugate transpose will have the effect of conjugating each eigenvalue. Thus, the eigenvalues of \(\mathcal{M}(T)\) are conjugates of the eigenvalues of \(\mathcal{M}(T^*)\).


\[\begin{aligned}
 \langle T-\lambda I v, v \rangle = \langle v, (T-\lambda I)^* \rangle = \langle v, T^* - \overline{\lambda} I v \rangle
\end{aligned}\]


\section{:noexport:}
\label{sec:orgaae94c2}
There exists some \(v\) s.t.
\[\begin{aligned}
Tv = \lambda v
\end{aligned}\]


\[\begin{aligned}
 \langle \lambda v, w \rangle = \langle Tv, w \rangle = \langle v, T^* w \rangle
\end{aligned}\]
\end{document}
