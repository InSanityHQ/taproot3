% Created 2021-09-27 Mon 11:53
% Intended LaTeX compiler: xelatex
\documentclass[letterpaper]{article}
\usepackage{graphicx}
\usepackage{grffile}
\usepackage{longtable}
\usepackage{wrapfig}
\usepackage{rotating}
\usepackage[normalem]{ulem}
\usepackage{amsmath}
\usepackage{textcomp}
\usepackage{amssymb}
\usepackage{capt-of}
\usepackage{hyperref}
\setlength{\parindent}{0pt}
\usepackage[margin=1in]{geometry}
\usepackage{fontspec}
\usepackage{svg}
\usepackage{cancel}
\usepackage{indentfirst}
\setmainfont[ItalicFont = LiberationSans-Italic, BoldFont = LiberationSans-Bold, BoldItalicFont = LiberationSans-BoldItalic]{LiberationSans}
\newfontfamily\NHLight[ItalicFont = LiberationSansNarrow-Italic, BoldFont       = LiberationSansNarrow-Bold, BoldItalicFont = LiberationSansNarrow-BoldItalic]{LiberationSansNarrow}
\newcommand\textrmlf[1]{{\NHLight#1}}
\newcommand\textitlf[1]{{\NHLight\itshape#1}}
\let\textbflf\textrm
\newcommand\textulf[1]{{\NHLight\bfseries#1}}
\newcommand\textuitlf[1]{{\NHLight\bfseries\itshape#1}}
\usepackage{fancyhdr}
\pagestyle{fancy}
\usepackage{titlesec}
\usepackage{titling}
\makeatletter
\lhead{\textbf{\@title}}
\makeatother
\rhead{\textrmlf{Compiled} \today}
\lfoot{\theauthor\ \textbullet \ \textbf{2021-2022}}
\cfoot{}
\rfoot{\textrmlf{Page} \thepage}
\renewcommand{\tableofcontents}{}
\titleformat{\section} {\Large} {\textrmlf{\thesection} {|}} {0.3em} {\textbf}
\titleformat{\subsection} {\large} {\textrmlf{\thesubsection} {|}} {0.2em} {\textbf}
\titleformat{\subsubsection} {\large} {\textrmlf{\thesubsubsection} {|}} {0.1em} {\textbf}
\setlength{\parskip}{0.45em}
\renewcommand\maketitle{}
\author{Taproot}
\date{\today}
\title{Axler 7.B exercise 7}
\hypersetup{
 pdfauthor={Taproot},
 pdftitle={Axler 7.B exercise 7},
 pdfkeywords={},
 pdfsubject={},
 pdfcreator={Emacs 28.0.50 (Org mode 9.4.4)}, 
 pdflang={English}}
\begin{document}

\tableofcontents

\begin{quote}
Suppose \(V\) is a complex inner product space and \(T \in  \mathcal{L}(V)\) is a normal operator such that \(T^9 = T^8\). Prove that \(T\) is self-adjoint and \(T^2 = T\).
\end{quote}

If \(T = 0\), then \(0^2 = 0\) and \(0\) is self-adjoint. Thus, let \(T \neq  0\).

In 7.1, Axler asserts that \(V\) is finite-dimensional.

By the complex spectral theorem, \(T\) has a diagonal matrix w.r.t. an orthonormal basis of \(V\).

Let these entries equal \(d_1, \ldots, d_n\). \(T^k\) will have on it's diagonal \(d_1^k, \ldots, d_n^k\). For each \(d_i\), \(d_i^8 = d_i^9\). The only values in \(\mathbb{F}\) that satisfy this are zero and one; thus every \(d_i\) must be a zero or a one.

Thus, \(T = T^2\) and \(T\) is self-adjoint.

\section{:noexport:}
\label{sec:orgde0130e}

\[\begin{aligned}
T T^* = T^* T
\end{aligned}\]

First, we will show that \(T^2 = T\). Suppose \(T\) is invertible. Then,
\[\begin{aligned}
T^9 &= T^8 \\
T^9 T^{-7}  &= T^8 T^{-7}\\
T^2 &= T
\end{aligned}\]
Suppose \(T\) is not invertible and not equal to zero. Then, \(T\) has some zero entries on it's diagonal and some non-zero entries on it's diagonal.
\end{document}
