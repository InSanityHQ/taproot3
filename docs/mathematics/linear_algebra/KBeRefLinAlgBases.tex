% Created 2021-09-27 Mon 12:03
% Intended LaTeX compiler: xelatex
\documentclass[letterpaper]{article}
\usepackage{graphicx}
\usepackage{grffile}
\usepackage{longtable}
\usepackage{wrapfig}
\usepackage{rotating}
\usepackage[normalem]{ulem}
\usepackage{amsmath}
\usepackage{textcomp}
\usepackage{amssymb}
\usepackage{capt-of}
\usepackage{hyperref}
\setlength{\parindent}{0pt}
\usepackage[margin=1in]{geometry}
\usepackage{fontspec}
\usepackage{svg}
\usepackage{cancel}
\usepackage{indentfirst}
\setmainfont[ItalicFont = LiberationSans-Italic, BoldFont = LiberationSans-Bold, BoldItalicFont = LiberationSans-BoldItalic]{LiberationSans}
\newfontfamily\NHLight[ItalicFont = LiberationSansNarrow-Italic, BoldFont       = LiberationSansNarrow-Bold, BoldItalicFont = LiberationSansNarrow-BoldItalic]{LiberationSansNarrow}
\newcommand\textrmlf[1]{{\NHLight#1}}
\newcommand\textitlf[1]{{\NHLight\itshape#1}}
\let\textbflf\textrm
\newcommand\textulf[1]{{\NHLight\bfseries#1}}
\newcommand\textuitlf[1]{{\NHLight\bfseries\itshape#1}}
\usepackage{fancyhdr}
\pagestyle{fancy}
\usepackage{titlesec}
\usepackage{titling}
\makeatletter
\lhead{\textbf{\@title}}
\makeatother
\rhead{\textrmlf{Compiled} \today}
\lfoot{\theauthor\ \textbullet \ \textbf{2021-2022}}
\cfoot{}
\rfoot{\textrmlf{Page} \thepage}
\renewcommand{\tableofcontents}{}
\titleformat{\section} {\Large} {\textrmlf{\thesection} {|}} {0.3em} {\textbf}
\titleformat{\subsection} {\large} {\textrmlf{\thesubsection} {|}} {0.2em} {\textbf}
\titleformat{\subsubsection} {\large} {\textrmlf{\thesubsubsection} {|}} {0.1em} {\textbf}
\setlength{\parskip}{0.45em}
\renewcommand\maketitle{}
\author{Exr0n}
\date{\today}
\title{Bases}
\hypersetup{
 pdfauthor={Exr0n},
 pdftitle={Bases},
 pdfkeywords={},
 pdfsubject={},
 pdfcreator={Emacs 28.0.50 (Org mode 9.4.4)}, 
 pdflang={English}}
\begin{document}

\tableofcontents

\#source Axler "Linear Algebra Done Right" chapter 2.B

\#flo \#ref \#disorganized \#incomplete

\section{Bases}
\label{sec:org8b9ec0b}
\subsection{Summary}
\label{sec:orgaecde93}
If it spans, and it's linearly independent, it's a basis!

\subsection{Axler2.27 \#definition basis}
\label{sec:org524d6f3}
\begin{quote}
A \emph{basis} of \(V\) is a list of vectors in \(V\) that is linearly
independent and spans \(V\). - Basically a linearly independent
spanning list, or the "minimum" amount of information contained in a
vector space
\end{quote}

\subsubsection{Other Results}
\label{sec:orgfdcb7bc}
\begin{itemize}
\item Axler2.29 "criterion for a basis"

\begin{itemize}
\item A list is a basis if and only if each vector in \(V\) can be written
as exactly one linear combination of the list
\end{itemize}

\item Axler2.31 all spanning lists contain a basis

\begin{itemize}
\item Intuitive. A spanning list might not be linearly independent, but
some subset of it must be.
\end{itemize}

\item Axler2.32 Any finite dimensional vector space has a basis

\begin{itemize}
\item Intuitive. It has a spanning list
\item Also, no infinite dimensional vector space has a basis, by
definition
\end{itemize}

\item Axler2.33 Linearly indepedent lists can be extended to a basis

\begin{itemize}
\item Intuitive. Do this by adding in vectors that "bring new information"
\end{itemize}

\item Axler2.34 Every subspace of \(V\) is part of a direct sum of \(V\)

\begin{itemize}
\item Intuitive. Kind of like saying there's an additive complement to
every subspace of \(V\)
\item Any vector space can be thought of the span of it's basis. Because
\(V\) has a basis, and one of \(U\)'s basises can be written as a
subsequence of \(V\)'s basis, that basis can be expanded and the
expanded elements spanned to form the complement vecspace.
\end{itemize}
\end{itemize}

\noindent\rule{\textwidth}{0.5pt}
\end{document}
