% Created 2021-09-12 Sun 22:49
% Intended LaTeX compiler: xelatex
\documentclass[letterpaper]{article}
\usepackage{graphicx}
\usepackage{grffile}
\usepackage{longtable}
\usepackage{wrapfig}
\usepackage{rotating}
\usepackage[normalem]{ulem}
\usepackage{amsmath}
\usepackage{textcomp}
\usepackage{amssymb}
\usepackage{capt-of}
\usepackage{hyperref}
\usepackage[margin=1in]{geometry}
\usepackage{fontspec}
\usepackage{indentfirst}
\setmainfont[ItalicFont = LiberationSans-Italic, BoldFont = LiberationSans-Bold, BoldItalicFont = LiberationSans-BoldItalic]{LiberationSans}
\newfontfamily\NHLight[ItalicFont = LiberationSansNarrow-Italic, BoldFont       = LiberationSansNarrow-Bold, BoldItalicFont = LiberationSansNarrow-BoldItalic]{LiberationSansNarrow}
\newcommand\textrmlf[1]{{\NHLight#1}}
\newcommand\textitlf[1]{{\NHLight\itshape#1}}
\let\textbflf\textrm
\newcommand\textulf[1]{{\NHLight\bfseries#1}}
\newcommand\textuitlf[1]{{\NHLight\bfseries\itshape#1}}
\usepackage{fancyhdr}
\pagestyle{fancy}
\usepackage{titlesec}
\usepackage{titling}
\makeatletter
\lhead{\textbf{\@title}}
\makeatother
\rhead{\textrmlf{Compiled} \today}
\lfoot{\theauthor\ \textbullet \ \textbf{2021-2022}}
\cfoot{}
\rfoot{\textrmlf{Page} \thepage}
\titleformat{\section} {\Large} {\textrmlf{\thesection} {|}} {0.3em} {\textbf}
\titleformat{\subsection} {\large} {\textrmlf{\thesubsection} {|}} {0.2em} {\textbf}
\titleformat{\subsubsection} {\large} {\textrmlf{\thesubsubsection} {|}} {0.1em} {\textbf}
\setlength{\parskip}{0.45em}
\renewcommand\maketitle{}
\author{Taproot}
\date{\today}
\title{Adjoint}
\hypersetup{
 pdfauthor={Taproot},
 pdftitle={Adjoint},
 pdfkeywords={},
 pdfsubject={},
 pdfcreator={Emacs 28.0.50 (Org mode 9.4.4)}, 
 pdflang={English}}
\begin{document}

\maketitle
\section{adjoint, \(T^*\)\hfill{}\textsc{def}}
\label{sec:org06e6fdb}
\begin{quote}
Suppose \$T \(\in\) \mathcal L(V, W). The \emph{adjoint} of \(T\) is the function \(T^* : W \to  V\) s.t.
\[\begin{aligned}
   \langle Tv, w \rangle = \langle v, T^* w \rangle
  \end{aligned}\]
\end{quote}
Apparently there's another meaning for 'adjoint' in linear algebra too, but it's not covered here.

This definition makes sense because of the \href{KBrefLinearFunctional.org}{Riesz Representation Theorem}\ldots{} :question:

Adjoints are kind of like complex conjugates, as seen in \href{KBrefConjugateTranspose.org}{Axler 7.10}

\section{results}
\label{sec:org7333cb1}

\subsection{Useful technique: 'flip \(T^*\) from one side of an inner product to become \(T\) on the other side'}
\label{sec:orgfc25d3e}
You can always do this by definition of adjoint.

\subsection{Axler7.5 the adjoint is a linear map}
\label{sec:org9b511b3}

\begin{quote}
If \(T \in   \mathcal{L} (V, W)\), then \(T^* \in  \mathcal{L} (W, V)\).
\end{quote}

\subsection{Axler7.6 Properties of the adjoint}
\label{sec:org9e92c4d}

\subsubsection{\((S+T)^* = S^* + T^*\) for all \(S, T \in  \mathcal{L}(V, W)\)}
\label{sec:orgdfe125a}

\subsubsection{\((\lambda T)^* = \overline{\lambda}  T^*\) for all \(\lambda \in  \mathbb{F}\) and \(T \in  \mathcal{L} (V, W)\)}
\label{sec:orgb12400f}

\subsubsection{\((T^*)^* = T\) for all \(T \in  L(V, W)\)}
\label{sec:org8b067ee}

\subsubsection{\(I^* = I\)}
\label{sec:org0ba84bc}

\subsubsection{\((ST)^* = T*S*\) for all \(T \in \mathcal{L} (V, W)\) and \(S \in  \mathcal{L} (W, U)\) where \(U\) is an inner product space over \(\mathbb{F}\)}
\label{sec:org1ce3227}

\subsection{Axler7.7 null space and range of \(T^*\)}
\label{sec:org33567c9}

Suppose \(T \in  \mathcal{L}(V, W)\). Then,

\subsubsection{\(\onull T^* = (\orange T)^\bot\)}
\label{sec:org98624f3}

\subsubsection{\(\orange T^* = (\onull T)^\bot\)}
\label{sec:orgfd56df7}

\subsubsection{\(\onull T = (\orange T^*)^\bot\)}
\label{sec:org20ec1e0}

\subsubsection{\(\orange T = (\onull T^*)^\bot\)}
\label{sec:org0981e21}
\end{document}
