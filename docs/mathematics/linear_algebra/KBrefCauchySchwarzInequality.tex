% Created 2021-09-11 Sat 16:42
% Intended LaTeX compiler: xelatex
\documentclass[letterpaper]{article}
\usepackage{graphicx}
\usepackage{grffile}
\usepackage{longtable}
\usepackage{wrapfig}
\usepackage{rotating}
\usepackage[normalem]{ulem}
\usepackage{amsmath}
\usepackage{textcomp}
\usepackage{amssymb}
\usepackage{capt-of}
\usepackage{hyperref}
\usepackage[margin=1in]{geometry}
\usepackage{fontspec}
\usepackage{indentfirst}
\setmainfont[ItalicFont = LiberationSans-Italic, BoldFont = LiberationSans-Bold, BoldItalicFont = LiberationSans-BoldItalic]{LiberationSans}
\newfontfamily\NHLight[ItalicFont = LiberationSansNarrow-Italic, BoldFont       = LiberationSansNarrow-Bold, BoldItalicFont = LiberationSansNarrow-BoldItalic]{LiberationSansNarrow}
\newcommand\textrmlf[1]{{\NHLight#1}}
\newcommand\textitlf[1]{{\NHLight\itshape#1}}
\let\textbflf\textrm
\newcommand\textulf[1]{{\NHLight\bfseries#1}}
\newcommand\textuitlf[1]{{\NHLight\bfseries\itshape#1}}
\usepackage{fancyhdr}
\pagestyle{fancy}
\usepackage{titlesec}
\usepackage{titling}
\makeatletter
\lhead{\textbf{\@title}}
\makeatother
\rhead{\textrmlf{Compiled} \today}
\lfoot{\theauthor\ \textbullet \ \textbf{2021-2022}}
\cfoot{}
\rfoot{\textrmlf{Page} \thepage}
\titleformat{\section} {\Large} {\textrmlf{\thesection} {|}} {0.3em} {\textbf}
\titleformat{\subsection} {\large} {\textrmlf{\thesubsection} {|}} {0.2em} {\textbf}
\titleformat{\subsubsection} {\large} {\textrmlf{\thesubsubsection} {|}} {0.1em} {\textbf}
\setlength{\parskip}{0.45em}
\renewcommand\maketitle{}
\author{Taproot}
\date{\today}
\title{Axler6.15 Cauchy-Schwarz Inequality}
\hypersetup{
 pdfauthor={Taproot},
 pdftitle={Axler6.15 Cauchy-Schwarz Inequality},
 pdfkeywords={},
 pdfsubject={},
 pdfcreator={Emacs 27.2 (Org mode 9.4.4)}, 
 pdflang={English}}
\begin{document}

\maketitle
\section{Cauchy-Schwarz Inequality\hfill{}\textsc{important}}
\label{sec:org84a6382}
'One of the most important inequalities in mathematics'
\begin{quote}
Suppose \(u, v \in V\) (where \(V\) is an inner product space). Then
\[\begin{aligned}
  \vert \langle u, v \rangle \vert \leq \lVert u \rVert \lVert v \rVert
  \end{aligned}\]

The inequality is an equality iff one of \(u, v\) is a scalar multiple of the other.
\end{quote}

\subsection{intuition}
\label{sec:orga15ce38}
For the Euclidean inner product, this is true because \(\langle u, v \rangle = \lVert u \rVert \lVert v \rVert \cos \theta\). However, the Cauchy-Schwarz inequality works for all inner product spaces, using the generalized Pythagorean theorem (instead of the law of cosines).
\subsection{proof is by \href{KBrefOrthogonalDecomposition.org}{the orthogonal decomposition}}
\label{sec:orgf8d2c26}

By homogeneity of norms,
\[\begin{aligned}
   \left\lVert \frac{ \langle u, v \rangle }{\lVert v \rVert} v \right\rVert ^2 = \left| \frac{ \langle u, v \rangle }{\lVert v \rVert}\right|^2 \lVert v \rVert ^2
   \end{aligned}\]

\subsection{results}
\label{sec:orge2bd2c8}
\subsubsection{triangle inequality}
\label{sec:orgedeb4d4}
\begin{quote}
Suppose \(u, v \in V\). Then

\[\begin{aligned}
    \lVert u+v \rVert \leq \lVert u \rVert + \lVert v \rVert
	\end{aligned}\]

The inequality is an equality if and only if one of \(u, v\) is a non-negative multiple of the other (degenerate triangle)
\end{quote}
This is proven by noticing that \(\langle u, v \rangle + \langle  v, u \rangle = \langle  u, v \rangle \overline{\langle v, u \rangle} = 2 Re \langle u, v \rangle \leq  2|\langle u, v \rangle| \leq 2\lVert u \rVert \lVert v \rVert\) by conjugate symmetry and Cauchy-Schwarz.

\subsubsection{Parallelogram Equality}
\label{sec:org9cc58cc}
\begin{quote}
Suppose \(u, v \in V\). Then
\[\begin{aligned}
    \lVert u+v \rVert^2 + \lVert u-v \rVert^2 = 2(\lVert u \rVert^2 + \lVert v \rVert^2)
	\end{aligned}\]
\end{quote}
\end{document}
