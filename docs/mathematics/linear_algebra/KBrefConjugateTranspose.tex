% Created 2021-09-12 Sun 22:49
% Intended LaTeX compiler: xelatex
\documentclass[letterpaper]{article}
\usepackage{graphicx}
\usepackage{grffile}
\usepackage{longtable}
\usepackage{wrapfig}
\usepackage{rotating}
\usepackage[normalem]{ulem}
\usepackage{amsmath}
\usepackage{textcomp}
\usepackage{amssymb}
\usepackage{capt-of}
\usepackage{hyperref}
\usepackage[margin=1in]{geometry}
\usepackage{fontspec}
\usepackage{indentfirst}
\setmainfont[ItalicFont = LiberationSans-Italic, BoldFont = LiberationSans-Bold, BoldItalicFont = LiberationSans-BoldItalic]{LiberationSans}
\newfontfamily\NHLight[ItalicFont = LiberationSansNarrow-Italic, BoldFont       = LiberationSansNarrow-Bold, BoldItalicFont = LiberationSansNarrow-BoldItalic]{LiberationSansNarrow}
\newcommand\textrmlf[1]{{\NHLight#1}}
\newcommand\textitlf[1]{{\NHLight\itshape#1}}
\let\textbflf\textrm
\newcommand\textulf[1]{{\NHLight\bfseries#1}}
\newcommand\textuitlf[1]{{\NHLight\bfseries\itshape#1}}
\usepackage{fancyhdr}
\pagestyle{fancy}
\usepackage{titlesec}
\usepackage{titling}
\makeatletter
\lhead{\textbf{\@title}}
\makeatother
\rhead{\textrmlf{Compiled} \today}
\lfoot{\theauthor\ \textbullet \ \textbf{2021-2022}}
\cfoot{}
\rfoot{\textrmlf{Page} \thepage}
\titleformat{\section} {\Large} {\textrmlf{\thesection} {|}} {0.3em} {\textbf}
\titleformat{\subsection} {\large} {\textrmlf{\thesubsection} {|}} {0.2em} {\textbf}
\titleformat{\subsubsection} {\large} {\textrmlf{\thesubsubsection} {|}} {0.1em} {\textbf}
\setlength{\parskip}{0.45em}
\renewcommand\maketitle{}
\author{Taproot}
\date{\today}
\title{Conjugate Transpose}
\hypersetup{
 pdfauthor={Taproot},
 pdftitle={Conjugate Transpose},
 pdfkeywords={},
 pdfsubject={},
 pdfcreator={Emacs 28.0.50 (Org mode 9.4.4)}, 
 pdflang={English}}
\begin{document}

\maketitle
\section{Axler7.8 conjugate transpose\hfill{}\textsc{def}}
\label{sec:orgfdb77c0}
\begin{quote}
The \emph{conjugate transpose} of an \$m\$-by-\(n\) matrix is the \$n\$-by-\(m\) matrix obtained by taking the transpose then the complex conjugate of each entry.
\end{quote}

If \(\mathbb{F} = \mathbb{R}\) then the conjugate transpose is just the transpose.
\section{Axler7.10 The matrix of \(T^*\) (\href{KBrefAdjoints.org}{adjoint})}
\label{sec:org7f57d4b}
\begin{quote}
Let \(T \in  \mathcal{L}(V, W)\). Suppose \(e_1, \ldots, e_n\) is an orthonormal basis of \(V\) and \(f_1, \ldots, f_m\) is an orthonormal basis of \(W\). Then,
\[\begin{aligned}
  \mathcal{M}(T^*, (f_1, \ldots, f_m), (e_1, \ldots, e_n))
  \end{aligned}\]

is the \emph{conjugate transpose} of

\[\begin{aligned}
  \mathcal{M}(T, (e_1, \ldots, e_n), (f_1, \dots, f_m))
  \end{aligned}\]
\end{quote}

However, since \textbf{this only works with orthonormal bases}, Axler decided to focus on adjoints instead of conjugate transposes. (but they are the same thing under orthonormal bases).
\end{document}
