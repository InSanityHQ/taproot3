% Created 2021-09-27 Mon 11:52
% Intended LaTeX compiler: xelatex
\documentclass[letterpaper]{article}
\usepackage{graphicx}
\usepackage{grffile}
\usepackage{longtable}
\usepackage{wrapfig}
\usepackage{rotating}
\usepackage[normalem]{ulem}
\usepackage{amsmath}
\usepackage{textcomp}
\usepackage{amssymb}
\usepackage{capt-of}
\usepackage{hyperref}
\setlength{\parindent}{0pt}
\usepackage[margin=1in]{geometry}
\usepackage{fontspec}
\usepackage{svg}
\usepackage{cancel}
\usepackage{indentfirst}
\setmainfont[ItalicFont = LiberationSans-Italic, BoldFont = LiberationSans-Bold, BoldItalicFont = LiberationSans-BoldItalic]{LiberationSans}
\newfontfamily\NHLight[ItalicFont = LiberationSansNarrow-Italic, BoldFont       = LiberationSansNarrow-Bold, BoldItalicFont = LiberationSansNarrow-BoldItalic]{LiberationSansNarrow}
\newcommand\textrmlf[1]{{\NHLight#1}}
\newcommand\textitlf[1]{{\NHLight\itshape#1}}
\let\textbflf\textrm
\newcommand\textulf[1]{{\NHLight\bfseries#1}}
\newcommand\textuitlf[1]{{\NHLight\bfseries\itshape#1}}
\usepackage{fancyhdr}
\pagestyle{fancy}
\usepackage{titlesec}
\usepackage{titling}
\makeatletter
\lhead{\textbf{\@title}}
\makeatother
\rhead{\textrmlf{Compiled} \today}
\lfoot{\theauthor\ \textbullet \ \textbf{2021-2022}}
\cfoot{}
\rfoot{\textrmlf{Page} \thepage}
\renewcommand{\tableofcontents}{}
\titleformat{\section} {\Large} {\textrmlf{\thesection} {|}} {0.3em} {\textbf}
\titleformat{\subsection} {\large} {\textrmlf{\thesubsection} {|}} {0.2em} {\textbf}
\titleformat{\subsubsection} {\large} {\textrmlf{\thesubsubsection} {|}} {0.1em} {\textbf}
\setlength{\parskip}{0.45em}
\renewcommand\maketitle{}
\author{Exr0n}
\date{\today}
\title{Injectivity (math)}
\hypersetup{
 pdfauthor={Exr0n},
 pdftitle={Injectivity (math)},
 pdfkeywords={},
 pdfsubject={},
 pdfcreator={Emacs 28.0.50 (Org mode 9.4.4)}, 
 pdflang={English}}
\begin{document}

\tableofcontents

\section{In the context of Linear Algebra (Axler 3.15)}
\label{sec:org3a7e52f}
\subsection{\#definition injective\hfill{}\textsc{def}}
\label{sec:orgb6b1d5a}
\begin{quote}
A function \(T : V \to W\) is called \emph{injective} if \(Tu = Tv\) implies \(u = v\)
\end{quote}
\subsection{\#aka one-to-one\hfill{}\textsc{aka}}
\label{sec:org91e39bd}
\subsection{Properties}
\label{sec:org22faa98}
\subsubsection{A map is injective iff it's null space equals \(\{0\}\)}
\label{sec:org064db90}
\subsubsection{A map to a smaller dimensional space is not injective (Axler3.23)}
\label{sec:org4ac36f6}
\begin{quote}
Suppose \(V\) and \(W\) are finite-dimensional vector spaces such that \(\text{dim }V > \text{dim }W\). Then no linear map from \(V\) to \(W\) is injective.
\end{quote}
\begin{enumerate}
\item Intuition
\label{sec:org9c9f1d1}
That makes sense, because if the output space has a smaller dimension, then there should be two inputs that go to the same output somewhere. Otherwise all the inputs just don't "fit".
\end{enumerate}
\subsection{Intuition}
\label{sec:org1c6e6cf}
\(Tu = Tv \implies u = v\) means that if the outputs are the same, then the inputs are the same, aka only one input goes to that one output. That's why it's called "one-to-one": only one input goes to that one output
\end{document}
