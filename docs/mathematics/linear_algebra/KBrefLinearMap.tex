% Created 2021-09-27 Mon 11:52
% Intended LaTeX compiler: xelatex
\documentclass[letterpaper]{article}
\usepackage{graphicx}
\usepackage{grffile}
\usepackage{longtable}
\usepackage{wrapfig}
\usepackage{rotating}
\usepackage[normalem]{ulem}
\usepackage{amsmath}
\usepackage{textcomp}
\usepackage{amssymb}
\usepackage{capt-of}
\usepackage{hyperref}
\setlength{\parindent}{0pt}
\usepackage[margin=1in]{geometry}
\usepackage{fontspec}
\usepackage{svg}
\usepackage{cancel}
\usepackage{indentfirst}
\setmainfont[ItalicFont = LiberationSans-Italic, BoldFont = LiberationSans-Bold, BoldItalicFont = LiberationSans-BoldItalic]{LiberationSans}
\newfontfamily\NHLight[ItalicFont = LiberationSansNarrow-Italic, BoldFont       = LiberationSansNarrow-Bold, BoldItalicFont = LiberationSansNarrow-BoldItalic]{LiberationSansNarrow}
\newcommand\textrmlf[1]{{\NHLight#1}}
\newcommand\textitlf[1]{{\NHLight\itshape#1}}
\let\textbflf\textrm
\newcommand\textulf[1]{{\NHLight\bfseries#1}}
\newcommand\textuitlf[1]{{\NHLight\bfseries\itshape#1}}
\usepackage{fancyhdr}
\pagestyle{fancy}
\usepackage{titlesec}
\usepackage{titling}
\makeatletter
\lhead{\textbf{\@title}}
\makeatother
\rhead{\textrmlf{Compiled} \today}
\lfoot{\theauthor\ \textbullet \ \textbf{2021-2022}}
\cfoot{}
\rfoot{\textrmlf{Page} \thepage}
\renewcommand{\tableofcontents}{}
\titleformat{\section} {\Large} {\textrmlf{\thesection} {|}} {0.3em} {\textbf}
\titleformat{\subsection} {\large} {\textrmlf{\thesubsection} {|}} {0.2em} {\textbf}
\titleformat{\subsubsection} {\large} {\textrmlf{\thesubsubsection} {|}} {0.1em} {\textbf}
\setlength{\parskip}{0.45em}
\renewcommand\maketitle{}
\author{Exr0n}
\date{\today}
\title{linear map}
\hypersetup{
 pdfauthor={Exr0n},
 pdftitle={linear map},
 pdfkeywords={},
 pdfsubject={},
 pdfcreator={Emacs 28.0.50 (Org mode 9.4.4)}, 
 pdflang={English}}
\begin{document}

\tableofcontents


\section{Definition}
\label{sec:org472167d}
\#definition Axler3.2 Linear Map
\#aka linear transformation
A \emph{linear map} from \(V\) to \(W\) is a function \(T : V \to W\) with the following properties:
\subsection{Additivity}
\label{sec:org539df96}
$$T(u+v) = Tu + Tv \forall u, v \in V$$
\subsection{Homogenity}
\label{sec:orgf3a89c4}
$$T(\lambda v) = \lambda(T v) \forall \lambda \in \mathbb{F}, v\in V$$
\section{Other Notation}
\label{sec:org4f09307}
\subsection{Set of Maps}
\label{sec:org2714ded}
\#definition Axler3.3 \(\mathcal{L}(V, W)\)
\begin{quote}
The set of all linear maps from \(V\) to \(W\) is denoted \(\mathcal{L}(V, W)\).
\end{quote}
\section{Examples}
\label{sec:org3c18571}
\subsection{zero (\(0\))}
\label{sec:org2353896}
Zero is a function \(0 : V \to W\) s.t. \(0v = 0 \forall v \in V\). (It takes all vectors in \(V\) and maps them to the additive identity of \(W\))
\subsection{identity (\(I\))}
\label{sec:orge15fa34}
The identity maps each from one vector space to itself (in the same vector space):
$$I \in \mathcal{L}(V, V), v\in V : Iv = v$$
\subsection{differentiation (\(D\))}
\label{sec:org34bfe05}
$$D \in \mathcal{L}\left(\mathcal{P}(\mathbb{R}), \mathcal{P}(\mathbb{R})\right) : Dp = p'$$
Basically stating that for two polynomials \(a, b \in \mathcal{P}(\mathbb{R})\), \(a'+b' = (a+b)'\) and with a constant \(\lambda \in \mathcal{R}\) \((\lambda a)' = \lambda a'\).
\subsection{integration}
\label{sec:org41fa46f}
\subsection{multiplication by \(x^2\)}
\label{sec:org1b984df}
$$T \in \mathcal{L}\left(\mathcal{P}(\mathbb{R}), \mathcal{P}(\mathbb{R})\right) : (Tp)(x) = x^2p(x)$$
 is a linear map
\subsection{backward shift}
\label{sec:org93e603c}
\(F^\infty\) is the vector space of all sequences of elements in \(\mathbb{F}\).
$$T \in \mathcal{L}\left(\mathbb{F}^\infty, \mathbb{F}^\infty\right) : T(x_1, x_2, x_3, \ldots) = (x_2, x_3, \ldots)$$
\subsection{\(\mathbb{F}^n \to \mathbb{F}^m\)}
\label{sec:org5b5740b}
Given a "coefficent matrix" \(A : A_{j,k}\in\mathbb{F} \forall j=1,\ldots,m; \forall k=1,\ldots,n\), define \(T \in \mathcal{L}(\mathbb{F}^n, \mathbb{F}^m)\):
$$T(x_1, \ldots, x_n) = (A_{1,1}x_1 + A_{1,2}x_2 + \cdots + A_{1,n}x_n,\ A_{2,1}x_1 + \cdots + A_{2, n}x_n,\ \ldots,\ A_{m, 1}x_1 + \cdots + A_{m, n} x_n)$$
Notice that this is equivalent to taking \(A\) as a \(m\times n\) matrix and dot producting it with the \(n \times 1\) matrix \(\begin{bmatrix}x_1 \\ x_2 \\ \vdots \\ x_n\end{bmatrix}\).
\section{Results}
\label{sec:orgfd0c0e7}
\subsection{Axler3.5 Linear maps and basis of domain}
\label{sec:orgede23e7}
If \(v_1, \ldots, v_n\) is a basis of \(V\) and \(w_1, \ldots, w_n \in W\), then there exists a unique linear map \(T : V\to W\) s.t.
$$T v_j = w_j \forall j \in 1, \ldots, n$$
\#aka given a basis \(v\) of \(V\), there is a unique linear map that maps \(v\) to each \(w \in W\).
\subsubsection{\#careful}
\label{sec:org9b94941}
\begin{enumerate}
\item same dimension
\label{sec:org658b0f1}
\(V\) and \(W\) are both of dimension \(n\).
\item same field
\label{sec:org5c001ab}
We defined \(V\) and \(W\) to both be vector spaces over the same field \(\mathbb{F}\) which is either \(\mathbb{R}\) or \(\mathbb{C}\).
\item \(v\) is a basis
\label{sec:orgc4b5865}
\(v_1, \ldots, v_n\) must be a basis of \(V\) (because that fact is used in the proof)
\end{enumerate}
\subsubsection{Questions}
\label{sec:org9c53d91}
\begin{enumerate}
\item {\bfseries\sffamily DONE} \#question what does it mean that "\(T\) is uniquely determined on \(\text{span}(v_1, \ldots, v_n)\)?\hfill{}\textsc{question}
\label{sec:org00f47e3}
There's no ambiguity and so we know exactly which map it's refering to, and thus it is uniquely determined.
\end{enumerate}
\end{document}
