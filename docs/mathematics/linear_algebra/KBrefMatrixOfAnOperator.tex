% Created 2021-09-27 Mon 11:52
% Intended LaTeX compiler: xelatex
\documentclass[letterpaper]{article}
\usepackage{graphicx}
\usepackage{grffile}
\usepackage{longtable}
\usepackage{wrapfig}
\usepackage{rotating}
\usepackage[normalem]{ulem}
\usepackage{amsmath}
\usepackage{textcomp}
\usepackage{amssymb}
\usepackage{capt-of}
\usepackage{hyperref}
\setlength{\parindent}{0pt}
\usepackage[margin=1in]{geometry}
\usepackage{fontspec}
\usepackage{svg}
\usepackage{cancel}
\usepackage{indentfirst}
\setmainfont[ItalicFont = LiberationSans-Italic, BoldFont = LiberationSans-Bold, BoldItalicFont = LiberationSans-BoldItalic]{LiberationSans}
\newfontfamily\NHLight[ItalicFont = LiberationSansNarrow-Italic, BoldFont       = LiberationSansNarrow-Bold, BoldItalicFont = LiberationSansNarrow-BoldItalic]{LiberationSansNarrow}
\newcommand\textrmlf[1]{{\NHLight#1}}
\newcommand\textitlf[1]{{\NHLight\itshape#1}}
\let\textbflf\textrm
\newcommand\textulf[1]{{\NHLight\bfseries#1}}
\newcommand\textuitlf[1]{{\NHLight\bfseries\itshape#1}}
\usepackage{fancyhdr}
\pagestyle{fancy}
\usepackage{titlesec}
\usepackage{titling}
\makeatletter
\lhead{\textbf{\@title}}
\makeatother
\rhead{\textrmlf{Compiled} \today}
\lfoot{\theauthor\ \textbullet \ \textbf{2021-2022}}
\cfoot{}
\rfoot{\textrmlf{Page} \thepage}
\renewcommand{\tableofcontents}{}
\titleformat{\section} {\Large} {\textrmlf{\thesection} {|}} {0.3em} {\textbf}
\titleformat{\subsection} {\large} {\textrmlf{\thesubsection} {|}} {0.2em} {\textbf}
\titleformat{\subsubsection} {\large} {\textrmlf{\thesubsubsection} {|}} {0.1em} {\textbf}
\setlength{\parskip}{0.45em}
\renewcommand\maketitle{}
\author{Taproot}
\date{\today}
\title{Axler5.22 Matrix of an Operator}
\hypersetup{
 pdfauthor={Taproot},
 pdftitle={Axler5.22 Matrix of an Operator},
 pdfkeywords={},
 pdfsubject={},
 pdfcreator={Emacs 28.0.50 (Org mode 9.4.4)}, 
 pdflang={English}}
\begin{document}

\tableofcontents

\section{Axler5.22 matrix of an operator, \(\mathcal{M} (T)\)\hfill{}\textsc{def}}
\label{sec:orgcc90e5e}
\begin{quote}
Suppose \(T \in  \mathcal{L} (V)\) and \(v_1, \ldots, v_n\) is a basis of \(V\). The \emph{matrix of \(T\)} wrt this basis is the \emph{n}-by-\emph{n} matrix
\[\begin{aligned}
  \mathcal{M} (T) = \begin{pmatrix}A_{1,1} & \cdots & A_{1, n} \\ \vdots & \ddots & \vdots \\ A_{n, 1} & \cdots & A_{n, n} \end{pmatrix}
  \end{aligned}\]
whose entries \(A_{j, k}\) are defined by
\[\begin{aligned}
  Tv_k = A_{1, k}v_1 + \cdots + A_{n,k}v_n
  \end{aligned}\]

Specify a basis with \(\mathcal{M} \left( T, (v_1, \ldots, v_n) \right)\)
\end{quote}
\subsection{intuition}
\label{sec:org8adf390}
\subsubsection{each column is where the map takes a basis vector}
\label{sec:org2dfaa89}
\section{Simplifying The Matrix Representation}
\label{sec:org296247e}
\subsection{'A central goal of linear algebra is to show that given an operator \(T \in  \mathcal{L} (V)\), there exists a basis of \(V\) wrt which \(T\) has a reasonably simple matrix'}
\label{sec:orgac97809}
\subsection{If by simple we mean "has many zeros" or RREF, then we know enough to ensure that there exists a basis s.t. the first column has zeros everywhere except the first row.}
\label{sec:orgede357a}

$\backslash$[\begin{aligned}
\begin{pmatrix}\lambda &&&\\0&*&&\\\vdots&&&\\0&&&\end{pmatrix}
\end{aligned}$\backslash$]
Where \(*\) denotes all the other entries. Find \(\lambda\) by taking the lone eigenvalue and letting it's eigenvector be the first basis vector.
\end{document}
