% Created 2021-09-12 Sun 22:49
% Intended LaTeX compiler: xelatex
\documentclass[letterpaper]{article}
\usepackage{graphicx}
\usepackage{grffile}
\usepackage{longtable}
\usepackage{wrapfig}
\usepackage{rotating}
\usepackage[normalem]{ulem}
\usepackage{amsmath}
\usepackage{textcomp}
\usepackage{amssymb}
\usepackage{capt-of}
\usepackage{hyperref}
\usepackage[margin=1in]{geometry}
\usepackage{fontspec}
\usepackage{indentfirst}
\setmainfont[ItalicFont = LiberationSans-Italic, BoldFont = LiberationSans-Bold, BoldItalicFont = LiberationSans-BoldItalic]{LiberationSans}
\newfontfamily\NHLight[ItalicFont = LiberationSansNarrow-Italic, BoldFont       = LiberationSansNarrow-Bold, BoldItalicFont = LiberationSansNarrow-BoldItalic]{LiberationSansNarrow}
\newcommand\textrmlf[1]{{\NHLight#1}}
\newcommand\textitlf[1]{{\NHLight\itshape#1}}
\let\textbflf\textrm
\newcommand\textulf[1]{{\NHLight\bfseries#1}}
\newcommand\textuitlf[1]{{\NHLight\bfseries\itshape#1}}
\usepackage{fancyhdr}
\pagestyle{fancy}
\usepackage{titlesec}
\usepackage{titling}
\makeatletter
\lhead{\textbf{\@title}}
\makeatother
\rhead{\textrmlf{Compiled} \today}
\lfoot{\theauthor\ \textbullet \ \textbf{2021-2022}}
\cfoot{}
\rfoot{\textrmlf{Page} \thepage}
\titleformat{\section} {\Large} {\textrmlf{\thesection} {|}} {0.3em} {\textbf}
\titleformat{\subsection} {\large} {\textrmlf{\thesubsection} {|}} {0.2em} {\textbf}
\titleformat{\subsubsection} {\large} {\textrmlf{\thesubsubsection} {|}} {0.1em} {\textbf}
\setlength{\parskip}{0.45em}
\renewcommand\maketitle{}
\author{Taproot}
\date{\today}
\title{Orthogonal Projection}
\hypersetup{
 pdfauthor={Taproot},
 pdftitle={Orthogonal Projection},
 pdfkeywords={},
 pdfsubject={},
 pdfcreator={Emacs 28.0.50 (Org mode 9.4.4)}, 
 pdflang={English}}
\begin{document}

\maketitle
\section{Axler6.53 orthogonal projection, \(P_U\)\hfill{}\textsc{def}}
\label{sec:orgecdb57e}
\begin{quote}
Suppose \(U\) is a finite-dimensional subspace of \(V\). The \emph{orthogonal projection} of \(V\) onto \(U\) is the operator \(P_U \in\mathcal{L} (V)\) defined as follows:

For \(v \in  V\), write \(v = u + w\), where \(u \in  U\) and \(w \in  U^\bot\). Then \(P_Uv = u\).
\end{quote}
In other words, \(P_U \in \mathcal{L} (V)\) takes \(v\) to the component of \(v\) that is in \(U\).

This concept is closely related to the \href{KBrefOrthogonalDecomposition.org}{Orthogonal Decomposition}
\subsection{Results}
\label{sec:org3a7abb8}
\subsubsection{Axler6.54 calculating \(P_U v\)}
\label{sec:org62ce1ed}

\[\begin{aligned}
    P_U v = \frac{\langle  v, x \rangle}{\lVert x \rVert ^2} x
	\end{aligned}\]

Because orthogonal decompositions and stuff
\subsubsection{Axler6.55 properties}
\label{sec:org91a357a}
Suppose \(U\) is a finite-dimensional subspace of \(V\) and \(v \in  V\). Then,
\begin{enumerate}
\item \(P_U \in \mathcal{L}(V)\)
\label{sec:org4a675d5}
\item \(P_U u = u \forall u \in  U\)
\label{sec:orgfc5394b}
\item \(P_U w = 0 \forall w \in  U^\bot\)
\label{sec:org53ed55b}
\item \(\orange P_U = U\)
\label{sec:orgf171a71}
\item \(\onull P_U = U^\bot\)
\label{sec:org0a694e6}
\item \(P_U ^2 = P_U\) (by $\backslash$#2 and $\backslash$#4)
\label{sec:org2111721}
\item \(\lVert P_U v \rVert \leq  \lVert v \rVert\)
\label{sec:org1bada39}
\item for every orthonormal basis \(e_1, \ldots, e_m\) of \(U\),
\label{sec:orgab3539e}

\[\begin{aligned}
     P_U v = \langle  v, e_1 \rangle e_1, + \cdots + \langle v, e_m \rangle e_m
	 \end{aligned}\]

(because \(P_U v \in  U\))
\end{enumerate}
\subsubsection{Axler6.56 Minimizing the distance to a subspace}
\label{sec:org26a2046}
See \href{KBrefMinimizingDistanceToSubspace.org}{Minimizing the distance to a subpsace}
\end{document}
