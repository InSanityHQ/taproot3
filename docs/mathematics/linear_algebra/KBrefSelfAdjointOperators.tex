% Created 2021-09-11 Sat 16:42
% Intended LaTeX compiler: xelatex
\documentclass[letterpaper]{article}
\usepackage{graphicx}
\usepackage{grffile}
\usepackage{longtable}
\usepackage{wrapfig}
\usepackage{rotating}
\usepackage[normalem]{ulem}
\usepackage{amsmath}
\usepackage{textcomp}
\usepackage{amssymb}
\usepackage{capt-of}
\usepackage{hyperref}
\usepackage[margin=1in]{geometry}
\usepackage{fontspec}
\usepackage{indentfirst}
\setmainfont[ItalicFont = LiberationSans-Italic, BoldFont = LiberationSans-Bold, BoldItalicFont = LiberationSans-BoldItalic]{LiberationSans}
\newfontfamily\NHLight[ItalicFont = LiberationSansNarrow-Italic, BoldFont       = LiberationSansNarrow-Bold, BoldItalicFont = LiberationSansNarrow-BoldItalic]{LiberationSansNarrow}
\newcommand\textrmlf[1]{{\NHLight#1}}
\newcommand\textitlf[1]{{\NHLight\itshape#1}}
\let\textbflf\textrm
\newcommand\textulf[1]{{\NHLight\bfseries#1}}
\newcommand\textuitlf[1]{{\NHLight\bfseries\itshape#1}}
\usepackage{fancyhdr}
\pagestyle{fancy}
\usepackage{titlesec}
\usepackage{titling}
\makeatletter
\lhead{\textbf{\@title}}
\makeatother
\rhead{\textrmlf{Compiled} \today}
\lfoot{\theauthor\ \textbullet \ \textbf{2021-2022}}
\cfoot{}
\rfoot{\textrmlf{Page} \thepage}
\titleformat{\section} {\Large} {\textrmlf{\thesection} {|}} {0.3em} {\textbf}
\titleformat{\subsection} {\large} {\textrmlf{\thesubsection} {|}} {0.2em} {\textbf}
\titleformat{\subsubsection} {\large} {\textrmlf{\thesubsubsection} {|}} {0.1em} {\textbf}
\setlength{\parskip}{0.45em}
\renewcommand\maketitle{}
\author{Taproot}
\date{\today}
\title{Self-adjoint operators}
\hypersetup{
 pdfauthor={Taproot},
 pdftitle={Self-adjoint operators},
 pdfkeywords={},
 pdfsubject={},
 pdfcreator={Emacs 27.2 (Org mode 9.4.4)}, 
 pdflang={English}}
\begin{document}

\maketitle
\section{Axler7.11 self-adjoint, Hermitian\hfill{}\textsc{def}}
\label{sec:org45a6286}
\begin{quote}
An operator \(T \in  \mathcal{L}(V)\) is called \emph{self-adjoint} if \(T = T^*\) aka it is \href{KBrefAdjoints.org}{adjoint} to itself. aka: \(T \in  \mathcal{L} (V)\) is self-adjoint iff
\[\begin{aligned}
   \langle Tv, w \rangle = \langle v, Tw \rangle
  \end{aligned}\]
\end{quote}
Because adjoint-ness is in some ways analygous to complex conjugation, a self-adjoint operator is somewhat analygous to real numbers (kinda like a number who equals its conjugates real, a map that equals its adjoint is "real")
\section{results}
\label{sec:org1f432fa}
\subsection{Axler7.13 Eigenvalues of self-adjoint operators are real}
\label{sec:org058d366}
\begin{quote}
Every eigenvalue of a self-adjoint operator is real.
\end{quote}
\subsection{Axler7.14 Over \(\mathbb{C}\), only the \(0\) operator has \(Tv\) being orthogonal to \(v\) for all \(v\)}
\label{sec:orgfa057d5}
For some \textbf{complex} vector space \(V\) and \(T \in  \mathcal{L}(V)\), if
\[\begin{aligned}
    \langle Tv, v \rangle = 0
   \end{aligned}\]
for all \(v \in  V\), then \(T = 0\).
\subsection{{\bfseries\sffamily TODO} Axler7.15 and Axler7.16??}
\label{sec:orga270c29}
\subsection{Every self-adjoint operator is \href{KBrefNormal.org}{normal}.}
\label{sec:org800487c}
\end{document}
