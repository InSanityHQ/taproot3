% Created 2021-09-27 Mon 12:03
% Intended LaTeX compiler: xelatex
\documentclass[letterpaper]{article}
\usepackage{graphicx}
\usepackage{grffile}
\usepackage{longtable}
\usepackage{wrapfig}
\usepackage{rotating}
\usepackage[normalem]{ulem}
\usepackage{amsmath}
\usepackage{textcomp}
\usepackage{amssymb}
\usepackage{capt-of}
\usepackage{hyperref}
\setlength{\parindent}{0pt}
\usepackage[margin=1in]{geometry}
\usepackage{fontspec}
\usepackage{svg}
\usepackage{cancel}
\usepackage{indentfirst}
\setmainfont[ItalicFont = LiberationSans-Italic, BoldFont = LiberationSans-Bold, BoldItalicFont = LiberationSans-BoldItalic]{LiberationSans}
\newfontfamily\NHLight[ItalicFont = LiberationSansNarrow-Italic, BoldFont       = LiberationSansNarrow-Bold, BoldItalicFont = LiberationSansNarrow-BoldItalic]{LiberationSansNarrow}
\newcommand\textrmlf[1]{{\NHLight#1}}
\newcommand\textitlf[1]{{\NHLight\itshape#1}}
\let\textbflf\textrm
\newcommand\textulf[1]{{\NHLight\bfseries#1}}
\newcommand\textuitlf[1]{{\NHLight\bfseries\itshape#1}}
\usepackage{fancyhdr}
\pagestyle{fancy}
\usepackage{titlesec}
\usepackage{titling}
\makeatletter
\lhead{\textbf{\@title}}
\makeatother
\rhead{\textrmlf{Compiled} \today}
\lfoot{\theauthor\ \textbullet \ \textbf{2021-2022}}
\cfoot{}
\rfoot{\textrmlf{Page} \thepage}
\renewcommand{\tableofcontents}{}
\titleformat{\section} {\Large} {\textrmlf{\thesection} {|}} {0.3em} {\textbf}
\titleformat{\subsection} {\large} {\textrmlf{\thesubsection} {|}} {0.2em} {\textbf}
\titleformat{\subsubsection} {\large} {\textrmlf{\thesubsubsection} {|}} {0.1em} {\textbf}
\setlength{\parskip}{0.45em}
\renewcommand\maketitle{}
\author{Huxley Marvit}
\date{\today}
\title{Chapter 1.B}
\hypersetup{
 pdfauthor={Huxley Marvit},
 pdftitle={Chapter 1.B},
 pdfkeywords={},
 pdfsubject={},
 pdfcreator={Emacs 28.0.50 (Org mode 9.4.4)}, 
 pdflang={English}}
\begin{document}

\tableofcontents

\#flo \#ref \#hw

\noindent\rule{\textwidth}{0.5pt}

\section{def of a vector space}
\label{sec:org2095523}
\begin{itemize}
\item \textbf{Props of addition and scalar multiplication in F\textsuperscript{N}}

\begin{itemize}
\item +: comutative, associative, identiy

\begin{itemize}
\item every element has an additive inverse
\end{itemize}

\item *: associative, identity
\item addition and scalar multiplication, connected by distributive props
\end{itemize}

\item let \emph{V} be a set with an addition and scalar multiplication that
satisfy the props,
\end{itemize}

\begin{verbatim}
**addition, scalar multiplication**
- addition: assigns an element u+v in V to each pair of elements u, v in V
- scalar multiplication: lv with l in f and v in V
\end{verbatim}

\begin{verbatim}
**vector space**
is V with addition and SCMUL with:

- commutativitity
- associativity
- additive idenitity
- additive inverse
- multiplicative identity
- distibutive properties
\end{verbatim}

\begin{itemize}
\item no multiplicative inverse?

\begin{itemize}
\item is this how you solve the 0 issue?
\end{itemize}

\item vec, point

\begin{itemize}
\item elements of vec space are called vecs or points
\end{itemize}

\item simplest vec space: \(\{0\}\)
\item f\textsuperscript{infin} is the set of all seqencues of elements of F

\begin{itemize}
\item additive identity: seqnece of all zeros
\end{itemize}

\item vector space can include a set of functions? not quite..

\begin{itemize}
\item let S be a set, and F\textsuperscript{S} be the set of functions from S to F
\item what?? \#review
\end{itemize}

\item let S be the interval [0,1] and F=R

\begin{itemize}
\item R\^{}[$\backslash$0, $\backslash$1] is the set of real valued function on the interval [0,1]
\item ??
\end{itemize}

\item F\textsuperscript{N} -> F\textsuperscript{1,2,\ldots{},n}
\item F\textsuperscript{infin} -> F\textsuperscript{1,2,\ldots{}}
\item vector spaces need unique additive inverse

\begin{itemize}
\item 0'=0'+0=0+0'=0

\begin{itemize}
\item nicer than my proof
\end{itemize}
\end{itemize}

\item unique additive inverse

\begin{itemize}
\item w=w+0=w+(v+w')=(w+v)=(w+v)+w'=0+w'=w'
\end{itemize}
\end{itemize}

\begin{verbatim}
V denotes a vector space over F
\end{verbatim}

\begin{verbatim}
1. no multiplicative inverse required?
2. what does the set of functions from S to F mean?
\end{verbatim}

\subsection{exercises}
\label{sec:org3ca2e10}
\begin{enumerate}
\item prove that -(-v) = v

\begin{enumerate}
\item -(-v) = -1(-1v) = (-1*(-1))v = 1v = v
\end{enumerate}

\item ab = 0, prove that a or b = 0

\begin{enumerate}
\item a=0/v = 0, v=0/a = 0
\end{enumerate}

\item empty set is not a vector space, it fails to satisfy only of the
reqs. which one?

\begin{enumerate}
\item no additive idenity

\begin{enumerate}
\item "there exists an element 0 in v" no there doesn't.
\end{enumerate}
\end{enumerate}
\end{enumerate}

homework: \href{KBxSolvingSystems.org}{KBxSolvingSystems}
\end{document}
