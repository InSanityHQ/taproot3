% Created 2021-09-12 Sun 22:49
% Intended LaTeX compiler: xelatex
\documentclass[letterpaper]{article}
\usepackage{graphicx}
\usepackage{grffile}
\usepackage{longtable}
\usepackage{wrapfig}
\usepackage{rotating}
\usepackage[normalem]{ulem}
\usepackage{amsmath}
\usepackage{textcomp}
\usepackage{amssymb}
\usepackage{capt-of}
\usepackage{hyperref}
\usepackage[margin=1in]{geometry}
\usepackage{fontspec}
\usepackage{indentfirst}
\setmainfont[ItalicFont = LiberationSans-Italic, BoldFont = LiberationSans-Bold, BoldItalicFont = LiberationSans-BoldItalic]{LiberationSans}
\newfontfamily\NHLight[ItalicFont = LiberationSansNarrow-Italic, BoldFont       = LiberationSansNarrow-Bold, BoldItalicFont = LiberationSansNarrow-BoldItalic]{LiberationSansNarrow}
\newcommand\textrmlf[1]{{\NHLight#1}}
\newcommand\textitlf[1]{{\NHLight\itshape#1}}
\let\textbflf\textrm
\newcommand\textulf[1]{{\NHLight\bfseries#1}}
\newcommand\textuitlf[1]{{\NHLight\bfseries\itshape#1}}
\usepackage{fancyhdr}
\pagestyle{fancy}
\usepackage{titlesec}
\usepackage{titling}
\makeatletter
\lhead{\textbf{\@title}}
\makeatother
\rhead{\textrmlf{Compiled} \today}
\lfoot{\theauthor\ \textbullet \ \textbf{2021-2022}}
\cfoot{}
\rfoot{\textrmlf{Page} \thepage}
\titleformat{\section} {\Large} {\textrmlf{\thesection} {|}} {0.3em} {\textbf}
\titleformat{\subsection} {\large} {\textrmlf{\thesubsection} {|}} {0.2em} {\textbf}
\titleformat{\subsubsection} {\large} {\textrmlf{\thesubsubsection} {|}} {0.1em} {\textbf}
\setlength{\parskip}{0.45em}
\renewcommand\maketitle{}
\author{Huxley Marvit}
\date{\today}
\title{Introduction to Groups and Matrices}
\hypersetup{
 pdfauthor={Huxley Marvit},
 pdftitle={Introduction to Groups and Matrices},
 pdfkeywords={},
 pdfsubject={},
 pdfcreator={Emacs 28.0.50 (Org mode 9.4.4)}, 
 pdflang={English}}
\begin{document}

\maketitle
\#ref \#ret

\noindent\rule{\textwidth}{0.5pt}

\section{[[\url{https://nuevaschool.instructure.com/courses/3718/assignments/61005}][The}
\label{sec:org61c8b9d}
Assignment]]
:CUSTOM\textsubscript{ID}: the-assignment
\href{Screen Shot 2021-08-28 at 6.12.09 PM.png.org}{Screen Shot
2021-08-28 at 6.12.09 PM.png} \href{8-28-21, 9:32 PM Microsoft Lens.pdf.org}{8-28-21, 9:32 PM Microsoft Lens.pdf}

\begin{verbatim}
For some reason the pdf isn't rendering, so I'll attach it as an additional file. Apologies for the handwriting..
\end{verbatim}

\subsubsection{\textbf{Tell us why you decided to sign up for this class.}}
\label{sec:orgf39a1f8}
I spend most of my free time doing programming projects with my friends,
and recently I've been doing a lot of Machine Learning. My 'excuse' for
taking linear algebra is that in my more recent ML projects I've had to
go lower level and I'm being held back by my understanding of linear
algebra and statistics, but frankly linear algebra just sounds really
cool and I enjoy Nueva math classes a lot.

\subsubsection{*read 1.a from the textbook. we will discuss any questions on}
\label{sec:orgc2efdb1}
monday! *
:CUSTOM\textsubscript{ID}: read-1.a-from-the-textbook.-we-will-discuss-any-questions-on-monday
q: is division also a lie? yes! q: do tuples all need the same type? why
call them n-tuples instead of lists? q: why is it called liniear
algebra? the explanation that it doesnt deal with geo doesnt explain it

\subsubsection{*Which of the number systems we discussed are groups under addition?}
\label{sec:orgb6eff0e}
Under multiplication?*
:CUSTOM\textsubscript{ID}: which-of-the-number-systems-we-discussed-are-groups-under-addition-under-multiplication

\begin{itemize}
\item natural

\begin{itemize}
\item + N: no identity
\item * N: no inverse
\end{itemize}

\item whole

\begin{itemize}
\item + N: no inverse
\item * N: no inverse
\end{itemize}

\item integers

\begin{itemize}
\item + Y
\item * N: no inverse
\end{itemize}

\item rational

\begin{itemize}
\item + Y
\item * Y : wrong! no inverse for 0. ish\ldots{}
\end{itemize}

\item real

\begin{itemize}
\item + Y
\item * Y
\end{itemize}

\item complex

\begin{itemize}
\item + Y
\item * Y missing zero!
\end{itemize}
\end{itemize}

\subsubsection{*Is there a multiplication for 3 by 1 vectors that satisfies all}
\label{sec:org6150983}
group requirements?*
:CUSTOM\textsubscript{ID}: is-there-a-multiplication-for-3-by-1-vectors-that-satisfies-all-group-requirements
Y -- Multiply equal indices.

E.g. \$
\begin{bmatrix} 1 & 2 & 3 \end{bmatrix}
\$ \$
\begin{bmatrix} 2 & 2 & 2 \end{bmatrix}
=
\begin{bmatrix} 1*2 & 2*2 & 3*2 \end{bmatrix}
=
\begin{bmatrix} 2 & 4 & 6 \end{bmatrix}
\$

\subsubsection{\textbf{Is there an identity for multiplication on 2 by 2 matrices?}}
\label{sec:orgf89a4cd}
Y -- \$
\begin{bmatrix} 1 & 0 \\ 0 & 1  \end{bmatrix}
\$

0s cancel the multiplications where the operation index doesn't match
the sum of the matrix indices.

expand to 1s along the diagonal
\end{document}
