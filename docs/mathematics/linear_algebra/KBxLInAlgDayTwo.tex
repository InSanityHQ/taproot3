% Created 2021-09-27 Mon 12:03
% Intended LaTeX compiler: xelatex
\documentclass[letterpaper]{article}
\usepackage{graphicx}
\usepackage{grffile}
\usepackage{longtable}
\usepackage{wrapfig}
\usepackage{rotating}
\usepackage[normalem]{ulem}
\usepackage{amsmath}
\usepackage{textcomp}
\usepackage{amssymb}
\usepackage{capt-of}
\usepackage{hyperref}
\setlength{\parindent}{0pt}
\usepackage[margin=1in]{geometry}
\usepackage{fontspec}
\usepackage{svg}
\usepackage{cancel}
\usepackage{indentfirst}
\setmainfont[ItalicFont = LiberationSans-Italic, BoldFont = LiberationSans-Bold, BoldItalicFont = LiberationSans-BoldItalic]{LiberationSans}
\newfontfamily\NHLight[ItalicFont = LiberationSansNarrow-Italic, BoldFont       = LiberationSansNarrow-Bold, BoldItalicFont = LiberationSansNarrow-BoldItalic]{LiberationSansNarrow}
\newcommand\textrmlf[1]{{\NHLight#1}}
\newcommand\textitlf[1]{{\NHLight\itshape#1}}
\let\textbflf\textrm
\newcommand\textulf[1]{{\NHLight\bfseries#1}}
\newcommand\textuitlf[1]{{\NHLight\bfseries\itshape#1}}
\usepackage{fancyhdr}
\pagestyle{fancy}
\usepackage{titlesec}
\usepackage{titling}
\makeatletter
\lhead{\textbf{\@title}}
\makeatother
\rhead{\textrmlf{Compiled} \today}
\lfoot{\theauthor\ \textbullet \ \textbf{2021-2022}}
\cfoot{}
\rfoot{\textrmlf{Page} \thepage}
\renewcommand{\tableofcontents}{}
\titleformat{\section} {\Large} {\textrmlf{\thesection} {|}} {0.3em} {\textbf}
\titleformat{\subsection} {\large} {\textrmlf{\thesubsection} {|}} {0.2em} {\textbf}
\titleformat{\subsubsection} {\large} {\textrmlf{\thesubsubsection} {|}} {0.1em} {\textbf}
\setlength{\parskip}{0.45em}
\renewcommand\maketitle{}
\date{\today}
\title{}
\hypersetup{
 pdfauthor={},
 pdftitle={},
 pdfkeywords={},
 pdfsubject={},
 pdfcreator={Emacs 28.0.50 (Org mode 9.4.4)}, 
 pdflang={English}}
\begin{document}

\tableofcontents



\section{Types of Numbers}
\label{sec:org463ab96}
algebra:

\begin{verbatim}
algebra is doing stuff to things
\end{verbatim}

\begin{itemize}
\item idea of a number changes -- 500yago they didnt know about negs
\end{itemize}

natural numbers are the most natural, apparently 0 not in natural, 0 in
whole

\(\mathbb{z}\) for integers, counting in german

rational numbers: a/b \(a,b \mathbb{e} \mathbb{z}\)

real numbers: infinite all the way down way more real numbers than
rational numbers

\begin{itemize}
\item Zero: important for groups -- starting point on number lines. true
neutral, \textbf{Additive Identity}

\begin{itemize}
\item \textbf{Multiplicative Identity}: 1
\item identity lets it keep it's identity? when the op doesn't change
\end{itemize}

\item negs: so we can deal with negs? so we can undo addition
\end{itemize}

\begin{verbatim}
subtraction is a lie! add negs
subtraction on the natural numbers is not closed
\end{verbatim}

\begin{verbatim}
closed: can't make a number not in the set
\end{verbatim}

\section{Groups}
\label{sec:org5e58754}
\begin{verbatim}
any set of mathematical elemements under one operation such that there is an identity each element has an inverse
\end{verbatim}

\begin{itemize}
\item they do not need to be \textbf{communitive}

\begin{itemize}
\item a+b = b+a
\end{itemize}

\item \textbf{associativity}

\begin{itemize}
\item (a+b)+c=a+(b+c)
\item order doesnt matter
\item most things we are doing will be associative
\item nice number systems are almost always associative
\end{itemize}
\end{itemize}

can add dimensions, like complex adding more leads to quaternions or
hamiltonians, then to sadonians?

\begin{verbatim}
called the cayley dickson construction, or smt
\end{verbatim}

\subsubsection{axioms:}
\label{sec:orgd530c32}
\begin{itemize}
\item there exists an identity
\item each element has an inverse
\item it's closed
\item associativity
\end{itemize}

\section{Matrices}
\label{sec:orgc80c0fa}
\begin{itemize}
\item can be called an array
\item 2d can use rows and columns as coords
\end{itemize}

\subsubsection{operations:}
\label{sec:org4fc93fd}
addition: only if same dimensions, loop through indicies dot: cross:
wrong! first row by first column with addition to make first entry,
first row by second column for second entry loop through indicies like
addition

\subsubsection{vectors:}
\label{sec:orga74ab53}
special case of matrix

column vec ( 1, 2 )

row vec ( 1, 2 )

cannot add diff dimensions

\subsubsection{representations}
\label{sec:org1e0b22a}
can draw up to 3, ish geometric is just arrow on graph to coords

adding vecs on the graph is just tip to tail, then first tip to last
tail for resultant just like phys

\begin{verbatim}
(
    a1
    a2
    .
    .
    .
    an
)
\end{verbatim}

is a vector of \(\mathbb{r}^n\)

\begin{verbatim}
matrix multiplication identity?
multiplication on group? multiplication on to collum vectors
\end{verbatim}

Homework: -
\href{KBxGroupAndMatricesIntro.org}{KBxGroupAndMatricesIntro}
\end{document}
