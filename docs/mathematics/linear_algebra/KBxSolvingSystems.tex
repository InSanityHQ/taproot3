% Created 2021-09-27 Mon 12:03
% Intended LaTeX compiler: xelatex
\documentclass[letterpaper]{article}
\usepackage{graphicx}
\usepackage{grffile}
\usepackage{longtable}
\usepackage{wrapfig}
\usepackage{rotating}
\usepackage[normalem]{ulem}
\usepackage{amsmath}
\usepackage{textcomp}
\usepackage{amssymb}
\usepackage{capt-of}
\usepackage{hyperref}
\setlength{\parindent}{0pt}
\usepackage[margin=1in]{geometry}
\usepackage{fontspec}
\usepackage{svg}
\usepackage{cancel}
\usepackage{indentfirst}
\setmainfont[ItalicFont = LiberationSans-Italic, BoldFont = LiberationSans-Bold, BoldItalicFont = LiberationSans-BoldItalic]{LiberationSans}
\newfontfamily\NHLight[ItalicFont = LiberationSansNarrow-Italic, BoldFont       = LiberationSansNarrow-Bold, BoldItalicFont = LiberationSansNarrow-BoldItalic]{LiberationSansNarrow}
\newcommand\textrmlf[1]{{\NHLight#1}}
\newcommand\textitlf[1]{{\NHLight\itshape#1}}
\let\textbflf\textrm
\newcommand\textulf[1]{{\NHLight\bfseries#1}}
\newcommand\textuitlf[1]{{\NHLight\bfseries\itshape#1}}
\usepackage{fancyhdr}
\pagestyle{fancy}
\usepackage{titlesec}
\usepackage{titling}
\makeatletter
\lhead{\textbf{\@title}}
\makeatother
\rhead{\textrmlf{Compiled} \today}
\lfoot{\theauthor\ \textbullet \ \textbf{2021-2022}}
\cfoot{}
\rfoot{\textrmlf{Page} \thepage}
\renewcommand{\tableofcontents}{}
\titleformat{\section} {\Large} {\textrmlf{\thesection} {|}} {0.3em} {\textbf}
\titleformat{\subsection} {\large} {\textrmlf{\thesubsection} {|}} {0.2em} {\textbf}
\titleformat{\subsubsection} {\large} {\textrmlf{\thesubsubsection} {|}} {0.1em} {\textbf}
\setlength{\parskip}{0.45em}
\renewcommand\maketitle{}
\author{Huxley Marvit}
\date{\today}
\title{Solving Systems Homework}
\hypersetup{
 pdfauthor={Huxley Marvit},
 pdftitle={Solving Systems Homework},
 pdfkeywords={},
 pdfsubject={},
 pdfcreator={Emacs 28.0.50 (Org mode 9.4.4)}, 
 pdflang={English}}
\begin{document}

\tableofcontents

\#ret \#hw

\noindent\rule{\textwidth}{0.5pt}

\section{Solving Systems}
\label{sec:orgef1d0f5}
Read 1.B! Have questions. Try a couple exercises. notes:
\href{KBxChapter1B.org}{KBxChapter1B}

Also, keep thinking about the group work questions from today:

\begin{itemize}
\item \textbf{What is the relationship between cross product and the magnitude of a
vector?}

\begin{itemize}
\item Which vector? Assuming resultant vector, \(||a||b|sin(\theta)|\)
would be the magnitude
\end{itemize}

\item \textbf{How does cosine relate to dot product? Can you prove it? (HINT: think
about the previous problem and the Law of Cosines.)}

\begin{itemize}
\item \(a \cdot b = |a||b|cos(\theta)\)
\item\relax [abcosthetaproof.pdf]
\end{itemize}

\item *Do your best to solve the following matrix equation using matrix
multiplications that correspond to row operations for systems,
specifically multiplying a row by a scalar, adding two rows, and
swapping the order of rows. You'll have to think about how to do these
things with matrices! It may help to keep in mind what a SOLVED matrix
equation looks like (in particular, what does the 3x3 matrix of
coefficients look like?).*
\end{itemize}

\$
\begin{bmatrix} 
    1 & -1 & 1 \\
    0 & 2 & 1\\
    2 & 1 & -2 \\
    \end{bmatrix}
\(\cdot\)
\begin{bmatrix} 
x \\
y \\
z \\
\end{bmatrix}
=
\begin{bmatrix} 
    -2 \\
    3 \\
    2 \\
    \end{bmatrix}
\$

\%\%\href{why.pdf.org}{why.pdf}\%\%

\href{Pasted image 20210904215239.png.org}{Pasted image
20210904215239.png}
\end{document}
