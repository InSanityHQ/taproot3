% Created 2021-09-12 Sun 22:50
% Intended LaTeX compiler: xelatex
\documentclass[letterpaper]{article}
\usepackage{graphicx}
\usepackage{grffile}
\usepackage{longtable}
\usepackage{wrapfig}
\usepackage{rotating}
\usepackage[normalem]{ulem}
\usepackage{amsmath}
\usepackage{textcomp}
\usepackage{amssymb}
\usepackage{capt-of}
\usepackage{hyperref}
\usepackage[margin=1in]{geometry}
\usepackage{fontspec}
\usepackage{indentfirst}
\setmainfont[ItalicFont = LiberationSans-Italic, BoldFont = LiberationSans-Bold, BoldItalicFont = LiberationSans-BoldItalic]{LiberationSans}
\newfontfamily\NHLight[ItalicFont = LiberationSansNarrow-Italic, BoldFont       = LiberationSansNarrow-Bold, BoldItalicFont = LiberationSansNarrow-BoldItalic]{LiberationSansNarrow}
\newcommand\textrmlf[1]{{\NHLight#1}}
\newcommand\textitlf[1]{{\NHLight\itshape#1}}
\let\textbflf\textrm
\newcommand\textulf[1]{{\NHLight\bfseries#1}}
\newcommand\textuitlf[1]{{\NHLight\bfseries\itshape#1}}
\usepackage{fancyhdr}
\pagestyle{fancy}
\usepackage{titlesec}
\usepackage{titling}
\makeatletter
\lhead{\textbf{\@title}}
\makeatother
\rhead{\textrmlf{Compiled} \today}
\lfoot{\theauthor\ \textbullet \ \textbf{2021-2022}}
\cfoot{}
\rfoot{\textrmlf{Page} \thepage}
\titleformat{\section} {\Large} {\textrmlf{\thesection} {|}} {0.3em} {\textbf}
\titleformat{\subsection} {\large} {\textrmlf{\thesubsection} {|}} {0.2em} {\textbf}
\titleformat{\subsubsection} {\large} {\textrmlf{\thesubsubsection} {|}} {0.1em} {\textbf}
\setlength{\parskip}{0.45em}
\renewcommand\maketitle{}
\author{Houjun Liu}
\date{\today}
\title{Series Convergence}
\hypersetup{
 pdfauthor={Houjun Liu},
 pdftitle={Series Convergence},
 pdfkeywords={},
 pdfsubject={},
 pdfcreator={Emacs 28.0.50 (Org mode 9.4.4)}, 
 pdflang={English}}
\begin{document}

\maketitle


\section{Series Convergence}
\label{sec:org0801755}
\subsection{Geometric Series}
\label{sec:orgd9c8069}
In \(\sum_{k=0}^\infty a(r^k)\), where \(|r|<1\), the series converges
to \(\sum_{k=0}^\infty a(r^k) = \frac{a}{1-r}\)

In \(\sum_{k=0}^n a(r^k)\),
\(\sum_{k=0}^n a(r^k) = \frac{a-ar^{n+1}}{1-r}\)

\subsection{nth term divergence test}
\label{sec:org1c2cbd2}
If \(\lim_{n \to \infty} a_n\) is not zero, the series \textbf{will} diverge.
The inverse is not necessarily true; that is, if this fails, use another
test to test convergence.

\subsection{Intergral Test}
\label{sec:org8ab88ab}
If the intergral to infinity is convergent, the sequence is convergent
as long as the sequence is continuous, positive, and decreasing. The
inverse applies, too.

\subsection{Power Series}
\label{sec:orga270074}
\(\sum^{\infty}_{n=1} \frac{1}{n^p}\)

If a p-series has a p > 1, the p-series will converge

If a p-series has a p <= 1, the p-series will diverge

\subsection{Comparison Test}
\label{sec:org61e8da6}
Both provided that \(a_n,b_n \geq 0\ \&\ a_n \leq b_n\)

\href{Pasted image 20210308082352.png.org}{Pasted image
20210308082352.png}

\href{Pasted image 20210308082201.png.org}{Pasted image
20210308082201.png}

Also, if \(\lim_{n \to \infty} \frac{a_n}{b_n} = C\ (0<c<\infty)\), the
two series will either both converge or both diverge. So you only need
to test one.

\subsection{Alternating Series Test}
\label{sec:orgb6261ea}
\href{Pasted image 20210309081249.png.org}{Pasted image
20210309081249.png}

\subsection{Ratio Test}
\label{sec:org3190f5c}
In a geometric series, the common ratio is simply
\(r = \frac{r^{n+1}}{r^n}\).

If \(r\) is an real value, \(|r|<1\), then series converges. If
\(|r| \geq 1\), the series diverges.

As limit goes to infinity in the \(r\), if the common ratio approaches
<1, that means that the ratio will get smaller and smaller, just like if
\(r\) were to be a real value and it was smaller than one. Meaning that
the series \textbf{converges.}

\href{Pasted image 20210310083028.png.org}{Pasted image
20210310083028.png}

And so, formally.

\href{Pasted image 20210310083142.png.org}{Pasted image
20210310083142.png}

The inverse is true, too.

\textbf{However, if the ratio is equal to one, the test is inconclusive.}

\noindent\rule{\textwidth}{0.5pt}

Absolute Convergence => series who converge and whose absolute value
converges

Conditional Convergence => series who converge and whose absolute value
does not converge

\subsection{So what is the error of a talor series? (Lagrange Error)}
\label{sec:org72dfdd0}
The error at point \(x\) of a \(n\)th degree talor polynomial centered
at \(a\) modeling a function with an absolute maximum value of \(M\) in
its \(n+1\)th dervitave between a bound containing \(x\) and \(a\):

\(|E(x)| \leq \frac{M(x-a)^{n+1}}{(n+1)!}\)

\subsection{Power Series}
\label{sec:orgba3d191}
\(f(x) = \sum_{n=0}^{\infty} a_n(x-c)^n= a_n(x-c)^0 + a_n(x-c)^1 ...\)

For instance, a geometric series is a special power series\ldots{}

\(g(x) = \sum_{n=0}^{\infty} = ax^n\)

This geometric series converges if \(|x|<1\), and so it has an interval
of convergence of \(-1 < x < 1\). If this converges, this function will
converge to \(\frac{a}{1-x}\)

\textbf{Interval of Convergence}: at what values of \(x\) does the series
converge?

\textbf{Radius of Convergence}: at what absolute distance from \(c\) (the
"centering" of the series) will the series converge?

To figure the interval of convergence, simply use the ratio test and
solve for \(x\) that makes the ratio \(< 1\). Then, think about the
inconclusive cases whereby ratio \(= 1\) --- then, use the comparison
test, or intergral test.

\href{Pasted image 20210317084838.png.org}{Pasted image
20210317084838.png}

*Derivatives, intergrals have the same radius of convergence as the
parent function, but their interval may be different due to different
behavior at endpoints*
\end{document}
