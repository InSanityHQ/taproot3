% Created 2021-09-11 Sat 16:43
% Intended LaTeX compiler: xelatex
\documentclass[letterpaper]{article}
\usepackage{graphicx}
\usepackage{grffile}
\usepackage{longtable}
\usepackage{wrapfig}
\usepackage{rotating}
\usepackage[normalem]{ulem}
\usepackage{amsmath}
\usepackage{textcomp}
\usepackage{amssymb}
\usepackage{capt-of}
\usepackage{hyperref}
\usepackage[margin=1in]{geometry}
\usepackage{fontspec}
\usepackage{indentfirst}
\setmainfont[ItalicFont = LiberationSans-Italic, BoldFont = LiberationSans-Bold, BoldItalicFont = LiberationSans-BoldItalic]{LiberationSans}
\newfontfamily\NHLight[ItalicFont = LiberationSansNarrow-Italic, BoldFont       = LiberationSansNarrow-Bold, BoldItalicFont = LiberationSansNarrow-BoldItalic]{LiberationSansNarrow}
\newcommand\textrmlf[1]{{\NHLight#1}}
\newcommand\textitlf[1]{{\NHLight\itshape#1}}
\let\textbflf\textrm
\newcommand\textulf[1]{{\NHLight\bfseries#1}}
\newcommand\textuitlf[1]{{\NHLight\bfseries\itshape#1}}
\usepackage{fancyhdr}
\pagestyle{fancy}
\usepackage{titlesec}
\usepackage{titling}
\makeatletter
\lhead{\textbf{\@title}}
\makeatother
\rhead{\textrmlf{Compiled} \today}
\lfoot{\theauthor\ \textbullet \ \textbf{2021-2022}}
\cfoot{}
\rfoot{\textrmlf{Page} \thepage}
\titleformat{\section} {\Large} {\textrmlf{\thesection} {|}} {0.3em} {\textbf}
\titleformat{\subsection} {\large} {\textrmlf{\thesubsection} {|}} {0.2em} {\textbf}
\titleformat{\subsubsection} {\large} {\textrmlf{\thesubsubsection} {|}} {0.1em} {\textbf}
\setlength{\parskip}{0.45em}
\renewcommand\maketitle{}
\author{Taproot}
\date{\today}
\title{Arc Length}
\hypersetup{
 pdfauthor={Taproot},
 pdftitle={Arc Length},
 pdfkeywords={},
 pdfsubject={},
 pdfcreator={Emacs 27.2 (Org mode 9.4.4)}, 
 pdflang={English}}
\begin{document}

\maketitle

\section{Formula}
\label{sec:org0cb8e39}

\[\begin{aligned}
   \int_{a}^{b} \sqrt{1+ f'^2(x)} dx
  \end{aligned}\]

\section{Derivation}
\label{sec:orgdd4a782}

Let \(S\) equal the total length of the curve.

Start with the pythagorean theorem: for any differentiable function,
\[ S = \sum_{a}^{b} dS \]
where \(dS\) can be calculated using the pythagorean theorem:

\[\begin{aligned}
  dS = \sqrt{\Delta y ^2 + \Delta x ^2 }
  \end{aligned}\]

We can simplify this by dividing \(dS\) by \(\Delta x\)

\[\begin{aligned}
  \frac{dS}{\Delta x} &=  \lim_{\Delta x \to  0} \sqrt{\frac{\Delta y^2}{\Delta x^2} + \frac{\Delta x^2}{\Delta x^2}}\\
  &=  \lim_{\Delta x \to  0} \sqrt{\frac{\Delta y^2}{\Delta x^2} + 1}\\
  &= \sqrt{f'^2(x) + 1}
  \end{aligned}\]

Now, to find \(dS\) again, we just have to multiply by \(dx\)

\[\begin{aligned}
  dS = \frac{dS}{\Delta x}dx = \sqrt{1+f'^2(x)}dx
  \end{aligned}\]
\end{document}
