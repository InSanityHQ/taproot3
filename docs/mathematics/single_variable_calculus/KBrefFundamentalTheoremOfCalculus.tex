% Created 2021-09-27 Mon 11:53
% Intended LaTeX compiler: xelatex
\documentclass[letterpaper]{article}
\usepackage{graphicx}
\usepackage{grffile}
\usepackage{longtable}
\usepackage{wrapfig}
\usepackage{rotating}
\usepackage[normalem]{ulem}
\usepackage{amsmath}
\usepackage{textcomp}
\usepackage{amssymb}
\usepackage{capt-of}
\usepackage{hyperref}
\setlength{\parindent}{0pt}
\usepackage[margin=1in]{geometry}
\usepackage{fontspec}
\usepackage{svg}
\usepackage{cancel}
\usepackage{indentfirst}
\setmainfont[ItalicFont = LiberationSans-Italic, BoldFont = LiberationSans-Bold, BoldItalicFont = LiberationSans-BoldItalic]{LiberationSans}
\newfontfamily\NHLight[ItalicFont = LiberationSansNarrow-Italic, BoldFont       = LiberationSansNarrow-Bold, BoldItalicFont = LiberationSansNarrow-BoldItalic]{LiberationSansNarrow}
\newcommand\textrmlf[1]{{\NHLight#1}}
\newcommand\textitlf[1]{{\NHLight\itshape#1}}
\let\textbflf\textrm
\newcommand\textulf[1]{{\NHLight\bfseries#1}}
\newcommand\textuitlf[1]{{\NHLight\bfseries\itshape#1}}
\usepackage{fancyhdr}
\pagestyle{fancy}
\usepackage{titlesec}
\usepackage{titling}
\makeatletter
\lhead{\textbf{\@title}}
\makeatother
\rhead{\textrmlf{Compiled} \today}
\lfoot{\theauthor\ \textbullet \ \textbf{2021-2022}}
\cfoot{}
\rfoot{\textrmlf{Page} \thepage}
\renewcommand{\tableofcontents}{}
\titleformat{\section} {\Large} {\textrmlf{\thesection} {|}} {0.3em} {\textbf}
\titleformat{\subsection} {\large} {\textrmlf{\thesubsection} {|}} {0.2em} {\textbf}
\titleformat{\subsubsection} {\large} {\textrmlf{\thesubsubsection} {|}} {0.1em} {\textbf}
\setlength{\parskip}{0.45em}
\renewcommand\maketitle{}
\author{Taproot}
\date{\today}
\title{Fundamental Theorem of Calculus}
\hypersetup{
 pdfauthor={Taproot},
 pdftitle={Fundamental Theorem of Calculus},
 pdfkeywords={},
 pdfsubject={},
 pdfcreator={Emacs 28.0.50 (Org mode 9.4.4)}, 
 pdflang={English}}
\begin{document}

\tableofcontents

\section{loose definition}
\label{sec:org3f6e73d}
\[\begin{aligned}
  \int \frac{d}{dx}f(x) dx = f(x)
  \end{aligned}\]
\section{formal definition}
\label{sec:orgb8254ff}
The theorem comes in two parts, apparently
\subsection{part 1}
\label{sec:org1768597}
\begin{quote}
If \(f(x)\) is continuous over an interval \([a, b]\), and the function \(F(x)\) is defined by
\[\begin{aligned}
   F(x) = \int_{a}^{x} f(t) dt
   \end{aligned}\]
then \(F'(x) = f(x)\) over \([a, b]\).
\end{quote}
\subsubsection{intuition}
\label{sec:org4d86ed2}
Note that its \(\int_{a}^{x} f(t) dt\) because \(x\) is an argument to the function and \(t\) is just the iteration variable.

Note that the integral can start anywhere to the left (arbitrary \(a\)) because that is removed as a constant when taking the derivative

Proof is by taking the limit form of a derivative of the integrals to \(x\) and \(x+h\), and seeing that it collapses to the mean value. As the range of the mean value expression goes to zero, the value converges to itself.
\subsubsection{results}
\label{sec:org818161c}
\begin{enumerate}
\item any integrable function and any continuous function has an anti-derivative
\label{sec:org95d3903}
\end{enumerate}
\subsection{part 2: the evaluation theorem}
\label{sec:org7d110bc}
\begin{quote}
If \(f(x)\) is continuous over the interval \([a, b]\) and \(F(x)\) is any anti-derivative of \(f(x)\), then
\[\begin{aligned}
   \int_{a}^{b} f(x) dx = F(b) - F(a)
   \end{aligned}\]
\end{quote}
\subsubsection{intuition}
\label{sec:org497824b}
If you can find the anti-derivative, then the sum between the regions is just the difference in the anti-derivative, which makes sense. Basically contiguous areas add up.
\section{an example}
\label{sec:org14b4584}
Imagine a function that has the bound of an integral as an argument:
\[\begin{aligned}
  g(x) = \int_0^x t\ dt = \frac{x^2}{2}\\
  \frac{d}{dx}g(x) = \frac{d}{dx}\int_0^x t\ dt = \frac{d}{dx}\frac{x^2}{2} = x
  \end{aligned}\]
\end{document}
