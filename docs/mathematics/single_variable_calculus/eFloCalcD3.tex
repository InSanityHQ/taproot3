% Created 2021-09-12 Sun 22:50
% Intended LaTeX compiler: xelatex
\documentclass[letterpaper]{article}
\usepackage{graphicx}
\usepackage{grffile}
\usepackage{longtable}
\usepackage{wrapfig}
\usepackage{rotating}
\usepackage[normalem]{ulem}
\usepackage{amsmath}
\usepackage{textcomp}
\usepackage{amssymb}
\usepackage{capt-of}
\usepackage{hyperref}
\usepackage[margin=1in]{geometry}
\usepackage{fontspec}
\usepackage{indentfirst}
\setmainfont[ItalicFont = LiberationSans-Italic, BoldFont = LiberationSans-Bold, BoldItalicFont = LiberationSans-BoldItalic]{LiberationSans}
\newfontfamily\NHLight[ItalicFont = LiberationSansNarrow-Italic, BoldFont       = LiberationSansNarrow-Bold, BoldItalicFont = LiberationSansNarrow-BoldItalic]{LiberationSansNarrow}
\newcommand\textrmlf[1]{{\NHLight#1}}
\newcommand\textitlf[1]{{\NHLight\itshape#1}}
\let\textbflf\textrm
\newcommand\textulf[1]{{\NHLight\bfseries#1}}
\newcommand\textuitlf[1]{{\NHLight\bfseries\itshape#1}}
\usepackage{fancyhdr}
\pagestyle{fancy}
\usepackage{titlesec}
\usepackage{titling}
\makeatletter
\lhead{\textbf{\@title}}
\makeatother
\rhead{\textrmlf{Compiled} \today}
\lfoot{\theauthor\ \textbullet \ \textbf{2021-2022}}
\cfoot{}
\rfoot{\textrmlf{Page} \thepage}
\titleformat{\section} {\Large} {\textrmlf{\thesection} {|}} {0.3em} {\textbf}
\titleformat{\subsection} {\large} {\textrmlf{\thesubsection} {|}} {0.2em} {\textbf}
\titleformat{\subsubsection} {\large} {\textrmlf{\thesubsubsection} {|}} {0.1em} {\textbf}
\setlength{\parskip}{0.45em}
\renewcommand\maketitle{}
\author{Exr0n}
\date{\today}
\title{Day 3}
\hypersetup{
 pdfauthor={Exr0n},
 pdftitle={Day 3},
 pdfkeywords={},
 pdfsubject={},
 pdfcreator={Emacs 28.0.50 (Org mode 9.4.4)}, 
 pdflang={English}}
\begin{document}

\maketitle


\section{Rate of Change (1, chemical reaction)}
\label{sec:org46a0713}
\begin{enumerate}
\item Average rate of change (slope) between \(t=20\) and \(t=30\) is
\(0.615\)
\item \(f\left(x\right)\ =\ \frac{\left(A_{0}\left(1-\exp\left(-k\left(x+p\right)\right)\right)-A_{0}\left(1-\exp\left(-k\left(x\right)\right)\right)\right)}{p}\)

\begin{enumerate}
\item Show that it looks like the tangent at \(x=25\):
\(y=f\left(25\right)\left(x-25\right)+51.444\)
\end{enumerate}

\item \href{https://www.desmos.com/calculator/ocjzjtyqjb}{Desmos Graph}
\end{enumerate}

\section{Rate of Change (2, washing machines)}
\label{sec:org747622d}
\begin{enumerate}
\item Average cost for \(100\) machines = \(\frac{11000}{100} = 110\)
\item Derivative is \(y = -0.2x + 100\), so we get \(80\)
\item By hard coding the numbers, we get
\(\left(2000+100\cdot101-0.1\left(101\right)^{2}\right)-\left(\left(2000+100\cdot100-0.1\left(100\right)^{2}\right)\right) = 79.9\)
which is roughly \(80\)
\item \href{20math401Marginal Cost}{Demos Graph}
\end{enumerate}

\section{Terminology}
\label{sec:org64f6c78}
\textbf{(slide 13 is confusing}, see questions.*)*

\section{Limits}
\label{sec:org80b2fdb}
\begin{enumerate}
\item Eq \(\frac{x^3-1}{x-1} \Rightarrow \{x^2+x+1 : x \neq 1\}\)
\end{enumerate}

\subsection{Limits Practice}
\label{sec:orgccfb2d2}
\begin{enumerate}
\item \(\lim_{x\to 10}2x+5 = 25\)
\item \(\lim_{x\to -2} \frac{x^2-x-6}{x-2} = -5\)
\item \(\lim_{x\to 4} \frac{x-4}{\sqrt{x}-2} \Rightarrow *\frac{\sqrt{x}+2}{\sqrt{x}+2} \Rightarrow \sqrt{x}+2 = 4\)
\item \(\lim_{x\to 0} \frac{sin x}{x}\): \(sin x = x\) for small \(x\)
(SHM), so we can treat it like \(\frac{x}{x}\) \#todo
\item \(\lim_{x\to 0} sin\frac{1}{x}\) Keeps changing\ldots{} Not sure how to
evaluate. \#todo
\item \(\lim_{x\to 2}\lfloor x \rfloor\)
\end{enumerate}

\noindent\rule{\textwidth}{0.5pt}
\end{document}
