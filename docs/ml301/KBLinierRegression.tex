% Created 2021-09-11 Sat 08:17
% Intended LaTeX compiler: xelatex
\documentclass[letterpaper]{article}
\usepackage{graphicx}
\usepackage{grffile}
\usepackage{longtable}
\usepackage{wrapfig}
\usepackage{rotating}
\usepackage[normalem]{ulem}
\usepackage{amsmath}
\usepackage{textcomp}
\usepackage{amssymb}
\usepackage{capt-of}
\usepackage{hyperref}
\usepackage[margin=1in]{geometry}
\usepackage{fontspec}
\usepackage{indentfirst}
\setmainfont[ItalicFont = LiberationSans-Italic, BoldFont = LiberationSans-Bold, BoldItalicFont = LiberationSans-BoldItalic]{LiberationSans}
\newfontfamily\NHLight[ItalicFont = LiberationSansNarrow-Italic, BoldFont       = LiberationSansNarrow-Bold, BoldItalicFont = LiberationSansNarrow-BoldItalic]{LiberationSansNarrow}
\newcommand\textrmlf[1]{{\NHLight#1}}
\newcommand\textitlf[1]{{\NHLight\itshape#1}}
\let\textbflf\textrm
\newcommand\textulf[1]{{\NHLight\bfseries#1}}
\newcommand\textuitlf[1]{{\NHLight\bfseries\itshape#1}}
\usepackage{fancyhdr}
\pagestyle{fancy}
\usepackage{titlesec}
\usepackage{titling}
\makeatletter
\lhead{\textbf{\@title}}
\makeatother
\rhead{\textrmlf{Compiled} \today}
\lfoot{\theauthor\ \textbullet \ \textbf{2021-2022}}
\cfoot{}
\rfoot{\textrmlf{Page} \thepage}
\titleformat{\section} {\Large} {\textrmlf{\thesection} {|}} {0.3em} {\textbf}
\titleformat{\subsection} {\large} {\textrmlf{\thesubsection} {|}} {0.2em} {\textbf}
\titleformat{\subsubsection} {\large} {\textrmlf{\thesubsubsection} {|}} {0.1em} {\textbf}
\setlength{\parskip}{0.45em}
\renewcommand\maketitle{}
\author{Huxley}
\date{\today}
\title{linier regresstion}
\hypersetup{
 pdfauthor={Huxley},
 pdftitle={linier regresstion},
 pdfkeywords={},
 pdfsubject={},
 pdfcreator={Emacs 27.2 (Org mode 9.4.4)}, 
 pdflang={English}}
\begin{document}

\maketitle


\section{One:}
\label{sec:orgc6fdb71}
\begin{itemize}
\item I would expect to see numbers similar to the y intercept and slope of
the line that the model is trying to fit (𝑦=0.3𝑥+1). In this
particular example, I would expect to see an intercept close to 1, and
a coefficients close to 0.3.
\end{itemize}

\section{Two:}
\label{sec:org40f8b06}
\begin{itemize}
\item I expected it to print out the corresponding y values when plugged
back into the original equation.
\end{itemize}

\section{Three:}
\label{sec:orgc4d4e34}
\begin{itemize}
\item I expected to see a line similar to the graph of 𝑦=0.3𝑥+1.
\end{itemize}

\section{Four}
\label{sec:org86e3c6c}
\begin{itemize}
\item I changed the equation of the line to
\texttt{data\_one\_x['y'] = 1 * data\_one\_x['x'] + 1} and verified that the code
still functioned. The output was
\texttt{Intercept: [1.]  Coefficients: [[file:1..org][1.]]} meaning that it
came to the correct answer, verifying that the code was working
properly.
\end{itemize}

\section{One}
\label{sec:org2ead8ba}
\begin{itemize}
\item I expected it to print numbers similar to the definition of the plane:
\texttt{y\_two\_x = 0.5 * x1\_two\_x - 2.7 * x2\_two\_x- 2 + noise\_two\_x}
\texttt{(0.5, -2.7, -2)}
\end{itemize}

\section{Two}
\label{sec:org2b8dd15}
\begin{itemize}
\item I expected to see a plane similar to the one defined above.
\end{itemize}

\section{Three}
\label{sec:org71a72dc}
\begin{itemize}
\item I decided to change the definition of the graph to
\texttt{y\_two\_x = 1 * x1\_two\_x + 1 * x2\_two\_x +1 + noise\_two\_x} and see if
the code still functioned. \texttt{print\_model\_fit} printed
\texttt{Intercept: 1.061603912300199  Coefficients: [0.97499882 0.96615802]},
showing that the code was working properly.
\end{itemize}

\section{Four}
\label{sec:orgc474c2f}
\begin{itemize}
\item The only major differences were in the visualization section. I would
imagine that these visualizations are very helpful with graphs
containing few dimensions, but become far less useful as the math
stays the same and the dimensions increase.
\end{itemize}
\end{document}
