% Created 2021-09-11 Sat 08:17
% Intended LaTeX compiler: xelatex
\documentclass[letterpaper]{article}
\usepackage{graphicx}
\usepackage{grffile}
\usepackage{longtable}
\usepackage{wrapfig}
\usepackage{rotating}
\usepackage[normalem]{ulem}
\usepackage{amsmath}
\usepackage{textcomp}
\usepackage{amssymb}
\usepackage{capt-of}
\usepackage{hyperref}
\usepackage[margin=1in]{geometry}
\usepackage{fontspec}
\usepackage{indentfirst}
\setmainfont[ItalicFont = LiberationSans-Italic, BoldFont = LiberationSans-Bold, BoldItalicFont = LiberationSans-BoldItalic]{LiberationSans}
\newfontfamily\NHLight[ItalicFont = LiberationSansNarrow-Italic, BoldFont       = LiberationSansNarrow-Bold, BoldItalicFont = LiberationSansNarrow-BoldItalic]{LiberationSansNarrow}
\newcommand\textrmlf[1]{{\NHLight#1}}
\newcommand\textitlf[1]{{\NHLight\itshape#1}}
\let\textbflf\textrm
\newcommand\textulf[1]{{\NHLight\bfseries#1}}
\newcommand\textuitlf[1]{{\NHLight\bfseries\itshape#1}}
\usepackage{fancyhdr}
\pagestyle{fancy}
\usepackage{titlesec}
\usepackage{titling}
\makeatletter
\lhead{\textbf{\@title}}
\makeatother
\rhead{\textrmlf{Compiled} \today}
\lfoot{\theauthor\ \textbullet \ \textbf{2021-2022}}
\cfoot{}
\rfoot{\textrmlf{Page} \thepage}
\titleformat{\section} {\Large} {\textrmlf{\thesection} {|}} {0.3em} {\textbf}
\titleformat{\subsection} {\large} {\textrmlf{\thesubsection} {|}} {0.2em} {\textbf}
\titleformat{\subsubsection} {\large} {\textrmlf{\thesubsubsection} {|}} {0.1em} {\textbf}
\setlength{\parskip}{0.45em}
\renewcommand\maketitle{}
\date{\today}
\title{}
\hypersetup{
 pdfauthor={},
 pdftitle={},
 pdfkeywords={},
 pdfsubject={},
 pdfcreator={Emacs 27.2 (Org mode 9.4.4)}, 
 pdflang={English}}
\begin{document}

\begin{center}
\begin{tabular}{l}
title: Zachary Second Reading Notes\\
author: Zachary Sayyah\\
course: PHYS201\\
source: \href{KBhPHYS201QuantumWorldBookNotesIndex.org}{KBhPHYS201QuantumWorldBookNotesIndex}\\
\end{tabular}
\end{center}

\section{Meet the Leptons}
\label{sec:org8ab7325}
\begin{itemize}
\item There are three families of subatomic particles

\begin{itemize}
\item Flavor 1: the electron and it's nuetrino
\item Flavor 2: the Muon and it's nuetrino
\item Flavor 3: the the tau lepton and it's nuetrino
\end{itemize}

\item Leptons carry one-half unit of spin and are either neutral or have 1
negative charge
\item For every lepton there's an anti-lepton with an opposite unit of
charge and the same mass
\item Leptons have no known size and in all theories describing them are
point particles
\end{itemize}

\subsection{Particles}
\label{sec:org71af41e}
\begin{itemize}
\item \textbf{Leptons} are fundemental spin one half particles that experience no
strong interactions and contain no quarks
\item \textbf{Baryons} are strong interacting particles that \emph{do} contain quarks
and have spin one half, but also 3/2 and 5/2 in some cases and are
relatively heavy
\item \textbf{Mesons} are composite strongly interacting particles also made of
quarks that have spin either 0 or 1.
\item \textbf{Quarks} are the fundamentally strongly interacting particles that are
constituents of Baryons and have baryonic charge giving them a charge
of 0 if two of them are united and 1 if 3 of them are
\item \textbf{Force carriers} are particles whose creation, annihilation, and
exchange create forces. It is these particles that we believe have no
substructure
\end{itemize}

\subsection{Electrons}
\label{sec:org6c3bfe2}
\begin{itemize}
\item Electrons are Leptons
\item Has an opposite which is the positron
\item Effectively launched particle physics as we know it
\item Is beta radiation
\item Equations to bridge relativity and quantum mechanics in order to
describe the electron predicted 1/2 spin and also anti-matter

\begin{itemize}
\item Some particles are their own anti-particles
\end{itemize}
\end{itemize}

\subsubsection{Radioactivity}
\label{sec:orgc0b888c}
\begin{itemize}
\item They realized that helium nucleus's shouldn't be able to escape the
nucleus of a particle and that it would have to escape through quantum
tunneling
\item Gamma radiation is produced by a change in quantum state of the
protons
\item Neutrino was suggested as a way to solve some of beta decay's issues

\begin{itemize}
\item So was neutron
\end{itemize}

\item There are 3 types of neutrinos known
\item We detected Nuetrinos through weak interactions with things around
them
\end{itemize}

\subsection{Muons}
\label{sec:orga0c507f}
\begin{itemize}
\item Cosmic radiation kept giving rise to the theory that there were
charged particles around 200x more massive than an electron
\item Muons appeared exactly like electrons but more massive

\begin{itemize}
\item For some reason however it doesn't decay into an electron and like a
gamma ray so we can assume it is somehow fundamentally different

\begin{itemize}
\item This brings us back to it being a different \emph{flavor}
\end{itemize}
\end{itemize}
\end{itemize}

\subsection{\#\#\# The Muon Neutrino}
\label{sec:org6886d73}
\end{document}
