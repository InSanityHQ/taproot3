% Created 2021-09-11 Sat 09:35
% Intended LaTeX compiler: xelatex
\documentclass[letterpaper]{article}
\usepackage{graphicx}
\usepackage{grffile}
\usepackage{longtable}
\usepackage{wrapfig}
\usepackage{rotating}
\usepackage[normalem]{ulem}
\usepackage{amsmath}
\usepackage{textcomp}
\usepackage{amssymb}
\usepackage{capt-of}
\usepackage{hyperref}
\usepackage[margin=1in]{geometry}
\usepackage{fontspec}
\usepackage{indentfirst}
\setmainfont[ItalicFont = LiberationSans-Italic, BoldFont = LiberationSans-Bold, BoldItalicFont = LiberationSans-BoldItalic]{LiberationSans}
\newfontfamily\NHLight[ItalicFont = LiberationSansNarrow-Italic, BoldFont       = LiberationSansNarrow-Bold, BoldItalicFont = LiberationSansNarrow-BoldItalic]{LiberationSansNarrow}
\newcommand\textrmlf[1]{{\NHLight#1}}
\newcommand\textitlf[1]{{\NHLight\itshape#1}}
\let\textbflf\textrm
\newcommand\textulf[1]{{\NHLight\bfseries#1}}
\newcommand\textuitlf[1]{{\NHLight\bfseries\itshape#1}}
\usepackage{fancyhdr}
\pagestyle{fancy}
\usepackage{titlesec}
\usepackage{titling}
\makeatletter
\lhead{\textbf{\@title}}
\makeatother
\rhead{\textrmlf{Compiled} \today}
\lfoot{\theauthor\ \textbullet \ \textbf{2021-2022}}
\cfoot{}
\rfoot{\textrmlf{Page} \thepage}
\titleformat{\section} {\Large} {\textrmlf{\thesection} {|}} {0.3em} {\textbf}
\titleformat{\subsection} {\large} {\textrmlf{\thesubsection} {|}} {0.2em} {\textbf}
\titleformat{\subsubsection} {\large} {\textrmlf{\thesubsubsection} {|}} {0.1em} {\textbf}
\setlength{\parskip}{0.45em}
\renewcommand\maketitle{}
\author{Houjun Liu}
\date{\today}
\title{Current}
\hypersetup{
 pdfauthor={Houjun Liu},
 pdftitle={Current},
 pdfkeywords={},
 pdfsubject={},
 pdfcreator={Emacs 27.2 (Org mode 9.4.4)}, 
 pdflang={English}}
\begin{document}

\maketitle


\section{Current}
\label{sec:orge98fd3c}
Current could be understood as the \emph{flow} of electricity on a circuit.
Note the difference between
\href{KBhPHYS201Voltage.org}{KBhPHYS201Voltage} --- which is a measure
of the \emph{pressure} of electrons --- and current, which is a measure of
the "volume of electrons per second".

Use the variable \(I\), a unit \(\frac{C}{s}\), \(Amps\), to measure
current. This also equals \(\frac{\Delta V}{Resistance}\). Big
resistance, little current. Current is measured in a unit
\(\frac{C}{s}\), which intuitively makes sense --- Current/second is
kind of like \(\frac{m^3}{s}\) --- it measures, roughly, the "amount of
flow"/second.

\definition{Current $I$}\{A value measured in unit \(\frac{C}{s}\), a.k.a. \(Amps\) that measures electron flow\}
\end{document}
