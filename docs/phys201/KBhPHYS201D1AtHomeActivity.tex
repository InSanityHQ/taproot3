% Created 2021-09-11 Sat 08:17
% Intended LaTeX compiler: xelatex
\documentclass[letterpaper]{article}
\usepackage{graphicx}
\usepackage{grffile}
\usepackage{longtable}
\usepackage{wrapfig}
\usepackage{rotating}
\usepackage[normalem]{ulem}
\usepackage{amsmath}
\usepackage{textcomp}
\usepackage{amssymb}
\usepackage{capt-of}
\usepackage{hyperref}
\usepackage[margin=1in]{geometry}
\usepackage{fontspec}
\usepackage{indentfirst}
\setmainfont[ItalicFont = LiberationSans-Italic, BoldFont = LiberationSans-Bold, BoldItalicFont = LiberationSans-BoldItalic]{LiberationSans}
\newfontfamily\NHLight[ItalicFont = LiberationSansNarrow-Italic, BoldFont       = LiberationSansNarrow-Bold, BoldItalicFont = LiberationSansNarrow-BoldItalic]{LiberationSansNarrow}
\newcommand\textrmlf[1]{{\NHLight#1}}
\newcommand\textitlf[1]{{\NHLight\itshape#1}}
\let\textbflf\textrm
\newcommand\textulf[1]{{\NHLight\bfseries#1}}
\newcommand\textuitlf[1]{{\NHLight\bfseries\itshape#1}}
\usepackage{fancyhdr}
\pagestyle{fancy}
\usepackage{titlesec}
\usepackage{titling}
\makeatletter
\lhead{\textbf{\@title}}
\makeatother
\rhead{\textrmlf{Compiled} \today}
\lfoot{\theauthor\ \textbullet \ \textbf{2021-2022}}
\cfoot{}
\rfoot{\textrmlf{Page} \thepage}
\titleformat{\section} {\Large} {\textrmlf{\thesection} {|}} {0.3em} {\textbf}
\titleformat{\subsection} {\large} {\textrmlf{\thesubsection} {|}} {0.2em} {\textbf}
\titleformat{\subsubsection} {\large} {\textrmlf{\thesubsubsection} {|}} {0.1em} {\textbf}
\setlength{\parskip}{0.45em}
\renewcommand\maketitle{}
\author{Houjun Liu and Exr0n}
\date{\today}
\title{D1 At home Activity}
\hypersetup{
 pdfauthor={Houjun Liu and Exr0n},
 pdftitle={D1 At home Activity},
 pdfkeywords={},
 pdfsubject={},
 pdfcreator={Emacs 27.2 (Org mode 9.4.4)}, 
 pdflang={English}}
\begin{document}

\maketitle
[Screen Shot 2020-08-24 at 11.18.55 AM.png] \#brokelink
(/Users/houliu/Desktop/Screen Shot 2020-08-24 at 11.18.55 AM.png)

\emph{Disclaimer} --- because of material differences, nothing happened\ldots{} I
am noting here the supposed responses.

\textbf{Senario 1}: the non-charged pen spins towards ("gets attracted")
towards the charged pen \textbf{Senario 2}: the pens repelled, spinning away
from each other - When both pens are rubbed, they have similar charges.
This (should) cause them to repel each other. \textbf{Senario 3}: the pen was
able to pick up chips of paper through charge, and the paper stuck to
the pen. Occasionally, pieces of paper starts bouncing towards and away
from the charged pen when the pen is not very close \textbf{Senario 4 (extra)}:
only the rubbed "charged" end of the pen was attracting \textbf{Senario 5}:
Scotch tape 2. Take a strip of scotch tape, fold over a tab, and stick
it on top of the first strip of scotch tape. Mark this piece of tape
\texttt{B} 3. Take a second strip, fold a tab over, and mark this \texttt{T}. 4. Grab
the top two tabs labeled \texttt{B} and \texttt{T}, and pull them up off the table
quickly. 5. Grab the two tabs separately, and pull them apart
quickly. 6. Take the two pieces, sticky side facing away from each
other, and bring them together. They should attract 7. Repeat with two
more pieces of tape. \texttt{B} with \texttt{B} should repel, opposites should
attract.

\noindent\rule{\textwidth}{0.5pt}

The rubbing is trying to create a \textbf{charge separation} --- getting
positive electrons "off the fur" and "on the pen". These electrons are
able to then come with the pen, and attract various things like paper
and pen.

\textbf{Electrostatics Worksheet} KB20200824134844.pdf
\end{document}
