% Created 2021-09-11 Sat 09:35
% Intended LaTeX compiler: xelatex
\documentclass[letterpaper]{article}
\usepackage{graphicx}
\usepackage{grffile}
\usepackage{longtable}
\usepackage{wrapfig}
\usepackage{rotating}
\usepackage[normalem]{ulem}
\usepackage{amsmath}
\usepackage{textcomp}
\usepackage{amssymb}
\usepackage{capt-of}
\usepackage{hyperref}
\usepackage[margin=1in]{geometry}
\usepackage{fontspec}
\usepackage{indentfirst}
\setmainfont[ItalicFont = LiberationSans-Italic, BoldFont = LiberationSans-Bold, BoldItalicFont = LiberationSans-BoldItalic]{LiberationSans}
\newfontfamily\NHLight[ItalicFont = LiberationSansNarrow-Italic, BoldFont       = LiberationSansNarrow-Bold, BoldItalicFont = LiberationSansNarrow-BoldItalic]{LiberationSansNarrow}
\newcommand\textrmlf[1]{{\NHLight#1}}
\newcommand\textitlf[1]{{\NHLight\itshape#1}}
\let\textbflf\textrm
\newcommand\textulf[1]{{\NHLight\bfseries#1}}
\newcommand\textuitlf[1]{{\NHLight\bfseries\itshape#1}}
\usepackage{fancyhdr}
\pagestyle{fancy}
\usepackage{titlesec}
\usepackage{titling}
\makeatletter
\lhead{\textbf{\@title}}
\makeatother
\rhead{\textrmlf{Compiled} \today}
\lfoot{\theauthor\ \textbullet \ \textbf{2021-2022}}
\cfoot{}
\rfoot{\textrmlf{Page} \thepage}
\titleformat{\section} {\Large} {\textrmlf{\thesection} {|}} {0.3em} {\textbf}
\titleformat{\subsection} {\large} {\textrmlf{\thesubsection} {|}} {0.2em} {\textbf}
\titleformat{\subsubsection} {\large} {\textrmlf{\thesubsubsection} {|}} {0.1em} {\textbf}
\setlength{\parskip}{0.45em}
\renewcommand\maketitle{}
\author{Houjun Liu}
\date{\today}
\title{Gravitational Fields, Newton's Law of Gravitation}
\hypersetup{
 pdfauthor={Houjun Liu},
 pdftitle={Gravitational Fields, Newton's Law of Gravitation},
 pdfkeywords={},
 pdfsubject={},
 pdfcreator={Emacs 27.2 (Org mode 9.4.4)}, 
 pdflang={English}}
\begin{document}

\maketitle


\section{Let's talk about Gravitational Fields}
\label{sec:orgcd1bea8}
Each object has what's called \textbf{gravitational field.} Surrounding each
object has what is effectively many tiny vectors getting weaker and
weaker as you move away from the Earth.

Remember, this is \emph{not the gravitational force} between two objects.
This is simply the \emph{gravitational FIELD} of one. To calculate the force
of gravity from the Gravitational Field, simply use multiply the mass of
the attractee to the gravitational field of the attractor, that is,
\(F_{grav} = M_{obj2} * GravField_{obj1}\).

\subsection{Newton's Law of Gravitation}
\label{sec:orgd3476a3}
And for actually calculating the gravitational field, you will need

\definition[where $G$ is a constant called "Gravitational constant"]{(A part of) Newton's Law of Gravitation}\{\(Grav. Field = \frac{G M_{source}}{R^2}\)\}
For good measure, here's the two equations combined to form the full
gravitational field.

\definition{Newton's Law of Gravitation}\{\(F_{grav} = M_{target} \frac{G M_{source}}{R^2}\)\}
It does not actually matter which object is the target and which one is
the source. Because of an magical property called the "Multiplication is
Commutative", swapping attractor and attractee will have the same
numerical result for gravitational force. (\emph{note! the field vectors are
still different though})

The units for \emph{Gravitational Field} is \(\frac{N}{kg}\), which, the
keen-eyed will see, equals \(\frac{m}{s^2}\), which, of course, is
acceleration.

And now for a old piece of news:

\definition{(Roughly) Earth's Gravitational Field}\{\(9.8\frac{N}{kg} = 9.8\frac{m}{s^2}\)\}
\subsection{Connection to Electric Fields}
\label{sec:org6cf8195}
Gravitational fields, w.r.t. Newton's Law of Gravitation, is actually
analogous to how the
\href{KBhPHYS201ElectricFields.org}{KBhPHYS201ElectricFields} Electric
field works. See there for some info on why that's the case and how it
could work.
\end{document}
