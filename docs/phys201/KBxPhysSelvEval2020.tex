% Created 2021-09-11 Sat 08:17
% Intended LaTeX compiler: xelatex
\documentclass[letterpaper]{article}
\usepackage{graphicx}
\usepackage{grffile}
\usepackage{longtable}
\usepackage{wrapfig}
\usepackage{rotating}
\usepackage[normalem]{ulem}
\usepackage{amsmath}
\usepackage{textcomp}
\usepackage{amssymb}
\usepackage{capt-of}
\usepackage{hyperref}
\usepackage[margin=1in]{geometry}
\usepackage{fontspec}
\usepackage{indentfirst}
\setmainfont[ItalicFont = LiberationSans-Italic, BoldFont = LiberationSans-Bold, BoldItalicFont = LiberationSans-BoldItalic]{LiberationSans}
\newfontfamily\NHLight[ItalicFont = LiberationSansNarrow-Italic, BoldFont       = LiberationSansNarrow-Bold, BoldItalicFont = LiberationSansNarrow-BoldItalic]{LiberationSansNarrow}
\newcommand\textrmlf[1]{{\NHLight#1}}
\newcommand\textitlf[1]{{\NHLight\itshape#1}}
\let\textbflf\textrm
\newcommand\textulf[1]{{\NHLight\bfseries#1}}
\newcommand\textuitlf[1]{{\NHLight\bfseries\itshape#1}}
\usepackage{fancyhdr}
\pagestyle{fancy}
\usepackage{titlesec}
\usepackage{titling}
\makeatletter
\lhead{\textbf{\@title}}
\makeatother
\rhead{\textrmlf{Compiled} \today}
\lfoot{\theauthor\ \textbullet \ \textbf{2021-2022}}
\cfoot{}
\rfoot{\textrmlf{Page} \thepage}
\titleformat{\section} {\Large} {\textrmlf{\thesection} {|}} {0.3em} {\textbf}
\titleformat{\subsection} {\large} {\textrmlf{\thesubsection} {|}} {0.2em} {\textbf}
\titleformat{\subsubsection} {\large} {\textrmlf{\thesubsubsection} {|}} {0.1em} {\textbf}
\setlength{\parskip}{0.45em}
\renewcommand\maketitle{}
\author{Huxley}
\date{\today}
\title{Physics Self Eval}
\hypersetup{
 pdfauthor={Huxley},
 pdftitle={Physics Self Eval},
 pdfkeywords={},
 pdfsubject={},
 pdfcreator={Emacs 27.2 (Org mode 9.4.4)}, 
 pdflang={English}}
\begin{document}

\maketitle
\#ret

\noindent\rule{\textwidth}{0.5pt}

\begin{verbatim}
Write about something you have discovered about yourself - through your study of physics, related to physics class, or even outside of physics- that is helpful to you as a lifelong learner. - Consider how much progress you have made in physics over the course of the semester.

If you have trouble with the above, answer one or more of the following questions, or none of the following questions, or any reflections you have on your own learning that do or do not relate to these questions.

You do not need to answer any one of these questions.  They are merely food for thought.

A)  Have you realized something about approaches or strategies that help you to learn better?  Understand concepts better? Recall topics/ideas easier? What would you suggest to someone else about how to succeed in learning physics?

B)  Have you discovered something about how to motivate yourself to take on challenge and to work through a problem even when there are difficulties? Are there different strategies you try to tackle a new problem?

C)  Have you come up with any ideas for techniques that you would like to try to help yourself learn faster?  Retain knowledge more easily?

D)  Are there any techniques of visualizing and framing concepts that you have discovered or improved upon that help you in physics?  Might you be able to apply similar techniques outside of physics.

E)  Are there any revelations that you have made that help you to learn and understand physics much easier?
\end{verbatim}

Physics class has actully deeply impacted the way I go about learning.

Solving symbolically is emphasized heavily in Physics class. I didn't
see the value of doing so until I realized that getting the integer
solution to our physics problems was not really the end goal. Instead,
the goal was to derive an equation that could then be used to solved the
problem. These equations would reveal deeper truths that would otherwise
stay hidden; in a learning environment (which, frankly, is most
environments) solving symbolically is undoubtedly the way to go. This
doesn't just apply to physics though --- I've found the concept of
solving symbolically very useful in other areas as well. Even in
conceptual spaces that don't focus on absolutes, like English, this
concept can still be applied and utilized.

In many years, perhaps I will not remember every kinematic equation, but
I will still stay true to the idea of solving symbolically.
\end{document}
