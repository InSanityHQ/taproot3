% Created 2021-09-11 Sat 09:36
% Intended LaTeX compiler: xelatex
\documentclass[letterpaper]{article}
\usepackage{graphicx}
\usepackage{grffile}
\usepackage{longtable}
\usepackage{wrapfig}
\usepackage{rotating}
\usepackage[normalem]{ulem}
\usepackage{amsmath}
\usepackage{textcomp}
\usepackage{amssymb}
\usepackage{capt-of}
\usepackage{hyperref}
\usepackage[margin=1in]{geometry}
\usepackage{fontspec}
\usepackage{indentfirst}
\setmainfont[ItalicFont = LiberationSans-Italic, BoldFont = LiberationSans-Bold, BoldItalicFont = LiberationSans-BoldItalic]{LiberationSans}
\newfontfamily\NHLight[ItalicFont = LiberationSansNarrow-Italic, BoldFont       = LiberationSansNarrow-Bold, BoldItalicFont = LiberationSansNarrow-BoldItalic]{LiberationSansNarrow}
\newcommand\textrmlf[1]{{\NHLight#1}}
\newcommand\textitlf[1]{{\NHLight\itshape#1}}
\let\textbflf\textrm
\newcommand\textulf[1]{{\NHLight\bfseries#1}}
\newcommand\textuitlf[1]{{\NHLight\bfseries\itshape#1}}
\usepackage{fancyhdr}
\pagestyle{fancy}
\usepackage{titlesec}
\usepackage{titling}
\makeatletter
\lhead{\textbf{\@title}}
\makeatother
\rhead{\textrmlf{Compiled} \today}
\lfoot{\theauthor\ \textbullet \ \textbf{2021-2022}}
\cfoot{}
\rfoot{\textrmlf{Page} \thepage}
\titleformat{\section} {\Large} {\textrmlf{\thesection} {|}} {0.3em} {\textbf}
\titleformat{\subsection} {\large} {\textrmlf{\thesubsection} {|}} {0.2em} {\textbf}
\titleformat{\subsubsection} {\large} {\textrmlf{\thesubsubsection} {|}} {0.1em} {\textbf}
\setlength{\parskip}{0.45em}
\renewcommand\maketitle{}
\author{Exr0n}
\date{\today}
\title{Notes on The Quantum World by Kenneth W Ford}
\hypersetup{
 pdfauthor={Exr0n},
 pdftitle={Notes on The Quantum World by Kenneth W Ford},
 pdfkeywords={},
 pdfsubject={},
 pdfcreator={Emacs 27.2 (Org mode 9.4.4)}, 
 pdflang={English}}
\begin{document}

\maketitle
\section{beneath the surface of things}
\label{sec:org42e5adb}
\subsection{scale}
\label{sec:orgedbb0ac}
\subsubsection{wood seems solid, but is mostly empty space (atoms). Quantum effects are noticable on even smaller scales}
\label{sec:org0095624}
Atoms are like Propeller analogy: easy for something small and fast to get through, but hard for something large and slow.
\subsection{unintuitive}
\label{sec:orgd2177e3}
Small (quantum mechanics) and fast (special relativity) things behave unintuitively
\subsection{particle history}
\label{sec:org731cc96}
\subsubsection{electron and proton}
\label{sec:org4f5feda}
known to exist in 1926
\subsubsection{photon}
\label{sec:org34888b6}
no mass, created/detsroyed easily, was thought of as a funky 'almost particle' until it was discovered that they behave similarly to electrons and that electrons can be created/destroyed easily too
\subsection{fundamental discoveries in 1924-1928}
\label{sec:orgf27001d}
\subsubsection{matter has wave properties}
\label{sec:orgbb6dc8a}
\subsubsection{fundamental laws are probabliistic}
\label{sec:orgb44455d}
\subsubsection{heisenburg uncertainty}
\label{sec:orgf1a0b48}
\subsubsection{discrete electron spin}
\label{sec:org7dc1838}
\subsubsection{every particle has an antiparticle}
\label{sec:orgf8cd6ee}
\subsubsection{multiple momentums simultaneously}
\label{sec:orgc7ea838}
\subsubsection{no two electrons can be in the same state of motion at the same time?}
\label{sec:org48c7827}
\subsection{unlike classical mechanics, properties and actions are not as distinct in the quantum world}
\label{sec:org69b229d}
\subsection{quantum mechanics won't follow common sense, which is to be expected because common sense is based on much larger things}
\label{sec:org1a8c243}
\subsection{subatomic vs fundamental particles}
\label{sec:org09771e6}
\subsubsection{hundreds of subatomic particles, but only a few are fundamental}
\label{sec:org4c0d96c}
\subsubsection{most are composite}
\label{sec:org62287e3}
\subsubsection{just like how there are hundreds of atoms but they are all composed of electrons, protons, neutrons}
\label{sec:org2c1754d}
\subsubsection{there are 24 fundamental particles excluding the higgs boson, graviton (which is not yet proven), and antiparticles.}
\label{sec:org4c9322c}

\section{how small is small? How fast is fast? (scale)}
\label{sec:orgb602629}
\subsection{convienent units}
\label{sec:orgac3718c}
\subsubsection{femtometer (fm, \(10^{-15} m\))}
\label{sec:org4953d6d}
\begin{enumerate}
\item roughly a proton diameter
\label{sec:org8b87768}
\item 6 miles is geometrically centered between the smallest particle probe and the radius of the known universe.
\label{sec:orgf1fa2c0}
\item smallest prob is \(10^{-18}\) meters. The plank length is \(10^{-35}\) meters
\label{sec:org4bc8c7d}
\end{enumerate}
\subsubsection{speed of light (c, \(3 \times 10^8 m/s^2\))}
\label{sec:orge2c5f15}
\begin{enumerate}
\item not as sped as distances are small
\label{sec:org6ff8f08}
\item atoms and molecules vibrate roughly \(10^{-5} c\) or \(10^{-6} c\)
\label{sec:org3fb749e}
\item the lighter something is, the faster it can go, so its likely that the rest-massless photon is the fastest
\label{sec:org61be641}
\item some theoretical tachyon business which can maybe go faster
\label{sec:orged46de5}
\end{enumerate}
\subsubsection{time}
\label{sec:org253e57a}
\begin{enumerate}
\item humans
\label{sec:org14a9a29}
\begin{enumerate}
\item image flashed for a hundreth of a second (10 ms) can be precieved but not a thousandth
\label{sec:org70c37b9}
\item average human reaction time is 150-300ms
\label{sec:org2f5593e}
\end{enumerate}
\item time to cross diameter of a proton at sped of light = \(10^{-23}s\)
\label{sec:orgfa3f4b2}
\item particles that live long enough to leave trails in the detector live roughly \(10^{-10}\) to \(10^{-6}\)
\label{sec:org8164829}
\item longest living is proton for 15 min (10\textsuperscript{3})
\label{sec:orgf493298}
\end{enumerate}
\subsubsection{mass}
\label{sec:org891f2fa}
\begin{enumerate}
\item mass is inertia, measured by how hard it is to accelerate them (change their motion)
\label{sec:org90be290}
\item often measured as energy via \(E = mc^2\)
\label{sec:org68b52e1}
\begin{enumerate}
\item proton = 938MeV is easier to say than \(1.67\times 10^{-27}kg\)
\label{sec:orga930ed0}
\end{enumerate}
\end{enumerate}
\subsubsection{electron volt (eV, energy auired by an electron being accelerated through an electric potential of 1 volt)}
\label{sec:org84b0698}
\begin{enumerate}
\item roughly a photon of red light
\label{sec:org7ef7cb0}
\item particle accelerators are made to create high energy particles that can then be converted to mass
\label{sec:orgff82093}
\item modern accelerators go to roughly 1TeV, while protons move with only 1eV on the surface of the sun and weigh almost 1GeV.
\label{sec:orga68a91b}
\end{enumerate}
\subsubsection{charge (\(e, 1.6\times 10^{-19}C\))}
\label{sec:org5101d8d}
\begin{enumerate}
\item protons and electrons have the same magnitude of charge, deemed one unit. quarks have fractional charges
\label{sec:org359d4bb}
\item open questions
\label{sec:org8b97e50}
\begin{enumerate}
\item why is charge quantized/descrete
\label{sec:orgd22f421}
\item what happens near charged particles? inf charge as dist \(\to\) zero
\label{sec:org0403ddd}
\item if particles are physically sized, why dont parts of the particle repel itself
\label{sec:org3a212ca}
\end{enumerate}
\end{enumerate}
\subsubsection{spin}
\label{sec:orgd001c62}
\begin{enumerate}
\item two types: spin and orbital motion
\label{sec:org9f8dd7d}
\item measured with angular momentum
\label{sec:orge931e69}
\item fundamental particles don't have a descernible spin but do have an angular momentum
\label{sec:org5d64592}
\item \(\hbar = \frac{h}{2\pi} = 1.05\times 10^{-34} kgm^2/s\)
\label{sec:org9ce6fdb}
\item orbital angular momentum must be a multiple of \(\hbar\) and spin angular momentum can be a multiple of \(\frac{1}{2}\hbar\)?
\label{sec:orga66f0ae}
\item a particle type can have many spins, but the change in spin is often so drastic that they are considered two different particles
\label{sec:orgc3e2f6e}
\end{enumerate}
\subsubsection{fundamental constants}
\label{sec:org8a235f3}
\begin{enumerate}
\item most units are chosen arbitrarily based on earths size or something, but there are two fundamental ones
\label{sec:org543f882}
\item plank's constant \(h\) defines the quantum scale\ldots{} larger \(h\) would make the universe 'lumpier' or 'more pixelated'
\label{sec:org09af4dc}
\item the speed of light \(c\) is the fastest speed, or something.
\label{sec:org19d1b47}
\item there is expceted to be a third constant to form a complete basis, but we haven't found one yet
\label{sec:orgc6a3767}
\begin{enumerate}
\item it would be a length or a time
\label{sec:orgeecc86c}
\end{enumerate}
\end{enumerate}
\section{meet the leptons}
\label{sec:orgef32688}
\subsection{types (flavor) electron, muon, tau}
\label{sec:orga7bffed}
\subsection{conserve \{charge, flavor, energy, momentum\}}
\label{sec:org65caeac}
\subsection{neutrinos are like the soul of its particle - tiny mass but same flavor}
\label{sec:org6b4f3c0}
\subsection{they have multiples of half unit spins or something}
\label{sec:org7df6965}
\subsection{any particles can be created as long as everything is conserved}
\label{sec:org57ac779}
\section{the rest of the extended family}
\label{sec:org9294ad2}
\subsection{quarks}
\label{sec:orgc02027a}
\subsubsection{six of them in groups of three}
\label{sec:org8de1e39}
\subsubsection{they group up in the wild to make up other composite particles}
\label{sec:org8edb33c}
\subsubsection{particles have integer charge, but quarks come in one-third multiples of charge}
\label{sec:org66bbd60}
\subsubsection{baryon number}
\label{sec:org294c84b}
\begin{enumerate}
\item another type of 'charge' that is also conserved
\label{sec:orgbae23dd}
\item protons and neutrons are both baryonic (meaning they have baryon number?)
\label{sec:orgc4a0bba}
\item like leptons, the lightest baryon cannot decay because there is nothing to decay into (the proton)
\label{sec:org87ee49c}
\item quarks have one-third baryon number also
\label{sec:orgf2f9ae8}
\end{enumerate}
\subsubsection{antiquarks}
\label{sec:org9f7b959}
\begin{enumerate}
\item makes up meson with another singular quark (For a total baryon number of 0)
\label{sec:org19ac553}
\item antiquark particles are unstable
\label{sec:org364c3ca}
\end{enumerate}
\subsubsection{color}
\label{sec:orgd5a2693}
\begin{enumerate}
\item red, green, blue, antired, antigreen, antiblue
\label{sec:org8b68560}
\item all three or a normal with an anti is colorless..???
\label{sec:org5b1c8a1}

ended page 71
\end{enumerate}
\subsection{composite particles}
\label{sec:org56db451}
\subsubsection{baryons vs mesons}
\label{sec:orged824c0}
\begin{enumerate}
\item baryons have half odd integral spins (1/2, 3/2, 5/2) while mesons have integral spins
\label{sec:org30269cc}
\item baryons are made of 3 quarks each while mesons are made up of a quark and an antiquark
\label{sec:orgc5dc547}
\item baryons are fermions and mesons are bosons. all are hadrons (strongly interacting) bc quarks are strongly interacting
\label{sec:org8fc9183}
\item mesons have baryon number zero
\label{sec:org241b4a1}
\end{enumerate}
\subsubsection{some baryons}
\label{sec:org4a3afca}
\begin{enumerate}
\item lightest are proton and neutron (made up up and down quarks)
\label{sec:orge0f89cc}
\item then heavier ones have strange quarks, and some even heavier have charm and bottom quarks
\label{sec:org99494aa}
\item have not found a baryon that contains a top quark yet
\label{sec:org54d7ada}
\item other than the proton, all 'baryons are unstable (radoactive)'
\label{sec:org3f8ac6b}
\item they all live a really short time (see the table) roughly \(10^{-10}\) to \(10^{-19}\) seconds
\label{sec:orgfcc4de9}
\end{enumerate}
\subsubsection{some mesons}
\label{sec:org3809a37}

\begin{enumerate}
\item the pion (lightest meson)
\label{sec:org2c71a0e}
\begin{enumerate}
\item Yukawa thought pions moving around gave rise to the strong nuclear force, but now we think its quarks exchanging gluons
\label{sec:orga6ec71d}
\item there is a charged version of the pion which is made of a down quark and an anti-up quark (written \(d\bar u\))
\label{sec:orgaa9b33e}
\item the uncharged pion is made up of 'a mixture, pratly an up quark and an anti-up quark, partly a down quark and an anti-down quark, so we write its composition as \(u \bar u \& d \bar d\)' what the heck\hfill{}\textsc{question}
\label{sec:org20f0ca3}
\item some different mesons have the same composition (neutral pion and eta)
\label{sec:org277915a}
\item mesons can decay entirely into leptons (while baryons cannot because they must conserve their non-zero baryon number)
\label{sec:orga5d7e51}
\end{enumerate}
\end{enumerate}
\subsection{force carriers}
\label{sec:orged04540}
\subsubsection{physics is about things and what happens to them, and those particles were the things. these are what happens}
\label{sec:org97408d7}
\subsubsection{all are bosons (integer spins) and there are no conservation laws so they can do whatever they want}
\label{sec:orgfcb1935}
\begin{enumerate}
\item not even conservation of angular momentum/spin?\hfill{}\textsc{question}
\label{sec:org84f5e7a}
\end{enumerate}
\subsubsection{there are six particle types, with one for each of the fundamental forces except the weak force which has three}
\label{sec:org221211a}
\begin{enumerate}
\item graviton (the weakest force) (but it is the one we see the most because all the other forces can be negative are balanced (ex positive and negative charges yet most matter is mostly neutral))
\label{sec:org150fab9}
\begin{enumerate}
\item its so weak that it doesn't really do anything in the subatomic world (can be ignored)
\label{sec:org43c671d}
\begin{enumerate}
\item but maybe some weird quantum stuff makes it actually important on smaller scales
\label{sec:org8dcb313}
\end{enumerate}
\item not much to say apparently, it cant be v precisely measured bc its so weak
\label{sec:org27450f0}
\end{enumerate}
\item weak force carriers (W and Z)
\label{sec:orgfa2877f}
\begin{enumerate}
\item very massive but lacking physical size??\hfill{}\textsc{question}
\label{sec:orgd197991}
\item they are actually three close siblings (much like the positive, negative, and neutral versions of pions)
\label{sec:orgf636185}
\item discovered at proton synchotron at CERN
\label{sec:org3fe7820}
\end{enumerate}
\item photon
\label{sec:org1198e9f}
\begin{enumerate}
\item proposed by Einstein in 1905, but not seen as a real particle until the 1930s
\label{sec:orgf4dfb09}
\item actually zero mass and zero size
\label{sec:orgbc3f5f9}
\item force carrier for electromagnitism
\label{sec:org763ea6d}
\item electroweak theory
\label{sec:orgc161d2b}
\begin{enumerate}
\item says that electromagnetic and weak forces are the same thing but on different scales
\label{sec:orgfdfb5ad}
\item suggests heaver force carrier means weaker force and shorter range
\label{sec:orge07f740}
\item except the electromagnetic force only cares about charged particles while the weak force affects all particles
\label{sec:orgb6056ff}
\end{enumerate}
\end{enumerate}
\item strong interaction gluons
\label{sec:orgbfab2d4}
\begin{enumerate}
\item there are 8 of them, and 8 more antiparticles
\label{sec:org0599eb1}
\item made up of two colors (red antiblue or blue antigreen) for 2\textsuperscript{3} = 8 combinations
\label{sec:org694024a}
\begin{enumerate}
\item are colors allowed to repeat (red antired) and is it always normal-anti (or is antired-green diff from green antired)\hfill{}\textsc{question}
\label{sec:org61e896f}
\end{enumerate}
\item quarks change color when they interact with a gluon
\label{sec:org1f28e5a}
\item gluons can also directly exert force on one another (while photons can only interact with eachother via charged particles)
\label{sec:org901c27c}
\begin{enumerate}
\item is there an example of this\hfill{}\textsc{question}
\label{sec:orgeec0b76}
\end{enumerate}
\item strong force holds quarks within the particle, and gets stronger as distance increases
\label{sec:org8ca4152}
\item you can break a quark out of an atom, but the energy will become more quarks and antiquarks and you might get a pion out (bruh)
\label{sec:org472e88e}
\end{enumerate}
\end{enumerate}
\subsection{feynmann diagrams}
\label{sec:orgd5f108c}
\subsubsection{start with something like a spacetime diagram}
\label{sec:org07bbdd8}
\begin{enumerate}
\item the path tells us everything there is to know about a particle
\label{sec:org28ad25d}
\item but at events (ex. where the path changes directions) any number of things might happen
\label{sec:org040429c}
\item subatomic events seem to be instant and unsurvivable (nothing goes in and also comes out)
\label{sec:orgf84e358}
\item particles can also move backwards in time, apparently
\label{sec:org383b103}
\end{enumerate}
\subsubsection{features of the feynmann diagram}
\label{sec:orgf77ea42}
\begin{enumerate}
\item every vertex has three points, with two fermion lines (particles) and one boson line (force carrier)
\label{sec:org1ebbff1}
\begin{enumerate}
\item \textbf{every interaction in the universe is the result of two fermions and a boson meeting in these triads}
\label{sec:org0d03976}
\end{enumerate}
\item arrows denote anti-ness (a forwards-in-time positron is the same as a backwards-in-time electron)
\label{sec:orgd6f01bd}
\item conservation laws are consistent at each vertex
\label{sec:org1514537}
\item composite particles don't interact with bosons, so we draw them as their constituent quarks
\label{sec:org16ba6f8}
\end{enumerate}
\subsubsection{gluon exchange between quarks}
\label{sec:orgba4fe08}
\begin{enumerate}
\item gluon particles are determined by the quark pairs that they interact with? so a gluon being released and then absorbed by a diff quark essentially swaps the colors of the two quarks?\hfill{}\textsc{question}
\label{sec:orgc905d3a}
\item is there a difference between a red quark sending a blue quark a red antiblue gluon and a blue quark sending a red quark a red antiblue glueon?\hfill{}\textsc{question}
\label{sec:orge522415}
\item quarks don't change their type (up down strange charm top bottom) when getting glueoned
\label{sec:orgc79d830}
\end{enumerate}
\end{document}
