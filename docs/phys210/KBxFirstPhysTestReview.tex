% Created 2021-09-11 Sat 09:36
% Intended LaTeX compiler: xelatex
\documentclass[letterpaper]{article}
\usepackage{graphicx}
\usepackage{grffile}
\usepackage{longtable}
\usepackage{wrapfig}
\usepackage{rotating}
\usepackage[normalem]{ulem}
\usepackage{amsmath}
\usepackage{textcomp}
\usepackage{amssymb}
\usepackage{capt-of}
\usepackage{hyperref}
\usepackage[margin=1in]{geometry}
\usepackage{fontspec}
\usepackage{indentfirst}
\setmainfont[ItalicFont = LiberationSans-Italic, BoldFont = LiberationSans-Bold, BoldItalicFont = LiberationSans-BoldItalic]{LiberationSans}
\newfontfamily\NHLight[ItalicFont = LiberationSansNarrow-Italic, BoldFont       = LiberationSansNarrow-Bold, BoldItalicFont = LiberationSansNarrow-BoldItalic]{LiberationSansNarrow}
\newcommand\textrmlf[1]{{\NHLight#1}}
\newcommand\textitlf[1]{{\NHLight\itshape#1}}
\let\textbflf\textrm
\newcommand\textulf[1]{{\NHLight\bfseries#1}}
\newcommand\textuitlf[1]{{\NHLight\bfseries\itshape#1}}
\usepackage{fancyhdr}
\pagestyle{fancy}
\usepackage{titlesec}
\usepackage{titling}
\makeatletter
\lhead{\textbf{\@title}}
\makeatother
\rhead{\textrmlf{Compiled} \today}
\lfoot{\theauthor\ \textbullet \ \textbf{2021-2022}}
\cfoot{}
\rfoot{\textrmlf{Page} \thepage}
\titleformat{\section} {\Large} {\textrmlf{\thesection} {|}} {0.3em} {\textbf}
\titleformat{\subsection} {\large} {\textrmlf{\thesubsection} {|}} {0.2em} {\textbf}
\titleformat{\subsubsection} {\large} {\textrmlf{\thesubsubsection} {|}} {0.1em} {\textbf}
\setlength{\parskip}{0.45em}
\renewcommand\maketitle{}
\author{Huxley Marvit}
\date{\today}
\title{First Phys Test Review}
\hypersetup{
 pdfauthor={Huxley Marvit},
 pdftitle={First Phys Test Review},
 pdfkeywords={},
 pdfsubject={},
 pdfcreator={Emacs 27.2 (Org mode 9.4.4)}, 
 pdflang={English}}
\begin{document}

\maketitle
\#flo

\noindent\rule{\textwidth}{0.5pt}

\section{The cat, and Mary}
\label{sec:org8bdecd9}
seperation is one meter and charge is +/- one coulomb, the force is
around 9 bil newtons cat woudnt feel great\ldots{}

\begin{verbatim}
why use this ungodly amount to define a coulomb?
\end{verbatim}

open question

F=kq1q1/r\textsuperscript{2} F>0 like charges F<0 opposite charges

equations gives you a scalar, not directions. tells you if they attract
or repel

electrons inside a conductor slosh around like a fluid

\section{If the charges on a conductor are stationary or static}
\label{sec:orgeed3c50}
electrons try to reach equilibrium for them to be still, there must be
no forces -> no e-field

\section{electric fields}
\label{sec:org4e46872}
\begin{itemize}
\item are perpendicular to the surface, and "skin deep"

\begin{itemize}
\item charge is zero when u go in

\begin{itemize}
\item except when charge is flowing
\end{itemize}
\end{itemize}
\end{itemize}

\begin{verbatim}
if charges are moving, all bets are off
\end{verbatim}

fields curve near the end when fields on top arnt there to cancel

treat parrelel planes as infinite

\section{the pufferfish}
\label{sec:org913f016}
gauss, the second. gauss, the first, is in a drawer at lick

\begin{verbatim}
when your're dead you don't need friends
\end{verbatim}
\end{document}
