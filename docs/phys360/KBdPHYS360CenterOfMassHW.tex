% Created 2021-09-11 Sat 09:36
% Intended LaTeX compiler: xelatex
\documentclass[letterpaper]{article}
\usepackage{graphicx}
\usepackage{grffile}
\usepackage{longtable}
\usepackage{wrapfig}
\usepackage{rotating}
\usepackage[normalem]{ulem}
\usepackage{amsmath}
\usepackage{textcomp}
\usepackage{amssymb}
\usepackage{capt-of}
\usepackage{hyperref}
\usepackage[margin=1in]{geometry}
\usepackage{fontspec}
\usepackage{indentfirst}
\setmainfont[ItalicFont = LiberationSans-Italic, BoldFont = LiberationSans-Bold, BoldItalicFont = LiberationSans-BoldItalic]{LiberationSans}
\newfontfamily\NHLight[ItalicFont = LiberationSansNarrow-Italic, BoldFont       = LiberationSansNarrow-Bold, BoldItalicFont = LiberationSansNarrow-BoldItalic]{LiberationSansNarrow}
\newcommand\textrmlf[1]{{\NHLight#1}}
\newcommand\textitlf[1]{{\NHLight\itshape#1}}
\let\textbflf\textrm
\newcommand\textulf[1]{{\NHLight\bfseries#1}}
\newcommand\textuitlf[1]{{\NHLight\bfseries\itshape#1}}
\usepackage{fancyhdr}
\pagestyle{fancy}
\usepackage{titlesec}
\usepackage{titling}
\makeatletter
\lhead{\textbf{\@title}}
\makeatother
\rhead{\textrmlf{Compiled} \today}
\lfoot{\theauthor\ \textbullet \ \textbf{2021-2022}}
\cfoot{}
\rfoot{\textrmlf{Page} \thepage}
\titleformat{\section} {\Large} {\textrmlf{\thesection} {|}} {0.3em} {\textbf}
\titleformat{\subsection} {\large} {\textrmlf{\thesubsection} {|}} {0.2em} {\textbf}
\titleformat{\subsubsection} {\large} {\textrmlf{\thesubsubsection} {|}} {0.1em} {\textbf}
\setlength{\parskip}{0.45em}
\renewcommand\maketitle{}
\author{Dylan Wallace}
\date{\today}
\title{Adv Mech Center Of Mass HW}
\hypersetup{
 pdfauthor={Dylan Wallace},
 pdftitle={Adv Mech Center Of Mass HW},
 pdfkeywords={},
 pdfsubject={},
 pdfcreator={Emacs 27.2 (Org mode 9.4.4)}, 
 pdflang={English}}
\begin{document}

\maketitle


\section{Problem 1}
\label{sec:orgbce639a}
\subsection{\((1a)\)}
\label{sec:org1e860b8}
$\backslash$[
\begin{aligned}
PE &= -W \\
W &= \int_{R_e}^\infty F(r) \,dr \\
\end{aligned}
$\backslash$] We know that the force applied to a point mass \(m\) by the
gravitational field of the earth (with mass \(M_e\)) with distance \(x\)
is modeled by \[F(r) = \frac{GmM_e}{r^2}\]. Therefore, our work integral
can be modified to be $\backslash$[
\begin{aligned}
W &= \int_{R_e}^\infty \frac{GmM_e}{r^2}\,dr \\
&= GmM_e \int_{R_e}^\infty \frac{1}{r^2} \,dr \\
&= GmM_e [-\frac{1}{r}]_{R_e}^\infty \\
&= -\frac{GmM_e}{R_e} \\
PE &= \frac{GmM_e}{R_e}
\end{aligned}
$\backslash$]

\subsection{\((1b)\)}
\label{sec:orgab18575}
$\backslash$[
\begin{aligned}
KE &= \frac{1}{2}mv^2 \\
KE &= PE \\
\frac{1}{2}mv^2 &= \frac{GmM_e}{R_e} \\
v &= \sqrt{\frac{2GM_e}{R_e}}
\end{aligned}
$\backslash$]

\subsection{\((1c)\)}
\label{sec:orge6c5af7}
$\backslash$[
\begin{aligned}
v &= \sqrt{\frac{2GM_e}{R_e}} \\
&= \sqrt{\frac{2\cdot 6.674 \cdot 10^{-11} \cdot 5.972 × 10^{24}}{6.371\cdot 10^{6}}} \\
&= 11185.7 m/s \\
&= 25020.1 mph \\
\end{aligned}
$\backslash$]

\section{Problem 2}
\label{sec:org062492c}
$\backslash$[
\begin{aligned}
\sum_{i=1}^{n} \vec{F}_{net,i} &= (\sum_{i=1}^{n} m_i) \ddot{\vec{r}}_{CM} \\
\sum_{i=1}^{n} m_i \ddot{\vec{r}}_{i} &= (\sum_{i=1}^{n} m_i) \ddot{\vec{r}}_{CM} \\
\int \int \sum_{i=1}^{n} m_i \ddot{\vec{r}}_{i} \,dt\,dt &= \int \int (\sum_{i=1}^{n} m_i) \ddot{\vec{r}}_{CM} \,dt\,dt \\
\int \sum_{i=1}^{n} m_i \dot{\vec{r}}_{i} \,dt + C_1 &= \int (\sum_{i=1}^{n} m_i) \dot{\vec{r}}_{CM} \,dt + C_1 \\
\sum_{i=1}^{n} m_i \vec{r}_{i} + C_1t + C_2 &= (\sum_{i=1}^{n} m_i) \vec{r}_{CM} + C_1t + C_2 \\
\end{aligned}
$\backslash$] Both constants are the same constant on both sides of the equation so
they will cancel out. The sum of all mass is just \(M\). $\backslash$[
\begin{aligned}
\vec{r}_{CM} &= \frac{1}{M} \sum_{i=1}^{n} m_i \vec{r}_{i} \\
\end{aligned}
$\backslash$]

\section{Problem 3}
\label{sec:orgebc406b}
Any force within a system will have an opposite force applied as well
(Newton's 3rd law). Therefore, forces within a system will cancel out
and will have no effect on the center of mass.

\section{Problem 4}
\label{sec:org78b2a66}
$\backslash$[
\begin{aligned}
\vec{v} &= \frac{<1, −4, 1> + 2<−3, −2, 6> + 3<2, 5, −3> + 4<−2, 4, 6>}{1 + 2 + 3 + 4} \\
&= <-0.7, 2.3, 2.8> \\
\end{aligned}
$\backslash$]

\href{Screen Shot 2021-09-05 at 7.09.00 PM.png.org}{Screen Shot
2021-09-05 at 7.09.00 PM.png}
\end{document}
