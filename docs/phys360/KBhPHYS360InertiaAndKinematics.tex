% Created 2021-09-11 Sat 11:24
% Intended LaTeX compiler: xelatex
\documentclass[letterpaper]{article}
\usepackage{graphicx}
\usepackage{grffile}
\usepackage{longtable}
\usepackage{wrapfig}
\usepackage{rotating}
\usepackage[normalem]{ulem}
\usepackage{amsmath}
\usepackage{textcomp}
\usepackage{amssymb}
\usepackage{capt-of}
\usepackage{hyperref}
\usepackage[margin=1in]{geometry}
\usepackage{fontspec}
\usepackage{indentfirst}
\setmainfont[ItalicFont = LiberationSans-Italic, BoldFont = LiberationSans-Bold, BoldItalicFont = LiberationSans-BoldItalic]{LiberationSans}
\newfontfamily\NHLight[ItalicFont = LiberationSansNarrow-Italic, BoldFont       = LiberationSansNarrow-Bold, BoldItalicFont = LiberationSansNarrow-BoldItalic]{LiberationSansNarrow}
\newcommand\textrmlf[1]{{\NHLight#1}}
\newcommand\textitlf[1]{{\NHLight\itshape#1}}
\let\textbflf\textrm
\newcommand\textulf[1]{{\NHLight\bfseries#1}}
\newcommand\textuitlf[1]{{\NHLight\bfseries\itshape#1}}
\usepackage{fancyhdr}
\pagestyle{fancy}
\usepackage{titlesec}
\usepackage{titling}
\makeatletter
\lhead{\textbf{\@title}}
\makeatother
\rhead{\textrmlf{Compiled} \today}
\lfoot{\theauthor\ \textbullet \ \textbf{2021-2022}}
\cfoot{}
\rfoot{\textrmlf{Page} \thepage}
\titleformat{\section} {\Large} {\textrmlf{\thesection} {|}} {0.3em} {\textbf}
\titleformat{\subsection} {\large} {\textrmlf{\thesubsection} {|}} {0.2em} {\textbf}
\titleformat{\subsubsection} {\large} {\textrmlf{\thesubsubsection} {|}} {0.1em} {\textbf}
\setlength{\parskip}{0.45em}
\renewcommand\maketitle{}
\author{Houjun Liu}
\date{\today}
\title{Rotational Inertia and Kinematics}
\hypersetup{
 pdfauthor={Houjun Liu},
 pdftitle={Rotational Inertia and Kinematics},
 pdfkeywords={},
 pdfsubject={},
 pdfcreator={Emacs 27.2 (Org mode 9.4.4)}, 
 pdflang={English}}
\begin{document}

\maketitle
\index{PHYS360!Derivations!Rotational Inertia and Kinematics}

\section{Deriving Rotational KE and Inertia}
\label{sec:org0019b9a}
Given \(m_i\), mass, \(\vec{r_i}'\), location of the center of mass, \(l_i\), \(\omega\), the angular velocity, figure a \(KE_{tot,rot}\). 

Because of the fact that the value \(\omega\) is in units \(\frac{d\theta}{dt}\), the rate of radians change, and we know of a radius of the spin \(l_i\), we could figure the velocity at which it is moving by simply scaling the change in radians up to a circle of radius \(l_i\), that is:

\begin{equation}
    V_i' = l_i \omega 
\end{equation}

(note that, to understand this, radians \(\frac{arc length}{radius}\))

And so, substituting into the statement of \(\sum^N_{i=1} \frac{1}{2}m_i\vec{v_i}'^2\)

\begin{align}
    KE_{rot} =& \sum^N_{i=1} \frac{1}{2}m_i\vec{v_i}'^2 \\
    =& \sum^N_{i=1} \frac{1}{2}m_i(l_i \omega)^2 \\
    =& \sum^N_{i=1} \frac{1}{2}m_i l_i^2 \omega^2 \\
    =& \frac{1}{2}\omega^2 \sum^N_{i=1} (m_i l_i^2)
\end{align}

\subsection{Rotational Inertia}
\label{sec:orgafb735c}
The right sum --- the mass times the distance away from maxis of rotation (\(\sum^N_{i=1} (m_i l_i^2)\)) --- is defined as the rotational (moment) of inertia (spinny mass). That is,

\begin{equation}
    I = \sum^N_{i=1} (m_i l_i^2)
\end{equation}

Replacing that value in the prior statement, the statement of \(KE_{rot}\) is defined as:

\begin{equation}
    KE_{rot} = \frac{1}{2}\omega^2I
\end{equation}


\subsection{Rotational Inertia for a Ring}
\label{sec:orgeba5487}
For a ring (that's perfectly circular) rotating on an axis perpendicular to the plane of the ring, the \(l_i\) --- distance from axis of rotation --- is the same value: namely, the radius \(R\) as the radius of a circle is the same for all positions. Meaning,

\begin{equation}
    l_i = R
\end{equation}

regardless of which value \(i\).

Hence, the value of \(KE_{rot}\) would be evaluated as\ldots{}

\begin{align}
    KE_{rot} =& \sum^N_{i=1}(m_il^2_i) \\
    =& \sum^N_{i=1}(m_iR^2) \\
    =& R^2 \sum^N_{i=1}m_i \\
\end{align}

Substituting \(M\) as the sum of all masses in the ring (\(M=\sum^N_{i=1}m_i\)), the statement is therefore:

\begin{equation}
    KE_{rot} = MR^2
\end{equation}

\subsection{Rotational Inertia of a Solid Sphere}
\label{sec:org15c6823}
I believe that the rotational inertia of \(I_{sphere}\) to be less than \(I_{disk}\). This is because, as the dimension of the object increases, it would be easier to change its velocity (a disk is easier to spin than a ring, etc.). Hence, my intuition states that \(I_{sphere}\) would be lower than \(I_{disk}\).

Mathematically, as \(M\) is staying at the same value, in the disk case has more mass closer to the axis of rotation --- meaning that the \(m_iR^2\) term would be smaller in more of the point masses than that of an object at a lower dimension. Hence, the sphere would have more points with lower \(m_iR^2\) terms than that of disk; hence, \(I_{sphere}\) would be less than \(I_{disk}\).

\section{Kinematics Equations}
\label{sec:org1e44f1a}
Given \(a=a_0\), initial velocity \(v_0\), and position \(y_0\), we derive the kinematics equations.

\begin{align}
    a(t) =& a_0 \\
    \int a(t) dt =& \int a_0 dt \\
    v(t) =& a_0t + C 
\end{align}

We are given that \(v(0)=v_0\). \(v(0) = C = v_0\), hence, \(C=v_0\). The velocity statement is therefore,

\begin{equation}
    v(t) = a_0x+v_0
\end{equation}

Continuing with integration:

\begin{align}
    v(t) =& a_0x + v_0 \\
    \int v(t) =& \int a_0x + v_0 dt \\
    y(t) =& \frac{1}{2}a_0x^2+v_0x+C \\
\end{align}

Again, substituting \(C = y_0\) by the same logic above --- \(y(0) = C = y_0\), we derive the statement for the position equation.

\begin{equation}
    y(t) = \frac{1}{2}a_0x^2 + v_0x + y_0
\end{equation}


\subsection{Proving \(v^2(t) = v_0^2 + 2a_0(y(t)-y_0)\)}
\label{sec:orgc95471b}
We start at the statement for \(v(t)\), squaring it, and substituting the necessary statements.

\begin{align}
    v(t) =& a_0x+v_0 \\
    \Rightarrow v^2(t) =& a_0^2 x^2 + 2a_0v_0x + v_0^2 \\
    v^2(t) =& v_0^2 + 2a_0 (\frac{1}{2} a_0 x^2 + v_0x) \\
    v^2(t) =& v_0^2 + 2a_0 (\frac{1}{2} a_0 x^2 + v_0x + y_0 - y_0) \\
    v^2(t) =& v_0^2 + 2a_0 (y(t) - y_0) 
\end{align}

It is therefore shown that:

\begin{equation}
    v^2(t) = v_0^2 + 2a_0 (y(t) - y_0) 
\end{equation}
\end{document}
