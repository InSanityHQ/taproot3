% Created 2021-09-11 Sat 09:36
% Intended LaTeX compiler: xelatex
\documentclass[letterpaper]{article}
\usepackage{graphicx}
\usepackage{grffile}
\usepackage{longtable}
\usepackage{wrapfig}
\usepackage{rotating}
\usepackage[normalem]{ulem}
\usepackage{amsmath}
\usepackage{textcomp}
\usepackage{amssymb}
\usepackage{capt-of}
\usepackage{hyperref}
\usepackage[margin=1in]{geometry}
\usepackage{fontspec}
\usepackage{indentfirst}
\setmainfont[ItalicFont = LiberationSans-Italic, BoldFont = LiberationSans-Bold, BoldItalicFont = LiberationSans-BoldItalic]{LiberationSans}
\newfontfamily\NHLight[ItalicFont = LiberationSansNarrow-Italic, BoldFont       = LiberationSansNarrow-Bold, BoldItalicFont = LiberationSansNarrow-BoldItalic]{LiberationSansNarrow}
\newcommand\textrmlf[1]{{\NHLight#1}}
\newcommand\textitlf[1]{{\NHLight\itshape#1}}
\let\textbflf\textrm
\newcommand\textulf[1]{{\NHLight\bfseries#1}}
\newcommand\textuitlf[1]{{\NHLight\bfseries\itshape#1}}
\usepackage{fancyhdr}
\pagestyle{fancy}
\usepackage{titlesec}
\usepackage{titling}
\makeatletter
\lhead{\textbf{\@title}}
\makeatother
\rhead{\textrmlf{Compiled} \today}
\lfoot{\theauthor\ \textbullet \ \textbf{2021-2022}}
\cfoot{}
\rfoot{\textrmlf{Page} \thepage}
\titleformat{\section} {\Large} {\textrmlf{\thesection} {|}} {0.3em} {\textbf}
\titleformat{\subsection} {\large} {\textrmlf{\thesubsection} {|}} {0.2em} {\textbf}
\titleformat{\subsubsection} {\large} {\textrmlf{\thesubsubsection} {|}} {0.1em} {\textbf}
\setlength{\parskip}{0.45em}
\renewcommand\maketitle{}
\author{Houjun Liu}
\date{\today}
\title{Deriving Rotational Energy}
\hypersetup{
 pdfauthor={Houjun Liu},
 pdftitle={Deriving Rotational Energy},
 pdfkeywords={},
 pdfsubject={},
 pdfcreator={Emacs 27.2 (Org mode 9.4.4)}, 
 pdflang={English}}
\begin{document}

\maketitle


\section{Position of \(m_i\)}
\label{sec:org8ae144f}
In a rigid body consisting of \(N\) point masses, the vector to the position of \(m_i\) is defined as \(\vec{r_i(t)}\), which is defined as follows:

\begin{equation}
    \vec{r_i(t)} = \vec{R_{CM}(t)} + \vec{r_i}'(t)
\end{equation}

whereas, \(\vec{R_{CM}(t)}\) is the position vector of the center of mass of the rigid body as a whole, and \(\vec{r_i}'(t)\) the vector from the center of mass to \(m_i\).

\section{Velocity of \(m_i\)}
\label{sec:orgf605b58}
The velocity of \(m_i\) is simply determined by the first derivative of the position equation as per above. Namely, that:

\begin{equation}
    \vec{v_i(t)} = \vec{V_{CM}(t)} + \vec{v_i}'(t)
\end{equation}

where, \(\vec{v_i(t)}\) is the velocity vector of \(m_i\), and \(\vec{V_{CM}(t)}\) is the velocity vector of the center of mass of the rigid body, and \(\vec{v_i}'(t)\) is the velocity vector from center of mass to \(m_i\).

\section{Deriving \(KE_{total}\)}
\label{sec:org161160d}

\subsection{Setting up}
\label{sec:orga5d3b3f}
From definition of \(KE_{total}\) itself, \(KE_{total}\) is the sum of all energies of each point mass in the rigid body.

\begin{equation}
    \sum^N_{i=1} \frac{1}{2}m_iv_i^2
\end{equation}


\subsection{Derivation, part 1}
\label{sec:org21fac2d}
Expanding this equation and substituting the value of \(v_i\), and additionally setting \(M = \sum m_i\) (namely, that \(M\) represents the total mass of the rigid body) we could derive:

\begin{align}
    \sum^N_{i=1} \frac{1}{2}m_iv_i^2 =& \sum^N_{i=1} \frac{1}{2}m_i(v_i \cdot v_i) \\
    =& \sum^N_{i=1} \frac{1}{2}m_i((\vec{V_{CM}} + \vec{v_i}') \cdot (\vec{V_{CM}} + \vec{v_i}')) \\
    =& \sum^N_{i=1} \frac{1}{2}m_i(\vec{V_{CM}}^2 + 2 \times (\vec{v_i}' \cdot \vec{V_{CM}}) + \vec{v_i}'^2)) \\
    =& \sum^N_{i=1} \frac{1}{2}m_i\vec{V_{CM}}^2 + \sum^N_{i=1} m_i \times (\vec{v_i}' \cdot \vec{V_{CM}}) + \sum^N_{i=1} \frac{1}{2}m_i\vec{v_i}'^2 \\
    =& \frac{1}{2} \vec{V_{CM}}^2 \sum^N_{i=1} m_i + \vec{V_{CM}} \sum^N_{i=1} m_i \vec{v_i}' + \sum^N_{i=1} \frac{1}{2}m_i\vec{v_i}'^2
\end{align}

\subsection{Dealing with the Middle Term}
\label{sec:org5511afe}
At this point, we must note that \(\sum^N_{i=1} m_i \vec{v_i}' = 0\). Per the definition of the center of mass, the following holds:

\begin{equation}
    \vec{r_{CM}} = (\frac{1}{M}) \sum_i m_i \vec{r_i}
\end{equation}

Changing reference frame to that of the center of mass itself, this equation therefore becomes:

\begin{equation}
    \vec{r_{CM}}' = (\frac{1}{M}) \sum_i m_i \vec{r_i}'
\end{equation}

It is important to realize here that \(\vec{r_{CM}}' = 0\) because of the fact that --- at the reference point of the center of mass, the center of mass is at a zero-vector distance away from itself.

In order to figure a statement with respect to the \emph{velocity} of \(r_i'\), we take the derivative of the previous equation with respect to time.

\begin{align}
    0 =& (\frac{1}{M}) \sum_i m_i \vec{r_i}' \\
    \Rightarrow& \frac{d}{dt} (\frac{1}{M}) \sum_i m_i \vec{r_i}' \\
    =& (\frac{1}{M}) \sum_i m_i \vec{v_i}'
\end{align}

Given that \(\frac{1}{M}\) could not be zero for an object with non-zero mass, it is concluded therefore that \(\sum_i m_i \vec{v_i}' = 0\).

\subsection{Derivation, part 2}
\label{sec:orge4d1c47}
As \(\sum_i m_i \vec{v_i}' = 0\), the \(KE_{total}\) work-in-progress equation's middle term (which contains the statement \(\sum_i m_i \vec{v_i}'\)) is therefore zero. Substituting that in and removing the term, we therefore result in:

\begin{equation}
     \sum^N_{i=1} \frac{1}{2}m_iv_i^2 = \frac{1}{2} \vec{V_{CM}}^2 \sum^N_{i=1} m_i + \sum^N_{i=1} \frac{1}{2}m_i\vec{v_i}'^2
\end{equation}

Replacing the definition of \(M = \sum m_i\), we result in

\begin{align}
     \sum^N_{i=1} \frac{1}{2}m_iv_i^2 &= \frac{1}{2} M \vec{V_{CM}}^2 + \sum^N_{i=1} \frac{1}{2}m_i\vec{v_i}'^2 \\
     KE_{total} &= \frac{1}{2} M \vec{V_{CM}}^2 + \sum^N_{i=1} \frac{1}{2}m_i\vec{v_i}'^2
\end{align}

The left term of this equation (\(\frac{1}{2} M \vec{V_{CM}}^2\)) is the clear original statement for \(KE_{translational}\). As component masses of a rigid body cannot experience translational motion about its center of origin, the second term is therefore rotational only and so \(KE_{rotational}\). 

Therefore:

\begin{equation}
    KE_{total} = KE_{translational}+KE_{rotational}
\end{equation}
\end{document}
