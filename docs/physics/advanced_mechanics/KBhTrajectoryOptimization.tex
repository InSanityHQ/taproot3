% Created 2021-09-22 Wed 21:47
% Intended LaTeX compiler: xelatex
\documentclass[letterpaper]{article}
\usepackage{graphicx}
\usepackage{grffile}
\usepackage{longtable}
\usepackage{wrapfig}
\usepackage{rotating}
\usepackage[normalem]{ulem}
\usepackage{amsmath}
\usepackage{textcomp}
\usepackage{amssymb}
\usepackage{capt-of}
\usepackage{hyperref}
\setlength{\parindent}{0pt}
\usepackage[margin=1in]{geometry}
\usepackage{fontspec}
\usepackage{svg}
\usepackage{cancel}
\usepackage{indentfirst}
\setmainfont[ItalicFont = LiberationSans-Italic, BoldFont = LiberationSans-Bold, BoldItalicFont = LiberationSans-BoldItalic]{LiberationSans}
\newfontfamily\NHLight[ItalicFont = LiberationSansNarrow-Italic, BoldFont       = LiberationSansNarrow-Bold, BoldItalicFont = LiberationSansNarrow-BoldItalic]{LiberationSansNarrow}
\newcommand\textrmlf[1]{{\NHLight#1}}
\newcommand\textitlf[1]{{\NHLight\itshape#1}}
\let\textbflf\textrm
\newcommand\textulf[1]{{\NHLight\bfseries#1}}
\newcommand\textuitlf[1]{{\NHLight\bfseries\itshape#1}}
\usepackage{fancyhdr}
\pagestyle{fancy}
\usepackage{titlesec}
\usepackage{titling}
\makeatletter
\lhead{\textbf{\@title}}
\makeatother
\rhead{\textrmlf{Compiled} \today}
\lfoot{\theauthor\ \textbullet \ \textbf{2021-2022}}
\cfoot{}
\rfoot{\textrmlf{Page} \thepage}
\titleformat{\section} {\Large} {\textrmlf{\thesection} {|}} {0.3em} {\textbf}
\titleformat{\subsection} {\large} {\textrmlf{\thesubsection} {|}} {0.2em} {\textbf}
\titleformat{\subsubsection} {\large} {\textrmlf{\thesubsubsection} {|}} {0.1em} {\textbf}
\setlength{\parskip}{0.45em}
\renewcommand\maketitle{}
\author{Houjun Liu}
\date{\today}
\title{Optimizing the Trajectory of a Bead Shooter}
\hypersetup{
 pdfauthor={Houjun Liu},
 pdftitle={Optimizing the Trajectory of a Bead Shooter},
 pdfkeywords={},
 pdfsubject={},
 pdfcreator={Emacs 28.0.50 (Org mode 9.4.4)}, 
 pdflang={English}}
\begin{document}

\maketitle
We begin by defining a system

\(\theta\) is the angle by which the shooter is aimed, the shooter shoots at \(v_0\), the projectile travels a distance of \(R\).

So, define a function \(R(\theta) = R\).

Hence, the goal of this project is to find local mix, min points (critical points that arn't inflection points), which means --- at a minimum\ldots{}

vs

\begin{equation}
    \frac{dR}{d\theta} = 0
\end{equation}

which would therefore indicate a \(\theta\) such that the distance would be the longest.

Hence, to get the longest distance, solve.

There was apparently \href{https://www.notion.so/shabangsystems/Projectiles-Trajectories-d9d491162e6844f9aefd2cc6dda8d334}{my old notes} on this. But not sure if its helpful.


\begin{align}
    &y(t), y_0=0, y_f=0 \\
&x(t), x_0=0, y_f=R \\
y(t) =& \frac{-1}{2} gt^2 + V_0_y t + y_0, V_0_y =  V_0 \sin\theta \\
y(t) =& \frac{-1}{2} gt^2 + V_0 \sin\theta t + y_0\\
x(t) =& 0 (g=0) + V_0_x t + x_0, V_0_x = V_0 \cos\theta  \\
x(f) =& 0 (g=0) + V_0 \cos\theta t + x_0 \\
0\ (end\ up\ on\ ground) = y_f = y(t_f) =& -\frac{1}{2}g t_f^2 + (v_0\sin\theta)t_f \\
R\ (want\ to\ travel\ R) = x_f = x(t_f) =& (v_0\cos\theta)t_f \\
\end{align}
\end{document}
