% Created 2021-09-27 Mon 12:02
% Intended LaTeX compiler: xelatex
\documentclass[letterpaper]{article}
\usepackage{graphicx}
\usepackage{grffile}
\usepackage{longtable}
\usepackage{wrapfig}
\usepackage{rotating}
\usepackage[normalem]{ulem}
\usepackage{amsmath}
\usepackage{textcomp}
\usepackage{amssymb}
\usepackage{capt-of}
\usepackage{hyperref}
\setlength{\parindent}{0pt}
\usepackage[margin=1in]{geometry}
\usepackage{fontspec}
\usepackage{svg}
\usepackage{cancel}
\usepackage{indentfirst}
\setmainfont[ItalicFont = LiberationSans-Italic, BoldFont = LiberationSans-Bold, BoldItalicFont = LiberationSans-BoldItalic]{LiberationSans}
\newfontfamily\NHLight[ItalicFont = LiberationSansNarrow-Italic, BoldFont       = LiberationSansNarrow-Bold, BoldItalicFont = LiberationSansNarrow-BoldItalic]{LiberationSansNarrow}
\newcommand\textrmlf[1]{{\NHLight#1}}
\newcommand\textitlf[1]{{\NHLight\itshape#1}}
\let\textbflf\textrm
\newcommand\textulf[1]{{\NHLight\bfseries#1}}
\newcommand\textuitlf[1]{{\NHLight\bfseries\itshape#1}}
\usepackage{fancyhdr}
\pagestyle{fancy}
\usepackage{titlesec}
\usepackage{titling}
\makeatletter
\lhead{\textbf{\@title}}
\makeatother
\rhead{\textrmlf{Compiled} \today}
\lfoot{\theauthor\ \textbullet \ \textbf{2021-2022}}
\cfoot{}
\rfoot{\textrmlf{Page} \thepage}
\renewcommand{\tableofcontents}{}
\titleformat{\section} {\Large} {\textrmlf{\thesection} {|}} {0.3em} {\textbf}
\titleformat{\subsection} {\large} {\textrmlf{\thesubsection} {|}} {0.2em} {\textbf}
\titleformat{\subsubsection} {\large} {\textrmlf{\thesubsubsection} {|}} {0.1em} {\textbf}
\setlength{\parskip}{0.45em}
\renewcommand\maketitle{}
\author{Exr0n}
\date{\today}
\title{Electrostatics Cheat Sheet}
\hypersetup{
 pdfauthor={Exr0n},
 pdftitle={Electrostatics Cheat Sheet},
 pdfkeywords={},
 pdfsubject={},
 pdfcreator={Emacs 28.0.50 (Org mode 9.4.4)}, 
 pdflang={English}}
\begin{document}

\tableofcontents



\section{Electrostatics}
\label{sec:org400db1c}
\subsection{Conduction vs Insulation}
\label{sec:org0da512f}
\begin{itemize}
\item Charge can flow through or over the surface of conductors:

\begin{itemize}
\item Metals, graphite, plasma
\end{itemize}

\item Insulators do not allow charge to flow along or through them.
\end{itemize}

\subsection{Transferred and Induced Charges}
\label{sec:org0b207eb}
\begin{itemize}
\item Charge can jump from a charged object to an uncharged object,
sometimes through insulators depending on voltage.
\item A charged object can induce a temporary charge \textbf{migration} in an
uncharged object, but the entire object is still neutral.
\end{itemize}

\subsection{Coulomb's Law}
\label{sec:org8c158a1}
\(F\vec{F} = \frac{1}{4\pi\epsilon_0}\left(\frac{q_1 q_2}{r^2}\right) = k\frac{q_1 q_2}{r^2}\)

\(k = 8.99_{x10^5} \frac{N m^2}{C^2}\)

\begin{center}
\begin{tabular}{lll}
Variable & Units & Description\\
\hline
\(q_1\), \(q_2\) & Coulomb (\(C\)) & The charge of each particle\\
\(r\) & Meters (\(m\)) & Distance between centers of charges\\
\end{tabular}
\end{center}

\(\epsilon_0\) and \(k\) are different ways of representing the
constant.

\subsubsection{Signs}
\label{sec:orga165d41}
\textbf{Be very careful with signs:}

If \(\vec{F} < 0\), charges repel each other.

If \(\vec{F} > 0\), charges attract each other.

\subsubsection{Multiple Charges}
\label{sec:org0fe6445}
You have to calculate each pairwise charge, and then add them up for
each particle. This is normal (vector) addition, so you can actually add
them (to get a vector field) and then apply it to a test particle
directly.

\subsubsection{Fields}
\label{sec:org78c90b4}
\(F_{elec} = k \frac{Q_1 Q_2}{R^2}\)

\(F_{grav} = G \frac{M_1 M_2}{R^2}\)

\(F_{elec} = \bf{E} Q_1\)

You can add fields together component-wise to get a combined field from
multiple charged particles.

\noindent\rule{\textwidth}{0.5pt}
\end{document}
