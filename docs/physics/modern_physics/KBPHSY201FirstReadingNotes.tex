% Created 2021-09-11 Sat 16:41
% Intended LaTeX compiler: xelatex
\documentclass[letterpaper]{article}
\usepackage{graphicx}
\usepackage{grffile}
\usepackage{longtable}
\usepackage{wrapfig}
\usepackage{rotating}
\usepackage[normalem]{ulem}
\usepackage{amsmath}
\usepackage{textcomp}
\usepackage{amssymb}
\usepackage{capt-of}
\usepackage{hyperref}
\usepackage[margin=1in]{geometry}
\usepackage{fontspec}
\usepackage{indentfirst}
\setmainfont[ItalicFont = LiberationSans-Italic, BoldFont = LiberationSans-Bold, BoldItalicFont = LiberationSans-BoldItalic]{LiberationSans}
\newfontfamily\NHLight[ItalicFont = LiberationSansNarrow-Italic, BoldFont       = LiberationSansNarrow-Bold, BoldItalicFont = LiberationSansNarrow-BoldItalic]{LiberationSansNarrow}
\newcommand\textrmlf[1]{{\NHLight#1}}
\newcommand\textitlf[1]{{\NHLight\itshape#1}}
\let\textbflf\textrm
\newcommand\textulf[1]{{\NHLight\bfseries#1}}
\newcommand\textuitlf[1]{{\NHLight\bfseries\itshape#1}}
\usepackage{fancyhdr}
\pagestyle{fancy}
\usepackage{titlesec}
\usepackage{titling}
\makeatletter
\lhead{\textbf{\@title}}
\makeatother
\rhead{\textrmlf{Compiled} \today}
\lfoot{\theauthor\ \textbullet \ \textbf{2021-2022}}
\cfoot{}
\rfoot{\textrmlf{Page} \thepage}
\titleformat{\section} {\Large} {\textrmlf{\thesection} {|}} {0.3em} {\textbf}
\titleformat{\subsection} {\large} {\textrmlf{\thesubsection} {|}} {0.2em} {\textbf}
\titleformat{\subsubsection} {\large} {\textrmlf{\thesubsubsection} {|}} {0.1em} {\textbf}
\setlength{\parskip}{0.45em}
\renewcommand\maketitle{}
\date{\today}
\title{}
\hypersetup{
 pdfauthor={},
 pdftitle={},
 pdfkeywords={},
 pdfsubject={},
 pdfcreator={Emacs 27.2 (Org mode 9.4.4)}, 
 pdflang={English}}
\begin{document}

\begin{center}
\begin{tabular}{l}
title: Zachary First Reading Notes\\
author: Zachary Sayyah\\
course: PHYS201\\
source: \href{KBhPHYS201QuantumWorldBookNotesIndex.org}{KBhPHYS201QuantumWorldBookNotesIndex}\\
\end{tabular}
\end{center}

\section{How to Deal with Large and Small}
\label{sec:org7a34b93}
\begin{itemize}
\item Scientific notation is required to deal with large and small
quantities

\begin{itemize}
\item This is required in much of particle physics since particles tend to
be very small and fast
\end{itemize}

\item People also tend to create more fitting units for a specific
application
\end{itemize}

\subsection{Units}
\label{sec:org5d823b6}
\textbf{Fentometers} are used as a unit of measurement in the atomic world.
They're \(10^{15}m\).

For \textbf{speed} we use fractions of the speed of light c \textasciitilde{}\(3*10^{8}m/s\)

\textbf{Volts} are used for charge.

\textbf{Particle Masses} can also be expressed in eV units. Particle masses are
actually pretty large with the eV unit.

\textbf{Planck Size} is about \(10^{-35}m\)

\textbf{Angular Momentum} can be measured in h-bars which are Planck's constant
divided by 2π

\subsection{Relative Scales Distance}
\label{sec:org8c0caf0}
\begin{itemize}
\item The nucleus takes up a very small amount of a particle

\begin{itemize}
\item Comparison drawn here is a basketball in an airport for a large
nucleus and a golf ball for smaller ones
\end{itemize}

\item Electrons occupy in a probability distribution the rest of the space
more or less
\item The only viable way to measure distances that small are through
scattering experiments involving shooting electrons at say a proton
and observing the scatter pattern

\begin{itemize}
\item The diameter of 1 proton is approximately 1 fermi
\end{itemize}

\item We live in a relative distance average
\item Short Wavelengths can also be observed to estimate the size of such
small particles
\item The Planck size is the smallest meaningful distance before spacetime
breaks down into quantum foam
\end{itemize}

\subsection{Relative Scales Speed}
\label{sec:org7629e51}
\begin{itemize}
\item The fastest anything can go so far as we know is the speed of light
\item It's hard to get anything close to the speed of light, but for stuff
like particle accelerators and cosmic rays it isn't super uncommon to
get close
\item Mass being the reluctance to accelerate means that the mass-less
photon should be the fastest particle requiring no energy to reach the
speed of light. For anything to go faster would be difficult.

\begin{itemize}
\item However, physicists have studied the Tachyon which is theoretically
capable of doing so but has not been discovered and also creates
strange circumstances
\end{itemize}
\end{itemize}

\subsection{Relative Scales Time}
\label{sec:orgd9575b9}
\begin{itemize}
\item The longest known time is the lifespan of the universe

\begin{itemize}
\item This is currently estimated to be about 13.7 billion years
\end{itemize}

\item The speed of light is the natural link between distance and time
measurements
\end{itemize}

\subsection{Relative Scales Mass}
\label{sec:org10e10a5}
\begin{itemize}
\item Mass is a measure of inertia meaning how hard something is to
accelerate
\item We measure particle's speed by measuring their resistance to
acceleration with knowledge of their speed
\item With particle masses it becomes more sensical to use MeV instead of kg
since the units make more sense
\end{itemize}

\subsection{Relative Scales Energy}
\label{sec:orgf12059d}
\begin{itemize}
\item Energy and its conservation make it perhaps one of the most important
things in physics
\item Kinetic energy and mass energy are the most important types when it
comes to particles

\begin{itemize}
\item Rest mas is different from mass
\end{itemize}

\item Mass represents a highly concentrated form of energy

\begin{itemize}
\item A little mass leads to a lot of energy while a lot of energy can
yield a little mass
\end{itemize}

\item In the subatomic world mass and energy are typically both measured
using the electron volt
\end{itemize}

\subsection{Relative Scales Charge}
\label{sec:org1febcfb}
\begin{itemize}
\item Electric charge is that thing that makes a particle attractive to
another type of particle
\item If the Gluons are overcome by the repulsion of the protons a nucleus
will break apart

\begin{itemize}
\item This is why there is a cap for how large an atom can be
realistically since it would require too much energy to keep
together than the gluons can offer
\end{itemize}

\item Negative and positive is entirely arbitrary they are just opposites
\end{itemize}

\subsection{Relative Scales Spin}
\label{sec:org58c77b6}
\begin{itemize}
\item Spin occurs with anything from the largest galaxies down to the
smallest particles
\item Angular momentum is used to measure both orbital motion and rotation
on one's own axis

\begin{itemize}
\item Fundemental particles have measurable angular momentum, but a rate
of rotation cannot be specified

\begin{itemize}
\item Planck's constant divided by 2π is the fundamental quantum unit of
rotation
\end{itemize}
\end{itemize}

\item Difference in spin is drastic enough for us to call particles with
different spin new particles
\item All electrons have the same spin

\begin{itemize}
\item Spin is quantized and things such as electrons are either "up" or
"down"
\end{itemize}
\end{itemize}
\end{document}
