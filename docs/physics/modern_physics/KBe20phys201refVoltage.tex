% Created 2021-09-12 Sun 22:49
% Intended LaTeX compiler: xelatex
\documentclass[letterpaper]{article}
\usepackage{graphicx}
\usepackage{grffile}
\usepackage{longtable}
\usepackage{wrapfig}
\usepackage{rotating}
\usepackage[normalem]{ulem}
\usepackage{amsmath}
\usepackage{textcomp}
\usepackage{amssymb}
\usepackage{capt-of}
\usepackage{hyperref}
\usepackage[margin=1in]{geometry}
\usepackage{fontspec}
\usepackage{indentfirst}
\setmainfont[ItalicFont = LiberationSans-Italic, BoldFont = LiberationSans-Bold, BoldItalicFont = LiberationSans-BoldItalic]{LiberationSans}
\newfontfamily\NHLight[ItalicFont = LiberationSansNarrow-Italic, BoldFont       = LiberationSansNarrow-Bold, BoldItalicFont = LiberationSansNarrow-BoldItalic]{LiberationSansNarrow}
\newcommand\textrmlf[1]{{\NHLight#1}}
\newcommand\textitlf[1]{{\NHLight\itshape#1}}
\let\textbflf\textrm
\newcommand\textulf[1]{{\NHLight\bfseries#1}}
\newcommand\textuitlf[1]{{\NHLight\bfseries\itshape#1}}
\usepackage{fancyhdr}
\pagestyle{fancy}
\usepackage{titlesec}
\usepackage{titling}
\makeatletter
\lhead{\textbf{\@title}}
\makeatother
\rhead{\textrmlf{Compiled} \today}
\lfoot{\theauthor\ \textbullet \ \textbf{2021-2022}}
\cfoot{}
\rfoot{\textrmlf{Page} \thepage}
\titleformat{\section} {\Large} {\textrmlf{\thesection} {|}} {0.3em} {\textbf}
\titleformat{\subsection} {\large} {\textrmlf{\thesubsection} {|}} {0.2em} {\textbf}
\titleformat{\subsubsection} {\large} {\textrmlf{\thesubsubsection} {|}} {0.1em} {\textbf}
\setlength{\parskip}{0.45em}
\renewcommand\maketitle{}
\author{Exr0n}
\date{\today}
\title{Voltage}
\hypersetup{
 pdfauthor={Exr0n},
 pdftitle={Voltage},
 pdfkeywords={},
 pdfsubject={},
 pdfcreator={Emacs 28.0.50 (Org mode 9.4.4)}, 
 pdflang={English}}
\begin{document}

\maketitle
\#ref \#incomplete

\section{Voltage}
\label{sec:org4336ed9}
\begin{itemize}
\item Units: \(\frac{Nm}{C} = \frac{J/C} = V\)
\item Amount of energy per unit of charge it takes to bring that charge to
that point

\begin{itemize}
\item If you have a ball and you are taking it up the hill, then it takes
energy to do that
\item When you let go, it will roll back down the hill
\item Voltage = energy per charge is similar to energy per kilogram of
raising the ball.
\item Field is the amount that it is resisting--the amount of force
required to move the charge.
\item \textbf{Analogous to gravitational potential energy}.
\end{itemize}
\end{itemize}

\subsection{Zero Point}
\label{sec:orgc769b12}
\begin{itemize}
\item If you have one positive and one negative, then the zero point of the
voltage is between the two charges
\item The zero can be defined anywhere, just like zero gravitational energy
can be anywhere

\begin{itemize}
\item However, conventionally, we define zero to be between two opposite
charges
\item We also define the voltage infinity distance away to be zero
\end{itemize}
\end{itemize}

\subsection{Equipotential}
\label{sec:orga28831e}
\begin{itemize}
\item A line that shows where voltage is the same
\href{srcPhETChargesFieldsEquipotentialLines.png.org}{srcPhETChargesFieldsEquipotentialLines.png}
\item Joules of electric potential energy
\item Scalar, while electric field is a vector \#\#\# Relationship with
Electric Field
\item Perpendicular to the electric field
\end{itemize}

\noindent\rule{\textwidth}{0.5pt}
\end{document}
