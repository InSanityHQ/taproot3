% Created 2021-09-12 Sun 22:49
% Intended LaTeX compiler: xelatex
\documentclass[letterpaper]{article}
\usepackage{graphicx}
\usepackage{grffile}
\usepackage{longtable}
\usepackage{wrapfig}
\usepackage{rotating}
\usepackage[normalem]{ulem}
\usepackage{amsmath}
\usepackage{textcomp}
\usepackage{amssymb}
\usepackage{capt-of}
\usepackage{hyperref}
\usepackage[margin=1in]{geometry}
\usepackage{fontspec}
\usepackage{indentfirst}
\setmainfont[ItalicFont = LiberationSans-Italic, BoldFont = LiberationSans-Bold, BoldItalicFont = LiberationSans-BoldItalic]{LiberationSans}
\newfontfamily\NHLight[ItalicFont = LiberationSansNarrow-Italic, BoldFont       = LiberationSansNarrow-Bold, BoldItalicFont = LiberationSansNarrow-BoldItalic]{LiberationSansNarrow}
\newcommand\textrmlf[1]{{\NHLight#1}}
\newcommand\textitlf[1]{{\NHLight\itshape#1}}
\let\textbflf\textrm
\newcommand\textulf[1]{{\NHLight\bfseries#1}}
\newcommand\textuitlf[1]{{\NHLight\bfseries\itshape#1}}
\usepackage{fancyhdr}
\pagestyle{fancy}
\usepackage{titlesec}
\usepackage{titling}
\makeatletter
\lhead{\textbf{\@title}}
\makeatother
\rhead{\textrmlf{Compiled} \today}
\lfoot{\theauthor\ \textbullet \ \textbf{2021-2022}}
\cfoot{}
\rfoot{\textrmlf{Page} \thepage}
\titleformat{\section} {\Large} {\textrmlf{\thesection} {|}} {0.3em} {\textbf}
\titleformat{\subsection} {\large} {\textrmlf{\thesubsection} {|}} {0.2em} {\textbf}
\titleformat{\subsubsection} {\large} {\textrmlf{\thesubsubsection} {|}} {0.1em} {\textbf}
\setlength{\parskip}{0.45em}
\renewcommand\maketitle{}
\author{Houjun Liu}
\date{\today}
\title{Electric fields}
\hypersetup{
 pdfauthor={Houjun Liu},
 pdftitle={Electric fields},
 pdfkeywords={},
 pdfsubject={},
 pdfcreator={Emacs 28.0.50 (Org mode 9.4.4)}, 
 pdflang={English}}
\begin{document}

\maketitle


\section{Electric Fields}
\label{sec:org53f7d75}
\noindent\rule{\textwidth}{0.5pt}

\subsection{\textbf{Calculation CheatSheet!}}
\label{sec:orga656351}
To recall, Coulomb's Law
\href{KBhPHYS201ColoumbsLaw.org}{KBhPHYS201ColoumbsLaw} looks like
\(F_{attraction} = k\frac{Q_1Q_2}{R^2}\), which is eerily similar to the
Force of Gravity, \(F_{grav} = G\frac{M_2M_2}{R^2}\).

And so, by the some token, we could also redefine electric force by
splitting the function in the same way as with gravity fields:

\definition{Electric Field}\{\(E = \frac{k \times Q_2}{R^2}\)\}
\definition[where $E$ is $Q_2$'s electric field]{Electric Force}\{\(F_{attraction} = E Q_1\)\}
Unsurprisingly, the units for \emph{Electric Field} is \(\frac{N}{C}\), and
no, before you get excited, there is nothing it equals.

\noindent\rule{\textwidth}{0.5pt}

\subsection{Directionality of Electric Fields}
\label{sec:orga2549f4}
With masses (and w.r.t.
\href{KBh\_PHYS201\_GravitationalFields.org}{KBh\textsubscript{PHYS201}\textsubscript{GravitationalFields}}
gravitational fields), it's easy. Masses always \emph{attracts} because
negative mass doesn't exist (\emph{yet}). But, with an electric field,
figuring out directions is harder.

So, we have two choices to dealing with directions:

\begin{enumerate}
\item Electric field of any \(Q\) has two values, one "attract field" and
one "repel field"
\item Drawing a single vector \(\vec{E}\), but remember that the direction
of the vector depends on what's dropped in it
\end{enumerate}

USE OPTION \textbf{2}.

In this manner, when we say, "this atom has a electric field vector in
this direction", we mean two things

\begin{enumerate}
\item When a positive test change is dropped onto that vector, it will
experience force in the same direction as the vector
\item When a negative “” “” “","” “” “” the opposite direction as the
vector
\end{enumerate}

\subsection{Illustrating Electric Fields}
\label{sec:org4e37dcf}
There are two ways of illustrating electric fields --- either drawing an
infinite amount of vectors (that's a lot of vectors), or drawing lines
originating from the main particle lining up all the vectors (Think! The
original Japanese flag.)

See
\href{KBhPHYS201IllustratingElectricFields.org}{KBhPHYS201IllustratingElectricFields}
illustrating electric fields.

\subsection{Electric Field Interactions}
\label{sec:org7e9df44}
See
\href{KBhPHYS201ElectricFieldInteractions.org}{KBhPHYS201ElectricFieldInteractions}
Electric Field Interactions.

\subsection{Conductors + Electric Field Interactions}
\label{sec:orgd0d61b4}
To get our modern world, \textbf{Conductors} --- metals and other elements in
which electrons could move freely --- are an important item to study and
model. Lots of problems involve interactions between electrons +
electric fields being placed in and around conductors.

See
\href{KBhPHYS201ConductorsEquilibrium.org}{KBhPHYS201ConductorsEquilibrium}

\subsection{Gravity and Voltage}
\label{sec:org3482168}
In addition to the measurements of field strength with \(\frac{N}{C}\),
there is a unit called Volts (\(V\)), measuring the amount of energy in
an electric field.

See \href{KBhPHYS201Voltage.org}{KBhPHYS201Voltage}

\subsection{An now, something interesting}
\label{sec:org9b39e11}
Take, a neutral conductor.

\href{./Screen Shot 2020-08-24 at 9.44.46 PM.png}{Screen Shot
2020-08-24 at 9.44.46 PM.png}

At the point of the cursor, there would be an electric field cause by
the central charge going outwards; at which point the following will
happen\ldots{}

\begin{enumerate}
\item The red (positive) charge attracts electrons to the inside of the
tube
\item These newly electrons set up their own electric fields equal and
opposite to the electric field by the central electron (because of
the Electric Field Deux. Gravitational Field thing)
\end{enumerate}

So, the conductor has a net electric field of 0. It's static.

Because of the fact that the neutral conductor had both 1) and 2) going
on, there is no tangent changes to the conductor (\textbf{think!} rule 2
aforementioned), and only field lines that are perpendicular (emitted by
the red, positive charge), will be passed out.

\subsection{Pressure of a field: voltage}
\label{sec:orgbbc880b}
\href{KBe20phys201refVoltage.org}{KBe20phys201refVoltage}

Annotated document:
\href{SRCelectrostaticsPacket1annotatedExr0n.pdf.org}{SRCelectrostaticsPacket1annotatedExr0n.pdf}
\end{document}
