% Created 2021-09-27 Mon 12:02
% Intended LaTeX compiler: xelatex
\documentclass[letterpaper]{article}
\usepackage{graphicx}
\usepackage{grffile}
\usepackage{longtable}
\usepackage{wrapfig}
\usepackage{rotating}
\usepackage[normalem]{ulem}
\usepackage{amsmath}
\usepackage{textcomp}
\usepackage{amssymb}
\usepackage{capt-of}
\usepackage{hyperref}
\setlength{\parindent}{0pt}
\usepackage[margin=1in]{geometry}
\usepackage{fontspec}
\usepackage{svg}
\usepackage{cancel}
\usepackage{indentfirst}
\setmainfont[ItalicFont = LiberationSans-Italic, BoldFont = LiberationSans-Bold, BoldItalicFont = LiberationSans-BoldItalic]{LiberationSans}
\newfontfamily\NHLight[ItalicFont = LiberationSansNarrow-Italic, BoldFont       = LiberationSansNarrow-Bold, BoldItalicFont = LiberationSansNarrow-BoldItalic]{LiberationSansNarrow}
\newcommand\textrmlf[1]{{\NHLight#1}}
\newcommand\textitlf[1]{{\NHLight\itshape#1}}
\let\textbflf\textrm
\newcommand\textulf[1]{{\NHLight\bfseries#1}}
\newcommand\textuitlf[1]{{\NHLight\bfseries\itshape#1}}
\usepackage{fancyhdr}
\pagestyle{fancy}
\usepackage{titlesec}
\usepackage{titling}
\makeatletter
\lhead{\textbf{\@title}}
\makeatother
\rhead{\textrmlf{Compiled} \today}
\lfoot{\theauthor\ \textbullet \ \textbf{2021-2022}}
\cfoot{}
\rfoot{\textrmlf{Page} \thepage}
\renewcommand{\tableofcontents}{}
\titleformat{\section} {\Large} {\textrmlf{\thesection} {|}} {0.3em} {\textbf}
\titleformat{\subsection} {\large} {\textrmlf{\thesubsection} {|}} {0.2em} {\textbf}
\titleformat{\subsubsection} {\large} {\textrmlf{\thesubsubsection} {|}} {0.1em} {\textbf}
\setlength{\parskip}{0.45em}
\renewcommand\maketitle{}
\author{Houjun Liu}
\date{\today}
\title{Magnetic Solids}
\hypersetup{
 pdfauthor={Houjun Liu},
 pdftitle={Magnetic Solids},
 pdfkeywords={},
 pdfsubject={},
 pdfcreator={Emacs 28.0.50 (Org mode 9.4.4)}, 
 pdflang={English}}
\begin{document}

\tableofcontents



\section{Magnetic Solids}
\label{sec:org4b1a219}
Diamagnetic Fields

\begin{itemize}
\item Contents induced that travels in the opposite direction of the
material
\item These "magnets" will repel permanent magnets when there is a
meaningful magnetic field is applied
\item Every material intrinsically has a diamagnetic property
\end{itemize}

Paramagnetic Fields

\begin{itemize}
\item Each atoms have "unpaired electrons" --- electrons are oriented at a
random spin
\item When a magnetic field is applied, the electrons align with the
direction of the magnetic field.
\item Upon removal of the magnetic field, the electrons get randomly spun
again
\end{itemize}

Ferromagnetic Fields

\begin{itemize}
\item Unpaired electrons already partially align to each other
\item Each block of aligned electron is called a "domain"
\item After applying a field, heat the material up => you get aligned
domains! You end up with a permanent magnet
\item Heating the magnet above currie temperature again will destroy the
magnet
\end{itemize}

Earth field generated by having a fluid core and spinning reasonably
quickly.

=> Auroras are caused by deflecting of the sun's solar flares through
the Earth's magnetic field
\end{document}
