% Created 2021-09-11 Sat 16:41
% Intended LaTeX compiler: xelatex
\documentclass[letterpaper]{article}
\usepackage{graphicx}
\usepackage{grffile}
\usepackage{longtable}
\usepackage{wrapfig}
\usepackage{rotating}
\usepackage[normalem]{ulem}
\usepackage{amsmath}
\usepackage{textcomp}
\usepackage{amssymb}
\usepackage{capt-of}
\usepackage{hyperref}
\usepackage[margin=1in]{geometry}
\usepackage{fontspec}
\usepackage{indentfirst}
\setmainfont[ItalicFont = LiberationSans-Italic, BoldFont = LiberationSans-Bold, BoldItalicFont = LiberationSans-BoldItalic]{LiberationSans}
\newfontfamily\NHLight[ItalicFont = LiberationSansNarrow-Italic, BoldFont       = LiberationSansNarrow-Bold, BoldItalicFont = LiberationSansNarrow-BoldItalic]{LiberationSansNarrow}
\newcommand\textrmlf[1]{{\NHLight#1}}
\newcommand\textitlf[1]{{\NHLight\itshape#1}}
\let\textbflf\textrm
\newcommand\textulf[1]{{\NHLight\bfseries#1}}
\newcommand\textuitlf[1]{{\NHLight\bfseries\itshape#1}}
\usepackage{fancyhdr}
\pagestyle{fancy}
\usepackage{titlesec}
\usepackage{titling}
\makeatletter
\lhead{\textbf{\@title}}
\makeatother
\rhead{\textrmlf{Compiled} \today}
\lfoot{\theauthor\ \textbullet \ \textbf{2021-2022}}
\cfoot{}
\rfoot{\textrmlf{Page} \thepage}
\titleformat{\section} {\Large} {\textrmlf{\thesection} {|}} {0.3em} {\textbf}
\titleformat{\subsection} {\large} {\textrmlf{\thesubsection} {|}} {0.2em} {\textbf}
\titleformat{\subsubsection} {\large} {\textrmlf{\thesubsubsection} {|}} {0.1em} {\textbf}
\setlength{\parskip}{0.45em}
\renewcommand\maketitle{}
\author{Houjun Liu}
\date{\today}
\title{PHET Simulation w Electric Fields}
\hypersetup{
 pdfauthor={Houjun Liu},
 pdftitle={PHET Simulation w Electric Fields},
 pdfkeywords={},
 pdfsubject={},
 pdfcreator={Emacs 27.2 (Org mode 9.4.4)}, 
 pdflang={English}}
\begin{document}

\maketitle


\section{Question 1}
\label{sec:orgd2013aa}
Place a single charge in the working area. Using the E-field sensor
(with "values" selected), and the measuring tape, confirm that the
E-field calculated by the PhET simulation agrees with the equation we
have used in class. (Note, the units for E-field that we learned in
class were N/C. The PhET simulation may express the units differently.
But the numerical values should be the same.)

I placed a sensor 1.013 metres away from the -1nC charge. The sensor
showed that the the charge had a voltage of 8.78, and my calculations
show that, per \(\frac{9*1}{1.013^2} N/C\), the electric field should be
about 8.8.

\section{Question 2}
\label{sec:org7b494de}
Place two positive charges in the working area. Where do you expect the
E field to be zero? Does the simulation confirm that?

There is a point in between the two electric changes in which the
electric field would be 0. And yes, the PHET simulation does show that.

\section{Question 3}
\label{sec:org0875e10}
Same as above, but use one positive and one negative charge.

None. There should not be given that the two charges are attracting each
other.

\section{Question 4}
\label{sec:org810c95e}
The E field at a given point can be thought of as the force that a +1 C
charge would feel if it were placed there. What does "electric
potential" or "voltage" appear to represent? The units mentioned in \#1
may be of interest as you consider this question.

Volts is a unit for Jouls / coulumb

\section{Question 5}
\label{sec:orgb92fb51}
Does electric potential appear to be a scalar or a vector?

A scalar, it seems. This makes sense given that energy is a scalar.

\section{Question 6}
\label{sec:org1d16d16}
What or where is the zero-point for electric potential?

When a position and negative charge is present, it seems like the point
between the two would have an electric potential of 0 volts.

\section{Question 7}
\label{sec:org5109271}
What is the relationship between the local E-field vector and a line of
constant electric potential? (You can explore this first by moving the
voltage sensor drag the little box, not the crosshairs and observing the
voltage values, then by plotting lines of constant potential).

It seems like the local e-field vector and the line of constant electric
potential are always perpendicular to each other. That is, the vector of
the charge does not have a component tangent to the constant field line
when the test charge is dropped on it.
\end{document}
