% Created 2021-09-12 Sun 22:49
% Intended LaTeX compiler: xelatex
\documentclass[letterpaper]{article}
\usepackage{graphicx}
\usepackage{grffile}
\usepackage{longtable}
\usepackage{wrapfig}
\usepackage{rotating}
\usepackage[normalem]{ulem}
\usepackage{amsmath}
\usepackage{textcomp}
\usepackage{amssymb}
\usepackage{capt-of}
\usepackage{hyperref}
\usepackage[margin=1in]{geometry}
\usepackage{fontspec}
\usepackage{indentfirst}
\setmainfont[ItalicFont = LiberationSans-Italic, BoldFont = LiberationSans-Bold, BoldItalicFont = LiberationSans-BoldItalic]{LiberationSans}
\newfontfamily\NHLight[ItalicFont = LiberationSansNarrow-Italic, BoldFont       = LiberationSansNarrow-Bold, BoldItalicFont = LiberationSansNarrow-BoldItalic]{LiberationSansNarrow}
\newcommand\textrmlf[1]{{\NHLight#1}}
\newcommand\textitlf[1]{{\NHLight\itshape#1}}
\let\textbflf\textrm
\newcommand\textulf[1]{{\NHLight\bfseries#1}}
\newcommand\textuitlf[1]{{\NHLight\bfseries\itshape#1}}
\usepackage{fancyhdr}
\pagestyle{fancy}
\usepackage{titlesec}
\usepackage{titling}
\makeatletter
\lhead{\textbf{\@title}}
\makeatother
\rhead{\textrmlf{Compiled} \today}
\lfoot{\theauthor\ \textbullet \ \textbf{2021-2022}}
\cfoot{}
\rfoot{\textrmlf{Page} \thepage}
\titleformat{\section} {\Large} {\textrmlf{\thesection} {|}} {0.3em} {\textbf}
\titleformat{\subsection} {\large} {\textrmlf{\thesubsection} {|}} {0.2em} {\textbf}
\titleformat{\subsubsection} {\large} {\textrmlf{\thesubsubsection} {|}} {0.1em} {\textbf}
\setlength{\parskip}{0.45em}
\renewcommand\maketitle{}
\author{Houjun Liu}
\date{\today}
\title{Quantum Mechanics}
\hypersetup{
 pdfauthor={Houjun Liu},
 pdftitle={Quantum Mechanics},
 pdfkeywords={},
 pdfsubject={},
 pdfcreator={Emacs 28.0.50 (Org mode 9.4.4)}, 
 pdflang={English}}
\begin{document}

\maketitle


\section{Quantum Mechanics}
\label{sec:org9476224}
\textbf{What is quantum mechanics?} Quantum => in small/discrete steps

The Quantum of US Currency => \$0.01

\subsection{Puzzle of the Blackbody Radiation}
\label{sec:org9dc3f67}
("black" => opaque): from solid materials, liquids

The radiation from hot, solid materials looks samey (bright yellow)
unlike every gas, however, had a spectral emission (think - neon
lights.)

But!

\href{Pasted image 20210303111558.png.org}{Pasted image
20210303111558.png}

The light spectrum did depend on temperature, so what happened? Why is
everything hot?

\textbf{Max Plank} => trying to model incoming light source from rays as
basically all absorbed and not bounced back.

\href{Pasted image 20210303111810.png.org}{Pasted image
20210303111810.png}

Max Plank's Model 1 in this manner matched well with observations at
long wavelengths (red hot). But, it predicted infinite brightness (it
will just "keep bouncing") as wavelength => 0, which is wrong. This is
the "ultraviolet catastrophie."

So, he made it better.

Max Plank's Model 2 is just Model 1, but an additional assumption that
when Energy Transfers from e\^{}- to EMWave, \(\delta E\) must be some
constant * frequency of light.

So, to synthesize high frequencies, this cop out had the effect of
supressing the infinite growth as \(\delta E\) would grow bigger and
bigger to the point where all your energy would not go into the EMWave
but to this transferring factor.

Which is like\ldots{} Kind of a cop out. But it did fit medium frequencies
better.

\noindent\rule{\textwidth}{0.5pt}

Einstein \texttt{> Light !} "wave"; instead, light are photon particles moving
through space.

\textbf{Impontant Knowledges::}

Energy of each photon is equal to the plank constant (h) times the
frequency (f). \(E = h*f\).

\href{Pasted image 20210308100848.png.org}{Pasted image
20210308100848.png}

\href{Pasted image 20210308101013.png.org}{Pasted image
20210308101013.png}

\(\lambda * f = c\)

\(E_{photon} = h \times f\)

Instead of Hertz, however, the frequency of F could better be
represented with \(\omega\), a unit of \(\frac{radians}{sec}\) that is
derived as \(2 \pi f (\frac{radians}{s})\)

So to calculate energy with \(\omega\), simply use
\(\bar{h} = \frac{h}{2\pi}\) and so \(E = \bar{h}\omega\)

\href{Pasted image 20210308111422.png.org}{Pasted image
20210308111422.png}

\subsection{Heisenberg Uncertainty}
\label{sec:org7d3360c}
\(\Delta E \times \Delta t = \bar{h}\) => "uncertainty of energy times
uncertanity in time is the reduced plank's contstant"

\href{Pasted image 20210308111709.png.org}{Pasted image
20210308111709.png}

Lifetime of the upper level => \(\Delta t\)

(Mean) lifetime of the "upper" energy level => \(\Delta t\). So,
\(\Delta E = \frac{\bar{h}}{\Delta t}\).

If \(\Delta t\) is small, \(\Delta E\) is large.

As long as the units of two deltas end up as \(J \times s\), they would
be related by the same way with \(\bar{h}\)

This \(\Delta P\) has an actual effect on our vision

\textbf{THIS IS IMPORTANT, TOO!}
\(\Delta \vec{p} \times \Delta \vec{x} \approx \bar{h}\).

\href{Pasted image 20210308112755.png.org}{Pasted image
20210308112755.png}

\href{Pasted image 20210308112058.png.org}{Pasted image
20210308112058.png}

\href{Pasted image 20210308112111.png.org}{Pasted image
20210308112111.png}

Meaning, in the subatomic world, everything exists based on differening
upper-energy-state-time based uncertainties.

"Diffraction through an apreture"

We could see a similar pattern in passing photons through a llit.
\(Slit large, \Delta P_x small\) \(Slit small, \Delta P_x large\).

\href{Pasted image 20210308112427.png.org}{Pasted image
20210308112427.png}

This limits the width of the lens of a camera because of the uncertanity
in momentum.

\href{Pasted image 20210308113318.png.org}{Pasted image
20210308113318.png}

\href{Pasted image 20210308113628.png.org}{Pasted image
20210308113628.png}

Taking the angle, and dividing it by 3000, which is \(\frac{1}{60}\)
degrees.

Even though Plank's constant is a tiny number, it effects how sharply
you eyes could see b/c of this uncertainty.

There are three "flavor"s of Leptons, each with two variations ---
creating six different leptons.

Lepton => "small", but they are not actually that small as what their
original namer had suggested.

\subsection{Famous Leptions}
\label{sec:orgbbea08a}
\begin{itemize}
\item The Electron

\begin{itemize}
\item Dirac's equations predicted the existance of a certain "positiron"
which would be the oppostite of an electron. After
self-determination (the "equation was too perfect to be wrong"), he
set out hard to try to prove it.
\end{itemize}
\end{itemize}

\textbf{Interactions in the small scale world occur through the creation and
annihilation of particles.}

Neutrinos interact only by weak interactions, which is (bar gravity,
which is the weakest physical interaction) a very weak physical
interaction.

\subsection{A table of particles}
\label{sec:orgfe5b5e8}
\begin{center}
\begin{tabular}{ll}
Particle & Wat\\
\hline
Leptons & Fundimental one-half spin particles, experience strong interaction, have no quarks\\
Baryons & Componsite particles made of quarks + has 1/2, or 3/2, or 5/2 spin\\
Mesons & Composite particles made of quarks + has 0, 1, or any interger spin\\
Quarks & Fundimental strongly interacting particles that never appear singly\\
Hadrons & Bayrons and Mesons that strongnly interact\\
Nucleaons & Neutrons and protons that reside in the nucleai\\
Fermions & Leptons, quarks, and nucleans: all have 1/2 odd interger spin\\
Bosons & Force carriers, like mesons, have intergin spin\\
\end{tabular}
\end{center}

Positive pion decays into a positive muon, an \emph{antimuon}, and a
neutrino.

The negative pion decays into a negative muon and an antineutrino.

A pair of electrons could collide and form a pair of tou particles.

Three flavours of leptons cannot interchange or become each other, but
they could interact.

\section{Photoelectric Effects}
\label{sec:orgf2653d0}
If you take a piece of conductor, for instance, a metal, and shine a EM
radiation on it (a light), you will know that there is a possiblilty for
electrons to escape the surface

\href{Pasted image 20210310112755.png.org}{Pasted image
20210310112755.png}

Most effective way of doing this: large Force, and shine for a long
time.

\textbf{Wrong but intuitive}:

Large force => large electric field => bright light. Long wavelength
light => long time => red light.

\textbf{But!} Long wavelength light, no matter how bright, ejected nothing.
Short wavelength light, no mater how dim, ejected electron.

Kinetic energy of the ejected electrons was related to the frequency of
the electron used. Higher frequencies gave electrons more frequency.

The brightness of light only had to do with electrons/sec. If you make
the light brighter, you just get more of electrons, but they have the
same energy.

this is because\ldots{}.

\textbf{Light is a particle! A photon.}

Each photon has an energy porportional to its frequency; that is,
\(E=hf\), where f is plank's constant and f the frequency.

So each e- in metal interacts with one photon at a time.

\href{Pasted image 20210310113322.png.org}{Pasted image
20210310113322.png}

A certain minimum amonut of energy is needed for electron to escape. The
minimum escape energy is called the "\textbf{work function}" of the metal.

Electron will be ejected as long as your kinetic energy gets there.

Energy of yoru photons goes to two places => satisfying the work
function + Kinetic energy of the ejected electron

Hence: \(h \times f = WF + KE_{e^-}\), where WF is the work function of
the material, and h, planks constant, f is the frequency

To measure the ejection, you need to chuck the whole apparutus in a
vaccume. Because if there are air molecules, it would absorbe the
electrons.

To continually eject electrons (otherwise, you would eject a few, your
metal becomes positive, and no more electrons for you), you also need to
collect the ejected electrons and put them back into the metal.

To figure out the amount of kinetic energy, simply figure out how much
voltage need to be added to stop the protons. If the stopping voltage is
small, it will need to fight the voltage but completes the circut. As
you increase the stopping voltage, you want to figure out when the
electrons don't have enough energy to complete the circut.

Increase Vstop untill current drops to 0.

\(V_{stop} \times Q_e = KE\). The kinetic energy of ejected electrons is
charge of an electron times the stopping voltage multiplied by.

Finally, plugging stuff into the previous hf equation:
\(h \times f = WF + Q_e \times V_{stop}\)

Quarks combine their fractional charges into particles of full charge.
The number of leptos (electron, muon, yadda) and bayons (neutron,
proton, pion, yadda) should be conserved though a decay

3 Quarks form a baryon => all colourless (RGB => W)

Hedron => Particles that are affected by Strong force; so all
assembleges of quarks.

\subsection{Heisenberg Uncertantiy}
\label{sec:orgebbc98f}
Classical physics: objectc has a position X and velocity V that are
completely defined

Quantum mechanics: wave function \(\psi\)
\(\psi(\vec(x),t) = a+bi (\vec(x),t)\).

\href{Pasted image 20210317093631.png.org}{Pasted image
20210317093631.png}

The wave function yields the probablitiy density of finding the object
at that location and at that time.

The momentum (\(\rho\)) of an object is associated with its wavelength
(\(\lambda\)).

If \(\rho\) is completely defined, its position \(x\) is completely
undefined

If \(x\) is completely defined, its position \(\rho\) is completely
undefined

\href{Pasted image 20210317093907.png.org}{Pasted image
20210317093907.png}

Heisnberg Uncertainty, formally:

\(\Delta x \Delta \rho \approx h\). This is handwavy.

\(\rho_0 = \frca{h}{\lambda}\) to verify heisenberg

\href{Pasted image 20210317094936.png.org}{Pasted image
20210317094936.png}

Through this, we could find out that\ldots{}

\begin{itemize}
\item \(\rho_0 = \frac{h}{\lambda} = \frac{\rho){}\)
\end{itemize}

\subsection{Bohr Model of the Atom}
\label{sec:orgc8040ce}
Planck's Model:

\href{Pasted image 20210317103124.png.org}{Pasted image
20210317103124.png}

But this is only for solid. that it glows "hotter and hotter"

Gasses, however, looks different:

\href{Pasted image 20210317103159.png.org}{Pasted image
20210317103159.png}

Whereby at discrete wavelengths/energies, it glows; yet in the other
states it does not.

This infers that there is only certain levels in which electron may
exist, and hence these jumps occur when energy levels shift (?)

Planck's constant for \(\bar{h}\) had units of \(J/s\). This is also the
units for angular momntum.

\subsection{Classical angular momentum}
\label{sec:org9a4b47e}
\href{Pasted image 20210317103521.png.org}{Pasted image
20210317103521.png}

Bohr's Model: completely classical with the addition that the angular
momentum must equal an interger \(N\) times Planck's constant
\(\bar{h}\). Meaning, \(angmom = N\bar{h}\).

So, the energy levels (angular momentums) in which electrons should
exist should be an interger of planck's reduced constant.

Goal: Electron Energy in terms of an N and a contsant (hbar)

\href{Pasted image 20210317103815.png.org}{Pasted image
20210317103815.png}

Circular motion is a bound system (\(KE=\frac{1}{2}mv^2\)). Hence, as R
increases, PE decreases.

\(PE=0 when R \to \infty\) \(PE_{electrical} = \frac{-k Q^2}{R}\);

\(|PE|=2KE\). \(E_{tot} = PE+KE = -KE\)

\href{Pasted image 20210317104437.png.org}{Pasted image
20210317104437.png}

Bohr Model: gets right --- energy level of Hydrogen

\sout{For every N, we have a unique energy + a unique angular momentum.}

Turns out, for every N, there are a variety of angular momentum that's
possible

Bohr's circular orbit and definite position theories are definitely
wrong.

There is, however, a maximum angular momentum for each N in units of
H-bar.

\subsection{Schroedinger's Equation}
\label{sec:org2252e1a}
\href{Pasted image 20210322093653.png.org}{Pasted image
20210322093653.png}

This equation is only useful if the particle is moving along on a
potential.

As time increases, the potential does not change in time

\href{Pasted image 20210322094031.png.org}{Pasted image
20210322094031.png}

The function \(\phi\) solution to schodinger's equation in probably
imaginary, but, multiplied by its complex conjugate, we could find the
probability density that the particle will be found at (x).

\href{Pasted image 20210322094217.png.org}{Pasted image
20210322094217.png}

And given that its a probability density, we know a few things

\(\int_{-\infty}^{infty} \phi * \phi dx = 1\) because "if you look
everywhere, you otta find it if it exist."

\(\phi(x)\) must be continuous.

The phi function must follow the property of being "bounded", meaning
that \(\phi {x<0 \rightarrow 0, x>L => \rightarrow}\)

Solving for the differential equation that we proposed:

\href{Pasted image 20210322095234.png.org}{Pasted image
20210322095234.png}

But wait! B squared here resulted in an negative number. That's either
imaginary (which actually works out because \(e^i = sin\), think talor),
or we just use a different function:

\href{Pasted image 20210322095510.png.org}{Pasted image
20210322095510.png}

But! Remember that this function is bounded between 0 and L. Hence, we
add multiple wave functions together like standing waves.

Continuity at X=L demand \(B*L = N*\pi\), whereby \(N=interger\). B !=0
because otherwise the particle would be just nonexistant anywhere.

For each N, a new standing wave is added. So the collapsed sum of the
wave functions would be the probablitiy densities.

\href{Pasted image 20210322100145.png.org}{Pasted image
20210322100145.png}

On classical physics, the probability of oscellations should be constant
(the particle could be at any point with equal probability.)

As N (location/"humps") increases, more and more smaller waves exist. So
as \(N\rightarrow \infty\), there would be so many oscillations that it
approaches the classical view

\href{Pasted image 20210322100406.png.org}{Pasted image
20210322100406.png}

\href{Pasted image 20210322100429.png.org}{Pasted image
20210322100429.png}

\href{Pasted image 20210322100450.png.org}{Pasted image
20210322100450.png}

\href{Pasted image 20210322100650.png.org}{Pasted image
20210322100650.png}

\href{Pasted image 20210322100713.png.org}{Pasted image
20210322100713.png}

As we know more certain its position (as L=>0, position is well known
(between 0 and L smaller)), the energy goes to infinity, which makes
momentum quite uncertain.

Visea versa.

\href{Pasted image 20210322100851.png.org}{Pasted image
20210322100851.png}

\href{Pasted image 20210322101650.png.org}{Pasted image
20210322101650.png}

Outside of the box, the expotentials on the two sides would contain a
stable velocity equaling to Q outside of the possible "existance"
region.

\href{Pasted image 20210322102027.png.org}{Pasted image
20210322102027.png}

Wave function is not 0 outside the box! According to quantum physics,
there is some probability of finding the object outside its physical
boundary box.

An analogy: \href{Pasted image 20210322102112.png.org}{Pasted image
20210322102112.png}

And this is "quantum tunneling"
\end{document}
