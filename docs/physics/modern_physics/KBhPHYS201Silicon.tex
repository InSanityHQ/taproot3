% Created 2021-09-11 Sat 16:41
% Intended LaTeX compiler: xelatex
\documentclass[letterpaper]{article}
\usepackage{graphicx}
\usepackage{grffile}
\usepackage{longtable}
\usepackage{wrapfig}
\usepackage{rotating}
\usepackage[normalem]{ulem}
\usepackage{amsmath}
\usepackage{textcomp}
\usepackage{amssymb}
\usepackage{capt-of}
\usepackage{hyperref}
\usepackage[margin=1in]{geometry}
\usepackage{fontspec}
\usepackage{indentfirst}
\setmainfont[ItalicFont = LiberationSans-Italic, BoldFont = LiberationSans-Bold, BoldItalicFont = LiberationSans-BoldItalic]{LiberationSans}
\newfontfamily\NHLight[ItalicFont = LiberationSansNarrow-Italic, BoldFont       = LiberationSansNarrow-Bold, BoldItalicFont = LiberationSansNarrow-BoldItalic]{LiberationSansNarrow}
\newcommand\textrmlf[1]{{\NHLight#1}}
\newcommand\textitlf[1]{{\NHLight\itshape#1}}
\let\textbflf\textrm
\newcommand\textulf[1]{{\NHLight\bfseries#1}}
\newcommand\textuitlf[1]{{\NHLight\bfseries\itshape#1}}
\usepackage{fancyhdr}
\pagestyle{fancy}
\usepackage{titlesec}
\usepackage{titling}
\makeatletter
\lhead{\textbf{\@title}}
\makeatother
\rhead{\textrmlf{Compiled} \today}
\lfoot{\theauthor\ \textbullet \ \textbf{2021-2022}}
\cfoot{}
\rfoot{\textrmlf{Page} \thepage}
\titleformat{\section} {\Large} {\textrmlf{\thesection} {|}} {0.3em} {\textbf}
\titleformat{\subsection} {\large} {\textrmlf{\thesubsection} {|}} {0.2em} {\textbf}
\titleformat{\subsubsection} {\large} {\textrmlf{\thesubsubsection} {|}} {0.1em} {\textbf}
\setlength{\parskip}{0.45em}
\renewcommand\maketitle{}
\author{Houjun Liu}
\date{\today}
\title{Silicon}
\hypersetup{
 pdfauthor={Houjun Liu},
 pdftitle={Silicon},
 pdfkeywords={},
 pdfsubject={},
 pdfcreator={Emacs 27.2 (Org mode 9.4.4)}, 
 pdflang={English}}
\begin{document}

\maketitle


\section{Silicon}
\label{sec:orgc68d7ed}
\begin{itemize}
\item Integrated circuits changed computer circutries

\item Circuts's sillicon purified as polycillion chunks

\begin{itemize}
\item The cubic seed will form a new cubic sillicon
\item Impurities added to sillicon to cause it to conduct
\item Negative charged free carrier (asinic) => n type
\item Positive charged carrier (boron) => p type
\end{itemize}

\item Christle ground to form ingots

\item Then, sliced thin as wafers

\item Wafers are then ground thin + removed of surface contaminates

\item Then, wafers are checked for resistivity

\item CMOS

\begin{itemize}
\item n-type transitior sandwich a p type region
\item A charge on the gate wolud cause the charge to go through from
source => drain
\item Vise, versa
\end{itemize}

\item Meaning, when the P-N circut combinations are on, the N-P combination
is off\\

\item High temperature used to grow sillicon dioxide to protect the sillicon
as sillicon interacts with pure exygen

\item Photoresist smeared on the wafer, and light is exposed to each part to
etch patters

\item Then, lazers/plasma/acid guides etching of the wafer suface

\item Plasma implimant impurities to cause conductivity

\item Photoresist then washed off

\item The wafer is then cleaned off
\end{itemize}

Then, the actual circut wires are introduced:

\begin{enumerate}
\item Deposition of sillican oxite
\item Photolithagraphy, masking + etching
\item Depositivion of tusten as pulig
\item Deposition + potterning of alluminum alloy as wires
\end{enumerate}

\noindent\rule{\textwidth}{0.5pt}

Lastly, the water is put into pieces to be placed onto circuts.
\end{document}
