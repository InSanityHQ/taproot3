% Created 2021-09-27 Mon 12:02
% Intended LaTeX compiler: xelatex
\documentclass[letterpaper]{article}
\usepackage{graphicx}
\usepackage{grffile}
\usepackage{longtable}
\usepackage{wrapfig}
\usepackage{rotating}
\usepackage[normalem]{ulem}
\usepackage{amsmath}
\usepackage{textcomp}
\usepackage{amssymb}
\usepackage{capt-of}
\usepackage{hyperref}
\setlength{\parindent}{0pt}
\usepackage[margin=1in]{geometry}
\usepackage{fontspec}
\usepackage{svg}
\usepackage{cancel}
\usepackage{indentfirst}
\setmainfont[ItalicFont = LiberationSans-Italic, BoldFont = LiberationSans-Bold, BoldItalicFont = LiberationSans-BoldItalic]{LiberationSans}
\newfontfamily\NHLight[ItalicFont = LiberationSansNarrow-Italic, BoldFont       = LiberationSansNarrow-Bold, BoldItalicFont = LiberationSansNarrow-BoldItalic]{LiberationSansNarrow}
\newcommand\textrmlf[1]{{\NHLight#1}}
\newcommand\textitlf[1]{{\NHLight\itshape#1}}
\let\textbflf\textrm
\newcommand\textulf[1]{{\NHLight\bfseries#1}}
\newcommand\textuitlf[1]{{\NHLight\bfseries\itshape#1}}
\usepackage{fancyhdr}
\pagestyle{fancy}
\usepackage{titlesec}
\usepackage{titling}
\makeatletter
\lhead{\textbf{\@title}}
\makeatother
\rhead{\textrmlf{Compiled} \today}
\lfoot{\theauthor\ \textbullet \ \textbf{2021-2022}}
\cfoot{}
\rfoot{\textrmlf{Page} \thepage}
\renewcommand{\tableofcontents}{}
\titleformat{\section} {\Large} {\textrmlf{\thesection} {|}} {0.3em} {\textbf}
\titleformat{\subsection} {\large} {\textrmlf{\thesubsection} {|}} {0.2em} {\textbf}
\titleformat{\subsubsection} {\large} {\textrmlf{\thesubsubsection} {|}} {0.1em} {\textbf}
\setlength{\parskip}{0.45em}
\renewcommand\maketitle{}
\author{Houjun Liu}
\date{\today}
\title{Special Relativity}
\hypersetup{
 pdfauthor={Houjun Liu},
 pdftitle={Special Relativity},
 pdfkeywords={},
 pdfsubject={},
 pdfcreator={Emacs 28.0.50 (Org mode 9.4.4)}, 
 pdflang={English}}
\begin{document}

\tableofcontents



\section{Special Relativity}
\label{sec:org7e84bbc}
\begin{itemize}
\item If events are apart for observers to see, two observers have to record
the time and compare notes
\item It will take time (sound/light) from an event to get from the event to
the observer; the travel time needs to be taken into account, and if
the objects are moving, calculating time travel would be extremely
tricky
\item Human senses also have certanty
\end{itemize}

\subsection{Space and Time}
\label{sec:org58fdf0c}
\begin{itemize}
\item Because it is impossible to observe a result independent of a
refrenece point, each observe would idealy want a clock of identical
construction
\item Laws of inertia (newton's first law) is not true for every coordinate
system, i.e\ldots{}.

\begin{itemize}
\item The rotationon of the Earth, if we set the coornate 0,0 at the
center of Earth, causes everything around earth to seem to be
rapidly rotating around earth
\item Hence, it would seem like a force would be wildly acting upon all
objects around Earth for their velocity to accelerate rapidly
\end{itemize}

\item If an object is translating through one coordinate system, it would be
uniformly translating to another coordinate system of the same type
that itself is uniformly translated uniformly
\end{itemize}

\subsection{The Principle of Relativity}
\label{sec:org9ad6fca}
\textbf{Note: this is not the theory of relativity}

\emph{Laws of Physics should be the same in all inertial refrence frame.}

\begin{itemize}
\item Classical mechanics => does apply
\item Electromagnatism => not sure
\end{itemize}

There is no true rest frame of the universe; because if so, the theories
of mechanics would be "wrong" in all other systems other than the rest
frame and will need adjustment. Furthermore, if there is a rest frame,
the "physics" on the moving frame would need to account for the velocity
of the moving frame.

However, measuring the motions of dfferent celestial bodies, it seems
like the motion w.r.t. any frame is governed by the same law. i.e. if
you are on earth, your velocity flips (you go around the sun) every half
a year. But, w.r.t. earth's frame at any time of the year, physics is no
different. So\ldots{}

\href{Pasted image 20210208105746.png.org}{Pasted image
20210208105746.png}

\subsubsection{But Light}
\label{sec:orgefa4d73}
Maxwell's equations of electromanitism. Speed of a \sout{wave} light:
\(\frac{1}{\sqrt{\epsilon_0 \mu_0}} = ~300,000km/s\)

But\ldots{} relative to what? It seems like this is a ground truth relative
to some mysterious \(K_0\), which, shoulden't exist. Meaning, the speed
of light should be inconsistent based on which set of rules you are
observing. But it isn't.

So how do we reconcile Relativistic principles and the theory of light?

\subsection{The Theory of Relativity}
\label{sec:org6043d74}
\subsubsection{When does the same time happen?}
\label{sec:org07e631e}
Put a guy at the midpoint of two things, put up some mirrors, and
observe if two things happen at once.

However, how do we know that light travels at the same speed from two
directions? Well, because Relativity Depends on this, we choose that
this is true.

Furthermore, if two events are simutaneous to one axis M, they cannot be
simutaneous on another axis M' that is moving w.r.t. M. Becauese.. THINK
--- if you are on a train, and moving towards a point, the light coming
from both ends equal for M would, in the case of M', come quicker from
the B side because you are moving towards it.

\noindent\rule{\textwidth}{0.5pt}

\section{Eistein's train}
\label{sec:org499587f}
Obserber M stands in the non-moving moment, and looks at two lights
shining at A and B. If they are simultaneous, the obserber M would see
both events D/c seconds after they happened, where D => distance between
A\&B, and c=>speed of light.

Obserber M', on the train, would instead see the left light at c-V, the
rgiht c+V, for M' is moving at a speed of V towards B

\href{Screen Shot 2021-02-08 at 5.12.23 PM.png.org}{Screen Shot
2021-02-08 at 5.12.23 PM.png}

For the A observer, he would think that the light arrived at d/(c-V),
and the B observer, d/(c+V). The smaller denominator one (A) would have
a greater time --- and that makes sense: you are moving away from it.

However, we can't just add c-V ond c+V to the speed of light when you
are moving! That's stupid. The speed of light should stay constant for
every observer --- whether M or M'.

The only way of reconcilling this problem, is that, through moving
though space-time, M' actually \emph{time traveled} --- that' from M''s
vantage point, A and B did not happen at the same time. In fact, B is
more recent than A for M'. In this way, M' could still move, see light
at the same speed, but see them at different instants --- \textbf{simutinaety
is relative to the vantage point}.

\section{Time Dialation}
\label{sec:orga1c394f}
The process by which, when

Let's say you (obs. M') are standing inside a mirror, and light traves
from you + bounces back.

\href{Pasted image 20210210094000.png.org}{Pasted image
20210210094000.png}

However, what if even A actually takes place on a moving surface
traveling to the right at a speed V w.r.t. another "stationary w.r.t.
ground" obserber M?

\href{Pasted image 20210210094151.png.org}{Pasted image
20210210094151.png}

First of all, note that distance L did not change --- *distances could
only be affected by the direction of lght travel, not perpandicular to
it.*

\href{Pasted image 20210210094431.png.org}{Pasted image
20210210094431.png}

To figure out the distance between event A and the mirror, it is simply
\(\frac{\delta t}{2} \times V\) (speed of prime travel times half the
distance between A and B).

And lastly, to figure out how much light travels, it is simply the
hipotenuse between L and \(\frac{\delta t}{2} \times V\).

\href{Pasted image 20210210094759.png.org}{Pasted image
20210210094759.png}

Notice! With this roundabout way, we have an equation with \(\delta t\)
on both sides: becaues we could figure out the distance of light travel
to mirror, doubling that divided by speed of light will result in
\(\delta t\) again!

Doing some algebra to seperate the \(\delta t\), we shall get that:

\href{Pasted image 20210210095050.png.org}{Pasted image
20210210095050.png}

But notice! We could divide the top and bottom by \(c\)

\href{Pasted image 20210210095115.png.org}{Pasted image
20210210095115.png}

Why is \(2L/c = \delta t'\)? Think! look at the primed (moving) frame.
W.r.t. the moving object, the light is traveling exactly perpendicular
up and down. So, the change in time in that case is just twice the
vertical distance divided by the speed of light.

Physicsts tend to define a \(\gamma\) and write the same equation as
follows:

\href{Pasted image 20210210095218.png.org}{Pasted image
20210210095218.png}

Now\ldots{}. Notice! If \(v\) is small, \(\gamma=1\) and hence
\(\delta t = \delta t'\)

As \(v \to c\), \(\gamma\) increases and hence, to M, M' will be
traveling quickly foward to time

\href{Pasted image 20210210100247.png.org}{Pasted image
20210210100247.png}

So. Knowledges: \(\gamma(v) = \frac{1}{\sqrt{1-(v^2/c^2)**2}}\). which
means, for speeds at \%c, \(\gamma(pc) = \frac{1}{\sqrt{1-pc^2}}\). And
finally, for stationary observer \(x\) and moving observer \(x'\),
\(\delta t = \delta t' \gamma\).

What's even cooler? this principle applies to the length of the moving
frame as well --- but backwards. That, at a moving frame,
\(D' = D \gamma\), where \(D\) is size of the moving frame ("length of
the spaceship") when it is not moving to the non-moving frame, and
\(D'\) the size of the moving frame when it is moving to the non-moving
frame.

Of course, beware of thinking this as the "length of the ship", though.
For instance, if your stationary frame is the spaceship, the "length of
the spaceship" would then be the distance between the planets.

Remember: two objects will need to be parallel to each other and one
traveling in a linear motion while the other stationary for this to
work!

\textbf{Displacement equales velocity multiplied by the time interval} (and
gamma.)

\section{Knowledges That Importance}
\label{sec:org85ecfdf}
\(\gamma(v) = \frac{1}{\sqrt{1-(v^2/c^2)**2}}\). which means, for speeds
at \%c, \(\gamma(pc) = \frac{1}{\sqrt{1-pc^2}}\)

\subsection{Time Dilation}
\label{sec:org4b3201e}
Should Ted be moving at a velocity \(V\) according to the refrence frame
of Sally. Time \(t\) for Ted would be \(\gamma \times t\) for sally.

\subsection{Length Contraction}
\label{sec:org7421af5}
Should Ted be moving at a velocity \(V\) according to the refrence frame
of Sally. Distance \(D\) away from Sally and Ted's shared \(D_0\) for
Ted would be \(\frac{D}{\gamma}\) for sally.

\href{Pasted image 20210224111827.png.org}{Pasted image
20210224111827.png}

=> "Finding the distance to origin, and multiply by gamma"
\end{document}
