% Created 2021-09-27 Mon 12:01
% Intended LaTeX compiler: xelatex
\documentclass[letterpaper]{article}
\usepackage{graphicx}
\usepackage{grffile}
\usepackage{longtable}
\usepackage{wrapfig}
\usepackage{rotating}
\usepackage[normalem]{ulem}
\usepackage{amsmath}
\usepackage{textcomp}
\usepackage{amssymb}
\usepackage{capt-of}
\usepackage{hyperref}
\setlength{\parindent}{0pt}
\usepackage[margin=1in]{geometry}
\usepackage{fontspec}
\usepackage{svg}
\usepackage{cancel}
\usepackage{indentfirst}
\setmainfont[ItalicFont = LiberationSans-Italic, BoldFont = LiberationSans-Bold, BoldItalicFont = LiberationSans-BoldItalic]{LiberationSans}
\newfontfamily\NHLight[ItalicFont = LiberationSansNarrow-Italic, BoldFont       = LiberationSansNarrow-Bold, BoldItalicFont = LiberationSansNarrow-BoldItalic]{LiberationSansNarrow}
\newcommand\textrmlf[1]{{\NHLight#1}}
\newcommand\textitlf[1]{{\NHLight\itshape#1}}
\let\textbflf\textrm
\newcommand\textulf[1]{{\NHLight\bfseries#1}}
\newcommand\textuitlf[1]{{\NHLight\bfseries\itshape#1}}
\usepackage{fancyhdr}
\pagestyle{fancy}
\usepackage{titlesec}
\usepackage{titling}
\makeatletter
\lhead{\textbf{\@title}}
\makeatother
\rhead{\textrmlf{Compiled} \today}
\lfoot{\theauthor\ \textbullet \ \textbf{2021-2022}}
\cfoot{}
\rfoot{\textrmlf{Page} \thepage}
\renewcommand{\tableofcontents}{}
\titleformat{\section} {\Large} {\textrmlf{\thesection} {|}} {0.3em} {\textbf}
\titleformat{\subsection} {\large} {\textrmlf{\thesubsection} {|}} {0.2em} {\textbf}
\titleformat{\subsubsection} {\large} {\textrmlf{\thesubsubsection} {|}} {0.1em} {\textbf}
\setlength{\parskip}{0.45em}
\renewcommand\maketitle{}
\author{Exr0n}
\date{\today}
\title{Bio Day 1}
\hypersetup{
 pdfauthor={Exr0n},
 pdftitle={Bio Day 1},
 pdfkeywords={},
 pdfsubject={},
 pdfcreator={Emacs 28.0.50 (Org mode 9.4.4)}, 
 pdflang={English}}
\begin{document}

\tableofcontents

\begin{itemize}
\item Homework \#hw

\begin{itemize}
\item Look through the
\href{https://nuevaschool.instructure.com/courses/3017/assignments/50846}{8th
grade bio site} Weeks 2 through 5 for roughly 40 mins.
\item Free association writing for 10 roughly minutes, maybe a paragraph
or two? Response: ``` I figured I would write for a few minutes
before reviewing the site to give you an idea of what content I
remembered, so here is that: The second law of thermodynamics states
something along the lines of "entropy increases in a closed system",
which means that things will generally reach equalibrium, find a
locally minimal energy state, or remove order. I like to think of it
as a emergent property of randomness, where atoms/molecules go
around randomly and in the process undo gradients, etc. Atoms have
electrons revolving around them, and those electrons stick with the
nucleus because the core is positively charged. Depending on the
size and contents of the atom, it might hold onto its outer
(valence) electrons more strongly than others. This element is said
to have higher electronegativity. When an atoms bond to form
molecules, electrons may be transfered (ionic) or they may not. When
electrons are transfered, each atom/molecule gains an electron which
changes it's charge. This happens when the electronegativity is very
high (> 3.4? I don't remember the number). Covalent bonds happen
when two atoms try to take the same electron, but the difference in
strength (electroneg) isn't strong enough for one to rip it away. If
there is a significant difference in strength, you end up with a
polar covalent bond (delta electroneg >= 0.4?). That means that one
side of the bond has a partial charge, because the electron spends
more time on that side of the bond. This is called a dipole, and
means that the molecule is polar. Polar molecules are hydrophilic,
which means they tend to interact with water (because water is polar
and can pull apart polar molecules).
\end{itemize}
\end{itemize}

I'm less sure about this stuff: Lipids (like phospholipids which make up
the animal cell walls) are fats, and they are made of carbohydrates?
Carbohydrates are molecules made of carbons and hydrogens. They are not
polar (no polar covalent or ionic bonds) and don't really interact with
water (hydrophobic). Protiens are strings of amino acids, which are
folded by other protiens that float around in the cytoplasm of a cell
based on mRNA?. Protiens do everything in a cell, and their shape is
determined by the polarity of the amino acids that make them up. That's
why protien folding as a computational problem is so important--because
if we can determine the shape of a given amino acid string then we can
design medicine faster (I love this kind of stuff). Enzymes are
catalysts that speed up reactions. I don't remeber how they relate to
other things.

Okay, that was 12 minutes. I'll now review the site and note updates
that I would make to the previous brain dump. 1. Nice, she my scale
models in! (Week 3) 2. Enzymes are protiens that help reduce the
activation energy for a reaction, such as by locking the orientation of
the reactants. Enzymes often need to be activated/deactivated and that
happens when other molecules (protiens?) change or block the activation
site. ```

\noindent\rule{\textwidth}{0.5pt}
\end{document}
