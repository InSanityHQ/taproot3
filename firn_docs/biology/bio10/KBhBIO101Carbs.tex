% Created 2021-09-27 Mon 12:01
% Intended LaTeX compiler: xelatex
\documentclass[letterpaper]{article}
\usepackage{graphicx}
\usepackage{grffile}
\usepackage{longtable}
\usepackage{wrapfig}
\usepackage{rotating}
\usepackage[normalem]{ulem}
\usepackage{amsmath}
\usepackage{textcomp}
\usepackage{amssymb}
\usepackage{capt-of}
\usepackage{hyperref}
\setlength{\parindent}{0pt}
\usepackage[margin=1in]{geometry}
\usepackage{fontspec}
\usepackage{svg}
\usepackage{cancel}
\usepackage{indentfirst}
\setmainfont[ItalicFont = LiberationSans-Italic, BoldFont = LiberationSans-Bold, BoldItalicFont = LiberationSans-BoldItalic]{LiberationSans}
\newfontfamily\NHLight[ItalicFont = LiberationSansNarrow-Italic, BoldFont       = LiberationSansNarrow-Bold, BoldItalicFont = LiberationSansNarrow-BoldItalic]{LiberationSansNarrow}
\newcommand\textrmlf[1]{{\NHLight#1}}
\newcommand\textitlf[1]{{\NHLight\itshape#1}}
\let\textbflf\textrm
\newcommand\textulf[1]{{\NHLight\bfseries#1}}
\newcommand\textuitlf[1]{{\NHLight\bfseries\itshape#1}}
\usepackage{fancyhdr}
\pagestyle{fancy}
\usepackage{titlesec}
\usepackage{titling}
\makeatletter
\lhead{\textbf{\@title}}
\makeatother
\rhead{\textrmlf{Compiled} \today}
\lfoot{\theauthor\ \textbullet \ \textbf{2021-2022}}
\cfoot{}
\rfoot{\textrmlf{Page} \thepage}
\renewcommand{\tableofcontents}{}
\titleformat{\section} {\Large} {\textrmlf{\thesection} {|}} {0.3em} {\textbf}
\titleformat{\subsection} {\large} {\textrmlf{\thesubsection} {|}} {0.2em} {\textbf}
\titleformat{\subsubsection} {\large} {\textrmlf{\thesubsubsection} {|}} {0.1em} {\textbf}
\setlength{\parskip}{0.45em}
\renewcommand\maketitle{}
\author{Houjun Liu}
\date{\today}
\title{Carbohydrates}
\hypersetup{
 pdfauthor={Houjun Liu},
 pdftitle={Carbohydrates},
 pdfkeywords={},
 pdfsubject={},
 pdfcreator={Emacs 28.0.50 (Org mode 9.4.4)}, 
 pdflang={English}}
\begin{document}

\tableofcontents



\section{Carbohydrates}
\label{sec:org6242b79}
Glucose, Cellulose, Lactose, etc. etc.

\subsection{Structures of Carbs}
\label{sec:org10ae8b8}
\begin{itemize}
\item Carbs have \textbf{6 carbons.}
\item Carbon chain with Hydrogen and Hydroxide
\item Dissolves well in water because of the slightly positive hydrogen and
the slightly negative OH
\end{itemize}

\begin{quote}
So, if you see a hexagon carbon ring, you are probably looking at a
carbohydrate
\end{quote}

You could take many "simple sugars" (glucose, for instance) and chain
them to build wonderfully complicated things (fiber!). To see how that
happens:
\href{KBhBIO101StructuresofCarbs.org}{KBhBIO101StructuresofCarbs}

\subsection{Uses of Carbs}
\label{sec:org08fc63d}
\begin{itemize}
\item Mitocondria

\begin{itemize}
\item Actually not strictly part of a cell!
\item Another organism (technically an organelle)

\begin{itemize}
\item Could move
\item Could replicate
\end{itemize}

\item Breaks down stored carbs (glycogen) into glucose and then eventually
smaller elements
\end{itemize}

\item Cell tagging

\begin{itemize}
\item As an authentication systems
\item 
\end{itemize}
\end{itemize}

\subsection{The Carbs Debate}
\label{sec:orgf24b01b}
\begin{itemize}
\item Fructose worse? Better? No difference?
\item Experiments differ

\begin{itemize}
\item Generally found no differences
\item Some found fructose to be a bit more obesity causing
\end{itemize}
\end{itemize}
\end{document}
