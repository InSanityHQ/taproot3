% Created 2021-09-27 Mon 12:01
% Intended LaTeX compiler: xelatex
\documentclass[letterpaper]{article}
\usepackage{graphicx}
\usepackage{grffile}
\usepackage{longtable}
\usepackage{wrapfig}
\usepackage{rotating}
\usepackage[normalem]{ulem}
\usepackage{amsmath}
\usepackage{textcomp}
\usepackage{amssymb}
\usepackage{capt-of}
\usepackage{hyperref}
\setlength{\parindent}{0pt}
\usepackage[margin=1in]{geometry}
\usepackage{fontspec}
\usepackage{svg}
\usepackage{cancel}
\usepackage{indentfirst}
\setmainfont[ItalicFont = LiberationSans-Italic, BoldFont = LiberationSans-Bold, BoldItalicFont = LiberationSans-BoldItalic]{LiberationSans}
\newfontfamily\NHLight[ItalicFont = LiberationSansNarrow-Italic, BoldFont       = LiberationSansNarrow-Bold, BoldItalicFont = LiberationSansNarrow-BoldItalic]{LiberationSansNarrow}
\newcommand\textrmlf[1]{{\NHLight#1}}
\newcommand\textitlf[1]{{\NHLight\itshape#1}}
\let\textbflf\textrm
\newcommand\textulf[1]{{\NHLight\bfseries#1}}
\newcommand\textuitlf[1]{{\NHLight\bfseries\itshape#1}}
\usepackage{fancyhdr}
\pagestyle{fancy}
\usepackage{titlesec}
\usepackage{titling}
\makeatletter
\lhead{\textbf{\@title}}
\makeatother
\rhead{\textrmlf{Compiled} \today}
\lfoot{\theauthor\ \textbullet \ \textbf{2021-2022}}
\cfoot{}
\rfoot{\textrmlf{Page} \thepage}
\renewcommand{\tableofcontents}{}
\titleformat{\section} {\Large} {\textrmlf{\thesection} {|}} {0.3em} {\textbf}
\titleformat{\subsection} {\large} {\textrmlf{\thesubsection} {|}} {0.2em} {\textbf}
\titleformat{\subsubsection} {\large} {\textrmlf{\thesubsubsection} {|}} {0.1em} {\textbf}
\setlength{\parskip}{0.45em}
\renewcommand\maketitle{}
\author{Houjun Liu}
\date{\today}
\title{Organelles in Eukaryotic Cells}
\hypersetup{
 pdfauthor={Houjun Liu},
 pdftitle={Organelles in Eukaryotic Cells},
 pdfkeywords={},
 pdfsubject={},
 pdfcreator={Emacs 28.0.50 (Org mode 9.4.4)}, 
 pdflang={English}}
\begin{document}

\tableofcontents



\section{Organelles in Eukaryotic Cells}
\label{sec:org23099d9}
\subsection{An Introduction.}
\label{sec:org3f3571c}
Many organells exist in a cell --- often more in Eukaryotic cells ---
that help execute the cell's functions. They serve a variety of
purposes, and help form the basics of cellular systems. Categorizing
them based on whether or not they have membranes
\href{KBhBIO101OrganellsBasedOnMembranes.org}{KBhBIO101OrganellsBasedOnMembranes}

\subsection{Chloroplast and Mitochondria}
\label{sec:orgdda4ae5}
\begin{itemize}
\item \textbf{Chloroplast} --- found in plants + does photosynthesis
\item \textbf{Mitochondria} --- found in animals + store ATP and extract energy
from ATP
\end{itemize}

\subsection{Rough Endoplasmic Reticulum (ER) and Smooth ER}
\label{sec:org6ae1511}
\begin{itemize}
\item \textbf{Rough ER} --- covered by ribosomes and folds
\href{KBhBIO101Proteins.org}{KBhBIO101Proteins} proteins
\item \textbf{Smooth ER} --- not covered by ribosomes and makes
\href{KBhBIO101Lipids.org}{KBhBIO101Lipids} lipids
\end{itemize}

\subsection{Ribosomes and Golgi apparatus}
\label{sec:orgf6eb09c}
\begin{itemize}
\item \textbf{Ribosomes} => synthesizes proteins
\item \textbf{Golgi apparatus} => packs, modifying, and moving proteins
\end{itemize}

\subsection{Cell Wall and Plasma Membrane}
\label{sec:orgc535da6}
\begin{itemize}
\item \textbf{Cell Wall} --- found in plants => surround the cell: hard
\item \textbf{Plasma membrane} --- found in animals => surround the cell: soft
\href{KBhBIO101Lipids.org}{KBhBIO101Lipids} lipids
\end{itemize}

\subsection{Cytosol, Cytoplasm and Cytoskeleton}
\label{sec:org67695e8}
\begin{itemize}
\item \textbf{Cytosol} => liquid found inside cells; the "cytoplasm" floats within
it
\item \textbf{Cytoplasm} => all the stuff within the cell
\end{itemize}

\subsection{Nucleus and Nucleolus}
\label{sec:org2b548d9}
\begin{itemize}
\item \textbf{Nucleus} => centre of the cell, stores DNA
\item \textbf{Nucleolus} => largest part of the nucleous that makes ribosomes
\end{itemize}

\subsection{Lysosomes and Food Vacuoles}
\label{sec:org51ecbbb}
\begin{itemize}
\item \textbf{Lysosomes} => vesticles that contains enzymes that breaks down
biomolecules
\item \textbf{Food Vacoules} => vesticels that stores food and other resources
\end{itemize}

\subsection{Cytoskeleton and Microtubules}
\label{sec:org9d869c7}
\begin{itemize}
\item \textbf{Cytoskeleton} => complex network of proteins + fibres that organize
the rest of the cell
\item \textbf{Microtubulues} => Polymers of tubulin protein that provides the main
structure of eukarotic cells
\end{itemize}

\subsection{Flagella and Cilia}
\label{sec:org8b10d0e}
\begin{itemize}
\item \textbf{Flagella} => a bacteria's tail --- allow them to move and also act as
an sensory organ. longer than a cilla, and moves in sinosoidial
pattern.
\item \textbf{Cilium} => a cell's "hair" --- provides sensory and communications
functions. Motil cilla could move about to "grab" things, and
non-motile cilla can't move. more abundant that the flagella, and
moves in circular pattern if they do move, and moves in circular
pattern if they do move
\end{itemize}
\end{document}
