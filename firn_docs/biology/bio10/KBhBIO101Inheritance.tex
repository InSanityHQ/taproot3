% Created 2021-09-27 Mon 12:01
% Intended LaTeX compiler: xelatex
\documentclass[letterpaper]{article}
\usepackage{graphicx}
\usepackage{grffile}
\usepackage{longtable}
\usepackage{wrapfig}
\usepackage{rotating}
\usepackage[normalem]{ulem}
\usepackage{amsmath}
\usepackage{textcomp}
\usepackage{amssymb}
\usepackage{capt-of}
\usepackage{hyperref}
\setlength{\parindent}{0pt}
\usepackage[margin=1in]{geometry}
\usepackage{fontspec}
\usepackage{svg}
\usepackage{cancel}
\usepackage{indentfirst}
\setmainfont[ItalicFont = LiberationSans-Italic, BoldFont = LiberationSans-Bold, BoldItalicFont = LiberationSans-BoldItalic]{LiberationSans}
\newfontfamily\NHLight[ItalicFont = LiberationSansNarrow-Italic, BoldFont       = LiberationSansNarrow-Bold, BoldItalicFont = LiberationSansNarrow-BoldItalic]{LiberationSansNarrow}
\newcommand\textrmlf[1]{{\NHLight#1}}
\newcommand\textitlf[1]{{\NHLight\itshape#1}}
\let\textbflf\textrm
\newcommand\textulf[1]{{\NHLight\bfseries#1}}
\newcommand\textuitlf[1]{{\NHLight\bfseries\itshape#1}}
\usepackage{fancyhdr}
\pagestyle{fancy}
\usepackage{titlesec}
\usepackage{titling}
\makeatletter
\lhead{\textbf{\@title}}
\makeatother
\rhead{\textrmlf{Compiled} \today}
\lfoot{\theauthor\ \textbullet \ \textbf{2021-2022}}
\cfoot{}
\rfoot{\textrmlf{Page} \thepage}
\renewcommand{\tableofcontents}{}
\titleformat{\section} {\Large} {\textrmlf{\thesection} {|}} {0.3em} {\textbf}
\titleformat{\subsection} {\large} {\textrmlf{\thesubsection} {|}} {0.2em} {\textbf}
\titleformat{\subsubsection} {\large} {\textrmlf{\thesubsubsection} {|}} {0.1em} {\textbf}
\setlength{\parskip}{0.45em}
\renewcommand\maketitle{}
\author{Houjun Liu}
\date{\today}
\title{Alleals traits and Inheritance}
\hypersetup{
 pdfauthor={Houjun Liu},
 pdftitle={Alleals traits and Inheritance},
 pdfkeywords={},
 pdfsubject={},
 pdfcreator={Emacs 28.0.50 (Org mode 9.4.4)}, 
 pdflang={English}}
\begin{document}

\tableofcontents



\section{Genetic Inheritance}
\label{sec:org6f9a773}
\textbf{As seen on "Blood Types!"}

RBCs have various carb styles. The presence/absence of two carb
modifications cause the difference of A\&B blood types.

One gene controlls the outcome: A\&B genes create attachment to two
different carbohydrates, A, B respectviely; O gene encodes a lack of
enzyme function, which means no carb modification. A person, of course,
has two alleals. If a person that has one A alleal and one B alleal,
both A\&B are expressed.

\begin{itemize}
\item A => AO, AA
\item B => BO, BB
\item AB => AB
\item O => OO
\end{itemize}

(psst\ldots{} this :point\textsubscript{up}: is a pundett square, just not in the square
form because that's apparently "too easy" and "does the work for you").

\textbf{O is the "recessive" trait: that anything like A or B will overtake the
O enzyme}

\begin{itemize}
\item AB+O => A, B, 50\% split
\item (AO|BO) + AB => A (50\% => AO, 25\% => BO), AB (25\%), B (25\% => AO, 50\%
=> BO)
\end{itemize}

These probabiltiy are not considered as a process by which these
probabilities are independently assorted into children (1/6 recision
probablity does not mean that the recessive gene will express in one out
of six children.) Instead, it means that EACH child has 1/6 chance of
the abnormality.

For more, see
\href{KBhBIO101GeneticInheritance.org}{KBhBIO101GeneticInheritance}
\end{document}
