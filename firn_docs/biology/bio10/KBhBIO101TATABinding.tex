% Created 2021-09-27 Mon 12:01
% Intended LaTeX compiler: xelatex
\documentclass[letterpaper]{article}
\usepackage{graphicx}
\usepackage{grffile}
\usepackage{longtable}
\usepackage{wrapfig}
\usepackage{rotating}
\usepackage[normalem]{ulem}
\usepackage{amsmath}
\usepackage{textcomp}
\usepackage{amssymb}
\usepackage{capt-of}
\usepackage{hyperref}
\setlength{\parindent}{0pt}
\usepackage[margin=1in]{geometry}
\usepackage{fontspec}
\usepackage{svg}
\usepackage{cancel}
\usepackage{indentfirst}
\setmainfont[ItalicFont = LiberationSans-Italic, BoldFont = LiberationSans-Bold, BoldItalicFont = LiberationSans-BoldItalic]{LiberationSans}
\newfontfamily\NHLight[ItalicFont = LiberationSansNarrow-Italic, BoldFont       = LiberationSansNarrow-Bold, BoldItalicFont = LiberationSansNarrow-BoldItalic]{LiberationSansNarrow}
\newcommand\textrmlf[1]{{\NHLight#1}}
\newcommand\textitlf[1]{{\NHLight\itshape#1}}
\let\textbflf\textrm
\newcommand\textulf[1]{{\NHLight\bfseries#1}}
\newcommand\textuitlf[1]{{\NHLight\bfseries\itshape#1}}
\usepackage{fancyhdr}
\pagestyle{fancy}
\usepackage{titlesec}
\usepackage{titling}
\makeatletter
\lhead{\textbf{\@title}}
\makeatother
\rhead{\textrmlf{Compiled} \today}
\lfoot{\theauthor\ \textbullet \ \textbf{2021-2022}}
\cfoot{}
\rfoot{\textrmlf{Page} \thepage}
\renewcommand{\tableofcontents}{}
\titleformat{\section} {\Large} {\textrmlf{\thesection} {|}} {0.3em} {\textbf}
\titleformat{\subsection} {\large} {\textrmlf{\thesubsection} {|}} {0.2em} {\textbf}
\titleformat{\subsubsection} {\large} {\textrmlf{\thesubsubsection} {|}} {0.1em} {\textbf}
\setlength{\parskip}{0.45em}
\renewcommand\maketitle{}
\author{Houjun Liu}
\date{\today}
\title{TATA Binding Protein}
\hypersetup{
 pdfauthor={Houjun Liu},
 pdftitle={TATA Binding Protein},
 pdfkeywords={},
 pdfsubject={},
 pdfcreator={Emacs 28.0.50 (Org mode 9.4.4)}, 
 pdflang={English}}
\begin{document}

\tableofcontents



\section{TATA Binding}
\label{sec:org9d976f4}
"DNA is long. How our cells know where to start transcribing?"

First, promoters exist to demarcate the start and the directionality of
transcription. In bacterias, this tipically means that there are two
regions that bind to the sigma subunit of the RNA polymerase, but in
eukareotes, this is much more complicated and involves something
called\ldots{}

:confetti: the TATA box! :confetti:

Before the start of the transcription site, the TATA box exists with a
sequence of usually T-A-T-A-(A/T)-A-(A/T) that the \textbf{TATA binding
protein} recognizes and binds to.

\subsection{TATA Binding Protein}
\label{sec:orgd8b9507}
This TATA binding protein will then serve as the landmark for the actual
RNA polymerease to find the promoter. Surprisingly, the TATA binding
protein is not gentle with the DNA: grabbing onto it and bending it
sharply to start transcription.

\subsection{Helpers to TATA}
\label{sec:orgb27cca9}
TATA is part of a larger transcription factor called TFIID. TFIID, upon
binding, calls for other transcription factor to bind, creating a big'ol
protein to decide whether or not transtription will happen.

These factors sometimes promote the start of transcription, others
inhibit the strate of transcription. TATA always is the anchoring point
that calls all of these together to start the process.

\subsection{So, How's the TATA sequence recognized?}
\label{sec:orgdef5874}
The TATA protein has a string of lysing and argenine, which inteacts
with the phosphate group of the DNA \& glues the proton to the DNA. The
protein also uses special amino acids to interact w/ DNA bases.

The simmetric nature of the two halves of the TATA binding proteins
means that it's able to recognize the proper binding sequence and jam
itself into the DNA.

It is thought that a gene duplication mutation from a long time ago
created this protein after two copies of the same gene is combined.

\subsection{SCA17}
\label{sec:orgb171397}
Sometimes, the transcription of the TBP goes wrong and create a repeat
in the CAG/CAA factors in the TBP; with an average of over 41 repeats.

This results in an abundance of glutamite, which means that this will
result in symptoms of psycoatric problems like involuntary movements,
dementia, and ataxia (not being able to control movement.)
\end{document}
