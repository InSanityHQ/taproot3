% Created 2021-09-27 Mon 12:01
% Intended LaTeX compiler: xelatex
\documentclass[letterpaper]{article}
\usepackage{graphicx}
\usepackage{grffile}
\usepackage{longtable}
\usepackage{wrapfig}
\usepackage{rotating}
\usepackage[normalem]{ulem}
\usepackage{amsmath}
\usepackage{textcomp}
\usepackage{amssymb}
\usepackage{capt-of}
\usepackage{hyperref}
\setlength{\parindent}{0pt}
\usepackage[margin=1in]{geometry}
\usepackage{fontspec}
\usepackage{svg}
\usepackage{cancel}
\usepackage{indentfirst}
\setmainfont[ItalicFont = LiberationSans-Italic, BoldFont = LiberationSans-Bold, BoldItalicFont = LiberationSans-BoldItalic]{LiberationSans}
\newfontfamily\NHLight[ItalicFont = LiberationSansNarrow-Italic, BoldFont       = LiberationSansNarrow-Bold, BoldItalicFont = LiberationSansNarrow-BoldItalic]{LiberationSansNarrow}
\newcommand\textrmlf[1]{{\NHLight#1}}
\newcommand\textitlf[1]{{\NHLight\itshape#1}}
\let\textbflf\textrm
\newcommand\textulf[1]{{\NHLight\bfseries#1}}
\newcommand\textuitlf[1]{{\NHLight\bfseries\itshape#1}}
\usepackage{fancyhdr}
\pagestyle{fancy}
\usepackage{titlesec}
\usepackage{titling}
\makeatletter
\lhead{\textbf{\@title}}
\makeatother
\rhead{\textrmlf{Compiled} \today}
\lfoot{\theauthor\ \textbullet \ \textbf{2021-2022}}
\cfoot{}
\rfoot{\textrmlf{Page} \thepage}
\renewcommand{\tableofcontents}{}
\titleformat{\section} {\Large} {\textrmlf{\thesection} {|}} {0.3em} {\textbf}
\titleformat{\subsection} {\large} {\textrmlf{\thesubsection} {|}} {0.2em} {\textbf}
\titleformat{\subsubsection} {\large} {\textrmlf{\thesubsubsection} {|}} {0.1em} {\textbf}
\setlength{\parskip}{0.45em}
\renewcommand\maketitle{}
\date{\today}
\title{}
\hypersetup{
 pdfauthor={},
 pdftitle={},
 pdfkeywords={},
 pdfsubject={},
 pdfcreator={Emacs 28.0.50 (Org mode 9.4.4)}, 
 pdflang={English}}
\begin{document}

\tableofcontents



\subsection{---}
\label{sec:org5dc79c6}
\section{Experiment Design}
\label{sec:orgdb7cd67}
\subsubsection{Mechanistic Understanding}
\label{sec:org810db4c}
\begin{itemize}
\item Once data has been collected, it is important that one tries to get a
mechanistic understanding.

\begin{itemize}
\item This means more why this pattern is occurring. \#\#\# Data vs Theory
\end{itemize}

\item Evidence is very different from theory.

\begin{itemize}
\item Evidence may describe a pattern, but as soon as any theory or
interpretation is applied it

\begin{itemize}
\item One of De's mentors called this religion vs evidence
\end{itemize}

\item Example: He takes a survey that has us all saying that we feel
relatively safe, but we don't feel calm and we feel stressed out

\begin{itemize}
\item With this data you can draw two conclusions

\begin{itemize}
\item We are in a world where we are stressed
\item We are a world where we aren't stressed
\end{itemize}

\item There has to be one or the other

\begin{itemize}
\item The evidence supports the first one, but one cannot prove
anything
\end{itemize}
\end{itemize}
\end{itemize}

\item The pattern is an extrapolation of the data

\begin{itemize}
\item It's considered religion
\end{itemize}
\end{itemize}
\end{document}
