% Created 2021-09-27 Mon 12:01
% Intended LaTeX compiler: xelatex
\documentclass[letterpaper]{article}
\usepackage{graphicx}
\usepackage{grffile}
\usepackage{longtable}
\usepackage{wrapfig}
\usepackage{rotating}
\usepackage[normalem]{ulem}
\usepackage{amsmath}
\usepackage{textcomp}
\usepackage{amssymb}
\usepackage{capt-of}
\usepackage{hyperref}
\setlength{\parindent}{0pt}
\usepackage[margin=1in]{geometry}
\usepackage{fontspec}
\usepackage{svg}
\usepackage{cancel}
\usepackage{indentfirst}
\setmainfont[ItalicFont = LiberationSans-Italic, BoldFont = LiberationSans-Bold, BoldItalicFont = LiberationSans-BoldItalic]{LiberationSans}
\newfontfamily\NHLight[ItalicFont = LiberationSansNarrow-Italic, BoldFont       = LiberationSansNarrow-Bold, BoldItalicFont = LiberationSansNarrow-BoldItalic]{LiberationSansNarrow}
\newcommand\textrmlf[1]{{\NHLight#1}}
\newcommand\textitlf[1]{{\NHLight\itshape#1}}
\let\textbflf\textrm
\newcommand\textulf[1]{{\NHLight\bfseries#1}}
\newcommand\textuitlf[1]{{\NHLight\bfseries\itshape#1}}
\usepackage{fancyhdr}
\pagestyle{fancy}
\usepackage{titlesec}
\usepackage{titling}
\makeatletter
\lhead{\textbf{\@title}}
\makeatother
\rhead{\textrmlf{Compiled} \today}
\lfoot{\theauthor\ \textbullet \ \textbf{2021-2022}}
\cfoot{}
\rfoot{\textrmlf{Page} \thepage}
\renewcommand{\tableofcontents}{}
\titleformat{\section} {\Large} {\textrmlf{\thesection} {|}} {0.3em} {\textbf}
\titleformat{\subsection} {\large} {\textrmlf{\thesubsection} {|}} {0.2em} {\textbf}
\titleformat{\subsubsection} {\large} {\textrmlf{\thesubsubsection} {|}} {0.1em} {\textbf}
\setlength{\parskip}{0.45em}
\renewcommand\maketitle{}
\author{Huxley}
\date{\today}
\title{Gel electrophoresis prep}
\hypersetup{
 pdfauthor={Huxley},
 pdftitle={Gel electrophoresis prep},
 pdfkeywords={},
 pdfsubject={},
 pdfcreator={Emacs 28.0.50 (Org mode 9.4.4)}, 
 pdflang={English}}
\begin{document}

\tableofcontents

\#ref \#ret

\noindent\rule{\textwidth}{0.5pt}

\section{Prompt}
\label{sec:orge98bddd}
\begin{enumerate}
\item Read through
\href{https://www.khanacademy.org/science/ap-biology/gene-expression-and-regulation/biotechnology/a/gel-electrophoresis}{this
brief article (Links to an external site.)} about the process of gel
electrophoresis.
\item Read the next lab protocol
(\href{https://docs.google.com/document/d/1Cr\_bvcfKP42KCDmol\_jTWAJKdojJhf7pMqpu18kyKho/edit?usp=sharing}{lab
3 gel electrophoresis (Links to an external site.)}) and note any
questions you have so that you can ask them in class before we start.
\item \textbf{Submit brief responses to the following}:

\begin{enumerate}
\item What is the specific feature of DNA that causes it to move toward
the positively charged side of the gel-running apparatus? 
\item Shorter DNA fragments travel farther along the gel than longer DNA
fragments. Why is this?
\item When you analyze your specific PCR product using gel
electrophoresis, what do you expect to see in the final image of
the stained gel (assume your PCR reaction was successful)?
\end{enumerate}
\end{enumerate}

\noindent\rule{\textwidth}{0.5pt}

\section{Responses}
\label{sec:orgb19ce49}
\begin{enumerate}
\item \textbf{What is the specific feature of DNA that causes it to move toward
the positively charged side of the gel-running apparatus?}

\begin{enumerate}
\item DNA fragments are negatively charged. Thus, they move toward the
positive charge.
\end{enumerate}

\item \textbf{Shorter DNA fragments travel farther along the gel than longer DNA
fragments. Why is this?}

\begin{enumerate}
\item DNA fragments have the same amount of charge regardless of their
mass. Thus, the fragments with less mass are more greatly
effected.
\end{enumerate}

\item *When you analyze your specific PCR product using gel
electrophoresis, what do you expect to see in the final image of the
stained gel (assume your PCR reaction was successful)?*

\begin{enumerate}
\item I would expect to see a band, or a lot more DNA, in the bp length
section of the segment that was multiplied with the PCR reaction.
\end{enumerate}
\end{enumerate}

\section{CW:}
\label{sec:orgdc151d6}
\begin{quote}


\begin{enumerate}
\item Move through the
\href{https://learn.genetics.utah.edu/content/labs/gel/}{gel
electrophoresis interactive tutorial} (enjoy the goofy sound
effects).
\item Using one or more sound sources, extend your learning about a
specific aspect of gel electrophoresis that interests you
(e.g. physical properties of agarose, movement of biomolecules
through agarose, staining DNA, DNA size standard ladders, invention
of the technique, automation, alternative applications of
electrophoresis, or another topic of your choosing). 
\item \textbf{Write and submit} a brief  (1 paragraph) summary of what you've
learned and include links to your sources.
\end{enumerate}
\end{quote}

The rabbit hole begins..
\url{https://en.wikipedia.org/wiki/Gel\_electrophoresis?scrlybrkr=940ae169}
\url{https://www.google.com/search?client=opera\&q=agarose\&sourceid=opera\&ie=UTF-8\&oe=UTF-8}
\url{https://en.wikipedia.org/wiki/Agar}
\url{https://en.wikipedia.org/wiki/Agarose\_gel\_electrophoresis}
\end{document}
