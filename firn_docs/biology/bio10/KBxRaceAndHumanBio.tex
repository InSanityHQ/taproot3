% Created 2021-09-27 Mon 12:01
% Intended LaTeX compiler: xelatex
\documentclass[letterpaper]{article}
\usepackage{graphicx}
\usepackage{grffile}
\usepackage{longtable}
\usepackage{wrapfig}
\usepackage{rotating}
\usepackage[normalem]{ulem}
\usepackage{amsmath}
\usepackage{textcomp}
\usepackage{amssymb}
\usepackage{capt-of}
\usepackage{hyperref}
\setlength{\parindent}{0pt}
\usepackage[margin=1in]{geometry}
\usepackage{fontspec}
\usepackage{svg}
\usepackage{cancel}
\usepackage{indentfirst}
\setmainfont[ItalicFont = LiberationSans-Italic, BoldFont = LiberationSans-Bold, BoldItalicFont = LiberationSans-BoldItalic]{LiberationSans}
\newfontfamily\NHLight[ItalicFont = LiberationSansNarrow-Italic, BoldFont       = LiberationSansNarrow-Bold, BoldItalicFont = LiberationSansNarrow-BoldItalic]{LiberationSansNarrow}
\newcommand\textrmlf[1]{{\NHLight#1}}
\newcommand\textitlf[1]{{\NHLight\itshape#1}}
\let\textbflf\textrm
\newcommand\textulf[1]{{\NHLight\bfseries#1}}
\newcommand\textuitlf[1]{{\NHLight\bfseries\itshape#1}}
\usepackage{fancyhdr}
\pagestyle{fancy}
\usepackage{titlesec}
\usepackage{titling}
\makeatletter
\lhead{\textbf{\@title}}
\makeatother
\rhead{\textrmlf{Compiled} \today}
\lfoot{\theauthor\ \textbullet \ \textbf{2021-2022}}
\cfoot{}
\rfoot{\textrmlf{Page} \thepage}
\renewcommand{\tableofcontents}{}
\titleformat{\section} {\Large} {\textrmlf{\thesection} {|}} {0.3em} {\textbf}
\titleformat{\subsection} {\large} {\textrmlf{\thesubsection} {|}} {0.2em} {\textbf}
\titleformat{\subsubsection} {\large} {\textrmlf{\thesubsubsection} {|}} {0.1em} {\textbf}
\setlength{\parskip}{0.45em}
\renewcommand\maketitle{}
\date{\today}
\title{}
\hypersetup{
 pdfauthor={},
 pdftitle={},
 pdfkeywords={},
 pdfsubject={},
 pdfcreator={Emacs 28.0.50 (Org mode 9.4.4)}, 
 pdflang={English}}
\begin{document}

\tableofcontents

\begin{center}
\begin{tabular}{l}
title: HW: Race and human biology/genetics\\
context: BIO201\\
author: Huxley\\
source: \#index\\
\end{tabular}
\end{center}

\#ret

\noindent\rule{\textwidth}{0.5pt}

\textbf{QUESTIONS:}

\textbf{Reflect on what you learned from this resource about the connection
between science/medicine and ideas about race:}

\begin{enumerate}
\item \textbf{What resource did you choose and what important ideas did you
explore that you would hope could be known more widely?} I chose the
final Ted Talk, \href{https://www.youtube.com/watch?v=QOSPNVunyFQ}{Skin
Color is an Illusion}. It talked about the important ideas
surrounding different skin colors and their health implications in
relation to vitamin D.
\end{enumerate}

\noindent\rule{\textwidth}{0.5pt}

\begin{enumerate}
\item \textbf{What was most surprising or interesting to you in particular and
why?} I didn't know Darwin didn't pick up on the connection between
geography and skin color. I always assumed he had.
\end{enumerate}

\noindent\rule{\textwidth}{0.5pt}

\begin{enumerate}
\item *How should science and/or medicine address their role in
perpetuating ideas about race and racism? (you can focus on the
specific issue you explored or think more broadly here). If your
resource didn't feel focused enough to answer this question, fell
free to instead tell me more here about what you learned and how it
connects to the genetics stuff we've been learning in class.* I don't
view the entirety of \texttt{science} or \texttt{medicine} as entities. Science
cannot act. I would argue instead that \texttt{science} is a way of viewing
the world, and perhaps the knowledge collected from this lens. People
can get it wrong, but the concept of \texttt{science} or \texttt{medicine} is not
the problem. Instead, \texttt{science} is being wielded poorly. Frankly,
fundamentally, the problem lies in larger issues and biases. Supposed
\texttt{science} has and will continue to be twisted into whatever before
being presented to the media.
\end{enumerate}
\end{document}
