% Created 2021-09-27 Mon 12:01
% Intended LaTeX compiler: xelatex
\documentclass[letterpaper]{article}
\usepackage{graphicx}
\usepackage{grffile}
\usepackage{longtable}
\usepackage{wrapfig}
\usepackage{rotating}
\usepackage[normalem]{ulem}
\usepackage{amsmath}
\usepackage{textcomp}
\usepackage{amssymb}
\usepackage{capt-of}
\usepackage{hyperref}
\setlength{\parindent}{0pt}
\usepackage[margin=1in]{geometry}
\usepackage{fontspec}
\usepackage{svg}
\usepackage{cancel}
\usepackage{indentfirst}
\setmainfont[ItalicFont = LiberationSans-Italic, BoldFont = LiberationSans-Bold, BoldItalicFont = LiberationSans-BoldItalic]{LiberationSans}
\newfontfamily\NHLight[ItalicFont = LiberationSansNarrow-Italic, BoldFont       = LiberationSansNarrow-Bold, BoldItalicFont = LiberationSansNarrow-BoldItalic]{LiberationSansNarrow}
\newcommand\textrmlf[1]{{\NHLight#1}}
\newcommand\textitlf[1]{{\NHLight\itshape#1}}
\let\textbflf\textrm
\newcommand\textulf[1]{{\NHLight\bfseries#1}}
\newcommand\textuitlf[1]{{\NHLight\bfseries\itshape#1}}
\usepackage{fancyhdr}
\pagestyle{fancy}
\usepackage{titlesec}
\usepackage{titling}
\makeatletter
\lhead{\textbf{\@title}}
\makeatother
\rhead{\textrmlf{Compiled} \today}
\lfoot{\theauthor\ \textbullet \ \textbf{2021-2022}}
\cfoot{}
\rfoot{\textrmlf{Page} \thepage}
\renewcommand{\tableofcontents}{}
\titleformat{\section} {\Large} {\textrmlf{\thesection} {|}} {0.3em} {\textbf}
\titleformat{\subsection} {\large} {\textrmlf{\thesubsection} {|}} {0.2em} {\textbf}
\titleformat{\subsubsection} {\large} {\textrmlf{\thesubsubsection} {|}} {0.1em} {\textbf}
\setlength{\parskip}{0.45em}
\renewcommand\maketitle{}
\author{Huxley}
\date{\today}
\title{SNP PCR research}
\hypersetup{
 pdfauthor={Huxley},
 pdftitle={SNP PCR research},
 pdfkeywords={},
 pdfsubject={},
 pdfcreator={Emacs 28.0.50 (Org mode 9.4.4)}, 
 pdflang={English}}
\begin{document}

\tableofcontents

\#ret

\noindent\rule{\textwidth}{0.5pt}

\%\%\#\# General

\subsection{SNiPs}
\label{sec:orgc026110}
\begin{itemize}
\item rs4680

\begin{itemize}
\item pain tolerance, worrier/warrior
\item \url{https://www.snpedia.com/index.php/Rs4680}
\end{itemize}

\item CYP1A2 gene

\begin{itemize}
\item specifically, rs2069514

\begin{itemize}
\item encodes for the 1C variant,
\item which encodes for slow caffeine metabolism
\end{itemize}
\end{itemize}

\item rs6902875

\begin{itemize}
\item episodic memory
\item \url{https://pubmed.ncbi.nlm.nih.gov/25317765/}
\end{itemize}
\end{itemize}

\section{Write-up}
\label{sec:org0335bec}
\%\%

\subsection{A : : \texttt{rs4680}}
\label{sec:orgfb143be}
Also known as Val158Me, rs4680 is an extensively studied SNP located in
the COMT gene. The COMT enzyme, encoded by the COMT gene, is responsible
for breaking down dopamine in the brain's prefrontal cortex
(\href{https://www.snpedia.com/index.php/Rs4680}{cite}). rs4680 causes the
enzyme to function roughly 25\% as efficiently as the wild type. The
result of wild-type versus rs4680 is commonly referred to as the warrior
versus worrier hypothesis
(\href{https://pubmed.ncbi.nlm.nih.gov/17008817/}{cite}). A worrier, one
with the rs4680 SNP, has higher dopamine levels. Thus, supposedly, they
should have lower pain tolerance, be more prone to stress as well as
more exploratory, and more efficient at information processing.
Conversely, the wild-type warriors should have higher pain tolerance, be
less prone to stress, less exploratory, and less efficient at cognition
in most conditions
(\href{https://www.huffpost.com/entry/stress-management\_b\_2671591}{cite}).

\subsection{B : : \texttt{rs2069514}}
\label{sec:org744323f}
rs2069514 is a 1C type allele of the CYP1A2 gene
(\href{https://www.snpedia.com/index.php/CYP1A2}{cite}). CYP1A2 encodes one
of the cytochrome P450 mixed function oxidase enzymes, all of which are
vital in the metabolism of xenobiotics
(\href{https://www.ncbi.nlm.nih.gov/pmc/articles/PMC4309856/}{cite}). One
such xenobiotic is caffeine, the processing of which is affected by 1C
and 1F type mutations on the CYP1A2 gene. Humans with 1F type mutations
are known as 'fast' caffeine metabolizers, whereas 1C type mutations
lead to 'slow' caffeine metabolism. Those who carry at least one 1C type
mutation will be slower at processing caffeine, and thus, will be more
affected by it. rs2069514 is one such 1C mutation, leading to decreased
activity by the CYP1A2 enzyme (cite
\href{https://www.snpedia.com/index.php/Rs2069514}{a},
\href{https://www.snpedia.com/index.php/CYP1A2}{b}).

\subsection{C : : \texttt{rs6902875}}
\label{sec:orgdaee102}
Much less is known about rs6902875, except that it is related to
significantly better episodic memory. This was tested specifically in
seniors (\href{https://www.snpedia.com/index.php/Rs6902875}{cite}). After
genome-wide linkage analysis on 467 LLFS (Long Life Family Study)
participants, a significant link between the 6q24 region and exceptional
episodic memory was found. More specifically, rs9321334 and rs6902875
were both nominally significantly associated
(\href{https://www.jwatch.org/na36191/2014/11/12/exceptional-memory-performance-elders-linked-candidate}{cite}).
The region harboring rs6902875 --- MOXD1 --- is required for the
synthesis of norepinephrine, a neurotransmitter involved with cognition.
When participants with the APOE E4 allele were removed, rs6902875 became
much more statistically significant than rs9321334
(\href{https://pubmed.ncbi.nlm.nih.gov/25317765/}{cite}).
\end{document}
