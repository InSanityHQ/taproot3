% Created 2021-09-27 Mon 11:51
% Intended LaTeX compiler: xelatex
\documentclass[letterpaper]{article}
\usepackage{graphicx}
\usepackage{grffile}
\usepackage{longtable}
\usepackage{wrapfig}
\usepackage{rotating}
\usepackage[normalem]{ulem}
\usepackage{amsmath}
\usepackage{textcomp}
\usepackage{amssymb}
\usepackage{capt-of}
\usepackage{hyperref}
\setlength{\parindent}{0pt}
\usepackage[margin=1in]{geometry}
\usepackage{fontspec}
\usepackage{svg}
\usepackage{cancel}
\usepackage{indentfirst}
\setmainfont[ItalicFont = LiberationSans-Italic, BoldFont = LiberationSans-Bold, BoldItalicFont = LiberationSans-BoldItalic]{LiberationSans}
\newfontfamily\NHLight[ItalicFont = LiberationSansNarrow-Italic, BoldFont       = LiberationSansNarrow-Bold, BoldItalicFont = LiberationSansNarrow-BoldItalic]{LiberationSansNarrow}
\newcommand\textrmlf[1]{{\NHLight#1}}
\newcommand\textitlf[1]{{\NHLight\itshape#1}}
\let\textbflf\textrm
\newcommand\textulf[1]{{\NHLight\bfseries#1}}
\newcommand\textuitlf[1]{{\NHLight\bfseries\itshape#1}}
\usepackage{fancyhdr}
\pagestyle{fancy}
\usepackage{titlesec}
\usepackage{titling}
\makeatletter
\lhead{\textbf{\@title}}
\makeatother
\rhead{\textrmlf{Compiled} \today}
\lfoot{\theauthor\ \textbullet \ \textbf{2021-2022}}
\cfoot{}
\rfoot{\textrmlf{Page} \thepage}
\renewcommand{\tableofcontents}{}
\titleformat{\section} {\Large} {\textrmlf{\thesection} {|}} {0.3em} {\textbf}
\titleformat{\subsection} {\large} {\textrmlf{\thesubsection} {|}} {0.2em} {\textbf}
\titleformat{\subsubsection} {\large} {\textrmlf{\thesubsubsection} {|}} {0.1em} {\textbf}
\setlength{\parskip}{0.45em}
\renewcommand\maketitle{}
\author{Taproot}
\date{\today}
\title{Curve Sketching}
\hypersetup{
 pdfauthor={Taproot},
 pdftitle={Curve Sketching},
 pdfkeywords={},
 pdfsubject={},
 pdfcreator={Emacs 28.0.50 (Org mode 9.4.4)}, 
 pdflang={English}}
\begin{document}

\tableofcontents


\section{Curve Sketching\hfill{}\textsc{unit2}}
\label{sec:org44a4f5c}

Goal: Draw a graph of \(f(x)\) using \(f'(x)\) and \(f''(x)\) (see \href{calculating-derivatives.org}{Calculating Derivatives}).
\emph{Warning!} Don't abandon precalculus knowledge and common sense! Lots of common sense is involved in this and the calculus mostly just fills in the gaps.

Two principles (really one) will aid this endeavour:
\begin{itemize}
\item If \(f' > 0\), \(f\) is increasing (and vice versa).
\item If \(f'' > 0\), \(f'\) is increasing (and vice versa).
\end{itemize}

\uline{Ex 1}. \(f(x) + 3x - x^3\)
\(f'(x) = 3-3x^2 \rightarrow 3(1-x)(1+x)\)
\(f'(x) > 0\) in the range \(-1<x<1\) and therefore \(f\) is increasing in that range.
\(f'(x) < 0\) in the range \(1<x\) and therefore \(f\) is increasing in that range (similarly for the range \(x < -1\).

General shape of function can be deduced from these facts.

\textbf{DEFINITION} If \(f'(x_0) = 0\), \(x_0\) is a \emph{critical point}. At this point \(f(x_0)\) is called the \emph{critical value}.
\end{document}
