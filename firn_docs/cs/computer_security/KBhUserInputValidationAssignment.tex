% Created 2021-09-28 Tue 09:33
% Intended LaTeX compiler: xelatex
\documentclass[letterpaper]{article}
\usepackage{graphicx}
\usepackage{grffile}
\usepackage{longtable}
\usepackage{wrapfig}
\usepackage{rotating}
\usepackage[normalem]{ulem}
\usepackage{amsmath}
\usepackage{textcomp}
\usepackage{amssymb}
\usepackage{capt-of}
\usepackage{hyperref}
\setlength{\parindent}{0pt}
\usepackage[margin=1in]{geometry}
\usepackage{fontspec}
\usepackage{svg}
\usepackage{cancel}
\usepackage{indentfirst}
\setmainfont[ItalicFont = LiberationSans-Italic, BoldFont = LiberationSans-Bold, BoldItalicFont = LiberationSans-BoldItalic]{LiberationSans}
\newfontfamily\NHLight[ItalicFont = LiberationSansNarrow-Italic, BoldFont       = LiberationSansNarrow-Bold, BoldItalicFont = LiberationSansNarrow-BoldItalic]{LiberationSansNarrow}
\newcommand\textrmlf[1]{{\NHLight#1}}
\newcommand\textitlf[1]{{\NHLight\itshape#1}}
\let\textbflf\textrm
\newcommand\textulf[1]{{\NHLight\bfseries#1}}
\newcommand\textuitlf[1]{{\NHLight\bfseries\itshape#1}}
\usepackage{fancyhdr}
\pagestyle{fancy}
\usepackage{titlesec}
\usepackage{titling}
\makeatletter
\lhead{\textbf{\@title}}
\makeatother
\rhead{\textrmlf{Compiled} \today}
\lfoot{\theauthor\ \textbullet \ \textbf{2021-2022}}
\cfoot{}
\rfoot{\textrmlf{Page} \thepage}
\renewcommand{\tableofcontents}{}
\titleformat{\section} {\Large} {\textrmlf{\thesection} {|}} {0.3em} {\textbf}
\titleformat{\subsection} {\large} {\textrmlf{\thesubsection} {|}} {0.2em} {\textbf}
\titleformat{\subsubsection} {\large} {\textrmlf{\thesubsubsection} {|}} {0.1em} {\textbf}
\setlength{\parskip}{0.45em}
\renewcommand\maketitle{}
\author{Houjun Liu}
\date{\today}
\title{User Input Validation Assignment}
\hypersetup{
 pdfauthor={Houjun Liu},
 pdftitle={User Input Validation Assignment},
 pdfkeywords={},
 pdfsubject={},
 pdfcreator={Emacs 28.0.50 (Org mode 9.4.4)}, 
 pdflang={English}}
\begin{document}

\tableofcontents


\section{Clojure, in brief\ldots{}}
\label{sec:orgf3562db}
The programming language \href{https://clojure.org/}{Clojure}, a variant of Lisp built atop the Java Virtual Machine, is known to be a safe language to low level attacks such as buffer overflow due to the insulation of its execution space.

To preserve the functionality of being a Lisp, however, the Clojure preserves most of the powerful syntax-manipulation tools like unquote slicing and compile-time macros.

Before we could introduce the vulnerabilities that could come with the misuse of these tools, it is important to introduce these tools first.

\subsection{Unquote slicing}
\label{sec:orgb8a7ca4}
"Unquoting" is a facility to introduce \texttt{code-mode} elements into \texttt{data-mode} lists. Here, for instance, is a regular data mode list that evaluates to value \texttt{(1 2 3 4)}:

\begin{verbatim}
'(1 2 3 4)
\end{verbatim}

\begin{verbatim}
(1 2 3 4)
\end{verbatim}


Leveraging "unquoting", we could instead execute a \emph{statement} that returns, for instance, \texttt{4} in place of the number \texttt{4} in the list instead of supplying the r-value \texttt{4} itself: therefore resulting in the same list but introducing live code during its parsing.

The syntax of unquoting is like so:

\begin{verbatim}
`(1 2 3 ~(* 2 2)) ; notice the ` instead of ' to denote a unquotable list
\end{verbatim}

\begin{verbatim}
(1 2 3 4)
\end{verbatim}


As you could see, the result is still \texttt{(1 2 3 4)}, yet the latter case leveraged \texttt{(* 2 2); => 4} to \emph{unquote} the value \texttt{4} into the list using live code.

\subsection{Macros}
\label{sec:orge25f854}
Marcos are a powerful way of manipulating the syntax of code. Essentially, they are compile-time unraveled directives that take arguments as \texttt{data-mode} lists and return a sliced \texttt{data-mode} expression.

Here's a basic example of the functionality of Macros. Take, for instance, the following expression adding two numbers:

\begin{verbatim}
(+ 1 2)
\end{verbatim}

\begin{verbatim}
3
\end{verbatim}


We will now write a macro whose behavior is to reverse the order of the first two expressions before evaluating, like so:

\begin{verbatim}
(defmacro rev-arg [a b c]
    `(~b ~a ~c))
\end{verbatim}

\begin{verbatim}
#'user/rev-arg
\end{verbatim}


Now, leveraging this macro, we could write the addition statement in an now admittedly weird manner:

\begin{verbatim}
(rev-arg 1 + 2)
\end{verbatim}

\begin{verbatim}
3
\end{verbatim}


As you could see, this macro did not \emph{evaluate} its arguments, but instead took them as \texttt{data-mode} items and swapped their order. The return value of the macro is yet another \texttt{data-mode} list representing the unwrapped expression (now with the order corrected), which is finally evaluated on runtime.

\section{Dangers of Macro Arguments}
\label{sec:orgaad1ea0}
Macros do not evaluate nor --- by default --- make any attempts to parse the contents of it arguments. Therefore, if an argument is unquoted wholesale into the result unwrapped expression, it is functionally equivalent to calling \texttt{eval} upon the statement is it will not be parsed at \emph{compile-time} but instead on \emph{run-time}.

Especially dangerous an action is if the argument of macros come from user input, as --- with functions --- the argument would either be passed consciously as fully data or evaluated with \texttt{eval} but is potentially sliced into the source tree and evaluated automatically on run-time for macros.

Here's the simplest example of runtime evaluation caused by a macro:

\begin{verbatim}
(defmacro secreteval [a]
    `(do
	(println "one")
	~a))
\end{verbatim}

\begin{verbatim}
#'user/secreteval
\end{verbatim}


And now, we call this macro to demonstrate the run-time eval.

\begin{verbatim}
(secreteval (println "two"))
\end{verbatim}

\begin{verbatim}
one
two
\end{verbatim}


The \texttt{stdout} result is printed in order of \texttt{one} \texttt{two}, meaning that, indeed, the execution of the argument \texttt{(println "two")} not only took place but took place \emph{after} the execution of the first part of the macro.

Compare this to the following functionally almost-similar \emph{function}:

\begin{verbatim}
(defn nosecreteval [a]
  (do
    (println "one")
    a))
\end{verbatim}

\begin{verbatim}
#'user/nosecreteval
\end{verbatim}


And calling it with the same input\ldots{}

\begin{verbatim}
(nosecreteval (println "two"))
\end{verbatim}

\begin{verbatim}
two
one
\end{verbatim}


As you could see, the argument is evaluated and passed evaluated \emph{before} the execution of the macro (likely compiled away by the JVM) and, should user data be passed this was as \emph{data}, will not be evaluated at all due to its persistent as a list.

\section{Preventing Macro Argument Evaluation}
\label{sec:org6d9ded0}
The simplest mechanism by which this type of evaluation could be prevented is via the use of assert statements or even list manipulation to establish basic assumptions regarding the argument.

\subsection{Nonexample}
\label{sec:org9533b19}
For instance, we write the following macro create a list of two elements.

\begin{verbatim}
(defmacro bad-concat [a]
    `(list 1 ~a))
\end{verbatim}

\begin{verbatim}
#'user/bad-concat
\end{verbatim}


Calling it with some "normal" elements will act as you'd expect.

\begin{verbatim}
(bad-concat 12)
\end{verbatim}

\begin{verbatim}
(1 12)
\end{verbatim}


However, you could introduce code execution by, for instance, passing an expression that produces a side effect.

\begin{verbatim}
(bad-concat (println "Code execution!"))
\end{verbatim}

\begin{verbatim}
Code execution!
\end{verbatim}


As you could see, the statement \texttt{code execution} is \emph{printed}, instead of all of \texttt{(println "aoenust")} being concatenated to the list to \texttt{1}, as the author of the program presumably desired.

\subsection{\texttt{assert} based solution}
\label{sec:orgcd31098}
One way to solve this is by asserting that the argument \texttt{a} is an \texttt{atom} --- meaning it does not contain \texttt{s-expressions} that could be accidentally evaluated. This would be implemented as follows:

\begin{verbatim}
(defmacro better-concat [a]
    (assert (not (list? a)))
    `(list 1 ~a))
\end{verbatim}

\begin{verbatim}
#'user/better-concat
\end{verbatim}


The first example returns what one would expect:

\begin{verbatim}
(better-concat 12)
\end{verbatim}

\begin{verbatim}
(1 12)
\end{verbatim}


And the statement with side-effects would, on \emph{compile-time} throw an \texttt{AssertError}.

\begin{verbatim}
(better-concat (println "Code execution!"))
\end{verbatim}

\begin{verbatim}
class clojure.lang.Compiler$CompilerException
\end{verbatim}

\subsection{More ergonomic solution}
\label{sec:org452d5cb}
There is an even more ergonomic solution to this problem. Instead of validating input, we could coerce the input to \texttt{data-mode} using the quote-expression, essentially sanitizing it. Hence, we could create an even more ergonomic concat like so:

\begin{verbatim}
(defmacro ergo-concat [a]
    `(list 1 '~a))
\end{verbatim}

\begin{verbatim}
#'user/ergo-concat
\end{verbatim}


The first example still remains the same:

\begin{verbatim}
(ergo-concat 12)
\end{verbatim}

\begin{verbatim}
(1 12)
\end{verbatim}


and the side-effect statement will perform more ergonomically: 

\begin{verbatim}
(ergo-concat (println "Code execution!"))
\end{verbatim}

\begin{verbatim}
(1 (println "Code execution!"))
\end{verbatim}


As you could see, the list is actually, properly sliced in instead of triggering executing code.

\section{Legal/Ethical Concerns}
\label{sec:orgeb90623}
The problem that exist with Clojure macro evaluation issues is that it is very easily (and perhaps commonly) created by even seasoned Clojurists.

Exploitation of these issues could take a variety of forms, but the main and easiest form of these payloads probably are introduced through programs that take Clojurescript/Clojure as extension languages, which is becoming more common as it leverages the extensibility of these languages.

One possible attacker may create an extension or a string on a fileserver fetched by the extension that is used and evaluated as an argument of a badly-written macro. This may subsequently access APIs (or, in worse cases where the plugin passes a macro exposed by server-side Clojure code, even perform server-side modifications.)

Such action would, therefore, create modifications and unlawful access to systems that may not even have been a member of the base system that was exploited (for instance, inserting \texttt{slurp("../../../test.txt")} to read a random file a few directories up.)

This would, of course, be under the purview of \texttt{18USC§1030.a1}, which defines "knowingly accessed a computer without authorization" as something punishable under \texttt{18USC§1030.c}. Furthermore, it would likely be an unethical use of both the plug in and the target system as it is directly creating the possibility and channels for completely unauthorized access.

Due to the proliferation of such code, however, and the fact that extensions are knowingly accessible from the public, it is also likely that the individual attacking may not know this loophole at all and proceeded to accidentally invoke a macro as a function: for instance, passing \texttt{slurp} to a symbol expecting to slurp local files but due to it actually being a macro results in access of server files.

As long as the unknowing "attacker" stops and notifies the owner of the system after realizing this, I don't believe it would be ethically challenging.
\end{document}
