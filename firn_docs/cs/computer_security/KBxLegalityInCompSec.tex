% Created 2021-09-27 Mon 12:02
% Intended LaTeX compiler: xelatex
\documentclass[letterpaper]{article}
\usepackage{graphicx}
\usepackage{grffile}
\usepackage{longtable}
\usepackage{wrapfig}
\usepackage{rotating}
\usepackage[normalem]{ulem}
\usepackage{amsmath}
\usepackage{textcomp}
\usepackage{amssymb}
\usepackage{capt-of}
\usepackage{hyperref}
\setlength{\parindent}{0pt}
\usepackage[margin=1in]{geometry}
\usepackage{fontspec}
\usepackage{svg}
\usepackage{cancel}
\usepackage{indentfirst}
\setmainfont[ItalicFont = LiberationSans-Italic, BoldFont = LiberationSans-Bold, BoldItalicFont = LiberationSans-BoldItalic]{LiberationSans}
\newfontfamily\NHLight[ItalicFont = LiberationSansNarrow-Italic, BoldFont       = LiberationSansNarrow-Bold, BoldItalicFont = LiberationSansNarrow-BoldItalic]{LiberationSansNarrow}
\newcommand\textrmlf[1]{{\NHLight#1}}
\newcommand\textitlf[1]{{\NHLight\itshape#1}}
\let\textbflf\textrm
\newcommand\textulf[1]{{\NHLight\bfseries#1}}
\newcommand\textuitlf[1]{{\NHLight\bfseries\itshape#1}}
\usepackage{fancyhdr}
\pagestyle{fancy}
\usepackage{titlesec}
\usepackage{titling}
\makeatletter
\lhead{\textbf{\@title}}
\makeatother
\rhead{\textrmlf{Compiled} \today}
\lfoot{\theauthor\ \textbullet \ \textbf{2021-2022}}
\cfoot{}
\rfoot{\textrmlf{Page} \thepage}
\renewcommand{\tableofcontents}{}
\titleformat{\section} {\Large} {\textrmlf{\thesection} {|}} {0.3em} {\textbf}
\titleformat{\subsection} {\large} {\textrmlf{\thesubsection} {|}} {0.2em} {\textbf}
\titleformat{\subsubsection} {\large} {\textrmlf{\thesubsubsection} {|}} {0.1em} {\textbf}
\setlength{\parskip}{0.45em}
\renewcommand\maketitle{}
\author{huxley narvit}
\date{\today}
\title{legality in computer security}
\hypersetup{
 pdfauthor={huxley narvit},
 pdftitle={legality in computer security},
 pdfkeywords={},
 pdfsubject={},
 pdfcreator={Emacs 28.0.50 (Org mode 9.4.4)}, 
 pdflang={English}}
\begin{document}

\tableofcontents

\#ret

\noindent\rule{\textwidth}{0.5pt}

\section{Ethics Reflection}
\label{sec:org1ee5feb}
\textbf{Stories}:
\href{https://groups.csail.mit.edu/mac/classes/6.805/articles/morris-worm.html}{The
Worm} \href{https://www.wired.com/2015/04/silk-road-1/}{Silk Road}

\#\#\# The Robert Morris Internet Worm \%\% released a self replicating, self
propogating worm which exploited the sendmail program slowed down a
bunch of stuff, spread like wildfire he got convicted pretty hard

\begin{verbatim}
why did he release it...?
\end{verbatim}

\%\%

\begin{enumerate}
\item Questions
\label{sec:orga427793}
\begin{enumerate}
\item \textbf{What are the ethical choices people faced? What, if any, actions
would you consider unethical? Why?}

\begin{enumerate}
\item Of course, Morris releasing the worm in first place was an ethical
choice, but so was convicting him. Morris, even if just curious
and not intending to hurt, still did hurt and acted irresponsibly.
Thus, I would classify his choice as unethical. Those who
convicted Morris had to do the same ethical classification I just
did, and I have no reason to believe what they did was unethical.
\end{enumerate}

\item *What, if any, actions in this story do you think should be illegal?
What actions are actually illegal? Specify laws that might be
relevant, even if no one was caught or prosecuted.*

\begin{enumerate}
\item Releasing the worm was illegal as it violated the computer Fraud
and Abuse Act. Releasing the worm malware should most likely be
illegal as it damaged others private property, but defining the
group that should be illegal is a much harder task.
\end{enumerate}

\item \textbf{What things do you think the people involved could have done to
achieve their goals while staying within legal and ethical bounds?}

\begin{enumerate}
\item Morris's goals are quite unclear to me, besides just sheer
curiosity about how his worm would fare in the real world.
Satisfying this curiosity without releasing the worm into the real
world is not an easy task. Morris could have tried to run some
type of simulation, or fixed his program to make sure it didn't
break machines then install a fail safe, but at the end of the day
the question was about the real world and a true test.
\end{enumerate}

\item *What would you consider appropriate punishment? If relevant to your
story, how does that compare to the punishments that were handed
down?*

\begin{enumerate}
\item The punishment given -- 3 years probation, 400 hours community
service, and a fine of \$10,050 -- seems a little much to me.
Changing those 400 hours to having Morris help prevent future
attacks would be more reasonable given that he has proven himself
to be capable of sending attacks.
\end{enumerate}

\item \textbf{What are the technical lessons that can be learned to improve
security?}

\begin{enumerate}
\item Communication networks between devices are scary and need to be
checked closely lest more self-replicating worms wreak havoc.
\end{enumerate}
\end{enumerate}
\end{enumerate}

\subsubsection{Silk Road}
\label{sec:orgebd2e6d}
\begin{enumerate}
\item Questions
\label{sec:orgdc76d99}
\begin{enumerate}
\item \textbf{What are the ethical choices people faced? What, if any, actions
would you consider unethical? Why?}

\begin{enumerate}
\item It's not entirely clear whether or not the Silk Road itself was
ethical. It certainly seemed ethical to DPR and his libertarian
followers -- they believed that each transaction was a "step
toward universal freedom." Almost all the choices surrounding the
Silk Road are thus ethical choices, from developing it to selling
and buying on it to trying to shut it down. However, the Silk Road
did not abide by the accepted social contract, and thus, on
average, it would be unethical.
\end{enumerate}

\item *What, if any, actions in this story do you think should be illegal?
What actions are actually illegal? Specify laws that might be
relevant, even if no one was caught or prosecuted.*

\begin{enumerate}
\item While I am not educated nearly enough to have informed opinions on
the topic, legalizing the drug trade seems like a good idea to me.
It allows for safer usage, actual regulation, and isn't conducive
to the creation of organized crime networks. While DPR broke many
laws with his network, there has also been discussion of the law
enforcement agencies conducting unlawful investigations like
looking for and seizing evidence without a search warrant. While
unlawful searches should, by definition, be illegal, the ethics of
it are also not entirely clear to me. There is an argument to be
made about not waiting to cut through all the bureaucracy when you
believe innocent people are getting hurt and you have the ability
to stop it.
\end{enumerate}

\item \textbf{What things do you think the people involved could have done to
achieve their goals while staying within legal and ethical bounds?}

\begin{enumerate}
\item I wonder, could DPR simply have moved somewhere where this trading
would be legal? I guess not, because the goal was not to create a
drug trade but to "step toward universal freedom," because in
DPR's eyes the current system wasn't free. I doubt changing the
system by going through the system would have led to the results
DPR desired, and I'm sure DPR knew that. Thus, when the
infrastructure to radically change the law is not set in place by
the law, sidestepping the legal system becomes the inherent
solution. As for ethical bounds, I am not qualified to comment,
but I will do so anyways! DPR genuinely believes in his cause, and
his network could have been a step in the right direction. Or it
could have simply been something that ruined a massive quantity of
lives. Without knowing, I can't classify DPR's actions from a
utilitarian perspective.
\end{enumerate}

\item *What would you consider appropriate punishment? If relevant to your
story, how does that compare to the punishments that were handed
down?*

\begin{enumerate}
\item DPR was sentenced to life in prison. From the perspective of a
legal system, any sidestep this large does warrant a lifetime in
prison regardless of the actual content of the actions. From the
perspective of a human, I can't say if it was appropriate for the
same reasons as the question above.
\end{enumerate}

\item \textbf{What are the technical lessons that can be learned to improve
security?}

\begin{enumerate}
\item Don't trust people on the internet, they could be cops out to get
you. Don't reveal your true IP to the world. Set up your
information flow such that when people "flip" it doesn't
compromise your operation. Be wary of social engineering tactics.
\end{enumerate}
\end{enumerate}

\item Feedback
\label{sec:org9e87430}
Excellent work reflecting on the ethical choices in the Morris worm and
Silk Road stories. A few comments for your consideration:

\begin{quote}
Those who convicted Morris had to do the same ethical classification
\end{quote}

*I like this perspective! Do you think that a justice system should, in
general, consider whether actions were ethical in addition to legal?
Does ours?* Ours -- originally -- was meant to, at least according to
FIJA (Fully Informed Jury Association). That's why Jury Nullification
exists. I think this is part of the whole built in distrust of rulers in
the American legal system. What the justice system should consider is a
much harder question. One could argue that the laws should be made in
such a way where considering the ethics woudn't change anything
substantial, and thus considering ethics in the justice system is fixing
the symptoms rather than the problem.

\begin{quote}
Morris's goals are quite unclear to me, besides just sheer curiosity
about howhis worm would fare in the real world.
\end{quote}

\textbf{That sounds about right to me.}

\begin{quote}
Satisfying this curiosity without releasingthe worm into the real
world is not an easy task.
\end{quote}

\textbf{What about a closed network?} Yes, you could get close to an answer,
but to get a true answer, releasing it into the world would be
essentially the only way.

\begin{quote}
Changing those 400 hours to having Morris help prevent future attacks
\end{quote}

*I agree; this seems both more reasonable and more useful. This seems
like a place where a restorative justice framework offers helpful
perspective.* What about when restoration is not possible?

\begin{quote}
Almost all the choices surrounding the Silk Road are thus ethical
choices
\end{quote}

\textbf{What about the clearly illegal choices made to sustain and protect Silk
Road, such as having someone killed?} Ah, my bad. I meant 'ethical
choices' in the sense that the choices operated within the realm of
ethics, not 'ethical choices' as in the choice was positive. My point
was that many of the choices made around the Silk Road were motivated by
ethics, or had ethical considerations, but operated in vastly different
ethical frameworks. As for the choices positivity, operating from a
ethics-as-emergent-property-of-community perspective, I concluded that
the maintainers choices were negative because they didn't follow the
social contract.

\begin{quote}
Without knowing, I can't classify DPR's actions from a utilitarian
perspective.
\end{quote}

*I think we can make some pretty good guesses, based on how the
government reacted and what actually happened, right? Like, maybe Silk
Road could have been the catalyst for a more open and free market, but
would the US government (or any government, really) have let that happen
without a fight? I'm guessing no, and that many lives would be ruined in
the process (see: War on Drugs).*

Not necessarily a catalyst, but a start -- I don't believe the effects
of the Silk Road are anywhere near completed. We can't make guesses
based on what happened because most things haven't happened yet. I'm
sure the Silk Road had a massive impact on crypto, which is looking like
it will lead us over the brink of some massive changes. Just look at El
Salvador.

And yes, radical change often ruins lives and causes pain, but that does
not mean radical change is bad or that we shound't cause it. We just
need to hope and do our best to make sure that the good from the change
outweighs the bad.

\begin{quote}
Don't trust people on the internet, they could be cops out to get you.
\end{quote}

\textbf{Or worse.} Uhoh
\end{enumerate}
\end{document}
