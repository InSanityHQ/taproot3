% Created 2021-09-27 Mon 12:02
% Intended LaTeX compiler: xelatex
\documentclass[letterpaper]{article}
\usepackage{graphicx}
\usepackage{grffile}
\usepackage{longtable}
\usepackage{wrapfig}
\usepackage{rotating}
\usepackage[normalem]{ulem}
\usepackage{amsmath}
\usepackage{textcomp}
\usepackage{amssymb}
\usepackage{capt-of}
\usepackage{hyperref}
\setlength{\parindent}{0pt}
\usepackage[margin=1in]{geometry}
\usepackage{fontspec}
\usepackage{svg}
\usepackage{cancel}
\usepackage{indentfirst}
\setmainfont[ItalicFont = LiberationSans-Italic, BoldFont = LiberationSans-Bold, BoldItalicFont = LiberationSans-BoldItalic]{LiberationSans}
\newfontfamily\NHLight[ItalicFont = LiberationSansNarrow-Italic, BoldFont       = LiberationSansNarrow-Bold, BoldItalicFont = LiberationSansNarrow-BoldItalic]{LiberationSansNarrow}
\newcommand\textrmlf[1]{{\NHLight#1}}
\newcommand\textitlf[1]{{\NHLight\itshape#1}}
\let\textbflf\textrm
\newcommand\textulf[1]{{\NHLight\bfseries#1}}
\newcommand\textuitlf[1]{{\NHLight\bfseries\itshape#1}}
\usepackage{fancyhdr}
\pagestyle{fancy}
\usepackage{titlesec}
\usepackage{titling}
\makeatletter
\lhead{\textbf{\@title}}
\makeatother
\rhead{\textrmlf{Compiled} \today}
\lfoot{\theauthor\ \textbullet \ \textbf{2021-2022}}
\cfoot{}
\rfoot{\textrmlf{Page} \thepage}
\renewcommand{\tableofcontents}{}
\titleformat{\section} {\Large} {\textrmlf{\thesection} {|}} {0.3em} {\textbf}
\titleformat{\subsection} {\large} {\textrmlf{\thesubsection} {|}} {0.2em} {\textbf}
\titleformat{\subsubsection} {\large} {\textrmlf{\thesubsubsection} {|}} {0.1em} {\textbf}
\setlength{\parskip}{0.45em}
\renewcommand\maketitle{}
\author{Houjun Liu}
\date{\today}
\title{English graded discussion prep}
\hypersetup{
 pdfauthor={Houjun Liu},
 pdftitle={English graded discussion prep},
 pdfkeywords={},
 pdfsubject={},
 pdfcreator={Emacs 28.0.50 (Org mode 9.4.4)}, 
 pdflang={English}}
\begin{document}

\tableofcontents



\section{English graded discussion prep}
\label{sec:org26a3f50}
\subsection{Quote bin}
\label{sec:org617ea26}
\textbf{Page Numbers}: 16, 42, 43, 44, 46, 47, 51, 60

\noindent\rule{\textwidth}{0.5pt}

Begin here:

\subsection{Jack's Intro 0-1}
\label{sec:org5b56ff7}
How does Marlow's inner perspective come into dialog with the darkness
along his journey?

Can we look beyond his perspective of the darkness?

This asks two questions. What does Marlow's language reveal to us about
his journey and how, ultimately, does his internalities reveals?

Two stances: Marlow Brings Darkness, or Marlow Sees Darkness

\subsection{The Discussion}
\label{sec:org97082be}
The leader's job is to "moderate" the two others; others, DO NOT ADVANCE
THE ARGUMENT. Let the leader do it.

Bounce off ideas, and repeat/reply/clarify/ask questions. Feel free to
bud in.

\subsubsection{Ryan's Opening --- Marlow's Perspective brings Darkness 1-5}
\label{sec:org424c94c}
\begin{itemize}
\item Initially, Ryan believes that it is difficult to see beyond Marlow's
perspective
\item Story is from his perspective => so, of course, its Marlow that welds
the darkness!
\item However, it is possible to gauge the darkness by analyzing the dialog
with other characters to find their perspectives to find out what they
say
\item Issue! only perspectives we get is through European colonists is
ultimately biased

\begin{itemize}
\item We mostly get European views
\item And the two times a native speaks, its\ldots{}

\begin{itemize}
\item "Eat 'im"
\item “and Mistah Kurtz, He Dead
\end{itemize}

\item Because we only have two instances --- and they are gramatically
incoherent sentences --- can't accurately gauge the darkness through
non-marlow
\item Mention Acheobe
\end{itemize}
\end{itemize}

\textbf{CONFLIGHT! line} There is possibility to gauge darkness of land through
simply analyzing the environment to take it as it is from objective
language of Marlow.

\subsubsection{Zach --- Darkness already present 5-9}
\label{sec:org415595e}
\begin{itemize}
\item Both his environment and himself is dark
\item Surroundings are already dark b/c of Europeans
\item Find quotes about

\begin{enumerate}
\item People dying; people chained together
\item How that influence marlow's view, before and after
\end{enumerate}

\item Provide transition: there are unseen dark properties of his
environment that he does not see; why?
\item \textbf{Some analysis} It does not really seem to darken Marlow --- why is he
not affected? Oh! Right! Because\ldots{} MARLOW IS A BLOODY RACIST!

\begin{itemize}
\item Undereacts to black people dying next to him
\item Compared to to like\ldots{} Kurtz dying, or his Helmsmen dying --- lots
of ceremony and discussion
\end{itemize}

\item Marlow bloody racist => dosen't see some darkness
\end{itemize}

\textbf{CONFLIGHT!} Maybe Marlow's racist lens could be looked at from the
opposite way! Darkness is, actually, everywhere\ldots{}

\subsubsection{Jack --- the darkness was always there (b/c Europeans), but Marlow's}
\label{sec:org060f379}
perspective also is a contributor to it. 9-12
:CUSTOM\textsubscript{ID}: jack-the-darkness-was-always-there-bc-europeans-but-marlows-perspective-also-is-a-contributor-to-it.-9-12

\begin{itemize}
\item Darkness is \textbf{everywhere} --- even on the Thems (pg 5) --- but Marlow
could only percieve the darkness that he choose to percieve. Marlow
does not bring darkness, and Marlow does not see Darknss in Africa.
Darkness is everywhere and it is just what Marlow's perception
illustrates
\item Hell example 43 --- language of hell, except language actually just
show reality
\item pg 50 --- he chose to not see darkne
\item "The earth seemed unearthly. we are accustomed \ldots{} but here--- you
could look at a thing mostrous and free": indeed, Marlow admitting
that perception shaped his senses --- racist marlow labels africa as
moster, and chooses to see it here.
\item Then realized that, if there were to be an example, a place to show
the objective example of the darkness, we could get another
perspective potentially through convos he has with other people to
pick up on things that they have said
\end{itemize}

Saliant points for Jack:

\begin{itemize}
\item Thems: 5
\item Hell: 43/44
\item Cannibals: 49 bottom
\item 
\end{itemize}
\end{document}
