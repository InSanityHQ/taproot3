% Created 2021-09-27 Mon 11:52
% Intended LaTeX compiler: xelatex
\documentclass[letterpaper]{article}
\usepackage{graphicx}
\usepackage{grffile}
\usepackage{longtable}
\usepackage{wrapfig}
\usepackage{rotating}
\usepackage[normalem]{ulem}
\usepackage{amsmath}
\usepackage{textcomp}
\usepackage{amssymb}
\usepackage{capt-of}
\usepackage{hyperref}
\setlength{\parindent}{0pt}
\usepackage[margin=1in]{geometry}
\usepackage{fontspec}
\usepackage{svg}
\usepackage{cancel}
\usepackage{indentfirst}
\setmainfont[ItalicFont = LiberationSans-Italic, BoldFont = LiberationSans-Bold, BoldItalicFont = LiberationSans-BoldItalic]{LiberationSans}
\newfontfamily\NHLight[ItalicFont = LiberationSansNarrow-Italic, BoldFont       = LiberationSansNarrow-Bold, BoldItalicFont = LiberationSansNarrow-BoldItalic]{LiberationSansNarrow}
\newcommand\textrmlf[1]{{\NHLight#1}}
\newcommand\textitlf[1]{{\NHLight\itshape#1}}
\let\textbflf\textrm
\newcommand\textulf[1]{{\NHLight\bfseries#1}}
\newcommand\textuitlf[1]{{\NHLight\bfseries\itshape#1}}
\usepackage{fancyhdr}
\pagestyle{fancy}
\usepackage{titlesec}
\usepackage{titling}
\makeatletter
\lhead{\textbf{\@title}}
\makeatother
\rhead{\textrmlf{Compiled} \today}
\lfoot{\theauthor\ \textbullet \ \textbf{2021-2022}}
\cfoot{}
\rfoot{\textrmlf{Page} \thepage}
\renewcommand{\tableofcontents}{}
\titleformat{\section} {\Large} {\textrmlf{\thesection} {|}} {0.3em} {\textbf}
\titleformat{\subsection} {\large} {\textrmlf{\thesubsection} {|}} {0.2em} {\textbf}
\titleformat{\subsubsection} {\large} {\textrmlf{\thesubsubsection} {|}} {0.1em} {\textbf}
\setlength{\parskip}{0.45em}
\renewcommand\maketitle{}
\author{Taproot}
\date{\today}
\title{}
\hypersetup{
 pdfauthor={Taproot},
 pdftitle={},
 pdfkeywords={},
 pdfsubject={},
 pdfcreator={Emacs 28.0.50 (Org mode 9.4.4)}, 
 pdflang={English}}
\begin{document}

\tableofcontents

\section{Prompt}
\label{sec:org3b0903a}
\subsection{Assignment guidelines}
\label{sec:org58aa3f3}

After reading and analyzing Kincaid’s book, you have a better understanding of rhetorical purpose and techniques. Using Kincaid as inspiration, and using at least three of her techniques (parentheses, tone, anaphora, personal pronouns, labyrinthine sentence, em-dashes, etc.), write a rhetorical pastiche on one of the prompts below.

In addition to your pastiche, please provide a short reflection. You should explain your rhetorical purpose as well as the techniques and elements of style you are trying to imitate (this means you will need to reflect a bit on Kincaid’s techniques, so that you can explain what aspects of her writing you’ve been inspired by). Your reflection is a chance to explain what you want to accomplish, in case it doesn’t fully come through in your piece, as well as to demonstrate your understanding of rhetorical techniques.

\subsection{Prompts}
\label{sec:org5b24a3a}

\subsubsection{Conflicts in identity and place}
\label{sec:orgdc79c86}
In A Small Place Kincaid describes a conflicted relationship with identity, place, and the past, at times nostalgic, at times combative. Write about some aspect of your life—a part of your identity, a favorite place, your spoken/written languages, a relationship—that evokes contradiction and conflict for you. You may employ an implicit argument in your essay (like Kincaid’s), but you \textbf{must} have an explicit argument to articulate in your reflection.

\subsubsection{Responsible travel}
\label{sec:org27ac7de}
Kincaid begins her text with a mockery of “you,” an assumed pasty-faced (read: white), European tourist visiting Antigua. Reflect on your position as a sometimes-tourist, privileged at least to some extent. Possible intended audiences include American friends and family unaware of the complicated power dynamics of travelling in non-Western and/or post-colonial countries, or the “locals” of said country: Antiguans, Peruvians, Costa Ricans, or others.


\subsubsection{Solutions for issues based in poverty}
\label{sec:org6f4edd1}
Critiques of philanthropy often suggest it is motivated by a colonial mindset (“white savior complex,” “white volunteerism”). Yet the fact remains that poverty proliferates and many aid organizations work to alleviate illness, destitution, and inequity. Use the Poverty Action Lab site (\url{https://www.povertyactionlab.org/} (Links to an external site.)) to suggest a solution for some \textbf{manageable} issue based in poverty. Note about navigating the site: You may find it useful to start using either “Regions” or “Sectors”; “evaluations” and “publications” motivate issues and provide solutions (policy proposals).

\subsection{Evaluation and assessment}
\label{sec:orga7283cb}
\begin{itemize}
\item Understanding Literature: Form and Function
\item Close Reading and Argumentation
\item Structure and Mechanics
\item The Writer’s Voice
\item Responsibility
\end{itemize}


\subsection{Formatting and due dates}
\label{sec:org1c868f9}

Rhetorical essay: 600-800 words, double-spaced, include MLA-style header and title

Reflection: at least 150 words, no more than 250 words, double-spaced



Introduce Assignment: Week of 11/9

Prewriting/notes/ideas due: 11/11

Rough draft due for peer editing: 11/18

Final due (including reflection): 11/20

\section{Outline}
\label{sec:org80b3b08}
Uh, does that even apply? I want to use parenthises and I want to write about conflicting relationship

\section{Draft 0.1}
\label{sec:orgd51deb5}
  Who (
    but not really who (
        not one person anyway, why should you assume that you already know who it will be?
    ), I mean, what kind of person (
        or really, personality, because that is what matters right? What is a person but their personality, maybe their looks, their phisique but that varies with age (
            but then again what doesn't vary with age? Personality is ever chaging, at times slower yet but often faster than looks might.
            ), so then, I suppose, who now?
    ), or rather, what kind of interaction (
        it boils down to this, doesn't it? It doesn't matter who they really are alone, because in the end what you see of them is what you see with them.. so really who is a good fit?
    ) warrents such dramatic measures?
) should you let get close to you?

\section{Draft 0.2}
\label{sec:org44ecf33}

  Who (
      or maybe not who\ldots{} because it's not a single person. How do you know you already know who it might be? Or rather, what kind of person--or maybe I'm asking what kind of personality? (
        Because what is a person if not their personality? (
            Looks, you say, physique. But that changes all the time. This decision is a big one, and its not something that should change with a dab of makeup. Just as you wouldn't avoid buying a house if a loud plane flew overhead while you were visiting. Of course, you might take pause if loud planes often fly overhead, just as you might take pause if their appearance is often untidy. But personality is what we are here for right? Atleast, that's what they all say.
        ) So what kind of personality are you seeking? (
            It's like walking into a vast store, with the various personalities laid out on the shelves for all to see. Kindness, people say. Intelligence, ambition, loyalty, forgiveness. Some of the aisles have well polished stripes, where countless reflecting minds have picked the common traits off the shelf. But might you shy away from the crowds and visit the dark aisles at the rear, with its dusty boxes and subpar lighting? Who knows, perhaps there will be something golden.
            ) How can one know what it is they want? And how can one find this store of clairity? (
                Because everyone's looking for it, I can tell you that. When they're alone, or maybe with a trusted friend, they plan out which aisles they will visit, in what order, and what to do if something is on sale or suddenly expensive. But do they ever make it to that store, are they ever able to carry out their meticulous plan? No, not usually. Usually the don't make it at all, or they get dizzy in the revolving door at the entrance and spend their time inside in a daze. Humans tend to let their emotions overrun them at the most inconveinient times
            ). All this is assuming that you know what you want, and to be fair, most people think they know what they want. But are you sure that you know what you want? (
                So far, you've only been considering what would work for you at this moment. But with everyone changing all the time, it's like trying to find a boat floating the perfect distance away, and expecting them to stay at that distance all the time. Even if you put down anchor, the slack in the chain will let you drift.. it won't last.
            ) They say that you become who you are around, but those around you too are perpetually changing. Should you choose someone seems to click with you now, or someone who's current image is who you want to be? (
            Should you choose someone who feels right, or someone who you think will be? Generally people choose logic, but how can you logic your way to happiness or fulfillment?
How can you even know who someone is at this moment, not to mention who they will become? Who should you let get close to you?

\section{Final Draft I guess}
\label{sec:orgbbfbb8e}
\url{https://docs.google.com/document/d/1AAtndWZm\_lVWKSKGRPZeWBOn\_2pEhDIUGnOzJSNaess/edit}
\end{document}
