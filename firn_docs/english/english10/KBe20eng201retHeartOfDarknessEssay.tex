% Created 2021-09-27 Mon 12:02
% Intended LaTeX compiler: xelatex
\documentclass[letterpaper]{article}
\usepackage{graphicx}
\usepackage{grffile}
\usepackage{longtable}
\usepackage{wrapfig}
\usepackage{rotating}
\usepackage[normalem]{ulem}
\usepackage{amsmath}
\usepackage{textcomp}
\usepackage{amssymb}
\usepackage{capt-of}
\usepackage{hyperref}
\setlength{\parindent}{0pt}
\usepackage[margin=1in]{geometry}
\usepackage{fontspec}
\usepackage{svg}
\usepackage{cancel}
\usepackage{indentfirst}
\setmainfont[ItalicFont = LiberationSans-Italic, BoldFont = LiberationSans-Bold, BoldItalicFont = LiberationSans-BoldItalic]{LiberationSans}
\newfontfamily\NHLight[ItalicFont = LiberationSansNarrow-Italic, BoldFont       = LiberationSansNarrow-Bold, BoldItalicFont = LiberationSansNarrow-BoldItalic]{LiberationSansNarrow}
\newcommand\textrmlf[1]{{\NHLight#1}}
\newcommand\textitlf[1]{{\NHLight\itshape#1}}
\let\textbflf\textrm
\newcommand\textulf[1]{{\NHLight\bfseries#1}}
\newcommand\textuitlf[1]{{\NHLight\bfseries\itshape#1}}
\usepackage{fancyhdr}
\pagestyle{fancy}
\usepackage{titlesec}
\usepackage{titling}
\makeatletter
\lhead{\textbf{\@title}}
\makeatother
\rhead{\textrmlf{Compiled} \today}
\lfoot{\theauthor\ \textbullet \ \textbf{2021-2022}}
\cfoot{}
\rfoot{\textrmlf{Page} \thepage}
\renewcommand{\tableofcontents}{}
\titleformat{\section} {\Large} {\textrmlf{\thesection} {|}} {0.3em} {\textbf}
\titleformat{\subsection} {\large} {\textrmlf{\thesubsection} {|}} {0.2em} {\textbf}
\titleformat{\subsubsection} {\large} {\textrmlf{\thesubsubsection} {|}} {0.1em} {\textbf}
\setlength{\parskip}{0.45em}
\renewcommand\maketitle{}
\author{Exr0n}
\date{\today}
\title{Exr0n Heart of Darkness Essay Index}
\hypersetup{
 pdfauthor={Exr0n},
 pdftitle={Exr0n Heart of Darkness Essay Index},
 pdfkeywords={},
 pdfsubject={},
 pdfcreator={Emacs 28.0.50 (Org mode 9.4.4)}, 
 pdflang={English}}
\begin{document}

\tableofcontents

\#ret \#disorganized \#incomplete

\section{Prompt}
\label{sec:orgf768cea}
\begin{verbatim}
Heart of Darkness Analytical Essay

English 10: Landscapes of the Self and Other



For your first literary analysis paper, you will be coming up with your own interpretive argument about some aspect of Heart of Darkness.



OPTION 1: Choose a recurring word, motif, pattern, or character

Choose a word, motif, pattern, or character that you’ve noticed throughout the book, and construct an analytical, argumentative essay around it. For example, you might want to look at specific moments where you see “light” and “dark” imagery. Or, maybe you want to look at every time the word “wilderness” shows up. Perhaps you want to analyze the role of women in the book, or the way that Marlow writes about the jungle. Again, if you choose a theme we’ve discussed in class, you need to go above and beyond what we’ve talked about—you need to add something new to the conversation. For this option, I recommend looking at a repetition or a character that/who seems to change throughout the book. If you look at a changing use of the same language or at the arc of a character, it will be easier for you to construct a forward-moving argument.



OPTION 2: Choose a moment in the text

Pick an excerpt of no more than 1/3-1/2 a page from the book, and construct an analytical, argumentative essay around it. To do this well you need to pay close attention to language: word choice, sentence structure, tone, other literary/rhetorical techniques, etc. If you choose a passage we’ve already discussed in class, you need to go above and beyond what we’ve talked about—you need to add something new to the conversation. Our recommendation is to choose an excerpt we haven’t treated in detail together. 



You will likely make connections to other parts of the text, particularly as you engage broader implications in your argument. However, the focus of your piece should be on your excerpt.



OPTION 3: Propose your own analytical adventure

If neither option 1 or 2 appeals to you, please schedule time to meet with me during tutorial to discuss a topic of your choice. 




Basic requirements:

Your essay will need to be 2-3 pages in length.
Your essay should include, at least, 2 thoughtfully selected and analyzed direct citations per body paragraph. 
You should double-space your paper, use 1” margins, use 12-point font (preferably Times New Roman), and use MLA formatting (a four-line header, last name and page number, citations, etc.).
Include a works cited entry for Heart of Darkness at the end of your essay.


Template Items Assessed for this Paper

Understanding Literature: Form and Function
Close Reading and Argumentation
Structure and Mechanics
The Writer’s Voice


Due Dates:

Topic selection and thesis approval 10/2 


A rough draft 10/5 
at least, one page double-spaced will be due at the end of class. If you want feedback on a rough draft earlier, I recommend seeing me during tutorial or tutor in the WRC.


Peer edit 10/7 


The final draft 10/12 will be due at the beginning of class
\end{verbatim}

\section{Evidence}
\label{sec:orgc4e95fe}
\begin{itemize}
\item I want to analyze the language on the second half of page 92 > "I
thought his memory was like the other memories of the dead that
accumulate in every man's life---a vague impress on the brain of
shadows that had fallen on it in their swift and final passage; but
before the high and ponderous door, between the tall houses of a
street as still and decorous as a well-kept alley in a cemetery, I had
a vision of him on the stretcher, opening his mouth voraciously, as if
to devour all the earth with all its mankind. He lived then before me;
he lived as much as he had ever lived---a shadow insatiable of
splendid appearances, of frightful realities; a shadow darker than the
shadow of the night, and draped nobly in the folds of a gorgeous
eloquence. The vision seemed to enter the house with me---the
stretcher, the phantom-bearers, the wild crowd of obedient worshipers
the gloom of the forests, the glitter of the reach between the murky
bends, the beat of the drum, regular and muffled like the beating of a
heart---the heart of a conquering darkness. It was a moment of triumph
for the wilderness, an invading and vengeful rush which, it seemed to
me, I would have to keep back alone for the salvation of another soul.
And the memory of what I had heard him say afar there, with the horned
shapes stirring at my back, in the glow of fires, within the patient
woods, those broken phrases came back to me, were heard again in their
ominous and terrifying simplicity. I remembered his abject pleading,
his abject threats, the colossal scale of his vile desires, the
meanness, the torment, the tempestuous anguish of his soul. And later
on I seemed to see his collected languid manner, when he said one day,
'This lot of ivory now is really mine. The Company did not pay for it.
I collected it myself at a very great personal risk. I am afraid they
will try to claim it as theirs though. H'm. It is a difficult case.
What do you think I ought to do---resist? Eh? I want no more than
justice.' \ldots{} He wanted no more than justice---no more than justice. I
rang the bell before a mahogany door on the first floor, and while I
waited he seemed to stare at me out of the glassy panel---stare with
that wide and immense stare embracing condemning, loathing all the
universe. I seemed to hear the whispered cry, 'The horror! The
horror!"
\end{itemize}

\section{Analysis}
\label{sec:org0afb008}
\subsection{How does it make me feel?}
\label{sec:org2ce407a}
\begin{itemize}
\item The text feels dreamy, abstract, and almost hallucinatory
\item Kurtz is seen as immortalized, due to his extreme intentions and the
raw tangibility of the original experience
\item the cadence of writing draws the reader and brings them along for an
involuntary, exhilarating series of visions
\item the above feelins show that Kurtz and his intentions are immortalized
and impressed upon marlow, perhaps the raw tangibility shows the
primal, nearly supernatural impact of the wilderness in the heart of
darkness

\begin{itemize}
\item because what is "natural" for them is the western world
\end{itemize}
\end{itemize}

\subsection{Literary Devices}
\label{sec:org188a29b}
\begin{itemize}
\item Alliteration
\item Tail alliteration
\item Parallel structure
\item Juxtaposition / nonsensicality
\item Zooming out
\end{itemize}

I'm taking a while to figure out how I actually want to write this. I
should just bite the bullet and pick some examples to get going.

\section{Thesis}
\label{sec:org1ee1694}
In "Heart of Darkness", Conrad uses parallel structure of auditory
repetition and logical juxtaposition to create the feeling of an
invading torrent of overwhelming experience.

\section{Outline}
\label{sec:org4257a2b}
\subsection{Intro}
\label{sec:org21b0c89}
\begin{itemize}
\item summarize plot
\item thesis
\end{itemize}

\subsection{Body 1}
\label{sec:orge64c26b}
\begin{itemize}
\item Auditory repetition, particularly alliteration and consonance, creates
a rhythm of pronunciation that drags the reader along.
\item "the gloom of the forests, the glitter of the reach"

\begin{itemize}
\item the parallel use of the structural phrases "the" and "of the", as
well as the initial adjective consonant "gl" emphasizes the constant
stream of emotionally provocative adjectives.
\end{itemize}

\item "the wild crowd of obedient worshipers"

\begin{itemize}
\item the spate of hard and soft stressed consonants creates a rhythmic,
almost musical pattern of tension and release
\item In "wild" and "crowd", the \emph{w} phonic leads the reader to the hard
\emph{d} phonic at the end of the word.
\item In "obedient", a number of hard sounds are stressed (\emph{b}, \emph{d}, \emph{t})
which creates rhythmic tension to be released by the softer sounds
in "worshipers"
\end{itemize}
\end{itemize}

\subsection{Body 2}
\label{sec:org68c8e8a}
\begin{itemize}
\item The juxtaposition of opposite adjectives creates a sense of
paranormality and mystifies the imagery.
\item "the wild crowd of obedient worshipers"

\begin{itemize}
\item wild and obedient are normally opposites, but here they are used to
emphasize the abnormality of the scene--the usually obedient
worshipers have gone wild in.
\end{itemize}

\item "the glitter of the reach between the murky bends"

\begin{itemize}
\item the murky bends normally do not glitter, but the frothing surface
may reflect some light despite silt in the water.
\end{itemize}

\item Both phrases use contradictory adjectives create nuanced, sensical
imagery whose complexity creates a fractal-like depth in the imagery.
\end{itemize}

\subsection{Body 3}
\label{sec:org227350a}
\begin{itemize}
\item The imagery in the excerpt increments in scale and intensity, creating
a sense of uncontrollable expanding cognizance.
\item "the stretcher, the phantom-bearers, the \ldots{} worshipers, the \ldots{}
forests".

\begin{itemize}
\item Each image is engulfed by the previous
\end{itemize}

\item From "the worshipers" to "the gloom" to "regular and muffled like the
beating of a heart"

\begin{itemize}
\item Each image is allocated increasing specificity of adjective,
starting with no detail at all to some description to an entire
simile.
\end{itemize}
\end{itemize}

\subsection{Conclusion}
\label{sec:org1682f76}
\begin{itemize}
\item The use of evolving parallel structure ties this repetition together
and coheres
\end{itemize}

\section{Essay}
\label{sec:org2ad783c}
Joseph Conrad's "Heart of Darkness" tells Marlow's experience with a
voraciously driven Mr. Kurtz and his death in the pre-colonial Congo. As
Marlow is visiting the late Kurtz's fiance he has a flashback to a
moment in the wilderness: "the stretcher, the phantom-bearers, the wild
crowd of obedient worshipers the gloom of the forests, the glitter of
the reach between the murky bends, the beat of the drum, regular and
muffled like the beating of a heart--the heart of a conquering darkness"
(91). In this excerpt, Conrad uses a three pronged parallel structure of
auditory repetition, logical juxtaposition, and progression of imagery
to illustrate an invading sensory overload. Auditory repetition,
particularly alliteration and consonance, creates a rhythm of language
that drags the reader along. When describing the scene, Marlow intones
"the gloom of the forests, the glitter of the reach". The structural
correspondence of the clauses creates a rhythmic flow and emphasizes the
differing adjectives; the alliterative hard consonant beginnings of each
word also add to the rhythmic feel. This repetitive structure imagines a
relentless march of stimulus flooding Marlow's senses. Marlow also notes
"the wild crowd of obedient worshipers", a lyric whose spate of hard and
soft consonants creates a rhythmic, nearly musical pattern of tension
and release. In "wild" and "crowd", the \emph{w} phonic leads the reader
along to a hard \emph{d} phonic at the end of the word--setting a pace for
the rest of the clause. Then, in "obedient", a string of hard consonants
makes the pronunciation feel fast paced and torrential. The softer
consonants in "worshipers" and the hard \emph{p} near the end creates a quick
trill to end the thought. The staccato repetition of the \emph{d} phonic also
creates a sense of urgency, furthering the avalanche of imagery. As
Marlow speaks in his trance-like state, Conrad uses the juxtaposition of
opposite adjectives to create a sense of paranormality and mystify the
imagery. For example, Marlow mentions "the wild crowd of obedient
worshipers" who are normally obedient but have become rowdy following
Kurtz's departure. Here, the use of contradictory adjectives highlights
the distinction between the mindset and actions of the individual native
and the collected tribe. Similarly, Marlow also speaks of "the glitter
of the reach between the murky bends", whose muddy water can still
reflect the dwindling sunlight at the right angle. This description
separates the beautiful, rippled surface of the river from the imprecise
depths below. Both use contradictory adjectives to create nuance.
Although the their adjectives seem paradoxical at first glance, the
imagery is subtle and sensical. This nuance increases the depth of
imagery and makes the scene more visceral for the reader; the
compactness of the complexity weights each clause and increases the
density of the sensory tsunami. Finally, instead of following a rigid
parallel structure, the imagery in the excerpt increments in scale and
intensity--increasing the sense uncontrollable expanding cognizance.
Marlow focuses first on stretcher, then the bearers, then the forest and
the river, and finally the oppressive drums and inner continent itself.
As each piece of imagery engulfs the previous, the reader imagines the
situation in a broader and broader context--as if through the viewfinder
of a camera, zooming out into space. It is as if Marlow is bringing us
along for a non-lucid out of body experience. Not only does the object
of focus become physically larger, but the nuance of description also
increases. At first, Kurtz's stretcher is simply a stretcher--but as the
vision goes on the forest is "gloomy" and the water glitters in a
specific reach between the bends. This culminates in the description of
the beating of a drum: "regular and muffled like the beating of a
heart--the heart of a conquering darkness". Not only does Conrad use an
entire simile and an em-dash, but he also ties it back to the title and
thus the entire progression of the story. This zoom-out reveals the
structure of the plot--the lay of the land, so to speak. This vision is
a defining moment of Marlow's life flashing before his eyes, the epitome
of sensory overload. Each aspect of Marlow's vision conveys a feeling of
being overwhelmed. The intensity of the vision and it's freshness in his
memory creates a sensory overload conveyed to the reader through
phonetics, language, and content.

\section{Editing}
\label{sec:org71fa571}
\begin{itemize}
\item No 1st or 2nd person
\item No colloquial/slang
\item Precise language
\item No "big, good, bad, thing, get, interesting, really"
\end{itemize}

\section{Feedback round 1}
\label{sec:org32f7399}
\begin{itemize}
\item Intro could be longer (because so much of it is a quote)
\item Thesis

\begin{itemize}
\item missing a so what
\item what does the intensity of the congo show?
\item connect to the meaning and intentions behind the novel
\end{itemize}
\end{itemize}

\section{Export}
\label{sec:org29eceab}
\url{https://docs.google.com/document/d/1j8QsHTNkcqGl8mAIR5ddp1CdKODspltTYg2Eaf-H0MA/edit}

\noindent\rule{\textwidth}{0.5pt}
\end{document}
