% Created 2021-09-27 Mon 12:02
% Intended LaTeX compiler: xelatex
\documentclass[letterpaper]{article}
\usepackage{graphicx}
\usepackage{grffile}
\usepackage{longtable}
\usepackage{wrapfig}
\usepackage{rotating}
\usepackage[normalem]{ulem}
\usepackage{amsmath}
\usepackage{textcomp}
\usepackage{amssymb}
\usepackage{capt-of}
\usepackage{hyperref}
\setlength{\parindent}{0pt}
\usepackage[margin=1in]{geometry}
\usepackage{fontspec}
\usepackage{svg}
\usepackage{cancel}
\usepackage{indentfirst}
\setmainfont[ItalicFont = LiberationSans-Italic, BoldFont = LiberationSans-Bold, BoldItalicFont = LiberationSans-BoldItalic]{LiberationSans}
\newfontfamily\NHLight[ItalicFont = LiberationSansNarrow-Italic, BoldFont       = LiberationSansNarrow-Bold, BoldItalicFont = LiberationSansNarrow-BoldItalic]{LiberationSansNarrow}
\newcommand\textrmlf[1]{{\NHLight#1}}
\newcommand\textitlf[1]{{\NHLight\itshape#1}}
\let\textbflf\textrm
\newcommand\textulf[1]{{\NHLight\bfseries#1}}
\newcommand\textuitlf[1]{{\NHLight\bfseries\itshape#1}}
\usepackage{fancyhdr}
\pagestyle{fancy}
\usepackage{titlesec}
\usepackage{titling}
\makeatletter
\lhead{\textbf{\@title}}
\makeatother
\rhead{\textrmlf{Compiled} \today}
\lfoot{\theauthor\ \textbullet \ \textbf{2021-2022}}
\cfoot{}
\rfoot{\textrmlf{Page} \thepage}
\renewcommand{\tableofcontents}{}
\titleformat{\section} {\Large} {\textrmlf{\thesection} {|}} {0.3em} {\textbf}
\titleformat{\subsection} {\large} {\textrmlf{\thesubsection} {|}} {0.2em} {\textbf}
\titleformat{\subsubsection} {\large} {\textrmlf{\thesubsubsection} {|}} {0.1em} {\textbf}
\setlength{\parskip}{0.45em}
\renewcommand\maketitle{}
\author{Houjun Liu}
\date{\today}
\title{Close Reading!}
\hypersetup{
 pdfauthor={Houjun Liu},
 pdftitle={Close Reading!},
 pdfkeywords={},
 pdfsubject={},
 pdfcreator={Emacs 28.0.50 (Org mode 9.4.4)}, 
 pdflang={English}}
\begin{document}

\tableofcontents



\section{Close Reading}
\label{sec:orgc57b044}
\definition{Close Reading}{Unpacking the meaning of individual words and phrases in a text.}
\subsection{Goals}
\label{sec:org7d53227}
\begin{itemize}
\item Unpacking the meaning of individual words and phrases in a text
\item Consider effects of author's writers

\begin{itemize}
\item W.r.t. the reader's experience
\item W.r.t. the authors too
\end{itemize}

\item Connecting a part => Whole

\begin{itemize}
\item Looking for connections
\item Look for implications
\item Look for how a text changes in its contexts "Marian's text w.r.t.
Beigin nationalism"
\end{itemize}

\item Also, a creative effort!
\end{itemize}

\subsection{Framework}
\label{sec:orgeee0806}
\#important One way of close reading is to use the following framework:

\definition[Where X is a literary technique, and Y a theme or a broader point]{Close Reading Framework}{The Author uses X to do Y}
When annotating, look for\ldots{}

\begin{itemize}
\item Repetition
\item Irony
\item Metaphores
\end{itemize}

etc. But, remember that \emph{this is only the X.} It is not enough to find a
device, but you need to analyze \emph{why} its there.

\subsection{The assessment}
\label{sec:org598f384}
The First Close Reading Assessment!

\href{KBhENG201CloseReadingParagraph.org}{KBhENG201CloseReadingParagraph}
\end{document}
