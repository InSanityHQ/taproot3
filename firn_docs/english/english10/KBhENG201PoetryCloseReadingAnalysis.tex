% Created 2021-09-27 Mon 12:02
% Intended LaTeX compiler: xelatex
\documentclass[letterpaper]{article}
\usepackage{graphicx}
\usepackage{grffile}
\usepackage{longtable}
\usepackage{wrapfig}
\usepackage{rotating}
\usepackage[normalem]{ulem}
\usepackage{amsmath}
\usepackage{textcomp}
\usepackage{amssymb}
\usepackage{capt-of}
\usepackage{hyperref}
\setlength{\parindent}{0pt}
\usepackage[margin=1in]{geometry}
\usepackage{fontspec}
\usepackage{svg}
\usepackage{cancel}
\usepackage{indentfirst}
\setmainfont[ItalicFont = LiberationSans-Italic, BoldFont = LiberationSans-Bold, BoldItalicFont = LiberationSans-BoldItalic]{LiberationSans}
\newfontfamily\NHLight[ItalicFont = LiberationSansNarrow-Italic, BoldFont       = LiberationSansNarrow-Bold, BoldItalicFont = LiberationSansNarrow-BoldItalic]{LiberationSansNarrow}
\newcommand\textrmlf[1]{{\NHLight#1}}
\newcommand\textitlf[1]{{\NHLight\itshape#1}}
\let\textbflf\textrm
\newcommand\textulf[1]{{\NHLight\bfseries#1}}
\newcommand\textuitlf[1]{{\NHLight\bfseries\itshape#1}}
\usepackage{fancyhdr}
\pagestyle{fancy}
\usepackage{titlesec}
\usepackage{titling}
\makeatletter
\lhead{\textbf{\@title}}
\makeatother
\rhead{\textrmlf{Compiled} \today}
\lfoot{\theauthor\ \textbullet \ \textbf{2021-2022}}
\cfoot{}
\rfoot{\textrmlf{Page} \thepage}
\renewcommand{\tableofcontents}{}
\titleformat{\section} {\Large} {\textrmlf{\thesection} {|}} {0.3em} {\textbf}
\titleformat{\subsection} {\large} {\textrmlf{\thesubsection} {|}} {0.2em} {\textbf}
\titleformat{\subsubsection} {\large} {\textrmlf{\thesubsubsection} {|}} {0.1em} {\textbf}
\setlength{\parskip}{0.45em}
\renewcommand\maketitle{}
\author{Houjun Liu}
\date{\today}
\title{Poetry Close Reading}
\hypersetup{
 pdfauthor={Houjun Liu},
 pdftitle={Poetry Close Reading},
 pdfkeywords={},
 pdfsubject={},
 pdfcreator={Emacs 28.0.50 (Org mode 9.4.4)}, 
 pdflang={English}}
\begin{document}

\tableofcontents



\section{Poetry Close Reading}
\label{sec:org1aea35b}
\section{Essay Template}
\label{sec:org681bffb}
\subsection{General Information}
\label{sec:orgf5244ca}
\begin{center}
\begin{tabular}{lll}
Due Date & Topic & Important Documents\\
\hline
Mar 22nd, 6PM & Poetry Close Reading & Poems\\
\end{tabular}
\end{center}

\subsection{Prompt}
\label{sec:org560d9ff}
A close reading analysis about how the poems we read, discussed and
analyzed in class inspired your original work.

\subsection{Claim Synthesis}
\label{sec:org5ab2b5a}
\subsubsection{Development phase -- How and So-What}
\label{sec:orgd7f5e90}
Effect of enjambment: "Hot // lies, rot-//ting hulls" => provides a sign
of resignation.

Rhythmic nature of \emph{Calypso} => sense of systemicism w.r.t. colonialism
\begin{itemize}
\item enviromental destruction
\end{itemize}

Us them dynamic => "Our hours lifted their lacy black veils--- // a
procession of grieving women."

\subsection{Defluffifying}
\label{sec:org6b9b71b}
CHOSEN THESIS CLAIM

\begin{itemize}
\item Point a
\item Point b
\item Point c
\end{itemize}

So what? SO WHAT

Now, defluffify by re-writing the three points + so what in as little
words as possible.

\textbf{RESTATED CLAIM}

\noindent\rule{\textwidth}{0.5pt}

There is always
\href{https://wp.ucla.edu/wp-content/uploads/2016/01/UWC\_handouts\_What-How-So-What-Thesis-revised-5-4-15-RZ.pdf}{UCLA
Writing Lab}
\end{document}
