% Created 2021-09-27 Mon 12:02
% Intended LaTeX compiler: xelatex
\documentclass[letterpaper]{article}
\usepackage{graphicx}
\usepackage{grffile}
\usepackage{longtable}
\usepackage{wrapfig}
\usepackage{rotating}
\usepackage[normalem]{ulem}
\usepackage{amsmath}
\usepackage{textcomp}
\usepackage{amssymb}
\usepackage{capt-of}
\usepackage{hyperref}
\setlength{\parindent}{0pt}
\usepackage[margin=1in]{geometry}
\usepackage{fontspec}
\usepackage{svg}
\usepackage{cancel}
\usepackage{indentfirst}
\setmainfont[ItalicFont = LiberationSans-Italic, BoldFont = LiberationSans-Bold, BoldItalicFont = LiberationSans-BoldItalic]{LiberationSans}
\newfontfamily\NHLight[ItalicFont = LiberationSansNarrow-Italic, BoldFont       = LiberationSansNarrow-Bold, BoldItalicFont = LiberationSansNarrow-BoldItalic]{LiberationSansNarrow}
\newcommand\textrmlf[1]{{\NHLight#1}}
\newcommand\textitlf[1]{{\NHLight\itshape#1}}
\let\textbflf\textrm
\newcommand\textulf[1]{{\NHLight\bfseries#1}}
\newcommand\textuitlf[1]{{\NHLight\bfseries\itshape#1}}
\usepackage{fancyhdr}
\pagestyle{fancy}
\usepackage{titlesec}
\usepackage{titling}
\makeatletter
\lhead{\textbf{\@title}}
\makeatother
\rhead{\textrmlf{Compiled} \today}
\lfoot{\theauthor\ \textbullet \ \textbf{2021-2022}}
\cfoot{}
\rfoot{\textrmlf{Page} \thepage}
\renewcommand{\tableofcontents}{}
\titleformat{\section} {\Large} {\textrmlf{\thesection} {|}} {0.3em} {\textbf}
\titleformat{\subsection} {\large} {\textrmlf{\thesubsection} {|}} {0.2em} {\textbf}
\titleformat{\subsubsection} {\large} {\textrmlf{\thesubsubsection} {|}} {0.1em} {\textbf}
\setlength{\parskip}{0.45em}
\renewcommand\maketitle{}
\author{Huxley}
\date{\today}
\title{GoST Culmination Assignment Planning}
\hypersetup{
 pdfauthor={Huxley},
 pdftitle={GoST Culmination Assignment Planning},
 pdfkeywords={},
 pdfsubject={},
 pdfcreator={Emacs 28.0.50 (Org mode 9.4.4)}, 
 pdflang={English}}
\begin{document}

\tableofcontents

\#flo \#ret \#disorganized \#incomplete

\noindent\rule{\textwidth}{0.5pt}

\begin{verbatim}
Instructions: 

For this assignment, you will be producing a project that shows your understanding and close reading of a thematic aspect of Roy’s _The God of Small Things_. As always, close reading can involve tracking a repetition of some type in the text, including but not limited to repetitions in diction, structure, syntax, or characters’ behaviors. You might also think about tracking a change of some kind in the novel. This project is focused on your close reading skills, but you will likely be applying them in a way that is different from a traditional analytical essay. For this project, “Writer’s Voice” will be assessed and will be loosely defined as either a) creative writing or b) the way you use visual elements to connect to your audience.

 Using any medium (suggestions below), create a piece of art or an artistic representation that captures one of the central themes of the novel. You are being given two weeks to complete this project, so the end result should reflect this extended period of time. This is not a project that you can pull off the night before it is due and receive an exemplary assessment. Plan accordingly. Some ideas: 
-   Make a TV news story on an aspect of the book with which you also hand in a script.
-   Create a copy of the Ayemenem local newspaper for a) a notable day from the story or b) a mundane day-in-the-life of Ayemenem’s inhabitants.
-   Prepare a funeral for a character who dies that (perhaps) doubles as a social justice call to action. These funeral logistics should include a written eulogy component.
-   Create a video in which you act out/interpret a scene in _The God of Small Things_ (this option lends itself to collaboration with peers).
-   Make a day-in-the-life of Rahel film.
-   Create a Spotify playlist as a soundtrack for the text and include explanations (liner notes) for songs that incorporate close reading.
-   Create a visual art option with written component that includes close reading.
-   Write a pastiche of _The God of Small Things_ and include written explanation/analysis.
-   Write a long poem out of fragments of the book (this is a kind of pastiche)--describe how you used pieces of the book to illustrate the important themes and how those influence the novel.
-   You may also suggest your own creative project
-   You may also write a traditional close reading or literary essay.

With the creative options above, you must include an expository writing piece of two pages (double-spaced) that includes close readings of textual evidence from the book. You may think of this exposition in two pieces: 1) deeper engagement with textual evidence and a thematic scope from _The God of Small Things_, and 2) connecting your creative work to this thematic scope. In short, show us how your creative work engages with a theme in the novel_._ How does your creative work derive from and represent your own (close) reading of the novel? While you are allowed to collaborate on any of the above projects (with prior approval from your teacher), each student will be required to turn in their own unique reflection.

**Timeline:**

By 2/11: Receive project prompt

February Break: No homework, but you are welcomed to start brainstorming project ideas

Week of 2/22: In-class work periods for project

Week of 3/1: Project due

**Goals**:

It is your job to make sure you demonstrate the following template items, no matter the

format you choose for the assignment:
[](https://archive.org/stream/in.ernet.dli.2015.201707/2015.201707.The-God_djvu.txt)
1.  Understanding Literature
2.  Close Reading
3.  Structure & Mechanics
4.  Writer’s Voice
5.  Responsibility
\end{verbatim}

close reading on wrist watch? model wrist watch in blender?

\section{apearence:}
\label{sec:orge8a9cdb}
cheap plastic, ten to two. under grass buried in the ground.

\section{qoute bin:}
\label{sec:org848471d}
"The Waiting filled Rahel until she was ready to burst. She looked at
her watch. It was ten to two. She thought of Julie Andrews and
Christopher Plummer kissing each other sideways so that their noses
didn't collide. She wondered whether people always kissed each other
sideways. She tried to think of whom to ask."

"Rahel's toy wristwatch had the time painted on it. Ten to two. One of
her ambitions was to own a watch on which she could change the time
whenever she wanted to (which according to her was what Time was meant
for in the first place)."

"From the way Ammu held her head, Rahel could tell that she was still
angry. Rahel looked at her watch. Ten to two. Still no train. She put
her chin on the window sill. She could feel the grey gristle of the felt
that cushioned the window glass pressing into her chinskin. She took off
her sunglasses to get a better look at the dead frog squashed on the
road. It was so dead and squashed so flat that it looked more like a
frog-shaped stain on the road than a frog. Rahel wondered if Miss Mitten
had been squashed into a Miss Mitten-shaped stain by the milk truck that
killed her."

"Rahel liked all this. Holding the handbag. Everyone pissing in front of
everyone. Like friends. She knew nothing then, of how precious a feeling
this was. Like Jriends. They would never be together like this again.
Ammu, Baby Kochamma and she. When Baby Kochamma finished, Rahel looked
at her watch. 'So long you took, Baby Kochamma,' she said. 'It's ten to
two.'"

“The back verandah of the History House (where a posse of Touchable
policemen converged, where an inflatable goose was burst) had been
enclosed and converted into the air>\^{} hotel kitchen. Nothing worse than
kebabs and caramel custard hap- pened there now. The Terror was past.
Overcome by the smell of food. Silenced by the humming of cooks. The
cheerful chop- chop-chopping of ginger and garlic. The disembowelling of
lesser mammals \textasciitilde{} pigs, goats. The dicing of meat. The scaling of fish.

Something lay buried in the ground. Under grass. Under twenty- three
years of June rain.

A small forgotten thing.

Nothing that the world would miss.

A child's plastic wristwatch with the time painted on it.

Ten to two it said.”

“The Q\textsuperscript{ntas} koala they took for their children.

And the pens and socks Police children with multi-coloured toes

They burst the goose with a cigarette Bang. And buncd the rubber scraps

Yooseless goose Too recognizable

The gasses one of them wore. The others laughed so he kept them on for a
while. The watch they all forgot It stayed behind in the History House.
In the back verandah A faulty record of the time. Ten to two.

They left

Six princes, their pockets stuffed with toys

A pair of two-egg twins.

And the God of Loss.

He couldn't walk. So they dragged him

Nobody saw them.

Bats, of course, are blind”

"The Past Isn't Dead. It Isn't Even Past" - william falkner

\section{analysis:}
\label{sec:orge2603dd}
about frozen time?

twins are stuck in time?

book plays with time.

rahel talks about being able to change the time whenever she wants, roy
does just that

time is representation

time, like frog, is dead? does not move, according to watch.

the book is frozen in time, in the sense that it doesnt progress
centered around where the watch it left

frozen time -> oxymoron

people are frozen, stuck in the past.

passage of time means nothing?

two interpretations

time does not pass, and past stays with us.

the past vs past, time vs Time

grass has grown, time has passed.

ten too two, could be ten two two when time is frozen, it becomes
meaningless.

she only mentions it in the point when she is waiting! what does that
mean

waiting is the only place where we are aware of time?

second definition, aware of time, but still unable to move past.

\section{here we are, outlining.}
\label{sec:org5f82951}
rahels watch has drawn on hands.

time is frozen.

frozen can be two ways:

unable to be changed, and unable to escape the past.

unable to be changed. represented by the unchanging discarded watch.

caste system, not escaped even in communist ranks.

baby kochamma and her love

always leaking pickle jars

"the past insnt dead. it isnt even past." faulker

represented by the grass growing.

the past traps us. estha cannot move on. everyones lives are forever
changed.

the watch is only mentioned when rahel is waiting, which is when we are
aware of time. time is still passing, but the past is never gone.

it's at the history house, history all about the pasts effect on the
future.

\section{batman begins.}
\label{sec:org3da52c6}
Link!

For my creative project, I chose to create a rendition of Rahel's watch.
I made my model as close as possible to how Rahel's watch is described
in the novel, from the toy-like shoddy manufacturing to the painted on
hands. These hands are placed at ten to two, a phrased frequently
repeated throughout the entire novel. Of course, a watch with painted on
hands defeats the very purpose of a watch; after all, how can it tell
the time if it never changes? Rahel wants to "own a watch on which she
could change the time whenever she wanted to (which according to her was
what Time was meant for in the first place)." (citation). Arundhati Roy,
the author of the book and the constructor of its world, does just this.
The story is told non-linearly; the laws of Time are changed and broken
constantly, whenever Roy wants to. The passage of time becomes almost
meaningless, as the story is not about chronology. The phrase repeated,
"it was ten to two,"(citation) is not ten before two, ten till two, or
even 1:50. Instead, it is deliberately chosen to be "ten to two," which
could easily be misinterpreted as ten two two. But it doesn't matter,
and that's the point. Frozen time is practically an oxymoron; time has
become meaningless. Furthermore, the book reads "it was," asserting that
the watch is correct about the time, which of course, it isn't. Time has
become frozen, and thus, in the classical sense of the word, becomes
meaningless, arbitrary, and changeable. But this begs the question, what
does frozen time mean?

Frozen time can take primarily two meanings, both of which appear in the
book. The first meaning is that time simply does not pass, or in
practice, does not create change. I represented this with the unchanging
watch in the center of the frame, where even "under twenty-three years
of June rain," Rahel's "plastic wristwatch with the time painted on it
[\ldots{}] stays buried. [\ldots{}] Ten to two it said." (citation). The passage
of time fails to change the positions of the hands, just as it fails to
change much else in the book. Immutable or unchanging things are, in a
sense, frozen in time. This theme of immutability appears often
throughout the book, perhaps most apparently with the caste system.
Despite Velutha's infinite kindness, he is still beaten to death. Even
the communist party, described as "a cocktail revolution," is prejudiced
against low class members. (citation) Even when the ideology directly
contradicts it, the caste system is still unchanged; it is frozen in
time. This theme appears again with Baby Kochamma's love for Father
Mulligan, when even "Father Mulligan's death did not alter the text of
the entries in Baby Kochamma's diary," all declaring her love
(citation). The theme even sneaks into smaller instances, such as
Mammachi's inability to stop her pickle jars from leaking leaves her
wondering "whether she would ever master the art of perfect
preservation" for her entire life (citation).

The second way time can be frozen is best demonstrated by a quote from
William Faulkner: "The past is never dead. It's not even past." The past
stays with us in the sense that we can never escape its effects. I chose
to represent this with the grass growing. The world is not frozen, time
still passes, and yet, the watch remains; untouched, trapped. This sense
of frozen time also appears throughout the book. The phrase "ten to two"
only ever appears when Rahel is waiting. The very first use of the
phrase appears while in the train station: "The Waiting filled Rahel
until she was ready to burst. She looked at her watch. It was ten to
two." (citation). When one is waiting, they are acutely aware of the
time. Time is passing, and yet it remains ten to two. The one time the
phrase appears when Rahel is not waiting is when the watch is left at
the History House. History itself is about the past and its impacts,
ones we can never escape. Estha becomes mute because of his trauma,
unable to escape the past. Chacko cannot escape his marriage, still
trapped in a unofficial one sided relationship. Margaret Kochamma cannot
escape Sophie Mol's death. After all, things can change forever in a
day.

Once the quietness arrived, it stayed and spread in

Estha.

Wether it's waiting for a train,
\end{document}
