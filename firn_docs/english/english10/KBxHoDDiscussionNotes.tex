% Created 2021-09-27 Mon 12:02
% Intended LaTeX compiler: xelatex
\documentclass[letterpaper]{article}
\usepackage{graphicx}
\usepackage{grffile}
\usepackage{longtable}
\usepackage{wrapfig}
\usepackage{rotating}
\usepackage[normalem]{ulem}
\usepackage{amsmath}
\usepackage{textcomp}
\usepackage{amssymb}
\usepackage{capt-of}
\usepackage{hyperref}
\setlength{\parindent}{0pt}
\usepackage[margin=1in]{geometry}
\usepackage{fontspec}
\usepackage{svg}
\usepackage{cancel}
\usepackage{indentfirst}
\setmainfont[ItalicFont = LiberationSans-Italic, BoldFont = LiberationSans-Bold, BoldItalicFont = LiberationSans-BoldItalic]{LiberationSans}
\newfontfamily\NHLight[ItalicFont = LiberationSansNarrow-Italic, BoldFont       = LiberationSansNarrow-Bold, BoldItalicFont = LiberationSansNarrow-BoldItalic]{LiberationSansNarrow}
\newcommand\textrmlf[1]{{\NHLight#1}}
\newcommand\textitlf[1]{{\NHLight\itshape#1}}
\let\textbflf\textrm
\newcommand\textulf[1]{{\NHLight\bfseries#1}}
\newcommand\textuitlf[1]{{\NHLight\bfseries\itshape#1}}
\usepackage{fancyhdr}
\pagestyle{fancy}
\usepackage{titlesec}
\usepackage{titling}
\makeatletter
\lhead{\textbf{\@title}}
\makeatother
\rhead{\textrmlf{Compiled} \today}
\lfoot{\theauthor\ \textbullet \ \textbf{2021-2022}}
\cfoot{}
\rfoot{\textrmlf{Page} \thepage}
\renewcommand{\tableofcontents}{}
\titleformat{\section} {\Large} {\textrmlf{\thesection} {|}} {0.3em} {\textbf}
\titleformat{\subsection} {\large} {\textrmlf{\thesubsection} {|}} {0.2em} {\textbf}
\titleformat{\subsubsection} {\large} {\textrmlf{\thesubsubsection} {|}} {0.1em} {\textbf}
\setlength{\parskip}{0.45em}
\renewcommand\maketitle{}
\author{Huxley}
\date{\today}
\title{}
\hypersetup{
 pdfauthor={Huxley},
 pdftitle={},
 pdfkeywords={},
 pdfsubject={},
 pdfcreator={Emacs 28.0.50 (Org mode 9.4.4)}, 
 pdflang={English}}
\begin{document}

\tableofcontents

\noindent\rule{\textwidth}{0.5pt}

\#flo

\section{One}
\label{sec:orgfeb51b8}
\begin{quote}
How does Conrad relate /define darkness and insanity throughout the
novel? 
Why does Conrad build up to Kurtz so gradually, what
perception of Kurtz does Conrad want to portray?
\end{quote}

\begin{itemize}
\item Nixie's conclustion

\begin{itemize}
\item Madnesss is being removed from a place you are used to

\begin{itemize}
\item Does she mean that it causes madness?
\item says that heads are the main symbol of madness
\item 
\end{itemize}
\end{itemize}

\item Jackson says

\begin{itemize}
\item That the meeting of kurtz is ironic
\item Builds up kurtzs, but when they meet him, it's anticlimatic
\end{itemize}

\item Daniel responds

\begin{itemize}
\item Talks about marlows changing perception of kurts
\item Connects back to insanity

\begin{itemize}
\item nice one
\end{itemize}
\end{itemize}

\item Madeline rebuts

\begin{itemize}
\item kurts was not as expected
\item but still liked kurts at the end
\end{itemize}

\item Nixie says

\begin{itemize}
\item About kurtz greed with ivory?

\begin{itemize}
\item not super related
\end{itemize}
\end{itemize}

\item Jackson responds

\begin{itemize}
\item breaks out of dillsution, "the horror! the horror!"
\end{itemize}

\item Nixie interjects

\begin{itemize}
\item Isolating causes kurtz madness

\begin{itemize}
\item relates back to her original point
\end{itemize}

\item ties into subcoincuiss

\begin{itemize}
\item Restates original point
\item hmmmmm
\end{itemize}
\end{itemize}

\item Jackson

\begin{itemize}
\item There is nothing at the end
\end{itemize}

\item Madeline

\begin{itemize}
\item About insanity:

\begin{itemize}
\item conrad uses drums to represent heart beat to make the jungle seem
dangerous
\item \ldots{}.
\end{itemize}
\end{itemize}

\item Daniel

\begin{itemize}
\item As you travel the heart of darkness, you travel into kurtz mind

\begin{itemize}
\item is this true? you dont know kurtz mind till near the end
\end{itemize}
\end{itemize}

\item Nixes asks

\begin{itemize}
\item How does conrad define madness?
\end{itemize}

\item Daniel adds

\begin{itemize}
\item Mostly about kurtz,
\item caused by some sort of 'darkness'
\end{itemize}
\end{itemize}

\section{Two}
\label{sec:org1dfbdb1}
\begin{quote}
How does Conrad relate /define darkness and insanity throughout the
novel? 
Why does Conrad build up to Kurtz so gradually, what
perception of Kurtz does Conrad want to portray?
\end{quote}

\begin{itemize}
\item Riley explains some context
\item Aime describes

\begin{itemize}
\item dehumanization
\item objecitifes the woman

\begin{itemize}
\item beucase shes beutiful physically, marlow has more respect
\end{itemize}

\item the intended was viewed as emotional and docile
\end{itemize}

\item Rowan adds

\begin{itemize}
\item Beuty is inherently primal

\begin{itemize}
\item stems from necessity

\begin{itemize}
\item this is an intresting point. fits with the "Europe no better"
thesis.
\end{itemize}
\end{itemize}
\end{itemize}

\item anoushka

\begin{itemize}
\item we dont see her often
\item justaposition off\ldots{}?

\begin{itemize}
\item hard to hear
\end{itemize}

\item strange descriptions of beuty

\begin{itemize}
\item ah, the justaposition of beuty and savage?
\end{itemize}
\end{itemize}

\item riley

\begin{itemize}
\item kurzt is like a cult leader

\item described similary in some fashions to the natives

\item sedces his follower, buts gets seduced himself by the heart of
darkness

\item \begin{quote}
takes priority over rational thinking
\end{quote}
\end{itemize}

\item aime says

\begin{itemize}
\item that marlow is emotinally detached until the end

\begin{itemize}
\item even when pointing our the horrid things that are happening
\end{itemize}
\end{itemize}

\item rowan responds

\begin{itemize}
\item describes violence to "his people" in much more detail

\begin{itemize}
\item huge contrast in the descriptions

\begin{itemize}
\item this is proof of racism(ish)! says rowan

\begin{itemize}
\item framed narritive means that this is weird
\end{itemize}
\end{itemize}
\end{itemize}
\end{itemize}
\end{itemize}

\section{Four (I know how to count trust me)}
\label{sec:org041e884}
\begin{quote}


\begin{enumerate}
\item Is the racism in Heart of Darkness a parody of European racism
during colonial times or is it Conrad's inherent racism?




\item How did the different people who told Marlow about Kurtz alter his
opinion on him, how much did their opinions influence Marlow and
why did they hold more/less value?


\item Calder starts
\item is conrad inherently racist
\item why do we keep talking about this? this is not related! (not what
calder said)
\end{enumerate}
\end{quote}

\begin{itemize}
\item Nico says

\begin{itemize}
\item Conrad is truly racist

\begin{itemize}
\item becuase black people speak once or twice and are dehumanized
\item its not satire
\end{itemize}
\end{itemize}

\item Calder respondes

\begin{itemize}
\item bassicly says it can be a depiction distinc from conrads actual
beliefs

\begin{itemize}
\item hard to distinguis

\begin{itemize}
\item this is a good point. thx calder
\end{itemize}
\end{itemize}
\end{itemize}

\item Isabella

\begin{itemize}
\item uses the n-word, therefore, is racist!
\end{itemize}

\item kayla

\begin{itemize}
\item pattern of the n-word, therefore, nword isnt parody, therefore, is
racist!
\item an author cannot have a racist protagonist withoutgh being racist

\begin{itemize}
\item uh
\end{itemize}
\end{itemize}

\item Nico says

\begin{itemize}
\item that marlow isnt protrayed as having any flaws, and therfore, conrad
is racist!
\end{itemize}

\item isabella responds

\begin{itemize}
\item actully marlow is painted with flaws
\end{itemize}

\item kayla adds

\begin{itemize}
\item conrad is writing about his experience through marlow

\begin{itemize}
\item hmmmm
\end{itemize}
\end{itemize}

\item nico says

\begin{itemize}
\item that conrad says that europeans have been dragged down by africa
\item therfore, he truly is, racist
\end{itemize}

\item isabella asks

\begin{itemize}
\item clarifying question
\end{itemize}

\item calder

\begin{itemize}
\item europe is infecting africa
\item europe is the heart of darkness

\begin{itemize}
\item oooh, intrestingg
\end{itemize}

\item maybe both cus they were infected
\end{itemize}
\end{itemize}

\begin{center}
\begin{tabular}{l}
\[Question\ Break\]\\
\end{tabular}
\end{center}

\begin{itemize}
\item isabella

\begin{itemize}
\item what
\end{itemize}

\item nico

\begin{itemize}
\item marlow describes kurtz to the intended not with his actual beliefs
but with how kurtz is described to him
\item flip flops alot
\end{itemize}

\item kayla disagrees with nico

\begin{itemize}
\item says that marlow doesnt belive what he said to the intended
\item did nico say this? he very well may have
\end{itemize}
\end{itemize}
\end{document}
