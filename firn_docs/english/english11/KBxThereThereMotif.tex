% Created 2021-09-27 Mon 12:02
% Intended LaTeX compiler: xelatex
\documentclass[letterpaper]{article}
\usepackage{graphicx}
\usepackage{grffile}
\usepackage{longtable}
\usepackage{wrapfig}
\usepackage{rotating}
\usepackage[normalem]{ulem}
\usepackage{amsmath}
\usepackage{textcomp}
\usepackage{amssymb}
\usepackage{capt-of}
\usepackage{hyperref}
\setlength{\parindent}{0pt}
\usepackage[margin=1in]{geometry}
\usepackage{fontspec}
\usepackage{svg}
\usepackage{cancel}
\usepackage{indentfirst}
\setmainfont[ItalicFont = LiberationSans-Italic, BoldFont = LiberationSans-Bold, BoldItalicFont = LiberationSans-BoldItalic]{LiberationSans}
\newfontfamily\NHLight[ItalicFont = LiberationSansNarrow-Italic, BoldFont       = LiberationSansNarrow-Bold, BoldItalicFont = LiberationSansNarrow-BoldItalic]{LiberationSansNarrow}
\newcommand\textrmlf[1]{{\NHLight#1}}
\newcommand\textitlf[1]{{\NHLight\itshape#1}}
\let\textbflf\textrm
\newcommand\textulf[1]{{\NHLight\bfseries#1}}
\newcommand\textuitlf[1]{{\NHLight\bfseries\itshape#1}}
\usepackage{fancyhdr}
\pagestyle{fancy}
\usepackage{titlesec}
\usepackage{titling}
\makeatletter
\lhead{\textbf{\@title}}
\makeatother
\rhead{\textrmlf{Compiled} \today}
\lfoot{\theauthor\ \textbullet \ \textbf{2021-2022}}
\cfoot{}
\rfoot{\textrmlf{Page} \thepage}
\renewcommand{\tableofcontents}{}
\titleformat{\section} {\Large} {\textrmlf{\thesection} {|}} {0.3em} {\textbf}
\titleformat{\subsection} {\large} {\textrmlf{\thesubsection} {|}} {0.2em} {\textbf}
\titleformat{\subsubsection} {\large} {\textrmlf{\thesubsubsection} {|}} {0.1em} {\textbf}
\setlength{\parskip}{0.45em}
\renewcommand\maketitle{}
\author{Huxley Marvit}
\date{\today}
\title{There There Motif}
\hypersetup{
 pdfauthor={Huxley Marvit},
 pdftitle={There There Motif},
 pdfkeywords={},
 pdfsubject={},
 pdfcreator={Emacs 28.0.50 (Org mode 9.4.4)}, 
 pdflang={English}}
\begin{document}

\tableofcontents

\#ret \#hw

\noindent\rule{\textwidth}{0.5pt}

\section{motif: maybe.}
\label{sec:org6b05ba3}
\href{https://docs.google.com/document/d/1AzruqoxZE4CwH\_yfuWJDRB9tF0qZuTVk8UNt14dpZFQ/edit}{assignment}

doesnt bother to answer because not expecting an answer

reflecs lived in contecxt of unanswered crimes

answer is too horrible

*Evidence (multiple instances) and close-reading analysis of one passage
(examine particular speaker choices in literary craft: style, syntax,
word choice/connotation, allusion, etc. What do these choices suggest or
connote?)*

\begin{itemize}
\item "That's how looking like a monster works out for me. The Drome. And
when I stand up, when I stand up real fucking tall like I can,
nobody'll fuck with me. Everybody runs like they seen a ghost. Maybe I
am a ghost. Maybe Maxine doesn't even know who I am. Maybe I'm the
opposite of a medicine person. Maybe I'm'a do something one day, and
everybody's gonna know about me. Maybe that's when I'll come to life.
Maybe that's when they'll finally be able to look at me, because
they'll have to." (18)\\
\item "Her bear wasn't named One Shoe, but maybe I should have considered
myself lucky to have a bear with two shoes and not just one. But then
bears don't wear shoes, so maybe I wasn't lucky" (38)
\item "After that we did nothing every day but find out what the meals were
and when they would be served. We stayed on the island because there
was no other choice. There was no house or life to go back to, no hope
that maybe we would get what we were asking for, that the government
would have mercy on us, spare our throats by sending boats of food and
electricians, builders, and contractors to fix the place up. The days
just passed, and nothing happened. [\ldots{}] 'We're gonna get outta here.
Don't you two worry,' our mom said to us one night from across the
cell. But I no longer trusted her. I was unsure of whose side she was
on, or if there were even sides anymore. Maybe there were only sides
like there were sides on the rocks at the edge of the island." (45)
\item "Dene thinks Calvin is nervous, but then Dene is nervous, he is always
nervous, so maybe it's projection. But projection as a concept is a
slippery slope because everything could be projection. He is regularly
subject to solipsism's recursive, drowning affect." (109)
\item "I think about the irresistible-force paradox. How there cannot be
both an irresistible force and an immovable object in existence at
once. But what is happening in my blocked, coiled, possibly knotted
bowels? Could it be the working out of an ancient paradox? If shitting
mysteriously stopped, then couldn't seeing, hearing, breathing, do so
in turn? No. It's all the shitty food. Paradoxes don't work out. They
cancel out. I'm overthinking it." (50)
\item "Opal carried the weight of Ronald's possible death around with her
for a year. She was scared to go back and check. She was afraid that
it didn't bother her that he was dead. That she killed him. She didn't
want to go and find out if he was still alive. But she didn't really
want to have killed him either. It was easier to let him stay maybe
dead. Possibly dead." (121)
\end{itemize}

\subsubsection{Analysis}
\label{sec:orgada94d0}
\begin{itemize}
\item "Opal carried the weight of Ronald's possible death around with her
for a year. She was scared to go back and check. She was afraid that
it didn't bother her that he was dead. That she killed him. She didn't
want to go and find out if he was still alive. But she didn't really
want to have killed him either. It was easier to let him stay maybe
dead. Possibly dead." (121)
\end{itemize}

Opal is not described as carrying the weight of Ronald's death -- Opal
is described as carrying the weight of his possible death. This
possibility, this state of "maybe" is what Opal is carrying. Opal
doesn't want Ronald to be "still alive," but she doesn't want to have
killed him either; she is choosing the weight of uncertainty over the
weight of responsibility. This state of "maybe" is not a transitional
state, as Opal let's Ronald stay "maybe dead." Instead, it is a state of
being. In a world of uncertainty and tragedy, letting reality sit in a
state of maybe becomes "easier," and weight of uncertainty becomes
comforting.

\noindent\rule{\textwidth}{0.5pt}

scared of own lack of morality? physical weight? "possible death" not
carrying the death itself but the uncertainty wants him to be dead, but
doesnt want to have killed him (responsibility) easier to sit in state
of "maybe dead"

\begin{itemize}
\item \textbf{Why are these particular passage(s) significant in the larger
development or role of the motif as a whole within the novel?}
\end{itemize}

On a broader level, \emph{There There} has a theme of unanswered questions,
where characters wonder about the world without seeking or even valuing
answers. This lack of expectation of answers reflects the characters
lived experiences -- growing up in an environment of unanswered crimes
and horrible truths. The book itself is not even asking for answers,
it's only detailing the modern day effects of the atrocities committed
in the past. These constant unanswered questions that the characters ask
encapsulate how they have been forced to interact with the world
surrounding them.
\end{document}
