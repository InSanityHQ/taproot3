% Created 2021-09-27 Mon 12:00
% Intended LaTeX compiler: xelatex
\documentclass[letterpaper]{article}
\usepackage{graphicx}
\usepackage{grffile}
\usepackage{longtable}
\usepackage{wrapfig}
\usepackage{rotating}
\usepackage[normalem]{ulem}
\usepackage{amsmath}
\usepackage{textcomp}
\usepackage{amssymb}
\usepackage{capt-of}
\usepackage{hyperref}
\setlength{\parindent}{0pt}
\usepackage[margin=1in]{geometry}
\usepackage{fontspec}
\usepackage{svg}
\usepackage{cancel}
\usepackage{indentfirst}
\setmainfont[ItalicFont = LiberationSans-Italic, BoldFont = LiberationSans-Bold, BoldItalicFont = LiberationSans-BoldItalic]{LiberationSans}
\newfontfamily\NHLight[ItalicFont = LiberationSansNarrow-Italic, BoldFont       = LiberationSansNarrow-Bold, BoldItalicFont = LiberationSansNarrow-BoldItalic]{LiberationSansNarrow}
\newcommand\textrmlf[1]{{\NHLight#1}}
\newcommand\textitlf[1]{{\NHLight\itshape#1}}
\let\textbflf\textrm
\newcommand\textulf[1]{{\NHLight\bfseries#1}}
\newcommand\textuitlf[1]{{\NHLight\bfseries\itshape#1}}
\usepackage{fancyhdr}
\pagestyle{fancy}
\usepackage{titlesec}
\usepackage{titling}
\makeatletter
\lhead{\textbf{\@title}}
\makeatother
\rhead{\textrmlf{Compiled} \today}
\lfoot{\theauthor\ \textbullet \ \textbf{2021-2022}}
\cfoot{}
\rfoot{\textrmlf{Page} \thepage}
\renewcommand{\tableofcontents}{}
\titleformat{\section} {\Large} {\textrmlf{\thesection} {|}} {0.3em} {\textbf}
\titleformat{\subsection} {\large} {\textrmlf{\thesubsection} {|}} {0.2em} {\textbf}
\titleformat{\subsubsection} {\large} {\textrmlf{\thesubsubsection} {|}} {0.1em} {\textbf}
\setlength{\parskip}{0.45em}
\renewcommand\maketitle{}
\author{Huxley Marvit}
\date{\today}
\title{Watson Notes}
\hypersetup{
 pdfauthor={Huxley Marvit},
 pdftitle={Watson Notes},
 pdfkeywords={},
 pdfsubject={},
 pdfcreator={Emacs 28.0.50 (Org mode 9.4.4)}, 
 pdflang={English}}
\begin{document}

\tableofcontents

\noindent\rule{\textwidth}{0.5pt}

\#reading : \href{GHMW Unit 1.pdf.org}{GHMW Unit 1.pdf} \#\# Watson \emph{\textasciitilde{}}
\texttt{The Evolution of International Society; a comparative, Historical Analysis}

Classification words of civilizations are not descriptive enough. Way to
broad.

Empires fall along a spectrum from absolute independence to absolute
empire. The absolutes are theoretical and do not occur in practice.

\section{Four Broad Categories:}
\label{sec:orge556bb0}
\begin{itemize}
\item Independence
\item Hegemony
\item Dominion
\item Empire
\end{itemize}

Order -> peace / prosperity, less freedom

Rules \emph{expected to} benefit all members of system

\begin{quote}
commitments to a collective security
\end{quote}

\textbf{Freedom of action in an independent states is limited by the pressures
of interdependence in a system}

\texttt{Hegemony = Being able to exert a *law* above the operatings of the system}

\^{} Sometimes though of as only one person

Broader:
\texttt{Hegemony =  Being albe to determine the relations between the members of an international society}

\texttt{Suzerainty = One state hold total political power over another}

\begin{quote}
shadowy overlord-ship
\end{quote}

hegemony requires tacit acceptance

\texttt{Dominion = Imperial authoirty determines the internal government of other communities, but retain their identity as seperate states and some control over their own affairs}

\texttt{Empire = direct administration of differnt communities from an imperial centre}

\texttt{=Continuum!=}

Community bound by: - custom - ethnic descent - religion - language

\^{} the importance of these fluctuates drastically over time

\section{Pendulum Theory}
\label{sec:org919acd9}
Hegemony on one side, dominion on the other, swings between and over
corrects.

\section{Examples}
\label{sec:org0e90e74}
\begin{itemize}
\item Suzerainty / Dominion?

\begin{itemize}
\item UK's control over India
\item "the thing that happened with Russia a few years ago"
\item China with Tawain?
\item China with Hong Kong
\end{itemize}

\item Empire

\begin{itemize}
\item Differentiate from "bloated state" with

\begin{itemize}
\item Core + periphery

\begin{itemize}
\item Provinces with separate rights, but not independent on paper
\end{itemize}
\end{itemize}

\item Holy roman empire

\begin{itemize}
\item Weird mix of empire + hegemony. Like feudalism
\end{itemize}
\end{itemize}
\end{itemize}

\begin{quote}
Are there patterns in world history?
\end{quote}

Takeover -> split -> slow reforming

\section{The State}
\label{sec:org4cbee1e}
What is a state? > a state is a human community that (successfully)
claims the monopoly of the legitimate use of physical force within a
given territory

Where do states come from?

\textbf{Social contract theory:} > The state rises from the cumulative
experience of a populations\ldots{}

Basically, the need for a leader / social contract arises with size.

\textbf{Bellicist theory:} with Charles Tilly > war makes states and states
make war

\subsubsection{Four Functions of a state}
\label{sec:org74f6f11}
\begin{quote}
Think of the state as like the mafia
\end{quote}

\href{Screen Shot 2020-08-26 at 2.51.49 PM.png.org}{Screen Shot
2020-08-26 at 2.51.49 PM.png}

\begin{itemize}
\item War Making

\begin{itemize}
\item Created taxes

\begin{itemize}
\item Leads to protection
\end{itemize}
\end{itemize}

\item State Making
\item Protection
\item Extraction

\begin{itemize}
\item Want rich people to stay and not be dangerous

\begin{itemize}
\item Solution: Legal system
\end{itemize}
\end{itemize}
\end{itemize}

\section{Bellicist theory group work}
\label{sec:orgc30183f}
\begin{verbatim}
With your group, brainstorm the following:
What hypotheses might you generate from the bellicist theory of the state? (i.e. “If I change variable X in a state’s history/situation/etc. the effect will be Y”)
Using history as your data, how might you test your hypothesis?
Bonus: Can you think of any specific examples in history (regions, events, states/empires) that might be good places to look for evidence for or against the bellicist theory?
\end{verbatim}
\end{document}
