% Created 2021-09-27 Mon 12:00
% Intended LaTeX compiler: xelatex
\documentclass[letterpaper]{article}
\usepackage{graphicx}
\usepackage{grffile}
\usepackage{longtable}
\usepackage{wrapfig}
\usepackage{rotating}
\usepackage[normalem]{ulem}
\usepackage{amsmath}
\usepackage{textcomp}
\usepackage{amssymb}
\usepackage{capt-of}
\usepackage{hyperref}
\setlength{\parindent}{0pt}
\usepackage[margin=1in]{geometry}
\usepackage{fontspec}
\usepackage{svg}
\usepackage{cancel}
\usepackage{indentfirst}
\setmainfont[ItalicFont = LiberationSans-Italic, BoldFont = LiberationSans-Bold, BoldItalicFont = LiberationSans-BoldItalic]{LiberationSans}
\newfontfamily\NHLight[ItalicFont = LiberationSansNarrow-Italic, BoldFont       = LiberationSansNarrow-Bold, BoldItalicFont = LiberationSansNarrow-BoldItalic]{LiberationSansNarrow}
\newcommand\textrmlf[1]{{\NHLight#1}}
\newcommand\textitlf[1]{{\NHLight\itshape#1}}
\let\textbflf\textrm
\newcommand\textulf[1]{{\NHLight\bfseries#1}}
\newcommand\textuitlf[1]{{\NHLight\bfseries\itshape#1}}
\usepackage{fancyhdr}
\pagestyle{fancy}
\usepackage{titlesec}
\usepackage{titling}
\makeatletter
\lhead{\textbf{\@title}}
\makeatother
\rhead{\textrmlf{Compiled} \today}
\lfoot{\theauthor\ \textbullet \ \textbf{2021-2022}}
\cfoot{}
\rfoot{\textrmlf{Page} \thepage}
\renewcommand{\tableofcontents}{}
\titleformat{\section} {\Large} {\textrmlf{\thesection} {|}} {0.3em} {\textbf}
\titleformat{\subsection} {\large} {\textrmlf{\thesubsection} {|}} {0.2em} {\textbf}
\titleformat{\subsubsection} {\large} {\textrmlf{\thesubsubsection} {|}} {0.1em} {\textbf}
\setlength{\parskip}{0.45em}
\renewcommand\maketitle{}
\author{Zachary Sayyah}
\date{\today}
\title{State theory}
\hypersetup{
 pdfauthor={Zachary Sayyah},
 pdftitle={State theory},
 pdfkeywords={},
 pdfsubject={},
 pdfcreator={Emacs 28.0.50 (Org mode 9.4.4)}, 
 pdflang={English}}
\begin{document}

\tableofcontents



\section{States}
\label{sec:org043ca88}
\href{https://docs.google.com/presentation/d/1vJnwwlECaFQkPPlls\_B68Ujhqd4zRIZwixMch\_TnO8g/edit}{Slides}
\#\# Why do States Start? - One theory called social contract theory is
that a state arises from the cumulative experience of a population's
self-government as it grows and requires more and more coordination. -
This has been used as a reference for state creation - Another theory
called the Bellicist theory (formed by Charles Tilly) of the state says
that states form to make war on other states. - States exist to control
territory from external threats - States exist to eliminate internal
rivals and insurgents - States exist to protect their people - States
take taxes and and revenue to form a state

\subsection{How Bellicist works on a State.}
\label{sec:org7baf969}
\begin{itemize}
\item An army is useful for both war making and state making according to
Bellicist theory.

\begin{itemize}
\item To be good at war making a state needs to efficiently tax it's
people to be succesful at war making.

\begin{itemize}
\item Can not be too much, or too little.

\begin{itemize}
\item Needs to beat rival
\end{itemize}

\item Needs to make wealthy people buy into government and pay them
taxes for protection.

\begin{itemize}
\item Can lead to problematic policies as the wealthy people matter
more for taxes
\end{itemize}
\end{itemize}
\end{itemize}
\end{itemize}
\end{document}
