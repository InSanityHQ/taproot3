% Created 2021-09-27 Mon 12:00
% Intended LaTeX compiler: xelatex
\documentclass[letterpaper]{article}
\usepackage{graphicx}
\usepackage{grffile}
\usepackage{longtable}
\usepackage{wrapfig}
\usepackage{rotating}
\usepackage[normalem]{ulem}
\usepackage{amsmath}
\usepackage{textcomp}
\usepackage{amssymb}
\usepackage{capt-of}
\usepackage{hyperref}
\setlength{\parindent}{0pt}
\usepackage[margin=1in]{geometry}
\usepackage{fontspec}
\usepackage{svg}
\usepackage{cancel}
\usepackage{indentfirst}
\setmainfont[ItalicFont = LiberationSans-Italic, BoldFont = LiberationSans-Bold, BoldItalicFont = LiberationSans-BoldItalic]{LiberationSans}
\newfontfamily\NHLight[ItalicFont = LiberationSansNarrow-Italic, BoldFont       = LiberationSansNarrow-Bold, BoldItalicFont = LiberationSansNarrow-BoldItalic]{LiberationSansNarrow}
\newcommand\textrmlf[1]{{\NHLight#1}}
\newcommand\textitlf[1]{{\NHLight\itshape#1}}
\let\textbflf\textrm
\newcommand\textulf[1]{{\NHLight\bfseries#1}}
\newcommand\textuitlf[1]{{\NHLight\bfseries\itshape#1}}
\usepackage{fancyhdr}
\pagestyle{fancy}
\usepackage{titlesec}
\usepackage{titling}
\makeatletter
\lhead{\textbf{\@title}}
\makeatother
\rhead{\textrmlf{Compiled} \today}
\lfoot{\theauthor\ \textbullet \ \textbf{2021-2022}}
\cfoot{}
\rfoot{\textrmlf{Page} \thepage}
\renewcommand{\tableofcontents}{}
\titleformat{\section} {\Large} {\textrmlf{\thesection} {|}} {0.3em} {\textbf}
\titleformat{\subsection} {\large} {\textrmlf{\thesubsection} {|}} {0.2em} {\textbf}
\titleformat{\subsubsection} {\large} {\textrmlf{\thesubsubsection} {|}} {0.1em} {\textbf}
\setlength{\parskip}{0.45em}
\renewcommand\maketitle{}
\author{Huxley}
\date{\today}
\title{Data Processing Responses}
\hypersetup{
 pdfauthor={Huxley},
 pdftitle={Data Processing Responses},
 pdfkeywords={},
 pdfsubject={},
 pdfcreator={Emacs 28.0.50 (Org mode 9.4.4)}, 
 pdflang={English}}
\begin{document}

\tableofcontents

\#ret

\noindent\rule{\textwidth}{0.5pt}

Original Submission:
\href{KBDataProcessingRett.org}{KBDataProcessingRett}

\section{Comment Response}
\label{sec:orge236665}
\begin{itemize}
\item 2b. How do you convert from the 1-10 rating scale to
positive/negative/neutral?

\begin{itemize}
\item You would have to pick ranges that define what is
positive/negative/neutral
\end{itemize}

\item 2c. Given that the reviews vary greatly in their length, is one of
these preferred over the other?

\begin{itemize}
\item BOW doesn't work well when length varies. TFIDF, however, does.
\end{itemize}

\item 2e. Why not accuracy, precision, or recall?

\begin{itemize}
\item Those would all work as well. I just decided to list one possible
validation metric as opposed to all of them.
\end{itemize}

\item 3a. This could work, but I think you will find this problem more
straightforward as a different kind of problem.

\begin{itemize}
\item I guess this problem could be done as a clustering problem,
representing each movie as a location in a multidimensional space
then placing users in the space and clustering.
\end{itemize}

\item 3b. Semi-supervised usually refers to models where we have some
labels, and we generate additional labels. Where do our labels come
from? How do we generate the additional ones?

\begin{itemize}
\item The original labels would most likely be generated. As time
progresses, we get labels back from the users that are interacting
with our model's recommendations.
\end{itemize}

\item 3c. Be more specific: which features would you use which techniques
on? For example, you could use bag of words or OHE on the names of the
stars, but you'd get pretty different results depending on which one
you picked.

\begin{itemize}
\item OHE: name of stars, genre. BOW or TFID: title.
\end{itemize}

\item 3e. A user not watching a recommendation is not necessarily good
signal on whether it's a good recommendation (i.e. I might really like
a movie, and just not have time to watch it right now). Conversely, a
user watching a movie is not necessarily signal that it was a good
recommendation (maybe I watched it because you recommended it, but I
hated it).

\begin{itemize}
\item This depends on the goal of the recommendation. If you get a good
recommendation and don't watch it, was it really a good
recommendation? If the goal is simply to increase watch time, and
thus increase the number of ads viewed, then yes watching or not
watching is a good metric. Think click-bait. Of course, this has to
be weighed against the long term effects of the recommendation.
\end{itemize}

\item 3f. I know you are kidding (I hope so, at least!), but the whole point
of us learning about ethics is precisely so that they do not crumble
in the face of capitalism!

\begin{itemize}
\item I think maybe we should shift the goal to rebuilding the already
crumbled ethics\ldots{}
\end{itemize}

\item 4b. Where do these labels come from?

\begin{itemize}
\item From the friends you asked.
\end{itemize}

\item 4c/d. Why WPIE/NN? Is there a simpler approach we might try first?

\begin{itemize}
\item BOW, TFIDF, or word vectors would work, but Facebook's method itself
would most likely work better. As for the type of model, Decision
Trees and Random Forests would work.
\end{itemize}

\item 4e. Why not accuracy, precision, or recall?

\begin{itemize}
\item Again, just decided to list one validation metric. Accuracy,
precision, and recall would also work.
\end{itemize}
\end{itemize}
\end{document}
