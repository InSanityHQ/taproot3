% Created 2021-09-27 Mon 12:00
% Intended LaTeX compiler: xelatex
\documentclass[letterpaper]{article}
\usepackage{graphicx}
\usepackage{grffile}
\usepackage{longtable}
\usepackage{wrapfig}
\usepackage{rotating}
\usepackage[normalem]{ulem}
\usepackage{amsmath}
\usepackage{textcomp}
\usepackage{amssymb}
\usepackage{capt-of}
\usepackage{hyperref}
\setlength{\parindent}{0pt}
\usepackage[margin=1in]{geometry}
\usepackage{fontspec}
\usepackage{svg}
\usepackage{cancel}
\usepackage{indentfirst}
\setmainfont[ItalicFont = LiberationSans-Italic, BoldFont = LiberationSans-Bold, BoldItalicFont = LiberationSans-BoldItalic]{LiberationSans}
\newfontfamily\NHLight[ItalicFont = LiberationSansNarrow-Italic, BoldFont       = LiberationSansNarrow-Bold, BoldItalicFont = LiberationSansNarrow-BoldItalic]{LiberationSansNarrow}
\newcommand\textrmlf[1]{{\NHLight#1}}
\newcommand\textitlf[1]{{\NHLight\itshape#1}}
\let\textbflf\textrm
\newcommand\textulf[1]{{\NHLight\bfseries#1}}
\newcommand\textuitlf[1]{{\NHLight\bfseries\itshape#1}}
\usepackage{fancyhdr}
\pagestyle{fancy}
\usepackage{titlesec}
\usepackage{titling}
\makeatletter
\lhead{\textbf{\@title}}
\makeatother
\rhead{\textrmlf{Compiled} \today}
\lfoot{\theauthor\ \textbullet \ \textbf{2021-2022}}
\cfoot{}
\rfoot{\textrmlf{Page} \thepage}
\renewcommand{\tableofcontents}{}
\titleformat{\section} {\Large} {\textrmlf{\thesection} {|}} {0.3em} {\textbf}
\titleformat{\subsection} {\large} {\textrmlf{\thesubsection} {|}} {0.2em} {\textbf}
\titleformat{\subsubsection} {\large} {\textrmlf{\thesubsubsection} {|}} {0.1em} {\textbf}
\setlength{\parskip}{0.45em}
\renewcommand\maketitle{}
\author{Houjun Liu}
\date{\today}
\title{Acien Regime}
\hypersetup{
 pdfauthor={Houjun Liu},
 pdftitle={Acien Regime},
 pdfkeywords={},
 pdfsubject={},
 pdfcreator={Emacs 28.0.50 (Org mode 9.4.4)}, 
 pdflang={English}}
\begin{document}

\tableofcontents



\section{Acien Régime}
\label{sec:org6aebd97}
The Old France was economy generally rural + dominated by subsistence
farming.

\subsection{Societal Structure}
\label{sec:org3306f4c}
Strict grid of social hierarchy resulted due to birthright => Great
Chain of Being

\textbf{Great Chain of Being} The entire "world" was, according to GCB,
organized structurally with god at the top, rocks at the bottom.

In the human world, King on the top --- God's divine representative
("L'état c'est moi" --- the state is me), then the clergy, then
aristocracy, and finally commoners.

In practice, this makes a\ldots{}

\subsection{Estates: "Caste" System}
\label{sec:org94beaf2}
French societies organized into three castes --- "estates". Old France
very Roman Catholic with the Church owning a large amount of resources
=> almost 10\% w/ the monarchs crowed in cathedrals.

\begin{itemize}
\item \textbf{First Estate} => clergy; enjoyed high status
\item \textbf{Second Estate} => aristocracy; provided military and monetary support

\begin{itemize}
\item \emph{Les Grands}: largest landholders w/ large houses
\item \emph{Seigneurs}: provincial nobles who simply owned estates in the
countryside
\end{itemize}

\item \textbf{Third Estate} => 97\% of the population

\begin{itemize}
\item Production!
\item Reproduction!
\item Work!
\item Relatively prosperous, but <40\% owned land
\item Most rented land from lords as tenant farmers/sharecroppers
\end{itemize}
\end{itemize}

\subsection{Infrastructure Disorganization}
\label{sec:orgee20710}
\begin{itemize}
\item No national currency, nor system of weights and measures, nor a market
\item Network of highways existed, but not very efficient
\end{itemize}

\subsection{Merchanitilist Economy}
\label{sec:org6d5a56c}
\begin{itemize}
\item Economic policy guided by \textbf{merchanitilist theory}

\begin{itemize}
\item Notion that precious metals holdings is the ultimate goal
\item Encouraged development of manufacturing to provide for global market
\item Development of the new \emph{bourgeoisie} class --- small merchants and
shopkeepers
\end{itemize}
\end{itemize}

\subsection{Need of Reform}
\label{sec:orgce41152}
The established merchantilist theory came under challenge by newer
philosophies like the free market theory.

\begin{itemize}
\item \textbf{Adam Smith}'s free market economy/baurseiosie challenged
mercantilistic economy
\end{itemize}
\end{document}
