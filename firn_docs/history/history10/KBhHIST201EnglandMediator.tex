% Created 2021-09-27 Mon 12:00
% Intended LaTeX compiler: xelatex
\documentclass[letterpaper]{article}
\usepackage{graphicx}
\usepackage{grffile}
\usepackage{longtable}
\usepackage{wrapfig}
\usepackage{rotating}
\usepackage[normalem]{ulem}
\usepackage{amsmath}
\usepackage{textcomp}
\usepackage{amssymb}
\usepackage{capt-of}
\usepackage{hyperref}
\setlength{\parindent}{0pt}
\usepackage[margin=1in]{geometry}
\usepackage{fontspec}
\usepackage{svg}
\usepackage{cancel}
\usepackage{indentfirst}
\setmainfont[ItalicFont = LiberationSans-Italic, BoldFont = LiberationSans-Bold, BoldItalicFont = LiberationSans-BoldItalic]{LiberationSans}
\newfontfamily\NHLight[ItalicFont = LiberationSansNarrow-Italic, BoldFont       = LiberationSansNarrow-Bold, BoldItalicFont = LiberationSansNarrow-BoldItalic]{LiberationSansNarrow}
\newcommand\textrmlf[1]{{\NHLight#1}}
\newcommand\textitlf[1]{{\NHLight\itshape#1}}
\let\textbflf\textrm
\newcommand\textulf[1]{{\NHLight\bfseries#1}}
\newcommand\textuitlf[1]{{\NHLight\bfseries\itshape#1}}
\usepackage{fancyhdr}
\pagestyle{fancy}
\usepackage{titlesec}
\usepackage{titling}
\makeatletter
\lhead{\textbf{\@title}}
\makeatother
\rhead{\textrmlf{Compiled} \today}
\lfoot{\theauthor\ \textbullet \ \textbf{2021-2022}}
\cfoot{}
\rfoot{\textrmlf{Page} \thepage}
\renewcommand{\tableofcontents}{}
\titleformat{\section} {\Large} {\textrmlf{\thesection} {|}} {0.3em} {\textbf}
\titleformat{\subsection} {\large} {\textrmlf{\thesubsection} {|}} {0.2em} {\textbf}
\titleformat{\subsubsection} {\large} {\textrmlf{\thesubsubsection} {|}} {0.1em} {\textbf}
\setlength{\parskip}{0.45em}
\renewcommand\maketitle{}
\author{Houjun Liu}
\date{\today}
\title{England, The Mediator, Grand Alliance}
\hypersetup{
 pdfauthor={Houjun Liu},
 pdftitle={England, The Mediator, Grand Alliance},
 pdfkeywords={},
 pdfsubject={},
 pdfcreator={Emacs 28.0.50 (Org mode 9.4.4)}, 
 pdflang={English}}
\begin{document}

\tableofcontents



\section{England, the Mediator}
\label{sec:org9e4796e}
\textbf{King Willam III} engineered a mediation position for England --- such
that they threw themselves to counter-balance any fights.

\begin{itemize}
\item England's \href{KBhHIST201RaisonDeEtat.org}{KBhHIST201RaisonDeEtat},
CLAIM @\href{KBhHIST201Kissinger.org}{KBhHIST201Kissinger} did not
require them to expand, for "national interest is in the preservation
of European balance."
\item Glorious revolution kicked James II off the throne, chose William of
Orange of the Netherlands as replacement
\end{itemize}

Willam used the fact that if France occupied Belgium, it would surely
eat up the Netherlands, to cause England to fight in the war: starting a
bitter fight between him and Louis XIV.

\subsection{Grand Alliance}
\label{sec:org693aa0d}
To fight the growing France, England formed the \textbf{grand alliance}

\begin{itemize}
\item Sweden
\item Spanish Savoy
\item Austria
\item Netherlands
\item England
\end{itemize}

all versus France.

Constantly fought, and left France to be strong but not dominant =>
\textbf{Textbook balance of power!}

Textbook example of
\href{KBhHIST201RaisonDeEtat.org}{KBhHIST201RaisonDeEtat}:
ideologically, England and France are on the same side. However, it is
not in England's best interest to join them

\begin{quote}
In this manner, Great Britain became the balancer of the European
equilibrium, first almost by default, later by conscious strategy.
Without Great Britain 's tenacious commitment to that role, France
would almost surely have achieved hegemony over Europe in the
eighteenth or nine- teenth century, and Germany would have done the
same in the modern period.
\end{quote}
\end{document}
