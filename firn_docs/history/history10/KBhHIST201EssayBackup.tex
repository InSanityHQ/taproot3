% Created 2021-09-27 Mon 12:00
% Intended LaTeX compiler: xelatex
\documentclass[letterpaper]{article}
\usepackage{graphicx}
\usepackage{grffile}
\usepackage{longtable}
\usepackage{wrapfig}
\usepackage{rotating}
\usepackage[normalem]{ulem}
\usepackage{amsmath}
\usepackage{textcomp}
\usepackage{amssymb}
\usepackage{capt-of}
\usepackage{hyperref}
\setlength{\parindent}{0pt}
\usepackage[margin=1in]{geometry}
\usepackage{fontspec}
\usepackage{svg}
\usepackage{cancel}
\usepackage{indentfirst}
\setmainfont[ItalicFont = LiberationSans-Italic, BoldFont = LiberationSans-Bold, BoldItalicFont = LiberationSans-BoldItalic]{LiberationSans}
\newfontfamily\NHLight[ItalicFont = LiberationSansNarrow-Italic, BoldFont       = LiberationSansNarrow-Bold, BoldItalicFont = LiberationSansNarrow-BoldItalic]{LiberationSansNarrow}
\newcommand\textrmlf[1]{{\NHLight#1}}
\newcommand\textitlf[1]{{\NHLight\itshape#1}}
\let\textbflf\textrm
\newcommand\textulf[1]{{\NHLight\bfseries#1}}
\newcommand\textuitlf[1]{{\NHLight\bfseries\itshape#1}}
\usepackage{fancyhdr}
\pagestyle{fancy}
\usepackage{titlesec}
\usepackage{titling}
\makeatletter
\lhead{\textbf{\@title}}
\makeatother
\rhead{\textrmlf{Compiled} \today}
\lfoot{\theauthor\ \textbullet \ \textbf{2021-2022}}
\cfoot{}
\rfoot{\textrmlf{Page} \thepage}
\renewcommand{\tableofcontents}{}
\titleformat{\section} {\Large} {\textrmlf{\thesection} {|}} {0.3em} {\textbf}
\titleformat{\subsection} {\large} {\textrmlf{\thesubsection} {|}} {0.2em} {\textbf}
\titleformat{\subsubsection} {\large} {\textrmlf{\thesubsubsection} {|}} {0.1em} {\textbf}
\setlength{\parskip}{0.45em}
\renewcommand\maketitle{}
\author{Houjun Liu}
\date{\today}
\title{Kennedy Essay Backup}
\hypersetup{
 pdfauthor={Houjun Liu},
 pdftitle={Kennedy Essay Backup},
 pdfkeywords={},
 pdfsubject={},
 pdfcreator={Emacs 28.0.50 (Org mode 9.4.4)}, 
 pdflang={English}}
\begin{document}

\tableofcontents

Paul Kennedy's \textit{The Rise and Fall of the Great Powers} presents a
theory in which he proposed that Europe's model of small nation-states
helped encourage competition and diversification that prevents
governmental overextension and domination that, according to Kennedy,
leads to eventual downfall. In proving his overall thesis, Kennedy made
a claim addressing the reason why China closed its maritime trade after
the 1433 expedition of Zheng He --- claiming that a conservative
following of Confucian principles and effective governmental obstinance
in Ming China's part lead to their eventual downfall in trade. In an
attempt to justify his arguments, Kennedy cited a lack of
=reconsideration'' in rebuilding trade and naval forces even when, according to his work, circumstances became clear to rebuild trade; he alleges that, instead of economic or domestic incentives, the Ming government refused to reopen due to its=conservative
Confucian'' nature. In Kennedy's claim arguing that the conservative
Confucian views caused Ming's closing of maritime trade, he presents a
faulty argument that ignores both the ineffectiveness of unnecessarily
reopening maritime trade and the discord in the infighting-plagued Ming
government; through his argument, Kennedy also misinterpreted Confucian
philosophy central to his argument for it, instead of an ideological ban
on trade, actually favors market circulation and advises governments not
to encourage price racketeering.

One main flaw in Kennedy's argument of the conservative Confucian cause
of the closing of maritime trade lies in the work's lack of
consideration to longer-term effects of reopening maritime trade under
the Ming's situation. In his argument, Kennedy claims that stopping
maritime activity
=does not appear to have been reconsidered when the disadvantanges \ldots{} became clear: \ldots{} the Chinese coastline and even cities \ldots{} were being attacked by Japanese pirates'' --- implying that there must be an alterer motive in not resuming maritime activity even in a situation that, hints Kennedy, is in dire need of redecision. Center to the this claim, however, is the notion that while there be no reason not to reopen trade, Ming China failed to do so. Unsurprisingly, this is untrue: after some political struggle, China in fact \textit{did} attempt an reopening of trade that, contrary to what Kennedy hints in his argument, did not rapidly resolve the economic situation in Ming china in both short and long term. As Kennedy mentioned, the closing of trade and subsequent deflation lead to rapid merchant attacks on coastal China --- especially in the coastal, trade-dependent region of Fujian. Yet, contrary to what Kennedy presented, the Ming government indeed=reversed
course [on trade]'', invalidating Kennedy's initial argument. Kennedy
failed to recognize the actual broader implications of simple trade
reopening that, indeed, actually lead to why China's economy ultimately
imploded. The Ming government reopened trade
=not only because it recognized its inability to stop smuggling, or because it had begun to appreciate how much Fujian's populace depended on trade. Beijing had come to realize that the nation desperately needed the merchants' most important good: silver.'' Silver, due to its relative stability, was a major asset at the time to the Ming government due to the disastrous hyperinflation/deflation cycles caused by repeated issuing of novel coins by each Emperor. Although this initial reopening of a silver-based economy proved to be reasonably successful in the short run, the Spanish discovery of the Potosí mine would actually ignite the broader economic downfall that Kennedy appears to associate with a simple ban on trade:=In
1642, so much silver has been produced that its value is falling even as
the mines slacken \ldots{} Like the Spanish king, the Ming emperor backs his
military ventures with Spanish silver, which his subjects must use to
pay their taxes. When the value of silver falls, the government runs out
of money.'' Contrary to Kennedy's claims, the Ming economy did not fail
due to a supposed lack of ``reconsideration'' on trade; it did
reconsider trading quite extensively, and instead imploded centuries
later due to a worldwide deflation caused by an influx of Spanish
silver.

Through his argument, Kennedy did --- in fact --- acknowledge one reason
that he claimed is a
=plausible'' reason for a closing of trade: the Mongols. Kennedy claims that maintaining=a
large navy was an expensive luxury.'' But, immediately following that
claim, Kennedy follows up with a seemingly contradicting statement
calling the Ming government to
=reconsider'' their anti-maritime decision due to the Japanese attacks aforementioned. Through this, Kennedy not only blindsided his own argument, but created another flaw in his argument by ignoring the longstanding political struggle for which the Mongolian attacks are, in fact, a scapegoat. Zheng He's voyages=had
become a target in political infighting --- one bureaucratic faction
championing them, \ldots{} [and] another trying to take down the first by
decrying their expense.'' With two bureaucratic factions championing
opposing causes --- sacrificing Zheng He as a means to an end --- it is
not surprising that one of the two would take an obvious and
understandable cause of the Mongols as an argument against the
expeditions through
=decrying'' the expense. The cancellation of the voyages by Yongle's son is, simply, a show of strength to the other=faction'':
power consolidation masqueraded as sensible economic policy. The Mongol
attacks, cited by those opposing Zheng He's voyages as economic
luxuriation, were in fact a perennial problem: the Ming was given
Mandate after overturning
=the Mongol-lead Yuan'', was --- as Kennedy correctly noted --- attacked (CITENEEDED) by Mongol invasions through its rule, and indeed was later overturned by the Manchus=who
had taken north \ldots{} [since] during Song times''. Attacks of China from
the north were constant and rampant, making it less than evident that an
elevation would cause China to destroy such a large naval investment.
Although there may exist many theories hypothesizing the reason why the
naval forces are destroyed --- infighting chief among which --- the
attacks from the North are more common then Kennedy claimed it to be:
not periodic and acute enough to warrant the destruction of such vast
naval forces and an subsequent ``reconsideration'' afterwards as Kennedy
had claimed.

Center to Kennedy's argument is his claim that the
=sheer conservatism'' in Ming's Confucian government brought the downfall to expeditions of Zheng He --- a claim that both egregiously misinterpreted the Confucian viewpoint on trade and paints an unfair caricature of the Ming government.=In
[the Ming's] `Restoration' atmosphere,'' claims Kennedy,
=the all-important officialdom was concerned to preserve and recapture the past, not to create a brighter future.'' In this argument, Kennedy hints that a preservation of the=past''
--- whichever model of the past that may be --- would necessarily bring
not only a full, unmitigated dismissal of present technologies, but also
a unexplainable dismissal of any possibility of improvement. If, per
Kennedy's claims, the
=sheer \ldots{} Confucian bureaucracy \ldots{} [was] a key element in China's retreat'' from all trade activities, then the conservative past that Kennedy claims China is emulating would need a stance greatly antagonistic to trade to warrant a full=destruction
of all sea-going vessels'' (CITENEEDED) and dramatic shutdown. However,
the views of the Confucian scholars in the 81 AD Han Salt and Iron
Debate disagrees:
=the purpose of merchants is circulation and the purpose of artisans is making tools. These matters should not become a major concern of the government.'' Through his claim, Kennedy candidly misinterpreted the Confucian viewpoint. Instead of a complete shutdown of trade, Confucian scholars claim that a well-ran government should not be \textit{concerned} with controlling trade themselves --- natural trade (i.e. creation and circulation), not monopolies and limits, should be encouraged. The Confucian view against merchants is not one that forbids their existence, but a view where=quick
traders and unscrupulous officials [that] buy when goods are cheat to
make high profits'' --- in other words, arbitraging and racketeering ---
is discouraged. Through the Salt and Iron debates, Confucian scholars
are trying to convince a struggling Han dynasty from imposing a strict
trade limitation. The traditional Confucian views that Kennedy claim
Ming China is desperately attempting to emulate in fact does not
correlate with the Ming government's action of limitation, rendering
Kennedy's claims invalid.

In Kennedy's claim that traditional Confucian views brought the Ming
government to stop all maritime trade, the author presents a problematic
claim that not only ignores both the ineffectiveness of trade opening
and the discord in the Ming leadership but also misinterpreted Confucian
philosophy central to his argument. Kennedy, through his faulty and
somewhat forced argumentation, applies a simplification to the rich and
nuanced Ming governmental system that simply is not warranted. Although
this calls for a reevaluation of his comparative analysis between
European and Oriental cultures in terms of competition, Kennedy's
primary arguments surrounding Europe in itself remains valid.
\end{document}
