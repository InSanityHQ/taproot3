% Created 2021-09-27 Mon 12:00
% Intended LaTeX compiler: xelatex
\documentclass[letterpaper]{article}
\usepackage{graphicx}
\usepackage{grffile}
\usepackage{longtable}
\usepackage{wrapfig}
\usepackage{rotating}
\usepackage[normalem]{ulem}
\usepackage{amsmath}
\usepackage{textcomp}
\usepackage{amssymb}
\usepackage{capt-of}
\usepackage{hyperref}
\setlength{\parindent}{0pt}
\usepackage[margin=1in]{geometry}
\usepackage{fontspec}
\usepackage{svg}
\usepackage{cancel}
\usepackage{indentfirst}
\setmainfont[ItalicFont = LiberationSans-Italic, BoldFont = LiberationSans-Bold, BoldItalicFont = LiberationSans-BoldItalic]{LiberationSans}
\newfontfamily\NHLight[ItalicFont = LiberationSansNarrow-Italic, BoldFont       = LiberationSansNarrow-Bold, BoldItalicFont = LiberationSansNarrow-BoldItalic]{LiberationSansNarrow}
\newcommand\textrmlf[1]{{\NHLight#1}}
\newcommand\textitlf[1]{{\NHLight\itshape#1}}
\let\textbflf\textrm
\newcommand\textulf[1]{{\NHLight\bfseries#1}}
\newcommand\textuitlf[1]{{\NHLight\bfseries\itshape#1}}
\usepackage{fancyhdr}
\pagestyle{fancy}
\usepackage{titlesec}
\usepackage{titling}
\makeatletter
\lhead{\textbf{\@title}}
\makeatother
\rhead{\textrmlf{Compiled} \today}
\lfoot{\theauthor\ \textbullet \ \textbf{2021-2022}}
\cfoot{}
\rfoot{\textrmlf{Page} \thepage}
\renewcommand{\tableofcontents}{}
\titleformat{\section} {\Large} {\textrmlf{\thesection} {|}} {0.3em} {\textbf}
\titleformat{\subsection} {\large} {\textrmlf{\thesubsection} {|}} {0.2em} {\textbf}
\titleformat{\subsubsection} {\large} {\textrmlf{\thesubsubsection} {|}} {0.1em} {\textbf}
\setlength{\parskip}{0.45em}
\renewcommand\maketitle{}
\author{Houjun Liu}
\date{\today}
\title{Models of History Systems}
\hypersetup{
 pdfauthor={Houjun Liu},
 pdftitle={Models of History Systems},
 pdfkeywords={},
 pdfsubject={},
 pdfcreator={Emacs 28.0.50 (Org mode 9.4.4)}, 
 pdflang={English}}
\begin{document}

\tableofcontents



\section{Models of History}
\label{sec:orgdbaa35d}
\subsection{Watson}
\label{sec:orgf49f9e7}
Watson's Model: scale from\ldots{}

\begin{center}
\begin{tabular}{llll}
Independence & .. & .. & Full Order\\
\hline
Indipendence States (no control) & Hegemony (some external control) & Dominion (some external, some internal control) & Empire (full control)\\
\end{tabular}
\end{center}

See \href{KBhHIST201Watson.org}{KBhHIST201Watson} Watson's Model of
States

\subsection{Arrigi}
\label{sec:org12b1908}
\begin{itemize}
\item When people talk about Hegemony, they often mean dominance
\end{itemize}

\begin{quote}
In order to achieve hegemony/dominion in a system, a state must
transform how a history operates
\end{quote}

\noindent\rule{\textwidth}{0.5pt}

A spectrum of governance with two extremes\ldots{}

\subsubsection{"Territorialist Model"}
\label{sec:orgb736cd2}
\begin{itemize}
\item Success measured with territory + control
\item Wealth and economy as byproduct
\end{itemize}

\subsubsection{"Capitalist Model"}
\label{sec:org4fd669b}
\begin{itemize}
\item Success measured with the control over resources and trade
\item Territorial acquisition as byproduct\\
\end{itemize}

\noindent\rule{\textwidth}{0.5pt}

Arrigi claims that there are two main modes of power: the Capitalist ---
controlling trade and resources --- and the territorialist ---
controlling land and people.

To reach \textbf{world hegemony} (become the world leader) --- change the world
order. For instance, Tang China turned the world from no model to a
Territorialist model; US turned the world from a Territorialist model to
a Capitalist model.

\subsubsection{Coercion vs Consent}
\label{sec:org43bc9e5}
\begin{itemize}
\item Means of power acquisition
\item Either\ldots{}

\begin{itemize}
\item Coercion --- force joining of a system via force (trade war, actual
war)
\item Consent --- use deals and negotiations to ask to join system
\end{itemize}
\end{itemize}

\subsection{Social Contract Theory}
\label{sec:org58c5fca}
\begin{quote}
The state arises from the cumulative experience of a populations'
self-government as it grows and requires more and more attention
\end{quote}

\subsection{Predatory THeory}
\label{sec:org69bb604}
\begin{verbatim}
"War makes states, and states make war" — Charles Tilly
\end{verbatim}

\subsubsection{Functions of a State}
\label{sec:orgf4059f3}
\begin{enumerate}
\item \textbf{War Making}: The act of eliminating rivals or potential external
threats outside of its own territories.
\item \textbf{State Making}: The act of eliminating internal rival forces and
insurgents from within its own territories.
\item \textbf{Protection}: The act of eliminating potential threats to its
population.
\item \textbf{Extraction}: The act of securing the means to execute the previous
three activities, such as the collection of taxes or revenue.
\end{enumerate}

\#flo \#disorganized

Merchantilist Empires => Merchants => Trade

\begin{itemize}
\item Portchugal and Spain focused on land aqusition

\begin{itemize}
\item King and queen doing conquring
\item Little privatization and more of a territarialist model
\end{itemize}

\item Dutch, French, and British empires focused more on actual trading

\begin{itemize}
\item Companies with charters

\begin{itemize}
\item But! Has own army
\item Coinage
\item and Court
\item (all vested on the authority of the government)
\end{itemize}

\item Mostly private investors and a capitalist model
\end{itemize}
\end{itemize}
\end{document}
