% Created 2021-09-27 Mon 12:00
% Intended LaTeX compiler: xelatex
\documentclass[letterpaper]{article}
\usepackage{graphicx}
\usepackage{grffile}
\usepackage{longtable}
\usepackage{wrapfig}
\usepackage{rotating}
\usepackage[normalem]{ulem}
\usepackage{amsmath}
\usepackage{textcomp}
\usepackage{amssymb}
\usepackage{capt-of}
\usepackage{hyperref}
\setlength{\parindent}{0pt}
\usepackage[margin=1in]{geometry}
\usepackage{fontspec}
\usepackage{svg}
\usepackage{cancel}
\usepackage{indentfirst}
\setmainfont[ItalicFont = LiberationSans-Italic, BoldFont = LiberationSans-Bold, BoldItalicFont = LiberationSans-BoldItalic]{LiberationSans}
\newfontfamily\NHLight[ItalicFont = LiberationSansNarrow-Italic, BoldFont       = LiberationSansNarrow-Bold, BoldItalicFont = LiberationSansNarrow-BoldItalic]{LiberationSansNarrow}
\newcommand\textrmlf[1]{{\NHLight#1}}
\newcommand\textitlf[1]{{\NHLight\itshape#1}}
\let\textbflf\textrm
\newcommand\textulf[1]{{\NHLight\bfseries#1}}
\newcommand\textuitlf[1]{{\NHLight\bfseries\itshape#1}}
\usepackage{fancyhdr}
\pagestyle{fancy}
\usepackage{titlesec}
\usepackage{titling}
\makeatletter
\lhead{\textbf{\@title}}
\makeatother
\rhead{\textrmlf{Compiled} \today}
\lfoot{\theauthor\ \textbullet \ \textbf{2021-2022}}
\cfoot{}
\rfoot{\textrmlf{Page} \thepage}
\renewcommand{\tableofcontents}{}
\titleformat{\section} {\Large} {\textrmlf{\thesection} {|}} {0.3em} {\textbf}
\titleformat{\subsection} {\large} {\textrmlf{\thesubsection} {|}} {0.2em} {\textbf}
\titleformat{\subsubsection} {\large} {\textrmlf{\thesubsubsection} {|}} {0.1em} {\textbf}
\setlength{\parskip}{0.45em}
\renewcommand\maketitle{}
\author{Houjun Liu}
\date{\today}
\title{The Industrial Revolution}
\hypersetup{
 pdfauthor={Houjun Liu},
 pdftitle={The Industrial Revolution},
 pdfkeywords={},
 pdfsubject={},
 pdfcreator={Emacs 28.0.50 (Org mode 9.4.4)}, 
 pdflang={English}}
\begin{document}

\tableofcontents



\section{The Industrial Revolution}
\label{sec:org223c190}
\subsection{Background: Post Napoleon Europe\ldots{}}
\label{sec:org25b80d9}
\textbf{Post-Napoleonic International System was "strange"}

\subsubsection{New International Order}
\label{sec:org2ea561c}
@\href{KBhHIST201MasonAndKennedy.org}{KBhHIST201MasonAndKennedy}

\begin{itemize}
\item Erosion of tariff barrieriers + widespread free trade ideals => new
international order
\item Struggle between 1793-1815 ("Great War") caused a \textbf{yearning for
stability}
\end{itemize}

The newfound yearning caused a longer term wish for stability\ldots{} But!
Wars did not end: \textbf{Regional/individual conflicts \{territory,
nationality, etc.\} persisted}. however, this did manage to limit the
scope of conflicts

\begin{itemize}
\item Europe's technical superiority increased the conflict for the less
developed
\item CLAIM: military power accompanied the economic globalization @Kennedy,
\href{KBhHIST201MasonAndKennedy.org}{KBhHIST201MasonAndKennedy}
\end{itemize}

\subsection{The Industrial Revolution}
\label{sec:org614840e}
\definition{Industrial Revolution}{The substitution of inanimate for animated sources of power through the conversion of heat into work.}
@Kennedy
\href{KBhHIST201MasonAndKennedy.org}{KBhHIST201MasonAndKennedy}'s
thesis claim Thesis CLAIM: international economy growth, Industrial
Revolution, European stability, and military modernization => favored
Great Powers

Of course, IR is a very \emph{slow} process.

\subsubsection{The Revolution}
\label{sec:orgf327206}
\begin{itemize}
\item At its nascent stages started affecting only a certain amount of
manufacturers + is a slow-moving process
\end{itemize}

\textbf{Is Technologically Beneficial}: allowed mankind to explore new sources
of energy. For instance, twenty-folded the production of driven looms

\textbf{Had cascading effect}: better access to technology turned textile
industry to be more productive which created a demand to more machines,
cotton, iron, and communication

Also other systemic benefits like the creation of a the wage-labor
factory work system => massive increase in productivity.

\subsubsection{Impacts of IR}
\label{sec:org65817e0}
The Industrial Revolution, of course, had lots of impacts. Please see
\href{KBhHIST201ImpactsofIR.org}{KBhHIST201ImpactsofIR}
\end{document}
