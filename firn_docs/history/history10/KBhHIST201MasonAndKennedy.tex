% Created 2021-09-27 Mon 12:00
% Intended LaTeX compiler: xelatex
\documentclass[letterpaper]{article}
\usepackage{graphicx}
\usepackage{grffile}
\usepackage{longtable}
\usepackage{wrapfig}
\usepackage{rotating}
\usepackage[normalem]{ulem}
\usepackage{amsmath}
\usepackage{textcomp}
\usepackage{amssymb}
\usepackage{capt-of}
\usepackage{hyperref}
\setlength{\parindent}{0pt}
\usepackage[margin=1in]{geometry}
\usepackage{fontspec}
\usepackage{svg}
\usepackage{cancel}
\usepackage{indentfirst}
\setmainfont[ItalicFont = LiberationSans-Italic, BoldFont = LiberationSans-Bold, BoldItalicFont = LiberationSans-BoldItalic]{LiberationSans}
\newfontfamily\NHLight[ItalicFont = LiberationSansNarrow-Italic, BoldFont       = LiberationSansNarrow-Bold, BoldItalicFont = LiberationSansNarrow-BoldItalic]{LiberationSansNarrow}
\newcommand\textrmlf[1]{{\NHLight#1}}
\newcommand\textitlf[1]{{\NHLight\itshape#1}}
\let\textbflf\textrm
\newcommand\textulf[1]{{\NHLight\bfseries#1}}
\newcommand\textuitlf[1]{{\NHLight\bfseries\itshape#1}}
\usepackage{fancyhdr}
\pagestyle{fancy}
\usepackage{titlesec}
\usepackage{titling}
\makeatletter
\lhead{\textbf{\@title}}
\makeatother
\rhead{\textrmlf{Compiled} \today}
\lfoot{\theauthor\ \textbullet \ \textbf{2021-2022}}
\cfoot{}
\rfoot{\textrmlf{Page} \thepage}
\renewcommand{\tableofcontents}{}
\titleformat{\section} {\Large} {\textrmlf{\thesection} {|}} {0.3em} {\textbf}
\titleformat{\subsection} {\large} {\textrmlf{\thesubsection} {|}} {0.2em} {\textbf}
\titleformat{\subsubsection} {\large} {\textrmlf{\thesubsubsection} {|}} {0.1em} {\textbf}
\setlength{\parskip}{0.45em}
\renewcommand\maketitle{}
\author{Houjun Liu}
\date{\today}
\title{Mason and Kennedy}
\hypersetup{
 pdfauthor={Houjun Liu},
 pdftitle={Mason and Kennedy},
 pdfkeywords={},
 pdfsubject={},
 pdfcreator={Emacs 28.0.50 (Org mode 9.4.4)}, 
 pdflang={English}}
\begin{document}

\tableofcontents



\section{Mason and Kennedy}
\label{sec:org05ec160}
\subsection{Modern Europe Political Philosophies}
\label{sec:orge9bfe59}
See
\href{KBhHIST201LiberalismAndNationalism.org}{KBhHIST201LiberalismAndNationalism}

1848 German Confederation of States created based on novel nationalistic
ideals anchored around German speaking countries.

\subsection{The Industrial Revolution}
\label{sec:org2014d63}
See
\href{KBhHIST201IndustrialRevolution.org}{KBhHIST201IndustrialRevolution}

\subsection{Germanic States + Struggles of Nationalism}
\label{sec:org208c8c9}
\textbf{The 1860s created two important new states in Europe => using warfare
and civic nationalism to create new connections}

See
\href{KBhHIST201GermanicNationalism.org}{KBhHIST201GermanicNationalism}

\subsubsection{Austria-Hungary}
\label{sec:orgf3b03fc}
After international weakenings of Austria as shown by
\href{KBhHIST201GermanicNationalism.org}{KBhHIST201GermanicNationalism},
the \emph{Ausgleich} (compromise) was signed creating a dual-monarchy of
Austria and Hungary: each got own constitution + parliament, but joined
together under the Hasburg crown.

\subsubsection{Creation of Nation-States}
\label{sec:orgd3b256d}
\textbf{Nation-States} (nationalized political identities) emerged in the
sixteenth century => a very slow process

See \href{KBhHIST201NationStates.org}{KBhHIST201NationStates}

\subsubsection{The Isms}
\label{sec:orge0cf34d}
\textbf{Seperatism}

CLAIM: when nationalism arises in multinational empires, national groups
attempt to break away from the larger empire => engendering
\textbf{separatism}. And it is indeed nationalism that caused the Ottoman
Empire to break up

\textbf{Socialism}

Lower-classes harboured socialism => a bottom-up approach of nationalism
where resources are divided evenly.

\textbf{Irrendentism}

Middle and upper class created process that promotes the taking of land
belonging to another state => fostered the creation of states like
Germany and Italy
\end{document}
