% Created 2021-09-27 Mon 12:00
% Intended LaTeX compiler: xelatex
\documentclass[letterpaper]{article}
\usepackage{graphicx}
\usepackage{grffile}
\usepackage{longtable}
\usepackage{wrapfig}
\usepackage{rotating}
\usepackage[normalem]{ulem}
\usepackage{amsmath}
\usepackage{textcomp}
\usepackage{amssymb}
\usepackage{capt-of}
\usepackage{hyperref}
\setlength{\parindent}{0pt}
\usepackage[margin=1in]{geometry}
\usepackage{fontspec}
\usepackage{svg}
\usepackage{cancel}
\usepackage{indentfirst}
\setmainfont[ItalicFont = LiberationSans-Italic, BoldFont = LiberationSans-Bold, BoldItalicFont = LiberationSans-BoldItalic]{LiberationSans}
\newfontfamily\NHLight[ItalicFont = LiberationSansNarrow-Italic, BoldFont       = LiberationSansNarrow-Bold, BoldItalicFont = LiberationSansNarrow-BoldItalic]{LiberationSansNarrow}
\newcommand\textrmlf[1]{{\NHLight#1}}
\newcommand\textitlf[1]{{\NHLight\itshape#1}}
\let\textbflf\textrm
\newcommand\textulf[1]{{\NHLight\bfseries#1}}
\newcommand\textuitlf[1]{{\NHLight\bfseries\itshape#1}}
\usepackage{fancyhdr}
\pagestyle{fancy}
\usepackage{titlesec}
\usepackage{titling}
\makeatletter
\lhead{\textbf{\@title}}
\makeatother
\rhead{\textrmlf{Compiled} \today}
\lfoot{\theauthor\ \textbullet \ \textbf{2021-2022}}
\cfoot{}
\rfoot{\textrmlf{Page} \thepage}
\renewcommand{\tableofcontents}{}
\titleformat{\section} {\Large} {\textrmlf{\thesection} {|}} {0.3em} {\textbf}
\titleformat{\subsection} {\large} {\textrmlf{\thesubsection} {|}} {0.2em} {\textbf}
\titleformat{\subsubsection} {\large} {\textrmlf{\thesubsubsection} {|}} {0.1em} {\textbf}
\setlength{\parskip}{0.45em}
\renewcommand\maketitle{}
\author{Houjun Liu}
\date{\today}
\title{Ottomans in the 1500s}
\hypersetup{
 pdfauthor={Houjun Liu},
 pdftitle={Ottomans in the 1500s},
 pdfkeywords={},
 pdfsubject={},
 pdfcreator={Emacs 28.0.50 (Org mode 9.4.4)}, 
 pdflang={English}}
\begin{document}

\tableofcontents



\section{The Ottomans}
\label{sec:org737e465}
When China was failing
\href{KBhHIST201MingChina1500.org}{KBhHIST201MingChina1500}, the
Ottomans grow w/ Muslimdom --- Ottomans were the largest muslim nation
in Europe + a serious threat to Christendom.

\begin{itemize}
\item Enjoys control of the silk read
\item Huge landmass
\item Large army (and, large cannons + siege trains)

\begin{itemize}
\item Strong Navy! => deployed frequently in the Black Sea,
Constantinople, North Africa
\end{itemize}
\end{itemize}

The Ottoman empire grew from small size. However, with Syria and Egypt
shrinking 16th century, Ottomans became the biggest Muslim empire. Its
less medieval nature represented more the new age, centralized monarchy

\subsection{The Rise}
\label{sec:org11d8c17}
See \href{KBhHIST201OttomansRise.org}{KBhHIST201OttomansRise}

\subsection{The Height}
\label{sec:orgb1d6285}
See \href{KBhHIST201OttomansSuccess.org}{KBhHIST201OttomansSuccess}

\subsection{The Fall}
\label{sec:org460df46}
See
\href{KBhHIST201OttomansFall1500s.org}{KBhHIST201OttomansFall1500s}

\subsection{CN 09162020}
\label{sec:org768e789}
\#disorganized

Gelvin, Chapter 3

\begin{itemize}
\item Weakened governmental systems caused "17 century crisis" => whole of
the world getting Romanitus
\item "Great Inflation" + the "Price Revolution"

\begin{itemize}
\item Governments used bureaucracies to disempower aristocrats

\begin{itemize}
\item \#why not silver based inflation
\end{itemize}

\item Rulers needed ways for rule legitimization => "Routinization of
Chrisma"

\begin{itemize}
\item Finding new means to find new means to find authorities => Could
not use expansion to assert authority anymore
\item Leveraged shiny palaces and overcomplicated rituals
\end{itemize}

\item Claims of why inflation happened:

\begin{itemize}
\item Demographic expansion => Population growth induced

\begin{itemize}
\item More people
\item Post-black-death
\end{itemize}

\item Dependence aforementioned in and of itself was reason for
inflation

\begin{itemize}
\item States spent a large amount of money
\item Kept debasing currencies, causing inflation
\item Resulting in the Vicious Cycle

\begin{itemize}
\item Debasement
\item Governments see debasement, and debase further to recouperate
\end{itemize}
\end{itemize}

\item \href{KBhHIST201ProblemsWithSilver.org}{KBhHIST201ProblemsWithSilver}
Problems with Silver, too!
\end{itemize}
\end{itemize}

\item Transition of "WE" => "MWS"

\begin{itemize}
\item 1500s World Empires” => division of world by huge empires

\begin{itemize}
\item Possible for several WE to exist (Ottoman, Safavid, Habsbug,
China)
\item Spread through millitary conquest or threatening thereof
\item Self-sustaining and independent
\item Each equivalent to the other \#why

\begin{itemize}
\item CLAIM: "no empire was tech. superior to any other. Nor was
empire organized in a manner that gave it particular advantage
over any other."
\end{itemize}
\end{itemize}

\item Late 1500s-now: "world empires =>"modern world system”

\begin{itemize}
\item "Politically fragmented but economically united"
\item Modern world spread influence by bringing outlying districts into
a single economic structure
\item Spread and grown through competition, where there are winners and
loosers
\end{itemize}
\end{itemize}

\item Argument: “how Europe became the core and others became the priperary:
winners are in the middle, and others who got pulled in are the
loosers
\end{itemize}
\end{document}
