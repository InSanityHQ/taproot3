% Created 2021-09-27 Mon 12:00
% Intended LaTeX compiler: xelatex
\documentclass[letterpaper]{article}
\usepackage{graphicx}
\usepackage{grffile}
\usepackage{longtable}
\usepackage{wrapfig}
\usepackage{rotating}
\usepackage[normalem]{ulem}
\usepackage{amsmath}
\usepackage{textcomp}
\usepackage{amssymb}
\usepackage{capt-of}
\usepackage{hyperref}
\setlength{\parindent}{0pt}
\usepackage[margin=1in]{geometry}
\usepackage{fontspec}
\usepackage{svg}
\usepackage{cancel}
\usepackage{indentfirst}
\setmainfont[ItalicFont = LiberationSans-Italic, BoldFont = LiberationSans-Bold, BoldItalicFont = LiberationSans-BoldItalic]{LiberationSans}
\newfontfamily\NHLight[ItalicFont = LiberationSansNarrow-Italic, BoldFont       = LiberationSansNarrow-Bold, BoldItalicFont = LiberationSansNarrow-BoldItalic]{LiberationSansNarrow}
\newcommand\textrmlf[1]{{\NHLight#1}}
\newcommand\textitlf[1]{{\NHLight\itshape#1}}
\let\textbflf\textrm
\newcommand\textulf[1]{{\NHLight\bfseries#1}}
\newcommand\textuitlf[1]{{\NHLight\bfseries\itshape#1}}
\usepackage{fancyhdr}
\pagestyle{fancy}
\usepackage{titlesec}
\usepackage{titling}
\makeatletter
\lhead{\textbf{\@title}}
\makeatother
\rhead{\textrmlf{Compiled} \today}
\lfoot{\theauthor\ \textbullet \ \textbf{2021-2022}}
\cfoot{}
\rfoot{\textrmlf{Page} \thepage}
\renewcommand{\tableofcontents}{}
\titleformat{\section} {\Large} {\textrmlf{\thesection} {|}} {0.3em} {\textbf}
\titleformat{\subsection} {\large} {\textrmlf{\thesubsection} {|}} {0.2em} {\textbf}
\titleformat{\subsubsection} {\large} {\textrmlf{\thesubsubsection} {|}} {0.1em} {\textbf}
\setlength{\parskip}{0.45em}
\renewcommand\maketitle{}
\author{Houjun Liu}
\date{\today}
\title{Europe Political Fragmentation}
\hypersetup{
 pdfauthor={Houjun Liu},
 pdftitle={Europe Political Fragmentation},
 pdfkeywords={},
 pdfsubject={},
 pdfcreator={Emacs 28.0.50 (Org mode 9.4.4)}, 
 pdflang={English}}
\begin{document}

\tableofcontents



\section{Political Fragmentation of Europe}
\label{sec:orgce6aa4e}
Europe is politically fragmented, but Mr. Kennedy thinks this is mucho
bueno

\subsection{Causes}
\label{sec:orgb235df7}
\begin{itemize}
\item Geographical --- no big plains but lots of rivers
\item Harsher climate makes difficult central control
\end{itemize}

\subsection{Effects}
\label{sec:org6a0ce7a}
\begin{itemize}
\item Hard to be fully unified, and hence hard to be assimilated
\item Diversity encouraged inter-dependence and trade (which, \#why is this a
good thing necessarily?)

\begin{itemize}
\item Trade included \emph{bulk} items primarily and not luxury-focused trade
of the east
\item Easy access to sea encouraged shipbuilding
\item Quick, common day trade encouraged building on an economy ---
credit, banks, common currency
\item Need for long-range fishing encourages the building of bigger ships,
and that's good because \#why?
\end{itemize}
\end{itemize}

\subsubsection{Development of strong military technology}
\label{sec:orgb9bd60e}
\begin{itemize}
\item Beginning around 14-1600s with the establishment of "gunpowder
empires"
\item Military powers begin concentrating

\begin{itemize}
\item Italy's use of the crossbowmen + pikes
\item France and England gained monopoly with artilirary
\end{itemize}

\item \textbf{Notice!} However, that there is nevertheless a \emph{variety} of powers
that made individual control difficult --- no single individual ever
gained an edge
\item Varied political entities caused difficulty in controlling the whole
continent
\item Lean army of smaller nations encouraged fighting with artillery, and
not direct combat --- driving technological innovation
\end{itemize}

\begin{quote}
To most European statesmen the loss of Hungary was of far greater
import than the establishment of factories in the Orient, and the
threat to Vienna more significant than their own challenges at Aden,
Goa and Malacca; only goverments bordering the Atlantic could, like
the later historians, ignore this fact
\end{quote}

Meaning\ldots{} Generally weaker individual states originally caused less
inter-continental fighting against the status-quo big nations
(\href{KBhHIST201MingChina1500.org}{KBhHIST201MingChina1500}/\href{KBhHIST201Ottomans1500.org}{KBhHIST201Ottomans1500})

\subsection{Advantages}
\label{sec:org5e8a995}
Fragmentation also encourage competition:

\begin{itemize}
\item Trading and merchants less stigmatized
(unlike\href{KBhHIST201ChinasDecline1500.org}{KBhHIST201ChinasDecline1500}
China's Governmental Decline in the 1500s) => making supressing
economic development difficult \#why
\item Encouraged small millitia (\emph{condottieri}) to compete for contracts
\item Created genuine innovation

\begin{itemize}
\item Experimentation with gunpowder
\item Innovation by hired army
\end{itemize}

\item Self-perpetuating cycle made attack and monopolization harder\\
\item Innovative techniques often drive other innovation (experiments with
cannons => more sturdy ships)
\end{itemize}
\end{document}
