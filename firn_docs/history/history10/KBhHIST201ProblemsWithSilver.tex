% Created 2021-09-27 Mon 12:00
% Intended LaTeX compiler: xelatex
\documentclass[letterpaper]{article}
\usepackage{graphicx}
\usepackage{grffile}
\usepackage{longtable}
\usepackage{wrapfig}
\usepackage{rotating}
\usepackage[normalem]{ulem}
\usepackage{amsmath}
\usepackage{textcomp}
\usepackage{amssymb}
\usepackage{capt-of}
\usepackage{hyperref}
\setlength{\parindent}{0pt}
\usepackage[margin=1in]{geometry}
\usepackage{fontspec}
\usepackage{svg}
\usepackage{cancel}
\usepackage{indentfirst}
\setmainfont[ItalicFont = LiberationSans-Italic, BoldFont = LiberationSans-Bold, BoldItalicFont = LiberationSans-BoldItalic]{LiberationSans}
\newfontfamily\NHLight[ItalicFont = LiberationSansNarrow-Italic, BoldFont       = LiberationSansNarrow-Bold, BoldItalicFont = LiberationSansNarrow-BoldItalic]{LiberationSansNarrow}
\newcommand\textrmlf[1]{{\NHLight#1}}
\newcommand\textitlf[1]{{\NHLight\itshape#1}}
\let\textbflf\textrm
\newcommand\textulf[1]{{\NHLight\bfseries#1}}
\newcommand\textuitlf[1]{{\NHLight\bfseries\itshape#1}}
\usepackage{fancyhdr}
\pagestyle{fancy}
\usepackage{titlesec}
\usepackage{titling}
\makeatletter
\lhead{\textbf{\@title}}
\makeatother
\rhead{\textrmlf{Compiled} \today}
\lfoot{\theauthor\ \textbullet \ \textbf{2021-2022}}
\cfoot{}
\rfoot{\textrmlf{Page} \thepage}
\renewcommand{\tableofcontents}{}
\titleformat{\section} {\Large} {\textrmlf{\thesection} {|}} {0.3em} {\textbf}
\titleformat{\subsection} {\large} {\textrmlf{\thesubsection} {|}} {0.2em} {\textbf}
\titleformat{\subsubsection} {\large} {\textrmlf{\thesubsubsection} {|}} {0.1em} {\textbf}
\setlength{\parskip}{0.45em}
\renewcommand\maketitle{}
\date{\today}
\title{Problems with Silver}
\hypersetup{
 pdfauthor={},
 pdftitle={Problems with Silver},
 pdfkeywords={},
 pdfsubject={},
 pdfcreator={Emacs 28.0.50 (Org mode 9.4.4)}, 
 pdflang={English}}
\begin{document}

\tableofcontents

Silver, given it's original value, is a highly valuable metal. However,
the wonderféul "American Silver", in Mann's words

\begin{quote}
threw the Spanish elite into delirium. Enraptured by the sudden wealth
and power, the monarchy launched a series of costly foreign wars.
\end{quote}

which, obviously, was not great for the Spanish. In addition, because
there was really simply so much silver, the Spanish king had to borrow
more and more money to be able to sustain the war as Silver prices kept
dropping.

From this sudden decrease in price, other things kept happening:

\begin{enumerate}
\item "The cost of living doubled in Europe in the last half of the 16th
century."
\item Wages did not keep pace with the boom, so people started mutinies
\end{enumerate}

Oh, right, this decrease in the price of silver also coincided with
China's increased demand for silver, leading to
\href{KBhHIST201ChinasDeclineWRTCurrency.org}{KBhHIST201ChinasDeclineWRTCurrency}
China's Decline w.r.t. Currency.
\end{document}
