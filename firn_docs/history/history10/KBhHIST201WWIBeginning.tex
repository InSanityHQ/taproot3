% Created 2021-09-27 Mon 12:00
% Intended LaTeX compiler: xelatex
\documentclass[letterpaper]{article}
\usepackage{graphicx}
\usepackage{grffile}
\usepackage{longtable}
\usepackage{wrapfig}
\usepackage{rotating}
\usepackage[normalem]{ulem}
\usepackage{amsmath}
\usepackage{textcomp}
\usepackage{amssymb}
\usepackage{capt-of}
\usepackage{hyperref}
\setlength{\parindent}{0pt}
\usepackage[margin=1in]{geometry}
\usepackage{fontspec}
\usepackage{svg}
\usepackage{cancel}
\usepackage{indentfirst}
\setmainfont[ItalicFont = LiberationSans-Italic, BoldFont = LiberationSans-Bold, BoldItalicFont = LiberationSans-BoldItalic]{LiberationSans}
\newfontfamily\NHLight[ItalicFont = LiberationSansNarrow-Italic, BoldFont       = LiberationSansNarrow-Bold, BoldItalicFont = LiberationSansNarrow-BoldItalic]{LiberationSansNarrow}
\newcommand\textrmlf[1]{{\NHLight#1}}
\newcommand\textitlf[1]{{\NHLight\itshape#1}}
\let\textbflf\textrm
\newcommand\textulf[1]{{\NHLight\bfseries#1}}
\newcommand\textuitlf[1]{{\NHLight\bfseries\itshape#1}}
\usepackage{fancyhdr}
\pagestyle{fancy}
\usepackage{titlesec}
\usepackage{titling}
\makeatletter
\lhead{\textbf{\@title}}
\makeatother
\rhead{\textrmlf{Compiled} \today}
\lfoot{\theauthor\ \textbullet \ \textbf{2021-2022}}
\cfoot{}
\rfoot{\textrmlf{Page} \thepage}
\renewcommand{\tableofcontents}{}
\titleformat{\section} {\Large} {\textrmlf{\thesection} {|}} {0.3em} {\textbf}
\titleformat{\subsection} {\large} {\textrmlf{\thesubsection} {|}} {0.2em} {\textbf}
\titleformat{\subsubsection} {\large} {\textrmlf{\thesubsubsection} {|}} {0.1em} {\textbf}
\setlength{\parskip}{0.45em}
\renewcommand\maketitle{}
\author{Houjun Liu}
\date{\today}
\title{European Tensions leading to WWI}
\hypersetup{
 pdfauthor={Houjun Liu},
 pdftitle={European Tensions leading to WWI},
 pdfkeywords={},
 pdfsubject={},
 pdfcreator={Emacs 28.0.50 (Org mode 9.4.4)}, 
 pdflang={English}}
\begin{document}

\tableofcontents



\section{Causes of WWI: European Tensions Building}
\label{sec:org33eb4e7}
\subsection{Europeans' Blindness}
\label{sec:orgd5d1997}
Claim @\href{KBhHIST201PalmerCh17.org}{KBhHIST201PalmerCh17}

Throughout the early 19th century, Europeans believed that their
technological innovations would defuse conflicts and bring overall
happiness due to the fear of fighting. Owning to this, they maintained
huge standing armies due to fears of internal conflicts and to keep
peace.

These are quite powerful armies --- a la\ldots{} Britian, for instance --
which had one of if not \emph{the} largest navy on the planet at the time.

\subsection{German Growth}
\label{sec:org097864b}
German unity brought boom in industry in especially the Germanic states
but also all of Europe => Germans wanted street cred ("a place in the
sun") for all their industrialization. Their actions directly added
pressure to the cause of WWI.

See
\href{KBhHIST201WWIStartWRTGermany.org}{KBhHIST201WWIStartWRTGermany}

At this point! Two camps began to form: the \textbf{German-Austrian-Italian}
alliance vs. the \textbf{Frenco-Russian} alliance: each having touch feelings
about the other.

\subsection{(More) Alliances Forming}
\label{sec:org1fa1f34}
Sprints of cross-floor coopreation did occur but unstable and rare.

*Everyone was looking at Britian because what they did determined how
the BOP will be constructed/disrupted. They prided themselves on
isolation and the lack of alliance, but when Germany supported Boers in
Boers vs. Britian, conflict began to brew.*

CLAIM (@\href{KBhHIST201PalmerCh17.org}{KBhHIST201PalmerCh17})
Britian's move here is critical and probably caused a war. Because
germany supported against Britian, they triggered Britian to become
allianced with the others against it.

\subsection{The Germans built a navy.}
\label{sec:orgb9b0925}
Of course, pissing Britian off is not a very good thing. Namely, you
probably want a good navy so that you could at least fight a little. So
Germans did just that. See
\href{KBhHIST201WWIStartWRTGermany.org}{KBhHIST201WWIStartWRTGermany}

\textbf{At this point, the three nations (E, F, R) of the "Triple Entente" were
acting together against the "Triple Alliance" (G, I, A).}

\subsection{Testing the Relationship}
\label{sec:org8a47bee}
Kaiser William II attempted to test the relationship between France and
England (the Entente), so Germany did a few things.

\begin{enumerate}
\item Kaiser William making a speech directly against the French claim in
Morocco. Britian backend the French side, and because of this tactic,
the two strengthed relationship in fear of the common Germanic enemy.

\item A gunboat appeared in Agadir and demanded that the French give up
some of the congo. British cabinet David Lloyd George protested this.
\end{enumerate}
\end{document}
