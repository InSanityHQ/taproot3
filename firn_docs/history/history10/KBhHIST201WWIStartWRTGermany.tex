% Created 2021-09-27 Mon 12:00
% Intended LaTeX compiler: xelatex
\documentclass[letterpaper]{article}
\usepackage{graphicx}
\usepackage{grffile}
\usepackage{longtable}
\usepackage{wrapfig}
\usepackage{rotating}
\usepackage[normalem]{ulem}
\usepackage{amsmath}
\usepackage{textcomp}
\usepackage{amssymb}
\usepackage{capt-of}
\usepackage{hyperref}
\setlength{\parindent}{0pt}
\usepackage[margin=1in]{geometry}
\usepackage{fontspec}
\usepackage{svg}
\usepackage{cancel}
\usepackage{indentfirst}
\setmainfont[ItalicFont = LiberationSans-Italic, BoldFont = LiberationSans-Bold, BoldItalicFont = LiberationSans-BoldItalic]{LiberationSans}
\newfontfamily\NHLight[ItalicFont = LiberationSansNarrow-Italic, BoldFont       = LiberationSansNarrow-Bold, BoldItalicFont = LiberationSansNarrow-BoldItalic]{LiberationSansNarrow}
\newcommand\textrmlf[1]{{\NHLight#1}}
\newcommand\textitlf[1]{{\NHLight\itshape#1}}
\let\textbflf\textrm
\newcommand\textulf[1]{{\NHLight\bfseries#1}}
\newcommand\textuitlf[1]{{\NHLight\bfseries\itshape#1}}
\usepackage{fancyhdr}
\pagestyle{fancy}
\usepackage{titlesec}
\usepackage{titling}
\makeatletter
\lhead{\textbf{\@title}}
\makeatother
\rhead{\textrmlf{Compiled} \today}
\lfoot{\theauthor\ \textbullet \ \textbf{2021-2022}}
\cfoot{}
\rfoot{\textrmlf{Page} \thepage}
\renewcommand{\tableofcontents}{}
\titleformat{\section} {\Large} {\textrmlf{\thesection} {|}} {0.3em} {\textbf}
\titleformat{\subsection} {\large} {\textrmlf{\thesubsection} {|}} {0.2em} {\textbf}
\titleformat{\subsubsection} {\large} {\textrmlf{\thesubsubsection} {|}} {0.1em} {\textbf}
\setlength{\parskip}{0.45em}
\renewcommand\maketitle{}
\author{Houjun Liu}
\date{\today}
\title{Start of WWI w.r.t. Germany}
\hypersetup{
 pdfauthor={Houjun Liu},
 pdftitle={Start of WWI w.r.t. Germany},
 pdfkeywords={},
 pdfsubject={},
 pdfcreator={Emacs 28.0.50 (Org mode 9.4.4)}, 
 pdflang={English}}
\begin{document}

\tableofcontents



\section{Germany's Provokations in WWI}
\label{sec:org42336fc}
Germany's quick success is a little scary for folks because they are
used to standard European power structures; this is also scary for
Bismark (Chancellor of Germany) because he wanted to keep peace.

\subsection{Signing of Allegiances}
\label{sec:org112d03a}
\begin{itemize}
\item Otto von Bismark feared that the new Germany could be broken by war,
so tried to keep peace.

\begin{itemize}
\item Bismark signed two sets of treaties. The first forming the \textbf{Triple
Alliance}, which lead to the alliance of AustriaHungary, Italy, and
Germany.
\item Also signed a second reverse insurance treaty with Russia, an enemy
of the first.
\item The reverse insurance alliance with Russia was way to finagly, which
lead to it collapsing under Bismark's successors and eventually lead
to France signing a treaty with Russia.
\end{itemize}
\end{itemize}

\subsection{Building of Navies}
\label{sec:org9e24cb9}
Reasons? see \href{KBhHIST201WWIBeginning.org}{KBhHIST201WWIBeginning}

\begin{enumerate}
\item By 1898-1912, the Germans rapidly grew their naval power ---
originally an important source of British power. The Brits feel
threatened.
\item The British policy of always having the biggest navy lead them to
enter an arms race vs. Germany.
\item The British government slowly began a series of diplomatic actions
that resulted in a readjustment of relationships with the French, a
convention with Russia, and eventually leading the three to form an
unofficial alliance + "act together". CLAIM: this perhaps is the
immediate cause of the war.
\end{enumerate}
\end{document}
