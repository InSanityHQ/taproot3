% Created 2021-09-27 Mon 12:00
% Intended LaTeX compiler: xelatex
\documentclass[letterpaper]{article}
\usepackage{graphicx}
\usepackage{grffile}
\usepackage{longtable}
\usepackage{wrapfig}
\usepackage{rotating}
\usepackage[normalem]{ulem}
\usepackage{amsmath}
\usepackage{textcomp}
\usepackage{amssymb}
\usepackage{capt-of}
\usepackage{hyperref}
\setlength{\parindent}{0pt}
\usepackage[margin=1in]{geometry}
\usepackage{fontspec}
\usepackage{svg}
\usepackage{cancel}
\usepackage{indentfirst}
\setmainfont[ItalicFont = LiberationSans-Italic, BoldFont = LiberationSans-Bold, BoldItalicFont = LiberationSans-BoldItalic]{LiberationSans}
\newfontfamily\NHLight[ItalicFont = LiberationSansNarrow-Italic, BoldFont       = LiberationSansNarrow-Bold, BoldItalicFont = LiberationSansNarrow-BoldItalic]{LiberationSansNarrow}
\newcommand\textrmlf[1]{{\NHLight#1}}
\newcommand\textitlf[1]{{\NHLight\itshape#1}}
\let\textbflf\textrm
\newcommand\textulf[1]{{\NHLight\bfseries#1}}
\newcommand\textuitlf[1]{{\NHLight\bfseries\itshape#1}}
\usepackage{fancyhdr}
\pagestyle{fancy}
\usepackage{titlesec}
\usepackage{titling}
\makeatletter
\lhead{\textbf{\@title}}
\makeatother
\rhead{\textrmlf{Compiled} \today}
\lfoot{\theauthor\ \textbullet \ \textbf{2021-2022}}
\cfoot{}
\rfoot{\textrmlf{Page} \thepage}
\renewcommand{\tableofcontents}{}
\titleformat{\section} {\Large} {\textrmlf{\thesection} {|}} {0.3em} {\textbf}
\titleformat{\subsection} {\large} {\textrmlf{\thesubsection} {|}} {0.2em} {\textbf}
\titleformat{\subsubsection} {\large} {\textrmlf{\thesubsubsection} {|}} {0.1em} {\textbf}
\setlength{\parskip}{0.45em}
\renewcommand\maketitle{}
\author{Huxley}
\date{\today}
\title{Cold War in the Developing World Planning}
\hypersetup{
 pdfauthor={Huxley},
 pdftitle={Cold War in the Developing World Planning},
 pdfkeywords={},
 pdfsubject={},
 pdfcreator={Emacs 28.0.50 (Org mode 9.4.4)}, 
 pdflang={English}}
\begin{document}

\tableofcontents

\#flo \#ref \#disorganized \#incomplete

\noindent\rule{\textwidth}{0.5pt}

asssignment:
\url{https://docs.google.com/document/d/18pAQO6g8R1x7d8ufiJEgW8v3zi-qmJeRJX5nueh9W\_Y/edit}

imagine links

third synth third historyagraphy third lit review

\url{https://www.jstor.org/stable/2936421?seq=1}

title?

Powers Edge: Ideological Manipulation in Foreign Countries

instating leaders in foreign coutries -> dictators that we cant do
anything about dictators because fighting against the wants of the
people

\section{Note-taking, begin!}
\label{sec:org6a84f19}
\begin{itemize}
\item America overthrew Sukarno in 1965, and replaced him with the right
wing Suharto who was big bad news

\item evidence that america was involed?

\item america had already participated in a revolt against sukar in 1958

\item america congragulated suharto and themselves when he smashed the
commies

\item this guy argues that america didnt play a large role

\item america prodded the army to take down sukaro but it didnt work

\begin{itemize}
\item they had "bassically given up"
\end{itemize}

\item Amercia gave up, the coup caught the CIA by surprise

\item lyndon b johson and w post watergate means we get info

\item cia was on a short leash, if they had plans to move against, then
washintion would know

\begin{itemize}
\item he didnt, and thus, CIA was not involved.
\item cia admitted that they didnt have a connection
\end{itemize}

\item indeonesia was important because:

\begin{itemize}
\item \begin{quote}
the sprawling archipelago bestrode the sea lanes connecting the
Indian Ocean to the Pacific;
\end{quote}

\item outflanked vietnam

\item \begin{quote}
it threatened havoc on the security of the Philippines and the
American bases there
\end{quote}

\item had a wealth of natural resources
\end{itemize}

\item sukarno was super charismatic, let him rise to the top

\begin{itemize}
\item survived a coup, and a buncha other things
\end{itemize}

\item america doesnt want to push sukarno into greater reliance on the
Chinese and the Russions (commies)

\item sukarno had a phycological war against malasia

\item CIA thought of sukarno more of a speaker rather than a thinker

\begin{itemize}
\item \begin{quote}
more improviser than ideologue
\end{quote}
\end{itemize}

\item sukarno needed the PKI, the PKI needed sukarno

\item sukarno has the same deal with the army, who also ballanced the PKI

\item american aid had to be low due to pressure from congress, but they
tried to do enough to keep sukarno from behaving even more badly

\item \begin{quote}
"When any nation offers us aid with political strings attached, then
I will tell them," here Sukarno switched to English for emphasis,
“'Go to hell with your aid!“'
\end{quote}

\begin{itemize}
\item people didnt like that.
\end{itemize}

\item cia thought that sukarno was becoming "mentally unhinged"

\item sukarno made another speech that was very clearly anti american

\item rusk said that internal anti comunist action could not develop
independent of sukarno

\item and that they should think of it as a communist dictatorship

\item sukarno declined americans a bunch (pg 16)

\item ball said:

\begin{itemize}
\item \begin{quote}
so long as sukarno is alve, the prospects of an army revolt
against him are slim
\end{quote}
\end{itemize}

\item their were rumors of sukarno being ill, tons of misinformation
\end{itemize}

idk\ldots{}

\begin{itemize}
\item america likes to take credit when things go right?

\item 
\end{itemize}

\section{planin}
\label{sec:org9313a84}
big (small) idea: we live in contexts, america provided the context for
suharto to do the coup

trying to fit the situation into the binary of blame is not nueanced
enough wheeee

\begin{enumerate}
\item \textbf{Title (Something catchy: short summary of topic)}

\begin{enumerate}
\item TBD
\end{enumerate}

\item \textbf{Lede/hook (\textasciitilde{}1 paragraph)}

\begin{enumerate}
\item description of the coup?

\begin{enumerate}
\item describe in way that makes it seem like america did it.
\end{enumerate}
\end{enumerate}

\item \textbf{Context for the case study (\textasciitilde{}1-2 paragraphs)}

\begin{enumerate}
\item talk about inddonesia and it's positioning, what it meant

\begin{enumerate}
\item indonesia had large strategic signifigance:

\begin{enumerate}
\item \begin{quote}
right in the middle of the sea lanes connecting "the
Indian Ocean to the Pacific; it outflanked Vietnam; it
threatened havoc on the security of the Philippines and
the American bases there. Economically the country's clout
rested on a wealth of natural resources, and although no
one could say just how abundant they might be, American
oil com- panies had bet some half-billion dollars on
Indonesia's future.6 Politically Sukarno had early claimed
stature among Third World neutralists by hosting the
seminal Bandung conference of 1955; nearly a decade later
he remained a heavyweight in the nonaligned movement."
\end{quote}
\end{enumerate}
\end{enumerate}
\end{enumerate}

\item \textbf{Source analysis (\textasciitilde{}1-2 paragraphs)}

\begin{enumerate}
\item summary?

\begin{enumerate}
\item Pick at least one reading and provide a brief review of the
article's main contributions. To do so, identify both the
purpose and the argument of the piece. What justifications does
the author provide about the topic selection and the
significance of the arguments?

\begin{enumerate}
\item \begin{quote}
This brings us back to the most serious deficiency of
covert warfare: its inevitable tendency to poison the well
of public opinion. Each actual operation discovered gives
rise to a hundred imagined, which for all their
illusiveness are no less con- vincing to an international
audience and damaging to the United States.
\end{quote}

\item Covert operations and secrecy -> intense speculation,
damages public opinion, truth is important. this is the
truth.

\item Argument: Sukarno's overthrow was not done by americans
\end{enumerate}

\item Here would also be the place to provide a brief summary of the
contents of the article: what specifics does the article focus
on and how do those choices help to advance the author's
argument?

\begin{enumerate}
\item America tried, failed. Gave up.
\item we had info, we didnt know. caught us by surprise.
\item america had incentive, made sense,
\item but we ultimently didnt do it.
\end{enumerate}
\end{enumerate}
\end{enumerate}

\item \textbf{"Literature review" (\textasciitilde{}1-2 pagagraphs)}

\begin{enumerate}
\item How does your secondary source engage with the theme of
triangularity and the global cold war?

\begin{enumerate}
\item it acknolwges the global cold war, and explores an edge case

\begin{enumerate}
\item intellectual expanse, how truly global the cold war was.
\end{enumerate}
\end{enumerate}

\item Here you should connect your reading to the larger field. What is
distinct about their approach and how effective is that approach
at capturing the triangular relationship of the global cold war?

\begin{enumerate}
\item another attempt to make sense and find truth from the knowledge
pool
\item another aspect of the global cold war: where it isnt, where
it's imagined to be
\end{enumerate}
\end{enumerate}

\item \textbf{Evaluation: How did the Cold War system shape development and/or
decolonization in your region?}

\begin{enumerate}
\item connection to Munro:

\begin{enumerate}
\item \begin{quote}
"The so-called Cold War\ldots{}was far less the confrontation of
the United States with Russia than America's expansion into
the entire world."
\end{quote}

\item the fact that this event was thought to be done by the
americans demstrates the expanse
\end{enumerate}

\item context creation, blame, ect
\end{enumerate}
\end{enumerate}

\section{writing? here we go.}
\label{sec:orge969ed6}
September, 1957: Eisenhower gives the command, and the CIA begins
planning its coup to overthrow Sukarno. 1958, it fails, forcing Sukarno
into greater reliance on the PKI --- the Communist party of Indonesia.
Sukarno gives his famous speech, declaring loudly to the United States:
"When any nation offers us aid with political strings attached, then I
will tell them, go to hell with your aid!" America's influence over
Sukarno had been admitted and destroyed in one fell swoop. A year later,
Sukarno becomes the first to drop out of the United Nations.
\url{https://journals.sagepub.com/doi/abs/10.1177/002070206502000205?journalCode=ijxa}
Sukarno then gives another speech, recorded by Francis Galbraith, the
American embassy's Chief of Mission. Galbraith wrote, "Sukarno declares
Indonesia in the camp of the Asian Communist countries and opposed to
the U.S." He continued, "It would be fatuous to pretend that this speech
is other principally than a declaration of enmity for us." September 30,
1965: Another coup takes place, six top army generals are seized and
murdered. Suharto, a far right-wing general, takes Sukarno's power.
Followed by this transfer was the massacring of Indonesian communists
--- and alleged communists --- that left the country in shock. Body
count estimates range from hundreds of thousands to over a million. The
United States congratulated Suharto and supported him as he then ruled
as dictator. H. W. Brands, the author of \emph{The Limits of Manipulation:
How the United States Didn't Topple Sukarno}, claims that America had
virtually no involvement, and the transfer of power from Sukarno to
Suharto "resulted instead from developments of essentially Indonesian
origin."

it's importance in the Cold War.

To understand the events that took place in Indonesia, one must first
understand Indonesia's surrounding context. The initial coup took place
just a few years after Eisenhower's election. One year after the
election, the CIA successfully overthrew Mohammed Mossadegh, the Iranian
Prime Minister at the time. Another year later, the CIA toppled
Guatemalan President Jacobo Árbenz, replacing him with Carlos Castillo
Armas who instated a military dictatorship. It was a time of covert
operations and government manipulation in developing countries. The
second coup, leading to the transfer of power from Sukarno to Suharto
took place just at the start of the Cold War's height. Chronologically
in the middle of the Vietnam war, the United States feared Sukarno would
create a similar situation to South Vietnam. \{CITES!\}

Indonesia itself had incredible value to the United States during the
cold war. It's unique location contributed greatly to this value;
Indonesia spanned the sea lanes linking the Pacific and Indian Oceans,
providing a crucial choke-point. It also outflanked Vietnam, providing
crucial positioning to the ongoing war.
\url{https://www.jstor.org/stable/resrep05921.11?seq=1\#metadata\_info\_tab\_contents}
Complimenting its locational and strategical advantages, Indonesia was
rich with natural resources. Along with an abundance of oil and natural
gas, it carried a slew of valuable metals. \{\}
\url{https://www.worldatlas.com/articles/what-are-the-major-natural-resources-of-indonesia.html}

However, H. W. Brands argues that America had virtually no involvement
with the coup that took down Sukarno and replaced him with Suharto.
Brands first states why the knowledge surrounding the event is available
to the public: Lyndon B. Johnson had an "insatiable appetite for
information," leading to much of the CIA's activity being
documented.(CITE) After the Bay of Pigs invasion, the CIA was put on a
far shorter leash. The documents on Indonesia later got revealed to the
public when they came up for review in the Watergate scandal. Brands
then goes on to describe America's dealings with Sukarno. The CIA
continually prodded the Indonesian army to move against Sukarno, and
after many months, "the Johnson administration, at a loss as to what
else it might do, had basically given up." (CITE) The United States then
decided to try and solidify relations with Sukarno, repeatedly meeting
and offering aid. Every time, Sukarno would send "the Americans home
empty-handed." (cite). Giving up again, the United States attempted to
resolve conflict in the Indonesian-Malaysian crisis and give aid only to
keep Sukarno from --- as Dean Rusk the Secretary of State at the time
put it --- "behaving even more badly than he has."(cite). After repeated
failure, the United States almost entirely withdrew from Indonesia.
Ellsworth Bunker, a top diplomat reported to Washington that "Indonesia
essentially will have to save itself."

Brands concludes with his thoughts on covert operations as a whole. He
acknowledges the credibility of America's involvement, stating that the
whole event would be "a nonstory except that the myth of American
responsibility has proved so hardy." (cite). By trying to get closer to
the truth, Brands hopes to dispel rumors surrounding the event, as "each
actual operation discovered gives rise to a hundred imagined." (cite).
Listing the many deficits of covert operations, Brands claims that the
"most serious deficiency of covert warfare" is its "inevitable tendency
to poison the well of public opinion."(cite). Rumors of covert
operations, in the minds of the public, are nearly impossible to remove
--- damaging the United States. Brands ends by stating: "while the
United States government remains in the business of covert warfare, the
list of indictments will continue to grow."

Brands grapples with the truly global nature of the Cold War by
exploring where the Cold war is not. Even in a scenario where the United
States did not meddle, it is still perceived as the puppet master
orchestrating major changes. The extent of the Cold War reaches farther
than reality --- the Cold War was so truly global that it is now assumed
to be the cause of major events like the ones Brands discusses. Brands
tries to unveil the truth by analyzing and compiling multiple primary
sources, disagreeing with the common assumptions. But as Brands
mentions, the nature of these assumptions --- rooted in the "top secret"
--- are very hard to dismantle. Lack of evidence simply increases
speculation. He demonstrates a way in which the modern day manifestation
of the Cold War, despite it being over, is global in its perception.

While very insightful, Brands tries to fit the events of Indonesia into
the binary of blame. After a detailed examination spanning roughly 25
pages, Brands concludes that "The United States did not overthrow
Sukarno, and it was not responsible for the hundreds of thousands of
deaths involved in the liquidation of the PKI." (cite). While
overthrowing Sukarno and being responsible for the deaths of those in
the PKI are separate, Brands implies that the former caused the latter.
While perhaps not involved directly with the coup that transferred power
from Sukarno to Suharto, the United States created a context where it
was easily assumed that the CIA was behind said coup. Suharto must have
known that after the coup, the United States would be on his side. It
was reasonable for Suharto to expect support for him and his
anti-communist agenda - much as the United States had done with so many
other power transfers. And the United States did in fact support him ---
publicly. While perhaps not directly involved, America created a context
which created a much greater incentive for the coup. This perspective
brings us back to Brands' binary of blame: was the United States
responsible or not responsible for the events in Indonesia? The truth is
much more nuanced; while the battles of the Cold War reached far, the
contexts it created and the events that sprung from those contexts
reached farther. Brands titles his paper: \textbf{The Limits of Manipulation:
How the United States Didn't Topple Sukarno}. While the United States
may not have been directly behind the coup, can it really be said that
they didn't manipulate the situation in Indonesia?
\end{document}
