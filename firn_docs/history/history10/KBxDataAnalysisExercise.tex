% Created 2021-09-27 Mon 12:00
% Intended LaTeX compiler: xelatex
\documentclass[letterpaper]{article}
\usepackage{graphicx}
\usepackage{grffile}
\usepackage{longtable}
\usepackage{wrapfig}
\usepackage{rotating}
\usepackage[normalem]{ulem}
\usepackage{amsmath}
\usepackage{textcomp}
\usepackage{amssymb}
\usepackage{capt-of}
\usepackage{hyperref}
\setlength{\parindent}{0pt}
\usepackage[margin=1in]{geometry}
\usepackage{fontspec}
\usepackage{svg}
\usepackage{cancel}
\usepackage{indentfirst}
\setmainfont[ItalicFont = LiberationSans-Italic, BoldFont = LiberationSans-Bold, BoldItalicFont = LiberationSans-BoldItalic]{LiberationSans}
\newfontfamily\NHLight[ItalicFont = LiberationSansNarrow-Italic, BoldFont       = LiberationSansNarrow-Bold, BoldItalicFont = LiberationSansNarrow-BoldItalic]{LiberationSansNarrow}
\newcommand\textrmlf[1]{{\NHLight#1}}
\newcommand\textitlf[1]{{\NHLight\itshape#1}}
\let\textbflf\textrm
\newcommand\textulf[1]{{\NHLight\bfseries#1}}
\newcommand\textuitlf[1]{{\NHLight\bfseries\itshape#1}}
\usepackage{fancyhdr}
\pagestyle{fancy}
\usepackage{titlesec}
\usepackage{titling}
\makeatletter
\lhead{\textbf{\@title}}
\makeatother
\rhead{\textrmlf{Compiled} \today}
\lfoot{\theauthor\ \textbullet \ \textbf{2021-2022}}
\cfoot{}
\rfoot{\textrmlf{Page} \thepage}
\renewcommand{\tableofcontents}{}
\titleformat{\section} {\Large} {\textrmlf{\thesection} {|}} {0.3em} {\textbf}
\titleformat{\subsection} {\large} {\textrmlf{\thesubsection} {|}} {0.2em} {\textbf}
\titleformat{\subsubsection} {\large} {\textrmlf{\thesubsubsection} {|}} {0.1em} {\textbf}
\setlength{\parskip}{0.45em}
\renewcommand\maketitle{}
\author{Huxley}
\date{\today}
\title{Data Analysis Exercise}
\hypersetup{
 pdfauthor={Huxley},
 pdftitle={Data Analysis Exercise},
 pdfkeywords={},
 pdfsubject={},
 pdfcreator={Emacs 28.0.50 (Org mode 9.4.4)}, 
 pdflang={English}}
\begin{document}

\tableofcontents

\#ref \#ret

\noindent\rule{\textwidth}{0.5pt}

\%\%For \textbf{each} of the four scenarios below, answer the following
questions. Please explain where your targets/rewards would come from
(\#2), how you would make your inputs numerical (\#3), and a bit of your
reasoning on ethical issues (\#6). Other questions do not need
explanation.

\begin{enumerate}
\item What type of machine learning problem (regression, classification,
clustering, generation, reinforcement learning) do you think this is?
\item If this is a supervised problem, what will you use as your targets
(aka labels)? If it is reinforcement learning, what will you use as
your rewards? If it is unsupervised, just say "unsupervised".
\item What processing do you need to do to your input data?
\item What type(s) of model(s) would you try? You may need to combine
models. Remember to start with the simplest thing that might work!
Types of models we've talked about are linear regression, decision
trees, random forest, logistic regression, naive bayes, support
vector machines, K-means, DBSCAN, hierarchical clustering, fully
connected neural networks, convolutional neural networks, recurrent
neural networks, generative adversarial networks, deep Q learning,
and evolutionary methods.
\item What validation metric(s) would you use to decide how well you're
doing?
\item What ethical challenges do the data collection, creation, and/or use
of this model create? If you feel there aren't any, just say
"None".\%\%
\end{enumerate}

\section{Scenarios:}
\label{sec:org79d602d}
\begin{enumerate}
\item \textbf{You want your model to learn to play
\href{https://en.wikipedia.org/wiki/Frogger}{Frogger}.}

\begin{enumerate}
\item Reinforcement learning
\item Points, as given by the game.
\item The agent would be passed some input vector with a grid
representing the game, perhaps in addition to some extra
normalized variables like time left and goals filled.
\item Perhaps deep Q learning with an RNN to be able to process
different obstacle speeds.
\item Regret
\item Botting in video games, when the same techniques are applied to
multiplayer?
\end{enumerate}

\item \textbf{You would like a model to write tweets in the style of a particular
author.}

\begin{enumerate}
\item Generation
\item Unsupervised
\item Padding, byte pair encoding, tokenization, ect.
\item GAN, RNN / LSTM for generator, RNN for discriminator
\item Human evaluation
\item Could be used to fake tweets or impersonate people which has many
larger implications. Data harvesting might also be a problem.
\end{enumerate}

\item *A company would like to be able to predict the next months' sales
for each of its products. You have a dataset that the company has
collected for many years, with data for a particular product on a
particular month in each row. Each row contains the number of sales
for the month, the number of sales from the previous month, the
average rating (1-5) of the product in the previous month, the number
of reviews in the previous month, the product type (e.g. "toaster",
"coffee maker", "rice cooker"), its price the previous month, and its
price for the current month.*

\begin{enumerate}
\item Regression
\item The number of sales for the month in the collected dataset
\item Normalize average rating, OHE product type
\item FNN
\item R\textsuperscript{2}
\item None
\end{enumerate}

\item *You would like to predict the presence of a certain disease using
chest x-ray data. You have a lot of x-ray images, a small amount of
which have been labeled as having the disease or not having the
disease. The rest of the images are unknown as to whether or not the
person has the disease.*

\begin{enumerate}
\item Classification
\item Semi-supervised, using the provided labels as well as generated
pseudo-labels
\item Down-sampling, grayscale?
\item CNN
\item F-score
\item Model explainability challenges, covered in our ethics
presentation last semester in ML.
\end{enumerate}
\end{enumerate}

\%\%\#\#\# Example:

Here is an example answer, taken from
\url{http://archive.ics.uci.edu/ml/datasets/Abalone}:

Scenario: You want to predict the age of an abalone (a type of
shellfish). You have a dataset that includes the age of the abalone, the
sex ('M', 'F', and 'I'), the length, the diameter, the height, and the
weight.

\begin{enumerate}
\item regression
\item age (included in dataset)
\item Length, diameter, height, and weight are numeric. I will scale them.
For the sex feature, make it one-hot-encoded.
\item linear regression. maybe a fully connected neural network later
\item R\textsuperscript{2}\%\%
\end{enumerate}
\end{document}
