% Created 2021-09-27 Mon 12:00
% Intended LaTeX compiler: xelatex
\documentclass[letterpaper]{article}
\usepackage{graphicx}
\usepackage{grffile}
\usepackage{longtable}
\usepackage{wrapfig}
\usepackage{rotating}
\usepackage[normalem]{ulem}
\usepackage{amsmath}
\usepackage{textcomp}
\usepackage{amssymb}
\usepackage{capt-of}
\usepackage{hyperref}
\setlength{\parindent}{0pt}
\usepackage[margin=1in]{geometry}
\usepackage{fontspec}
\usepackage{svg}
\usepackage{cancel}
\usepackage{indentfirst}
\setmainfont[ItalicFont = LiberationSans-Italic, BoldFont = LiberationSans-Bold, BoldItalicFont = LiberationSans-BoldItalic]{LiberationSans}
\newfontfamily\NHLight[ItalicFont = LiberationSansNarrow-Italic, BoldFont       = LiberationSansNarrow-Bold, BoldItalicFont = LiberationSansNarrow-BoldItalic]{LiberationSansNarrow}
\newcommand\textrmlf[1]{{\NHLight#1}}
\newcommand\textitlf[1]{{\NHLight\itshape#1}}
\let\textbflf\textrm
\newcommand\textulf[1]{{\NHLight\bfseries#1}}
\newcommand\textuitlf[1]{{\NHLight\bfseries\itshape#1}}
\usepackage{fancyhdr}
\pagestyle{fancy}
\usepackage{titlesec}
\usepackage{titling}
\makeatletter
\lhead{\textbf{\@title}}
\makeatother
\rhead{\textrmlf{Compiled} \today}
\lfoot{\theauthor\ \textbullet \ \textbf{2021-2022}}
\cfoot{}
\rfoot{\textrmlf{Page} \thepage}
\renewcommand{\tableofcontents}{}
\titleformat{\section} {\Large} {\textrmlf{\thesection} {|}} {0.3em} {\textbf}
\titleformat{\subsection} {\large} {\textrmlf{\thesubsection} {|}} {0.2em} {\textbf}
\titleformat{\subsubsection} {\large} {\textrmlf{\thesubsubsection} {|}} {0.1em} {\textbf}
\setlength{\parskip}{0.45em}
\renewcommand\maketitle{}
\author{Huxley}
\date{\today}
\title{Mason Chapter Five}
\hypersetup{
 pdfauthor={Huxley},
 pdftitle={Mason Chapter Five},
 pdfkeywords={},
 pdfsubject={},
 pdfcreator={Emacs 28.0.50 (Org mode 9.4.4)}, 
 pdflang={English}}
\begin{document}

\tableofcontents

\#flo \#disorganized \#incomplete

\noindent\rule{\textwidth}{0.5pt}

\section{Marks?}
\label{sec:orgf6196ac}
no it's Marx.

\begin{itemize}
\item 1848, communist maanifesto time
\item led to the soviet union
\end{itemize}

\subsection{Carl Marks}
\label{sec:org36a4ad0}
\begin{itemize}
\item Father advocated for constitutionalism
\item moved around a bunch
\item see pg 22 for summary
\end{itemize}

\subsection{The manifesto.}
\label{sec:orgcd2f22e}
\begin{itemize}
\item Joins a secret society for communism
\item in the manifesto:

\begin{itemize}
\item \begin{quote}
Marx and Engels argued that history should not be un- derstood as
a story of great individuals or of conflict among states but of
social classes and their struggles with each other.
\end{quote}

\item ends with

\begin{itemize}
\item \begin{quote}
"Let the ruling classes tremble at a communist revolu- tion. The
proletarians have nothing to lose but their chains. They have a
world to win. Proletarians of all countries, unite!"
\end{quote}
\end{itemize}
\end{itemize}

\item Marx was super poor
\item Engels:

\begin{itemize}
\item \begin{quote}
\textbf{*} "Philosophers have so far explained the world in various ways:
the point, however, is to change it."
:CUSTOM\textsubscript{ID}: philosophers-have-so-far-explained-the-world-in-various-ways-the-point-however-is-to-change-it.
\end{quote}
\end{itemize}
\end{itemize}

\subsection{the theory}
\label{sec:org00883e3}
\begin{itemize}
\item atempted to create a scientific theory of hist and econ

\item about means of production

\item those who control the means of production control the society

\item religion is

\begin{itemize}
\item \begin{quote}
is simply a tool of the dominant class to keep the lower classes
in their place in this world, with the expectation of a better
existence in the hereafter. Religion, in the words of Engels, is
the “opiate of the masses.
\end{quote}
\end{itemize}

\item Marx was very deterministic?

\item all historty is about and driven by class struggles

\item \begin{quote}
All societies begin in the primitive-communal stage, move through a
system of slavery (the dominant class being the slave owners), then
feudalism, then capitalism, and eventually communism, at which point
classes would no longer exist.
\end{quote}

\item ** Believed that this was inevitable in every society
:CUSTOM\textsubscript{ID}: believed-that-this-was-inevitable-in-every-society

\item says that capatilism sows the seads of its own destruction

\begin{itemize}
\item workers only get paid a fraction of the value they produce
\item thus, they cant afford what they produce
\item leads to accumulation of goods that people cannot buy
\item which leads to peridoc crisis
\item and forces entrepaneus to scale back and lay off workers
\item which leads to increasingly bad econonic crises and the
"immerseration" of the workers
\item eventually the workers will just revolt ---
\item is this true? woudnt the market just adjust?
\end{itemize}
\end{itemize}

\subsection{the idea? weird flow man}
\label{sec:orgfe3b3cf}
\begin{itemize}
\item \begin{quote}
when the workers own the means of pro- duction, the entire economic
substructure will collapse and re-form, as will the superstructure
of society.
\end{quote}

\item adam smith talks about private vice creating public virute

\item marx says that the communist enviroment will foster people who dont
engage in this private vice

\item and thus, the communal aspect works

\item communism fosters the creation of a "new man", which will build a "new
society"

\item \begin{quote}
When social classes disappear, so too will poverty, exploitation,
re- sentment, greed, and crime, so that there will be no need for a
police force. Indeed, because government simply perpetuates the
supremacy of the dominant class, without social classes there will
be no need for government at all. The state, according to Engels,
will simply "wither away."
\end{quote}

\item along with the states will be natinal boundries, and then the whole
planet will all start hugging eachother
\end{itemize}

\subsection{the idea, but later}
\label{sec:org3d1dee3}
\begin{itemize}
\item product of enlightenment and such
\item idea of economic determinism is still largely present today
\item bunch people adopted it
\item to see this, see pg 33
\end{itemize}

\subsection{Actual manifesto}
\label{sec:orgf2b5f52}
\begin{itemize}
\item \begin{quote}
class conflict---the idea that the social order is divided into
classes based on conflicting economic interests.
\end{quote}

\item i dont have time to annotate whoooops
\end{itemize}

\begin{center}
\begin{tabular}{l}
\# Chapter 7, begin.\\
\hline
\#\# Nationalism?\\
- dates and things, nationalist ideals begin. see pg 32 - the concept of a nation was new, as it was derived from the enlightenment - about shared 'things' -- political goals, idealogy, race, ect. - > A people has to feel these common ties to be a nation. - nationalism is a form of seperatism, therefore the states dont like it - > Often, this nationalism took a very different form, called irredentism, which is the demand for territory belonging to another state. This top-down nationalism, used by national leaders making irredentist claims, fostered the creation of unified states in Germany and Italy.\\
\end{tabular}
\end{center}

\begin{itemize}
\item \begin{quote}
PRELUDE TO UNIFICATION: THE CRIMEAN WAR
\end{quote}

\item \emph{foreshadowing the first world war}

\item and then, unity!(?)

\item revolutionry nationlism vs controlled nationalism (liberal, with a
constitutional monarchy)

\item \begin{quote}
led his thousand "Redshirts" in a seizure of power
\end{quote}

\item oh boy

\item welp, lotta dates.
\end{itemize}
\end{document}
