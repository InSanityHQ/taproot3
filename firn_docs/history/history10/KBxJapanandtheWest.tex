% Created 2021-09-27 Mon 12:00
% Intended LaTeX compiler: xelatex
\documentclass[letterpaper]{article}
\usepackage{graphicx}
\usepackage{grffile}
\usepackage{longtable}
\usepackage{wrapfig}
\usepackage{rotating}
\usepackage[normalem]{ulem}
\usepackage{amsmath}
\usepackage{textcomp}
\usepackage{amssymb}
\usepackage{capt-of}
\usepackage{hyperref}
\setlength{\parindent}{0pt}
\usepackage[margin=1in]{geometry}
\usepackage{fontspec}
\usepackage{svg}
\usepackage{cancel}
\usepackage{indentfirst}
\setmainfont[ItalicFont = LiberationSans-Italic, BoldFont = LiberationSans-Bold, BoldItalicFont = LiberationSans-BoldItalic]{LiberationSans}
\newfontfamily\NHLight[ItalicFont = LiberationSansNarrow-Italic, BoldFont       = LiberationSansNarrow-Bold, BoldItalicFont = LiberationSansNarrow-BoldItalic]{LiberationSansNarrow}
\newcommand\textrmlf[1]{{\NHLight#1}}
\newcommand\textitlf[1]{{\NHLight\itshape#1}}
\let\textbflf\textrm
\newcommand\textulf[1]{{\NHLight\bfseries#1}}
\newcommand\textuitlf[1]{{\NHLight\bfseries\itshape#1}}
\usepackage{fancyhdr}
\pagestyle{fancy}
\usepackage{titlesec}
\usepackage{titling}
\makeatletter
\lhead{\textbf{\@title}}
\makeatother
\rhead{\textrmlf{Compiled} \today}
\lfoot{\theauthor\ \textbullet \ \textbf{2021-2022}}
\cfoot{}
\rfoot{\textrmlf{Page} \thepage}
\renewcommand{\tableofcontents}{}
\titleformat{\section} {\Large} {\textrmlf{\thesection} {|}} {0.3em} {\textbf}
\titleformat{\subsection} {\large} {\textrmlf{\thesubsection} {|}} {0.2em} {\textbf}
\titleformat{\subsubsection} {\large} {\textrmlf{\thesubsubsection} {|}} {0.1em} {\textbf}
\setlength{\parskip}{0.45em}
\renewcommand\maketitle{}
\author{Huxley}
\date{\today}
\title{Japan and the West Notes}
\hypersetup{
 pdfauthor={Huxley},
 pdftitle={Japan and the West Notes},
 pdfkeywords={},
 pdfsubject={},
 pdfcreator={Emacs 28.0.50 (Org mode 9.4.4)}, 
 pdflang={English}}
\begin{document}

\tableofcontents

\#flo \#ref \#disorganized

\noindent\rule{\textwidth}{0.5pt}

\section{The west? Japan?}
\label{sec:org54e829c}
\begin{itemize}
\item were highly civilized when 'they alloed the westerners to discorver
them'
\item westernization of japan,

\begin{itemize}
\item \begin{quote}
(ish) looked like it had been opened by westerners but it actully
had exploded from within \#\# background: two centuries of
isolation: 1640---1854
\end{quote}
\end{itemize}

\item japan was isolated, no one was alowwed to leave and foreneris were not
allowed to enter
\item these policies were based on experience
\item shoguns and stuff
\item miltatary formed a dictatorship
\item class lies began to blur as people went into povrerty, then merchants
and artisans started to prosper
\item many people stopped believing in buddhism
\end{itemize}

\subsection{the opening.}
\label{sec:org8c46df1}
\begin{itemize}
\item nobles were heavily in debt

\begin{itemize}
\item couldnt produce from agriculture,
\item so turned to foreing trade.
\end{itemize}

\item buncha stuff
\end{itemize}

\subsection{meji era}
\label{sec:org8230b42}
\begin{itemize}
\item forced the shogun to resign
\item turned it into a modern nation state
\item abolished feudalism
\item people adopted shintoism
\item industrilization and financial modernization happened, led to increase
in wealth and massive boom in population
\end{itemize}

\subsection{russia and stuff}
\label{sec:org0f4e9ee}
\begin{itemize}
\item russian gov needed

\begin{itemize}
\item \begin{quote}
atmosphere of crisis and expansion to stifle criticism of tsarism
at home;
\end{quote}
\end{itemize}

\item war broke our in 1904 between japan and russia

\item buncha dates and stuff

\item \begin{quote}
The Japanese victory set off long chains of repercussions in at
least three different directions. First, the Russian government,
frustrated in its foreign policy in East Asia, shifted its attention
back to Europe, where it re sumed an active role in the affairs of
the Balkans
\end{quote}

\item led to russian revolution

\item \begin{quote}
The moral was clear. Everywhere leaders of subjugated peoples
concluded, from the Japanese precedent, that they must bring Western
science and industry to their own countries, but that they must do
it, as the Japa nese had done, by getting rid of control by the
Europeans, supervising the process of modernization themselves, and
preserving their own native na tional character.
\end{quote}

\item sum of ending:

\begin{itemize}
\item \begin{quote}
The Japanese victory and Russian defeat can therefore be seen as
steps in three mighty developments: the First World War, the
Russian Revolution, and the Revolt of Asia. These three together
put an end to Europe's world supremacy and to confident ideas
about the in evitable progress and expansion of European
civilization; or at least they so transmuted them as to make the
world of the twentieth century far different from that of the
nineteenth.
\end{quote}
\end{itemize}
\end{itemize}
\end{document}
