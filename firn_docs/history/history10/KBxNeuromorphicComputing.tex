% Created 2021-09-27 Mon 12:00
% Intended LaTeX compiler: xelatex
\documentclass[letterpaper]{article}
\usepackage{graphicx}
\usepackage{grffile}
\usepackage{longtable}
\usepackage{wrapfig}
\usepackage{rotating}
\usepackage[normalem]{ulem}
\usepackage{amsmath}
\usepackage{textcomp}
\usepackage{amssymb}
\usepackage{capt-of}
\usepackage{hyperref}
\setlength{\parindent}{0pt}
\usepackage[margin=1in]{geometry}
\usepackage{fontspec}
\usepackage{svg}
\usepackage{cancel}
\usepackage{indentfirst}
\setmainfont[ItalicFont = LiberationSans-Italic, BoldFont = LiberationSans-Bold, BoldItalicFont = LiberationSans-BoldItalic]{LiberationSans}
\newfontfamily\NHLight[ItalicFont = LiberationSansNarrow-Italic, BoldFont       = LiberationSansNarrow-Bold, BoldItalicFont = LiberationSansNarrow-BoldItalic]{LiberationSansNarrow}
\newcommand\textrmlf[1]{{\NHLight#1}}
\newcommand\textitlf[1]{{\NHLight\itshape#1}}
\let\textbflf\textrm
\newcommand\textulf[1]{{\NHLight\bfseries#1}}
\newcommand\textuitlf[1]{{\NHLight\bfseries\itshape#1}}
\usepackage{fancyhdr}
\pagestyle{fancy}
\usepackage{titlesec}
\usepackage{titling}
\makeatletter
\lhead{\textbf{\@title}}
\makeatother
\rhead{\textrmlf{Compiled} \today}
\lfoot{\theauthor\ \textbullet \ \textbf{2021-2022}}
\cfoot{}
\rfoot{\textrmlf{Page} \thepage}
\renewcommand{\tableofcontents}{}
\titleformat{\section} {\Large} {\textrmlf{\thesection} {|}} {0.3em} {\textbf}
\titleformat{\subsection} {\large} {\textrmlf{\thesubsection} {|}} {0.2em} {\textbf}
\titleformat{\subsubsection} {\large} {\textrmlf{\thesubsubsection} {|}} {0.1em} {\textbf}
\setlength{\parskip}{0.45em}
\renewcommand\maketitle{}
\author{Huxley}
\date{\today}
\title{Neuromorphic Computing}
\hypersetup{
 pdfauthor={Huxley},
 pdftitle={Neuromorphic Computing},
 pdfkeywords={},
 pdfsubject={},
 pdfcreator={Emacs 28.0.50 (Org mode 9.4.4)}, 
 pdflang={English}}
\begin{document}

\tableofcontents

\#flo \#ref \#ret \#incomplete

\noindent\rule{\textwidth}{0.5pt}

\section{Final Exploration}
\label{sec:org8abe090}
Note: I did a dive into trying to label the axes of a word vector space,
and I also tried to do some stuff with summarization models, but neither
of those projects worked out within the time frame. Delegated to summer
they go. Instead, here are my notes on

\subsection{Neuromorphic Computing!}
\label{sec:orge20448c}
\href{https://www.youtube.com/watch?v=-dl1FPrpw1A\&ab\_channel=NeuroInspiredComputationalElementsWorkshop}{Cool
talk}

Notes on the talk:

\begin{itemize}
\item Talks about autonomous drone racing

\begin{itemize}
\item only possible recently due to tech limitations
\item Bird brain vs Drone "brain"

\begin{itemize}
\item Parrot

\begin{itemize}
\item 50mW, 2.2G

\item Can learn words

\item Can nav new environments at 35km/h

\item \begin{quote}
Can learn to manipulate cups for drinking
\end{quote}
\end{itemize}
\end{itemize}

\item Drone

\begin{itemize}
\item 18000 mW, 40g

\item Pre trained flying at walking pace

\item \begin{quote}
Can't learn anything online
\end{quote}
\end{itemize}

\item Main idea is the birds adaptability

\begin{itemize}
\item Can learn to "really understand what a cup can be useful for"
despite it not being in its evolutionary past
\item way beyond current autonomous drones
\end{itemize}

\item We have a lot to learn from nature!
\end{itemize}

\item Deep learning is very power hungry

\begin{itemize}
\item increasingly so -- grew by 300,000x in the last 6 years
\item not on a trajectory to close the gap with the parrot!
\end{itemize}

\item Deep learning has slow generalization

\begin{itemize}
\item Training currently is mostly offline and batched
\item example of a child looking at a few pictures of a cat: they can now
easily tell what is a cat, and can even recognize cartoons of a cat.
\end{itemize}

\item looking to the brain!

\begin{itemize}
\item they implemented neural networks with temporal states in each neuron

\begin{itemize}
\item with some course of evolution
\end{itemize}

\item current tools break down on these new models

\begin{itemize}
\item arn't differentiable, cant do stochastic descent, ect.
\end{itemize}
\end{itemize}

\item motivates a different type of computing?

\begin{itemize}
\item memory elements is embedded in the synaptic neurons
\item redefines what is efficient

\begin{itemize}
\item conventional is forced into vectorized approach
\end{itemize}
\end{itemize}

\item their new chip doesn't have floating point numbers, multiply
accumulators, or off-chip DRAM

\begin{itemize}
\item but they can do all this amazing stuff

\begin{itemize}
\item highlights that they are genuinely operating in a totally
different paradigm
\end{itemize}
\end{itemize}

\item results!

\begin{itemize}
\item sensing domain

\begin{itemize}
\item in spiking event based paradigm
\item IO no longer a bottleneck
\item can natively send event from sensor to chip, can do gesture reg
with just 15mW for both the camera and chip combined
\item early new live learning
\item doing new things with touch
\item and created odor recognition\ldots{}?

\begin{itemize}
\item 3000x more efficient than deep autoencoder
\end{itemize}
\end{itemize}

\item robotics and drone space

\begin{itemize}
\item adaptive robotic arms that can learn real world forces like
friction and remap itself
\item iCub scene understanding
\item can use visual cues to and live learning for directionality
\item event based camera input for horizon tracking with crazy specs
\item super small 35 neuron network to achieve smooth MAV landings

\begin{itemize}
\item like some kind of insect brain
\end{itemize}
\end{itemize}

\item abstract

\begin{itemize}
\item graph search can use inherent parallelism to be over 100x faster

\begin{itemize}
\item with "temporally propagated wavefronts"
\end{itemize}

\item much better and faster similarity search
\item can easily model heat diffusion
\item combinatorial optimization 40x faster, 2000x lower energy

\begin{itemize}
\item like real world train scheduling problems
\end{itemize}
\end{itemize}

\item can map problem space onto energy and time gain for their new
neuromorphic techniques

\href{Screen Shot 2021-06-11 at 3.48.43 PM.png.org}{Screen Shot
2021-06-11 at 3.48.43 PM.png}

\item gains mostly in RNNs instead of feed forward

\begin{itemize}
\item like the brain
\end{itemize}
\end{itemize}

\item 1000x gains, not percentage wise

\begin{itemize}
\item on a programmable chip!
\item hardware is still very early
\end{itemize}
\end{itemize}
\end{document}
