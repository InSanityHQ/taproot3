% Created 2021-09-27 Mon 12:00
% Intended LaTeX compiler: xelatex
\documentclass[letterpaper]{article}
\usepackage{graphicx}
\usepackage{grffile}
\usepackage{longtable}
\usepackage{wrapfig}
\usepackage{rotating}
\usepackage[normalem]{ulem}
\usepackage{amsmath}
\usepackage{textcomp}
\usepackage{amssymb}
\usepackage{capt-of}
\usepackage{hyperref}
\setlength{\parindent}{0pt}
\usepackage[margin=1in]{geometry}
\usepackage{fontspec}
\usepackage{svg}
\usepackage{cancel}
\usepackage{indentfirst}
\setmainfont[ItalicFont = LiberationSans-Italic, BoldFont = LiberationSans-Bold, BoldItalicFont = LiberationSans-BoldItalic]{LiberationSans}
\newfontfamily\NHLight[ItalicFont = LiberationSansNarrow-Italic, BoldFont       = LiberationSansNarrow-Bold, BoldItalicFont = LiberationSansNarrow-BoldItalic]{LiberationSansNarrow}
\newcommand\textrmlf[1]{{\NHLight#1}}
\newcommand\textitlf[1]{{\NHLight\itshape#1}}
\let\textbflf\textrm
\newcommand\textulf[1]{{\NHLight\bfseries#1}}
\newcommand\textuitlf[1]{{\NHLight\bfseries\itshape#1}}
\usepackage{fancyhdr}
\pagestyle{fancy}
\usepackage{titlesec}
\usepackage{titling}
\makeatletter
\lhead{\textbf{\@title}}
\makeatother
\rhead{\textrmlf{Compiled} \today}
\lfoot{\theauthor\ \textbullet \ \textbf{2021-2022}}
\cfoot{}
\rfoot{\textrmlf{Page} \thepage}
\renewcommand{\tableofcontents}{}
\titleformat{\section} {\Large} {\textrmlf{\thesection} {|}} {0.3em} {\textbf}
\titleformat{\subsection} {\large} {\textrmlf{\thesubsection} {|}} {0.2em} {\textbf}
\titleformat{\subsubsection} {\large} {\textrmlf{\thesubsubsection} {|}} {0.1em} {\textbf}
\setlength{\parskip}{0.45em}
\renewcommand\maketitle{}
\author{Huxley}
\date{\today}
\title{WW1 Essay Planning}
\hypersetup{
 pdfauthor={Huxley},
 pdftitle={WW1 Essay Planning},
 pdfkeywords={},
 pdfsubject={},
 pdfcreator={Emacs 28.0.50 (Org mode 9.4.4)}, 
 pdflang={English}}
\begin{document}

\tableofcontents

\#flo \#ret \#disorganized \#incomplete

\noindent\rule{\textwidth}{0.5pt}

\section{Ope there goes gravity}
\label{sec:orgf09b837}
prompt:

\begin{verbatim}
**

The political scientist Kenneth Waltz argues that the causes of war can be analyzed at three different levels: the individual human level, the state level, and the international system level. Those who view things from the first level believe that war is best explained by “selfishness,” “misdirected aggressive impulses,” or “stupidity” within the human psyche. Those who favor the second level believe there are hostile or aggressive or revisionist states who, because of their form of government or other domestic issues, behave in a warlike manner while other states simply want to keep the peace (the status quo). Those who favor the third level believe that the international system itself, because it is an anarchy with “no system of law enforceable” between states, and in which each state acts according to its own interest and reserves the right to use force to achieve its aims, makes war inevitable. 



Analyze World War 1 according to one (or a blend) of these levels of analysis. Make an argument that combines an explanation of the general causes of the war with the specific sequence of events (including events that prolonged the war beyond the initial outbreak). I would suggest using a chronological organization to your essay.

**
\end{verbatim}

resources: \href{GHMW Unit 4.pdf.org}{GHMW Unit 4.pdf} palmer reading
\url{https://drive.google.com/drive/folders/1KTggTDz3Yl7fT9MxwG4l25qMPNyiUioe?usp=sharing}

levels of analysis:

\begin{itemize}
\item individual
\item state
\item systemic
\end{itemize}

example: put an egg in a highway, do you blame the car that chrushes it?

how do we think about inevitability?

relative usefullness of each one

how do they intersect? system provides inevitability, but is that
emergent out of the qualities of the individual?

potential essay: top down, each one informs the next. simply map to
analysis levels

claim: not actully three seperate levels? higher up levels cannot be
generated withought the previous?

systems need to account for randomness, which is inevitable. thus,
higher level analysis?

the fact that ferdinand dying caused this meant that there was a higher
level issue?

systems: random chance events will happen

was it a random chance event? was this the 'natural progression of
things'?

or is the aproach, 'random chance events are inevitable (in a system
with so many actors), thus the system must be set up for this? or it is
part of the system?'

top down:

\begin{itemize}
\item because of this system, it was inevitable that a state would act out.
\item lets look at this state. because of this state, it was inevitable that
a person would act out
\item lets look at that person
\item discuss merits of different levels of analysis?
\end{itemize}

bottom up:

\begin{itemize}
\item this person did this, and that caused the war.
\end{itemize}

usefullness of analysis at different levels?

war is a local minima in the system

nukes have allowed us to jump peaks and reach a lower minima

warring state has incentive to bring allies into war

non warring states have incentive to allie with winning side and gain
power with them, as well as prevent others from becoming to powerful --
BOP dynamic from resan detant

this brings the entire system into war

as more actors join into the war, you MUST join the war. one side of the
war will win, and massively gain power. you cannot stay at your old
divided power level.

just as peace is an emergent property of raison detant (BOP), global war
can be as well? easily "excitable"

prisoners dillema sqaure, when tensions are high, best desition is to
attack. also best descition to bring allies, and best desition for
allies to join in. tragedy of the commons. war breaks out

OLD: [aa] => 1,1. [da, ad] => 0,3 3,0. [dd] => 2,2. || a NEW: [aa, ad,
da] => 0,0. [dd] => 2,2. || d

system has changed, MAD has allowed us to jump peaks to a lower minima.

war only breaks out when tenstions are high, otherwise you would just
defend.

once tensions become high, war becomes inevitable, and can be triggered
at the smallest event.

\^{}\^{} goes in bp3

franz ferdinand died, and that caused the war. if that could cause the
war, something was up.

germany caused the tenstions which caused the war. but \{transition\}

system level game theory analysis

thesis ideas:

levels of analysis are usefull at the level you can impact? doesnt work
if higher level informs your level of impact\ldots{}

causation is defined by what is usefull?

potential outline:

start with small explanatin: ferdinand and states || but were they realy
to blame? egg in street

systemic explanation

which requires tenstion

thesis: tension changed the local minima to global warfare

switch tenstion and system explanation?

everyone had huge armies, and they all took it for granted that a war
was coming even though few wanted one pg. 1-2

"In the last years before 1914 the idea that war was bound to break out
sooner or later probably made some statesmen, in some countries, more
willing to unleash it."is thiss cus "it would happen anyways" or cus "we
want to be first"?

formed the triple alliance out of fear that the german empire would be
torn to peices in another european war

french formed their own alliance in response

britain and germany had a naval race

germans felt encircled by france and russia alliance even more concerned
when france joined them

\subsection{Quote Bin}
\label{sec:org7533569}
\begin{itemize}
\item ferdinand and states

\begin{itemize}
\item Ferdinand

\begin{itemize}
\item On June 28, 1914, a young Bosnian revolutionary, a member of the
Serbian secret soci ety called "Union of Death" and commonly known
as the Black Hand, acting with the knowledge of certain Serbian
officials, assassinated the heir to the Habsburg empire, the
Archduke Francis Ferdinand,
\item states

\begin{itemize}
\item and above
\item The Germans, issuing their famous "blank check," encouraged the
Austrians to be firm.
\item The Austrians, thus reassured, dispatched a drastic ultimatum to
Serbia, demanding among other things that Austrian officials be
permitted to collaborate in investigating and punishing the
perpetrators of the assassination.
\item The Serbs counted on Russian support, even to the point of war,
judging that Russia could not yield in a Balkan crisis, for the
third time in six years, without losing its influence in the
Balkans altogether.
\item The Russians in turn counted on France; and France, terrified at
the possibility of being some day caught alone in a war with
Germany and determined to keep Russia as an ally at any cost, in
effect gave a blank check to Russia.
\item It has often been said that had the German government know as a
positive fact that England would fight, the war might not have
come. Hence the evasiveness of British policy is made a
contributing cause of the war.
\end{itemize}
\end{itemize}
\end{itemize}

\item tension and alliances

\begin{itemize}
\item triple alliance

\begin{itemize}
\item The Germans complained of being "encircled" by France and Russia.
They dreaded the day when they might have to face a war on two
fronts.\\
\item Bismarck after 1871 feared that in another European war his new
German Empire might be tom to pieces. [\ldots{}] in 1879 he formed a
military alliance with Austria-Hungary, to which Italy was added
in 1882. Thus was formed the Triple Alliance,
\end{itemize}

\item triple entente

\begin{itemize}
\item The French, faced by the Triple Alliance, soon seized the
opportunity to form their own alliance with Russia
\item By 1907 Eng land, France, and Russia were acting together. The
older Triple Alliance faced a newer Triple Entente,
\end{itemize}

\item tension

\begin{itemize}
\item that would arise in the future. Each power felt that it must stand
by its allies whatever the specific issue. This was because all
lived in the fear of war, of some nameless future war in which
allies would be necessary.
\item Never had the European states maintained such huge armies in
peacetime as at the be ginning of the twentieth century. One, two,
or even three years of compul sory military service for all young
men became the rule. In 1914 each of the Continental Great Powers
had not only a huge standing army but millions of trained reserves
among the civilian population
\item The Germans, who already felt encircled by the alliance of France
and Russia, naturally watched with concern the drift of England
into the Franco-Russian camp.
\item Once united (or almost united), the Germans entered upon their
industrial revolution. Manufacturing, finance, shipping,
population grew phenomenally. By 1900, for example, Germany
produced more steel than France and Britain combined,
\end{itemize}
\end{itemize}

\item systemic

\begin{itemize}
\item "In the last years before 1914 the idea that war was bound to break
out sooner or later probably made some statesmen, in some countries,
\textbf{more willing to unleash it.}"
\item Each power felt that it must stand by its allies whatever the
specific issue.
\end{itemize}
\end{itemize}

\subsection{Outline (finally (hopefully))}
\label{sec:org0f2b2d4}
\begin{enumerate}
\item THESIS: In a system of raison detan't, increase in tension and
\label{sec:orgb732b17}
the formation of alliances shifted the local minima from peace to global
warfare.
:CUSTOM\textsubscript{ID}: thesis-in-a-system-of-raison-detant-increase-in-tension-and-the-formation-of-alliances-shifted-the-local-minima-from-peace-to-global-warfare.

\begin{itemize}
\item Ferdinand and states

\begin{itemize}
\item Ferdiand got assinated by the black hand, this caused the war.

\begin{itemize}
\item for ferdinand getting assinated to cause such a large scale war,
larger factors must have been at play.
\end{itemize}

\item exactly what state caused what is heavily debated

\begin{itemize}
\item it could have been the germans, who attempted to destroy serbia
\item could have been serbia, who backed the black hand's assination
\item the fact of the matter is, qoute "In the last years before 1914
the idea [was] that war was bound to break out sooner or later."
and this wasnt due to one meddling state.
\end{itemize}
\end{itemize}

\item tenstion and alliances

\begin{itemize}
\item to better understand the true causes, we first must understand the
situation.
\item alliances formed, tension formed
\item alliances:

\begin{itemize}
\item formed the triple alliance

\begin{itemize}
\item out of fear of a european war
\end{itemize}

\item formed the triple entente

\begin{itemize}
\item scared the germans even more
\end{itemize}

\item these alliances also contributed healvily to growing --
\end{itemize}

\item tension:

\begin{itemize}
\item large armies,
\item naval race,
\item technological and economic growth
\end{itemize}
\end{itemize}

\item systemic

\begin{itemize}
\item BOP prisoners diilema reward square is 0,0 for the AD scenario

\begin{itemize}
\item once you have alliances, it becomes 0,3 again
\end{itemize}

\item tension ensures that the other wants to attack, meaning it is clear
that you must first.
\item ALSO theres the snowball effect

\begin{itemize}
\item OUTSIDER: if there is a large war you must join in or be left in
the dust.
\item INSIDER: best option is to bring in allies
\item leads to massive wars
\end{itemize}

\item with tension and alliances, local minima becomes massive wars as
opposed to peace
\end{itemize}
\end{itemize}

conclustion -- nukes?
\end{enumerate}

\subsection{Goddamn. Writing. Time. (0.7th draft)}
\label{sec:org83edef5}
In a system of raison detan't, increase in tension and the formation of
alliances shifted the local minima from peace to global warfare.

analyzing individul actors in a broader system

While many individual actors could be blamed, analysis at this level is
not useful for understanding causation.

The assassination of Franz Ferdinand caused World War One. In 1914, a
Serbian member of the Black Hand "assassinated the heir to the Hapsburg
empire." (citation). This murder caused the Austrian government to try
and end Serbia's independence; global warfare ensued. However, for the
assassination of Ferdinand to cause such a large scale event, larger
factors must have been at play. The situation was akin to an egg sitting
in a busy highway -- at some point, the egg will get cracked. Yes, the
driver who runs the egg over caused it to crack, but a deeper and more
useful understanding of causation comes from analyzing the situation
rather than the one of many possible drivers. At a higher level of
removal, it could be argued that Serbia was the cause of the war, as the
assassin was "acting with the knowledge of certain Serbian
officials."(citation). Or perhaps it was Austria, who issued a "drastic
ultimatum to Serbia." (citation). But Austria only did so after Germany
delivered "their famous 'blank check,' encourag[ing] the Austrians to be
firm."(citation). Trying to determine which point in this causational
chain truly led to World War One is a fruitless exercise, as the fact of
the matter is, "in the last years before 1914 the idea [was] that war
was bound to break out sooner or later" -- and this wasn't due to one
meddling state (citation).

To better understand the cause of the war, we must first understand the
geopolitical situation at the time -- a time of growing alliances and
tensions. In 1817, the fear of warfare led Germany to form "a military
alliance with Austria-Hungary, to which Italy was added in 1882."
(citation). This alliance led to the creation of another; faced by this
new threat, France allied with Russia, and later, England. Tension only
grew, as Germany "who already felt encircled by the alliance of France
and Russia, naturally watched with concern the drift of England into the
Franco-Russian camp."(citation). The Continental Great Powers were now
riddled with alliances. These alliances contributed to the massive
amount of tension, evident by the "huge standing army" and "millions of
trained reserves among the civilian population" of each Continental
Great Power. (citation). This tension created a feeling of impending
war, "in which allies would be necessary." (citation). This feeling
tightened existing alliances and led to the creation of new ones, in
turn contributing to even greater tension. Alliances formed, tension
grew, and the with this change the local minima began to shift. \{ could
mention: Massive industrial and economic growth \}

The emergent property from raison d'etat changes from a balance of power
to large scale conflict when alliances and tension are introduced. A
system of states all enacting raison d'etat is known to lead to a
balance of power, as when any one state gains power, the others act in
their own best interest and crush the rising state. Out of pure self
interest arises peace. We can represent this scenario with a payoff
matrix -- two actors, each with two possible actions: attack or defend.
If both states defend, they each get a score of two. If both attack,
they each lose resources, and any power gained would be crushed; each
state gets a zero. If one state attacks while the other defends, the
same scenario occurs. In this case, defending is clearly the best
action. However, as alliances form, this payoff matrix changes. If an
alliance attacks, it becomes much more likely that they will reap a
reward, as crushing them requires an alliance of equal or greater size.
If one doesn't form, the alliance gets a score of three, and the
defending party a zero. Thus, larger alliances form as a preemptive
measure. This alliance formation then starts a sort of snowball effect.
For any insider -- an attacking or defending party -- the best course of
action would be to try and recruit more allies. One side is expected to
win, and effectively gain the power of their fallen foes. And hence,
outsiders, if the war is sufficiently large, must join the fray lest
they be left in the dust with their old divided power level. The global
minima has shifted. The emergent property from raison d'etat is no
longer peace but rather large-scale warfare.

\begin{itemize}
\item raison d'etat leads to balance of power

\begin{itemize}
\item states all crush others when they gain power cus it's their best
interest
\item we can represent this scenario with a payoff matrix, with two actors
who can either defend or attack

\begin{itemize}
\item if both actors defend, they get a reward, say two.
\item if both actors attack, any resulting power gained would just be
crushed by the state system, so each state would get 0 or lower.
\item if one actor attacks and another defends, then any resulting power
would againbe chrushed, resulting in 0.
\end{itemize}

\item the best option is clearly to defend.
\end{itemize}

\item when large allainces form, it becomes a lot more likely that the AD
scenario results in a reward, 0,3.

\begin{itemize}
\item if the balance of power system is to be preserved, allainces of
greater size must be formed (temp?)
\end{itemize}

\item Unsure of whether or not attacking will be fruitfull. With tension,
you want to attack first.
\end{itemize}

**

support exploring the causes of World War One through

We want to, need to, build systems such that the emergent properties
lead to peace - even stable peace. Uncomfortable as it sounds, the
modern mutually assured destruction of the atomic era is one such model.

In the modern day, this methodology still applies (nukes).

Emergent properties exerting downward causal control.

Rather than the classic "chain of causality" it is really a cycle of
causality.

**
\end{document}
