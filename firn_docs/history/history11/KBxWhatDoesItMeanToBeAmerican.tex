% Created 2021-09-27 Mon 12:01
% Intended LaTeX compiler: xelatex
\documentclass[letterpaper]{article}
\usepackage{graphicx}
\usepackage{grffile}
\usepackage{longtable}
\usepackage{wrapfig}
\usepackage{rotating}
\usepackage[normalem]{ulem}
\usepackage{amsmath}
\usepackage{textcomp}
\usepackage{amssymb}
\usepackage{capt-of}
\usepackage{hyperref}
\setlength{\parindent}{0pt}
\usepackage[margin=1in]{geometry}
\usepackage{fontspec}
\usepackage{svg}
\usepackage{cancel}
\usepackage{indentfirst}
\setmainfont[ItalicFont = LiberationSans-Italic, BoldFont = LiberationSans-Bold, BoldItalicFont = LiberationSans-BoldItalic]{LiberationSans}
\newfontfamily\NHLight[ItalicFont = LiberationSansNarrow-Italic, BoldFont       = LiberationSansNarrow-Bold, BoldItalicFont = LiberationSansNarrow-BoldItalic]{LiberationSansNarrow}
\newcommand\textrmlf[1]{{\NHLight#1}}
\newcommand\textitlf[1]{{\NHLight\itshape#1}}
\let\textbflf\textrm
\newcommand\textulf[1]{{\NHLight\bfseries#1}}
\newcommand\textuitlf[1]{{\NHLight\bfseries\itshape#1}}
\usepackage{fancyhdr}
\pagestyle{fancy}
\usepackage{titlesec}
\usepackage{titling}
\makeatletter
\lhead{\textbf{\@title}}
\makeatother
\rhead{\textrmlf{Compiled} \today}
\lfoot{\theauthor\ \textbullet \ \textbf{2021-2022}}
\cfoot{}
\rfoot{\textrmlf{Page} \thepage}
\renewcommand{\tableofcontents}{}
\titleformat{\section} {\Large} {\textrmlf{\thesection} {|}} {0.3em} {\textbf}
\titleformat{\subsection} {\large} {\textrmlf{\thesubsection} {|}} {0.2em} {\textbf}
\titleformat{\subsubsection} {\large} {\textrmlf{\thesubsubsection} {|}} {0.1em} {\textbf}
\setlength{\parskip}{0.45em}
\renewcommand\maketitle{}
\author{huxley}
\date{\today}
\title{what does it mean to be american?}
\hypersetup{
 pdfauthor={huxley},
 pdftitle={what does it mean to be american?},
 pdfkeywords={},
 pdfsubject={},
 pdfcreator={Emacs 28.0.50 (Org mode 9.4.4)}, 
 pdflang={English}}
\begin{document}

\tableofcontents

\#inclass

\section{so, what does it mean?}
\label{sec:org12c4781}
\emph{freewrite:}

\noindent\rule{\textwidth}{0.5pt}

america -> really just a place more connected to silicon valley

don't understand what "not america" is

i did not choose to live in america, thus im not proud of it

america as a set of idealogies

being american = believing the idealogies? or the specific social
contract

silicon valley -> fluidity of ideas and innovation, and the resources
and hope to do it?

built on distrust of authority heredity authroity soecifically

stuck in the same place is not american via capitalism and poverty
cycles

american dream is everyone has the ability to climb via capitalslm

\noindent\rule{\textwidth}{0.5pt}

\href{https://docs.google.com/document/d/15X8knvkTgq1R4WklZ0GcP2RzaiYshO1L5mzkb1yhBQo/edit\#}{the
reading} and
\href{https://nuevaschool.instructure.com/courses/3776/modules/items/219697}{the
slides}

\subsubsection{the american dream}
\label{sec:org4bb2d81}
\begin{verbatim}
anyone, regardless of where they were born or what class they were born into, can attain their own version of success in a society in which upward mobility is possible for everyone.
\end{verbatim}

\begin{itemize}
\item nuclear family and heavy community? but also capitalism?
\item expressing oneself is american
\end{itemize}

\begin{verbatim}
is republicanism more intertwined with the founding energy of the country?
\end{verbatim}

classical, most likely.

\begin{quote}
can do spirit - \emph{lauren}
\end{quote}

\begin{itemize}
\item american exceptionalism

\begin{itemize}
\item we are the best, but also uniqe in history
\item outside of the historyicl trajectory
\item allowed because of lack of fedaulism
\end{itemize}
\end{itemize}

\begin{verbatim}
what is best?
\end{verbatim}

\begin{itemize}
\item america in the world
\end{itemize}

\subsubsection{patriotism}
\label{sec:orge7eeea1}
trumpian

\begin{verbatim}
One of the core lessons of Trumpian politics is that Americans are starved for a meaningful politics of what it means to be American. Getting rid of the vainglorious Trump administration is only a partial solution. The causes of his rise remain.
\end{verbatim}

\begin{verbatim}
the left says that america is just crap
\end{verbatim}

he says that the left says that america is iredeamable

why reclaim?

it has been effective

\begin{verbatim}
```ad-question
How might we articulate a sense of nationalism that mitigates the lure of exclusion and ethno-cultural definitions of citizenship? Come up with a pitch for how the idea of America might be used to advance such a vision of nationalism. You can choose to ground this notion of nationalism within a set of leftist policies, conservative policies, and/or be ideologically agnostic.
\end{verbatim}

```
\end{document}
