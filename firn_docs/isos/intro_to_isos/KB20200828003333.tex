% Created 2021-09-27 Mon 12:02
% Intended LaTeX compiler: xelatex
\documentclass[letterpaper]{article}
\usepackage{graphicx}
\usepackage{grffile}
\usepackage{longtable}
\usepackage{wrapfig}
\usepackage{rotating}
\usepackage[normalem]{ulem}
\usepackage{amsmath}
\usepackage{textcomp}
\usepackage{amssymb}
\usepackage{capt-of}
\usepackage{hyperref}
\setlength{\parindent}{0pt}
\usepackage[margin=1in]{geometry}
\usepackage{fontspec}
\usepackage{svg}
\usepackage{cancel}
\usepackage{indentfirst}
\setmainfont[ItalicFont = LiberationSans-Italic, BoldFont = LiberationSans-Bold, BoldItalicFont = LiberationSans-BoldItalic]{LiberationSans}
\newfontfamily\NHLight[ItalicFont = LiberationSansNarrow-Italic, BoldFont       = LiberationSansNarrow-Bold, BoldItalicFont = LiberationSansNarrow-BoldItalic]{LiberationSansNarrow}
\newcommand\textrmlf[1]{{\NHLight#1}}
\newcommand\textitlf[1]{{\NHLight\itshape#1}}
\let\textbflf\textrm
\newcommand\textulf[1]{{\NHLight\bfseries#1}}
\newcommand\textuitlf[1]{{\NHLight\bfseries\itshape#1}}
\usepackage{fancyhdr}
\pagestyle{fancy}
\usepackage{titlesec}
\usepackage{titling}
\makeatletter
\lhead{\textbf{\@title}}
\makeatother
\rhead{\textrmlf{Compiled} \today}
\lfoot{\theauthor\ \textbullet \ \textbf{2021-2022}}
\cfoot{}
\rfoot{\textrmlf{Page} \thepage}
\renewcommand{\tableofcontents}{}
\titleformat{\section} {\Large} {\textrmlf{\thesection} {|}} {0.3em} {\textbf}
\titleformat{\subsection} {\large} {\textrmlf{\thesubsection} {|}} {0.2em} {\textbf}
\titleformat{\subsubsection} {\large} {\textrmlf{\thesubsubsection} {|}} {0.1em} {\textbf}
\setlength{\parskip}{0.45em}
\renewcommand\maketitle{}
\author{Zachary Sayyah}
\date{\today}
\title{Theaetus Reading Notes}
\hypersetup{
 pdfauthor={Zachary Sayyah},
 pdftitle={Theaetus Reading Notes},
 pdfkeywords={},
 pdfsubject={},
 pdfcreator={Emacs 28.0.50 (Org mode 9.4.4)}, 
 pdflang={English}}
\begin{document}

\tableofcontents



\section{Notes}
\label{sec:org01c3f36}
\subsubsection{Highlighted bits}
\label{sec:org70dfc19}
\begin{itemize}
\item One should not believe another person who lacks the qualifications to
have an informed opinion.

\begin{itemize}
\item What is expertise though? What makes a person qualified?
\item One interpretation is that knowledge, or expertise is simply a
perspective.

\begin{itemize}
\item However, that expertise is always true to you, so that cannot be a
good way of defining expertise
\end{itemize}

\item Socrates soon proposes that one can not get knowledge through
experiences, but only through logic.

\begin{itemize}
\item Knowledge is now defined more as true judgment.

\begin{itemize}
\item They dismiss this by saying that knowledge can't be true
judgment since a person can make a true judgment without having
any knowledge.

\begin{itemize}
\item They now narrow this definition to true judgment with some
reasoning behind it.

\begin{itemize}
\item However, this reasoning is based off of knowledge which puts
you into a catch 22 situation
\end{itemize}
\end{itemize}
\end{itemize}
\end{itemize}
\end{itemize}
\end{itemize}

\textbf{Discussion point:} Does knowledge have to be correct, or complete to be
knowledge? The working and used definition of knowledge is less about
what is actually correct as much as what is seemingly correct to the
observer. Even if you're perceiving something incorrectly, that to you
is knowledge.  This is the first definition they discuss and Socrates
deems it incorrect due to the fact that an uninformed person can take
something as knowledge that isn't correct and can be disproved by a more
experienced person. However, the fact remains that to that person, even
that uninformed perception is considered knowledge by them and anyone
else who has the same perception until they deem it disproved. In fact,
things widely considered to be false now in some cases been in the past
considered to be true and common knowledge. To those people who believed
it to be common knowledge, it was common knowledge since they lacked
sufficient evidence/reasoning to change their perspective.
\end{document}
