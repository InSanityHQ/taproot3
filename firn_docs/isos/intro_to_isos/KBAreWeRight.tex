% Created 2021-09-27 Mon 12:02
% Intended LaTeX compiler: xelatex
\documentclass[letterpaper]{article}
\usepackage{graphicx}
\usepackage{grffile}
\usepackage{longtable}
\usepackage{wrapfig}
\usepackage{rotating}
\usepackage[normalem]{ulem}
\usepackage{amsmath}
\usepackage{textcomp}
\usepackage{amssymb}
\usepackage{capt-of}
\usepackage{hyperref}
\setlength{\parindent}{0pt}
\usepackage[margin=1in]{geometry}
\usepackage{fontspec}
\usepackage{svg}
\usepackage{cancel}
\usepackage{indentfirst}
\setmainfont[ItalicFont = LiberationSans-Italic, BoldFont = LiberationSans-Bold, BoldItalicFont = LiberationSans-BoldItalic]{LiberationSans}
\newfontfamily\NHLight[ItalicFont = LiberationSansNarrow-Italic, BoldFont       = LiberationSansNarrow-Bold, BoldItalicFont = LiberationSansNarrow-BoldItalic]{LiberationSansNarrow}
\newcommand\textrmlf[1]{{\NHLight#1}}
\newcommand\textitlf[1]{{\NHLight\itshape#1}}
\let\textbflf\textrm
\newcommand\textulf[1]{{\NHLight\bfseries#1}}
\newcommand\textuitlf[1]{{\NHLight\bfseries\itshape#1}}
\usepackage{fancyhdr}
\pagestyle{fancy}
\usepackage{titlesec}
\usepackage{titling}
\makeatletter
\lhead{\textbf{\@title}}
\makeatother
\rhead{\textrmlf{Compiled} \today}
\lfoot{\theauthor\ \textbullet \ \textbf{2021-2022}}
\cfoot{}
\rfoot{\textrmlf{Page} \thepage}
\renewcommand{\tableofcontents}{}
\titleformat{\section} {\Large} {\textrmlf{\thesection} {|}} {0.3em} {\textbf}
\titleformat{\subsection} {\large} {\textrmlf{\thesubsection} {|}} {0.2em} {\textbf}
\titleformat{\subsubsection} {\large} {\textrmlf{\thesubsubsection} {|}} {0.1em} {\textbf}
\setlength{\parskip}{0.45em}
\renewcommand\maketitle{}
\author{Zachary Sayyah}
\date{\today}
\title{How do we know we're right about climate change}
\hypersetup{
 pdfauthor={Zachary Sayyah},
 pdftitle={How do we know we're right about climate change},
 pdfkeywords={},
 pdfsubject={},
 pdfcreator={Emacs 28.0.50 (Org mode 9.4.4)}, 
 pdflang={English}}
\begin{document}

\tableofcontents



\section{Notes}
\label{sec:orgd838e8d}
\subsubsection{Do most people believe in it?}
\label{sec:org15195e5}
\begin{itemize}
\item There are still scientists that are divided about the issue of climate
change, but they are few and far between

\begin{itemize}
\item The Majority of Americans think that scientists \textbf{are} divided about
climate change
\end{itemize}

\item Only a little more than half of Americans think that global
temperatures have risen

\begin{itemize}
\item This is concerning since it is not really a concept that is under
heavy scientific debate
\end{itemize}
\end{itemize}

\subsubsection{Are we Sure it's Real?}
\label{sec:org08eb5fc}
\begin{itemize}
\item Could we be wrong on climate change? We've been wrong in the past even
when almost the entire scientific community has a general consensus

\begin{itemize}
\item Before the twentieth century it was much easier to publish less
supported conclusions

\begin{itemize}
\item The scientific process in general has gotten stricter and more
accurate
\end{itemize}
\end{itemize}

\item In the case of climate science since political action is required,
there has been an unusual effort to make new scientific discovery
about this subject accessible to everyone and not just experts
\item Scientists predicted a long time ago that greenhouse gas emissions
could change the climate and there is now overwhelming evidence that
the climate is changing
\end{itemize}

\#\#\# Why is there Uncertainty In the General Public - Political
uncertainty is different from scientific uncertainty - A lot of people
may be uninformed for this reason - Sometimes it is hard to make a
distinction between uncertainty in future events based upon current
scientific understanding and understanding in the current situation - A
lot of scientists see their jobs as knowledge finders/creators and not
communicators of this knowledge - Some even sneer at those who are
trying to communicate the consensus calling them populerizers(I give up
on spelling).

\#\#\# What now - While no scientific conclusion is provable, our current
understanding is our best and relatively strong - It is therefore the
most logical to take action on - However, there is no one measurement of
scientific reliability

\subsection{C Plus Plus that Reflects my Current Feelings on this}
\label{sec:orgad511d1}
\begin{verbatim}
#include <iostream>

int main()
{
    std::cout << "Why does isos post readings so late :(" << std::endl;
}
\end{verbatim}

The above code is written in C++ and has been scientifically proven to
be better than it's C\# equivalent. This has caused uncertainty about the
future of climate simulations and in extension the future of our lives
as most simulation type technologies are written in C\#.
\end{document}
