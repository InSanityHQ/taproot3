% Created 2021-09-27 Mon 12:02
% Intended LaTeX compiler: xelatex
\documentclass[letterpaper]{article}
\usepackage{graphicx}
\usepackage{grffile}
\usepackage{longtable}
\usepackage{wrapfig}
\usepackage{rotating}
\usepackage[normalem]{ulem}
\usepackage{amsmath}
\usepackage{textcomp}
\usepackage{amssymb}
\usepackage{capt-of}
\usepackage{hyperref}
\setlength{\parindent}{0pt}
\usepackage[margin=1in]{geometry}
\usepackage{fontspec}
\usepackage{svg}
\usepackage{cancel}
\usepackage{indentfirst}
\setmainfont[ItalicFont = LiberationSans-Italic, BoldFont = LiberationSans-Bold, BoldItalicFont = LiberationSans-BoldItalic]{LiberationSans}
\newfontfamily\NHLight[ItalicFont = LiberationSansNarrow-Italic, BoldFont       = LiberationSansNarrow-Bold, BoldItalicFont = LiberationSansNarrow-BoldItalic]{LiberationSansNarrow}
\newcommand\textrmlf[1]{{\NHLight#1}}
\newcommand\textitlf[1]{{\NHLight\itshape#1}}
\let\textbflf\textrm
\newcommand\textulf[1]{{\NHLight\bfseries#1}}
\newcommand\textuitlf[1]{{\NHLight\bfseries\itshape#1}}
\usepackage{fancyhdr}
\pagestyle{fancy}
\usepackage{titlesec}
\usepackage{titling}
\makeatletter
\lhead{\textbf{\@title}}
\makeatother
\rhead{\textrmlf{Compiled} \today}
\lfoot{\theauthor\ \textbullet \ \textbf{2021-2022}}
\cfoot{}
\rfoot{\textrmlf{Page} \thepage}
\renewcommand{\tableofcontents}{}
\titleformat{\section} {\Large} {\textrmlf{\thesection} {|}} {0.3em} {\textbf}
\titleformat{\subsection} {\large} {\textrmlf{\thesubsection} {|}} {0.2em} {\textbf}
\titleformat{\subsubsection} {\large} {\textrmlf{\thesubsubsection} {|}} {0.1em} {\textbf}
\setlength{\parskip}{0.45em}
\renewcommand\maketitle{}
\author{Huxley}
\date{\today}
\title{science - not broke?}
\hypersetup{
 pdfauthor={Huxley},
 pdftitle={science - not broke?},
 pdfkeywords={},
 pdfsubject={},
 pdfcreator={Emacs 28.0.50 (Org mode 9.4.4)}, 
 pdflang={English}}
\begin{document}

\tableofcontents

\#flo \#disorganized

\noindent\rule{\textwidth}{0.5pt}

\section{starting on part 4? ok}
\label{sec:org358526a}
Science isnt broken its just low yield

talks about the emotional requirments, ie. beliving that this / the next
idea is the right one to maintainn motiviation

isnt this inherently flawed?

relatively empty reading

\section{Discusstion point}
\label{sec:orgfa9ad4b}
The root of this problem, human error and bias, is only becoming
magnified as time goes on. As media becomes our main source of truth,
and our entertainment becomes media's main source of income, our society
as a whole becomes detached from truth. It doesn't necessarily stray
away from truth, as the truth becomes entirely irrelevant. The truth
becomes irrelevant in the face of the catchy headline, whether or not
that headline aligns with the truth. This detachment is not limited to
scientific findings, but rather to anything that is supposedly based in
truth. We see it in our science, we see it in our entertainment, we see
it in our politics, and soon we may see it in our thoughts.
\end{document}
