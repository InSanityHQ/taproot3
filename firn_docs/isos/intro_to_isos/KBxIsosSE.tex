% Created 2021-09-27 Mon 12:02
% Intended LaTeX compiler: xelatex
\documentclass[letterpaper]{article}
\usepackage{graphicx}
\usepackage{grffile}
\usepackage{longtable}
\usepackage{wrapfig}
\usepackage{rotating}
\usepackage[normalem]{ulem}
\usepackage{amsmath}
\usepackage{textcomp}
\usepackage{amssymb}
\usepackage{capt-of}
\usepackage{hyperref}
\setlength{\parindent}{0pt}
\usepackage[margin=1in]{geometry}
\usepackage{fontspec}
\usepackage{svg}
\usepackage{cancel}
\usepackage{indentfirst}
\setmainfont[ItalicFont = LiberationSans-Italic, BoldFont = LiberationSans-Bold, BoldItalicFont = LiberationSans-BoldItalic]{LiberationSans}
\newfontfamily\NHLight[ItalicFont = LiberationSansNarrow-Italic, BoldFont       = LiberationSansNarrow-Bold, BoldItalicFont = LiberationSansNarrow-BoldItalic]{LiberationSansNarrow}
\newcommand\textrmlf[1]{{\NHLight#1}}
\newcommand\textitlf[1]{{\NHLight\itshape#1}}
\let\textbflf\textrm
\newcommand\textulf[1]{{\NHLight\bfseries#1}}
\newcommand\textuitlf[1]{{\NHLight\bfseries\itshape#1}}
\usepackage{fancyhdr}
\pagestyle{fancy}
\usepackage{titlesec}
\usepackage{titling}
\makeatletter
\lhead{\textbf{\@title}}
\makeatother
\rhead{\textrmlf{Compiled} \today}
\lfoot{\theauthor\ \textbullet \ \textbf{2021-2022}}
\cfoot{}
\rfoot{\textrmlf{Page} \thepage}
\renewcommand{\tableofcontents}{}
\titleformat{\section} {\Large} {\textrmlf{\thesection} {|}} {0.3em} {\textbf}
\titleformat{\subsection} {\large} {\textrmlf{\thesubsection} {|}} {0.2em} {\textbf}
\titleformat{\subsubsection} {\large} {\textrmlf{\thesubsubsection} {|}} {0.1em} {\textbf}
\setlength{\parskip}{0.45em}
\renewcommand\maketitle{}
\author{Huxley}
\date{\today}
\title{ISOS Self Eval}
\hypersetup{
 pdfauthor={Huxley},
 pdftitle={ISOS Self Eval},
 pdfkeywords={},
 pdfsubject={},
 pdfcreator={Emacs 28.0.50 (Org mode 9.4.4)}, 
 pdflang={English}}
\begin{document}

\tableofcontents

\#ret

\noindent\rule{\textwidth}{0.5pt}

\begin{itemize}
\item We read difficult texts in this class. What techniques did you use for
getting the most out of readings and for formulating useful discussion
points?

\begin{itemize}
\item I approached the readings without assuming that what I was reading
was 'thruthful.' This led me to analyse the texts more deeply, and
forced me to only accept ideas that I had thought through. For
discussion points, I tried to approach them with curiosity and ask
one of many genuine questions that I was left with after doing the
reading.
\end{itemize}

\item Our class is dependent on students engaging in discussions. What kind
of role do you play in discussion, and what helps you be better at
that role? What goals do you have for yourself in next semester's
discussions?

\begin{itemize}
\item I would like to think I provide unique viewpoints to the discussion.
Sometimes, unique viewpoints which reveal assumptions don't allow
the class to progress through the material at the desired pace. Next
semester, I hope to find a balance.
\end{itemize}

\item What kind of role did you play in your group project, the Bad Science
presentation? What goals do you have for yourself in future group
projects?

\begin{itemize}
\item I played the role of a person working equally with their friends. We
all contributed, we all had fun, and we all learned. I would like to
continue this in the future.
\end{itemize}

\item How might you apply what you learned in ISOS this fall to being a
scientist in your science classes or to using scientific information
to make informed choices as a citizen?

\begin{itemize}
\item I am now a Postmodernist. There is no subjective truth. Nothing is
real. Seriously speaking, I think that not recognizing ones own
assumptions or taking some truth for granted is incredibly
dangerous, and ISOS is a class that helps us avoid this danger. More
specifically, perhaps the most interesting idea we have discussed is
the idea of robustness, and how we can view ideas as a sort of
interconnected web; as this web moves away from being linear and
becomes increasingly interconnected, one idea in the web being
proved 'false' is much less destructive and can be remedied much
more quickly. This is a way I will try to design my own systems in
the future.
\end{itemize}
\end{itemize}
\end{document}
