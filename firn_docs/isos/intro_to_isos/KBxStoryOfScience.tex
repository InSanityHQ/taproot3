% Created 2021-09-27 Mon 12:02
% Intended LaTeX compiler: xelatex
\documentclass[letterpaper]{article}
\usepackage{graphicx}
\usepackage{grffile}
\usepackage{longtable}
\usepackage{wrapfig}
\usepackage{rotating}
\usepackage[normalem]{ulem}
\usepackage{amsmath}
\usepackage{textcomp}
\usepackage{amssymb}
\usepackage{capt-of}
\usepackage{hyperref}
\setlength{\parindent}{0pt}
\usepackage[margin=1in]{geometry}
\usepackage{fontspec}
\usepackage{svg}
\usepackage{cancel}
\usepackage{indentfirst}
\setmainfont[ItalicFont = LiberationSans-Italic, BoldFont = LiberationSans-Bold, BoldItalicFont = LiberationSans-BoldItalic]{LiberationSans}
\newfontfamily\NHLight[ItalicFont = LiberationSansNarrow-Italic, BoldFont       = LiberationSansNarrow-Bold, BoldItalicFont = LiberationSansNarrow-BoldItalic]{LiberationSansNarrow}
\newcommand\textrmlf[1]{{\NHLight#1}}
\newcommand\textitlf[1]{{\NHLight\itshape#1}}
\let\textbflf\textrm
\newcommand\textulf[1]{{\NHLight\bfseries#1}}
\newcommand\textuitlf[1]{{\NHLight\bfseries\itshape#1}}
\usepackage{fancyhdr}
\pagestyle{fancy}
\usepackage{titlesec}
\usepackage{titling}
\makeatletter
\lhead{\textbf{\@title}}
\makeatother
\rhead{\textrmlf{Compiled} \today}
\lfoot{\theauthor\ \textbullet \ \textbf{2021-2022}}
\cfoot{}
\rfoot{\textrmlf{Page} \thepage}
\renewcommand{\tableofcontents}{}
\titleformat{\section} {\Large} {\textrmlf{\thesection} {|}} {0.3em} {\textbf}
\titleformat{\subsection} {\large} {\textrmlf{\thesubsection} {|}} {0.2em} {\textbf}
\titleformat{\subsubsection} {\large} {\textrmlf{\thesubsubsection} {|}} {0.1em} {\textbf}
\setlength{\parskip}{0.45em}
\renewcommand\maketitle{}
\author{Huxley}
\date{\today}
\title{Story Of Science Planning}
\hypersetup{
 pdfauthor={Huxley},
 pdftitle={Story Of Science Planning},
 pdfkeywords={},
 pdfsubject={},
 pdfcreator={Emacs 28.0.50 (Org mode 9.4.4)}, 
 pdflang={English}}
\begin{document}

\tableofcontents

\#ref \#ret

\noindent\rule{\textwidth}{0.5pt}

\section{Lets go.}
\label{sec:org6f2994f}
\subsection{Story of Haber | rough notes}
\label{sec:orgac930e3}
\begin{itemize}
\item \textbf{Haber invented the Haber-Bosch process, responsible for artificial
nitrogen fixation}

\item used in fertilizers, incredibly important

\begin{itemize}
\item lack of fixed nitrogen is actually one of the obstacles that origin
of life theories hypothesize around
\item N2 makes up around 70\% of the air we breath, but is crazy hard to
fix into usable nitrogen because of its strong triple bond
\end{itemize}

\item Without the process, a third of the world would be unfed

\item Process:
\({\displaystyle {\ce {N2 + 3 H2 -> 2 NH3}}\quad \Delta H^{\circ }=-91.8~{\text{kJ/mol}}}\)

\item *But, Haber was an \emph{evil} guy. *

\begin{itemize}
\item led the German chemical warfare teams in WWI

\item He invented poison gas used in world war one - And, thus, created
Haber's rule: \(c\times t = k\) - c: concentration, t: time to
breath given an effect to produce, k: constant based on the poison
itself

\begin{itemize}
\item Also created poisons used in pesticides
\item And created the poison gas the Nazi's used to murder his own
relatives
\end{itemize}

\item Wife committed suicide after a fight with Haber\ldots{}.
\end{itemize}

\item \textbf{Caused so much pain and suffering, was an evil person. And yet, is
responsible for the life of a third of the human race}

\begin{itemize}
\item how do we think about this? how does morality handle a situation
like that?
\item dark, but interesting story, interesting context, interesting
science!
\end{itemize}
\end{itemize}

\subsection{Part two! (of part one)}
\label{sec:org22ce610}
\begin{verbatim}
Assignment 1b - Research the context and the science: Research and write
some notes/analysis to help you cement your understanding of 

1) the context of the time period, place, and set of people of your moment; 

and 2) the science of your story. Make sure you know the details of the arguments
and claims, for this really makes a good history of science story. Some of
them are absurd; some of them seem absurd, and are genius!
\end{verbatim}

\subsubsection{Context}
\label{sec:org28572f3}
\begin{itemize}
\item \textbf{Haber}

\begin{itemize}
\item trained as organic chemist
\item switched to physical chem

\begin{itemize}
\item mainly industrial processes
\end{itemize}

\item wrote a book,
\texttt{Experimental Investigations on the Decomposition and Combustion of Hydrocarbons}
\item also wrote: \texttt{The Theoretical Basis of Technical Electrochemistry}
\item worked on nitrobenzeins, the hydrogen oxygen fuel cell, and the
glass electrode, then wrote
\texttt{“The Electrolytic Processes of Organic Chemistry}
\item got intrested in thermodynamics, wrote
\texttt{The Thermodynamics of Technical Gas Reactions}
\item then he worked on nitrogen fixation! got a nobel for it
\end{itemize}

\item \textbf{WWI}

\begin{itemize}
\item entirely devoted research and resources to germany during wartime
preparation
\item nitrates could be used in explosives
\item pioneered chemical warfare, chlorine gas, ect.
\item became Chief of Germany's Chemical Warfare Service
\item \textbf{General}

\begin{itemize}
\item their was a period of high tension before ww1, eventually set off
by the assination of franz ferdinand
\item germany was growing rapidly, caused a security dillema

\begin{itemize}
\item everyone has massive armies sitting and building, which in turn
caused others to build armies before wartime
\end{itemize}

\item in germany, their was a sentiment of wanting to be in the
spotlight
\item also, general consensus was that war was inevitable when a state
grows, but also good for states.
\item during the war, trench warfare made poisan gas incredibly
effective
\end{itemize}
\end{itemize}
\end{itemize}

\subsubsection{The actual process}
\label{sec:org4915392}
\begin{itemize}
\item \textbf{Overview}

\begin{itemize}
\item using high pressures and catalysts
\item fixing nitrogen from the air with hydrogen to produce amonia
\item inside container normally made form rthenium or iron

\begin{itemize}
\item temp of >425c, psi > 200
\end{itemize}

\item converted to fluid ammonia
\end{itemize}

\item \textbf{Closer look}

\begin{itemize}
\item \({\displaystyle \ \mathrm {N\_{2}} +3\ \mathrm {H\_{2}} \quad \rightleftharpoons \quad \ 2\ \mathrm {NH\_{3}} \quad \quad {\Delta H=-92.28\;\mathrm {kJ} }\ ({\Delta H\_{298K}=-46.14\;\\mathrm {kJ/mol} })}![{\displaystyle \ \mathrm {N_{2}} +3\ \mathrm {H_{2}} \quad \rightleftharpoons \quad \ 2\ \mathrm {NH_{3}} \quad \quad {\Delta H=-92.28\;\mathrm {kJ} }\ ({\Delta H_{298K}=-46.14\;\mathrm {kJ/mol} })}]\)
\item exothermic, of course
\item equilibrium const:
\({\displaystyle K_{eq}={\frac {p^{2}\mathrm {(NH_{3})} }{p\mathrm {(N_{2})} \cdot p^{3}\mathrm {(H_{2})} }}}\)
\end{itemize}
\end{itemize}

\subsection{First draft.}
\label{sec:orge289037}
idea: yin yang, with infographic

\begin{enumerate}
\item yin:
\label{sec:org8182f1a}
Invented the Haber-Bosch Process

Process of taking nitrogen in the air and "fixing" it

\({\displaystyle \ \mathrm {N_{2}} +3\ \mathrm {H_{2}} \quad \rightleftharpoons \quad \ 2\ \mathrm {NH_{3}} \quad \quad {\Delta H=-92.28\;\mathrm {kJ} }\ ({\Delta H_{298K}=-46.14\;\mathrm {kJ/mol} })}\)

This way, it can be used in fertilizers

Previously, the main source was bat guano, which was treated like gold

Arguably one of the most important inventions ever

50\% of the worlds food production is reliant on it

Billions of lives are reliant on it

Haber won a Nobel Prize for it

\item yang:
\label{sec:org40b30eb}
During WWI, used the same process for explosives

Led the team pioneering chemical weapons

Invented chlorine gas, causing horrible horrible deaths and extending
the war

Was thought of as a war criminal

When he went home, his wife thought he was immoral and they fought

He ignored her and threw a party. Overnight, she took her own life.

Their son found his dead mom in the morning, and Haber left him alone
and went to work.

His son later also committed suicide

Later on, the same chemical weapons he discovered were used by the Nazis

It was also used to kill many of Haber's own family members in the gas
chambers
\end{enumerate}

\subsection{Reflection thingy}
\label{sec:orgddf567d}
My peer reviewer did not leave a significant quantity of actionable
constructive criticism. Most of the questions were answered with "yes."
The feedback that I did act on was giving names to the categories of the
text. I also, as suggested, changed the background imagery and the yin
yang symbolism. I think thiss feedback was valuable and made the final
product better in the end.
\end{document}
