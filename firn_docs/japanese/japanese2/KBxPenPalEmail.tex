% Created 2021-09-27 Mon 12:01
% Intended LaTeX compiler: xelatex
\documentclass[letterpaper]{article}
\usepackage{graphicx}
\usepackage{grffile}
\usepackage{longtable}
\usepackage{wrapfig}
\usepackage{rotating}
\usepackage[normalem]{ulem}
\usepackage{amsmath}
\usepackage{textcomp}
\usepackage{amssymb}
\usepackage{capt-of}
\usepackage{hyperref}
\setlength{\parindent}{0pt}
\usepackage[margin=1in]{geometry}
\usepackage{fontspec}
\usepackage{svg}
\usepackage{cancel}
\usepackage{indentfirst}
\setmainfont[ItalicFont = LiberationSans-Italic, BoldFont = LiberationSans-Bold, BoldItalicFont = LiberationSans-BoldItalic]{LiberationSans}
\newfontfamily\NHLight[ItalicFont = LiberationSansNarrow-Italic, BoldFont       = LiberationSansNarrow-Bold, BoldItalicFont = LiberationSansNarrow-BoldItalic]{LiberationSansNarrow}
\newcommand\textrmlf[1]{{\NHLight#1}}
\newcommand\textitlf[1]{{\NHLight\itshape#1}}
\let\textbflf\textrm
\newcommand\textulf[1]{{\NHLight\bfseries#1}}
\newcommand\textuitlf[1]{{\NHLight\bfseries\itshape#1}}
\usepackage{fancyhdr}
\pagestyle{fancy}
\usepackage{titlesec}
\usepackage{titling}
\makeatletter
\lhead{\textbf{\@title}}
\makeatother
\rhead{\textrmlf{Compiled} \today}
\lfoot{\theauthor\ \textbullet \ \textbf{2021-2022}}
\cfoot{}
\rfoot{\textrmlf{Page} \thepage}
\renewcommand{\tableofcontents}{}
\titleformat{\section} {\Large} {\textrmlf{\thesection} {|}} {0.3em} {\textbf}
\titleformat{\subsection} {\large} {\textrmlf{\thesubsection} {|}} {0.2em} {\textbf}
\titleformat{\subsubsection} {\large} {\textrmlf{\thesubsubsection} {|}} {0.1em} {\textbf}
\setlength{\parskip}{0.45em}
\renewcommand\maketitle{}
\author{Huxley}
\date{\today}
\title{Pen Pen Email}
\hypersetup{
 pdfauthor={Huxley},
 pdftitle={Pen Pen Email},
 pdfkeywords={},
 pdfsubject={},
 pdfcreator={Emacs 28.0.50 (Org mode 9.4.4)}, 
 pdflang={English}}
\begin{document}

\tableofcontents

\#ret \#incomplete

\noindent\rule{\textwidth}{0.5pt}

\section{Desc}
\label{sec:org80e6451}
\url{https://docs.google.com/document/d/1n-AR4qiePq5XU96Gq5vSlWiHlzrEFTt7VU6ESKZ6VZE/edit}

\section{English}
\label{sec:org8ecee08}
Me: 1. I am 15 and I have six people in my family; I live in California
in the U.S. 2. My favorite sports are soccer, biking, and table
tennis. 3. My favorite classes are Machine Learning, Physics, and
Calculus. 4. I want to visit Japan, but I can't because of covid. 5. I
love to eat japanese food, but it's not very good in america.

!Me:

\begin{enumerate}
\item How many people are in your family? How old are you?
\item Have you ever had a part time job? If yes, what job? Did you like it?
\item Why do you like PE, English, and History? How many times do you have
each class every week?
\item What type of food do you like? Have you tried things like blue cheese
and burritos?
\item How many hours of homework do you have every day?
\end{enumerate}

\section{Japanese}
\label{sec:orgc091c65}
はじめまして、ハクスリーともうします。

こんかいは、いろいろなしつもんにこたえてくれて、ほんとうにありがとうございます。たいへんたすかります。

私は十五さいです。かぞくは六にんです。カリフォルニアにすんでいます。

スポーツにかんしては、サッカーとサイクリングとピンポンがすきです。

いちばんすきなクラスは、きかいがくしゅうとぶつりとびせきぶんです。

日本にいきたいんですが、コロナウイルスのせいでまだいけないんです。日本で、かんこうしたり、おいしいものをたべたりしたいです。

日本りょうりがだいすきですが、アメリカでの日本りょうりはあまりよくないんです。

川西さんはなんさいですか。かぞくはなんにんですか。

アルバイトをしたことがありますか。あるなら、どんなアルバイトでしたか。すきですか。

なぜ体育と英語と歴史がすきですか。しゅうになんかいクラスがありますか。

どんなたべものがすきですか。ブルーチーズやブリトーをたべたことがありますか。

まいにち、なんじかんぐらいしゅくだいをしますか。

日本語と英語と両方でこたえてください。できれば3月末までにこたえてもらえるとたすかります。

よろしくおねがいします。

ハクスリー
\end{document}
