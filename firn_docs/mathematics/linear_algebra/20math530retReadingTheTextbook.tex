% Created 2021-09-27 Mon 12:03
% Intended LaTeX compiler: xelatex
\documentclass[letterpaper]{article}
\usepackage{graphicx}
\usepackage{grffile}
\usepackage{longtable}
\usepackage{wrapfig}
\usepackage{rotating}
\usepackage[normalem]{ulem}
\usepackage{amsmath}
\usepackage{textcomp}
\usepackage{amssymb}
\usepackage{capt-of}
\usepackage{hyperref}
\setlength{\parindent}{0pt}
\usepackage[margin=1in]{geometry}
\usepackage{fontspec}
\usepackage{svg}
\usepackage{cancel}
\usepackage{indentfirst}
\setmainfont[ItalicFont = LiberationSans-Italic, BoldFont = LiberationSans-Bold, BoldItalicFont = LiberationSans-BoldItalic]{LiberationSans}
\newfontfamily\NHLight[ItalicFont = LiberationSansNarrow-Italic, BoldFont       = LiberationSansNarrow-Bold, BoldItalicFont = LiberationSansNarrow-BoldItalic]{LiberationSansNarrow}
\newcommand\textrmlf[1]{{\NHLight#1}}
\newcommand\textitlf[1]{{\NHLight\itshape#1}}
\let\textbflf\textrm
\newcommand\textulf[1]{{\NHLight\bfseries#1}}
\newcommand\textuitlf[1]{{\NHLight\bfseries\itshape#1}}
\usepackage{fancyhdr}
\pagestyle{fancy}
\usepackage{titlesec}
\usepackage{titling}
\makeatletter
\lhead{\textbf{\@title}}
\makeatother
\rhead{\textrmlf{Compiled} \today}
\lfoot{\theauthor\ \textbullet \ \textbf{2021-2022}}
\cfoot{}
\rfoot{\textrmlf{Page} \thepage}
\renewcommand{\tableofcontents}{}
\titleformat{\section} {\Large} {\textrmlf{\thesection} {|}} {0.3em} {\textbf}
\titleformat{\subsection} {\large} {\textrmlf{\thesubsection} {|}} {0.2em} {\textbf}
\titleformat{\subsubsection} {\large} {\textrmlf{\thesubsubsection} {|}} {0.1em} {\textbf}
\setlength{\parskip}{0.45em}
\renewcommand\maketitle{}
\author{Exr0n}
\date{\today}
\title{ret Reading The Textbook!}
\hypersetup{
 pdfauthor={Exr0n},
 pdftitle={ret Reading The Textbook!},
 pdfkeywords={},
 pdfsubject={},
 pdfcreator={Emacs 28.0.50 (Org mode 9.4.4)}, 
 pdflang={English}}
\begin{document}

\tableofcontents

\#ret

\section{Exercises}
\label{sec:orga4d32a0}
\subsection{1.A.2}
\label{sec:org43526ea}
$\backslash$[
\begin{split}
\left(\frac{-1+\sqrt{3}i}{2}\right)^3 =
\left(\frac{-1+\sqrt{3}i}{2}\right)\left(\frac{-2-2\sqrt{3}i}{4}\right) =
\frac{2+\cancel{2\sqrt{3}i-2\sqrt{3}i}-2*3*i^2}{8} = \frac{8}{8} = 1
\end{split}
$\backslash$]

\subsection{1.A.10}
\label{sec:org23bf89b}
$\backslash$[
\begin{aligned}
(4, -3, 1, 7) + 2(x_1, x_2, x_3, x_4) = (5, 9, -6, 8)\\
4+2x_1 &= 5,\\-3+2x_2 &= 9,\\1+2x_3 &= -6,\\7+2x_4 &= 8\\
x &= (\frac{1}{2}, 6, \frac{-7}{2}, \frac{1}{2})\\
\end{aligned}
$\backslash$] Not sure how to do this with matrices?

\subsection{1.A.15}
\label{sec:org3528f60}
$\backslash$[
\begin{split}
\lambda(x+y)\\
&=\lambda(x_1+y_1, x_2+y_2, x_3+y_3\ ...\ x_n+y_n)\\
&=(\lambda(x_1+y_1), \lambda(x_2+y_2), \lambda(x_3+y_3)\  ...\ \lambda(x_n+y_n))\\
&=(\lambda x_1 + \lambda y_1, \lambda x_2 + \lambda y_2, \lambda x_3 + \lambda y_3 \ ...\ \lambda x_n + \lambda y_n)\\
&=(\lambda x_1, \lambda x_2, \lambda x_3\ ...\ \lambda x_n) + (\lambda y_1, \lambda y_2, \lambda y_3\ ...\ \lambda y_n)\\
&= \lambda(x) + \lambda(y) = \lambda x + \lambda y
\end{split}
$\backslash$]

\section{Matrices for Solving Systems}
\label{sec:orgbc77136}
I'm not sure what I should notice, although it's interesting that they
are all 2x2 matrices that are (or can be decomposed into) one number
away from the identity. I think we mentioned that they were "essential
matrices" or something?

\section{Geometric Interpretation of Dot Product}
\label{sec:org72e5941}
We talked about it in class, and learned it in physics, but a dot
product \(A \cdot B\) can be interpreted as the magnitude of \(A\)'s
projection onto \(B\) multiplied by the magnitude of \(B\).
\(A \cdot B = |A||B|cos\theta\).

\section{Dot Product on Vectors as a Group}
\label{sec:org1d9a2ef}
No. Dot product returns a scalar, which means that this operation is
distinctly not closed.

After class on 3 Sep, Daniel mentioned that it might be a group if you
define a modified dot product where you take the normal dot product and
put it in the direction of the second vector. However, this doesn't work
because the for any given \(N\)x\(1\) matrixe \(A\) the identity \(e\)
has to satisfy \(A\cdot e = e \cdot A = A\). Thus, the definition that
relies on the direction of the second operand will break when the
identity is on one of the sides. *Because dot product relies on the
angle between the two vectors, I think it would be difficult to find an
angle for an identity vector that works with all other angles of
vectors. I'm not sure how to formalize this\ldots{}* \#todo

\section{Inverse of a matrix}
\label{sec:org727fbba}
I tried this for the previous homework when we were to determine if 2x2
matrices were groups under multiplication, but didn't end up getting
anywhere. I will try again\ldots{}

\href{srcIdentityMatrixFormula.png.org}{srcIdentityMatrixFormula.png}

I got something like \(w = \frac{1-\frac{bc}{bc-ad}}{a}\), which I don't
think is correct. It's also been an hour and a half, so I think I'll
have to leave this here for now. \#todo

\noindent\rule{\textwidth}{0.5pt}
\end{document}
