% Created 2021-09-27 Mon 12:03
% Intended LaTeX compiler: xelatex
\documentclass[letterpaper]{article}
\usepackage{graphicx}
\usepackage{grffile}
\usepackage{longtable}
\usepackage{wrapfig}
\usepackage{rotating}
\usepackage[normalem]{ulem}
\usepackage{amsmath}
\usepackage{textcomp}
\usepackage{amssymb}
\usepackage{capt-of}
\usepackage{hyperref}
\setlength{\parindent}{0pt}
\usepackage[margin=1in]{geometry}
\usepackage{fontspec}
\usepackage{svg}
\usepackage{cancel}
\usepackage{indentfirst}
\setmainfont[ItalicFont = LiberationSans-Italic, BoldFont = LiberationSans-Bold, BoldItalicFont = LiberationSans-BoldItalic]{LiberationSans}
\newfontfamily\NHLight[ItalicFont = LiberationSansNarrow-Italic, BoldFont       = LiberationSansNarrow-Bold, BoldItalicFont = LiberationSansNarrow-BoldItalic]{LiberationSansNarrow}
\newcommand\textrmlf[1]{{\NHLight#1}}
\newcommand\textitlf[1]{{\NHLight\itshape#1}}
\let\textbflf\textrm
\newcommand\textulf[1]{{\NHLight\bfseries#1}}
\newcommand\textuitlf[1]{{\NHLight\bfseries\itshape#1}}
\usepackage{fancyhdr}
\pagestyle{fancy}
\usepackage{titlesec}
\usepackage{titling}
\makeatletter
\lhead{\textbf{\@title}}
\makeatother
\rhead{\textrmlf{Compiled} \today}
\lfoot{\theauthor\ \textbullet \ \textbf{2021-2022}}
\cfoot{}
\rfoot{\textrmlf{Page} \thepage}
\renewcommand{\tableofcontents}{}
\titleformat{\section} {\Large} {\textrmlf{\thesection} {|}} {0.3em} {\textbf}
\titleformat{\subsection} {\large} {\textrmlf{\thesubsection} {|}} {0.2em} {\textbf}
\titleformat{\subsubsection} {\large} {\textrmlf{\thesubsubsection} {|}} {0.1em} {\textbf}
\setlength{\parskip}{0.45em}
\renewcommand\maketitle{}
\author{Exr0n}
\date{\today}
\title{Linear Independence}
\hypersetup{
 pdfauthor={Exr0n},
 pdftitle={Linear Independence},
 pdfkeywords={},
 pdfsubject={},
 pdfcreator={Emacs 28.0.50 (Org mode 9.4.4)}, 
 pdflang={English}}
\begin{document}

\tableofcontents



\section{Overview}
\label{sec:org598d8c4}
\subsection{Intuition}
\label{sec:orgb3d3cc3}
\begin{itemize}
\item If you have a vector list \(v\) that is a linear combination of
vectors in \(V\), or equivalently,
\item \[v = a_1v_1 + ... + a_mv_m \text{where} v_1, ..., v_m \in V\]
\item And those choices of \(a_1, ..., a_m\) are unique, then this is a
linear independence? \#\# \#definition linearly independent > - The empty
list \(()\) is linearly independent > - A list \(v_1, ..., v_m\) of
vectors in \(V\) is called \textbf{\emph{linearly independent}} if the only choice
of \(a_1, ..., a_m \in F\) that makes \(a_1v_1 + ... + a_mv_m\) equal
\(0\) is \(a_1 = ... = a_m = 0\)
\item \^{}\^{}\^{} what the heck is that last part about everything equaling \(0\)??
\#todo-exr0n
\href{KBe20math530floQuestions.org}{KBe20math530floQuestions}
\end{itemize}

\noindent\rule{\textwidth}{0.5pt}
\end{document}
