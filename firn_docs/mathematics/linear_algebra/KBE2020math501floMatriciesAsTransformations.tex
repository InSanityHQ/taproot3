% Created 2021-09-27 Mon 12:03
% Intended LaTeX compiler: xelatex
\documentclass[letterpaper]{article}
\usepackage{graphicx}
\usepackage{grffile}
\usepackage{longtable}
\usepackage{wrapfig}
\usepackage{rotating}
\usepackage[normalem]{ulem}
\usepackage{amsmath}
\usepackage{textcomp}
\usepackage{amssymb}
\usepackage{capt-of}
\usepackage{hyperref}
\setlength{\parindent}{0pt}
\usepackage[margin=1in]{geometry}
\usepackage{fontspec}
\usepackage{svg}
\usepackage{cancel}
\usepackage{indentfirst}
\setmainfont[ItalicFont = LiberationSans-Italic, BoldFont = LiberationSans-Bold, BoldItalicFont = LiberationSans-BoldItalic]{LiberationSans}
\newfontfamily\NHLight[ItalicFont = LiberationSansNarrow-Italic, BoldFont       = LiberationSansNarrow-Bold, BoldItalicFont = LiberationSansNarrow-BoldItalic]{LiberationSansNarrow}
\newcommand\textrmlf[1]{{\NHLight#1}}
\newcommand\textitlf[1]{{\NHLight\itshape#1}}
\let\textbflf\textrm
\newcommand\textulf[1]{{\NHLight\bfseries#1}}
\newcommand\textuitlf[1]{{\NHLight\bfseries\itshape#1}}
\usepackage{fancyhdr}
\pagestyle{fancy}
\usepackage{titlesec}
\usepackage{titling}
\makeatletter
\lhead{\textbf{\@title}}
\makeatother
\rhead{\textrmlf{Compiled} \today}
\lfoot{\theauthor\ \textbullet \ \textbf{2021-2022}}
\cfoot{}
\rfoot{\textrmlf{Page} \thepage}
\renewcommand{\tableofcontents}{}
\titleformat{\section} {\Large} {\textrmlf{\thesection} {|}} {0.3em} {\textbf}
\titleformat{\subsection} {\large} {\textrmlf{\thesubsection} {|}} {0.2em} {\textbf}
\titleformat{\subsubsection} {\large} {\textrmlf{\thesubsubsection} {|}} {0.1em} {\textbf}
\setlength{\parskip}{0.45em}
\renewcommand\maketitle{}
\author{Exr0n}
\date{\today}
\title{Matricies as transformations}
\hypersetup{
 pdfauthor={Exr0n},
 pdftitle={Matricies as transformations},
 pdfkeywords={},
 pdfsubject={},
 pdfcreator={Emacs 28.0.50 (Org mode 9.4.4)}, 
 pdflang={English}}
\begin{document}

\tableofcontents

\#flo

\section{Flo}
\label{sec:orga8b9d4f}
\subsection{Thought Processes}
\label{sec:org884578a}
\begin{itemize}
\item As geometric transformations
\item As algebraic transformations
\end{itemize}

\subsection{Examples}
\label{sec:org625781f}
\(\begin{matrix}x &0 \\ 0 &y\end{matrix}\) | Scale by \(x\) and \(y\)
\(\begin{matrix}0 &1 \\ -1 & 0\end{matrix}\) | Rotate -90deg (easier to
visualize geometrically, also two reflections (over \(y=x\) and
\(y=0\))) \(\begin{matrix}1 &1 \\ 0 &1\end{matrix}\) | Add \(y\) to
\(x\) (easier to visualize algebraically, also a shear)

\subsection{Related}
\label{sec:orgf207e59}
\begin{itemize}
\item \href{KBe2020math530retPracticeMultiplyMatrixIdentfyGroups.org}{KBe2020math530retPracticeMultiplyMatrixIdentfyGroups}
\end{itemize}

\section{Rotation Matrices}
\label{sec:org3c6c6fc}
\begin{itemize}
\item We can get 90deg rotations decently easily, but what about other
angles?
\end{itemize}

\noindent\rule{\textwidth}{0.5pt}
\end{document}
