% Created 2021-09-27 Mon 12:03
% Intended LaTeX compiler: xelatex
\documentclass[letterpaper]{article}
\usepackage{graphicx}
\usepackage{grffile}
\usepackage{longtable}
\usepackage{wrapfig}
\usepackage{rotating}
\usepackage[normalem]{ulem}
\usepackage{amsmath}
\usepackage{textcomp}
\usepackage{amssymb}
\usepackage{capt-of}
\usepackage{hyperref}
\setlength{\parindent}{0pt}
\usepackage[margin=1in]{geometry}
\usepackage{fontspec}
\usepackage{svg}
\usepackage{cancel}
\usepackage{indentfirst}
\setmainfont[ItalicFont = LiberationSans-Italic, BoldFont = LiberationSans-Bold, BoldItalicFont = LiberationSans-BoldItalic]{LiberationSans}
\newfontfamily\NHLight[ItalicFont = LiberationSansNarrow-Italic, BoldFont       = LiberationSansNarrow-Bold, BoldItalicFont = LiberationSansNarrow-BoldItalic]{LiberationSansNarrow}
\newcommand\textrmlf[1]{{\NHLight#1}}
\newcommand\textitlf[1]{{\NHLight\itshape#1}}
\let\textbflf\textrm
\newcommand\textulf[1]{{\NHLight\bfseries#1}}
\newcommand\textuitlf[1]{{\NHLight\bfseries\itshape#1}}
\usepackage{fancyhdr}
\pagestyle{fancy}
\usepackage{titlesec}
\usepackage{titling}
\makeatletter
\lhead{\textbf{\@title}}
\makeatother
\rhead{\textrmlf{Compiled} \today}
\lfoot{\theauthor\ \textbullet \ \textbf{2021-2022}}
\cfoot{}
\rfoot{\textrmlf{Page} \thepage}
\renewcommand{\tableofcontents}{}
\titleformat{\section} {\Large} {\textrmlf{\thesection} {|}} {0.3em} {\textbf}
\titleformat{\subsection} {\large} {\textrmlf{\thesubsection} {|}} {0.2em} {\textbf}
\titleformat{\subsubsection} {\large} {\textrmlf{\thesubsubsection} {|}} {0.1em} {\textbf}
\setlength{\parskip}{0.45em}
\renewcommand\maketitle{}
\author{Exr0n}
\date{\today}
\title{flo 11}
\hypersetup{
 pdfauthor={Exr0n},
 pdftitle={flo 11},
 pdfkeywords={},
 pdfsubject={},
 pdfcreator={Emacs 28.0.50 (Org mode 9.4.4)}, 
 pdflang={English}}
\begin{document}

\tableofcontents

\#flo

\section{Polynomials}
\label{sec:orged450a9}
\begin{itemize}
\item See \href{KBrefPolynomial.org}{KBrefPolynomial}
\end{itemize}

\subsection{0 polynomial}
\label{sec:orgd0e4712}
\begin{itemize}
\item Has degree \(-infty\)
\item Degrees are usually positive, except for the \(0\) degree
\item "that's too hard, and we're not going to do it here"
\end{itemize}

\subsection{Identically zero}
\label{sec:org7ba3132}
\begin{itemize}
\item Like \(0\) or \(0 x^0\)
\item Most polynomials are sometimes zero, but polynomials that are
"identically zero" means that it's always zero (instead of just
sometimes zero)
\end{itemize}

\subsection{\(\mathcal{P}_m(F)\)}
\label{sec:org2dbb7cd}
\begin{itemize}
\item Polynomials with coefficients in \(F\) whose highest degree is \(m\)
\item It can't be "whose degree is exactly \(m\)" because otherwise you
won't have the identity and it won't be closed under addition (in the
case where coefficient sum \(a_m + b_m = 0\))
\end{itemize}

\subsubsection{It's a finite dimensional vector space}
\label{sec:org257ddec}
\begin{itemize}
\item \[a_0z^0+...+a_mz^m + b_0z^0 + ... + b_mz^m = (a_0+b_0)z^0 + ... + (a_m+b_m)z^m\]
\end{itemize}

\subsection{Proof of 2.16}
\label{sec:org1c7e07a}
\begin{itemize}
\item Structure: proof by contradiction
\end{itemize}

\section{Linear Independence}
\label{sec:org84119ff}
\begin{itemize}
\item "non-trivial" means "simplest possible", which has usually got the
most zeros
\item See
\href{KB20math530refLinearIndependence.org}{KB20math530refLinearIndependence}
\end{itemize}

\subsection{2.21 Linear Dependence Lemma 2.21}
\label{sec:org067fae7}
\#toexpand - it's saying that any linearly independent list has a vector
inside that doesn't "contribute anything", and that if you remove it
you'l have the same span. Implicitly, maybe through induction?) if you
remove a dependent vector enough times then you get a linearly
independent list. - The list \((1, 1, 1), (2, 2, 2), (3, 3, 3)\) is
really dependent, but \((0), (0), (0)\) is the most dependent (you have
to remove all to get independence).

\begin{html}
<p style="page-break-before: always">
\end{html}

\begin{html}
</p>
\end{html}

\begin{itemize}
\item how
\item to
\item pagebreak
\end{itemize}

\noindent\rule{\textwidth}{0.5pt}

\section{Exercise 2.A.1}
\label{sec:org7672df5}
\subsection{Lemma}
\label{sec:org2c6f116}
\begin{quote}
If vectors \(v_1, v_2, v_3, v_4\) span \(V\), then the list
\[v_1-v_2, v_2-v_3, v_3-v_4, v_4\] also spans \(V\).
\end{quote}

\subsection{Proof}
\label{sec:org01aee3e}
We prove the lemma by showing that any vector \(v \in V\) can be written
in the form \(a_1v_1 + a_2v_2 + a_3+v_3 + a_4v_4\) can also be written
as a linear combination of the form \[
b_1 (v_1-v_2) + b_2 (v_2-v_3) + b_3(v_3-v_4) + b_4v_4
\]

If we set $\backslash$[
\begin{aligned}
b_1 &= a_1\\
b_2 &= a_1 + a_2\\
b_3 &= a_1 + a_2 + a_3\\
b_4 &= a_1 + a_2 + a_3 + a_4\\
\end{aligned}
$\backslash$] then the two combinations will be equivalent:

\(\begin{aligned} &a_1(v_1-v_2) + (a_1+a_2)(v_2-v_3) + (a_1+a_2+a_3)(v_3-v_4) + (a_1+a_2+a_3+a_4)v_4\\ = &a_1v_1 \cancel{- a_1v_2 + a_1v_2} +a_2v_2 \cancel{- (a_1+a_2)v_3 + (a_1+a_2)v_3} +a_3v_3 - \cancel{(a_1+a_2+a_3)v_4 + (a_1+a_2+a_3)v_4} + a_4v_4\\ = a_1v_1 + a_2v_2 + a_3v_3 + a_4 v_4 \end{aligned}\)

\noindent\rule{\textwidth}{0.5pt}
\end{document}
