% Created 2021-09-27 Mon 12:03
% Intended LaTeX compiler: xelatex
\documentclass[letterpaper]{article}
\usepackage{graphicx}
\usepackage{grffile}
\usepackage{longtable}
\usepackage{wrapfig}
\usepackage{rotating}
\usepackage[normalem]{ulem}
\usepackage{amsmath}
\usepackage{textcomp}
\usepackage{amssymb}
\usepackage{capt-of}
\usepackage{hyperref}
\setlength{\parindent}{0pt}
\usepackage[margin=1in]{geometry}
\usepackage{fontspec}
\usepackage{svg}
\usepackage{cancel}
\usepackage{indentfirst}
\setmainfont[ItalicFont = LiberationSans-Italic, BoldFont = LiberationSans-Bold, BoldItalicFont = LiberationSans-BoldItalic]{LiberationSans}
\newfontfamily\NHLight[ItalicFont = LiberationSansNarrow-Italic, BoldFont       = LiberationSansNarrow-Bold, BoldItalicFont = LiberationSansNarrow-BoldItalic]{LiberationSansNarrow}
\newcommand\textrmlf[1]{{\NHLight#1}}
\newcommand\textitlf[1]{{\NHLight\itshape#1}}
\let\textbflf\textrm
\newcommand\textulf[1]{{\NHLight\bfseries#1}}
\newcommand\textuitlf[1]{{\NHLight\bfseries\itshape#1}}
\usepackage{fancyhdr}
\pagestyle{fancy}
\usepackage{titlesec}
\usepackage{titling}
\makeatletter
\lhead{\textbf{\@title}}
\makeatother
\rhead{\textrmlf{Compiled} \today}
\lfoot{\theauthor\ \textbullet \ \textbf{2021-2022}}
\cfoot{}
\rfoot{\textrmlf{Page} \thepage}
\renewcommand{\tableofcontents}{}
\titleformat{\section} {\Large} {\textrmlf{\thesection} {|}} {0.3em} {\textbf}
\titleformat{\subsection} {\large} {\textrmlf{\thesubsection} {|}} {0.2em} {\textbf}
\titleformat{\subsubsection} {\large} {\textrmlf{\thesubsubsection} {|}} {0.1em} {\textbf}
\setlength{\parskip}{0.45em}
\renewcommand\maketitle{}
\author{Exr0n}
\date{\today}
\title{linalg flow 5}
\hypersetup{
 pdfauthor={Exr0n},
 pdftitle={linalg flow 5},
 pdfkeywords={},
 pdfsubject={},
 pdfcreator={Emacs 28.0.50 (Org mode 9.4.4)}, 
 pdflang={English}}
\begin{document}

\tableofcontents



\section{Participation}
\label{sec:orga8b95b3}
\begin{itemize}
\item Unmute yourself
\end{itemize}

\section{Homework Review}
\label{sec:org6a47736}
\begin{itemize}
\item From homework
\href{20math530retReadingTheTextbook.org}{20math530retReadingTheTextbook}
\end{itemize}

\subsection{Is Dot Product Nice?}
\label{sec:org16d918b}
\begin{itemize}
\item Nice = group properties

\begin{itemize}
\item They aren't because its not closed
\item However, we still like dot product because it can easily tell us if
the thing is perpendicular
\end{itemize}
\end{itemize}

\subsection{Inverse of a matrix}
\label{sec:org4c3be12}
\begin{itemize}
\item Use 2 systems of equations (2 variables, 2 equations, twice)
\href{KBe20math530srcMatrixInverse.png.org}{KBe20math530srcMatrixInverse.png}
\item \(y = \frac{c}{bc-ad} = \frac{-c}{ad-bc}\)
\item Determinant determines whether its possible to have an inverse
(because if it's zero, then it's not possible!)

\begin{itemize}
\item A matrix with no inverse is \textbf{SINGULAR}
\item Determinant of \(A\) is zero
\item A has no inverse
\item \href{https://mathworld.wolfram.com/InvertibleMatrixTheorem.html}{invertable
matrix theorem}
\end{itemize}
\end{itemize}

\section{Proof Attempt Discussion Page?}
\label{sec:org4b14c7e}
\section{Small Groups}
\label{sec:org4611d8e}
\begin{enumerate}
\item Calculate cross products
\item Graph cross products
\item Cross Product geometry?

\begin{itemize}
\item It's the perpendicular!
\item \#bonushw its perpendicular
\end{itemize}

\item Determinant geometric interpretation?

\begin{itemize}
\item It's the perpendicular! IF you crossproduct-ify
\item \(\begin{bmatrix}x\\y\end{bmatrix}\Rightarrow\left|\begin{bmatrix}i&j\\x&y\end{bmatrix}\right| = iy-jx = \begin{bmatrix}y\\-x\end{bmatrix}\)
\#\# Taking the Determinant (why \sout{-}-?)
\end{itemize}

\item We take the sub-matrices on a torus

\begin{itemize}
\item But if you wrap everything around properly then you have a plus in
front of every coefficient
\item But if you don't wrap it, then the determinant ends up being the
negative, so that's why there's the whole plus minus thing.
\end{itemize}
\end{enumerate}

\noindent\rule{\textwidth}{0.5pt}
\end{document}
