% Created 2021-09-27 Mon 12:03
% Intended LaTeX compiler: xelatex
\documentclass[letterpaper]{article}
\usepackage{graphicx}
\usepackage{grffile}
\usepackage{longtable}
\usepackage{wrapfig}
\usepackage{rotating}
\usepackage[normalem]{ulem}
\usepackage{amsmath}
\usepackage{textcomp}
\usepackage{amssymb}
\usepackage{capt-of}
\usepackage{hyperref}
\setlength{\parindent}{0pt}
\usepackage[margin=1in]{geometry}
\usepackage{fontspec}
\usepackage{svg}
\usepackage{cancel}
\usepackage{indentfirst}
\setmainfont[ItalicFont = LiberationSans-Italic, BoldFont = LiberationSans-Bold, BoldItalicFont = LiberationSans-BoldItalic]{LiberationSans}
\newfontfamily\NHLight[ItalicFont = LiberationSansNarrow-Italic, BoldFont       = LiberationSansNarrow-Bold, BoldItalicFont = LiberationSansNarrow-BoldItalic]{LiberationSansNarrow}
\newcommand\textrmlf[1]{{\NHLight#1}}
\newcommand\textitlf[1]{{\NHLight\itshape#1}}
\let\textbflf\textrm
\newcommand\textulf[1]{{\NHLight\bfseries#1}}
\newcommand\textuitlf[1]{{\NHLight\bfseries\itshape#1}}
\usepackage{fancyhdr}
\pagestyle{fancy}
\usepackage{titlesec}
\usepackage{titling}
\makeatletter
\lhead{\textbf{\@title}}
\makeatother
\rhead{\textrmlf{Compiled} \today}
\lfoot{\theauthor\ \textbullet \ \textbf{2021-2022}}
\cfoot{}
\rfoot{\textrmlf{Page} \thepage}
\renewcommand{\tableofcontents}{}
\titleformat{\section} {\Large} {\textrmlf{\thesection} {|}} {0.3em} {\textbf}
\titleformat{\subsection} {\large} {\textrmlf{\thesubsection} {|}} {0.2em} {\textbf}
\titleformat{\subsubsection} {\large} {\textrmlf{\thesubsubsection} {|}} {0.1em} {\textbf}
\setlength{\parskip}{0.45em}
\renewcommand\maketitle{}
\author{Exr0n}
\date{\today}
\title{Mini Quiz: Subspace Addition forms Groups?}
\hypersetup{
 pdfauthor={Exr0n},
 pdftitle={Mini Quiz: Subspace Addition forms Groups?},
 pdfkeywords={},
 pdfsubject={},
 pdfcreator={Emacs 28.0.50 (Org mode 9.4.4)}, 
 pdflang={English}}
\begin{document}

\tableofcontents



\section{Problem}
\label{sec:orgccbc95b}
\begin{quote}
Do subspaces form a group under subspace addition?
\end{quote}

\begin{itemize}
\item Properties for a group:
\href{KBe2020math530refGroups.org}{KBe2020math530refGroups}
\item Closed, Identity, Inverses, Associative
\end{itemize}

\section{Working it out}
\label{sec:org73093f6}
I don't actually remember the exact definition of subspace addition. If
I remember correctly from the proof of Axler exercise 1.C.14, the sum of
two subspaces is the subspace where each vector is the sum of two
vectors in the original two subspaces?

\subsection{Closed}
\label{sec:orge357368}
I don't remember if it was guaranteed to be a subspace, but it must have
the identity (because the constituents both had the identity), it is
closed under scalar multiplication (because you could take the sum
apart, multiply the bits from each smaller subspace which are closed
under scalar multiplication, and then put them back together again). I
think it is closed under addition because both parts are closed under
addition. This is by no means a rigorous proof, but it is as close as I
can get without knowing the actual definition of a subspace sum, and I
think its reasonably convincing.

\subsection{Identity}
\label{sec:orgcbe034e}
If the above is true, then subspaces are closed under subspace addition.
The identity subspace would be the one with only the field additive
identity, because there is only one element so the resulting subspace
sum has the same number of elements as the other original subspace, and
because the identity vector plus any vector of the other subspace will
be that other vector by definition.

\subsection{Inverse}
\label{sec:orgd7eda03}
Because the subspace sum is all of the possible outputs when adding each
vector in the two subspaces, if a subspace has two or more unique
elements then it's not possible to have an inverse subspace: it would
not be possible to create a subspace and force the parings such that the
resulting subpsace sum is the degenerate one (\({0}\)).

\section{Checking}
\label{sec:orga766fe4}
Now I will look back at the textbook to see what the formal definition
of subspace sum is and if my previous conclusions were valid. The quiz
assignment says to avoid help from classes or websites, so I am not sure
if I am allowed to look at the textbook. If not, please grade me on the
previous sections only.

It seems like my notion of subspace sums is roughly correct, so I am
pretty sure that \[
\fbox{ No, subspaces do not form a group under subspace addition due to a lack of inverses. }
\]

\noindent\rule{\textwidth}{0.5pt}
\end{document}
