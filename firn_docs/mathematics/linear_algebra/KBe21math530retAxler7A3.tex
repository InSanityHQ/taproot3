% Created 2021-09-27 Mon 11:53
% Intended LaTeX compiler: xelatex
\documentclass[letterpaper]{article}
\usepackage{graphicx}
\usepackage{grffile}
\usepackage{longtable}
\usepackage{wrapfig}
\usepackage{rotating}
\usepackage[normalem]{ulem}
\usepackage{amsmath}
\usepackage{textcomp}
\usepackage{amssymb}
\usepackage{capt-of}
\usepackage{hyperref}
\setlength{\parindent}{0pt}
\usepackage[margin=1in]{geometry}
\usepackage{fontspec}
\usepackage{svg}
\usepackage{cancel}
\usepackage{indentfirst}
\setmainfont[ItalicFont = LiberationSans-Italic, BoldFont = LiberationSans-Bold, BoldItalicFont = LiberationSans-BoldItalic]{LiberationSans}
\newfontfamily\NHLight[ItalicFont = LiberationSansNarrow-Italic, BoldFont       = LiberationSansNarrow-Bold, BoldItalicFont = LiberationSansNarrow-BoldItalic]{LiberationSansNarrow}
\newcommand\textrmlf[1]{{\NHLight#1}}
\newcommand\textitlf[1]{{\NHLight\itshape#1}}
\let\textbflf\textrm
\newcommand\textulf[1]{{\NHLight\bfseries#1}}
\newcommand\textuitlf[1]{{\NHLight\bfseries\itshape#1}}
\usepackage{fancyhdr}
\pagestyle{fancy}
\usepackage{titlesec}
\usepackage{titling}
\makeatletter
\lhead{\textbf{\@title}}
\makeatother
\rhead{\textrmlf{Compiled} \today}
\lfoot{\theauthor\ \textbullet \ \textbf{2021-2022}}
\cfoot{}
\rfoot{\textrmlf{Page} \thepage}
\renewcommand{\tableofcontents}{}
\titleformat{\section} {\Large} {\textrmlf{\thesection} {|}} {0.3em} {\textbf}
\titleformat{\subsection} {\large} {\textrmlf{\thesubsection} {|}} {0.2em} {\textbf}
\titleformat{\subsubsection} {\large} {\textrmlf{\thesubsubsection} {|}} {0.1em} {\textbf}
\setlength{\parskip}{0.45em}
\renewcommand\maketitle{}
\author{Taproot}
\date{\today}
\title{Axler 7.A exercise 3}
\hypersetup{
 pdfauthor={Taproot},
 pdftitle={Axler 7.A exercise 3},
 pdfkeywords={},
 pdfsubject={},
 pdfcreator={Emacs 28.0.50 (Org mode 9.4.4)}, 
 pdflang={English}}
\begin{document}

\tableofcontents

\begin{quote}
Suppose \(T \in  \mathcal{L}(V)\) and \(U\) is a subspace of \(V\). Prove that \(U\) is invariant under \(T\) iff \(U^\bot\) is invariant under \(T^*\).
\end{quote}

For all pairs \(u \in  U\) and \(w \in  U^\perp\),

\[\begin{aligned}
 \langle Tu, w \rangle = 0\\
 \langle u, T^*w \rangle = 0
\end{aligned}\]

This implies that the range of \(T^*\BigR|_{U^\perp} \subseteq U^\perp\), aka that \(T^*\) is invariant under \(U^\perp\)

This implies both directions, since \(U = U^{\perp ^\perp }\) and \(T = (T^*)^*\).

\section{:noexport:}
\label{sec:org6d3da10}

For all \(u \in  U\), \(Tu = u' \in  U\).
Let \(w = U^\perp\). Then, \$T\textsuperscript{*}w = \$
\[\begin{aligned}
 \langle u, T^*w \rangle = \langle Tu, w \rangle = \langle u', w \rangle
\end{aligned}\]
\end{document}
