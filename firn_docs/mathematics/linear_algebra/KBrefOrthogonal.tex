% Created 2021-09-27 Mon 11:52
% Intended LaTeX compiler: xelatex
\documentclass[letterpaper]{article}
\usepackage{graphicx}
\usepackage{grffile}
\usepackage{longtable}
\usepackage{wrapfig}
\usepackage{rotating}
\usepackage[normalem]{ulem}
\usepackage{amsmath}
\usepackage{textcomp}
\usepackage{amssymb}
\usepackage{capt-of}
\usepackage{hyperref}
\setlength{\parindent}{0pt}
\usepackage[margin=1in]{geometry}
\usepackage{fontspec}
\usepackage{svg}
\usepackage{cancel}
\usepackage{indentfirst}
\setmainfont[ItalicFont = LiberationSans-Italic, BoldFont = LiberationSans-Bold, BoldItalicFont = LiberationSans-BoldItalic]{LiberationSans}
\newfontfamily\NHLight[ItalicFont = LiberationSansNarrow-Italic, BoldFont       = LiberationSansNarrow-Bold, BoldItalicFont = LiberationSansNarrow-BoldItalic]{LiberationSansNarrow}
\newcommand\textrmlf[1]{{\NHLight#1}}
\newcommand\textitlf[1]{{\NHLight\itshape#1}}
\let\textbflf\textrm
\newcommand\textulf[1]{{\NHLight\bfseries#1}}
\newcommand\textuitlf[1]{{\NHLight\bfseries\itshape#1}}
\usepackage{fancyhdr}
\pagestyle{fancy}
\usepackage{titlesec}
\usepackage{titling}
\makeatletter
\lhead{\textbf{\@title}}
\makeatother
\rhead{\textrmlf{Compiled} \today}
\lfoot{\theauthor\ \textbullet \ \textbf{2021-2022}}
\cfoot{}
\rfoot{\textrmlf{Page} \thepage}
\renewcommand{\tableofcontents}{}
\titleformat{\section} {\Large} {\textrmlf{\thesection} {|}} {0.3em} {\textbf}
\titleformat{\subsection} {\large} {\textrmlf{\thesubsection} {|}} {0.2em} {\textbf}
\titleformat{\subsubsection} {\large} {\textrmlf{\thesubsubsection} {|}} {0.1em} {\textbf}
\setlength{\parskip}{0.45em}
\renewcommand\maketitle{}
\author{Taproot}
\date{\today}
\title{Orthogonal}
\hypersetup{
 pdfauthor={Taproot},
 pdftitle={Orthogonal},
 pdfkeywords={},
 pdfsubject={},
 pdfcreator={Emacs 28.0.50 (Org mode 9.4.4)}, 
 pdflang={English}}
\begin{document}

\tableofcontents

\section{orthogonal\hfill{}\textsc{def}}
\label{sec:org8778c94}
\begin{quote}
Two vectors \(u, v \in V\) are called \emph{orthogonal} if \(\langle u, v \rangle = 0\)
\end{quote}
\section{results}
\label{sec:org3f5f927}
\subsection{orthogonal \textasciitilde{}= perpendicular}
\label{sec:org21df811}
\subsection{Axler 6.12 orthogonality and zero}
\label{sec:org44a5e3a}
\subsubsection{0 is orthogonal to every vector in \(V\)}
\label{sec:orgd5138a9}
\subsubsection{0 is the only vector in \(V\) that is orthogonal to itself}
\label{sec:orgdbb9b17}
\subsection{Axler 6.13 Pythagorean Theorem}
\label{sec:orgf0d1192}
\begin{quote}
Suppose \(u\) and \(v\) are orthogonal vectors in \(V\). Then
\[\begin{aligned}
   \lVert u+v \rVert^2 = \lVert u \rVert^2 + \lVert v \rVert^2
   \end{aligned}\]
\end{quote}
\subsubsection{proof with more algebra written out}
\label{sec:orgbe339d4}

\[\begin{aligned}
	\lVert u+v \rVert^2 &= \langle u+v, u+v \rangle\\
	&= \langle u, u+v \rangle + \langle v, u+v \rangle\\
	&= \langle u,u \rangle + \cancelto{0}{\langle u, v \rangle} + \cancelto{0}{\langle v, u \rangle} + \langle v,v \rangle\\
	&= \lVert u \rVert^2 + \lVert v \rVert^2
	\end{aligned}\]
\end{document}
