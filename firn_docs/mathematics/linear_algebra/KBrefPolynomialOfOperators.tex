% Created 2021-09-27 Mon 11:53
% Intended LaTeX compiler: xelatex
\documentclass[letterpaper]{article}
\usepackage{graphicx}
\usepackage{grffile}
\usepackage{longtable}
\usepackage{wrapfig}
\usepackage{rotating}
\usepackage[normalem]{ulem}
\usepackage{amsmath}
\usepackage{textcomp}
\usepackage{amssymb}
\usepackage{capt-of}
\usepackage{hyperref}
\setlength{\parindent}{0pt}
\usepackage[margin=1in]{geometry}
\usepackage{fontspec}
\usepackage{svg}
\usepackage{cancel}
\usepackage{indentfirst}
\setmainfont[ItalicFont = LiberationSans-Italic, BoldFont = LiberationSans-Bold, BoldItalicFont = LiberationSans-BoldItalic]{LiberationSans}
\newfontfamily\NHLight[ItalicFont = LiberationSansNarrow-Italic, BoldFont       = LiberationSansNarrow-Bold, BoldItalicFont = LiberationSansNarrow-BoldItalic]{LiberationSansNarrow}
\newcommand\textrmlf[1]{{\NHLight#1}}
\newcommand\textitlf[1]{{\NHLight\itshape#1}}
\let\textbflf\textrm
\newcommand\textulf[1]{{\NHLight\bfseries#1}}
\newcommand\textuitlf[1]{{\NHLight\bfseries\itshape#1}}
\usepackage{fancyhdr}
\pagestyle{fancy}
\usepackage{titlesec}
\usepackage{titling}
\makeatletter
\lhead{\textbf{\@title}}
\makeatother
\rhead{\textrmlf{Compiled} \today}
\lfoot{\theauthor\ \textbullet \ \textbf{2021-2022}}
\cfoot{}
\rfoot{\textrmlf{Page} \thepage}
\renewcommand{\tableofcontents}{}
\titleformat{\section} {\Large} {\textrmlf{\thesection} {|}} {0.3em} {\textbf}
\titleformat{\subsection} {\large} {\textrmlf{\thesubsection} {|}} {0.2em} {\textbf}
\titleformat{\subsubsection} {\large} {\textrmlf{\thesubsubsection} {|}} {0.1em} {\textbf}
\setlength{\parskip}{0.45em}
\renewcommand\maketitle{}
\author{Taproot}
\date{\today}
\title{Polynomials of Operators}
\hypersetup{
 pdfauthor={Taproot},
 pdftitle={Polynomials of Operators},
 pdfkeywords={},
 pdfsubject={},
 pdfcreator={Emacs 28.0.50 (Org mode 9.4.4)}, 
 pdflang={English}}
\begin{document}

\tableofcontents

\section{\(p(T)\)\hfill{}\textsc{def}}
\label{sec:org0a4cc94}
\begin{quote}
Suppose \(T \in  \mathcal{L} (V)\) and \(p \in \mathcal{P} (\mathbb{F} )\) is a polynomial given by

\[\begin{aligned}
  P(z) = a_0 + a_1z + a_2z^2 + \cdots + a_m z^m
  \end{aligned}\]
for \$z \(\in\)  \mathbb{F} \$. Then \(p(T)\) is the operator defined by
\[\begin{aligned}
  p(T) = a_0I + a_1T + a_2T^2 + \cdots + a_m T^m
  \end{aligned}\]
\end{quote}
\section{using\hfill{}\textsc{deps}}
\label{sec:org7f58ffe}
\subsection{\href{KBrefOperatorExponents.org}{Operator Exponents}}
\label{sec:org71d251c}
\section{intuition}
\label{sec:org2b070f1}
\subsection{Exactly how you would expect it to work}
\label{sec:orgfb1eba5}
\section{results}
\label{sec:org3705493}
\subsection{For some operator \(T \in  \mathcal{L} (V)\) the function from \(\mathcal{P} (\mathbb{F} )\) to \(\mathcal{L} (V)\)}
\label{sec:org248446d}
\[\begin{aligned}
   p \mapsto p(T)
   \end{aligned}\]
is linear
\end{document}
