% Created 2021-09-27 Mon 11:52
% Intended LaTeX compiler: xelatex
\documentclass[letterpaper]{article}
\usepackage{graphicx}
\usepackage{grffile}
\usepackage{longtable}
\usepackage{wrapfig}
\usepackage{rotating}
\usepackage[normalem]{ulem}
\usepackage{amsmath}
\usepackage{textcomp}
\usepackage{amssymb}
\usepackage{capt-of}
\usepackage{hyperref}
\setlength{\parindent}{0pt}
\usepackage[margin=1in]{geometry}
\usepackage{fontspec}
\usepackage{svg}
\usepackage{cancel}
\usepackage{indentfirst}
\setmainfont[ItalicFont = LiberationSans-Italic, BoldFont = LiberationSans-Bold, BoldItalicFont = LiberationSans-BoldItalic]{LiberationSans}
\newfontfamily\NHLight[ItalicFont = LiberationSansNarrow-Italic, BoldFont       = LiberationSansNarrow-Bold, BoldItalicFont = LiberationSansNarrow-BoldItalic]{LiberationSansNarrow}
\newcommand\textrmlf[1]{{\NHLight#1}}
\newcommand\textitlf[1]{{\NHLight\itshape#1}}
\let\textbflf\textrm
\newcommand\textulf[1]{{\NHLight\bfseries#1}}
\newcommand\textuitlf[1]{{\NHLight\bfseries\itshape#1}}
\usepackage{fancyhdr}
\pagestyle{fancy}
\usepackage{titlesec}
\usepackage{titling}
\makeatletter
\lhead{\textbf{\@title}}
\makeatother
\rhead{\textrmlf{Compiled} \today}
\lfoot{\theauthor\ \textbullet \ \textbf{2021-2022}}
\cfoot{}
\rfoot{\textrmlf{Page} \thepage}
\renewcommand{\tableofcontents}{}
\titleformat{\section} {\Large} {\textrmlf{\thesection} {|}} {0.3em} {\textbf}
\titleformat{\subsection} {\large} {\textrmlf{\thesubsection} {|}} {0.2em} {\textbf}
\titleformat{\subsubsection} {\large} {\textrmlf{\thesubsubsection} {|}} {0.1em} {\textbf}
\setlength{\parskip}{0.45em}
\renewcommand\maketitle{}
\author{Exr0n}
\date{\today}
\title{Surjectivity of Functions}
\hypersetup{
 pdfauthor={Exr0n},
 pdftitle={Surjectivity of Functions},
 pdfkeywords={},
 pdfsubject={},
 pdfcreator={Emacs 28.0.50 (Org mode 9.4.4)}, 
 pdflang={English}}
\begin{document}

\tableofcontents

\section{In the context of Linear Algebra (Axler 3.20) \#defintion surjective\hfill{}\textsc{def}}
\label{sec:orgfd76e86}
\begin{quote}
A function \(T : V \to W\) is called \emph{surjective} if its range equals \(W\).
\end{quote}
\subsection{\#aka onto\hfill{}\textsc{aka}}
\label{sec:orge8ad297}
\subsection{Properties}
\label{sec:orgd36e66b}
\subsubsection{A non-surjective map can be made surjective by changing the output space. (intuitive, not in book)}
\label{sec:orgbf893b2}
\subsubsection{A map to a larger dimensional space is not surjective (Axler3.24)}
\label{sec:orgf1bc621}
\begin{quote}
Suppose \(V\) and \(W\) are finite-dimensional spaces such that \(\text{dim }V < \text{dim }U\). Then no linear map from \(V\) to \(W\) is surjective.
\end{quote}
\begin{enumerate}
\item Intuition
\label{sec:orgcffe983}
Surjectivity means that every element in the output space is mapped to, and that makes this intuitively true: If the number of possible inputs (and by extension, possible outputs) is smaller dimension than the output space, how can every output be mapped to?
\end{enumerate}
\end{document}
