% Created 2021-09-27 Mon 12:03
% Intended LaTeX compiler: xelatex
\documentclass[letterpaper]{article}
\usepackage{graphicx}
\usepackage{grffile}
\usepackage{longtable}
\usepackage{wrapfig}
\usepackage{rotating}
\usepackage[normalem]{ulem}
\usepackage{amsmath}
\usepackage{textcomp}
\usepackage{amssymb}
\usepackage{capt-of}
\usepackage{hyperref}
\setlength{\parindent}{0pt}
\usepackage[margin=1in]{geometry}
\usepackage{fontspec}
\usepackage{svg}
\usepackage{cancel}
\usepackage{indentfirst}
\setmainfont[ItalicFont = LiberationSans-Italic, BoldFont = LiberationSans-Bold, BoldItalicFont = LiberationSans-BoldItalic]{LiberationSans}
\newfontfamily\NHLight[ItalicFont = LiberationSansNarrow-Italic, BoldFont       = LiberationSansNarrow-Bold, BoldItalicFont = LiberationSansNarrow-BoldItalic]{LiberationSansNarrow}
\newcommand\textrmlf[1]{{\NHLight#1}}
\newcommand\textitlf[1]{{\NHLight\itshape#1}}
\let\textbflf\textrm
\newcommand\textulf[1]{{\NHLight\bfseries#1}}
\newcommand\textuitlf[1]{{\NHLight\bfseries\itshape#1}}
\usepackage{fancyhdr}
\pagestyle{fancy}
\usepackage{titlesec}
\usepackage{titling}
\makeatletter
\lhead{\textbf{\@title}}
\makeatother
\rhead{\textrmlf{Compiled} \today}
\lfoot{\theauthor\ \textbullet \ \textbf{2021-2022}}
\cfoot{}
\rfoot{\textrmlf{Page} \thepage}
\renewcommand{\tableofcontents}{}
\titleformat{\section} {\Large} {\textrmlf{\thesection} {|}} {0.3em} {\textbf}
\titleformat{\subsection} {\large} {\textrmlf{\thesubsection} {|}} {0.2em} {\textbf}
\titleformat{\subsubsection} {\large} {\textrmlf{\thesubsubsection} {|}} {0.1em} {\textbf}
\setlength{\parskip}{0.45em}
\renewcommand\maketitle{}
\date{\today}
\title{Lake Problem}
\hypersetup{
 pdfauthor={},
 pdftitle={Lake Problem},
 pdfkeywords={},
 pdfsubject={},
 pdfcreator={Emacs 28.0.50 (Org mode 9.4.4)}, 
 pdflang={English}}
\begin{document}

\tableofcontents

\#flo \#ref \#ret \#disorganized \#incomplete

\noindent\rule{\textwidth}{0.5pt}

\section{the lake problem}
\label{sec:org00f2004}
\textbf{150 word description of the mathematical technique you used to
approximate the lake bed.}

After looking at quite a few bathymetries of real lakes, we realized
that when the geometry of the lake's edge changed substantially, there
would often be a raised ridge in the intersection. Each of these
distinct regions would have a deep spot, then be surrounded either by
the lake edge or a ridge. Our goal with our point placement was to
locate these deep spots, then use our remaining points to locate the
ridges and to see the change in slope at the same time. After receiving
our points, we inputted them into the 3D modeling software Blender, and
used a marching cubes algorithm to create a mesh which we then converted
to a .stl and 3D printed. We also created a programmatic solution which
involved looping through every point and taking the weighted average of
the nearby known point's heights. We weighted exponentially based upon
the distance of the source points to our target points, hoping to
produce some nice curves.
\end{document}
