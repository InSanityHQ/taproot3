% Created 2021-09-27 Mon 12:03
% Intended LaTeX compiler: xelatex
\documentclass[letterpaper]{article}
\usepackage{graphicx}
\usepackage{grffile}
\usepackage{longtable}
\usepackage{wrapfig}
\usepackage{rotating}
\usepackage[normalem]{ulem}
\usepackage{amsmath}
\usepackage{textcomp}
\usepackage{amssymb}
\usepackage{capt-of}
\usepackage{hyperref}
\setlength{\parindent}{0pt}
\usepackage[margin=1in]{geometry}
\usepackage{fontspec}
\usepackage{svg}
\usepackage{cancel}
\usepackage{indentfirst}
\setmainfont[ItalicFont = LiberationSans-Italic, BoldFont = LiberationSans-Bold, BoldItalicFont = LiberationSans-BoldItalic]{LiberationSans}
\newfontfamily\NHLight[ItalicFont = LiberationSansNarrow-Italic, BoldFont       = LiberationSansNarrow-Bold, BoldItalicFont = LiberationSansNarrow-BoldItalic]{LiberationSansNarrow}
\newcommand\textrmlf[1]{{\NHLight#1}}
\newcommand\textitlf[1]{{\NHLight\itshape#1}}
\let\textbflf\textrm
\newcommand\textulf[1]{{\NHLight\bfseries#1}}
\newcommand\textuitlf[1]{{\NHLight\bfseries\itshape#1}}
\usepackage{fancyhdr}
\pagestyle{fancy}
\usepackage{titlesec}
\usepackage{titling}
\makeatletter
\lhead{\textbf{\@title}}
\makeatother
\rhead{\textrmlf{Compiled} \today}
\lfoot{\theauthor\ \textbullet \ \textbf{2021-2022}}
\cfoot{}
\rfoot{\textrmlf{Page} \thepage}
\renewcommand{\tableofcontents}{}
\titleformat{\section} {\Large} {\textrmlf{\thesection} {|}} {0.3em} {\textbf}
\titleformat{\subsection} {\large} {\textrmlf{\thesubsection} {|}} {0.2em} {\textbf}
\titleformat{\subsubsection} {\large} {\textrmlf{\thesubsubsection} {|}} {0.1em} {\textbf}
\setlength{\parskip}{0.45em}
\renewcommand\maketitle{}
\date{\today}
\title{MVC PS\#4}
\hypersetup{
 pdfauthor={},
 pdftitle={MVC PS\#4},
 pdfkeywords={},
 pdfsubject={},
 pdfcreator={Emacs 28.0.50 (Org mode 9.4.4)}, 
 pdflang={English}}
\begin{document}

\tableofcontents



\section{Class Problems}
\label{sec:org5eca4c6}
\subsection{\((1)\) and \((2)\)}
\label{sec:org9fba1e6}
\((2)\): If the xy-angle, z-angle, and magnitude of a point in 3D space
are represented by the variables \(\theta, \phi, l\), then the vector
representation of the point will be equal to
\(<\sin{\phi}\cdot\cos{\theta}, \sin{\phi}\cos{\theta}, \cos{\phi}>\cdot l\).
Therefore, the answer to \((1)\) is
\(<\frac{\pi}{4}, \frac{\pi}{4}, \frac{\sqrt{3}}{2}>\).

\section{Vectors}
\label{sec:org91d04af}
\subsection{\((2)\)}
\label{sec:orge0f30e5}
Magnitude: \(\sqrt{10}\)

Direction: \(<\frac{3}{\sqrt{10}}, -\frac{1}{\sqrt{10}}>\)

\subsection{\((5)\)}
\label{sec:orgc518d43}
Magnitude: \(\sqrt{21}\)

Direction:
\(<\frac{1}{\sqrt{21}}, -\frac{2}{\sqrt{21}}, \frac{4}{\sqrt{21}}>\)

\subsection{\((9)\)}
\label{sec:org9055d2b}
Magnitude: \(\frac{\sqrt{5}}{2}\)

Direction: \(<-\sqrt{\frac{3}{5}}, \sqrt{\frac{2}{5}}>\)

\subsection{\((30)\)}
\label{sec:orgffd685b}
See Drawings Section

\subsection{Drawings}
\label{sec:org55c0f2b}
\href{KBdPS4img.jpg.org}{KBdPS4img.jpg}
\end{document}
