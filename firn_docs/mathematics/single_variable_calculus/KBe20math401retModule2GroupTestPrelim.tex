% Created 2021-09-27 Mon 11:53
% Intended LaTeX compiler: xelatex
\documentclass[letterpaper]{article}
\usepackage{graphicx}
\usepackage{grffile}
\usepackage{longtable}
\usepackage{wrapfig}
\usepackage{rotating}
\usepackage[normalem]{ulem}
\usepackage{amsmath}
\usepackage{textcomp}
\usepackage{amssymb}
\usepackage{capt-of}
\usepackage{hyperref}
\setlength{\parindent}{0pt}
\usepackage[margin=1in]{geometry}
\usepackage{fontspec}
\usepackage{svg}
\usepackage{cancel}
\usepackage{indentfirst}
\setmainfont[ItalicFont = LiberationSans-Italic, BoldFont = LiberationSans-Bold, BoldItalicFont = LiberationSans-BoldItalic]{LiberationSans}
\newfontfamily\NHLight[ItalicFont = LiberationSansNarrow-Italic, BoldFont       = LiberationSansNarrow-Bold, BoldItalicFont = LiberationSansNarrow-BoldItalic]{LiberationSansNarrow}
\newcommand\textrmlf[1]{{\NHLight#1}}
\newcommand\textitlf[1]{{\NHLight\itshape#1}}
\let\textbflf\textrm
\newcommand\textulf[1]{{\NHLight\bfseries#1}}
\newcommand\textuitlf[1]{{\NHLight\bfseries\itshape#1}}
\usepackage{fancyhdr}
\pagestyle{fancy}
\usepackage{titlesec}
\usepackage{titling}
\makeatletter
\lhead{\textbf{\@title}}
\makeatother
\rhead{\textrmlf{Compiled} \today}
\lfoot{\theauthor\ \textbullet \ \textbf{2021-2022}}
\cfoot{}
\rfoot{\textrmlf{Page} \thepage}
\renewcommand{\tableofcontents}{}
\titleformat{\section} {\Large} {\textrmlf{\thesection} {|}} {0.3em} {\textbf}
\titleformat{\subsection} {\large} {\textrmlf{\thesubsection} {|}} {0.2em} {\textbf}
\titleformat{\subsubsection} {\large} {\textrmlf{\thesubsubsection} {|}} {0.1em} {\textbf}
\setlength{\parskip}{0.45em}
\renewcommand\maketitle{}
\author{Exr0n}
\date{\today}
\title{Module 2 Group Test Prelim}
\hypersetup{
 pdfauthor={Exr0n},
 pdftitle={Module 2 Group Test Prelim},
 pdfkeywords={},
 pdfsubject={},
 pdfcreator={Emacs 28.0.50 (Org mode 9.4.4)}, 
 pdflang={English}}
\begin{document}

\tableofcontents


\section{Openstax Calc vol 1 chap 2.4 ex 134}
\label{sec:org120347d}
$$g(t) = \frac{1}{t}+1$$ which is basically \(\frac{1}{x}\) shied up by one, so there is an \(\boxed{\text{infinite discontinuity at} x=0}\)
\section{136}
\label{sec:orgfb14dab}
There is a jump discontinuity at \(x=2\), because normally \(y=\frac{x}{x}\) simplifies to \(y=1\), but the sign flips at \(x=2\).
\section{142}
\label{sec:org077865a}
$$f(y) = \frac{\sin(\pi y)}{\tan(\pi y)} = \frac{\cancel{\sin(\pi y)} \cos(\pi y)}{\cancel{\sin(\pi y)}}$$
So there is a removable discontinuity at \(y=1\), because there is a discontinuity but it can be removed with algebra.
\section{148}
\label{sec:org82a7817}
$$\begin{aligned}e^{4k} = 4+3\\e^{4k}=7\\4k=\ln(7)\\k=\frac{\ln(7)}{4}\end{aligned}$$
\section{{\bfseries\sffamily TODO} 174}
\label{sec:org6d86ef0}
Prove \(f(x)\) is continues everywhere, meaning show that \(\forall c\in \mathbb{R}\)
$$\lim_{x\to c} f(x) = f(c)$$
Because we can always evaluate \(f(x)\), the limit always exists.
\section{Paul's online math notes Section 2-9: 23}
\label{sec:org06f19d8}
The IVT states that when a function is continuous over a closed interval \([a, b]\), then for all \(\min\{f(a), f(b)\} \le y \le \max\{f(a), f(b)\}\) there exists some \(a \le c \le b\) s.t. \(f(c) = y\).
In this case, we have \(f(4) = 193\) and \(f(8) = -511\). \(f(x)\) is a polynomial, so it is continuous over the range. Because our values stradle zero, there must be some value \(4 \le c \le 8\) s.t. \(f(c) = 0\).
\section{Boundedness theorem}
\label{sec:org6688a69}
Given a function \(f(x)\) that is continuous on a closed interval \([a, b]\), there exists some \(M \in \mathbb R\) s.t. \(f(c) \le M\) for all \(a \le c \le b\) aka \(M\) is an upper bound on \(f(x)\) over the interval \([a, b]\). There's also one that's less than all \(c\). Doesn't work for open intervals.
\subsection{\((0, 1]\): not continuous, not a closed interval}
\label{sec:orgdd37987}
\subsection{\([0, 1)\): not a closed interval}
\label{sec:org8d396fa}
\subsection{\((0, 1]\): not a closed interval}
\label{sec:org14c3f57}
\subsection{\((0, 1]\): not continuous, not a closed interval}
\label{sec:org347a8df}
\subsection{\(f(x) = \frac{1}{x}\): not continuous}
\label{sec:orgfe1c2d8}
\section{Epilouge}
\label{sec:org1ea6a49}
Other than Problem 5, this took roughly 40 minutes. I still don't know how to do problem 5..
\end{document}
