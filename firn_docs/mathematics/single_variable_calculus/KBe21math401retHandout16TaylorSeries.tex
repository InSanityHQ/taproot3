% Created 2021-09-27 Mon 11:53
% Intended LaTeX compiler: xelatex
\documentclass[letterpaper]{article}
\usepackage{graphicx}
\usepackage{grffile}
\usepackage{longtable}
\usepackage{wrapfig}
\usepackage{rotating}
\usepackage[normalem]{ulem}
\usepackage{amsmath}
\usepackage{textcomp}
\usepackage{amssymb}
\usepackage{capt-of}
\usepackage{hyperref}
\setlength{\parindent}{0pt}
\usepackage[margin=1in]{geometry}
\usepackage{fontspec}
\usepackage{svg}
\usepackage{cancel}
\usepackage{indentfirst}
\setmainfont[ItalicFont = LiberationSans-Italic, BoldFont = LiberationSans-Bold, BoldItalicFont = LiberationSans-BoldItalic]{LiberationSans}
\newfontfamily\NHLight[ItalicFont = LiberationSansNarrow-Italic, BoldFont       = LiberationSansNarrow-Bold, BoldItalicFont = LiberationSansNarrow-BoldItalic]{LiberationSansNarrow}
\newcommand\textrmlf[1]{{\NHLight#1}}
\newcommand\textitlf[1]{{\NHLight\itshape#1}}
\let\textbflf\textrm
\newcommand\textulf[1]{{\NHLight\bfseries#1}}
\newcommand\textuitlf[1]{{\NHLight\bfseries\itshape#1}}
\usepackage{fancyhdr}
\pagestyle{fancy}
\usepackage{titlesec}
\usepackage{titling}
\makeatletter
\lhead{\textbf{\@title}}
\makeatother
\rhead{\textrmlf{Compiled} \today}
\lfoot{\theauthor\ \textbullet \ \textbf{2021-2022}}
\cfoot{}
\rfoot{\textrmlf{Page} \thepage}
\renewcommand{\tableofcontents}{}
\titleformat{\section} {\Large} {\textrmlf{\thesection} {|}} {0.3em} {\textbf}
\titleformat{\subsection} {\large} {\textrmlf{\thesubsection} {|}} {0.2em} {\textbf}
\titleformat{\subsubsection} {\large} {\textrmlf{\thesubsubsection} {|}} {0.1em} {\textbf}
\setlength{\parskip}{0.45em}
\renewcommand\maketitle{}
\author{Exr0n}
\date{\today}
\title{Handout 16 Problems}
\hypersetup{
 pdfauthor={Exr0n},
 pdftitle={Handout 16 Problems},
 pdfkeywords={},
 pdfsubject={},
 pdfcreator={Emacs 28.0.50 (Org mode 9.4.4)}, 
 pdflang={English}}
\begin{document}

\tableofcontents


\section{Complete the Representation}
\label{sec:org8a864af}
\begin{center}
\begin{tabular}{lll}
Function & First four terms & Generalized\\
\hline
\(\frac{1}{1-2x}\) & \(1+2x+4x^2+8x^3+\cdots\) & \(\sum_{k=0} 2^k x^k\)\\
\(\cos(3x)\) & \(1-\frac{9x^2}{2!}+\frac{81x^4}{4!}-\frac{729x^6}{6!}+\cdots\) & \(\sum_{k=0} \frac{(-1)^k (3x)^{2k}}{2k!}\)\\
\(\frac{e^x}{e^2}\) & \(\frac{1}{e^2} + \frac{x}{e^2} + \frac{x^2}{e^2 2!} + \cdots\) & \(\sum_{k=0} \frac{x^k}{e^2 k!}\)\\
\(\sin(x^2)\) & \(x^2-\frac{x^6}{3!} + \frac{x^{10}}{5!} + \frac{x^{14}}{7!} + \cdots\) & \(\sum_{k=0} \frac{(-1)^k x^{2^{2k+1}}}{(2k+1)!}\)\\
\(\frac{1}{1+x^4}\) & \(1 - x^4 + x^8 - x^{16} + \cdots\) & \(\sum_{k=0} (-x^4)^k\)\\
\(e^{\left((x-1)^2\right)}\) & \(1+(x-1)^2 + \frac{(x-1)^4}{2!} + \frac{(x-1)^6}{3!} + \cdots\) & \(\sum_{k=0} \frac{(x-1)^{2k}}{k!}\)\\
\(\frac{\cos(x)-1}{x^2}\) & \(-\frac{1}{2!} + \frac{x^2}{4!} - \frac{x^4}{ 6!} + \cdots\) & \(\sum_{k=1} \frac{(-1)^k x^{2(k-1)}}{(2k)!}\)\\
\(2x \ln (1+2x)\) & \((2x)(2x)-\frac{(2x)(2x)^2}{2} + \frac{(2x)(2x)^3}{3} - \frac{(2x)(2x)^4}{4} + \cdots\) & \(\sum_{k=1}\frac{2x(-1)^{k-1}(2x)^k}{k}\)\\
\(\frac{2x}{1+x^2}\) & \(2x - 2x^3 + 2x^5 - 2x^7 + \cdots\) & \(\sum_{k=0}2x (-1)^k x^{2k}\)\\
\end{tabular}
\end{center}
\section{page 3}
\label{sec:orgcca87be}
\subsection{a: skipped}
\label{sec:orgf3e1d9a}
\subsection{find maclaurin series for \(f'(x)\) where \(f(x) = \sum_{k=0} \frac{(2x)^{k+1}}{k+1}\)}
\label{sec:orgc3ccaef}
\[ \xcancel{\frac{d}{dx} \frac{(2x)^{n+1}}{n+1} = \frac{\cancel{(n+1)^2} (2x)^n (2)}{\cancel{(n+1)^2}} = 2(2x)^n} \]

Instead of using the quotient rule, \(\frac{1}{k+1}\) is a constant for each term so we can just use the chain and power rules:

\[\begin{aligned}
   \frac{d}{dx} \frac{(2x)^{k+1}}{k+1} = \frac{1}{k+1}\frac{d}{dx}(2x)^{k+1} = \frac{1}{\cancel{k+1}}\cancel{(k+1)}(2x)^k(2) = 2(2x)^k
   \end{aligned}\]


So, our series is just
\[\begin{aligned}
   \sum_{k=0} 2(2x)^k = 2 + 4x + 8x^2 + 16x^3 + \cdots
   \end{aligned}\]
\subsection{estimate \(f'\left(-\frac{1}{3}\right)\)}
\label{sec:org9c83683}

When only using the first 4 terms:
\[\begin{aligned}
   2 + 4\frac{-1}{3} + 8 \left(\frac{-1}{3}\right)^2 + 16 \left(\frac{-1}{3}\right)^2 = \frac{10}{3}
   \end{aligned}\]

For the entire sequence:

\[\begin{aligned}
   \sum_{k=0} 2\left(\frac{-2}{3}\right)^k = 2 \sum_{k=0}\left(-\frac{2}{3}\right)^k = \frac{2}{1--\frac{2}{3}} = \frac{2}{\frac{5}{3}} = \frac{6}{5}
   \end{aligned}\]

because the series is geometric.

\section{page 4}
\label{sec:orgc6cf3eb}
\subsection{find \(1-\frac{x^2}{3!}+\frac{x^4}{5!}-\frac{x^6}{7!}+\cdots\)}
\label{sec:orga45188d}
That series is just the taylor series for
\[\begin{aligned}
   f(x) = \frac{\sin x}{x}
   \end{aligned}\]
So the derivative at zero is zero, and the second derivative:
\[\begin{aligned}
\frac{d}{dx} \frac{x \cos x - \sin x}{x^2} =
\frac{x^2 \left( -x \sin x \cancel{+ \cos x - \cos x} \right) - \left( x \cos x - \sin x \right)(2x) }{x^4}\\
= \frac{-x^3 \sin x - 2x\left( x \cos x - \sin x \right) }{x^4}
   \end{aligned}\]
is undefined at zero. However, the top of the fraction will be negative (\(x^3 \sin x\) is like \(x^4\) when \(x \approx 0\) and \(x \cos x - \sin x = x ( \cos x - \frac{\sin x}{x}\)), so the second derivative is zero at \(x\). (Checked with desmos). Thus, the function has a local maximum at \(x = 0\).

\subsection{show approximation at \(x=1\) is within \(\epsilon < \frac{1}{100}\) with \(1-\frac{1}{3!}\)}
\label{sec:org90a82db}

\[\begin{aligned}
   f(1) - \left(1-\frac{1}{3!}\right) &= \frac{1^4}{5!} - \frac{1^6}{7!} + \cdots\\
   &= \frac{1}{5!} - \frac{1}{7!} + \cdots\\
   &< \frac{1}{5!} = \frac{1}{120} < \frac{1}{100}
   \end{aligned}\]

\subsection{solution to the differential equation \(xy' +y = \cos x\)}
\label{sec:orge26c427}

\[\begin{aligned}
   xy' + y = \cos x \implies y &= \cos x - xy'\\
   &= \cos x -  \cancel x \frac{x \cos x - \sin x}{x^{\cancel 2}}\\
   &= \cos x - \frac{x \cos x - \sin x}{x}\\
   &= \cos x - \frac{\cancel x \cos x}{\cancel x} + \frac{\sin x}{x}\\
   &= \cancel{\cos x - \cos x} + \frac{\sin x}{x}\\
   y &= \frac{\sin x}{x}
   \end{aligned}\]
\end{document}
