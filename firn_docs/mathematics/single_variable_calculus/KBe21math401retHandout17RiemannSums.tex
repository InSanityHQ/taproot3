% Created 2021-09-27 Mon 11:53
% Intended LaTeX compiler: xelatex
\documentclass[letterpaper]{article}
\usepackage{graphicx}
\usepackage{grffile}
\usepackage{longtable}
\usepackage{wrapfig}
\usepackage{rotating}
\usepackage[normalem]{ulem}
\usepackage{amsmath}
\usepackage{textcomp}
\usepackage{amssymb}
\usepackage{capt-of}
\usepackage{hyperref}
\setlength{\parindent}{0pt}
\usepackage[margin=1in]{geometry}
\usepackage{fontspec}
\usepackage{svg}
\usepackage{cancel}
\usepackage{indentfirst}
\setmainfont[ItalicFont = LiberationSans-Italic, BoldFont = LiberationSans-Bold, BoldItalicFont = LiberationSans-BoldItalic]{LiberationSans}
\newfontfamily\NHLight[ItalicFont = LiberationSansNarrow-Italic, BoldFont       = LiberationSansNarrow-Bold, BoldItalicFont = LiberationSansNarrow-BoldItalic]{LiberationSansNarrow}
\newcommand\textrmlf[1]{{\NHLight#1}}
\newcommand\textitlf[1]{{\NHLight\itshape#1}}
\let\textbflf\textrm
\newcommand\textulf[1]{{\NHLight\bfseries#1}}
\newcommand\textuitlf[1]{{\NHLight\bfseries\itshape#1}}
\usepackage{fancyhdr}
\pagestyle{fancy}
\usepackage{titlesec}
\usepackage{titling}
\makeatletter
\lhead{\textbf{\@title}}
\makeatother
\rhead{\textrmlf{Compiled} \today}
\lfoot{\theauthor\ \textbullet \ \textbf{2021-2022}}
\cfoot{}
\rfoot{\textrmlf{Page} \thepage}
\renewcommand{\tableofcontents}{}
\titleformat{\section} {\Large} {\textrmlf{\thesection} {|}} {0.3em} {\textbf}
\titleformat{\subsection} {\large} {\textrmlf{\thesubsection} {|}} {0.2em} {\textbf}
\titleformat{\subsubsection} {\large} {\textrmlf{\thesubsubsection} {|}} {0.1em} {\textbf}
\setlength{\parskip}{0.45em}
\renewcommand\maketitle{}
\author{Exr0n}
\date{\today}
\title{Handout 17 Riemann Sums}
\hypersetup{
 pdfauthor={Exr0n},
 pdftitle={Handout 17 Riemann Sums},
 pdfkeywords={},
 pdfsubject={},
 pdfcreator={Emacs 28.0.50 (Org mode 9.4.4)}, 
 pdflang={English}}
\begin{document}

\tableofcontents


\section{Reading}
\label{sec:org10f4173}
\setcounter{subsubsection}{7}
\subsection{Definition of a Definite Integral}
\label{sec:org9c84218}
For each interval \([x_i, x_{i+1}]\), we choose \(x_i^*\) in the interval to be the position of the minimum (for lower bound) or maximum (upper bound) value.

\section{Problems}
\label{sec:org5ae97e7}

\subsection{exr1.3}
\label{sec:org56257f1}
Using the left edge: -8.4375

Summation notation for left edge approximation:
\[\begin{aligned}
   \sum_{i=0}^n \underbrace{\frac{b-a}{n}}_{\text{ width }} \underbrace{f\left(a+\frac{b-a}{n}i\right)}_\text{ height }
   \end{aligned}\]

\subsection{exr1.4 (in class)}
\label{sec:org0280737}
0.21875 using the left estimate

\subsection{exr1.5}
\label{sec:org1216fb8}

\subsubsection{left estimate}
\label{sec:org812e43c}

34.7 feet (add all except last number and divide by two, because we are stopping at 3.0 seconds in.)

\subsubsection{right estimate}
\label{sec:org9ec909c}
44.8 feet (add the last number and drop the zero from the beginning)

\subsubsection{middle estimate}
\label{sec:orgc5ac8ff}
Not enough information to do it for \(\Delta x = 0.5\), so I will use \(n=3\) aka \(\Delta x = 1\)

\[\begin{aligned}
    6.2 + 14.9 + 19.4 = 40.5 \text{ feet }
	\end{aligned}\]

\subsection{exr1.6}
\label{sec:org89a76d1}

\subsubsection{\(\int_0^1 \sqrt{x^2+1}dx\)}
\label{sec:org34bf266}
\(\sqrt{x^2+1}\) is the length the hypotenuse of a triangle with leg-lengths 1 and \(x\). Because \(x\) is continuous, this is like the area of a right triangle with leg-lengths 1 and 1, which is \boxed\{\frac{1}{2}\}.


\begin{enumerate}
\item {\bfseries\sffamily TODO} Wolfram Alpha doesn't agree though
\label{sec:orgd4ad53d}

Probably because as you take approximations, there will be overlap, so the actual value is bigger than I think it is.

I also don't know how to take the anti-chain-rule, so I don't know how to integrate the function symbolically.
\end{enumerate}

\subsubsection{\(\int_0^3 (x-1)dx\)}
\label{sec:orgfb217c5}
Not sure area wise, but the anti-derivative is guess-able:

\[\begin{aligned}
    \frac{d}{dx}\left( \frac{x^2}{2} -x\right) = x-1
	\end{aligned}\]


\[\begin{aligned}
    \frac{3^2}{2}-3 = 1.5
	\end{aligned}\]

\subsection{exr1.7}
\label{sec:org63be5a8}

\subsubsection{right endpoint approx for \(y=x^2\)}
\label{sec:orgfa2093d}

\[\begin{aligned}
    \sum_{i=1}^{n+1} \Delta x f(i \Delta x) = \sum_{i=1}^{n+1} \frac{1}{n} \left( \frac{i}{n}\right) ^2
	\end{aligned}\]
where \(\Delta x = \frac{1}{n}\)

\subsubsection{general form for left-side riemann sum}
\label{sec:orgc2ab568}
See exr1.3

\subsection{exr1.11}
\label{sec:orgfddff9c}

\[\begin{aligned}
   \int_\pi^{2\pi} \cos(x)dx = \lim_{n\to \infty} \sum_{i=0}^n \frac{\pi}{n}\cos\left( \pi + \frac{i\pi}{n}\right)
   \end{aligned}\]

\subsection{exr1.12}
\label{sec:org11e4434}

\subsubsection{1}
\label{sec:orgb461617}
\[\begin{aligned}
    \lim_{n \to \infty} \sum_{k=0}^n \Delta x \sqrt{4+(1+k\Delta x)^2} \text{ where }\Delta x = \frac{2}{n}
	\end{aligned}\]

\subsubsection{2}
\label{sec:orgaf265b1}
\[\begin{aligned}
    \lim_{n \to  \infty} \sum_{k=0}^n \Delta x(2+k\Delta x)^2 + \frac{1}{2+k\Delta x} \text{ where } \Delta x = \frac{3}{n}
	\end{aligned}\]

\subsection{exr1.13}
\label{sec:orge050e33}

\subsubsection{1}
\label{sec:org26eb802}

\[\begin{aligned}
    \int_0^1 \frac{e^x}{1+x} dx
	\end{aligned}\]

\subsubsection{2}
\label{sec:org47e0615}

\[\begin{aligned}
    \int_2^5 x\sqrt{1+x^3}dx
	\end{aligned}\]

\subsubsection{3}
\label{sec:org2188c0f}

\[\begin{aligned}
    \int_1^3 \frac{x}{x^2+4}dx
	\end{aligned}\]

\subsection{exr1.14}
\label{sec:orgf6d23e5}
Graphically, it's the right triangle from origin to B minus the one from origin to A. Algebraically,

\[\begin{aligned}
   \frac{d}{dx}\frac{1}{2}x^2 = x\\
&\implies \int_a^b xdx = \frac{1}{2}b^2 + c - \left(\frac{1}{2} a^2 + c\right) = \frac{b^2-a^2}{2}
   \end{aligned}\]
\end{document}
