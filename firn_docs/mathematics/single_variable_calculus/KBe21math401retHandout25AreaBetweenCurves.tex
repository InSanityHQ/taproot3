% Created 2021-09-27 Mon 11:53
% Intended LaTeX compiler: xelatex
\documentclass[letterpaper]{article}
\usepackage{graphicx}
\usepackage{grffile}
\usepackage{longtable}
\usepackage{wrapfig}
\usepackage{rotating}
\usepackage[normalem]{ulem}
\usepackage{amsmath}
\usepackage{textcomp}
\usepackage{amssymb}
\usepackage{capt-of}
\usepackage{hyperref}
\setlength{\parindent}{0pt}
\usepackage[margin=1in]{geometry}
\usepackage{fontspec}
\usepackage{svg}
\usepackage{cancel}
\usepackage{indentfirst}
\setmainfont[ItalicFont = LiberationSans-Italic, BoldFont = LiberationSans-Bold, BoldItalicFont = LiberationSans-BoldItalic]{LiberationSans}
\newfontfamily\NHLight[ItalicFont = LiberationSansNarrow-Italic, BoldFont       = LiberationSansNarrow-Bold, BoldItalicFont = LiberationSansNarrow-BoldItalic]{LiberationSansNarrow}
\newcommand\textrmlf[1]{{\NHLight#1}}
\newcommand\textitlf[1]{{\NHLight\itshape#1}}
\let\textbflf\textrm
\newcommand\textulf[1]{{\NHLight\bfseries#1}}
\newcommand\textuitlf[1]{{\NHLight\bfseries\itshape#1}}
\usepackage{fancyhdr}
\pagestyle{fancy}
\usepackage{titlesec}
\usepackage{titling}
\makeatletter
\lhead{\textbf{\@title}}
\makeatother
\rhead{\textrmlf{Compiled} \today}
\lfoot{\theauthor\ \textbullet \ \textbf{2021-2022}}
\cfoot{}
\rfoot{\textrmlf{Page} \thepage}
\renewcommand{\tableofcontents}{}
\titleformat{\section} {\Large} {\textrmlf{\thesection} {|}} {0.3em} {\textbf}
\titleformat{\subsection} {\large} {\textrmlf{\thesubsection} {|}} {0.2em} {\textbf}
\titleformat{\subsubsection} {\large} {\textrmlf{\thesubsubsection} {|}} {0.1em} {\textbf}
\setlength{\parskip}{0.45em}
\renewcommand\maketitle{}
\author{Taproot}
\date{\today}
\title{Handout 25 Additional Answers}
\hypersetup{
 pdfauthor={Taproot},
 pdftitle={Handout 25 Additional Answers},
 pdfkeywords={},
 pdfsubject={},
 pdfcreator={Emacs 28.0.50 (Org mode 9.4.4)}, 
 pdflang={English}}
\begin{document}

\tableofcontents

\setcounter{section}{10}

\section{cubic and a line}
\label{sec:org73dcfca}

\subsection{show tangency}
\label{sec:org062283f}

\[\begin{aligned}
   \frac{dy}{dx} = \frac{d}{dx}(4 x ^2 - x ^3) = 8 x - 3 x^2&\Bigr|_3\\
   &= 24 - 27 = -3\\
   \frac{dy}{dx} = \frac{d}{dx}(18 - 3 x) = -3 \\
   \end{aligned}\]

\subsection{area between curves}
\label{sec:org8bc175b}

\[\begin{aligned}
    \int_{3}^{6} 18-3x - 4x^2 + x^3 dx &\to \frac{1}{4}x^4 - \frac{1}{3}4x^3 - \frac{1}{2} 3x^2 + 18 x +C\\
	&= \frac{1}{4}(6)^4 - \frac{1}{3}4(6)^3 - \frac{1}{2} 3(6)^2 + 18 (6) \\
	&- \frac{1}{4}(3)^4 + \frac{1}{3}4(3)^3 + \frac{1}{2} 3(3)^2 - 18 (3)\\
	&= \boxed{\frac{261}{4}}
   \end{aligned}\]

\section{estimate area}
\label{sec:org8ac5755}

Right handed Riemann Sum:
\[\begin{aligned}
  0.5 + 4 + 10 + 13 + 10 + 0 = 37.5
  \end{aligned}\]

\section{estimate area again}
\label{sec:org015135d}

\[\begin{aligned}
  4(200 + 2700 + 1100 + 4000 + 200) = 32800
  \end{aligned}\]

\section{area between curves}
\label{sec:orgaf6b83d}

\[\begin{aligned}
  \int_{0}^{10} 2200e^{0.024t} dx - \int_0^{10} 1360e^{0.018t} dx &= \frac{1}{0.024} 2200e^{0.024t} - \frac{1}{0.018} 1360e^{0.018t}\\
  \implies &\quad \ \frac{1}{0.024} 2200e^{0.24} - \frac{1}{0.018} 1360e^{0.18} - \frac{1}{0.024} 2200 + \frac{1}{0.018} 1360
  &\approx  9964
  \end{aligned}\]

The area represents the population over those ten years.

\section{meaning of area}
\label{sec:orgbc3114e}
The shaded region represents the profit made between producing 50 units and 100 units.

\section{slicing pizza into three using parallel cuts}
\label{sec:org37bebb8}
The problem of placing slices is the same if we only worry about the top half of the pizza. Thus, we can choose some \(x\) for the first slice s.t.

\[\begin{aligned}
  2\int_{-7}^{x} \sqrt{7^2 - t^2} dt &= \int_{x}^{7} \sqrt{7^2 - t^2} dt\\
  2\int_{-7}^{x} \sqrt{7^2 - t^2} dt &- \int_{x}^{7} \sqrt{7^2 - t^2} dt = 0\\
  2\int_{-7}^{x} \sqrt{7^2 - t^2} dt &+ \int_{7}^{x} \sqrt{7^2 - t^2} dt = 0\\
  2\left( \int_{0}^{x} \sqrt{7^2 - t^2} dt - \int_{0}^{-7} \sqrt{7^2 - t^2} dt \right)  &+\left( \int_{0}^{x} \sqrt{7^2 - t^2} dt - \int_{0}^{7} \sqrt{7^2 - t^2} dt \right)  = 0\\
  2\left( \int_{0}^{x} \sqrt{7^2 - t^2} dt + \int_{-7}^{0} \sqrt{7^2 - t^2} dt \right)  &+\left( \int_{0}^{x} \sqrt{7^2 - t^2} dt - \int_{0}^{7} \sqrt{7^2 - t^2} dt \right)  = 0\\
  2\int_{0}^{x} \sqrt{7^2 - t^2} dt + 2\int_{-7}^{0} \sqrt{7^2 - t^2} dt  &+\int_{0}^{x} \sqrt{7^2 - t^2} dt - \int_{0}^{7} \sqrt{7^2 - t^2} dt = 0\\
  3\int_{0}^{x} \sqrt{7^2 - t^2} dt + 2\int_{-7}^{0} \sqrt{7^2 - t^2} dt  &- \int_{0}^{7} \sqrt{7^2 - t^2} dt = 0\\
  3\int_{0}^{x} \sqrt{7^2 - t^2} dt + \int_{-7}^{0} \sqrt{7^2 - t^2} dt &= 0\\
  3\int_{0}^{x} \sqrt{7^2 - t^2} dt + \frac{49\pi}{4}  &= 0\\
  \end{aligned}\]
Now, we need to use trigonometric substitution, apparently.
\[\begin{aligned}
  x = a \sin \theta, dx = a \cos  \theta d \theta
  \end{aligned}\]


Or, you could use David's method, which is just better (cut horizontally instead of vertically)
\[\begin{aligned}
  \int_{-7}^{7} \sqrt{49-x^2} - a dx = \frac{49\pi }{3}\\
  \int_{-7}^{7} \sqrt{49-x^2} dx -\int_{-7}^{7}  a dx = \frac{49\pi }{3}\\
  \frac{49\pi }{2} -\int_{-7}^{7}  a dx = \frac{49\pi }{3}\\
  \frac{49\pi }{2} -\left( 7a - -7a \right)  = \frac{49\pi }{3}\\
  \frac{49\pi }{6} = 14a\\
  a = \frac{49\pi }{84} = \frac{7\pi }{12}
  \end{aligned}\]
Since \(a\) is the upper half of the center portion, the width of each slice is \(2a = \frac{7 \pi}{6}\)

\section{tractrix}
\label{sec:org9d2c6be}

\subsection{derivative}
\label{sec:org7031b3d}

At any moment, if the boat is at \((x, y)\) and the puller is at \((0, h)\), then the velocity of the boat is in the direction
\[\begin{aligned}
  \frac{\Delta y}{\Delta x}
  \end{aligned}\]

Where \(\Delta x = -x\) and \(\Delta y\) can be found using the Pythagorean Theorem

\[\begin{aligned}
  &&L ^2 &= \Delta y ^2 + x ^2\\
  \implies && \Delta y &= \sqrt{L^2 - x ^2}
  \end{aligned}\]

Thus, the boat is moving in the direction

\[\begin{aligned}
  \frac{\sqrt{L^2-x^2}}{ -x}
  \end{aligned}\]

\subsection{integral}
\label{sec:orge44f031}

\[\begin{aligned}
   \int \frac{\sqrt{L^2-x^2}}{-x} dx &= - \int \frac{1}{x} \sqrt{L^2 - x^2} dx\\
   \text{Let } x = L \sin \theta, dx = L \cos  \theta d\theta \\
   &= - \int \frac{1}{L \sin \theta } \sqrt{L^2 - L ^2 \sin  ^2 \theta } dx\\
   &= - \int \frac{1}{L \sin \theta } L \sqrt{1 - \sin  ^2 \theta } dx\\
   &= - \int \frac{1}{L \sin \theta } L \sqrt{\cos  ^2 \theta } dx\\
   &= - \int \frac{L \cos  \theta }{L \sin  \theta } L \cos  \theta  d\theta \\
   &= - \int L \frac{\cos ^2 \theta}{\sin  \theta} d\theta\\
   &= - L \int \frac{1}{\sin \theta } d\theta  + L \int \sin  \theta  d\theta\\
   &= L \ln  \lvert \csc \theta  + \cot  \theta  \rvert - L \cos  \theta +C\\
   &= L \ln  \left\lvert \frac{L}{x}  + \frac{\sqrt{L^2-x^2}}{x} \right\rvert - \sqrt{L^2-x^2} +C
   \end{aligned}\]

\section{{\bfseries\sffamily TODO} water displacement}
\label{sec:orgeeb3837}

Plan: find a function \(f(r)\) which represents the amount of water displaced for any radius, then take the derivative and find roots.
\end{document}
