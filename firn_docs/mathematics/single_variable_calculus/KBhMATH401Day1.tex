% Created 2021-09-27 Mon 12:03
% Intended LaTeX compiler: xelatex
\documentclass[letterpaper]{article}
\usepackage{graphicx}
\usepackage{grffile}
\usepackage{longtable}
\usepackage{wrapfig}
\usepackage{rotating}
\usepackage[normalem]{ulem}
\usepackage{amsmath}
\usepackage{textcomp}
\usepackage{amssymb}
\usepackage{capt-of}
\usepackage{hyperref}
\setlength{\parindent}{0pt}
\usepackage[margin=1in]{geometry}
\usepackage{fontspec}
\usepackage{svg}
\usepackage{cancel}
\usepackage{indentfirst}
\setmainfont[ItalicFont = LiberationSans-Italic, BoldFont = LiberationSans-Bold, BoldItalicFont = LiberationSans-BoldItalic]{LiberationSans}
\newfontfamily\NHLight[ItalicFont = LiberationSansNarrow-Italic, BoldFont       = LiberationSansNarrow-Bold, BoldItalicFont = LiberationSansNarrow-BoldItalic]{LiberationSansNarrow}
\newcommand\textrmlf[1]{{\NHLight#1}}
\newcommand\textitlf[1]{{\NHLight\itshape#1}}
\let\textbflf\textrm
\newcommand\textulf[1]{{\NHLight\bfseries#1}}
\newcommand\textuitlf[1]{{\NHLight\bfseries\itshape#1}}
\usepackage{fancyhdr}
\pagestyle{fancy}
\usepackage{titlesec}
\usepackage{titling}
\makeatletter
\lhead{\textbf{\@title}}
\makeatother
\rhead{\textrmlf{Compiled} \today}
\lfoot{\theauthor\ \textbullet \ \textbf{2021-2022}}
\cfoot{}
\rfoot{\textrmlf{Page} \thepage}
\renewcommand{\tableofcontents}{}
\titleformat{\section} {\Large} {\textrmlf{\thesection} {|}} {0.3em} {\textbf}
\titleformat{\subsection} {\large} {\textrmlf{\thesubsection} {|}} {0.2em} {\textbf}
\titleformat{\subsubsection} {\large} {\textrmlf{\thesubsubsection} {|}} {0.1em} {\textbf}
\setlength{\parskip}{0.45em}
\renewcommand\maketitle{}
\author{Houjun Liu}
\date{\today}
\title{Day 1 with Veena}
\hypersetup{
 pdfauthor={Houjun Liu},
 pdftitle={Day 1 with Veena},
 pdfkeywords={},
 pdfsubject={},
 pdfcreator={Emacs 28.0.50 (Org mode 9.4.4)}, 
 pdflang={English}}
\begin{document}

\tableofcontents



\section{General Aftercare}
\label{sec:orge9b7e8f}
\begin{itemize}
\item Assignments on Canvas (preferably a PDF)
\item Collaborate as much as possible

\begin{itemize}
\item Learn and share ideas together
\item Collaborate well together
\end{itemize}

\item Nikhil TAing!! ;)
\item \textasciitilde{}30 mins of HW/class period. \emph{time} the assignments and write it down
on top.
\item Tests are take home, duh (COVID NOISES!!), and are Assigned Wednesday,
Due on Monday)
\end{itemize}

\textbf{Expectations} * A notebook should be maintained + some solved sample
problems * Homework assigned each class * HW graded for Habits of Mind
rubric * One graded assignments every two weeks or so

Textbooks: \begin{center}
\includegraphics[width=.9\linewidth]{./Screen Shot 2020-08-24 at 1.11.22 PM.png}
\end{center}

\noindent\rule{\textwidth}{0.5pt}

\section{Knowledge Points This Semester}
\label{sec:org59cb555}
\begin{itemize}
\item Limits

\begin{itemize}
\item Eplison delta proofs
\item Limit laws
\item Evaluating functions
\item Prove limit laws
\end{itemize}

\item Continuity

\begin{itemize}
\item Types of continuity + discontinuity
\item Define continuity
\item Immediate value theorem

\begin{itemize}
\item Application +
\item Boundedness
\end{itemize}
\end{itemize}

\item Derivatives

\begin{itemize}
\item Limit definition of derivatives
\item Define differentialibity
\item Understand how the first and second order derivatives
\item Talor Series approximations
\item L'Hospital rules for limits w/ indeterminate rations, indeterminate
products, indeterminate products
\end{itemize}

\item \begin{itemize}
\item a final project
\end{itemize}
\end{itemize}

\textbf{Everything you use on tests must be derived by you.}

=> Make test corrections + consider reassessing (immediately) if
necessary + meet with instructors \& TAs during \emph{Wednesday lunch} or
\emph{Friday tutorial}

\noindent\rule{\textwidth}{0.5pt}

\section{So, what \emph{is} Calculus?}
\label{sec:orgda70480}
\begin{itemize}
\item The analysis of change
\item Study of curves
\item Study of rate-of-change
\end{itemize}

\subsection{Rate of change}
\label{sec:orgd6ae890}
We have seen this before: \textbf{Slopes!}

\begin{quote}
The rate of change tells you the relation in the trend of the graph.
Think! Negative and positive functions!
\end{quote}

\definition{First order rate of change}{How much is the function changing over a period of time?}
\definition{Second order difference}{How much is the rate of change changing over time?}
\end{document}
