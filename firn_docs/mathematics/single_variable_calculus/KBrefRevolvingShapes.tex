% Created 2021-09-27 Mon 11:53
% Intended LaTeX compiler: xelatex
\documentclass[letterpaper]{article}
\usepackage{graphicx}
\usepackage{grffile}
\usepackage{longtable}
\usepackage{wrapfig}
\usepackage{rotating}
\usepackage[normalem]{ulem}
\usepackage{amsmath}
\usepackage{textcomp}
\usepackage{amssymb}
\usepackage{capt-of}
\usepackage{hyperref}
\setlength{\parindent}{0pt}
\usepackage[margin=1in]{geometry}
\usepackage{fontspec}
\usepackage{svg}
\usepackage{cancel}
\usepackage{indentfirst}
\setmainfont[ItalicFont = LiberationSans-Italic, BoldFont = LiberationSans-Bold, BoldItalicFont = LiberationSans-BoldItalic]{LiberationSans}
\newfontfamily\NHLight[ItalicFont = LiberationSansNarrow-Italic, BoldFont       = LiberationSansNarrow-Bold, BoldItalicFont = LiberationSansNarrow-BoldItalic]{LiberationSansNarrow}
\newcommand\textrmlf[1]{{\NHLight#1}}
\newcommand\textitlf[1]{{\NHLight\itshape#1}}
\let\textbflf\textrm
\newcommand\textulf[1]{{\NHLight\bfseries#1}}
\newcommand\textuitlf[1]{{\NHLight\bfseries\itshape#1}}
\usepackage{fancyhdr}
\pagestyle{fancy}
\usepackage{titlesec}
\usepackage{titling}
\makeatletter
\lhead{\textbf{\@title}}
\makeatother
\rhead{\textrmlf{Compiled} \today}
\lfoot{\theauthor\ \textbullet \ \textbf{2021-2022}}
\cfoot{}
\rfoot{\textrmlf{Page} \thepage}
\renewcommand{\tableofcontents}{}
\titleformat{\section} {\Large} {\textrmlf{\thesection} {|}} {0.3em} {\textbf}
\titleformat{\subsection} {\large} {\textrmlf{\thesubsection} {|}} {0.2em} {\textbf}
\titleformat{\subsubsection} {\large} {\textrmlf{\thesubsubsection} {|}} {0.1em} {\textbf}
\setlength{\parskip}{0.45em}
\renewcommand\maketitle{}
\author{Taproot}
\date{\today}
\title{Revolving Shapes}
\hypersetup{
 pdfauthor={Taproot},
 pdftitle={Revolving Shapes},
 pdfkeywords={},
 pdfsubject={},
 pdfcreator={Emacs 28.0.50 (Org mode 9.4.4)}, 
 pdflang={English}}
\begin{document}

\tableofcontents

\section{an example: semicircle revolved around the x-axis to create a sphere}
\label{sec:org246a374}
We can make cuts perpendicular to the axis of rotation. In this case, you end up with a bunch of circular disks, where the height of each slice is your semicircle function.

Thus, the volume of the disk is
\[\begin{aligned}
   \pi f^2(x_i) \Delta x = (a^2-x^2) \pi \Delta x
  \end{aligned}\]

This is kinda like a Riemann Sum, but with more stuff added on. We can take the limit of the sum

\[\begin{aligned}
   \lim_{n \to \infty} \sum_{k=1}^{n} \pi (a^2 - x_i^2) \Delta x
  \end{aligned}\]

Where \(\Delta x = \frac{1}{n}\) and \(x_i = -a + \frac{2ak}{n}\)

Expressed as an integral:

\[\begin{aligned}
   \int_{-a}^{a} \pi (a^2-x^2) dx &\to  \int \pi a^2 dx - \int \pi x^2 dx \\
   &= \pi a^2 x - \pi \frac{1}{3}x^3\\
   &\to \pi a^3 - \pi \frac{1}{3} a^3 + \pi a^3 + \pi \frac{1}{3} (-a)^3\\
   &= 2\pi a^3 - \pi \frac{2}{3} a^3\\
   &= \frac{4}{3} \pi a^3
  \end{aligned}\]
\section{now lets try a cone}
\label{sec:org592ba06}

Rotate
\[\begin{aligned}
  y = -ax +b
  \end{aligned}\]
Around the y-axis. Then, each circle (which is layed out flat) has thickness \(dy\) and radius \(x\) or \(\frac{y-b}{-a}\)

The volume of the disk is then
\[\begin{aligned}
  \pi  \left(\frac{y-b}{-a}\right)  ^2 dy
  \end{aligned}\]

Or using \(r, h\) as the radius and height of the cone,

\[\begin{aligned}
   \pi  \left(r-\frac{r}{h}y \right)  ^2 dy
  \end{aligned}\]

And we can take the integral of that from \(0\) to \(h\)

\[\xcancel{\begin{aligned}
  \int_{0}^{h}  \pi  \left(r-\frac{r}{h}y \right)  ^2 dy &\to \pi  \int \left(r - \frac{r}{h}y \right)  ^2 dx \\
  \text{Let } u = r-\frac{r}{h}y, du = -\frac{r}{h} dx\\
  &= \pi -\frac{h}{r} \int u ^2 du\\
  &= -\pi \frac{h}{r} \frac{1}{3}u^3 +C\\
  &= -\pi \frac{h}{r} \frac{1}{3} \left(r-\frac{r}{h}y \right)  ^3
  \end{aligned}}\]


\[\begin{aligned}
  \int_{0}^{h}  \pi  \left(r-\frac{r}{h}y \right)  ^2 dy &\to \pi  \int r^2 +   \left(\frac{r}{h}y \right)  ^2 - 2 r  \left(\frac{r}{h}y \right)  dx \\
  &= \frac{1}{3} \pi r^3 + \frac{1}{3}\frac{r^2}{h^2}y^3
  \end{aligned}\]
\end{document}
