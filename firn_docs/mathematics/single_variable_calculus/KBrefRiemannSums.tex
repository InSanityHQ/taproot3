% Created 2021-09-27 Mon 11:53
% Intended LaTeX compiler: xelatex
\documentclass[letterpaper]{article}
\usepackage{graphicx}
\usepackage{grffile}
\usepackage{longtable}
\usepackage{wrapfig}
\usepackage{rotating}
\usepackage[normalem]{ulem}
\usepackage{amsmath}
\usepackage{textcomp}
\usepackage{amssymb}
\usepackage{capt-of}
\usepackage{hyperref}
\setlength{\parindent}{0pt}
\usepackage[margin=1in]{geometry}
\usepackage{fontspec}
\usepackage{svg}
\usepackage{cancel}
\usepackage{indentfirst}
\setmainfont[ItalicFont = LiberationSans-Italic, BoldFont = LiberationSans-Bold, BoldItalicFont = LiberationSans-BoldItalic]{LiberationSans}
\newfontfamily\NHLight[ItalicFont = LiberationSansNarrow-Italic, BoldFont       = LiberationSansNarrow-Bold, BoldItalicFont = LiberationSansNarrow-BoldItalic]{LiberationSansNarrow}
\newcommand\textrmlf[1]{{\NHLight#1}}
\newcommand\textitlf[1]{{\NHLight\itshape#1}}
\let\textbflf\textrm
\newcommand\textulf[1]{{\NHLight\bfseries#1}}
\newcommand\textuitlf[1]{{\NHLight\bfseries\itshape#1}}
\usepackage{fancyhdr}
\pagestyle{fancy}
\usepackage{titlesec}
\usepackage{titling}
\makeatletter
\lhead{\textbf{\@title}}
\makeatother
\rhead{\textrmlf{Compiled} \today}
\lfoot{\theauthor\ \textbullet \ \textbf{2021-2022}}
\cfoot{}
\rfoot{\textrmlf{Page} \thepage}
\renewcommand{\tableofcontents}{}
\titleformat{\section} {\Large} {\textrmlf{\thesection} {|}} {0.3em} {\textbf}
\titleformat{\subsection} {\large} {\textrmlf{\thesubsection} {|}} {0.2em} {\textbf}
\titleformat{\subsubsection} {\large} {\textrmlf{\thesubsubsection} {|}} {0.1em} {\textbf}
\setlength{\parskip}{0.45em}
\renewcommand\maketitle{}
\author{Taproot}
\date{\today}
\title{Riemann Sums}
\hypersetup{
 pdfauthor={Taproot},
 pdftitle={Riemann Sums},
 pdfkeywords={},
 pdfsubject={},
 pdfcreator={Emacs 28.0.50 (Org mode 9.4.4)}, 
 pdflang={English}}
\begin{document}

\tableofcontents

\section{Slicing into Rectangles}
\label{sec:orgceee43f}
The general idea of Riemann sums is to slice a curve into vertical non-overlapping rectangles to approximate the area between the curve and the x-axis. This can be expressed mathematically as a summation given the function \(f(x)\), the range \([a, b]\), and the number of rectangles \(n\):

\[\begin{aligned}
  \sum_{k=1}^{n} \frac{b-a}{n} f(a + k\frac{b-a}{n})
  \end{aligned}\]

This can be written more concisely by defining \(\Delta x = \frac{b-a}{n}\) and \(x_k = a + k \Delta x\):

\[\begin{aligned}
  \sum_{k=1}^{n} \Delta x f(x_i)
  \end{aligned}\]

These estimates all have the right endpoint of the rectangle touching the curve. You could also use the left endpoint, or use the minimum value one and add a triangle to form a trapezoid.

\section{Area Interpretation}
\label{sec:org371bf5d}
Areas under curves can be estimated if you recognize the function. For example:
\[\begin{aligned}
  \int_{0}^{1} \sqrt{1-x^2} dx
  \end{aligned}\]
Traces out a quarter of a semicircle, so the area under this curve is \(\frac{\pi}{4}\)

\section{Upper and Lower Bound}
\label{sec:org410aae1}
To get an upper and lower bound approximation using a Riemann sum, you cannot always take the left or right edge. Instead, you have to take the minimum or maximum in an interval, usually denoted \(f(x_i^*)\).
\section{the Definite Integral}
\label{sec:org27b66d6}
Finally, we can define the definite integral as a limit of Riemann sums.

\[\begin{aligned}
  \int_{a}^{b} f(x) dx =\lim_{n \to \infty } \sum_{k=1}^{n} f(x_i) \Delta x
  \end{aligned}\]

Where once again, \(\Delta x = \frac{b-a}{n}\) and \(x_k = a+k\Delta x\)

These integrals can be evaluated directly with a lot of algebra and some \href{file:///home/hliu/projects/Taproot/math/countingandprobability/KBrefSumFromOneToN.org}{triangular number tricks}.
\end{document}
