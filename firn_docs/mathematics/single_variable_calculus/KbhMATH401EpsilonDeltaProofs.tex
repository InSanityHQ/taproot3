% Created 2021-09-27 Mon 12:03
% Intended LaTeX compiler: xelatex
\documentclass[letterpaper]{article}
\usepackage{graphicx}
\usepackage{grffile}
\usepackage{longtable}
\usepackage{wrapfig}
\usepackage{rotating}
\usepackage[normalem]{ulem}
\usepackage{amsmath}
\usepackage{textcomp}
\usepackage{amssymb}
\usepackage{capt-of}
\usepackage{hyperref}
\setlength{\parindent}{0pt}
\usepackage[margin=1in]{geometry}
\usepackage{fontspec}
\usepackage{svg}
\usepackage{cancel}
\usepackage{indentfirst}
\setmainfont[ItalicFont = LiberationSans-Italic, BoldFont = LiberationSans-Bold, BoldItalicFont = LiberationSans-BoldItalic]{LiberationSans}
\newfontfamily\NHLight[ItalicFont = LiberationSansNarrow-Italic, BoldFont       = LiberationSansNarrow-Bold, BoldItalicFont = LiberationSansNarrow-BoldItalic]{LiberationSansNarrow}
\newcommand\textrmlf[1]{{\NHLight#1}}
\newcommand\textitlf[1]{{\NHLight\itshape#1}}
\let\textbflf\textrm
\newcommand\textulf[1]{{\NHLight\bfseries#1}}
\newcommand\textuitlf[1]{{\NHLight\bfseries\itshape#1}}
\usepackage{fancyhdr}
\pagestyle{fancy}
\usepackage{titlesec}
\usepackage{titling}
\makeatletter
\lhead{\textbf{\@title}}
\makeatother
\rhead{\textrmlf{Compiled} \today}
\lfoot{\theauthor\ \textbullet \ \textbf{2021-2022}}
\cfoot{}
\rfoot{\textrmlf{Page} \thepage}
\renewcommand{\tableofcontents}{}
\titleformat{\section} {\Large} {\textrmlf{\thesection} {|}} {0.3em} {\textbf}
\titleformat{\subsection} {\large} {\textrmlf{\thesubsection} {|}} {0.2em} {\textbf}
\titleformat{\subsubsection} {\large} {\textrmlf{\thesubsubsection} {|}} {0.1em} {\textbf}
\setlength{\parskip}{0.45em}
\renewcommand\maketitle{}
\author{Houjun Liu}
\date{\today}
\title{Epsilon Delta Proofs}
\hypersetup{
 pdfauthor={Houjun Liu},
 pdftitle={Epsilon Delta Proofs},
 pdfkeywords={},
 pdfsubject={},
 pdfcreator={Emacs 28.0.50 (Org mode 9.4.4)}, 
 pdflang={English}}
\begin{document}

\tableofcontents



\section{Epsilon Delta Proofs}
\label{sec:orgfe93402}
The secrets of the limit

\subsection{Formal Definition of a Limit}
\label{sec:org170e843}
\definition[for all $\epsilon > 0$, there exists a $\delta$ such that $if\ 0<|x-a|<\delta,\ then\  0<|f(x)-L|<\epsilon$]{Limit Definition}\{\(\lim_{x\to a} f(x) = L\)\}

\subsection{An Epsilon Delta Proof}
\label{sec:orgf4a9865}
Let's prove \(\lim_{x\to 2} x^2 = 4\) together!

The crux of the proof is to come up with a value \(\delta\) that is a
function of \(\epsilon\) assuming that \(0 < \epsilon\) that meets
\(0<|x-a|<\delta\).

Oh, here's some symbols

\begin{center}
\begin{tabular}{ll}
Symbol & Definition\\
\hline
\(\forall\) & For all\\
\(\exists\) & There exisits\\
\(s.t.\) & Such that\\
\end{tabular}
\end{center}

And so, the formal and pretentious definition of a limit:

\(\lim_{x\to a} f(x) = L\ where\ \forall \epsilon > 0, \exists \delta > 0,\ s.t.\ 0<|x-a|<\delta \to |f(x) -L|<\epsilon.\)

This needs to go before \textbf{every Epsilon Delta proof.}

\begin{itemize}
\item Step 1: Re-write the Definition Above w.r.t. the function
\item Step 2: Do scratch work to identify delta
\item Step 3: Plug it in to verify
\end{itemize}
\end{document}
