% Created 2021-09-27 Mon 12:02
% Intended LaTeX compiler: xelatex
\documentclass[letterpaper]{article}
\usepackage{graphicx}
\usepackage{grffile}
\usepackage{longtable}
\usepackage{wrapfig}
\usepackage{rotating}
\usepackage[normalem]{ulem}
\usepackage{amsmath}
\usepackage{textcomp}
\usepackage{amssymb}
\usepackage{capt-of}
\usepackage{hyperref}
\setlength{\parindent}{0pt}
\usepackage[margin=1in]{geometry}
\usepackage{fontspec}
\usepackage{svg}
\usepackage{cancel}
\usepackage{indentfirst}
\setmainfont[ItalicFont = LiberationSans-Italic, BoldFont = LiberationSans-Bold, BoldItalicFont = LiberationSans-BoldItalic]{LiberationSans}
\newfontfamily\NHLight[ItalicFont = LiberationSansNarrow-Italic, BoldFont       = LiberationSansNarrow-Bold, BoldItalicFont = LiberationSansNarrow-BoldItalic]{LiberationSansNarrow}
\newcommand\textrmlf[1]{{\NHLight#1}}
\newcommand\textitlf[1]{{\NHLight\itshape#1}}
\let\textbflf\textrm
\newcommand\textulf[1]{{\NHLight\bfseries#1}}
\newcommand\textuitlf[1]{{\NHLight\bfseries\itshape#1}}
\usepackage{fancyhdr}
\pagestyle{fancy}
\usepackage{titlesec}
\usepackage{titling}
\makeatletter
\lhead{\textbf{\@title}}
\makeatother
\rhead{\textrmlf{Compiled} \today}
\lfoot{\theauthor\ \textbullet \ \textbf{2021-2022}}
\cfoot{}
\rfoot{\textrmlf{Page} \thepage}
\renewcommand{\tableofcontents}{}
\titleformat{\section} {\Large} {\textrmlf{\thesection} {|}} {0.3em} {\textbf}
\titleformat{\subsection} {\large} {\textrmlf{\thesubsection} {|}} {0.2em} {\textbf}
\titleformat{\subsubsection} {\large} {\textrmlf{\thesubsubsection} {|}} {0.1em} {\textbf}
\setlength{\parskip}{0.45em}
\renewcommand\maketitle{}
\author{Dylan Wallace}
\date{\today}
\title{Physics 250 Day 1}
\hypersetup{
 pdfauthor={Dylan Wallace},
 pdftitle={Physics 250 Day 1},
 pdfkeywords={},
 pdfsubject={},
 pdfcreator={Emacs 28.0.50 (Org mode 9.4.4)}, 
 pdflang={English}}
\begin{document}

\tableofcontents



\section{Experiments}
\label{sec:orgc4be434}
Basically, we just rubbed a bunch of things on each other and checked
the resulting charge with an electrometer.

\subsection{Interesting results}
\label{sec:org3e28ac6}
\begin{itemize}
\item Combs are great for static electricity
\item Rubbing some objects on others caused similar charges, while other
object caused different charges
\item These notes are in hindsight so I legit don't remember too much
\end{itemize}

\section{Explanation}
\label{sec:org1316067}
\begin{itemize}
\item Opposite charges attract; similar charges repel
\item When charged object is brought close to a conductor, electrons in the
conductor will flow and polarize the conductor
\item When charged object is brought close to an insulator, atoms inside the
insulator will be polarized. With small objects, this can make the
whole object be basically polarized.
\item When a charged object makes contact with a conductor, the electrons
will be shared between objects.
\end{itemize}

\section{Homework}
\label{sec:org4ef0251}
\subsection{Lecture Notes}
\label{sec:orgec7a406}
Might not be complete.

\subsubsection{Electrostatics Basics}
\label{sec:org8c15298}
\begin{itemize}
\item There are Insulators and Conductors

\begin{itemize}
\item Insulators: Don't share electrons
\item Conductors: Share electrons
\item Learn why this is in solid state physics
\end{itemize}

\item List of charges when rubbed

\begin{itemize}
\item Plastics usually become negative
\item Fur, elastics usually become positive
\end{itemize}

\item Electrons can be shared between materials
\item Electrons can move somewhat freely (depending on the material) within
an object

\begin{itemize}
\item Especially when close to another charged object!
\end{itemize}

\item Even in materials where electrons can't move freely (e.g. paper, other
insulators), polarization can cause a "chain reaction" and "polarize"
the object as a whole
\end{itemize}

\subsubsection{Quantification}
\label{sec:org1b13a3b}
\begin{itemize}
\item Coulomb's Law

\begin{itemize}
\item Given two point charges, Q1 and Q2, and a distance r

\item \(F = k \frac{q_1 q_2}{r^2}\)

\begin{itemize}
\item \(k\) is \(8.99\times 10^{9}Nm^{2}C^{-2}\)
\item \(r\) is in meters
\item \(q_1\), \(q_2\) in Coulombs (\(C\))
\item if \(F > 0\), then force is repulsion
\item if \(F < 0\), then force is attraction
\end{itemize}

\item Sample Problem: Find distance (\(r\)) given \(q_1\), \(q_2\), and
\(F\) $\backslash$[
\begin{aligned}
q_1 &= 50uC &= 50\times 10^{-6}C \\
q_2 &= 1uC &= 1\times 10^{-6}C \\
F_{12} &= 2N \\
k &= 8.99\times 10^{9}Nm^{2}C^{-2} \\
F &= k \frac{q_1q_2}{r^2} \\
r^2 &= k \frac{q_1q_2}{F} &= 8.99\times 10^{9}Nm^{2}C^{-2} \cdot 50\times 10^{-12}C^{2} \div 2N \\
&= 224.75 \times 10^{-3}m \\
r &= \sqrt{224.75 \times 10^{-3}}m \\
&= 474\times 10^{-3}m
\end{aligned}
$\backslash$]

\item In more complicated setups, certain things such as acceleration
won't be constant because it is determinant on force, which is
determined by distance from other charges.

\begin{itemize}
\item This complicates things so don't expect it to be simple.
\end{itemize}
\end{itemize}
\end{itemize}

\subsubsection{Vector Fields}
\label{sec:org3d2603a}
\begin{itemize}
\item Fields of vectors

\begin{itemize}
\item Vector magnitude is in \(NC^{-1}\) (Newtons per Coulomb)
\item Behave in interesting ways i guess i dunno
\item Calculate using a hypothetical proton
\end{itemize}
\end{itemize}
\end{document}
