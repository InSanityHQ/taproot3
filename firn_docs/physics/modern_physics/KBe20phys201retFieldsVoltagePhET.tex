% Created 2021-09-27 Mon 12:02
% Intended LaTeX compiler: xelatex
\documentclass[letterpaper]{article}
\usepackage{graphicx}
\usepackage{grffile}
\usepackage{longtable}
\usepackage{wrapfig}
\usepackage{rotating}
\usepackage[normalem]{ulem}
\usepackage{amsmath}
\usepackage{textcomp}
\usepackage{amssymb}
\usepackage{capt-of}
\usepackage{hyperref}
\setlength{\parindent}{0pt}
\usepackage[margin=1in]{geometry}
\usepackage{fontspec}
\usepackage{svg}
\usepackage{cancel}
\usepackage{indentfirst}
\setmainfont[ItalicFont = LiberationSans-Italic, BoldFont = LiberationSans-Bold, BoldItalicFont = LiberationSans-BoldItalic]{LiberationSans}
\newfontfamily\NHLight[ItalicFont = LiberationSansNarrow-Italic, BoldFont       = LiberationSansNarrow-Bold, BoldItalicFont = LiberationSansNarrow-BoldItalic]{LiberationSansNarrow}
\newcommand\textrmlf[1]{{\NHLight#1}}
\newcommand\textitlf[1]{{\NHLight\itshape#1}}
\let\textbflf\textrm
\newcommand\textulf[1]{{\NHLight\bfseries#1}}
\newcommand\textuitlf[1]{{\NHLight\bfseries\itshape#1}}
\usepackage{fancyhdr}
\pagestyle{fancy}
\usepackage{titlesec}
\usepackage{titling}
\makeatletter
\lhead{\textbf{\@title}}
\makeatother
\rhead{\textrmlf{Compiled} \today}
\lfoot{\theauthor\ \textbullet \ \textbf{2021-2022}}
\cfoot{}
\rfoot{\textrmlf{Page} \thepage}
\renewcommand{\tableofcontents}{}
\titleformat{\section} {\Large} {\textrmlf{\thesection} {|}} {0.3em} {\textbf}
\titleformat{\subsection} {\large} {\textrmlf{\thesubsection} {|}} {0.2em} {\textbf}
\titleformat{\subsubsection} {\large} {\textrmlf{\thesubsubsection} {|}} {0.1em} {\textbf}
\setlength{\parskip}{0.45em}
\renewcommand\maketitle{}
\author{Exr0n}
\date{\today}
\title{Fields and Voltage PhET Exploration}
\hypersetup{
 pdfauthor={Exr0n},
 pdftitle={Fields and Voltage PhET Exploration},
 pdfkeywords={},
 pdfsubject={},
 pdfcreator={Emacs 28.0.50 (Org mode 9.4.4)}, 
 pdflang={English}}
\begin{document}

\tableofcontents

\#ret

Run this PhET simulation (Links to an external site.). Using both an
Electric Field Sensor, and Voltage inspector (cross hairs), explore the
following topics. Submit your notes when you are finished. Be sure your
notes include both the questions and their answers!

\begin{enumerate}
\item Place a single charge in the working area. Using the E-field sensor
(with "values" selected), and the measuring tape, confirm that the
E-field calculated by the PhET simulation agrees with the equation we
have used in class. (Note, the units for E-field that we learned in
class were N/C. The PhET simulation may express the units
differently. But the numerical values should be the same.)

\begin{enumerate}
\item The numbers seem to check out, (voltage is \textasciitilde{}9 volts 1m away) since
the charges are in nanoCoulombs
\end{enumerate}

\item Place two positive charges in the working area. Where do you expect
the E field to be zero? Does the simulation confirm that?

\begin{enumerate}
\item Between the charges
\href{srcPhETChargesFieldsNeutralBetweenPositives.png.org}{srcPhETChargesFieldsNeutralBetweenPositives.png}
\end{enumerate}

\item Same as above, but use one positive and one negative charge.

\begin{enumerate}
\item E won't ever be zero, since the charges don't cancel each other
out. However, if you are far enough away the field becomes
negligible.
\href{srcPhETChargesFieldsNegligableField.png.org}{srcPhETChargesFieldsNegligableField.png}
\end{enumerate}

\item The E field at a given point can be thought of as the force that a +1
C charge would feel if it were placed there. What does "electric
potential" or "voltage" appear to represent? The units mentioned in
\#1 may be of interest as you consider this question.

\begin{enumerate}
\item Apparently \(\frac{N}{C}\) is equivalent to \(\frac{V}{m}\), then
the volt is \(\frac{Nm}{C}\) aka \(\frac{J}{C}\) (Joules per
Coulomb)
\item Then, "voltage is the difference in energy when you move a charge"
\end{enumerate}

\item Does electric potential appear to be a scalar or a vector?

\begin{enumerate}
\item Seems like a scalar, there's no arrow.
\end{enumerate}

\item What or where is the zero-point for electric potential?

\begin{enumerate}
\item On the line perpendicular to the segment between the charges
through the midpoint of the charges.
\end{enumerate}

\item What is the relationship between the local E-field vector and a line
of constant electric potential? (You can explore this first by moving
the voltage sensor (drag the little box, not the crosshairs) and
observing the voltage values, then by plotting lines of constant
potential).

\begin{enumerate}
\item Constant voltage:
\href{20phys201srcPhETChargesAndFieldsConstantVoltage.png.org}{20phys201srcPhETChargesAndFieldsConstantVoltage.png}
\item The lines seem to be perpendicular to the field vectors.
\end{enumerate}
\end{enumerate}

When you are finished, you can play electric field hockey (Links to an
external site.)for fun!

\noindent\rule{\textwidth}{0.5pt}
\end{document}
