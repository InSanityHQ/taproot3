% Created 2021-09-27 Mon 12:02
% Intended LaTeX compiler: xelatex
\documentclass[letterpaper]{article}
\usepackage{graphicx}
\usepackage{grffile}
\usepackage{longtable}
\usepackage{wrapfig}
\usepackage{rotating}
\usepackage[normalem]{ulem}
\usepackage{amsmath}
\usepackage{textcomp}
\usepackage{amssymb}
\usepackage{capt-of}
\usepackage{hyperref}
\setlength{\parindent}{0pt}
\usepackage[margin=1in]{geometry}
\usepackage{fontspec}
\usepackage{svg}
\usepackage{cancel}
\usepackage{indentfirst}
\setmainfont[ItalicFont = LiberationSans-Italic, BoldFont = LiberationSans-Bold, BoldItalicFont = LiberationSans-BoldItalic]{LiberationSans}
\newfontfamily\NHLight[ItalicFont = LiberationSansNarrow-Italic, BoldFont       = LiberationSansNarrow-Bold, BoldItalicFont = LiberationSansNarrow-BoldItalic]{LiberationSansNarrow}
\newcommand\textrmlf[1]{{\NHLight#1}}
\newcommand\textitlf[1]{{\NHLight\itshape#1}}
\let\textbflf\textrm
\newcommand\textulf[1]{{\NHLight\bfseries#1}}
\newcommand\textuitlf[1]{{\NHLight\bfseries\itshape#1}}
\usepackage{fancyhdr}
\pagestyle{fancy}
\usepackage{titlesec}
\usepackage{titling}
\makeatletter
\lhead{\textbf{\@title}}
\makeatother
\rhead{\textrmlf{Compiled} \today}
\lfoot{\theauthor\ \textbullet \ \textbf{2021-2022}}
\cfoot{}
\rfoot{\textrmlf{Page} \thepage}
\renewcommand{\tableofcontents}{}
\titleformat{\section} {\Large} {\textrmlf{\thesection} {|}} {0.3em} {\textbf}
\titleformat{\subsection} {\large} {\textrmlf{\thesubsection} {|}} {0.2em} {\textbf}
\titleformat{\subsubsection} {\large} {\textrmlf{\thesubsubsection} {|}} {0.1em} {\textbf}
\setlength{\parskip}{0.45em}
\renewcommand\maketitle{}
\author{Houjun Liu}
\date{\today}
\title{Digital Logic}
\hypersetup{
 pdfauthor={Houjun Liu},
 pdftitle={Digital Logic},
 pdfkeywords={},
 pdfsubject={},
 pdfcreator={Emacs 28.0.50 (Org mode 9.4.4)}, 
 pdflang={English}}
\begin{document}

\tableofcontents



\section{Digital Logic}
\label{sec:orga870edd}
\subsection{Binary}
\label{sec:org4d034a4}
1011010.0 => \(1*2^6 + 0*2^5 + 1*2^4 + 1*2^3 + 0*2^2 + 1*2^1 + 0*2^0\)

\begin{itemize}
\item In binary, 2 conditions could represent all numbers
\item Low Voltage => 0
\item High Voltage => 1
\end{itemize}

1011010 + 011101

Here's a truth table:

\begin{center}
\begin{tabular}{rrrrrrr}
Signal A & Signal B & A OR B & A AND B & A XOR B & A NOR B & A XNOR B\\
\hline
0 & 0 & 0 & 0 & 0 & 1 & 1\\
0 & 1 & 1 & 0 & 1 & 0 & 0\\
1 & 0 & 1 & 0 & 1 & 0 & 0\\
1 & 1 & 1 & 1 & 0 & 0 & 1\\
\end{tabular}
\end{center}

\subsection{Logic gates}
\label{sec:orgd220c4e}
\textbf{OR Gates}: a mystery?

\subsection{Binary Operations}
\label{sec:org3f3deb8}
\begin{itemize}
\item \textbf{A+B} => \{A XOR B => ones digit; A AND B => carry digit\}
\end{itemize}
\end{document}
