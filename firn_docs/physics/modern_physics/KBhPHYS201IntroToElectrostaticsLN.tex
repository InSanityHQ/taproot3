% Created 2021-09-27 Mon 12:02
% Intended LaTeX compiler: xelatex
\documentclass[letterpaper]{article}
\usepackage{graphicx}
\usepackage{grffile}
\usepackage{longtable}
\usepackage{wrapfig}
\usepackage{rotating}
\usepackage[normalem]{ulem}
\usepackage{amsmath}
\usepackage{textcomp}
\usepackage{amssymb}
\usepackage{capt-of}
\usepackage{hyperref}
\setlength{\parindent}{0pt}
\usepackage[margin=1in]{geometry}
\usepackage{fontspec}
\usepackage{svg}
\usepackage{cancel}
\usepackage{indentfirst}
\setmainfont[ItalicFont = LiberationSans-Italic, BoldFont = LiberationSans-Bold, BoldItalicFont = LiberationSans-BoldItalic]{LiberationSans}
\newfontfamily\NHLight[ItalicFont = LiberationSansNarrow-Italic, BoldFont       = LiberationSansNarrow-Bold, BoldItalicFont = LiberationSansNarrow-BoldItalic]{LiberationSansNarrow}
\newcommand\textrmlf[1]{{\NHLight#1}}
\newcommand\textitlf[1]{{\NHLight\itshape#1}}
\let\textbflf\textrm
\newcommand\textulf[1]{{\NHLight\bfseries#1}}
\newcommand\textuitlf[1]{{\NHLight\bfseries\itshape#1}}
\usepackage{fancyhdr}
\pagestyle{fancy}
\usepackage{titlesec}
\usepackage{titling}
\makeatletter
\lhead{\textbf{\@title}}
\makeatother
\rhead{\textrmlf{Compiled} \today}
\lfoot{\theauthor\ \textbullet \ \textbf{2021-2022}}
\cfoot{}
\rfoot{\textrmlf{Page} \thepage}
\renewcommand{\tableofcontents}{}
\titleformat{\section} {\Large} {\textrmlf{\thesection} {|}} {0.3em} {\textbf}
\titleformat{\subsection} {\large} {\textrmlf{\thesubsection} {|}} {0.2em} {\textbf}
\titleformat{\subsubsection} {\large} {\textrmlf{\thesubsubsection} {|}} {0.1em} {\textbf}
\setlength{\parskip}{0.45em}
\renewcommand\maketitle{}
\author{Houjun Liu}
\date{\today}
\title{Electrostatics Index}
\hypersetup{
 pdfauthor={Houjun Liu},
 pdftitle={Electrostatics Index},
 pdfkeywords={},
 pdfsubject={},
 pdfcreator={Emacs 28.0.50 (Org mode 9.4.4)}, 
 pdflang={English}}
\begin{document}

\tableofcontents

First, let's begin with\ldots{}

\section{Electrostatics Cheat Sheet}
\label{sec:org0aa7cb5}
\href{KB20200825215200.org}{KB20200825215200}

\section{An atom}
\label{sec:org053c38f}
We begin by recognizing the fact that \textbf{it's the electron that can move
around in an atom.}.

For now, materials could be either \textbf{Conductors} or \textbf{Insulators.}

\begin{itemize}
\item \textbf{Conductors}

\begin{itemize}
\item \(e^-\) move freely
\item Think! Metal
\end{itemize}

\item \textbf{Insulators}

\begin{itemize}
\item \(e^-\) cannot move freely
\item Think! Wood/Glass/Plastic
\end{itemize}
\end{itemize}

Objects have different charge properties
\href{KBhPHYS201AtomChargeProps.org}{KBhPHYS201AtomChargeProps}, and
they interact with each other in specific ways:

\begin{itemize}
\item \textbf{Like changes tend to repel}
\item \textbf{Different changes tend to attract}
\end{itemize}

\href{KBhPHYS201AtomInteractions.org}{KBhPHYS201AtomInteractions}

\subsubsection{The Rods and Paper Experiment}
\label{sec:org4d70645}
Recall the day one at-home experiment
\href{KBhPHYS201D1AtHomeActivity.org}{KBhPHYS201D1AtHomeActivity}.
Let's see how the interactions we saw relates to the physical world:

See
\href{KBhPHYS201ElectrostaticPolarization.org}{KBhPHYS201ElectrostaticPolarization},
the analysis of the Rods and Paper Experiment

\subsubsection{The Electroscope}
\label{sec:orgdeb863d}
See \href{KBhPHYS201Electroscope.org}{KBhPHYS201Electroscope}, the
electroscope.

\section{Quantifying electrical force!}
\label{sec:org01b9962}
See \href{KBhPHYS201ColoumbsLaw.org}{KBhPHYS201ColoumbsLaw}, Coulomb's
Law

\section{Gravity + Gravitational Fields!}
\label{sec:org7fa3a5f}
Each object has what's called \textbf{gravitational field.} Surrounding each
object has what is effectively many tiny vectors getting weaker and
weaker as you move away from the Earth. You could calculate the force of
gravity just by knowing\ldots{}

\begin{enumerate}
\item The mass of what you are calculating.
\item How far away is the other object's mass.
\end{enumerate}

Then, out pops a value that tells you the magnitude of force that an
object would exert on another object w.r.t. their mass that was dropped
right where that vector was.

To see how we could do this, and how it relates to electrostatics, see
\href{KBhPHYS201GravitationalFields.org}{KBhPHYS201GravitationalFields}
Newton's Law of Gravitation.

\section{Electric Fields}
\label{sec:orge8b3a6a}
See \href{KBhPHYS201ElectricFields.org}{KBhPHYS201ElectricFields}
Electric Fields.

\section{Applications of Electrostatics}
\label{sec:org0da1742}
\subsection{Van de Graff Generator}
\label{sec:org76ce0c4}
See \href{KBhPHYS201VanDeGraff.org}{KBhPHYS201VanDeGraff}

\subsection{Lazer Printers}
\label{sec:org41961a1}
See \href{KBhPHYS201LazerPrinters.org}{KBhPHYS201LazerPrinters}

\section{Resistance and Current}
\label{sec:org18f1f6d}
See
\href{KBhPHYS201ResistanceConductivity.org}{KBhPHYS201ResistanceConductivity}
\end{document}
