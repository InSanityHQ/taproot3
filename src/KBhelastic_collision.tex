% Created 2022-05-04 Wed 23:54
% Intended LaTeX compiler: xelatex
\documentclass[letterpaper]{article}
\usepackage{graphicx}
\usepackage{longtable}
\usepackage{wrapfig}
\usepackage{rotating}
\usepackage[normalem]{ulem}
\usepackage{amsmath}
\usepackage{amssymb}
\usepackage{capt-of}
\usepackage{hyperref}
\usepackage[margin=1in]{geometry}
\setlength{\parindent}{0pt}
\usepackage[margin=1in]{geometry}
\usepackage{fontspec}
\usepackage{svg}
\usepackage{tikz}
\usepackage{cancel}
\usepackage{pgfplots}
\usepackage{indentfirst}
\setmainfont[ItalicFont = HelveticaNeue-Italic, BoldFont = HelveticaNeue-Bold, BoldItalicFont = HelveticaNeue-BoldItalic]{HelveticaNeue}
\newfontfamily\NHLight[ItalicFont = HelveticaNeue-LightItalic, BoldFont       = HelveticaNeue-UltraLight, BoldItalicFont = HelveticaNeue-UltraLightItalic]{HelveticaNeue-Light}
\newcommand\textrmlf[1]{{\NHLight#1}}
\newcommand\textitlf[1]{{\NHLight\itshape#1}}
\let\textbflf\textrm
\newcommand\textulf[1]{{\NHLight\bfseries#1}}
\newcommand\textuitlf[1]{{\NHLight\bfseries\itshape#1}}
\usepackage{fancyhdr}
\usepackage{csquotes}
\pagestyle{fancy}
\usepackage{titlesec}
\usepackage{titling}
\makeatletter
\lhead{\textbf{\@title}}
\makeatother
\rhead{\textrmlf{Written} \today}
\lfoot{\theauthor\ \textbullet \ \textbf{2021-2022}}
\cfoot{}
\rfoot{\textrmlf{Page} \thepage}
\renewcommand{\tableofcontents}{}
\titleformat{\section} {\Large} {\textrmlf{\thesection} {|}} {0.3em} {\textbf}
\titleformat{\subsection} {\large} {\textrmlf{\thesubsection} {|}} {0.2em} {\textbf}
\titleformat{\subsubsection} {\large} {\textrmlf{\thesubsubsection} {|}} {0.1em} {\textbf}
\setlength{\parskip}{0.45em}
\renewcommand\maketitle{}
\author{Houjun Liu}
\date{\today}
\title{Elastic Collision}
\hypersetup{
 pdfauthor={Houjun Liu},
 pdftitle={Elastic Collision},
 pdfkeywords={},
 pdfsubject={},
 pdfcreator={Emacs 28.0.91 (Org mode 9.5.2)}, 
 pdflang={English}}
\begin{document}

\maketitle
\tableofcontents


\section{Elastic collision}
\label{sec:org0e469a7}
We are given that the object \(m_1\) collides with the rod with velocity \(v_0\), and the rod is floating in free space. Given \(m_1\), \(v_0\), \(m_2\), \(I_0\), and \(r\), we are to figure to the final velocity of \(m_1\) after collision \(v_f\), the velocity of \(m_2\) after collision \(v_{CM}\), and of course the rotation of the rod after collision \(\omega\).

We are assuming that this collision elastic.

We have, then, for conservation of linear momentum:

\begin{equation}
 m_1 v_0 = m_1v_f + m_2 v_{CM} 
\end{equation}

Furthermore, we understand that kinetic energy is also conserved here; therefore:

\begin{align}
&\frac{1}{2} m_1{v_0}^2 = \left(\frac{1}{2} m_1{v_f}^2\right)+\left(\frac{1}{2} m_2{v_{CM}}^2\right)+\left(\frac{1}{2} I_0{\omega}^2\right)\\
\Rightarrow & m_1{v_0}^2 = \left( m_1{v_f}^2\right)+\left( m_2{v_{CM}}^2\right)+\left( I_0{\omega}^2\right)
\end{align}

as the point mass does not have any rotational inertia, and the rod is not rotating at the start.

Lastly, we understand that the angular momentum is conserved through a collision; letting the origin as the center of mass of the rod:

\begin{align}
   &m_1 r^2 \left(\frac{v_0}{r}\right) = m_1 r^2 \left(\frac{v_f}{r}\right) + I_0 \omega\\
\Rightarrow &m_1 r v_0 = m_1 r v_f + I_0 \omega
\end{align}

We now have a system of three equations that can be combined to solve for three unknowns \(v_f\), \(v_{CM}\), and \(\omega\).

We will begin with the kinetic energy expression:

\begin{equation}
 m_1{v_0}^2 = \left( m_1{v_f}^2\right)+\left( m_2{v_{CM}}^2\right)+\left( I_0{\omega}^2\right)   
\end{equation}

Now, this last term seems a little concerning to deal with. Therefore, we will use the final expression to simplify it:

\begin{align}
   &m_1 rv_0 = m_1rv_f + I_0 \omega\\
\Rightarrow & m_1 rv_0 - m_1rv_f = I_0 \omega
\end{align}

and also:

\begin{equation}
 \frac{m_1 rv_0 - m_1rv_f}{I_0} = \omega   
\end{equation}

Therefore, \(I_0 \omega^2\) would be:

\begin{equation}
    \frac{(m_1 rv_0 - m_1rv_f)^2}{I_0}
\end{equation}

Supplying that back:

\begin{align}
 &m_1{v_0}^2 =  m_1{v_f}^2+m_2{v_{CM}}^2+\left( I_0{\omega}^2\right)\\
\Rightarrow & m_1{v_0}^2 = m_1{v_f}^2+ m_2{v_{CM}}^2+\frac{(m_1 rv_0 - m_1rv_f)^2}{I_0}\\
\Rightarrow & m_1{v_0}^2 = m_1{v_f}^2+ m_2{v_{CM}}^2+\frac{(r(m_1 v_0 - m_1v_f))^2}{I_0}\\
\Rightarrow & m_1{v_0}^2 = m_1{v_f}^2+ m_2{v_{CM}}^2+\frac{r^2(m_1 v_0 - m_1v_f)^2}{I_0}\\
\Rightarrow & m_1{v_0}^2 -m_1{v_f}^2 =  m_2{v_{CM}}^2+\frac{r^2(m_1 v_0 - m_1v_f)^2}{I_0}\\
\Rightarrow & m_1I_0{v_0}^2 -m_1I_0{v_f}^2 =  m_2I_0{v_{CM}}^2+r^2(m_1 v_0 - m_1v_f)^2
\end{align}

At this point, we realize that we haven't use the first expression yet. Therefore, we will endeavor to use it to rid of the one \(v_{CM}\) we have here.

\begin{align}
 &m_1 v_0 = m_1v_f + m_2 v_{CM} \\
\Rightarrow &m_1 v_0 -m_1v_f =  m_2 v_{CM} \\
\Rightarrow &\frac{m_1 v_0 -m_1v_f}{m_2} = v_{CM} 
\end{align}

Ok, so now, supplying all of this back:

\begin{align}
& m_1I_0{v_0}^2 -m_1I_0{v_f}^2 =  m_2I_0{v_{CM}}^2+r^2(m_1 v_0 - m_1v_f)^2\\
\Rightarrow & m_1I_0{v_0}^2 -m_1I_0{v_f}^2 =  m_2I_0{\left(\frac{m_1 v_0 -m_1v_f}{m_2}\right)}^2+r^2(m_1 v_0 - m_1v_f)^2\\
\Rightarrow & m_1I_0{v_0}^2 -m_1I_0{v_f}^2 =  m_2I_0{\frac{(m_1 v_0 -m_1v_f)^2}{{m_2}^2}}+r^2(m_1 v_0 - m_1v_f)^2\\
\Rightarrow & m_1m_2I_0{v_0}^2 -m_1m_2I_0{v_f}^2 =  m_2I_0(m_1 v_0 -m_1v_f)^2+m_2r^2(m_1 v_0 - m_1v_f)^2\\
\Rightarrow & m_1m_2I_0{v_0}^2 -m_1m_2I_0{v_f}^2 =  m_2I_0((m_1 v_0)^2 -2((m_1)^2v_fv_0) \\&+ (m_1 v_f)^2)+m_2r^2((m_1 v_0)^2 -2((m_1)^2v_fv_0) + (m_1 v_f)^2)\\
\Rightarrow & I_0{v_0}^2 -I_0{v_f}^2 =  m_1I_0({v_0}^2 -2v_fv_0 + {v_f}^2)+m_1r^2({v_0}^2 -2v_fv_0 + {v_f}^2)\\
\Rightarrow & I_0{v_0}^2 -I_0{v_f}^2 =  (m_1I_0{v_0}^2 -2m_1I_0v_fv_0 + m_1I_0{v_f}^2)+(m_1r^2{v_0}^2 -2m_1r^2v_fv_0 +m_1r^2 {v_f}^2)\\
\Rightarrow & -I_0{v_f}^2 +2m_1I_0v_fv_0 - m_1I_0{v_f}^2+2m_1r^2v_fv_0 -m_1r^2 {v_f}^2=  (m_1I_0{v_0}^2)+(m_1r^2{v_0}^2 )- I_0{v_0}^2 \\
\Rightarrow & -(I_0+m_1I_0+m_1r^2){v_f}^2 +2m_1v_0(I_0+r^2)v_f =  (m_1I_0{v_0}^2)+(m_1r^2{v_0}^2 )- I_0{v_0}^2 \\
\Rightarrow & -(I_0+m_1I_0+m_1r^2){v_f}^2 +2m_1v_0(I_0+r^2)v_f =  (m_1I_0+m_1r^2-I_0){v_0}^2 \\
\Rightarrow & -(I_0+m_1I_0+m_1r^2){v_f}^2 +2m_1v_0(I_0+r^2)v_f-(m_1I_0+m_1r^2-I_0){v_0}^2 = 0  
\end{align}

There are probably more elegant ways to do this, but now, its quadratic equation time!

\begin{equation}
   x = \frac{-b\pm\sqrt{b^2-4ac}}{2a} 
\end{equation}

\begin{equation}
   \begin{cases}
   a = -(I_0+m_1I_0+m_1r^2)\\ 
   b = 2m_1v_0(I_0+r^2)\\ 
   c = -(m_1I_0+m_1r^2-I_0){v_0}^2\\ 
   x = v_f\\ 
\end{cases}
\end{equation}

Let's tackle that middle term first.

\begin{align}
   b^2 &=  (2m_1v_0(I_0+r^2))^2\\
&= 4{m_1}^2{v_0}^2(I_0+r^2)^2\\
&= 4{m_1}^2{v_0}^2({I_0}^2+2I_0r^2+r^4)\\
&= (4{m_1}^2{v_0}^2{I_0}^2+8{m_1}^2{v_0}^2I_0r^2+4{m_1}^2{v_0}^2r^4)
\end{align}

and,

\begin{align}
   4ac &=  4(I_0+m_1I_0+m_1r^2)(m_1I_0+m_1r^2-I_0){v_0}^2\\
&= 4{v_0}^2((m_1{I_0}^2+m_1I_0r^2-{I_0}^2)+((m_1I_0)^2+{m_1}^2I_0r^2-m_1{I_0}^2)+({m_1}^2r^2I_0+(m_1r^2)^2-m_1I_0r^2))\\
&= 4{m_1}^2{v_0}^2{I_0}^2+8{m_1}^2{v_0}^2I_0r^2+4{m_1}^2{v_0}^2r^4 - 4{I_0}^2{v_0}^2
\end{align}

Therefore, subtracting the two terms, we will note that they are only different by:

\begin{equation}
   b^2-4ac = 4{I_0}^2{v_0}^2
\end{equation}

Finally, then; performing the square root:

\begin{equation}
    \sqrt{b^2-4ac}  = 2{I_0}{v_0}
\end{equation}

Subtracting negative \(b\) to this expression:

\begin{align}
   &-2m_1v_0(I_0+r^2) - 2{I_0}{v_0}\\
\Rightarrow &-2m_1v_0I_0- 2{I_0}{v_0}-2m_1v_0r^2 
\end{align}

And finally, dividing by \(2a\):

\begin{align}
   \frac{-2m_1v_0I_0- 2{I_0}{v_0}-2m_1v_0r^2}{-2(I_0+m_1I_0+m_1r^2)} &=\frac{m_1v_0I_0+{I_0}{v_0}+m_1v_0r^2}{(I_0+m_1I_0+m_1r^2)}\\
&= v_0
\end{align}

Ok, I suppose this is a very valid solution, but not a very exciting one ("nothing happened"). Let's rewind and add negative \(b\) instead.

Subtracting negative \(b\) to this expression:

\begin{align}
   &-2m_1v_0(I_0+r^2) + 2{I_0}{v_0}\\
\Rightarrow &-2m_1v_0I_0- 2{I_0}{v_0}+2m_1v_0r^2 
\end{align}

And finally, dividing by \(2a\):

\begin{align}
   \frac{-2m_1v_0I_0- 2{I_0}{v_0}+2m_1v_0r^2}{-2(I_0+m_1I_0+m_1r^2)} &=\frac{m_1v_0I_0+{I_0}{v_0}-m_1v_0r^2}{(I_0+m_1I_0+m_1r^2)}\\
&= \frac{(m_1I_0+{I_0}-m_1r^2)v_0}{(I_0+m_1I_0+m_1r^2)}\ \blacksquare
\end{align}

Phew. Now, we simply have to leverage the top expression to figure the rest of the expressions.

We have that:

\begin{align}
\frac{m_1 v_0 -m_1v_f}{m_2} = v_{CM}
\end{align}

from the previous figuring of \(v_f\), therefore:

\begin{align}
&\frac{m_1 v_0 -m_1v_f}{m_2} = v_{CM}\\
\Rightarrow &\frac{m_1 v_0 -m_1\frac{(m_1I_0+{I_0}-m_1r^2)v_0}{(I_0+m_1I_0+m_1r^2)}}{m_2} = v_{CM} \\
\Rightarrow &m_1 v_0 -m_1\frac{(m_1I_0+{I_0}-m_1r^2)v_0}{(I_0+m_1I_0+m_1r^2)} = m_2v_{CM} \\
\Rightarrow &\frac{m_1 v_0(I_0+m_1I_0+m_1r^2)}{(I_0+m_1I_0+m_1r^2)} -\frac{m_1v_0(m_1I_0+{I_0}-m_1r^2)}{(I_0+m_1I_0+m_1r^2)} = m_2v_{CM} \\
\Rightarrow &\frac{2m_1 v_0m_1r^2}{(I_0+m_1I_0+m_1r^2)}  = m_2v_{CM} \\
\Rightarrow &v_{CM} = \frac{2m_1 v_0m_1r^2}{m_2(I_0+m_1I_0+m_1r^2)}\ \blacksquare
\end{align}

We finally have, to figure \(\omega\):

\begin{equation}
m_1 rv_0 - m_1rv_f = I_0 \omega
\end{equation}

Therefore:

\begin{align}
 &m_1 rv_0 - m_1rv_f = I_0 \omega   \\
\Rightarrow &\frac{r(m_1 v_0 - m_1v_f)}{I_0} = \omega\\
\Rightarrow &(m_1 v_0 - m_1v_f) = \frac{I_0}{r}\omega
\end{align}

From halfway through the expression block before, we have figured this result, that:

\begin{equation}
 (m_1 v_0 - m_1v_f) = \frac{2m_1 v_0m_1r^2}{(I_0+m_1I_0+m_1r^2)}
\end{equation}

Therefore, we have finally that:

\begin{equation}
    \omega = \frac{2m_1 v_0m_1r}{(I_0+m_1I_0+m_1r^2)I_0}\ \blacksquare
\end{equation}

And we are done.

\section{Rigid Body Kinetic Energy}
\label{sec:org988cefc}
We will start with the known expression that:

\begin{equation}
   KE = \sum_i \frac{1}{2} m_i{v_i}^2
\end{equation}

Because of the fact a point \(v_i\) can be defined as a sum of the velocity from the origin plus the displace from from origin (\(v_i = v_{CM}+v'_i\)), we can rewrite the kinetic energy expression:

\begin{equation}
   KE = \sum_i  \frac{1}{2}  m_i (V_{CM}+v'_i)(V_{CM}+v'_i)
\end{equation}

Now, we shall foil the above expression:

\begin{align}
   KE &= \sum_i  \frac{1}{2}  m_i ({V_{CM}}^2+2v_{CM}v'_i+{v'_i}^2) \\
&= \sum_i  \frac{1}{2}  m_i {V_{CM}}^2+ \sum_i  m_i V_{CM}v'_i+\sum_i  \frac{1}{2}  m_i {v'_i}^2 \\
&= \frac{1}{2}  M {V_{CM}}^2+ \sum_i  m_i V_{CM}v'_i+\sum_i  \frac{1}{2} m_i {v'_i}^2 \\
&= \frac{1}{2}  M {V_{CM}}^2+ V_{CM}\sum_i  m_i v'_i+\sum_i  \frac{1}{2} m_i {v'_i}^2 
\end{align}

At which point, we realize that we have in the middle arrived at the definition of the center of mass in the reference frame of the center of mass---meaning that it is indeed \(0\) because the center of mass is at the origin of the center of mass. Moving on, then:

\begin{align}
   KE &= \frac{1}{2}  M {V_{CM}}^2+ V_{CM}\sum_i  m_i v'_i+\sum_i  \frac{1}{2} m_i {v'_i}^2 \\
   &= \frac{1}{2}  M {V_{CM}}^2+\frac{1}{2} \sum_i m_i {v'_i}^2 
\end{align}

We now note that this is a rigid body. Therefore, each point mass will not "slip" from the center of mass---they remain the same distance relative each other. Therefore, we can leverage the no slipping assumption to claim that \(v'_i = r'_i \omega\).

\begin{align}
   KE &= \frac{1}{2}  M {V_{CM}}^2+\frac{1}{2} \sum_i m_i ( r'_i \omega )^2 \\ 
   &= \frac{1}{2}  M {V_{CM}}^2+\frac{1}{2} \sum_i m_i r'_i^2 \omega^2  \\
   &= \frac{1}{2}  M {V_{CM}}^2+\frac{1}{2} \omega^2 \sum_i m_i r'_i^2   \\
   &= \frac{1}{2}  M {V_{CM}}^2+\frac{1}{2} I \omega^2\ \blacksquare
\end{align}
\end{document}