% Created 2022-05-09 Mon 22:29
% Intended LaTeX compiler: xelatex
\documentclass[letterpaper]{article}
\usepackage{graphicx}
\usepackage{longtable}
\usepackage{wrapfig}
\usepackage{rotating}
\usepackage[normalem]{ulem}
\usepackage{amsmath}
\usepackage{amssymb}
\usepackage{capt-of}
\usepackage{hyperref}
\usepackage[margin=1in]{geometry}
\setlength{\parindent}{0pt}
\usepackage[margin=1in]{geometry}
\usepackage{fontspec}
\usepackage{svg}
\usepackage{tikz}
\usepackage{cancel}
\usepackage{pgfplots}
\usepackage{indentfirst}
\setmainfont[ItalicFont = HelveticaNeue-Italic, BoldFont = HelveticaNeue-Bold, BoldItalicFont = HelveticaNeue-BoldItalic]{HelveticaNeue}
\newfontfamily\NHLight[ItalicFont = HelveticaNeue-LightItalic, BoldFont       = HelveticaNeue-UltraLight, BoldItalicFont = HelveticaNeue-UltraLightItalic]{HelveticaNeue-Light}
\newcommand\textrmlf[1]{{\NHLight#1}}
\newcommand\textitlf[1]{{\NHLight\itshape#1}}
\let\textbflf\textrm
\newcommand\textulf[1]{{\NHLight\bfseries#1}}
\newcommand\textuitlf[1]{{\NHLight\bfseries\itshape#1}}
\usepackage{fancyhdr}
\usepackage{csquotes}
\pagestyle{fancy}
\usepackage{titlesec}
\usepackage{titling}
\makeatletter
\lhead{\textbf{\@title}}
\makeatother
\rhead{\textrmlf{Written} \today}
\lfoot{\theauthor\ \textbullet \ \textbf{2021-2022}}
\cfoot{}
\rfoot{\textrmlf{Page} \thepage}
\renewcommand{\tableofcontents}{}
\titleformat{\section} {\Large} {\textrmlf{\thesection} {|}} {0.3em} {\textbf}
\titleformat{\subsection} {\large} {\textrmlf{\thesubsection} {|}} {0.2em} {\textbf}
\titleformat{\subsubsection} {\large} {\textrmlf{\thesubsubsection} {|}} {0.1em} {\textbf}
\setlength{\parskip}{0.45em}
\renewcommand\maketitle{}
\author{Houjun Liu}
\date{\today}
\title{MVC 2 PS\#28}
\hypersetup{
 pdfauthor={Houjun Liu},
 pdftitle={MVC 2 PS\#28},
 pdfkeywords={},
 pdfsubject={},
 pdfcreator={Emacs 28.0.91 (Org mode 9.5.2)}, 
 pdflang={English}}
\begin{document}

\maketitle
\tableofcontents


\section{Evaluating a Cylindrical Integral}
\label{sec:org11a4d8c}
\begin{quote}
Considering the function: 

\begin{equation}
   f(x,y,z) = \sqrt{x^2+y^2} 
\end{equation}
\end{quote}

To evaluate the integral, we will convert it to cylindrical coordinates. We note first that the integral is to be evaluated inside the cylinder of \(x^2+y^2 = 16\), which means that we wish to evaluate it in a circle with center at the origin with radius \(4\).

Furthermore, we understand that the bounds of the function are to be evaluated between \([-5, -4]\).

If we set up the integral, we will get:

\begin{equation}
   \int_{-5}^{-4} \int_C\ \sqrt{x^2+y^2}\ dx\ dy\ dz 
\end{equation}

This is convenient. We can evaluate the inner integral first like in \(\mathbb{R}^2\to\mathbb{R}^1\), then simply evaluate the other integral after.

Let's do so.

Note that the inner integral is a normal cylindrical coordinate setup. Therefore, we can take the following substitution:

\begin{equation}
   \sqrt{x^2+y^2} = r 
\end{equation}

Furthermore, that:

\begin{equation}
   dx\ dy = dr\ d\theta 
\end{equation}

With the appropriate bounds, then:

\begin{align}
   &\int_0^{2\pi} \int_0^4 r\ dr\ d\theta\\
\Rightarrow &\int_0^{2\pi} \left \frac{r^2}{2}\right|_0^4 d\theta\\
\Rightarrow &\int_0^{2\pi} 8\ d\theta\\
\Rightarrow &16\pi
\end{align}

Finally, we will take the integral of this value \(dz\):

\begin{equation}
   \int_{-5}^{-4} 16\pi\ dz  = 16\pi
\end{equation}

Therefore, the value of the integral is \(16\pi\).

\section{Uselessly Spherical Integral}
\label{sec:org79e92b8}
We first recall that the differential volume can be written as:

\begin{equation}
   dV = \rho^2 \sin \phi\ d\rho\ d\phi\ d\theta
\end{equation}

To take this integral, then, we have to figure the distance \(\rho\) to a rectangle for every point \((\phi, \theta)\).

We will take this integral w.r.t. one quadrant of the four parts of the rectangle, then we will scale it up by \(4\). Each of the quadrants are divided into again two equally-sized pieces: one sweeping from the baseline until halfway up, the other sweeping halfway up until the center line. These are the areas where either the adjacent side is consistently \(\frac{b}{2}\) or the opposite side is consistently \(\frac{a}{2}\).

The switch happens exactly at the triangle where the bottom side has length \(\frac{b}{2}\) and the right side has length \(\frac{a}{2}\). This triangle, per the definition of arctan, has an angle of:

\begin{equation}
   \theta = \tan^{-1}\left(\frac{a}{b}\right) 
\end{equation}

Therefore, from \(\theta = [0, \tan^{-1}\left(\frac{a}{b}\right)]\), we have sidelength \(\frac{b}{2\cos\theta}\). From \(\theta = [\tan^{-1}\left(\frac{a}{b}\right), \frac{\pi}{2}]\), we have sidelength \(\frac{a}{2\sin\theta}\).

By the same token, for the vertical declination, from \(\theta = [0, \tan^{-1}\left(\frac{a}{b}\right)]\), \(\phi = [0,\tan^{-1}\left(\frac{c}{b}\cos\theta\right)]\), we have \(\rho=\frac{b}{2\cos\theta\cos\phi}\). From \(\theta = [\tan^{-1}\left(\frac{a}{b}\right), \frac{\pi}{2}]\), \(\phi = [0,\tan^{-1}\left(\frac{c}{a}\sin\theta\right)]\), we have \(\rho=\frac{a}{2\sin\theta\cos\phi}\). For all other \(\phi\), \(\phi = [\tan^{-1}\left(\frac{c}{b}\cos\theta\right), \frac{\pi}{2}]\), we have the \(\frac{c}{2}\) term taking precedence and hence \(\rho = \frac{c}{2\sin\phi}\). 

Therefore, we will have to take the quarter-integral in three parts:

\begin{align}
&\int_0^{\tan^{-1}\left(\frac{a}{b}\right)} \int_0^{\tan^{-1}\left(\frac{c}{b}\cos\theta\right)} \frac{b}{2\cos\theta\cos\phi}\ d\phi\ d\theta\\
&\int_0^{\tan^{-1}\left(\frac{a}{b}\right)}\frac{b}{2\cos\theta} \int_0^{\tan^{-1}\left(\frac{c}{b}\cos\theta\right)} \frac{1}{\cos\phi}\ d\phi\ d\theta\\
&\int_0^{\tan^{-1}\left(\frac{a}{b}\right)}\frac{b}{2\cos\theta} \left \frac{\sin \phi}{\cos^2\phi} \right|_0^{\tan^{-1}\left(\frac{c}{b}\cos\theta\right)}\ d\theta\\
&\int_0^{\tan^{-1}\left(\frac{a}{b}\right)} \frac{1}{2} \, c \sqrt{\frac{c^{2} \cos\left(\theta\right)^{2} + b^{2}}{b^{2}}} \cos\left(\theta\right)^{2} \ d\theta\\
&\frac{1}{2b} \int_0^{\tan^{-1}\left(\frac{a}{b}\right)} \cos\left(\theta\right)^{2}\sqrt{c^{2} \cos\left(\theta\right)^{2} + b^{2}}  \ d\theta
\end{align}

At this point, I am not quite sure where next to go in solving this problem. I am currently conferring with my classmates for a solution---the current problem is that the \(\cos\theta\) term is being added to an inseparable constant \(b^2\), which renders the square root not very simplifiable. A trig sub here doesn't work super well, as there is an extra \(b^2\) term that's not here simplified.  
\end{document}