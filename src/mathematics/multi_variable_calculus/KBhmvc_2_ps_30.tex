% Created 2022-05-26 Thu 23:52
% Intended LaTeX compiler: xelatex
\documentclass[letterpaper]{article}
\usepackage{graphicx}
\usepackage{longtable}
\usepackage{wrapfig}
\usepackage{rotating}
\usepackage[normalem]{ulem}
\usepackage{amsmath}
\usepackage{amssymb}
\usepackage{capt-of}
\usepackage{hyperref}
\usepackage[margin=1in]{geometry}
\setlength{\parindent}{0pt}
\usepackage[margin=1in]{geometry}
\usepackage{fontspec}
\usepackage{svg}
\usepackage{tikz}
\usepackage{cancel}
\usepackage{pgfplots}
\usepackage{indentfirst}
\setmainfont[ItalicFont = HelveticaNeue-Italic, BoldFont = HelveticaNeue-Bold, BoldItalicFont = HelveticaNeue-BoldItalic]{HelveticaNeue}
\newfontfamily\NHLight[ItalicFont = HelveticaNeue-LightItalic, BoldFont       = HelveticaNeue-UltraLight, BoldItalicFont = HelveticaNeue-UltraLightItalic]{HelveticaNeue-Light}
\newcommand\textrmlf[1]{{\NHLight#1}}
\newcommand\textitlf[1]{{\NHLight\itshape#1}}
\let\textbflf\textrm
\newcommand\textulf[1]{{\NHLight\bfseries#1}}
\newcommand\textuitlf[1]{{\NHLight\bfseries\itshape#1}}
\usepackage{fancyhdr}
\usepackage{csquotes}
\pagestyle{fancy}
\usepackage{titlesec}
\usepackage{titling}
\makeatletter
\lhead{\textbf{\@title}}
\makeatother
\rhead{\textrmlf{Written} \today}
\lfoot{\theauthor\ \textbullet \ \textbf{2021-2022}}
\cfoot{}
\rfoot{\textrmlf{Page} \thepage}
\renewcommand{\tableofcontents}{}
\titleformat{\section} {\Large} {\textrmlf{\thesection} {|}} {0.3em} {\textbf}
\titleformat{\subsection} {\large} {\textrmlf{\thesubsection} {|}} {0.2em} {\textbf}
\titleformat{\subsubsection} {\large} {\textrmlf{\thesubsubsection} {|}} {0.1em} {\textbf}
\setlength{\parskip}{0.45em}
\renewcommand\maketitle{}
\author{Houjun Liu}
\date{\today}
\title{MVC 2 PS\#30}
\hypersetup{
 pdfauthor={Houjun Liu},
 pdftitle={MVC 2 PS\#30},
 pdfkeywords={},
 pdfsubject={},
 pdfcreator={Emacs 28.0.91 (Org mode 9.5.2)}, 
 pdflang={English}}
\begin{document}

\maketitle
\tableofcontents


\section{Jacobian Determinant for Polar}
\label{sec:org9d2dab1}
We are to determine (pun not intended) the polar correction factor for a double integral, \(dA= r\ dr\ d\theta\).

To do this, we will have to first figure the change of bases expressions such that we can take:

\begin{equation}
   f(x,y) = g(r, \theta) 
\end{equation}

Fortunately, this is already derived to use from before.

\begin{equation}
   \begin{cases}
   x = r\cos\theta \\
   y = r\sin\theta \\
\end{cases}
\end{equation}

Therefore, we have that:

\begin{equation}
   f(x,y) = f(r\cos\theta, r\sin\theta) 
\end{equation}

And therefore, we can figure \(J_{r,\theta}\):

\begin{equation}
   J = \begin{bmatrix} 
cos\theta & -r\sin\theta \\
sin\theta & r\cos\theta \\
\end{bmatrix} 
\end{equation}

Taking its determinant, then:

\begin{equation}
   det(J) = r\cos^2\theta +r\sin^2\theta = r
\end{equation}

And therefore, the change-of-basis result would be:

\begin{equation}
   dx\ dy = r\ dr\ d\theta 
\end{equation}

\section{Jacobian Determinant for Spherical}
\label{sec:orgdd1c57f}
We again need to figure a correction factor for \(dx\ dy\ dz = \rho^2\ \sin\phi\ d\rho\ d\theta\ d\phi\).

We therefore have to figure a change of bases for the expression:

\begin{equation}
   f(x,y,z) = g(\rho, \theta, \phi) 
\end{equation}

We can leverage the shape of the object to determine the parameterization:

\begin{equation}
   \begin{cases}
   x = \rho\sin\phi\cos\theta \\
   y = \rho\sin\phi\sin\theta \\
   z = \rho\cos\phi \\
\end{cases}
\end{equation}

We will now figure the matrix for \(J_{\rho, \theta, \phi}\):

\begin{equation}
   J = \begin{bmatrix} 
sin\phi\cos\theta & -\rho\ sin\phi\sin\theta & \rho\ cos\phi\cos\theta \\
sin\phi\sin\theta & \rho\ sin\phi\cos\theta & \rho\ cos\phi\sin\theta \\
cos\phi & 0 & -\rho \sin \phi\\
\end{bmatrix} 
\end{equation}

\begin{verbatim}
var("rho phi theta")
M = matrix([[sin(phi)*cos(theta), -rho*sin(phi)*sin(theta), rho*cos(phi)*cos(theta)], [sin(phi)*sin(theta), rho*sin(phi)*cos(theta), rho*cos(phi)*sin(theta)], [cos(phi), 0, -rho*sin(phi)]])
M
M.det().full_simplify()
\end{verbatim}

\begin{verbatim}
(rho, phi, theta)
[     cos(theta)*sin(phi) -rho*sin(phi)*sin(theta)  rho*cos(phi)*cos(theta)]
[     sin(phi)*sin(theta)  rho*cos(theta)*sin(phi)  rho*cos(phi)*sin(theta)]
[                cos(phi)                        0            -rho*sin(phi)]
-rho^2*sin(phi)
\end{verbatim}


Not quite sure why Sage didn't simply \((-\rho)^2\) into \(\rho^2\), but, we can see that:

\begin{equation}
   dx\ dy\ dz = \rho^2\sin\phi\ d\rho\ d \theta\ d\phi 
\end{equation}

\section{Surface area in polar}
\label{sec:orgdcd8111}
Given the fact that:

\begin{equation}
  dA = \sqrt{1 + \left(\frac{\partial f}{\partial x}\right)^2 + \left(\frac{\partial f}{\partial y}\right)^2}\ dx\ dy
\end{equation}

We are to figure the corresponding for a function in polar.

I suppose we can work this out as if we are doing traditional u-substitution: that is, we are to find a function that corrects for the correction factor as well as the \(dx\) and \(dy\) components.

Recall again, that:

\begin{equation}
   \begin{cases}
   x = r\cos\theta \\
   y = r\sin\theta \\
\end{cases}
\end{equation}

Furthermore: per the chain and total derivative rule, we have that:

\begin{equation}
   \frac{\partial f}{\partial x} = \frac{\partial f}{\partial \theta}\cdot \frac{\partial \theta}{\partial x} + \frac{\partial f}{\partial r}\cdot \frac{\partial r}{\partial x}
\end{equation}

and,

\begin{equation}
   \frac{\partial f}{\partial y} = \frac{\partial f}{\partial \theta}\cdot \frac{\partial \theta}{\partial y} + \frac{\partial f}{\partial r}\cdot \frac{\partial r}{\partial y}
\end{equation}

To actually figure this, then, we have to find expressions for each of the rightward partials.

Take, first, \(\frac{\partial \theta}{\partial x}\); we have:

\begin{align}
   &x = r\ cos\theta \\
\Rightarrow &\frac{\partial}{\partial x} x = \frac{\partial}{\partial x}r\ cos\theta \\
\Rightarrow &1 = -r\ sin\theta \frac{\partial \theta}{\partial x}\\
\Rightarrow &\frac{\partial \theta}{\partial x} = \frac{-1}{r\ sin\theta} 
\end{align}

Furthermore, for \(\frac{\partial r}{\partial x}\)

We have, trivially:

\begin{equation}
\frac{\partial r}{\partial x} = \frac{1}{cos\theta}
\end{equation}

Therefore:

\begin{equation}
   \frac{\partial f}{\partial x} = -\frac{\partial f}{\partial \theta}\cdot \frac{1}{r\sin\theta} + \frac{\partial f}{\partial r}\cdot \frac{1}{\cos\theta}
\end{equation}

We can repeat this for \(y\):

For \(\frac{\partial \theta}{\partial y}\); we have:

\begin{align}
   &y = r\ sin\theta \\
\Rightarrow &\frac{\partial}{\partial x} x = \frac{\partial}{\partial x}r\ cos\theta \\
\Rightarrow &1 = r\ cos\theta \frac{\partial \theta}{\partial x}\\
\Rightarrow &\frac{\partial \theta}{\partial x} = \frac{1}{r\ cos\theta} 
\end{align}

Furthermore, for \(\frac{\partial r}{\partial y}\)

We have, trivially:

\begin{equation}
\frac{\partial r}{\partial x} = \frac{1}{sin\theta}
\end{equation}

Therefore:

\begin{equation}
   \frac{\partial f}{\partial y} = \frac{\partial f}{\partial \theta}\cdot \frac{1}{r\cos\theta} + \frac{\partial f}{\partial r}\cdot \frac{1}{\sin\theta}
\end{equation}

Lastly, we recall that \(dx\ dy = r\ dr\ d\theta\)

Finally, putting it all together:

\begin{align}
  dA &= \sqrt{1 + \left(\frac{\partial f}{\partial x}\right)^2 + \left(\frac{\partial f}{\partial y}\right)^2}\ dx\ dy\\
&= \sqrt{1 + \left(\frac{\partial f}{\partial r}\cdot \frac{1}{\cos\theta}-\frac{\partial f}{\partial \theta}\right)^2 + \left(\frac{\partial f}{\partial \theta}\cdot \frac{1}{r\cos\theta} + \frac{\partial f}{\partial r}\cdot \frac{1}{\sin\theta}\right)^2}\ dx\ dy\\
\end{align}
\end{document}