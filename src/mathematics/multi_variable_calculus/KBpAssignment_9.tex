% Created 2021-10-20 Wed 00:07
% Intended LaTeX compiler: pdflatex
\documentclass[11pt]{article}
\usepackage[utf8]{inputenc}
\usepackage[T1]{fontenc}
\usepackage{graphicx}
\usepackage{grffile}
\usepackage{longtable}
\usepackage{wrapfig}
\usepackage{rotating}
\usepackage[normalem]{ulem}
\usepackage{amsmath}
\usepackage{textcomp}
\usepackage{amssymb}
\usepackage{capt-of}
\usepackage{hyperref}
\author{Peter Choi}
\date{\today}
\title{Assignment 9}
\hypersetup{
 pdfauthor={Peter Choi},
 pdftitle={Assignment 9},
 pdfkeywords={},
 pdfsubject={},
 pdfcreator={Emacs 27.2 (Org mode 9.4.4)}, 
 pdflang={English}}
\begin{document}

\maketitle
\tableofcontents


\section{Problem 1}
\label{sec:org5dc98c4}
\subsection{a)}
\label{sec:orge6e4a91}
Inorder for the boat to be going in the right direction we know that \(\vec C+\vec S=\alpha\vec D\), where \(\vec C\) is the current of the river, \(\vec S\) is the speed of the boat, \(\alpha\) is some scalar and \(\vec D\) is the vector that goes from the boatman's starting point to their desired endpoint.

We can set the boatman's start point as \((0,0)\), and thus \(\vec D=\langle3,2\rangle\). We also know that \(\vec C= \langle0,-3.5\rangle\). Lastly, \(\vec S=\langle13\sin(\theta),13\cos(\theta)\rangle\), where \(\theta\) is the angle between the side of the river and \(\vec S\).

We can then plug in these values into the equation written above:

\(\vec C + \vec S = \alpha\vec D\newline\Rightarrow \langle 0,-3.5\rangle+\langle13\sin(\theta),13\cos(\theta)\rangle=\alpha\langle3,2\rangle\newline\Rightarrow\langle13\sin(\theta),-3.5+13\cos(\theta)\rangle=\langle\alpha3,\alpha2\rangle\newline\Rightarrow13\sin(\theta)=\alpha3,-3.5+13\cos(\theta)=\alpha2\newline\Rightarrow6\alpha=26\sin(\theta),6\alpha=-10.5+39\cos(\theta)\newline\Rightarrow26\sin(\theta)=-10.5+39\cos(\theta)\)

plug it into wolfram alpha:

\(\theta \approx0.75686\) radians or \(\approx 43.36^{\circ}\)
\subsection{b)}
\label{sec:org8913fe4}
The net velocity of the boat is \(\vec S+\vec C=\langle13\sin(\theta),13\cos(\theta)-3.5\rangle\), where \(\theta\) is the answer to part a. To get the speed of the boat we find the magnitude of this vector:

\(|\vec S+\vec C|=\sqrt{(13\sin(\theta))^2+(13\cos(\theta)-3.5)}\approx10.7282\) km/h

Now we need to find the distance traveled by the boat, which should be the magnitude of \(\vec D\):

\(|\vec D|=\sqrt{2^2+3^2}=\sqrt{4+9}=\sqrt{13}\approx 3.60555\) km

To get the time it took to take the trip we divide the distace by the speed:

\({3.60555\over10.7282}=0.336\) hours, which is \(20.2\) minutes

\section{Problem 2}
\label{sec:orgaff522f}
\end{document}