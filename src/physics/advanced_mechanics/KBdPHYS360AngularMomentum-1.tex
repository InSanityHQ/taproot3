% Created 2022-06-19 Sun 13:43
% Intended LaTeX compiler: xelatex
\documentclass[letterpaper]{article}
\usepackage{graphicx}
\usepackage{grffile}
\usepackage{longtable}
\usepackage{wrapfig}
\usepackage{rotating}
\usepackage[normalem]{ulem}
\usepackage{amsmath}
\usepackage{textcomp}
\usepackage{amssymb}
\usepackage{capt-of}
\usepackage{hyperref}
\usepackage[margin=1in]{geometry}
\setlength{\parindent}{0pt}
\usepackage[margin=1in]{geometry}
\usepackage{fontspec}
\usepackage{svg}
\usepackage{tikz}
\usepackage{cancel}
\usepackage{pgfplots}
\usepackage{indentfirst}
\setmainfont[ItalicFont = HelveticaNeue-Italic, BoldFont = HelveticaNeue-Bold, BoldItalicFont = HelveticaNeue-BoldItalic]{HelveticaNeue}
\newfontfamily\NHLight[ItalicFont = HelveticaNeue-LightItalic, BoldFont       = HelveticaNeue-UltraLight, BoldItalicFont = HelveticaNeue-UltraLightItalic]{HelveticaNeue-Light}
\newcommand\textrmlf[1]{{\NHLight#1}}
\newcommand\textitlf[1]{{\NHLight\itshape#1}}
\let\textbflf\textrm
\newcommand\textulf[1]{{\NHLight\bfseries#1}}
\newcommand\textuitlf[1]{{\NHLight\bfseries\itshape#1}}
\usepackage{fancyhdr}
\usepackage{csquotes}
\pagestyle{fancy}
\usepackage{titlesec}
\usepackage{titling}
\makeatletter
\lhead{\textbf{\@title}}
\makeatother
\rhead{\textrmlf{Written} \today}
\lfoot{\theauthor\ \textbullet \ \textbf{2021-2022}}
\cfoot{}
\rfoot{\textrmlf{Page} \thepage}
\renewcommand{\tableofcontents}{}
\titleformat{\section} {\Large} {\textrmlf{\thesection} {|}} {0.3em} {\textbf}
\titleformat{\subsection} {\large} {\textrmlf{\thesubsection} {|}} {0.2em} {\textbf}
\titleformat{\subsubsection} {\large} {\textrmlf{\thesubsubsection} {|}} {0.1em} {\textbf}
\setlength{\parskip}{0.45em}
\renewcommand\maketitle{}
\author{Dylan Wallace}
\date{\today}
\title{Torque and Angular Momentum 1 - Revised}
\hypersetup{
 pdfauthor={Dylan Wallace},
 pdftitle={Torque and Angular Momentum 1 - Revised},
 pdfkeywords={},
 pdfsubject={},
 pdfcreator={Emacs 28.0.91 (Org mode 9.4.6)}, 
 pdflang={English}}
\begin{document}

\maketitle
\tableofcontents


\section{1)}
\label{sec:org7f25491}
\begin{aligned}
\vec{L} &= \vec{p} \times m\vec{v} \\
\end{aligned}

The circle has a circumference of \(2\pi R\), and it takes \(\frac{2\pi}{\omega}\) seconds to travel that distance, so the tangential velocity must be \(\vec{v} = 2\pi R \div \frac{2\pi}{\vec{\omega}} &= R\vec{\omega}\) 
Therefore,
\begin{aligned}
\vec{L} &= \vec{R} \times mR\vec{\omega} \\
\end{aligned}

The vector is pointing out of the page, as the object is rotating counterclockwise.
\begin{aligned}
|\vec{L}| &= mR|\vec{R}||\vec{\omega}|\\
&= mR^2\omega \\
\end{aligned}

We know that \(\sin{\theta}\) is 1 because the vectors are perpendicular.

\section{2)}
\label{sec:orgc43aabe}
We can think of \(\vec{r}\) as the sum of some initial position vector and velocity times the time.

\begin{aligned}
\vec{r} &= \vec{r}_{0} + \vec{v}t \\
\end{aligned}

Then, we can look at angular momentum:

\begin{aligned}
\vec{L} &= \vec{r} \times m\vec{v} \\
&= (\vec{r}_{0} + \vec{v}t)\times m\vec{v} \\
&= (\vec{r}_{0}\times m\vec{v}) + (\vec{v}t\times m\vec{v}) \\
&= \vec{r}_{0} \times m\vec{v} \\
\end{aligned}

We see that angular momentum is not reliant on t. As such, it is conserved.

\section{3)}
\label{sec:orgbd395ca}
We are given:
\begin{aligned}
\vec{L} &= \vec{p}\timesm\vec{v} \\
\frac{d\vec{L}}{dt} &= \frac{d\vec{p}}{dt}\times m\vec{v} + \frac{d\,m\vec{v}}{dt}\times \vec{p} \\
&= \vec{v} \times m\vec{v} + m\vec{a} \times \vec{p}
&= 0 + \vec{p} \times m\vec{a} \\
&= \vec{p} \times\vec{F} \\
&= \vec{\tau} \\
\end{aligned}
\end{document}