% Created 2022-06-15 Wed 15:27
% Intended LaTeX compiler: xelatex
\documentclass[letterpaper]{article}
\usepackage{graphicx}
\usepackage{longtable}
\usepackage{wrapfig}
\usepackage{rotating}
\usepackage[normalem]{ulem}
\usepackage{amsmath}
\usepackage{amssymb}
\usepackage{capt-of}
\usepackage{hyperref}
\usepackage[margin=1in]{geometry}
\setlength{\parindent}{0pt}
\usepackage[margin=1in]{geometry}
\usepackage{fontspec}
\usepackage{svg}
\usepackage{tikz}
\usepackage{cancel}
\usepackage{pgfplots}
\usepackage{indentfirst}
\setmainfont[ItalicFont = HelveticaNeue-Italic, BoldFont = HelveticaNeue-Bold, BoldItalicFont = HelveticaNeue-BoldItalic]{HelveticaNeue}
\newfontfamily\NHLight[ItalicFont = HelveticaNeue-LightItalic, BoldFont       = HelveticaNeue-UltraLight, BoldItalicFont = HelveticaNeue-UltraLightItalic]{HelveticaNeue-Light}
\newcommand\textrmlf[1]{{\NHLight#1}}
\newcommand\textitlf[1]{{\NHLight\itshape#1}}
\let\textbflf\textrm
\newcommand\textulf[1]{{\NHLight\bfseries#1}}
\newcommand\textuitlf[1]{{\NHLight\bfseries\itshape#1}}
\usepackage{fancyhdr}
\usepackage{csquotes}
\pagestyle{fancy}
\usepackage{titlesec}
\usepackage{titling}
\makeatletter
\lhead{\textbf{\@title}}
\makeatother
\rhead{\textrmlf{Written} \today}
\lfoot{\theauthor\ \textbullet \ \textbf{2021-2022}}
\cfoot{}
\rfoot{\textrmlf{Page} \thepage}
\renewcommand{\tableofcontents}{}
\titleformat{\section} {\Large} {\textrmlf{\thesection} {|}} {0.3em} {\textbf}
\titleformat{\subsection} {\large} {\textrmlf{\thesubsection} {|}} {0.2em} {\textbf}
\titleformat{\subsubsection} {\large} {\textrmlf{\thesubsubsection} {|}} {0.1em} {\textbf}
\setlength{\parskip}{0.45em}
\renewcommand\maketitle{}
\author{Dylan Wallace}
\date{\today}
\title{Torque and Angular Momentum 2}
\hypersetup{
 pdfauthor={Dylan Wallace},
 pdftitle={Torque and Angular Momentum 2},
 pdfkeywords={},
 pdfsubject={},
 pdfcreator={Emacs 28.0.91 (Org mode 9.5.2)}, 
 pdflang={English}}
\begin{document}

\maketitle
\tableofcontents


\section{1)}
\label{sec:org765d0ce}
To finish the proof\ldots{}
Given two objects, \(A\) and \(B\), with a force \(F\) between them,
the torque on \(A\) and \(B\) is given by

\begin{aligned}
\tau_{A} &= \vec{r}_{A} \times \vec{F}_{A} \\
\tau_{B} &= \vec{r}_{B} \times \vec{F}_{B} \\
\end{aligned}

where \(\vec{F}_{A}\) is the foce applied by \(B\) on \(A\), and vice versa.
We know that because of N-3 \(\vec{F}_{A} = -\vec{F}_{B}\). (We also know that the forces point towards each object.)
Therefore,
\begin{aligned}
\tau_{AB} &= \tau_{A} + \tau_{B} \\
&= \vec{r}_{A} \times \vec{F}_{A} + \vec{r}_{B} \times \vec{F}_{B} \\
&= \vec{r}_{A} \times \vec{F}_{A} + \vec{r}_{B} \times -\vec{F}_{A} \\
\end{aligned}

We know that the direction of the two cross products are orthogonal to the plane that the two objects' position vectors and the origin of the system form.

\begin{aligned}
\tau_{AB} &= \vec{r}_{A} \times \vec{F}_{A} + \vec{r}_{B} \times -\vec{F}_{A} \\
&= |\vec{r}_{A}||\vec{F}_{A}|\sin{\theta_{A}} - |\vec{r}_{B}||\vec{F}_{A}|\sin{\theta_{B}} \\
&= |\vec{r}_{A}|\sin{\theta_{A}} - |\vec{r}_{B}|\sin{\theta_{B}} \\
\end{aligned}

The law of sines states that for a triangle \(\triangle ABC\), \(\frac{\overline{\rm BC}}{\sin{\theta_{A}}} = \frac{\overline{\rm AC}}{\sin{\theta_{B}}}\). We know that this applies in our particular proof because the objects \(A\), \(B\), and the origin form a triangle. As such,

\begin{aligned}
|\vec{r}_{A}|\sin{\theta_{A}} &= |\vec{r}_{B}|\sin{\theta_{B}} \\
\tau_{AB} &= 0 \\
\end{aligned}

The internal torque of any two objects of a system is zero, so the total internal torque must also be zero.

\section{2)}
\label{sec:orgd9dd50a}
We know that for one of the \(\vec{L}\) s: 
\begin{aligned}
\vec{r_1} &= R\hat{i} + h\hat{k} \\
\vec{L}_{1} &= \vec{r_1} \times m\vec{v_1} \\
\vec{v_1} &= R\omega\hat{j}\\
\vec{L}_{1} &= (R\hat{i} + h\hat{k}) \times mR\omega\hat{j} \\
&= -hmR\omega\hat{i} + mR^2\omega\hat{k}\\
\end{aligned}

We can show that the angular momentum of the two masses are symmetric by showing that \(\vec{r}\times \vec{v}\) is symmetric for both masses:

\begin{aligned}
\vec{r_{2}} &= -R\hat{i} + h\hat{k} \\
\vec{v_{2}} &= -R\omega\hat{j} \\
\\
\vec{r_1}\times \vec{v_1} &= (R\hat{i} + h\hat{k}) \times R\omega\hat{j} \\
&= -hR\omega\hat{i} + R^2\omega\hat{k} \\
\vec{r_2}\times \vec{v_2} &= (-R\hat{i} + h\hat{k}) \times R\omega\hat{j} \\
&= hR\omega\hat{i} + R^2\omega\hat{k} \\
\end{aligned}

Now we merely multiply by the mass (which is the same for both masses) to find the angular momentum:

\begin{aligned}
\vec{L_1} &= m(\vec{r_1}\times \vec{v_1}) \\
&= -hmR\omega\hat{i} + mR^2\omega\hat{k} \\
\vec{L_2} &= m(\vec{r_2}\times \vec{v_2}) \\
&= hmR\omega\hat{i} + mR^2\omega\hat{k} \\
\end{aligned}

We can add the two to get the aggregate angular momentum of the system:

\begin{aligned}
\vec{L} &= \vec{L}_{1} + \vec{L}_{2} \\
&= (-hmR\omega\hat{i} + mR^2\omega\hat{k}) + (hmR\omega\hat{i} + mR^2\omega\hat{k}) \\
&= 2mR^2\omega\hat{k} \\
\end{aligned}

\section{3)}
\label{sec:org7047123}
\subsection{a)}
\label{sec:orgbbb92cf}
We can think of the total angular momentum of an $$N$$-mass system as the sum of the z-components of their angular momentum. We only have to worry about their z components because it is given that the masses are symmetric about the center, and as such, the x and y components should cancel out.
\(\vec{L}_{N} &= \sum_{i=1}^{N} m_{i}l_{i}^2 \cdot \omega\hat{k}\)
\subsection{b)}
\label{sec:org857255d}
We can think of the total angular momentum as the above sum as \(N\) approaches infinity.
\begin{aligned}
\vec{L} &= \hat{k}\omega\sum_{i=1}^{N} m_{i}l_{i}^2 \\
\end{aligned}
We can think of \(l\) as a function of \(m\). We can turn our sum into an integral over the volume:

\begin{aligned}
\vec{L} &= \hat{k}\omega \int_{V} l^2 dm \\
dm &= \frac{M}{V_0} dV \\
\vec{L} &= \hat{k}\omega \int_{V} l^2 \frac{M}{V_0} dV \\
\end{aligned}

\section{4)}
\label{sec:orgba3716e}

We treat the rod as a line segment with length L and mass M, with density \(\lambda = M/L\). We also know the angular velocity as \(\vec{\omega} = \omega\hat{z}\).
Then, for each point on the line segment, we can find the angular momentum at that point. We can represent this as a function:

\begin{aligned}
\vec{L}(r) &= r\hat{x} \times \lambda r\omega\hat{y} \\
&= \lambda r^2 \omega \hat{z} \\
\end{aligned}

Then, we can integrate \(\vec{L}(r)\) from \(-L/2\) to \(L/2\):

\begin{aligned}
\vec{L}_{cum} &= \int_{-L/2}^{L/2} \vec{L}(r) dr \\
&= \int_{-L/2}^{L/2} \lambda r^2 \omega \hat{z} dr \\
&= [\frac{r^3}{3}]_{-L/2}^{L/2} \lambda\omega\hat{z} \\
&= 2\frac{L^3}{24} \lambda \omega \hat{z} \\
&= \frac{L^3}{12}\lambda\omega\hat{z} \\
&= \frac{L^3}{12}\frac{M}{L}\omega \hat{z} \\
&= \frac{1}{12}ML^2\omega \hat{z} \\
\end{aligned}






We can represent the angular momentum of each point on the rod and integrate over them to find the total angular momentum.
\begin{aligned}
L(r) &= r\hat{i} \times m_r\vec{v}_r \\
\vec{v}_r &= r\omega\hat{j} \\
L(r) &= r\hat{i} \times r m_r \omega\hat{j} \\
&= |rm_r\omega||r|\hat{k} \\
&= r^2 m_r\omega\hat{k}
\end{aligned}

We can now integrate this with respect to \(r\), from \(-\frac{L}{2}\) to \(\frac{L}{2}\). As we are integrating, we will remove \(m_r\) from our function and replace it with \(\lambda\), as the point mass will be infinitessimally small but will sum to \(M\), and the integral without the mass component will sum to \(L\).

\begin{aligned}
\vec{L} &= \int_{-\frac{L}{2}}^{\frac{L}{2}} r^2 \omega \lambda \hat{k} dr \\
&= \omega\lambda\hat{k} \int_{-\frac{L}{2}}^{\frac{L}{2}} r^2 dr \\
&= \omega\lambda\hat{k} \cdot \left[\frac{r^3}{3}\right]_{-\frac{L}{2}}^{\frac{L}{2}} \\
&= \omega\lambda\hat{k} \cdot 2\frac{L^3}{24} \\
&= \omega\lambda\hat{k}\cdot \frac{L^3}{12} \\
&= \omega\hat{k} \cdot \frac{1}{12}L^2M \\
&= \frac{1}{12}ML^2\omega\hat{k} \\
\end{aligned}


\section{5)}
\label{sec:orgf947884}

We know that to find the angular momentum of the ring, we need to sum the angular momentum of every point on the disk.
We know the density of the ring itself. We also know that the infinitessimal area of a ring given an infinitessimal thickness is \(da &= 2\pi r\,dr\).

Then, we can find the infinitessimal mass of that ring:
\begin{aligned}
dm &= \sigma\,da \\
&= \frac{M}{\pi R^2}\,da \\
&= \frac{M}{\pi R^2} 2\pi r\,dr \\
&= \frac{2Mr}{R^2}\,dr \\
\end{aligned}

We also know how to find the angular momentum of a ring:
\begin{aligned}
\vec{L}_{ring} &= r\hat{i}\times mv\hat{j} \\
&= r\hat{i}\times mr\omega\hat{j} \\
&= mr^2\omega \hat{k} \\
\end{aligned}

Plugging in what we found for \(dm\) for \(m\), we get

\begin{aligned}
\vec{L}_{ring} &= r^2\omega\hat{k}\,dm \\
&= r^2\omega\hat{k}\, \frac{2Mr}{R^2}\,dr \\
&= \frac{2Mr^3\omega}{R^2}\,dr \\
\end{aligned}

We then integrate:

\begin{aligned}
\vec{L} &= \int_{0}^{R} \vec{L}_{ring} &= \int_{0}^{R} \frac{2Mr^3\omega}{R^2}\hat{k} \,dr \\
&= \frac{2M\omega}{R^2}\hat{k} \int_{0}^{R} r^3\,dr \\
&= \frac{2M\omega}{R^2} [\frac{r^4}{4}]^{R}_{0} \hat{k} \\
&= \frac{2M\omega}{R^2}\cdot \frac{R^4}{4} \hat{k} \\
&= frac{2M\omegaR^2}{4}\hat{k} \\
&= \frac{1}{2}MR^2\omega\hat{k} \\
\end{aligned}
\end{document}